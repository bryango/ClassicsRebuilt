%From: Pierre Deligne <deligne@math.ias.edu>
%Date: Thu, 23 Jan 1997 17:52:08 -0500
%Subject: set7-1.tex

\input amstex
\documentstyle{amsppt}
\magnification=1200
\pagewidth{6.5 true in}
\pageheight{8.9 true in}
\loadeusm

\catcode`\@=11
\def\logo@{}
\catcode`\@=13

\NoRunningHeads

\input pictex.tex

\font\boldtitlefont=cmb10 scaled\magstep2

\def\lam{{\lambda}}
\def\eps{{\varepsilon}}

\def\psibar{\bar{\psi}}
\def\Omegabar{\bar{\Omega}}
\def\Lbar{\bar{L}}
\def\Sbar{\bar{S}}
\def\Xbar{\bar{X}}
\def\Ybar{\bar{Y}}

\def\Ghat{\hat{G}}
\def\Dhat{\hat{D}}
\def\Gammahat{\hat{\Gamma}}

%\def\dspace{\lineskip=2pt\baselineskip=18pt\lineskiplimit=0pt}
\def\halfspace{\lineskip=1pt\baselineskip=15pt\lineskiplimit=0pt}
%\def\wedgeop{\operatornamewithlimits{\wedge}\limits}
%\def\w{{\mathchoice{\,{\scriptstyle\wedge}\,}
%  {{\scriptstyle\wedge}}
%  {{\scriptscriptstyle\wedge}}{{\scriptscriptstyle\wedge}}}}
%\def\Le{{\mathchoice{\,{\scriptstyle\le}\,}
%  {\,{\scriptstyle\le}\,}
%  {\,{\scriptscriptstyle\le}\,}{\,{\scriptscriptstyle\le}\,}}}
\def\Ge{{\mathchoice{\,{\scriptstyle\ge}\,}
  {\,{\scriptstyle\ge}\,}
  {\,{\scriptscriptstyle\ge}\,}{\,{\scriptscriptstyle\ge}\,}}}
\def\vrulesub#1{\hbox{\vrule height7pt depth5pt\,}_{#1}}
%\def\mapright#1{\smash{\mathop{\,\longrightarrow\,}%
%     \limits^{#1}}}
%\def\plus{{\sssize +}}
%\def\upvee{{\sssize\vee}}
%\def\underSim#1{\mathop{\vtop{\ialign{##\crcr
%  \hfil {\bigsym\char'030}\hfil\crcr
%  \noalign{\nointerlineskip}${\ssize #1}$\crcr
%  \noalign{\nointerlineskip}\crcr}}}}

\def\eps{{\varepsilon}}
\def\Lam{{\Lambda}}
%\def\mynabla{{\nabla\!}}

%\def\xtil{\widetilde{x}}

%\def\Adot{\Dot{A}}
%\def\Bdot{\Dot{B}}
%\def\Xdot{\Dot{X}}
%\def\xdot{\Dot{x}}
%\def\psidot{\Dot{\psi}}

\def\dbC{{\Bbb C}}
%\def\dbR{{\Bbb R}}

\def\Tr{\text{\rm Tr}} \def\ln{\text{\rm ln}}
\def\End{\text{\rm End}} \def\Sym{\text{\rm Sym}}
\def\I{\text{\rm I}} \def\free{\text{\rm free}}
\def\II{\text{\rm II}}
\def\III{\text{\rm III}}


\def\scr#1{{\fam\eusmfam\relax#1}}

\def\scrF{{\scr F}}
\def\scrH{{\scr H}}
\def\scrL{{\scr L}}
\def\scrO{{\scr O}}


\document
\line{{\boldtitlefont Witten's Problems}, Set Seven --- 
N$^{\text{o}}$. 1 
\hfill (solution written by D. Freed)}
\smallskip
\hbox to \hsize{\hrulefill}

\bigskip\noindent
{\bf Problem:} 

\smallskip
(1) (a) 
Show that in $\phi^4$ theory the anomalous 
dimension of $\phi$ vanishes
in the one loop approximation and that the one-loop 
beta function is therefore
given by a one-loop diagram that we have computed before.  
The one-loop
beta function is in particular $\beta(\lambda)=A\lambda^2$ 
where $\lambda$ is
the $\phi^4$ coupling, and $A$ is the coefficient of the 
one-loop divergence.

(b) Convince yourself that the sign of the logarithmic 
divergence in this
one-loop diagram (whose significance we by now appreciate) 
can be determined
without any computation; it is determined by the 
fact that in {\it free}
scalar field theory the two point 
function $\langle \phi^2(x)\phi^2(0)\rangle$
is positive.  (This is actually a point that has been 
made previously.)

\bigskip\bigskip\noindent
{\bf Solution:}

\smallskip
\halfspace
\noindent
(a)\enspace
The anomalous dimension of $\phi$ gets contributions from
$2$-point function diagrams.
All such (connected) diagrams have naive quadratic
divergences.
Let $p$ be the momentum flowing in (and by conservation
flowing out) and $m$ the mass.
Then with a cutoff $\Lam$ the diagram diverges as
$$
f\fracwithdelims()pm
\Lam^2+g\fracwithdelims()pm
\log\fracwithdelims()\Lam m
p^2+h\fracwithdelims()pm\log\fracwithdelims()\Lam m
m^2
$$
for some functions $f$, $g$, $h$.
This follows by dimensional analysis.
(In fact, since the second derivative of a diagram with
respect to $p$ is finite, we see that $f$, $g$, $h$ are
simple polynomials.)
Now the only $1$-loop diagram is shown, and the loop
integral is\vadjust{\bigskip $$\vbox{\beginpicture
\setcoordinatesystem units < .50cm, .50cm>
\setlinear
%
% Fig ELLIPSE
%
\linethickness= 0.500pt
\ellipticalarc axes ratio  0.667:0.667  360 degrees 
	from 11.779 17.844 center at 11.113 17.844
%
% Fig POLYLINE object
%
\linethickness= 0.500pt
\putrule from  8.858 17.177 to 13.335 17.177
%
% Fig POLYLINE object
%
\linethickness= 0.500pt
\putrule from 12.287 17.177 to 12.954 17.177
%
% arrow head
%
\plot 12.700 17.113 12.954 17.177 12.700 17.240 /
%
%
% Fig POLYLINE object
%
\linethickness= 0.500pt
\putrule from  9.335 17.177 to 10.001 17.177
%
% arrow head
%
\plot  9.747 17.113 10.001 17.177  9.747 17.240 /
%
%
% Fig TEXT object
%
\put{$\ssize p$} [lB] at 12.700 17.716
%
% Fig TEXT object
%
\put{$\ssize p$} [lB] at  9.779 17.685
\linethickness=0pt
\putrectangle corners at  8.858 18.510 and 13.335 17.177
\endpicture}
$$}
\noindent
independent of $p$.
For this diagram, then, $g\equiv 0$ and there is no
renormalization to the coefficient of $\vert
d\phi\vert^2$ and so no contribution to the anomalous
dimension of $\phi$.

The diagram which contributes to the $1$-loop
$\beta$-function is the 
\vadjust{\bigskip $$\vbox{\beginpicture
\setcoordinatesystem units < .50cm, .50cm>
\setlinear
%
% Fig ELLIPSE
%
\linethickness= 0.500pt
\ellipticalarc axes ratio  1.715:1.715  360 degrees 
	from 12.478 17.844 center at 10.763 17.844
%
% Fig POLYLINE object
%
\linethickness= 0.500pt
\plot 12.478 17.812 14.573 19.114 /
%
% Fig POLYLINE object
%
\linethickness= 0.500pt
\plot 12.446 17.748 14.541 16.447 /
%
% Fig POLYLINE object
%
\linethickness= 0.500pt
\plot  6.953 16.478  9.049 17.780 /
%
% Fig POLYLINE object
%
\linethickness= 0.500pt
\plot  6.953 19.114  9.049 17.812 /
%
% Fig POLYLINE object
%
\linethickness= 0.500pt
\putrule from 10.700 19.558 to 10.954 19.558
%
% arrow head
%
\plot 10.700 19.494 10.954 19.558 10.700 19.622 /
%
%
% Fig POLYLINE object
%
\linethickness= 0.500pt
\putrule from 10.922 16.161 to 10.795 16.161
\putrule from 10.795 16.161 to 10.700 16.161
%
% arrow head
%
\plot 10.954 16.224 10.700 16.161 10.954 16.097 /
%
%
% Fig POLYLINE object
%
\linethickness= 0.500pt
\plot 14.034 18.796 13.906 18.637 /
%
% arrow head
%
\plot 14.016 18.875 13.906 18.637 14.115 18.796 /
%
%
% Fig POLYLINE object
%
\linethickness= 0.500pt
\plot 13.875 16.828 13.748 16.954 /
%
% arrow head
%
\plot 13.972 16.820 13.748 16.954 13.882 16.730 /
%
%
% Fig POLYLINE object
%
\linethickness= 0.500pt
\plot  7.493 18.764  7.620 18.669 /
%
% arrow head
%
\plot  7.379 18.771  7.620 18.669  7.455 18.872 /
%
%
% Fig POLYLINE object
%
\linethickness= 0.500pt
\plot  7.652 16.891  7.747 16.954 /
%
% arrow head
%
\plot  7.571 16.761  7.747 16.954  7.500 16.866 /
%
%
% Fig TEXT object
%
\put{$\ssize q$} [lB] at 10.763 20.003
%
% Fig TEXT object
%
\put{\null\kern-5pt $\ssize q-(p_1+p_2)$} [lB] at  9.938 15.399
%
% Fig TEXT object
%
\put{$\ssize x_1$} [lB] at  6.191 19.082
%
% Fig TEXT object
%
\put{$\ssize x_2$} [lB] at  6.636 15.875
%
% Fig TEXT object
%
\put{$\ssize x_3$} [lB] at 14.922 19.018
%
% Fig TEXT object
%
\put{\raise3pt\hbox{$\ssize x_4$}} [lB] at 14.700 15.939
%
% Fig TEXT object
%
\put{$\ssize p_1$} [lB] at  7.779 19.018
%
% Fig TEXT object
%
\put{$\ssize p_2$} [lB] at  7.811 16.447
%
% Fig TEXT object
%
\put{$\ssize p_3$} [lB] at 13.430 18.955
%
% Fig TEXT object
%
\put{$\ssize p_4$} [lB] at 13.906 17.177
\linethickness=0pt
\putrectangle corners at  6.191 20.510 and 14.922 15.399
\endpicture}
$$}
integral
$$
\Dhat(p_1,p_2,p_3,p_4)=\lam^2\int
\tfrac{d^4q}{(2\pi)^4}\tfrac{1}{(q^2+m^2)((q-p_1-p_2)^2+m^2)}
\cdot\delta(p_1+p_2+p_3+p_4)
$$
which has a leading divergence $A\lam^2\log\frac\Lam m$
if an ultraviolet cutoff $\Lam$ is imposed.
(Here $\Gammahat$ is a distribution supported on the
subspace $(p_1+p_2+p_3+p_4=0$.)

\medskip\noindent
(b)\enspace
We want to understand {\it a priori} that $A>0$.
For this it is best to Fourier transform to position
space.
This is straightforward and leads to
$$
D(x_1,x_2,x_3,x_4)=G(x-y)^2
\delta(x_1-x)\delta(x_2-x)\delta(x_3-y)
\delta(x_4-y),
$$
where $G$ is the propagator.
(So $D$ is supported on the subspace $x_1=x_2$,
$x_3=x_4$ and its value there is the square of the
Green's function.)
In general, the position space prescription for computing
a {\it one particle irreducible} Feynman diagram as part
of the effective action is the same as usual --- we
attach propagators to the legs and integrate over the
positions in the internal vertices --- except that we
replace the propagators on the {\it external} legs by
$\delta$-functions.
For convenience we take $y=0$ and write
$D(x)=G(x)^2$.

Now introduce the composite operator
$\scrO(x)=\phi^2(x)$.
Then it is immediate that the two point function
$\left<\scrO(x)\scrO(0)\right>_{\free}$ in the {\it free}
theory is computed by the diagram shown.

$$
\vbox{\beginpicture
\setcoordinatesystem units < .50cm, .50cm>
\setlinear
%
% Fig CIRCULAR ARC object
%
\linethickness= 0.500pt
\circulararc 76.238 degrees from  9.684 17.526 center at  8.831 16.397
%
% Fig CIRCULAR ARC object
%
\linethickness= 0.500pt
\circulararc 76.238 degrees from  7.938 17.209 center at  8.831 18.305
%
% Fig TEXT object
%
\put{\null\kern-.95pt$\ssize x\raise.50pt\hbox{$\ssize\otimes$}$} [lB] at  7.557 17.177
%
% Fig TEXT object
%
\put{$\ssize \otimes 0$} [lB] at  9.525 17.209
\linethickness=0pt
\putrectangle corners at  7.557 17.875 and  9.684 16.859
\endpicture}
$$

\noindent
We evaluate this diagram as $G(x)^2$, so it is precisely
equal to $D(x)$.
It is manifest that $D(x)\Ge0$, of course, since it is
written as the square of a real function.
But we would like to invoke instead {\it reflection
positivity} since the argument is useful in more general
situations.

Reflection positivity asserts that if $\scrO_\Omega$ is a
(complex) operator in a Euclidean field theory, supported
in some set $\Omega$ lying on one side of a hyperplane,
and if $\Omegabar$ denotes the reflection of $\Omega$ in
the hyperplane, then
$$
\left<\scrO_\Omega\overline{\scrO_{\Omegabar}}\right>
\Ge 0.
$$
If $\scrO_\Omega$ takes values in a vector space $V$,
then the assertion is that
$\left<\scrO_\Omega\overline{\scrO_{\Omegabar}}\right>$
is a nonnegative hermitian form on $V^*$.
$$
\vbox{\beginpicture
\setcoordinatesystem units < .50cm, .50cm>
\setlinear
%
% Fig POLYLINE object
%
\linethickness= 0.500pt
\putrule from  5.048 16.542 to 19.558 16.542
\linethickness= 0.500pt
%
% Fig INTERPOLATED PT SPLINE
%
\plot  8.827 19.336 	 8.703 19.283
	 8.598 19.233
	 8.511 19.184
	 8.441 19.136
	 8.341 19.036
	 8.287 18.923
	 8.269 18.830
	 8.269 18.733
	 8.284 18.634
	 8.313 18.537
	 8.354 18.444
	 8.407 18.359
	 8.470 18.285
	 8.541 18.224
	 8.642 18.180
	 8.756 18.168
	 8.878 18.180
	 9.005 18.205
	 9.132 18.232
	 9.253 18.251
	 9.364 18.252
	 9.462 18.224
	 9.544 18.162
	 9.621 18.072
	 9.681 17.971
	 9.716 17.875
	 9.698 17.797
	 9.684 17.716
	 9.718 17.635
	 9.770 17.554
	 9.835 17.474
	 9.910 17.398
	 9.990 17.328
	10.071 17.266
	10.151 17.215
	10.224 17.177
	10.311 17.147
	10.416 17.127
	10.533 17.115
	10.655 17.110
	10.777 17.108
	10.892 17.110
	10.996 17.112
	11.081 17.113
	11.192 17.113
	11.317 17.113
	11.385 17.113
	11.456 17.113
	11.530 17.113
	11.605 17.113
	11.683 17.113
	11.763 17.114
	11.845 17.115
	11.928 17.116
	12.012 17.118
	12.096 17.121
	12.182 17.123
	12.267 17.127
	12.353 17.131
	12.438 17.136
	12.523 17.141
	12.607 17.148
	12.691 17.155
	12.772 17.163
	12.853 17.172
	12.931 17.182
	13.008 17.193
	13.082 17.205
	13.153 17.218
	13.222 17.233
	13.287 17.248
	13.349 17.266
	13.462 17.304
	13.523 17.330
	13.590 17.361
	13.660 17.398
	13.733 17.439
	13.809 17.484
	13.885 17.533
	13.962 17.586
	14.038 17.641
	14.113 17.699
	14.185 17.758
	14.253 17.819
	14.317 17.881
	14.428 18.006
	14.510 18.129
	14.546 18.217
	14.571 18.322
	14.588 18.438
	14.596 18.558
	14.598 18.679
	14.594 18.795
	14.585 18.899
	14.573 18.986
	14.540 19.107
	14.514 19.178
	14.484 19.250
	14.451 19.322
	14.417 19.389
	14.351 19.494
	14.277 19.581
	14.180 19.676
	14.068 19.776
	13.946 19.876
	13.822 19.972
	13.701 20.060
	13.589 20.135
	13.494 20.193
	13.376 20.254
	13.306 20.288
	13.231 20.322
	13.151 20.357
	13.067 20.391
	12.981 20.426
	12.892 20.459
	12.804 20.491
	12.716 20.522
	12.629 20.550
	12.545 20.575
	12.465 20.597
	12.389 20.615
	12.319 20.628
	12.256 20.637
	12.177 20.643
	12.083 20.644
	11.980 20.640
	11.872 20.632
	11.764 20.618
	11.662 20.598
	11.570 20.573
	11.494 20.542
	11.368 20.443
	11.304 20.373
	11.241 20.296
	11.180 20.219
	11.121 20.146
	11.017 20.034
	10.905 19.947
	10.834 19.896
	10.759 19.844
	10.682 19.794
	10.607 19.749
	10.538 19.712
	10.478 19.685
	10.389 19.659
	10.282 19.639
	10.160 19.622
	10.097 19.615
	10.033 19.608
	 9.968 19.602
	 9.905 19.596
	 9.784 19.584
	 9.676 19.572
	 9.588 19.558
	 9.504 19.540
	 9.397 19.515
	 9.291 19.488
	 9.207 19.463
	 9.084 19.412
	 8.988 19.367
	 8.928 19.338
	 8.858 19.304
	/
\linethickness= 0.500pt
%
% Fig INTERPOLATED PT SPLINE
%
\plot  8.795 13.557 	 8.671 13.610
	 8.567 13.660
	 8.480 13.709
	 8.409 13.757
	 8.309 13.857
	 8.255 13.970
	 8.238 14.063
	 8.237 14.160
	 8.252 14.259
	 8.281 14.356
	 8.323 14.449
	 8.375 14.534
	 8.438 14.608
	 8.509 14.669
	 8.610 14.713
	 8.724 14.725
	 8.847 14.713
	 8.973 14.688
	 9.100 14.661
	 9.221 14.642
	 9.333 14.641
	 9.430 14.669
	 9.513 14.731
	 9.589 14.821
	 9.649 14.922
	 9.684 15.018
	 9.666 15.096
	 9.652 15.176
	 9.686 15.258
	 9.738 15.339
	 9.803 15.419
	 9.878 15.495
	 9.958 15.565
	10.040 15.627
	10.119 15.678
	10.192 15.716
	10.279 15.746
	10.384 15.766
	10.501 15.778
	10.623 15.783
	10.745 15.785
	10.861 15.783
	10.964 15.781
	11.049 15.780
	11.160 15.780
	11.286 15.780
	11.354 15.780
	11.424 15.780
	11.498 15.780
	11.574 15.780
	11.652 15.780
	11.732 15.779
	11.813 15.778
	11.896 15.777
	11.980 15.775
	12.065 15.772
	12.150 15.770
	12.236 15.766
	12.321 15.762
	12.407 15.757
	12.491 15.752
	12.576 15.745
	12.659 15.738
	12.741 15.730
	12.821 15.721
	12.899 15.711
	12.976 15.700
	13.050 15.688
	13.121 15.675
	13.190 15.660
	13.255 15.645
	13.317 15.627
	13.430 15.589
	13.492 15.563
	13.558 15.532
	13.628 15.495
	13.701 15.454
	13.777 15.409
	13.854 15.360
	13.931 15.307
	14.007 15.252
	14.081 15.194
	14.153 15.135
	14.222 15.074
	14.286 15.012
	14.396 14.887
	14.478 14.764
	14.514 14.676
	14.540 14.571
	14.556 14.455
	14.565 14.335
	14.566 14.214
	14.562 14.098
	14.553 13.994
	14.541 13.906
	14.509 13.786
	14.483 13.715
	14.452 13.643
	14.419 13.571
	14.385 13.504
	14.319 13.399
	14.245 13.312
	14.149 13.217
	14.036 13.117
	13.915 13.017
	13.790 12.921
	13.669 12.833
	13.557 12.758
	13.462 12.700
	13.344 12.639
	13.275 12.605
	13.199 12.571
	13.119 12.536
	13.035 12.502
	12.949 12.467
	12.861 12.434
	12.772 12.402
	12.684 12.371
	12.597 12.343
	12.513 12.318
	12.433 12.296
	12.357 12.278
	12.287 12.265
	12.224 12.256
	12.145 12.250
	12.052 12.249
	11.948 12.253
	11.840 12.261
	11.732 12.275
	11.630 12.295
	11.538 12.320
	11.462 12.351
	11.337 12.450
	11.273 12.520
	11.209 12.597
	11.148 12.674
	11.089 12.747
	10.986 12.859
	10.873 12.946
	10.803 12.997
	10.727 13.049
	10.650 13.099
	10.575 13.144
	10.506 13.181
	10.446 13.208
	10.358 13.234
	10.250 13.254
	10.129 13.271
	10.065 13.278
	10.001 13.285
	 9.937 13.291
	 9.873 13.297
	 9.752 13.309
	 9.644 13.321
	 9.557 13.335
	 9.472 13.353
	 9.365 13.378
	 9.259 13.405
	 9.176 13.430
	 9.052 13.481
	 8.956 13.526
	 8.896 13.555
	 8.827 13.589
	/
%
% Fig TEXT object
%
\put{$\ssize\Omega$} [lB] at 15.431 18.574
%
% Fig TEXT object
%
\put{$\ssize\bar{\Omega}$} [lB] at 15.335 14.065
%
% Fig TEXT object
%
%\put{\SetFigFont{12}{14.4}{rm}\_} [lB] at 15.367 14.446
%
% Fig TEXT object
%
\put{$\ssize\pi$} [lB] at 20.320 16.383
%
% Fig TEXT object
%
\put{$\ssize X_+$} [lB] at 18.256 17.209
%
% Fig TEXT object
%
\put{$\ssize X_-$} [lB] at 18.256 15.589
\linethickness=0pt
\putrectangle corners at  5.048 20.637 and 20.320 12.256
\endpicture}
$$

We can understand this heuristically by writing a path
integral over the half-space.
Namely, suppose for simplicity that we are dealing with
a complex scalar field $\phi$ with action $S(\phi)$.
Then for $\phi_0$ a function on $\pi$ we set
$$
Z_{X_+}(\phi_0)=\int\limits
\Sb
\phi\colon\,X_+\to\dbC\\
\phi\vrulesub{\pi}=\phi_0
\endSb
\scrO_\Omega(\phi)e^{-S_{X_+}(\phi)}D\phi.
$$
This should be viewed as a vector in some Hilbert space
of functions of $\phi_0$.
Now $X_-\cong \Xbar_+$, where $\Xbar_+$ is $X_+$ with the
reversed orientation.
The Euclidean action has the property that
$S_{\Ybar}=\overline{S_Y}$ for any spacetime
$Y$: \ orientation reversal conjugates the action.
It follows that
$$
\align
\overline{Z_{X_+}(\phi_0)} &=\int\limits
\Sb
\phi\colon\,X_+\to\dbC\\
\phi\vrulesub{\pi}=\phi_0
\endSb
\overline{\scrO_\Omega(\phi)}e^{-\overline{S_{X_+}(\phi)}}
D\phi\\
&=\int\limits
\Sb
\phi\colon\,X_-\to\dbC\\
\phi\vrulesub{\pi}=\phi_0
\endSb
\overline{\scrO_\Omega(\phi)}e^{-S_{X_-}(\phi)}D\phi
\endalign
$$
The norm square of $Z_{X_+}(\phi)$ in $\scrH$ is computed
by integrating over $\phi_0$:
$$
\align
\Vert Z_{X_+}(\phi_0)\Vert_{\scrH}^2
&=\int\limits_{\phi_0\colon\,\pi\to\dbC}Z_{X_+}(\phi_0)
\overline{Z_{X_+}(\phi_0)}D\phi_0\\
&=\int\limits_{\phi\colon X\to\dbC}\scrO_\Omega
(\phi)\overline{\scrO_{\Omegabar}(\phi)}e^{-S_X(\phi)}D\phi\\
&=\left<\scrO_\Omega\overline{\scrO_{\Omegabar}}\right>,
\endalign
$$
where the measure is normalized so that the partition function
is one.
The left hand side is obviously positive, hence so is the
right.

In our example $\scrO_{(x)}=\phi^2(x)$ is real, and to
show that
$D(x)=\left<\scrO(x)\scrO(0)\right>_{\free}\Ge 0$ we
use reflection positivity for the perpendicular
bisector of the line segment connecting $0$ and $x$.
Now $D(x)$ diverges as $x\to 0$, and since
$D(x)\Ge 0$ the coefficient of the divergence is
positive.
This implies that the divergence in $D$ with a
momentum cutoff $\Lam$ has a positive coefficient.


\enddocument




