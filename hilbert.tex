
\section{Isomorphism of Hilbert Spaces} 
\label{isom}

The $AdS$/CFT correspondence is a statement about
the equivalence of two quantum theories: string
theory (or M theory) on \adsp\ $\times$ (compact space)
and CFT$_{p+1}$. The two quantum theories
are equivalent if there is an isomorphism
between their Hilbert spaces, and moreover
if the operator algebras on the Hilbert spaces
are equivalent. In this section, we
discuss the isomorphism of the Hilbert spaces,
following \cite{Horowitz:1998bj,Witten:1998zw,
Maldacena:1998bw,Banks:1998dd}.
Related issues have been discussed 
in \cite{Barbon:1998ix,Barbon:1998cr,Abel:1999rq,
Chamblin:1999pz,Li:1999jy,Peet:1998cr,
Martinec:1998wm,Martinec:1999ja,
Martinec:1999sa}.
% The equivalence of the operator algebras will
% be discussed in subsequent sections. 

States in the Hilbert space of CFT$_{p+1}$
fall into representations of the global conformal
group $SO(2,p+1)$ on $\IR^{p,1}$. At the same time,
the isometry group of \ads\ is also $SO(2,p+1)$, and 
we can use it to classify states in 
the string theory. Thus, it is useful to compare
states in the two theories by organizing them into
representations of $SO(2,p+1)$. 
The conformal group $SO(2,p+1)$ has $\frac{1}{2}(p+2)(p+3)$
generators, $J_{ab} = -J_{ba}$ ($a,b = 0,1, \cdots, p+2$),
obeying the commutation relation
\beq
   [J_{ab}, J_{cd}] = -i ( g_{ac} J_{bd} \pm {\sl permutations})
\label{commutations} 
\eeq
with the metric $g_{ab}={\rm diag} (-1,+1,+1, \cdots, +1, -1)$.
In CFT$_{p+1}$, they are identified with
 the Poincar\'e generators $P_\mu$ and $M_{\mu\nu}$,
the dilatation $D$ and the special conformal generators $K_\mu$
($\mu, \nu = 0, \cdots, p$), by the formulas
\beq
   J_{p+2,p+1} = D, ~ J_{\mu,p+2} = \frac{1}{2}(K_\mu + P_\mu),
~ J_{\mu,p+1} = \frac{1}{2}(K_\mu - P_\mu), ~ J_{\mu\nu} = M_{\mu\nu}.
\eeq

Since the field theory on $\IR^{p,1}$ has no scale,  
the spectrum of the Hamiltonian $P_0$ is
continuous and there is no normalizable
ground state with respect to $P_0$. 
This is also the case for the string theory on 
\adsp. The Killing vector $\partial_t$
corresponding to $P_0$ has the norm
\beq
   || \partial_t || = Ru,
\eeq
and it vanishes as $u \rightarrow 0$. Consequently,
a stationary wave solution of the linearized supergravity
on \ads\ has a continuous frequency spectrum with respect to 
the timelike coordinate $t$. It is not easy to compare the
spectrum of $P_0$ of the two theories.

It is more useful to compare the two Hilbert spaces using the maximum compact
subgroup $SO(2) \times SO(p+1)$ of the conformal group \cite{Horowitz:1998bj}. 
The Minkowski space $\IR^{p,1}$ is conformally embedded in 
the Einstein Universe $\IR \times S^p$, 
and $SO(2) \times SO(p+1)$ is its isometry group. 
In particular, the generator $J_{0,p+2}=\frac{1}{2}(P_0 + K_0)$
of $SO(2)$ is the Hamiltonian 
for the CFT on $\IR \times S^p$. Now we have a scale in the problem, 
which is the radius of $S^p$, and the Hamiltonian $\frac{1}{2}(P_0 + K_0)$
has a mass gap. In string theory on \adsp , 
the generator $\frac{1}{2}(P_0 + K_0)$ corresponds to the global 
time translation 
along the coordinate $\tau$. This is a globally well-defined
coordinate on \ads\ and the Killing vector $\partial_\tau$
is everywhere non-vanishing:
\beq
  || \partial_\tau || = \frac{R}{\cos \theta}. 
\eeq
Therefore, a stationary wave solution with respect to $\tau$
is normalizable and has a discrete frequency spectrum.  
In fact, as we saw in section \ref{kkcatalogue}, the frequency
is quantized in such a way that bosonic fields in the
supergravity multiplet are periodic and their superpartners 
are anti-periodic ($i.e.$ obeying the supersymmetry preserving
Ramond boundary condition) in the $\tau$-direction
with the period $2\pi R$.

\subsection{Hilbert Space of String Theory}

With the techniques that  are currently available, 
we can make reliable  statements 
about the Hilbert space structure of  string theory on \ads\ 
only when the curvature radius $R$ of \ads\ is 
much larger than the string length $l_s$. 
In this section we will study some of the properties of the 
Hilbert space that we can see in the $AdS$ description. We will 
concentrate on the $AdS_5 \times S^5$ case, but it is easy to
 generalize this  to other cases.

We first consider the case that corresponds to the 't Hooft limit
$g_s \to 0 $, $g_s N$ fixed and large, so that we can trust the gravity 
approximation. 





\medskip
\noindent
(1) $E \ll  m_s $; {\it Gas of Free Gravitons}

\smallskip

The Hilbert space for low energies is well approximated by the
Fock space of gravitons and their superpartners on $AdS_5\times S^5$.
Since $\tau$ is a globally defined timelike coordinate on \ads ,
we can consider stationary wave solutions in the linearized
supergravity, which are the normalizable states discussed in section
\ref{pfinads}. The frequency $\omega$ of a stationary mode is quantized
in the unit set by the curvature radius $R$ (\ref{energyquanta}),
so  one may effectively view the supergravity particles
in \ads\ as confined in a box of size $R$. 
%More explicitly, 
%for a scalar field of mass $M$, the frequency $\omega$
%of a stationary solution on \adsp\ is given by
%\eqn{spec}{
%    \omega = \frac{1}{2R}\left[ (p+1) +  \sqrt{(p+1)^2 + 4 M^2R^2}
%      + 2 k \right],
%~~~ (k=0,1,2,...).} 
%For a given $M$, these stationary wave solutions
%make an irreducible highest weight representation of $SO(2,p+1)$, with 
%$k=0$ corresponding
%to the highest weight state. There are similar formulae for other
%tensor fields and fermions. 
%In the cases of maximum supersymmetry, \ads$_4 \times S^7$, \ads$_5
%\times S^5$, and \ads$_7 \times S^4$, the mass spectrum of the
%supergravity multiplet is such that the $\omega$'s given by
%(\ref{spec}) are always 
%integers times $R^{-1}$ for bosonic fields
%and halves of odd integers times $R^{-1}$ for fermionic
%fields, including the Kaluza-Klein tower
%coming from excitations on the compact space.
%This is a consequence of the fact that
%a single particle state is either a chiral primary state 
%or its descendant with respect to the 
%superconformal algebras, which are $OSp(6,2|4)$, $SU(2,2|4)$
%and $OSp(8|4)$ respectively \cite{Freedman:1984na}.

The operator $H=\frac{1}{2R}(P_0 + K_0)$ 
corresponds\footnote{
The factor $\frac{1}{2R}$ in the relation between $H$ and
$(P_0+K_0)$ is fixed by
the commutation relations (\ref{commutations}).}  to
the Killing vector $\partial_\tau$ on \ads.
Thus, a single particle state of frequency $\omega$
gives an eigenstate of $H$.
Since the supergraviton is a BPS particle, 
its energy eigenvalue $\omega$ 
is exact, free from corrections 
either by first quantized string effects ($\sim l_s/R$) 
or by quantum gravity effects 
($\sim l_P/R$). The energy of multiparticle states
may receive corrections, but they become important only
when the energy $E$ becomes comparable to the gravitational
potential $E^2/(m_P^8 R^7)$, $i.e.$ $E \sim m_P^8 R^7$.
For the energies we are considering  this effect is negligible. 

Therefore, the Hilbert space for $E \ll m_s$ is identified
with the Fock space of free supergravity particles. 
For $E \gg R^{-1}$, the
entropy $S(E)$ ($= \log N(E)$ where $N(E)$ 
is the density of states) behaves as
%\beq
%     S(E) \sim   (ER)^{\frac{9}{10}} ,
%\eeq
%
%contributing to the thermal spectrum,
%$i.e.$ $d = p+1 +{\rm dim}$(compact space $\gg l_s, l_P$).
%The proportionality constant is related to the number
%of particles  
%in the supergravity multiplet, and is independent of $l_s$ and
%$l_P$.
% For simplicity, in the
%following we will assume that the size of the compact
%space is of the order of $R$ and that $d=9$. In this case 
\beq
   S(E) \sim (ER)^{\frac{9}{10}},
 \label{fock}
\eeq
since we effectively have a gas in ten dimensions (we will ignore
multiplicative numerical factors in the entropy in this section).

\medskip
\noindent
(2) $ m_s < E \ll  m_s/g_s^2 $; {\it Gas of Free Strings}

\smallskip

When the energy $E$ becomes comparable to the string scale $m_s$, we
have to take into account excitations on the string worldsheet.
Although we do not know the exact first quantized spectrum 
of string theory on \ads, 
we can estimate the effects of the worldsheet excitations
when $l_s \ll R$.
The mass $m$ of a first quantized string state is a function of 
$l_s$ and $R$.  When $l_s \ll R$, the worldsheet dynamics is 
perturbative and we can expand $m$ in powers of $l_s/R$, with the
leading term given by the string spectrum on flat space 
($R = \infty$). Therefore, for a string state corresponding to the
$n$-th excited level of the string on flat space,
the (mass)$^2$ is given by
\beq
     m^2 = l_s^{-2} \left( n + O(l_s^2/R^2) \right).
\eeq
%Substituting this into (\ref{spec}) or similar formulae
%for tensor or fermionic fields, we find that this
%single string state gives an eigenstate of $H=\frac{1}{2R}(P_0+K_0)$, 
%with the eigenvalue
%\beq
%\omega  \simeq
%      \frac{n^{\frac{1}{2}}}{l_s}  + 
%            \frac{1}{R} \left[ \frac{1}{2}(p+1) + k 
%  + O\left(\frac{1}{m_s R} \right) \right],~~(k=0,1,2,...).
%\eeq
Unlike the single particle supergravity states discussed in the previous
paragraph, string excitations need  not carry integral
eigenvalues of $H$ (in units of $R^{-1}$). As they
are not BPS particles, they are generically 
unstable in string perturbation theory. 

The free string spectrum in 10 dimensions gives
the Hagedorn density of states 
\beq
  S(E) \simeq  E l_s.
\label{string}
\eeq
Thus, the entropy of supergravity particles (\ref{fock})
becomes comparable to that of excited strings (\ref{string})
when
\beq
 (ER)^{\frac{9}{10}} \sim E l_s, 
\eeq
namely 
\beq
E \sim  m_s^{10}  R^9.
\eeq 
For $ m_s^{10}  R^9 < E $, excited strings dominate
the Hilbert space. 
The free string formula (\ref{string}) 
is reliable until the energy hits another transition
point $E \sim m_s/g_s^2 $. 
%We are assuming here that
%$m_s^{10} R^2 \ll m_P^8$. If this is not satisfied,
%there is no regime where the entropy shows the 
%Hagedorn behavior
%(\ref{string}).
We are assuming that $R^9 < l_s^9/g_s^2$, which is true in the 
't Hooft region. 

\medskip
\noindent
(3) $m_s/g_s^2  \ll E \ll m_P^8 R^7$; {\it Small Black Hole}

\smallskip

As we increase the energy, the gas of free strings starts
collapsing to make a black hole. The black hole can be described
by the classical supergravity when the horizon radius $r_+$ 
becomes larger than the string length $l_s$. Furthermore, if the
horizon size $r_+$ is smaller than $R$,
the geometry near the black hole can be approximated by the 10-dimensional
Schwarzschild solution. The energy $E$ and the entropy $S$ of such
a black hole is given by
\begin{eqnarray}
   E &\sim &   m_P^8 r_+^7 
\nonumber \\
  S & \sim & (m_P r_+)^8.
\label{tendformula}
\end{eqnarray}
Therefore, the entropy is estimated to be
\beq
    S(E) \sim  (E l_P )^{\frac{8}{7}}.
\label{smallbh}
\eeq
We can trust this estimate when $l_s \ll r_+ \ll R$,
namely $m_P^8 l_s^7 \ll E \ll m_P^8 R^7$.
Comparing this with the Hagedorn density of states
in the regime (2) given by (\ref{string}), we find that 
the transition to (\ref{smallbh}) takes place at
\beq
 E \sim {m_s \over g_s^2} .
\eeq
%(Once again, we are assuming $m_s^{10} R^2 \ll m_P^8$.
%If this inequality does not hold, we need to compare
%(\ref{smallbh}) with the gas of free gravitons
%(\ref{fock}).)
For $E \gg m_P^8 l_s^7$, the entropy formula
(\ref{smallbh}) is reliable and the black hole
entropy exceeds that of the gas of free strings. 
Therefore, in this regime, the Hilbert space
is dominated by black hole states. 

\medskip
\noindent
(4) $m_P^8 R^7 < E$; {\it Large Black Hole}

\smallskip
The above analysis assumes that the size of the black hole,
characterized by the horizon radius $r_+$, is small compared 
to the radii $R$ of \ads$_5$ and $S^5$. 
As we increase the energy, the radius $r_+$ grows 
and eventually becomes comparable to $R$. Beyond this point, we 
can no longer use the 10-dimensional Schwarzschild solution to 
estimate the number of states. According to (\ref{tendformula}),
the horizon size becomes comparable to $R$ when the energy of
the black hole reaches the scale $E  \sim m_P^8 R^7$. Beyond 
this energy scale,  we have to use a solution which is asymptotically 
$AdS_5$ \cite{Hawking:1983dh}, 
\beq
   ds^2 = - f(r) d\tau^2
+ \frac{1}{f(r)}dr^2
 + r^2 d\Omega_3^2,
\label{adsbh}
\eeq
where
\beq
  f(r) = 1 + \frac{r^2}{R^2} - \frac{r_+^{2}}{r^{2}}
 \left(1+ \frac{r_+^2}{R^2} \right),
\eeq
and $r=r_+$ is the location of the out-most horizon. 
By studying the asymptotic behavior of the metric,
one finds that the black hole carries the energy
\beq
  E \sim  
  \frac{r_+^2}{{\bf l}_P^3}
\left( 1 + \frac{r_+^2}{R^2} \right).
\label{adsbhenergy}
\eeq
Here ${\bf l}_P$ is the five-dimensional Planck length,
related to the 10-dimensional Planck scale $l_P$
and the compactification scale $R$ as
\beq
  {\bf l}_P^3 = l_P^8 R^{-5}.
\eeq
The entropy of the \ads\ Schwarzschild solution is
given by
\beq
  S \sim \left(\frac{r_+}{{\bf l}_P} \right)^3.
\eeq
For $r_+ \gg R$, (\ref{adsbhenergy}) becomes
$E \sim  r_+^{4}{\bf l}_P^{-3} R^{-2}$, and the 
entropy as a function of energy is
\beq
  S \sim \left(\frac{ER^2}{{\bf l}_P}\right)^{\frac{3}{4}}
= \left(\frac{R}{l_P} \right)^{2} (ER)^{\frac{3}{4}}.
\label{adsbhentropy}
\eeq
As the energy increases, the horizon size expands as
$  R \ll r_+ \rightarrow \infty $,
and the supergravity approximation continues to be reliable.
For $E \rightarrow \infty$, the only stringy and quantum gravity
corrections are due to the finite size $R$ of
the \ads\ radius of curvature and of the compact space, and such
corrections are suppressed by factors of $l_s/R$ and $l_P/R$.
The leading $l_s/R$ corrections to (\ref{adsbhentropy})
were studied in \cite{Gubser:1998nz}, and
found to be of the order of $(l_s/R)^3$.

\medskip
\noindent
$\circ$ Summary

\smallskip
The above analysis gives the following picture about the
structure of the Hilbert space of string theory on \ads\
when $ l_s\ll R $ and $g_s \ll 1$.


\begin{figure}[htb]
\begin{center}
\epsfxsize=3.8in\leavevmode\epsfbox{hfig8.eps}
\end{center}
\caption{The behavior of the entropy $S$ as a function of
the energy $E$ in \ads$_5$ .}
\label{ent}
\end{figure} 



\noindent
(1) For energies $E \ll m_s$, the Hilbert space is the
Fock space of supergravity particles and the spectrum is
quantized in the unit of $R^{-1}$. For 
$E \ll m_s^{10} R^9$, the entropy is given by that
of the gas of free supergravity particles in 10 dimensions: 
\beq
S \sim (ER)^{\frac{9}{10}}.
\eeq 

\noindent
(2)  For $m_s^{10} R^9 < E \ll m_P^8 l_s^7 $, stringy
excitations become important, and the entropy grows
linearly in energy: 
\beq
S \sim El_s.
\eeq 

\noindent
(3) 
For $m_P^8 l_s^7 \ll E \ll m_P^8 R^7$,
the black hole 
starts to show up in the Hilbert space. 
For $E \ll m_P^8 R^7$, the size of the black hole
horizon is smaller than $R$, and the entropy is given by
that of the 10-dimensional Schwarzschild solution:
\beq
S \sim  (El_P)^{\frac{8}{7}}.
\eeq

\noindent
(4) For $m_P^8 R^7 < E$, the size of the black hole
horizon becomes larger than $R$. We then have to use
the \adsp\ Schwarzschild solution, and the entropy is
given by:
\beq
S \sim \left(\frac{R}{l_P} \right)^{2}
(ER)^{\frac{3}{4}}.
\label{largebhent}
\eeq

\noindent
The behavior of the entropy is depicted in figure
\ref{ent}. 

In the small black hole regime (3),
the system has a negative specific heat. This
corresponds to the well-known instability
of the flat space at finite temperature \cite{Gross:1982cv}. 
On the other hand, the \ads\ Schwarzschild solution
has a positive specific heat and it is thermodynamically stable.
This means that, if we consider a canonical ensemble, the
free string regime (2) and the small black hole regime
(3) will be missed. When set in contact
with a heat bath of temperature $T \sim m_s$, the
system will continue to absorb heat until its energy
reaches $E \sim m_P^8 R^7$, 
the threshold of the large black hole regime (4). 
In fact the jump from (1) to (4) takes place at much
lower temperature since 
the temperature equivalent of $E \sim m_P^8 R^7$
derived from (\ref{largebhent}) in the regime (4)
is $T \sim R^{-1}$. Therefore, once the temperature is
raised to $T \sim R^{-1}$ a black hole forms.
 The behavior of the canonical
ensemble will be discussed in more detail in section \ref{FiniteT}. 

Finally let us notice that in the case that $g_s \sim 1$ we 
do not have the Hagedorn phase, and we go directly from the gas of
gravitons to the small black hole phase. 



\subsection{Hilbert Space of Conformal Field Theory}

Next, let us turn to a discussion of the Hilbert space of
the CFT$_{p+1}$. The generator $J_{0,p+2}=
\frac{1}{2}(P_0+K_0)$ is the Hamiltonian of the CFT on $S^p$
with the unit radius. In the Euclidean CFT, the conformal
group $SO(2,p+1)$ turns into $SO(1,p+2)$ by the Wick rotation, 
and the Hamiltonian $\frac{1}{2}(P_0+K_0)$ and the dilatation 
operator $D$ can be rotated into each other by an internal 
isomorphism of the group. Therefore, if there is a conformal 
field $\phi_h(x)$ of dimension $h$ with respect to the dilatation 
$D$, then there is a corresponding eigenstate $|h \rangle$ of 
$\frac{1}{2}(P_0 + K_0)$
on $S^p$ with the same eigenvalue $h$. In two-dimensional
conformal field theory, this phenomenon is well-known as
the state-operator correspondence, but in fact it holds
for any CFT$_{p+1}$ :
\beq
  \phi_h(x) \rightarrow |h \rangle = \phi_h(x=0) |0 \rangle. 
\eeq

As discussed in section \ref{chiralops}, in maximally 
supersymmetric cases there is a one-to-one 
correspondence between chiral primary operators of CFT$_{p+1}$ and
the supergravity particles on the dual \adsp\ $\times$ (compact
space). This makes it possible to identify a state in the Fock space
of the supergravity particles on \ads\ with a state in the
CFT Hibert space generated by the chiral primary fields. 

To be specific, let us consider the ${\cal N}=4$ $SU(N)$ super Yang-Mills
theory in four dimensions and its dual, type IIB 
string theory on \ads$_5 \times S^5$. 
The string scale $m_s$ and the 10-dimensional
Planck scale $m_P$ are related to the
gauge theory parameters, $g_{YM}$ and $N$, by
\beq
   m_s \simeq (g_{YM}^2 N)^{\frac{1}{4}} R^{-1}, ~~ 
m_P  \simeq N^{\frac{1}{4}} R^{-1} .
\eeq
The four energy regimes of string theory on $AdS_5\times S^5$
 are translated into the gauge theory energy scales 
(measured in the units of the inverse $S^3$ radius) in the 't Hooft limit as
follows:

\medskip
\noindent
(1) $E \ll (g_{YM}^2N)^{\frac{1}{4}}$

The Hilbert space consists of the chiral primary
states, their superconformal descendants and their
products.  Because of the large-$N$ factorization, 
a product of gauge invariant operators 
receives corrections only at subleading orders in the $1/N$
expansion. 
This fits well with the supergravity
description of multi-graviton states, where we
estimated that their energy $E$
becomes comparable to the gravitational potential
when $E \sim m_P^8 R^7$, which in the gauge theory
scale corresponds to $E \sim N^2$. 
The entropy for $1 \ll E \ll (g_{YM}^2 N)^{\frac{1}{4}}$
is then given by
\beq
  S \sim E^{\frac{9}{10}}.
\eeq

\noindent
(2) $(g_{YM}^2N)^{\frac{1}{4}} < E \ll (g_{YM}^2
    N)^{-\frac{7}{2}} N^2$

Each single string state is identified with a single
trace operator in the gauge theory. Supergravity
particles correspond to chiral primary states and
stringy excitations to non-chiral primaries.
Since stringy excitations have an energy $\sim m_s$,
the \ads/CFT correspondence
requires that non-chiral conformal fields have to 
have large anomalous dimensions 
$\Delta \sim m_s R = (g_{YM}^2 N)^{\frac{1}{4}}$.
In the 't Hooft limit ($N \gg (g_{YM}^2 N)^\gamma$
for any $\gamma$), we can consider
the regime $(g_{YM}^2 N)^{\frac{5}{2}} <
E \ll (g_{YM}^2 N)^{-\frac{7}{2}} N^2$ where
the entropy shows the Hagedorn behavior
\beq
  S \sim (g_{YM}^2 N)^{-\frac{1}{4}} E.
\eeq
Apparently, the entropy in this regime is dominated by
the non-chiral fields.


\noindent
(3) $ (g_{YM}^2
    N)^{-\frac{7}{2}} N^2 < E < N^2$

The string theory Hilbert space consists of
states in the small black hole. It would
be interesting to find a gauge theory interpretation 
of the 10-dimensional Schwarzschild black hole.
The entropy in this regime behaves as
\beq
  S \sim N^{-\frac{2}{7}} E^{\frac{8}{7}}.
\eeq

\noindent
(4) $N^2 < E$

The string theory Hilbert space consists of
states in the large black hole. The entropy is given
by 
\beq
 S \sim N^{\frac{1}{2}} E^{\frac{3}{4}}.
\eeq
The $E^{\frac{3}{4}}$ scaling of the entropy is what one expects  
for a conformal field theory in $(3+1)$ dimensions at high energies
(compared to the radius of the sphere). 
It is interesting to note that the $N$ dependence of $S$
is the same as that of $N^2$ free particles in 
$(3+1)$ dimensions, although the precise numerical 
coefficient in $S$ differs from the one that is obtained
from the number of particles in the ${\cal N}=4$
Yang-Mills multiplet by a numerical factor \cite{Gubser:1996de}.





