%% This is a plain TeX file
%%
\magnification=1200
\hsize=6.5 true in
\vsize=8.7 true in
\input epsf.tex

%%%%%%%%%%%%%%%%%%%%%%%%%%%%%%%%%%%%%%%%%%%%%%%%%%%%%

\font\eightrm=cmr8
\font\sevenrm=cmr7
\font\sixrm=cmr6
\font\fiverm=cmr5
\font\eighti=cmmi8
\font\sixi=cmmi6
\font\fivei=cmmi5
\font\eightsy=cmsy8
\font\sixsy=cmsy6
\font\fivesy=cmsy5
\font\tenex=cmex10
\font\eightit=cmti8
\font\eightsl=cmsl8
\font\eighttt=cmtt8
\font\eightbf=cmbx8
\font\sixbf=cmbx6
\font\fivebf=cmbx5
\font\eightmsb=msbm8
\font\sixmsb=msbm6

\def\eightpoint{\def\rm{\fam0\eightrm}% switch to 8-point type
\textfont0=\eightrm \scriptfont0=\sixrm 
\scriptscriptfont0=\fiverm
  \textfont1=\eighti \scriptfont1=\sixi 
\scriptscriptfont1=\fivei
  \textfont2=\eightsy \scriptfont2=\sixsy 
\scriptscriptfont2=\fivesy
  \textfont3=\tenex \scriptfont3=\tenex 
\scriptscriptfont3=\tenex
  \textfont\itfam=\eightit  \def\it{\fam\itfam\eightit}%
  \textfont\slfam=\eightsl  \def\sl{\fam\slfam\eightsl}%
  \textfont\ttfam=\eighttt  \def\tt{\fam\ttfam\eighttt}%
  \textfont\bffam=\eightbf  \scriptfont\bffam=\sixbf
 \scriptscriptfont\bffam=\fivebf  \def\bf{\fam\bffam\eightbf}%
  \normalbaselineskip=9pt
  \setbox\strutbox=\hbox{\vrule height7pt depth2pt width0pt}%
  \let\sc=\sixrm  \normalbaselines\rm}

%%%%%%%%%%%%%%%%%%%%%%%%%%%%%%%%%%%%%%%%%%%%%%%%%%%%%%%%

\font\dotless=cmr10 %for the roman i or j to be
                    %used with accents on top.
                    %(\dotless\char'020=i)
                    %(\dotless\char'021=j)
\font\itdotless=cmti10
\def\itumi{{\"{\itdotless\char'020}}}
\def\itumj{{\"{\itdotless\char'021}}}
\def\umi{{\"{\dotless\char'020}}}
\def\umj{{\"{\dotless\char'021}}}
\font\smaller=cmr5
\font\boldtitlefont=cmb10 scaled\magstep2
\font\smallboldtitle=cmb10 scaled \magstep1
\font\ninerm=cmr9
\font\dun=cmdunh10 %scaled\magstep1

\footline={\hfil {\tenrm VI.\folio}\hfil}

\def\eps{{\varepsilon}}
\def\Eps{{\varepsilon}}
\def\kap{{\kappa}}
\def\lam{{\lambda}}
\def\Lam{{\Lambda}}
\def\mynabla{{\nabla\!}}

\def\Bmu{{B_{\mu\nu}}}
\def\Gmu{{G_{\mu\nu}}}

\def\xdot{{\dot x}}
\def\xddot{{\ddot x}}

\def\undertext#1{$\underline{\vphantom{y}\hbox{#1}}$}
\def\nspace{\lineskip=1pt\baselineskip=12pt%
     \lineskiplimit=0pt}
\def\dspace{\lineskip=2pt\baselineskip=18pt%
     \lineskiplimit=0pt}

\def\half{\raise4.5pt\hbox{{\vtop{\ialign{##\crcr
  \hfil\rm $1$\hfil\crcr
   \noalign{\nointerlineskip}--\crcr
   \noalign{\nointerlineskip\vskip-1pt}$2$\crcr}}}}}
\def\third{\raise4.5pt\hbox{{\vtop{\ialign{##\crcr
  \hfil\rm $1$\hfil\crcr
  \noalign{\nointerlineskip}--\crcr
  \noalign{\nointerlineskip\vskip-1pt}$3$\crcr}}}}}
\def\fourth{\raise4.5pt\hbox{{\vtop{\ialign{##\crcr
  \hfil\rm $1$\hfil\crcr
  \noalign{\nointerlineskip}--\crcr
  \noalign{\nointerlineskip\vskip-1pt}$4$\crcr}}}}}
\def\sixth{\raise4.5pt\hbox{{\vtop{\ialign{##\crcr
  \hfil\rm $1$\hfil\crcr
  \noalign{\nointerlineskip}--\crcr
  \noalign{\nointerlineskip\vskip-1pt}$6$\crcr}}}}}
\def\eighth{\raise4.5pt\hbox{{\vtop{\ialign{##\crcr
  \hfil\rm $1$\hfil\crcr
  \noalign{\nointerlineskip}--\crcr
  \noalign{\nointerlineskip\vskip-1pt}$8$\crcr}}}}}

\def\w{{\mathchoice{\,{\scriptstyle\wedge}\,}
  {{\scriptstyle\wedge}}
  {{\scriptscriptstyle\wedge}}{{\scriptscriptstyle\wedge}}}}
\def\Le{{\mathchoice{\,{\scriptstyle\le}\,}
{\,{\scriptstyle\le}\,}
{\,{\scriptscriptstyle\le}\,}{\,{\scriptscriptstyle\le}\,}}}
\def\Ge{{\mathchoice{\,{\scriptstyle\ge}\,}
{\,{\scriptstyle\ge}\,}
{\,{\scriptscriptstyle\ge}\,}{\,{\scriptscriptstyle\ge}\,}}}
\def\plus{{\hbox{$\scriptscriptstyle +$}}}
\def\xdot{\dot{x}}
\def\Condition#1{\item{#1}}
\def\Firstcondition#1{\hangindent\parindent{#1}\enspace
     \ignorespaces}
\def\Proclaim#1{\medbreak
  \medskip\noindent{\bf#1\enspace}\it\ignorespaces}
\def\finishproclaim{\par\rm
     \ifdim\lastskip<\smallskipamount\removelastskip
     \penalty55\medskip\fi}
\def\Item#1{\par\smallskip\hang\indent%
  \llap{\hbox to\parindent {#1\hfill\enspace}}\ignorespaces}
\def\vrulesub#1{{\,\vrule height7pt depth5pt}_{\,#1}}
\def\underbrake#1#2{\mathop{#1}\limits_{\raise3pt
  \hbox{%
\vrule height 3pt depth 0pt
  %\kern.1pt
  \hbox to #2{\hrulefill}
  \kern-3.4pt
  \vrule height 3pt depth 0pt}}}


\def\Diff{{\rm Diff}}  \def\expt{{\rm expt}}
\def\cm{{\rm cm}} \def\SO{{\rm SO}}
\def\mass{{\rm mass}} \def\diag{{\rm diag}}
\def\Weyl{{\rm Weyl}} 

\def\cbar{\bar{c}}
\def\bbar{\bar{b}}
\def\epsbar{\bar{\varepsilon}}
\def\Gammabar{\bar{\Gamma}}
\def\Sbar{\bar{S}}


\def\Ptil{\widetilde{P}}
\def\betatil{\tilde{\beta}}

\def\dbR{{\Bbb R}}

%These two files (in this order!!) are necessary
%in order to use AMS Fonts 2.0 with Plain TeX

\input amssym.def
\input amssym.tex

%Capital roman double letters(Blackboard bold)
\def\db#1{{\fam\msbfam\relax#1}}

\def\dbA{{\db A}} \def\dbB{{\db B}}
\def\dbC{{\db C}} \def\dbD{{\db D}}
\def\dbE{{\db E}} \def\dbF{{\db F}}
\def\dbG{{\db G}} \def\dbH{{\db H}}
\def\dbI{{\db I}} \def\dbJ{{\db J}}
\def\dbK{{\db K}} \def\dbL{{\db L}}
\def\dbM{{\db M}} \def\dbN{{\db N}}
\def\dbO{{\db O}} \def\dbP{{\db P}}
\def\dbQ{{\db Q}} \def\dbR{{\db R}}
\def\dbS{{\db S}} \def\dbT{{\db T}}
\def\dbU{{\db U}} \def\dbV{{\db V}}
\def\dbW{{\db W}} \def\dbX{{\db X}}
\def\dbY{{\db Y}} \def\dbZ{{\db Z}}

\font\teneusm=eusm10  \font\seveneusm=eusm7 
\font\fiveeusm=eusm5 
\newfam\eusmfam 
\textfont\eusmfam=\teneusm 
\scriptfont\eusmfam=\seveneusm 
\scriptscriptfont\eusmfam=\fiveeufm 
\def\scr#1{{\fam\eusmfam\relax#1}}


%Upper-case Script Letters:

\def\scrA{{\scr A}}   \def\scrB{{\scr B}}
\def\scrC{{\scr C}}   \def\scrD{{\scr D}}
\def\scrE{{\scr E}}   \def\scrF{{\scr F}}
\def\scrG{{\scr G}}   \def\scrH{{\scr H}}
\def\scrI{{\scr I}}   \def\scrJ{{\scr J}}
\def\scrK{{\scr K}}   \def\scrL{{\scr L}}
\def\scrM{{\scr M}}   \def\scrN{{\scr N}}
\def\scrO{{\scr O}}   \def\scrP{{\scr P}}
\def\scrQ{{\scr Q}}   \def\scrR{{\scr R}}
\def\scrS{{\scr S}}   \def\scrT{{\scr T}}
\def\scrU{{\scr U}}   \def\scrV{{\scr V}}
\def\scrW{{\scr W}}   \def\scrX{{\scr X}}
\def\scrY{{\scr Y}}   \def\scrZ{{\scr Z}}

\def\gr#1{{\fam\eufmfam\relax#1}}

%Euler Fraktur letters (German)
\def\grA{{\gr A}}	\def\gra{{\gr a}}
\def\grB{{\gr B}}	\def\grb{{\gr b}}
\def\grC{{\gr C}}	\def\grc{{\gr c}}
\def\grD{{\gr D}}	\def\grd{{\gr d}}
\def\grE{{\gr E}}	\def\gre{{\gr e}}
\def\grF{{\gr F}}	\def\grf{{\gr f}}
\def\grG{{\gr G}}	\def\grg{{\gr g}}
\def\grH{{\gr H}}	\def\grh{{\gr h}}
\def\grI{{\gr I}}	\def\gri{{\gr i}}
\def\grJ{{\gr J}}	\def\grj{{\gr j}}
\def\grK{{\gr K}}	\def\grk{{\gr k}}
\def\grL{{\gr L}}	\def\grl{{\gr l}}
\def\grM{{\gr M}}	\def\grm{{\gr m}}
\def\grN{{\gr N}}	\def\grn{{\gr n}}
\def\grO{{\gr O}}	\def\gro{{\gr o}}
\def\grP{{\gr P}}	\def\grp{{\gr p}}
\def\grQ{{\gr Q}}	\def\grq{{\gr q}}
\def\grR{{\gr R}}	\def\grr{{\gr r}}
\def\grS{{\gr S}}	\def\grs{{\gr s}}
\def\grT{{\gr T}}	\def\grt{{\gr t}}
\def\grU{{\gr U}}	\def\gru{{\gr u}}
\def\grV{{\gr V}}	\def\grv{{\gr v}}
\def\grW{{\gr W}}	\def\grw{{\gr w}}
\def\grX{{\gr X}}	\def\grx{{\gr x}}
\def\grY{{\gr Y}}	\def\gry{{\gr y}}
\def\grZ{{\gr Z}}	\def\grz{{\gr z}}

%\overfullrule=0pt

\parindent=20pt
\line{\dun --- DRAFT ---\hfill{\rm IASSNS-HEP-97/72}}

\bigskip\bigskip
\centerline{\boldtitlefont Lectures 9 and 10}
\medskip
\centerline{\smallboldtitle VI. Strings on General
Manifolds}

\medskip
\centerline{Eric D'Hoker}

\frenchspacing

\dspace
\bigskip
In the preceding lectures, we have quantized strings
in flat Minkowski space-time.
The existence of a Hilbert space of physical string
states (i.e. the consistent elimination
of negative norm states)
requires $\Weyl(\Sigma)$ and $\Diff(\Sigma)$
invariance of the full quantum theory.
This condition fixes the dimension of space-time to be
$D=26$ and results in the appearance of massless spin
$1$ and spin $2$ string states which behave like
Yang-Mills particles and gravitons, respectively.
The spectrum also contains massless dilatons and
anti-symmetric tensor states.

In the present lecture, we wish to extend the analysis
of the consistent elimination of negative norm states
to the case of arbitrary space-time manifolds, with
general metric $G_{\mu\nu}$, anti-symmetric tensor
$B_{\mu\nu}$ and dilaton $\Phi$, and possible other
background fields.
On physical grounds, we shall assume that the
space-times of interest have Minkowski signature
$(-,+,\cdots,+)$.
For simplicity, we assume that the dimension of
space-time $D$ is such that $D\Ge 4$ in order to avoid
dealing with certain peculiarities that arise in
$D=2,3$.
We expect again that the space-time dimension $D$ will
be fixed by the condition that all negative norm states
be consistently eliminated.
The precise value of $D$ where this happens may,
however, depend on the background fields $G_{\mu\nu}$,
$B_{\mu\nu}$ and $\Phi$, so we shall leave $D$ to be
determined later on.
We shall also assume that space-time $M$ has at least
$2$ non-compact dimensions, including the time
direction, so that scattering can be defined in $M$.

Given the signature of the space-time manifold $M$,
the Fock space of states produced by the quantization
of the map $x$ will contain again negative norm
states. 
These states are to be eliminated before a
Hilbert space of string states can be constructed.
To achieve elimination of negative norm states, it
is clearly necessary that the worldsheet 
quantum field theory of
the map $x$, defined in terms of the background
$G_{\mu\nu}$, $B_{\mu\nu}$, $\Phi$ (and perhaps other
fields), together with the Faddeev-Popov ghost fields
$b$, $c$, $\bbar$, $\cbar$, be invariant under
$\Diff(\Sigma)$ and $\Weyl(\Sigma)$ transformations.
In this case, the BRST quantization scheme of \S{IV} may be
set up, just as in the case of flat Minkowski $M$.

For certain classes of space-time manifolds $M$, this
condition can be shown to be also sufficient.
When $M=\dbR^2\times M_{D-2}$, where $\dbR^2$
includes the time direction, so that the metric $G$ on
$M_{D-2}$ is positive, we have the usual Hilbert space
construction for the states arising from the $M_{D-2}$ part
of space-time, since these states have positive norm. 
One can show that the BRST
quantization scheme for all of $M$ leads to a Hilbert
space.
The proof proceeds by passing to the lightcone gauge
variables, which may be defined for the $\dbR^2$ part
of $M$ just as in the case of flat $M$.
In lightcone gauge, only positive norm, physical states
survive, and may be mapped onto BRST cohomology classes
in a one-to-one fashion, just as for flat $M$.
For more general $M$, I am not aware of a proof that
a Hilbert space may be constructed, but it is
generally believed to be the case, at least ``under
certain conditions''.
For example, a space-time $M$ with a black hole is
expected to cause problems.

Henceforth, we shall restrict to space-time manifolds
$M$ and background fields, such that
$\Weyl(\Sigma)\ltimes \Diff(\Sigma)$ invariance of the
$x$, $b$, $c$, $\bbar$, $\cbar$ system leads to a
Hilbert space. 
Equivalently, the quantum field theory of the $x$
field is a conformal field theory of central charge
$c=26$.
Thus, we have reduced the conditions of consistent string
propagation in space-time manifolds $M$ background
fields to a problem in quantum field theory: that of
identifying and 
parametrizing conformal field theories of the map $x$
with central charge $c=26$.

\bigskip\noindent
\Item{\bf A)} {\bf Perturbation Theory Around General
Field Configurations}

Before launching into a study of the conditions for
conformal invariance of the field
theory of the map $x$ with general space-time $M$,
general $G_{\mu\nu}$, $B_{\mu\nu}$, $\Phi$, (and
possibly other background fields), we wish to clarify
the physical significance of string quantization in
non-flat background field configurations.
We proceed by comparison with perturbation theory in
quantum field theory (QFT).

In QFT, we start with canonical fields $\phi$ and a
classical action $S[\phi]$ which we assume to be
independent of $\hslash$.
Perturbation theory (in the loop expansion) is carried
out by choosing a field configuration $\phi_0$, and
expanding the field $\phi$ in powers of
$\sqrt{\hbar\,\,}$: \
$\phi=\phi_0+\sqrt{\hbar\,\,}\varphi$.
The Feynman rules are then obtained by an expansion
of the action $S[\phi]$ in powers of
$\sqrt{\hbar\,\,}$:
$$
\matrix{\displaystyle
{1\over\hslash}\,S[\phi] ={1\over\hslash}\,S[\phi_0] 
  &\displaystyle\!\!\! +{1\over\sqrt{\hslash\,\,}}\int\varphi
  {\delta S\over\delta\phi}\,[\phi_0]+\half
  \int\!\!\!\int
\varphi\varphi{\delta^2S\over\delta\phi^2}\,[\phi_0]\hfill
\displaystyle +{\textstyle 1\over\textstyle 6}
  \sqrt{\hslash\,\,}
  \int\!\!\!\int\!\!\!\int\varphi\varphi\varphi\,{\delta^3
S\over\delta\phi^3}\,[\phi_0]+\cdots\hfill\cr
\noalign{\bigskip}
\displaystyle 
x\raise2pt\hbox to 35pt{\hrulefill}\!\bullet\hfill
 &\qquad\displaystyle 
{1\over\sqrt{\hslash\,\,}}{\delta S\over\delta\phi(x)}\,
 [\delta_0]\hfill\cr
\noalign{\bigskip}
\displaystyle x \raise2pt\hbox to 35pt{\hrulefill} y\hfill
&\qquad\displaystyle \left({\delta^2 S\over
\delta\phi(x)\delta\phi(y)}\,
  [\phi_0]\right)^{-1}\hfill\cr
\noalign{\bigskip}
\displaystyle x\raise2pt\hbox to 35pt{\hrulefill}
\raise4pt\hbox{$\diagup$}\,\raise10pt\hbox{$y$} \kern-16pt
\lower4.25pt\hbox{$\diagdown$}\,
\lower8pt\hbox{$z$}\hfill &\qquad\displaystyle
\sqrt{\hslash\,\,}{\delta^3 S\over
\delta\phi(x)\delta\phi(y)\delta\phi(z)}\,[\phi_0]\hfill\cr
&\vdots\hfill\cr}
\kern-31pt\raise20pt\hbox{(6.A.1)}
$$
Both the Feynman rules for the $\varphi$ propagator
and for the $\varphi$ interaction vertices depend upon
the background field $\phi_0$ in a non-trivial way, as
is evident from the Feynman rule expansion in (6.A.1).

In string theory, we have no equivalent of the
canonical field $\phi$ and of the classical action
$S[\phi]$ that is as well understood as in QFT.
(Many attempts at constructing a string field theory
formulation have been made over the years, but have
met with only limited success.)
In string theory, our starting point consists only of 
the Feynman rules,
which are given in terms of a summation over surfaces
$\Sigma$, and maps $\Sigma\to M$, and which yield on-shell
transition amplitudes only.
The r\^{o}le of $\phi_0$ in QFT is played in string
theory by the background fields $\Gmu$,
$\Bmu$ $\Phi$, and possibly other fields.
And indeed, the Feynman rules of string theory depend
upon these background fields, just as the Feynman
rules of QFT depended upon $\phi_0$.

Fortunately, much crucial physical information may be
reconstructed from the Feynman rules: we already know
that they yield the transition amplitudes for
physical states, i.e. the $S$-matrix (at least in
perturbation theory).
In particular, it is clear from QFT that 
certain transition amplitudes yield direct information
on the background $\phi_0$.
The most important one for our purposes is that {\it
$\phi_0$ is a classical solution}, i.e. $\delta
S/\delta\phi[\phi_0]=0$ if and only if all {\it
tadpole graphs are zero to tree level}.
The string theory equivalent of this criterion is that
a given background field configuration $\Gmu$, $\Bmu$,
$\Phi$ (and possibly other fields) are a {\it solution
to the string equations of motion to tree level} (i.e.
classically) {\it if and only if all tadpole graphs
vanish}.
This criterion gives us a powerful tool to investigate
whether a certain configuration is a classical
solution or not. 
In QFT, we do not usually investigate the Feynman
rules to decide whether a configuration $\phi_0$ is a
solution, because we can directly check whether
$\delta S/\delta\phi[\phi_0]$ vanishes or not.
In string theory, this alternative is not available,
at present.

We have established previously that the existence
of a Hilbert space of string states requires the map
$x$ to define a conformal field theory of central
charge $c=26$.
It is a simple consequence of conformal invariance on
the sphere (established for $x$ in flat space-time $M$
in \S{III}) that the $1$-point function of any primary
conformal field of weight
$(1,1)$, describing an on-shell physical state,
must vanish to tree level.
Combining this result with that of the preceeding
paragraph, we discover that if $x$ defines a conformal
field theory of central charge $26$, for a certain
background configuration $\Gmu$, $\Bmu$, $\Phi$, (and
possibly other fields), then this configuration is a
classical (tree level) solution to the string
equation!
In other words, demanding that the backgrounds $\Gmu$,
$\Bmu$, $\Phi\,\,\,$, $\ldots$ define a $c=26$ CFT
will enforce on $\Gmu$, $\Bmu$, $\Phi$ (and possibly
other fields), 
the string equations of motion and guarantee the
existence of a Hilbert space of physical states.
While in QFT, we knew the equations of motion for
$\phi_0$ directly from the action, in string theory the
equations on the background fields will be novel.

We finish this subsection by noticing that other
useful information may be obtained from the Feynman
rules.
First, a configuration $\Gmu$, $\Bmu$,
$\Phi\,\,,\ldots$ is a solution to the full
(perturbative) quantum string equations if the sum of
all tadpoles cancels to all orders.
This is in analogy with QFT, where a solution to the
full (perturbative) quantum fields equations is an
extremum of the full effective action, so that the sum
of all tadpole graphs cancels.
Secondly, the presence of tachyons in the spectrum
signals that the background around which the
perturbative expansion was formulated is not a local
minimum.

\vfill\eject

\noindent
\Item{\bf B)} {\bf Renormalization of Generalized
Non-linear sigma Models}

Let $M$ be a Riemannian manifold of dimension $D$,
with metric $\Gmu$, and fields $\Bmu$, $\Phi$ defined
in \S{I}.
The starting point for string dynamics in the
background of these fields is the generalized 
non-linear sigma model action for the maps
$x\colon\,\Sigma\to M$, given by $S=S_{G,B}+S_\Phi$ and
$$
\eqalign{
S_{G,B}[x,g] &={1\over
8\pi\ell^2}\int\nolimits_{\Sigma}d\mu_g \partial_m
x^\mu\partial_n x^\nu\{g^{mn}\Gmu (x)+\varepsilon^{mn}
\Bmu(x)\}\cr
S_\Phi[x,g] &={1\over 2\pi}\int\nolimits_{\Sigma}
  d\mu_g R_g\Phi(x)\cr}
\eqno{(6.B.1)}
$$
Recall that the term involving $\Bmu$ is the pullback
of a two form $B=\Bmu dx^\mu\w dx^\nu$ under the map
$x$, written out in coordinates.
Here, $\varepsilon^{mn}$ is the tensor defined by
$$
\varepsilon^{mn}\equiv\epsbar^{mn}/\sqrt{\det\,g_{pq}\,\,}
\qquad\qquad
\epsbar^{01}=-\epsbar^{10}=1\,\,.
\eqno{(6.B.2)}
$$
We have re-exhibited the dependence on the string
tension $\kappa$, introduced in \S{I} via the
parameter $\ell$.
It is standard
notation to express $\kappa$ in terms of the {\it Regge 
slope parameter} $\alpha'$; we find it convenient to
use the Planck length scale $\ell$, instead, 
with the following
relations
$$
\kappa={1\over 4\pi\alpha'}={1\over
8\pi\ell^2}\,\,,
\qquad\qquad
\alpha'=2\ell^2\,\,,\qquad \ell>0\,\,.
\eqno{(6.B.3)}
$$
It is possible to include a further term in $S$
corresponding to the dynamics of the tachyon 
state, to which we shall associated the field $U$:
$$
S_U[x,g]={1\over 4\pi}\int\nolimits_{\Sigma}
d\mu_g U(x)\eqno{(6.B.4)}
$$

Recall from Lecture 1 that the classical action $S$ is
invariant under
$$
\matrix{
{\rm (a)} &\quad \Diff(\Sigma)\hfill \cr
\noalign{\bigskip}
{\rm (b)} &\quad \Diff(M)\hfill \cr
\noalign{\bigskip}
{\rm (c)} &\quad
U(1)_B  \hbox{ gauge invariance, acting by $B\to
B+d\gamma$, for some $\gamma\in\Omega^{(1)}(M)$.}\cr}
$$
$\Weyl(\Sigma)$ transformations on $g$ leave $S_{G,B}$
invariant, but will develop anomalies upon
quantization.
$S_\Phi$ and $S_U$ fail to be Weyl invariant, even
classically.

We naturally have two notions of dimension in the
problem:
the worldsheet or $\Sigma$-dimension, and the
space-time or $M$-dimension. 
The $\Sigma$-dimension of
$x$, $g$, $G$, $B$, $\Phi$ and $\ell$ vanishes, while
that of $\partial$ and $R_g$ is $1$ and $2$,
respectively.
The $M$-dimension of $x$ and $\ell$ is $-1$, and
that of $g$, $G$, $B$, $\Phi$ vanishes.

We begin by discussing the renormalization of the
quantum field theory defined by $S$, for general
fields $\Gmu$, $\Bmu$, $\Phi$.
We shall not discuss here the effects of including the
tachyon field in $S$.
We shall see that in certain regularization schemes,
such as dimensional regularization, its contribution
can be ignored.
Also, in the superstring, the tachyon will be absent
anyway.

\Proclaim{\rm 1) {\it Low Energy or $\alpha'$-expansion}.}
\finishproclaim

The parameter $\ell$ is a very small length-scale (see
Prob. 5), compared to scales accessible at present
(whose order of magnitude we denote by $L_{\expt}$)
$$
\ell\sim 10^{-33}\cm\qquad\qquad\qquad\qquad
L_{\expt}\sim 10^{-17}\cm\,\,.\eqno{(6.B.5)}
$$
Thus, we consider the quantum theory defined by $S$ in
an expansion in powers of $\ell$ (or equivalently
in powers of $\alpha'$), and we
are guaranteed that this approximation will be
reliable if we just retain the leading contributions.
The $\alpha'=2\ell^2$ expansion is an expansion in increasing
powers of $M$-derivatives on the fields, hence the name
{\it low-energy expansion}.

If it were only for $S_{G,B}$, the parameter
$\ell^2$ would play
precisely the same r\^{o}le as $\hbar$ plays in quantum field
theory, and the expansion is powers of $\ell$ would
yield a loop expansion.
$S_\Phi$ has a different dependence in $\ell$ though,
as each factor of $\Phi$ effectively contributes an
extra factor of $\ell^2$, and this
will rearrange the orders of the loop expansion.
Perturbation theory may be carried out as usual by
letting $x=x_0+\ell y$ for some fixed reference
configuration $x_0$.
Expansion of the fields $G$, $B$, $\Phi$, for example
of
$$
\Gmu(x)=\Gmu(x_0)+\ell\partial_\kap\Gmu(x_0)y^\kap
+\half \ell^2\partial_\kap\partial_\lam\Gmu
(x_0)y^\kap y^\lam+\cdots
\eqno{(6.B.6)}
$$
reveals that $S$ is a quantum field theory with an
infinite number of couplings.
It is seen from (6.B.6) that the independent couplings
of this quantum field theory correspond to the
successive derivatives of the fields $G$, $B$ and
$\Phi$ at the expansion point $x_0$.

\vfill\eject

\Proclaim{\rm 2) {\it Regularization schemes}.}
\finishproclaim

It is possible to choose regularization schemes that
preserve the classical symmetries $(\Diff(\Sigma),
\Diff(M), U(1)_B)$ of $S$.
We shall always assume that such a regulator is used,
and give here two examples.
It is of course not possible to choose a
regularization scheme that also preserves
$\Weyl(\Sigma)$ invariance.
This failure precisely results in the appearance of the
Weyl anomaly as was already discovered for flat $M$ and
vanishing $B$ and $\Phi$ in \S{IV}.

\medskip\noindent
a)\enspace
Dimensional regularization consists in replacing the
dimensionlity of $\Sigma$ by an arbitrary complex number
$2-\eps$, and multiplying the action $S$ by a factor of
$\Lam^{-\eps}$. 
Here $\Lam$ is the dimensional 
regularization renormalization scale with
$\Sigma$-dimension $1$.
Dimensional regularization has the advantage that it
never generates quadratic divergences, so that if we start
with $S_U=0$ (no contribution of the
tachyon to the action), then no tachyon dependence will ever
be generated upon renormalization.
Dimensional regularization has the disadvantage that
the $\Eps^{mn}$ tensor of (6.B.2) is specifically
$2$-dimensional, so that an additional ad hoc
prescription for its regularization 
must be provided, and the results should be verified using the
Ward identities to make sure that the ad hoc
prescription is consistent with all symmetries.

\medskip\noindent
b)\enspace
Heat-kernel methods have already been used in the
definition of determinants resulting from Gaussian
quantum theories.
They may also be used for QFT with non-trivial
interactions by 
replacing its propagator by the heat-kernel
regularization propagator.
Let $\Delta_0$ be the second order operator on the
field $y$ at
$x_0$, and $\Delta_0^{-1}$ the associated propagator
for $y$.
We define the heat-kernel propagator by using a small
real constant $\eta>0$:
$$
\Delta_\eta^{-1}=\Delta_0^{-1}\cdot
e^{-\Delta_0\eta^2}\,\,.\eqno{(6.B.7)}
$$

\Proclaim{\rm 3) {\it $\Diff(M)$ and $U(1)_B$ invariant
renormalization}.}
\finishproclaim

Since $x$ and $y$ have zero $\Sigma$-dimension, all
terms listed in (6.B.1) for
$S$ have $\Sigma$-dimension $2$ and thus $S$
is power counting renormalizable.
Under the renormalization group flow, the change in
each one of the infinite number of coupling constants
(i.e. the Taylor coefficients of the expansion of
$G$, $B$, $\Phi$ in powers of $y$ as shown for $G$ in
(6.B.6)) 
will be given by a beta function associated with that
coupling constant.
An infinite number of renormalization conditions is
required to specify the initial condition of the
theory completely.
These conditions amount to providing the functions
$\Gmu(x)$, $\Bmu(x)$ and $\Phi(x)$ at some reference
renormalization scale which we denote by $\Lam_0$.
For the metric, for example, the renormalization group
equations will read:
$$
\cases{
\Lam{d\over d\Lam}\Gmu(x)=\beta_{\mu\nu}(x) &\cr
\noalign{\bigskip}
\Gmu(x)\vrulesub{\Lam_0}=\Gmu^0(x) &\cr}
\eqno{(6.B.8)}
$$
Here $\beta_{\mu\nu}(x)$ depends on $x$ only through
the dependence of $\beta_{\mu\nu}$ on the fields
$G_{\mu\nu}$, $B_{\mu\nu}$ and $\Phi$, as well as the
derivatives of these fields.

Now, we impose upon the renormalization procedure that
it preserve the $\Diff(M)$ and $U(1)_B$ invariance of $S$.
The existence of a regulator that preserves these
symmetries guarantees that there will be no anomalies
in these symmetries.
This implies that $\beta_{\mu\nu}$ must be a
$U(1)_B$-invariant, symmetric rank $2$ $\Diff(M)$ tensor.
Similarly for the $\beta$-functions of the fields $B$
and $\Phi$.

We conclude with a remark on the nature of the
renormalizability of the action $S$.
Customarily, renormalizable QFT's are power counting
renormalizable, but depend only upon a finite number of
input parameters, which may be viewed as initial
conditions on the renormalization group equations.
In four dimensional QFTs of particle dynamics, the
fact that only a finite number of input parameters is
needed renders the theory physically predictive.
Indeed, once those parameters have been determined
(even approxiamtely), the theory is specified and all
physical quantities are in principle calculable.
Here, however, the number of input parameters in the
theory of action $S$ is infinite.
In string theory, this is precisely what is needed in
order to be able to formulate string theory in infinite
dimensional families of backgrounds, which will
correspond to solutions to string equations of motion.

\Proclaim{\rm 4) {\it Irrelevance of Higher Mass Couplings}.}
\finishproclaim

The fields $\Gmu$, $\Bmu$, $\Phi$ are associated with
the graviton, anti-symmetric tensor and
dilaton states in string theory, respectively. 
Those are precisely all the {\it massless states} of the
bosonic string.
A natural question arises as to whether fields
associated with {\it massive string states} should
also be included in $S$.
In other words, we wish to find out whether $S$ 
is the most general action possible for the dynamics of
bosonic string theory.

Using the correspondence (established in \S{II})
between variations in the
background fields (such as $\Gmu$, $\Bmu$ and $\Phi$) and
vertex operators for the associated physical states,
we may infer the structure of possible contributions
to $S$ from massive states, directly from the
corresponding vertex operators.
A vertex operator of a string state of
square mass $M^2=-2+2p$ is of the form
$$
V(\eps,k)=\int\nolimits_{\Sigma}d\mu_g
\Eps_{\mu_1\ldots\mu_{2p}}(k)
P^{\mu_1\ldots\mu_{2p}}(\partial x,
\partial^2x,\ldots;k)e^{ik\cdot x}
\eqno{(6.B.9)}
$$
where $\Eps$ is a polarization tensor characterizing
the string state and $P$ is a
polynomial of degree $2p$ in derivatives on $x$, so
that $P$ has $\Sigma$-dimension $2p$.
Denoting the field associated with this state by $E_{\mu_1
\ldots\mu_{2p}}(x)$, the contribution to the action 
$S$ will take the form
$$
S_E[x,g]=\int\nolimits_{\Sigma}d\mu_g\,E_{\mu_1\ldots
\mu_{2p}}(x)\Ptil^{\mu_1\ldots\mu_{2p}}(\partial x,
\partial^2 x,\ldots\,)\,\,.
\eqno{(6.B.10)}
$$
Here $\Ptil$ is again a polynomial of worldsheet
dimension $2p$.
For $p=0$, we recover the tachyon action, which we
omit.
For $p=1$, we recover the action $S$, associated with
massless string states.

For $p\Ge 2$, $S_E$ produces a non-renormalizable, or
{\it irrelevant}, contribution to the worldsheet action.
Addition of $S_E$ to $S$ yields a quantum field theory
that can also be defined solely from the action $S$,
but with modified background field values $G$, $B$,
$\Phi$.
In other words, the family of renormalized QFT's
defined by $S+S_E$ coincides with the family of
renormalized QFT's defined solely by $S$.

This means that bosonic string theory is completely specified
by the background fields associated with massless
states only (we continue to ignore the tachyon).
Once those have been specified, the dynamics of all massive
states is a consequence of string dynamics and is
completely determined.
Of course, as argued in the beginning of this lecture,
the condition of
consistent string propagation will further limit the
possible backgrounds to those yielding a conformal
field theory of central charge $c=26$.

\bigskip\noindent
\Item{\bf C)} {\bf General Structure of Weyl Dependence}

We now determine the general structure of the Weyl
dependence of Green functions in the quantum field
theory specified by the action $S$ in terms of fixed
background fields $\Gmu$, $\Bmu$, $\Phi$.
We consider unnormalized correlation functions of
operators $\phi_1\ldots\phi_n$, inserted at points
$\xi_1,\ldots,\xi_n\in\Sigma$.

In \S{E}, we shall provide a concrete and useful
reformulation of correlation functions in terms of the
effective action, via the covariant background field
method.
{}From this reformulation, it will be immediate to
establish in perturbation theory the properties that we
shall use formally in this section and in the next one
(\S{D}).

Under a $\Weyl(\Sigma)$ transformation $g\to
e^{2\delta\sigma}g$, with support away from
$\xi_1\ldots\xi_n$, we have by definition of the stress
tensor $T_{mn}$:
$$
\delta_\sigma\left<\phi_1\ldots\phi_n\right>_g=
{1\over 2\pi}\int\nolimits_{\Sigma}
d\mu_g\delta\sigma
\left<T_m^{\,\,\,m}\phi_1\ldots\phi_n\right>_g \,\,.
\eqno{(6.C.1)}
$$
We assume throughout that correlation functions are
covariant under $\Diff(\Sigma)$, so that
$$
\nabla^m T_{mn}=0\,\,.
\eqno{(6.C.2)}
$$
$T_m^{\,\,\,m}$ receives contributions from the explicit
Weyl-non-invariance of $S_\Phi$, from Weyl anomalies,
and from the interplay between non-invariance of
$S_\Phi$ and Weyl anomalies.
All such contributions to $T_m^{\,\,\,m}$ are {\it local
functions of $x$}, (i.e. dependent on $x$ and on
finite order derivatives on $x$) and {\it polynomial
in the derivatives of $x$}
(see Witten's lectures).

Collecting the various criteria, we find that 
$T_m^{\,\,\,m}$ has to
be

\medskip
\itemitem{1)}
local in $x$ and polynomial in derivatives of $x$;

\itemitem{2)}
$\Diff(\Sigma)$ scalar;

\itemitem{3)}
of $\Sigma$-dimension $2$;

\itemitem{4)}
of $M$-dimension $0$;

\itemitem{5)}
$\Diff(M)$-invariant;

\itemitem{6)}
$U(1)_B$ invariant;

\itemitem{7)}
invariant under shifts in $\Phi$ by a constant.

\medskip\noindent
Only the last point may need justification: a shift in
$\Phi$ by a constant yields a shift in the action $S$
by a term proportional to the Euler number, which is
Weyl independent.

Making use of a combination of
these criteria, we find that $T_m^{\,\,\,m}$
must involve precisely $2$ derivatives on $\Sigma$,
and must be determined in terms of $4$ unknown local functions
$\beta^G$, $\beta^B$, $\beta^\Phi$ and $\beta^V$ of
$x$:
$$
\eqalign{
T_m^{\,\,\,m} &=\partial_m x^\mu \partial_n
x^\nu\Bigl(\beta_{\mu\nu}^G(x)g^{mn}+\beta_{\mu\nu}^B(x)
\Eps^{mn}\Bigr)\cr
&\qquad +\beta^\Phi(x)R_g+\beta_\mu^V(x)
g^{mn}D_m^*\partial_n x^n\,\,.\cr}
\eqno{(6.C.3)}
$$
Here, $D_m^*$ is the covariant derivative on
$T^*\Sigma\otimes TM$ pulled back to $\Sigma$ by the
map $x$, and is defined by{\baselineskip=9pt\footnote{$^*$}%
{\eightpoint We shall denote the pullback under the map
$x$ of covariant derivatives $D$ on $TM$ by $D^*$.

}}
$$
D_m^*\partial_n x^\mu=\nabla_m \partial_n x^\mu+
\Gamma_{\nu \rho}^\mu\partial_m x^\nu\partial_n x^\rho\,\,,
\eqno{(6.C.4)}
$$
where $\nabla_m$ is the standard covariant derivative
on $T^*\Sigma$, as defined in \S{II}, and
$\Gamma$ is the affine connection of the metric
$\Gmu$.
The functions $\beta^G$, $\beta^B$, $\beta^\Phi$,
$\beta^V$ are called {\it beta functions}, even though in this
context they govern a change of
Weyl scale rather than a change of renormalization
scale.

The structure of $T_m^{\,\,\,m}$ reveals a 
puzzle:
the action $S$ involves only the three fields $G$, $B$,
$\Phi$, but the Weyl variation of the correlation
functions produces four beta functions.
Indeed, there is an additional beta function
$\beta^V$.
If only {\it constant} Weyl transformations
$\delta_\sigma$ are performed in (6.C.1), the term involving
$\beta^V$ may be integrated by parts and effectively absorbed
into $\beta^G$.
For general Weyl transformations, however, the term
involving $\beta^V$ is truly independent.

The resolution of this puzzle lies in
the fact that upon performing a Weyl transformation
on $g$, we are free to simultaneously transform $x$. 
On general grounds, the form of this transformation is
given by $\delta_\sigma x^\mu=\ell^2
V^\mu(x)\delta\sigma$, where $V^\mu(x)$ is a vector
field on $M$.
We shall make it clear later on how to implement this
additional effect of a Weyl transformation in actual
calculations.
Using the freedom to make this additional
transformation, we see that  $V^\mu(x)$ may be chosen
so as to compensate for the $\beta^V$-function.
The remaining $\beta$-functions $\beta_{\mu,\nu}^G$,
$\beta_{\mu\nu}^B$, $\beta^\Phi$ then have precisely as many
components as the fields $G_{\mu\nu}$, $B_{\mu\nu}$ and
$\Phi$ whose Weyl variation they describe.

Assuming now that the $\beta_\mu^V$ term has been
cancelled by a suitable transformation
$\delta_\sigma x^\mu$, we have the most general
expression for $T_m^{\,\,\,m}$ as follows
(we continue to use the same notation for $\beta$'s):
$$
T_m^{\,\,\,m}=\partial_m x^\mu\partial_n x^\nu
\left(\beta_{\mu\nu}^G(x)g^{mn}+\beta_{\mu\nu}^B(x)
\varepsilon^{mn}\right)+\beta^\Phi(x)R_g\,\,.
\eqno{(6.C.5)}
$$

(This phenomenon of augmenting Weyl transformations on
$g$ by further transformations on the fields was
already encountered in Problem Set \#1.
There, we studied a special case of action $S$ where
$D=1$, $G=1$, $B=0$, $\Phi=Qx$ and
$$
S[x,g]={1\over 4\pi}\int\nolimits_{\Sigma}
d\mu_g\left(\half\,x\Delta_g x+QR_g x\right)
\eqno{(6.C.6)}
$$
and discovered that upon a Weyl transformation,
$\delta_\sigma x=Q\delta\sigma$.)

\Proclaim{\it Classical contribution to $T_m{}^m$}.
\rm

The contribution to $T_m^{\,\,\,m}$ 
arising from the variation of the classical
action with respect to $\delta_\sigma x^\mu$ is given by
$$
T_m^{\,\,\,m}(\delta_\sigma x^\mu)=g^{mn}
\scrD_m^*\partial_n x^\mu V^\nu(x)\Gmu(x)\,\,.
\eqno{(6.C.7)}
$$
To define the covariant derivative $\scrD_m^*$, we need
the {\it torsion tensor}, $H_{\mu\nu\rho}$, which is the
field strength of the field
$\Bmu$, given by $H=dB$, or in component notation
$$
H_{\mu\nu\rho}\equiv\partial_\mu
B_{\nu\rho}+\partial_\nu B_{\rho\mu}+\partial_\rho
\Bmu\,\,.
\eqno{(6.C.8)}
$$
Then, we have
$$
\scrD_m^*\partial_n x^\mu\equiv D_m^*\partial_n
x^\mu-\half\,H_{\nu\rho}^\mu g_{m\rho}
\Eps^{qp}\partial_q x^\nu\partial_n x^\rho
\eqno{(6.C.9)}
$$
so that $\scrD_m^*$ is a covariant derivative with
respect to a connection with torsion
$\Gamma+{1\over2}\,H$ on $TM$.

It is now clear that a vector field
$V^\mu(x)$ may be chosen so as to
cancel $\beta_\mu^V$ in the expression for
$T_m^{\,\,\,m}$ in (6.C.3).
To lowest order in $\ell^2$, we have
$$
V^\mu(x)\Gmu(x)=-\beta_\nu^V(x)+O(\ell^2)\,\,.
\eqno{(6.C.10)}
$$
Notice that this cancellation of the $\beta^V$ term
produces modifications in the functions $\beta^G$ and
$\beta^B$ and possibly also $\beta^\Phi$.

\bigskip\noindent
\Item{\bf D)} {\bf General Structure of Weyl Anomaly
in Low Energy Expansion}

An exact computation of $\beta^G$, $\beta^B$ and
$\beta^\Phi$ is possible only if $G$, $B$ and $\Phi$
exhibit a high degree of symmetry.
(e.g. flat $\Gmu$, $\Bmu=\Phi=0$.)
For general $G$, $B$ and $\Phi$, only approximate
expressions can be obtained.
Fortunately, the constant $\ell$ is exceedingly small
compared to $L_{\expt}$, so an expansion in power of
$\ell$ should be very reliable.
On general grounds, the expansion of the functions
$\beta^G$, $\beta^B$ and $\beta^\Phi$ has the
following structure
$$
\beta(x)=\sum\limits_{p=0}^\infty \ell^{2p}\beta^{(p)}
(x)\eqno{(6.D.1)}
$$
where the coefficients $\beta^{(p)}(x)$ are

\medskip
\itemitem{1)}
local in $x$ and function of $x$, but not of derivatives
of $x$;

\itemitem{2)}
independent of $g$;

\itemitem{3)}
of $\Sigma$ dimension $0$;

\itemitem{4)}
of $M$ dimension $2+2p$ for $\beta^G$, $\beta^B$;
$2p$ for $\beta^\Phi$;

\itemitem{5)}
$\Diff(M)$ tensors;

\itemitem{6)}
$U(1)_B$ invariant, dependent upon $B$ only through
$H=dB$;

\itemitem{7)}
invariant under shifts in $\Phi$ by a constant.

\medskip\noindent
Expanding up to and including terms with two
derivatives (total) on the fields $G$, $B$, $\Phi$ 
requires expansion up to order $p=0$ for $\beta^G$,
$\beta^B$ and orders $p=0,2$ for $\beta^\Phi$.
Up to this order, we find the most general possible
expression satisfying the above criteria:
$$
\eqalign{
\beta_{\mu\nu}^G &=a_1R_{\mu\nu}^G+a_2\Gmu+a_3\Gmu R^G+
a_4 H_{\mu\rho\sigma}H_\nu^{\rho\sigma}
+a_5 \Gmu H_{\rho\sigma\tau}H^{\rho\sigma\tau}\cr
&\qquad + a_6D_\mu D_\nu\Phi+a_7 \Gmu D^2\Phi +
  a_8 G_{\mu\nu}D^\rho\Phi\,D_\rho\Phi\cr
\beta_{\mu\nu}^B &=b_1 D^\kap H_{\kap\mu\nu}+b_2
  D^\kap \Phi H_{\kap\mu\nu}\cr
\beta^\Phi &=c_0+\ell^2\left\{c_1 R^G+c_2 D^2\Phi+
  c_3 D^\kap\Phi D_\kap \Phi+c_4 H_{\rho\sigma\tau}
  H^{\rho\sigma\tau}\right\}\cr}
\eqno{(6.D.2)}
$$
Here, $R^G$ and $R_{\mu\nu}^G$ are the Ricci scalar
and tensor of the metric $G$, indices are raised and
lowered with the help of the metric $G$, and $D_\mu$ is the
covariant derivative with respect to the affine
connection $\Gamma$. 
On a vector field $V^\mu$, $D_\mu$ acts by
$$
D_\mu V^\kap\equiv \partial_\mu V^\kap+\Gamma_{\mu\nu}^\kap
V^\nu\,\,.
\eqno{(6.D.3)}
$$
The coefficients $a_1,\ldots,a_8$, $b_1$, $b_2$,
$c_0,c_1,\ldots,c_4$ are now the only remaining
unknowns, and we shall determine them shortly.

It is very useful to investigate the $\ell$-dependence
of the graphs contributing to each coefficient.
This counting is given by the usual loop expansion,
except that each factor of $\Phi$ contributes a factor
of $\ell^2$ additionally.
Within this counting, $a_8$ would arise from a graph
with $-1$ loops, and thus must be absent: $a_8=0$.
We have furthermore
$$
\matrix{
0-{\rm loops}\colon\hfill &a_6,a_7,b_2,c_3\hfill \cr
\noalign{\bigskip}
1-{\rm loop}\colon\hfill &a_1,\ldots,a_5,b_1,c_0,c_2\hfill \cr
\noalign{\bigskip}
2-{\rm loops}\colon\hfill &c_1,c_4\,\,.\hfill \cr}
$$

Two of these coefficients may be obtained from the
Weyl anomaly calculation in flat $\Gmu$, with
$\Bmu=\Phi=0$.
We have
$$
\matrix{
{\scriptscriptstyle\bullet}\hfill &a_2=0:\hfill
&\hbox{no field renormalization in free $x$-field
theory;}\hfill\cr
\noalign{\bigskip}
{\scriptscriptstyle\bullet}\hfill &c_0=+D/6:\hfill
  &\hbox{the central charge of $D$ scalars
$x^\mu$.}\hfill\cr}
\eqno{(6.D.4)}
$$

The tree-level coefficients may be obtained from
classical calculations, and are thus also easily
gotten.
One finds (see Problem Set \#7)
$$
\matrix{
{\scriptscriptstyle\bullet}\hfill &a_6=1\,,\hfill
  &a_7=0\hfill\cr
\noalign{\bigskip}
{\scriptscriptstyle\bullet}\hfill &b_2=1/2\,\,.\hfill &\cr
\noalign{\bigskip}
{\scriptscriptstyle\bullet}\hfill &c_3=2\,\,.\hfill &\cr}
\eqno{(6.D.5)}
$$

It remains to obtain the $1$-loop coefficients $a_1$,
$a_3$, $a_4$, $a_5$, $b_1$, $c_2$ and the $2$-loop
coefficients $c_1$ and $c_4$.
To do so, we use the background field method.

\vfill\eject

\noindent
\Item{\bf E)} {\bf Background Field Quantization
Method}

The background field quantization method is a powerful
tool for obtaining the effective action, which is the
generating functional for $1$-particle irreducible
(1PI) Feynman diagrams, in terms of a functional integral.
When local symmetries are present, the method may be
organized in a manifestly covariant way, thereby
greatly enhancing the effectiveness of perturbative
expansion methods.
We start by explaining the background field method for
a general scalar field $\phi$, and then adapt the method
to the case of sigma models.

Let $S$ be the classical action for the classical
field $\phi$, given in terms of classical coupling
constants.
We shall designate by coupling constants all
the parameters entering the
definition of $S$.
Their dependence will not be exhibited.
By replacing classical field and coupling constants by
{\it bare} quantities, we obtain the {\it bare action}
$S_0[\phi_0]$ in terms of the {\it bare field}
$\phi_0$ and {\it bare coupling constants}.
We shall assume that the 
field $\phi_0$ is multiplicatively renormalized by
a field renormalization factor $Z_\phi$, which 
depends only upon the coupling constants and the
renormalization scale $\Lam$, but not on
the field $\phi$.
Let $J$ be a source function. 
Then the generating
functional $W[J]$, which is the generating functional
for {\it renormalized connected Feynman diagrams}, is
defined by
$$
e^{W[J]}\equiv\int D\phi_0\,e^{-S_0[\phi_0]+Z_\phi\int
J\phi_0}\,\,.
\eqno{(6.E.1)}
$$

Alternatively, the above functional integral may be
expressed directly in terms of the {\it
renormalized field} $\phi\equiv Z_\phi\phi_0$.
It is standard to introduce the {\it renormalized
action} $S_R[\phi]$ by
$$
S_0[\phi\,Z_\phi^{-1}]\equiv S_R[\phi]
\eqno{(6.E.2)}
$$
($S_R$ consists of the classical action, plus renormalization
counterterms.)
Then, we have
$$
e^{W[J]}=\int D\phi\,e^{-S_R[\phi]+\int J\phi}\,\,.
\eqno{(6.E.3)}
$$

The effective action for connected, $1PI$ renormalized
Feynman diagrams is denoted by $\Gamma[\varphi]$ and
is defined as the Legendre transform of $W[J]$:
$$
\Gamma[\varphi]\equiv\int
J\varphi-W[J]\qquad\qquad\qquad
J(x)={\delta\Gamma[\varphi]\over \delta\varphi(x)}\,\,.
\eqno{(6.E.4)}
$$
Expressing $W[J]$ in terms of $\Gamma[\varphi]$,
eliminating $J$, and shifting $\phi$ by $\varphi$ in
(6.E.3), we
obtain the final formula for $\Gamma[\varphi]$:
$$
e^{-\Gamma[\varphi]}=\int D\phi\,\exp\left\{-S_R[\varphi+
\phi]+\int\phi{\delta\Gamma[\varphi]\over\delta\varphi}\right\}
\eqno{(6.E.5)}
$$

At first sight, it would appear that we have gained
little: the right hand side depends upon
$\Gamma[\varphi]$ itself, and we have only obtained an
implicit equation for $\Gamma[\varphi]$.
However, a loop expansion (in powers of $\hslash$ in QFT, which
we shall perform in terms of $\ell^2$ here, for later
use) reveals that the above expression yields a
powerful recursive equation for $\Gamma[\varphi]$.
To see this, redefine $\phi\to\ell\phi$, and expand $S_R$
and $\Gamma[\varphi]$ around the classical action.
We make use of the fact that $\Gamma[\varphi]$ admits
a Laurent expansion in powers of $\ell^2$ (not just
$\ell$). 
We have
$$
\eqalignno{
S_R[\varphi+\ell\phi] &={1\over\ell^2}\,S[\varphi]
  +{1\over\ell}\int\phi\,{\delta S\over \delta\varphi}
\,[\varphi]+\Sbar[\varphi;\phi](\ell) &(6.E.6)\cr
\noalign{\hbox{and}}
\Gamma[\varphi]\hfill &={1\over\ell^2}\,S[\varphi]+
  \Gammabar[\varphi](\ell)&(6.E.7)\cr}
$$
Here, $\Sbar$ and $\Gammabar$ admit a Taylor series
expansion in $\ell$ and $\ell^2$, respectively.
Substituting these general forms into (6.E.5),
we find:
$$
e^{-\Gammabar[\varphi](\ell)}=\int D\phi\,
\exp\left\{-\Sbar[\varphi;\phi](\ell)+\ell\int\phi\,
{\delta\Gammabar[\varphi](\ell)\over\delta\varphi}\right\}
\eqno{(6.E.8)}
$$
from which it is transparent that $\Gammabar$ is
determined by a nice recursive equation.
For example, to order $\ell^0$, we have a standard
formula for the $1$-loop effective action
$$
e^{-\Gammabar[\varphi](0)}=\int D\phi\,
e^{-\Sbar[\varphi;\phi](0)}\,\,.
\eqno{(6.E.9)}
$$

\bigskip\noindent
\Item{\bf F)} {\bf Covariant Expansion Methods}

Applying the background field method to the non-linear
sigma model with action $S$, given in (6.B.1), 
the quantum field $\phi$ becomes the
coordinate $x^\mu$ of a point on the Riemannian
manifold $M$, while the classical field $\varphi$
becomes a reference point $x_0{}^\mu$ on $M$.
Addition of coordinates --- as would apparently
be needed in the
background field method developed above --- is not
covariant under $\Diff(M)$.
It is beneficial to make use of the Riemann normal
coordinate expansion instead, since this method yields
$\Diff(M)$ covariant results.

For $x_0^\mu$ and $x^\mu$ sufficiently close to one
another, there is a unique shortest geodesic curve
$C$, parametrized by $x_\tau^\mu$, that interpolates
between $x_0^\mu$ for $\tau=0$ and $x^\mu$ for
$\tau=\ell$.
The function $x_\tau^\mu$ obeys the geodesic equation
$$
{D\over
D\tau}\,\xdot_\tau^\mu=\xddot_\tau^\mu+\Gamma_{\nu\rho}^\mu
(x_\tau)\xdot_\tau^\nu\,\xdot_\tau^\rho=0
\eqno{(6.F.1)}
$$
The tangent vector to $C$ at $\tau=0$ is a vector
field $\xi^\mu\equiv\xdot_0^\mu$ and $x^\mu$ may be
parametrized (in a neighborhood of $x_0^\mu$) by
$x_0^\mu$ and $\xi^\mu$: \
$x^\mu=e^{\ell\xi}x_0^\mu$.
$$
\vbox{\epsfxsize=2.5in\epsfbox{fig1.eps}}
$$

$\Diff(M)$-tensors admit covariant expansions in
powers of $\ell$.
For example, the expansion of rank $0$ tensors
(scalars, such as $\Phi$) is given by
$$
\Phi(x)=\Phi(x_0)+D_\kappa\Phi(x_0)\ell\xi^\kappa
+\half\,D_\kappa\,D_\lam\Phi(x_0)\ell^2
\xi^\kappa\xi^\lam+O(\ell^3).\eqno{(6.F.2)}
$$
while the expansion of rank $2$ tensors (such
as $G$, $B$) is given by
$$
\eqalign{
T_{\mu\nu}(x) &=T_{\mu\nu}(x_0)+D_\kap T_{\mu\nu}(x_0)
\ell\xi^\kap +\half\{D_\kap D_\lam T_{\mu\nu}(x_0)\cr
&\qquad\qquad -\third\,R_{\kap\mu\lam}^\rho
T_{\rho\nu}(x_0)-\third\,R_{\kap\nu\lam}^\rho
T_{\mu\rho}(x_0)\}\ell^2\xi^\kap\xi^\lam+
  O(\ell^3)\,\,.\cr}
\eqno{(6.F.3)}
$$
Here, $D_\kap$ is the covariant derivative with
respect to the affine connection
$\Gamma_{\nu\rho}^\mu$, and
$R_{\kap\mu\lam}^\rho$ is the associated Riemann
curvature.

As a result, the generalized non-linear sigma model
action $S[x,g]$ of \S{B} may be similarly expanded
around $x_0$, in a power series in $\ell$, by setting
$x=e^{\ell\xi}x_0$.
$$
\eqalign{
S[x,g]=&S[x_0,g]+\ell\int\nolimits_{\Sigma}
d\mu_g\,\xi^\mu\,S_\mu^{(1)}[x_0,g]+\Sbar[x_0,\xi;g]\cr
&S_\mu^{(1)}[x_0,g]={1\over\sqrt{\det\,g}}\,\,
{\delta S[e^{\xi'} x_0,g]\over\delta\xi'{^{\mu}}}
\,\,{\vrule height15pt depth 8pt}_{\,\xi'=0}
\cr}
\eqno{(6.F.4)}
$$
Here, $\Sbar$ admits a Taylor series expansion in
$\ell$.
Notice that $S$ and $S_\mu^{(1)}$ are each the sum of a
term of order $\ell^{-2}$ and a term of order
$\ell^0$ arising from the dilaton action $S_\Phi$.
The effective action $\Gamma[x_0,g]$ is now obtained
just as in (6.E.8), except with the field $\varphi$
replaced by $x_0$ and $\phi$ replaced by $\xi$.
$$
e^{-\Gamma[x_0,g]}=\int
D\xi^\mu
\exp\left\{-S [e^{\ell\xi}x_0,g]+\ell\int\nolimits_{\Sigma}
d\mu_g\xi^\mu\Gamma_\mu^{(1)}[x,g]\right\}
\eqno{(6.F.5)}
$$
It will be understood throughout that $S$ stands for
the renormalized action, denoted by $S_R$ previously.
We have also used the definition
$$
\Gamma_\mu^{(1)}[x_0,g]\equiv
{1\over\sqrt{\det\,g}}\,\,
{\delta\Gamma[e^{\xi'}x_0,g]\over\delta\xi'{^{\mu}}}
\,\,{\vrule height15pt depth 8pt}_{\,\xi'=0}\,\,.
\eqno{(6.F.5')}
$$
The functional measure $D\xi^\mu$ is defined through
the $L^2$ norm on $TM$, which is given explicitly by
$$
\Vert\xi^\mu\Vert^2\equiv\int\nolimits_{\Sigma}
d\mu_g\,G_{\mu\nu}(x_0)\xi^\mu\xi^\nu\,\,.
\eqno{(6.F.6)}
$$
Using again the property that $\Gamma[x_0,g]$ admits a
Laurent expansion in terms of $\ell^2$ rather than
$\ell$, we have
$$
\Gamma[x_0,g]=S[x_0,g]+\Gammabar[x_0,g]
\eqno{(6.F.7)}
$$
where $\Gammabar$ admits a Taylor series expansion
in $\ell^2$, and is given by
$$
e^{-\Gammabar[x_0,g]}=\int D\xi^\mu\,
e^{-\Sbar[x,g]+\ell\int\nolimits_{\Sigma}
  d\mu_g\xi^\mu\,\Gammabar_\mu^{(1)}[x_0,g]}\,\,.
\eqno{(6.F.8)}
$$
Here, $\Sbar$ is defined as in $(6.F.4)$, and
$\Gammabar_\mu^{(1)}$ is defined as in $(6.F.5')$ by
replacing $\Gamma$ by $\Gammabar$.

It remains to list the explicit expansion terms that
we shall need to evaluate $\beta^G$, $\beta^B$ and
$\beta^\Phi$, up to order $\ell^0$ in $\beta^G$ and
$\beta^B$ and $\ell^0$, $\ell^2$ in $\beta^\Phi$.
They are given by
$$
\Sbar 
=\Sbar_0+\ell\,\Sbar_1+\ell^2\Sbar_2+O(\ell^3)\eqno{(6.F.9)}
$$
where
$$
\eqalignno{
\Sbar_0 &={1\over 8\pi}\int\nolimits_{\Sigma}
  d\mu_g\Bigl\{\scrD_m^*\xi^\mu\scrD_n^*\xi^\nu\Gmu(x_0)
  g^{mn} +\scrR_{\mu\nu\rho\sigma}(x_0)
  \partial_m x_0^\mu\partial_n x_0^\rho
  \xi^\nu\xi^\sigma(g^{mn}-\Eps^{mn})\Bigr\}\cr
\Sbar_1 &={1\over 8\pi}\int\nolimits_{\Sigma} d\mu_g
\Bigl\{\third\,H_{\mu\nu\rho}\xi^\mu\scrD_m^*\xi^\nu\scrD_n^*
  \xi^\rho\Eps^{mn}\Bigr\}&(6.F.10)\cr
\Sbar_2 &={1\over 8\pi}\int\nolimits_{\Sigma}
d\mu_g  \Bigl\{\third\,R_{\mu\nu\rho\sigma}\xi^\nu\xi^\rho
  \scrD_m^*\xi^\mu\scrD_n^*\xi^\sigma g^{mn}
-\half\,\scrR_{\mu\nu\rho\sigma}
\xi^\nu\xi^\rho\scrD_m^*\xi^\mu\scrD_n^*\xi^\sigma\Eps^{mn}\cr
&\qquad\qquad\qquad +2D_\mu D_\nu\Phi(x_0)
  R_g\xi^\mu\xi^\nu\Bigr\}\,\,.\cr}
$$
Here, the covariant derivative with torsion on
$TM$ and pulled back to $\Sigma$, is defined by
$$
\scrD_m^*\xi^\mu\equiv D_m^*\xi^\mu+\half\,
H^\sigma{}_{\nu\rho} g_{mp}\Eps^{pq}\partial_q
  x^\nu\xi^\rho\,\,.
\eqno{(6.F.11)}
$$
(This definition agrees with that of (6.C.9).)
Furthermore, we have the following covariant
derivatives with torsion on $TM$
$$
\scrD_\mu\xi^\nu\equiv D_\mu\xi^\nu+\half\,
H^\sigma{}_{\mu\rho}\xi^\rho
\eqno{(6.F.12)}
$$
whose structure relations yield the torsion $H$ and
the curvature tensor $\scrR$ of the connection with
torsion $\Gamma+{1\over 2}\,H$:
$$
\left[\scrD_\mu,\scrD_\nu\right]\xi^\rho=H^\sigma{}_{\mu\nu}
\scrD_\sigma\xi^\rho+\scrR_{\sigma\mu\nu}^\rho
\xi^\sigma\,\,.
\eqno{(6.F.13)}
$$
It is easy to express $\scrR$ in terms of the Riemann
tensor $R$ and the torsion tensor $H$:
$$
\eqalign{
\scrR_{\mu\nu\rho\sigma} &=R_{\mu\nu\rho\sigma}+
\half\,D_\rho H_{\sigma\mu\nu}-\half\,D_\sigma
  H_{\rho\mu\nu}\cr
&\qquad +\fourth\,H_{\rho\mu\alpha}H_{\sigma\nu}{}^\alpha
-\fourth\,H_{\sigma\mu\alpha}H_{\rho\nu}{}^\alpha\cr}
\eqno{(6.F.14)}
$$

\bigskip\noindent
\Item{\bf G)} {\bf Reformulation as an $\SO(1,D-1)$
gauge theory}

The difficulty encountered in the above formulation is
that the quadratic form representing the kinetic term
for the field $\xi^\mu$ in (6.F.10) depends in a
non-trivial way on the metric $G_{\mu\nu}$ and on the
background field $x_0^\mu$.
As a result, perturbation theory will involve determining the
$\xi$-propagator in arbitrary $x_0$ background, which
is an impossible task.
One may get around this complication by changing
variables and absorbing the $G_{\mu\nu}$ and $x_0^\mu$
dependence into the new field.

One introduces an orthonormal frame, which we denote by
$e_\mu{}^a$, with $a=0,1,\ldots,D-1$ representing the
tangent space direction of $TM$.
We have, by definition
$$
G_{\mu\nu}=e_\mu{}^a\,e_\nu{}^b\,\eta_{ab}\eqno{(6.G.1)}
$$
where $\eta_{ab}=\diag(-\,+\cdots +)$ is the
$\SO(1,D-1)$ invariant flat metric.
The inverse frame is denoted by $e_a{}^\mu$, so that
$e_a{}^\mu\,e_\mu{}^b=\delta_a{}^b$ and
$e_\mu{}^a\,e_a{}^\nu=\delta_\mu{}^\nu$.
We also introduce the \break
$\SO(1,D-1)$-valued connection
$\omega_\mu$, with components $\omega_\mu{}^a{}_b$, such
hat under the covariant derivative with respect to the
affine connection $\Gamma$, and the spin connection
$\omega_\mu$, the frame $e_\mu{}^a$ is covariantly
constant
$$
D_\mu e_\nu{}^a\equiv \partial_\mu
e_\nu{}^a-\Gamma_{\mu\nu}^\kappa
e_\kappa{}^a+\omega_\mu{}^a{}_b\,e_\nu{}^b=0
\eqno{(6.G.2)}
$$
The covariant derivative $\scrD$, including the torsion
connection ${1\over 2}\,H$, is similarly defined.
For example, on $e_\nu{}^a$, we have
$$
\scrD_\mu e_\nu{}^a=D_\mu
e_\nu{}^a+\half\,H_{\mu\nu}{}^\rho
e_\rho{}^a\eqno{(6.G.3)}
$$
We now rewrite all tensors with respect to the
orthonormal frame basis.
Thus, we have
$$
\eqalign{
H_{\mu\nu\rho} &=e_\mu{}^a e_\nu{}^b e_\rho{}^c
H_{abc}\cr
\scrR_{\mu\nu\rho\sigma} &=e_\mu{}^a e_\nu{}^b
e_\rho{}^c e_\sigma{}^d
  \scrR_{abcd}\cr
\xi^\mu &=\xi^a e_a{}^\mu\cr
\scrD_a &=e_a{}^\mu \scrD_\mu\cr}
\eqno{(6.G.4)}
$$
where all tensors are evaluated at the same point on
$M$, which in our background calculation will be
$x_0{}^\mu$.
We also introduce the pullback of the tangent vector 
$\partial_m x_0^\mu$, expressed in orthonormal frame
basis:
$$
e_m^*{}^a\equiv \partial_m x_0{}^\mu e_\mu{}^a(x_0)
\eqno{(6.G.5)}
$$

The expression for the quantum contributions to the
effective action $\Gammabar[x_0,g]$ of (6.F.8) is now
$$
e^{-\Gamma[x_0,g]}=\int D\xi^a
\exp\left\{-\Sbar[e^{\ell\xi}x_0,g]+\ell
\int\nolimits_{\Sigma}d\mu_g \xi^a\Gamma_a^{(1)}\right\}
\eqno{(6.G.6)}
$$
The functional measure $D\xi^a$ is again defined with
respect to the $L^2$ norm
$$
\Vert\xi^a\Vert^2=\int\nolimits_{\Sigma} d\mu_g
\xi^a\xi^b\eta_{ab}\eqno{(6.G.7)}
$$
which does not depend upon $x_0$ or $G_{\mu\nu}$
anymore.
The action $\Sbar$ may be expanded, just as in
(6.F.10), and we obtain
$$
\eqalignno{
\Sbar_0 &={1\over 8\pi}\int\nolimits_{\Sigma}d\mu_g
  \{\scrD_m^*\xi^a\scrD_n^*\xi^b\eta_{ab}g^{mn}
+\scrR_{abcd}\,e_m^*{}^ae_n^*{}^c\xi^b\xi^d
(g^{mn}-\eps^{mn})\} &(6.G.8a)\cr
\Sbar_1 &={1\over 8\pi}\int\nolimits_{\Sigma}d\mu_g
\{\third\,H_{abc}\xi^a\scrD_m^*\xi^b\scrD_n^*\xi^c
  \Eps^{mn}\}&(6.G.8b)\cr
\Sbar_2 &={1\over 8\pi}\int\nolimits_{\Sigma}d\mu_g
\{\third(\scrR_{abcd}-\fourth\,H_{caf}H_{db}{}^f)
\xi^b\xi^c\scrD_m^*\xi^a\scrD_n^* \xi^d g^{mn} &(6.G.8c)\cr
&\qquad -\half\,\scrR_{abcd}\xi^b\xi^c\scrD_m^*
  \xi^a\scrD_n^*\xi^d\eps^{mn}
+2\scrD_a\scrD_b\Phi\,R_g\,\xi^a\xi^b\}\,\,.\cr}
$$
The pullback covariant derivative is now very simple:
$$
\scrD_m^*\xi^a=\nabla_m\xi^a+A_m{}^{ab}
\,\xi^b\eqno{(6.G.9)}
$$
The effective $\SO(1,D-1)$ gauge field $A_m$ is defined by the
pullback of the spin connection $\omega_\mu$:
$$
A_{mb}^a(\xi)\equiv
\partial_m x_0^\mu(\xi)
\omega_\mu{}^a{}_b \eqno{(6.G.10)}
$$
In particular, the kinetic term for the $\xi^a$-field
is now also independent of the metric $G_{\mu\nu}$ and
the background field $x_0{}^\mu$.
As a result, standard perturbation methods in $\ell$
may be used to evaluate the effective action
$\Gamma[x_0,g]$, with a standard propagation for
$\xi^a$.

\bigskip\noindent
\Item{\bf H)} {\bf Weyl Variation of the Effective
Action}

The definition of the trace of the stress tensor for
Green functions introduced in (6.C.1)
 may be translated into an equation for
the Weyl variation of the effective action,
$\Gamma[x_0,g]$.
One finds
$$
\delta_\sigma\Gamma[x_0,g]=-{1\over 2\pi}
\int\nolimits_{\Sigma}d\mu_g\,\delta\sigma\,T_m^{\,\,\,m}\,\,.
\eqno{(6.H.1)}
$$
where $T_m{}^m$ denotes the expectation value of the
operator $T_m{}^m$ in the bakcground field $x_0$, and
metric $g$.
The derivation of (6.H.1) from (6.C.1)
is given in Appendix A.
It is understood that $x_0$ is also allowed to
transform under $\Weyl(\Sigma)$, as explained in
\S{C}.
The arguments developed in \S{C} for what the most
general structure of $T_m^{\,\,\,m}$ is, carry over
here, and we find, for a suitable $\Weyl(\Sigma)$
action on $x_0$:
$$
T_m^{\,\,\,m}=\partial_m x_0^\mu\partial _n
x_0^\nu(\beta_{\mu\nu}^G
g^{mn}+\beta_{\mu\nu}^B\Eps^{mn})+
\beta^\Phi R_g \eqno{(6.H.2)}
$$
where $\beta^G$, $\beta^B$ and $\beta^\Phi$ are
evaluated at $x_0$.
This definition of $T_m^{\,\,\,m}$ and $\beta^G$,
$\beta^B$, and $\beta^\Phi$ is now completely precise,
and these quantities may be directly evaluated from
$\Gamma[x_0,g]$.
Notice that the variation $\delta_\sigma\Gamma$
obtained by substituting (6.H.2) into (6.H.1) must
satisfy an integrability condition, known as the
Wess-Zumino consistency condition.
This condition provides a relation between $\beta^G$,
$\beta^B$ and $\beta^\Phi$, which we shall discuss
in \S{I}.

The effective action $\Gamma[x_0,g]$ is the sum of the
classical action $S[x_0,g]$ and the quantum
corrections $\Gammabar[x_0,g]$.
The contribution of $S$ to $\beta^G$, $\beta^B$ and
$\beta^\Phi$ was already evaluated previously
(Problem Set \#7).
It remains to evaluate the Weyl transformation of
$\Gammabar$ to the order we are computing.
We begin by evaluating the $O(\ell^0)$ contribution,
which arises solely from $\Sbar_0$.
We shall treat the contributions to $T_m^{\,\,\,m}$
perturbatively in the number of $M$-derivatives
applied to the fields $G$, $B$ and $\Phi$.
This expansion is reliable since we know from general
considerations that we should only retain contributions
to $T_m{}^m$ with
two derivatives.
For example, the $\scrR$-term in $\Sbar_0$ can contribute at
most to first order.

The form of the first term in $\Sbar_0$ is that of a
gauge field --- the spin connection $\omega_\mu$ --- minimally
coupled to $\xi^a$.
By power counting its one-loop graphs with one insertion of
$\omega_\mu\,\omega_\nu$ and with two insertions of
$\omega_\mu\,\partial_\nu$ are the only 
ones that could potentially contribute to the Weyl anomaly.
$$
\vbox{\epsfxsize=3.5in\epsfbox{fig1a.eps}}
$$
\line{\null\kern3true cm graph $\omega_\mu\omega_\nu$
  \kern3.75true cm 
graph $\omega_\mu\partial_\nu$\hfil}

\noindent
All other graphs are absolutely convergent and obey
anomaly-free Ward identities.
By gauge invariance, however, the degree of divergence
is lowered by $2$ and the sum of the two 
graphs does not contribute
to the Weyl anomaly either (see also Gawedzki's lectures).
Thus, the only contribution from the first term in
$\Sbar_0$ is to the central charge: $+D/6$
contribution to $\beta^\Phi$, which we have already
discussed previously by fixing $c_0=+D/6$.

It remains to evaluate the contributions from the
first order expansion of the $\scrR$-term in
$\Sbar_0$ to $\beta^G$ and $\beta^B$. 
We directly use the Ansatz for these functions in
terms of $G$, $B$, $\Phi$:
we denote their contribution to this order by
$\betatil^G$, $\betatil^B$, $\betatil^\Phi$.
Clearly $\betatil^\Phi=0$, and the Weyl
transformation of this part is given by
$$
\eqalign{
&-{1\over
2\pi}\int\nolimits_{\Sigma}d\mu_g\,\delta\sigma\,\partial_m
x_0^\mu\partial_n
x_0^\nu\left(\betatil_{\mu\nu}^G(x_0)
g^{mn}+\betatil_{\mu\nu}^B(x_0)\Eps^{mn}\right)\cr
&=\delta_\sigma\Bigl<-{1\over
8\pi}\int\nolimits_{\Sigma}d\mu_g\,\scrR_{\mu a\rho b}
(x_0)\partial_m x_0^\mu\partial_n x_0^\rho \xi^a
\xi^ b(g^{mn}-\Eps^{mn})\Bigr>\,\,,\cr}
\eqno{(6.H.3)}
$$
where $\left<\qquad\right>$ stands for the normalized
correlation function taken with respect to the action
given by only the first term in $\Sbar_0$.
But, since $x_0$ was arbitrary throughout, we may
identify the symmetric and antisymmetric tensor parts:
$$
\eqalign{
\betatil_{\mu\nu}^G(x_0)\delta\sigma &=
\delta_\sigma\left<\xi^a\xi^b\right>{1\over 8}
  (\scrR_{\mu a\nu b}+\scrR_{\nu a\mu b})\cr
\betatil_{\mu\nu}^B(x_0)\delta\sigma &=-
\delta_\sigma\left<\xi^a\xi^b\right>{1\over 8}
  (\scrR_{\mu a\nu b}-\scrR_{\nu a\mu b})\cr}
\eqno{(6.H.4)}
$$
The correlation function of $\xi$'s to this order is
independent of $x_0$, and we we find
$$
\delta_\sigma\left<\xi^a\xi^b\right>=
2\delta\sigma\eta^{ab}\,\,.
\eqno{(6.H.5)}
$$
Making use of the expression for $\scrR$ in terms of
$R^G$ and $H$ of (6.F.14), we find
$$
\eqalign{
\betatil_{\mu\nu}^G &=\half\,R_{\mu\nu}^G
-\eighth\,H_{\mu\alpha\beta}H_\nu{}^{\alpha\beta}\cr
\betatil_{\mu\nu}^B &=-\fourth\,D^\alpha
H_{\alpha\mu\nu}\,\,.\cr}
\eqno{(6.H.6)}
$$
We now combine this result with the one obtained to
tree level: this yields the full $\beta$-functions
$\beta^G$, $\beta^B$ to order $\ell^0$; we also
collect the results obtained thus far for $\beta^\Phi$:
$$
\eqalign{
\beta_{\mu\nu}^G
&=\half\,R_{\mu\nu}^G-\eighth\,H_{\mu\alpha\beta}
H_\nu^{\,\,\,\alpha\beta}+D_\mu D_\nu\Phi\cr
\beta_{\mu\nu}^B &=-\fourth D^\alpha H_{\alpha\mu\nu}+
 \half\,D^\alpha\Phi\,H_{\alpha\mu\nu}\cr
\beta^\Phi &=D/6+\ell^2\{2D_\alpha \Phi
  D^\alpha\Phi-2D_\alpha D^\alpha\Phi+c_1
  R^G+c_4 H^2\}\,\,.\cr}
\eqno{(6.H.7)}
$$
Here, we have included the coefficient of $D_\alpha
D^\alpha\Phi$, which is obtained from a simple
$1$-loop calculation, from the last term in $\Sbar_2$.
The calculation of $c_1$, $c_4$ requires a full
$2$-loop computation, with the following Feynman
diagram contributions:
$$
\vbox{\epsfxsize=3.5in\epsfbox{fig2.eps}}
$$
The result is
$$
c_1=-\half\,;\qquad c_4={\textstyle 1\over\textstyle 24}\,\,.
\eqno{(6.H.8)}
$$

\bigskip
\Item{\bf I)} {\bf Low Energy String Field Equations and
String Effective Action}

The string field equations are obtained by including
also the effects of the Faddeev-Popov ghosts $b$ $c$;
this effect is limited to an addition of $-13/3$ to
$\beta^\Phi$.
We then obtain the following equations for string
dynamics:
$$
\cases{
\beta_{\mu\nu}^G =\half\,R_{\mu\nu}^G-\eighth\,
  H_{\mu\alpha\beta}H_\nu^{\,\,\,\alpha\beta}+
  D_\mu D_\nu\Phi=0 &\cr
\noalign{\bigskip}
\beta_{\mu\nu}^B=-\fourth\,D^\alpha
H_{\alpha\mu\nu}+\half\,D^\alpha\Phi\,H_{\alpha\mu\nu}=0&\cr
\noalign{\bigskip}
\beta^\Phi=(D-26)/6+\ell^2\{2D_\alpha\Phi\,D^\alpha\Phi
  -2D_\alpha D^\alpha\Phi-\half\,R^G+{\textstyle 1
  \over\textstyle 24}\,H^2\}=0\,\,.&\cr}
\eqno{(6.I.1)}
$$
to leading order in the expansion in powers of
derivatives on $G$, $B$ and $\Phi$.

There is an interesting consistency condition, 
whose origin may be understood as follows.
On a (locally) flat worldsheet, conformal
invariance requires only that $\beta^G=\beta^B=0$. 
One expects that these conditions should suffice to make
the model conformal on non-flat worldsheets as well.
This in turn suggests that the equation $\beta^\Phi=0$
is not independent from the equations 
$\beta^G=\beta^B=0$. 
In fact, this interdependence
may indeed be established with the help of the
Bianchi identities:
$$
\eqalign{
&D^\mu R_{\mu\nu}^G=\half\,D_\nu\,R^G\cr
&D^\mu(H_{\mu\alpha\beta}H_\nu^{\,\,\,\alpha\beta})=D^\mu
H_{\mu\alpha\beta}H_\nu^{\,\,\,\alpha\beta}
+\sixth\,D_\nu H^2\,\,,\cr}
\eqno{(6.I.2)}
$$
where $H^2\equiv
H_{\alpha\beta\gamma}H^{\alpha\beta\gamma}$.
The following relation is then found to hold
to this order in powers of $\ell$:
$$
D^\mu \beta_{\mu\nu}^G=-{1\over 2\ell^2}\,D_\nu
\beta^\Phi+\half\,\beta_{\alpha\beta}^B
H_\nu^{\,\,\,\alpha\beta}+2\beta_{\mu\nu}^G
D^\mu\Phi\,\,,
\eqno{(6.I.3)}
$$
so that the equations
$\beta^G=\beta^B=0$ imply that $\beta^\Phi$ is
constant.
We see that it suffices indeed to render the
quantum-field theory of the $x$-field
conformal on a (locally) flat worldsheet.
The only quantity left undetermined is then the
central charge.
There are arguments in the physics literature that
such an equation for $\beta^\Phi$ holds to all orders
in $\ell$.

The only way to satisfy $\beta^\Phi=0$, given that $D$
must be an integer, and that $\ell$ is an expansion
parameter, is to set $D=26$.

The above equations can be put in a more standard form,
which reflects the fact that $\Bmu$ and $\Phi$ may be
viewed as matter fields, coupling to the metric $\Gmu$
by their energy-momentum tensor.
Alternatively, the above field equations are seen to
derive from an action
$$
I(G,B,\Phi)={1\over 2\kappa^2}\int
d^{26}x\sqrt{\det\,G}\,\, e^{-2\Phi}
\Bigl\{R^G+4D_\mu\Phi D^\mu\Phi-{\textstyle
1\over\textstyle 12}\,H^2\Bigr\}\,\,,
\eqno{(6.I.4)}
$$
provided the dimension is critical $D=26$.
The way the dilaton coupling enters exposes the fact
that $e^{2\Phi_0}$ is the string loop expansion
parameter, or string coupling constant, when the dilaton
vacuum expection value is $\Phi_0$.
The presence of the factor
$e^{-2\Phi_0}$ reveals that this effective action
arises from string tree level effects.
(One loop contributions would have no exponential
$\Phi$ dependence.)
It is possible to put the action in a more standard
form, by performing a Weyl transformation on $G$.
(For later use, we consider arbitrary dimension $D_i$
on results will then have application also to
superstring theory, where $D=10$.)
$$
\Gmu\to G'_{\mu\nu}=e^{-4\Phi/(D-2)}\Gmu\,\,.
\eqno{(6.I.5)}
$$
The action in this metric assumes the form
$$
I(G',B,\Phi)={1\over 2\kappa^2}\int d^{D}x\sqrt{\det\,G'}\,\,
\Bigl\{R^{G'}-{4\over D-2}\,D_\mu\Phi D^\mu\Phi
-{{\textstyle 1}\over{\textstyle 12}}\,e^{-8\Phi/(D-2)}
H^2\Bigr\}\,\,.
\eqno{(6.I.6)}
$$
The metrics $G$ and $G'$ are usually called the {\it
string metric} and the {\it Einstein metric},
respectively.

The action $I(G',B,\Phi)$ is in fact well-known in
field theory.
If we ignore the peculiar value of the $D=26$
dimension, and arbitrarily set $D=10$ instead, we recover the
action for the 
bosonic part of the $N=1$ supergravity multiplet in
$D=10$ which is part of the so-called Chapline-Manton action
(see \S{X}).
When we deal will with superstrings, we shall see that the
superstring calculation to this order yields the same
as the bosonic string, except for the dimension!

Finally, we point out that the field equations (6.I.1)
are consistent with the low energy limit of transition
amplitudes of strings, scattering in flat Minkowski
space-time.
The simplest example of such a low energy limit was
encountered in \S{II}, where we argued that the
transversality conditions on the graviton polarization
tensor required conditions that arise as the low
energy limit of the linearized limit of
$R_{\mu\nu}^G=0$.
In fact, string scattering amplitudes may be used
directly to compute the string field equations.
Unfortunately, it is difficult to compute the effective
action directly from string transition amplitudes,
since the latter are always evaluated on-shell only.

\Item{\bf J)} {\bf A first Look at Compactification}

One of the most remarkable results of string theory is
that the dimension of space-time (at long distances) is
dynamically determined: $D=26$ for bosonic strings and
$D=10$ for superstrings in flat (or nearly flat)
space-time.
The nagging aspect of this result, however, is that the
number of dimensions we observe is only $4$.
Well, at least $10$ is {\it larger} than $4$!

In attempts at unifying general relativity and
Maxwell's electromagnetism, Kaluza and Klein proposed
in the 1920's that space-time could really be of higher
dimension.
And that as long as space-time effectively appears to
be $4$-dimensional at all length scales where present
day experiments can probe it, the extra dimensions
would not be directly observable.
This scheme is usually termed Kaluza-Klein theory.

The most straightforward way to realize such
unobservable extra dimensions is to consider
space-times $M$ of the form $M=\dbR^4\times K$.
Here, $\dim\,K=\dim\,M-4$, and $K$ is a compact
manifold, whose (largest) size will be denoted by
$R_K$.
As long as $R_K\ll L_{{\rm expt}}$, 
the $K$ manifold should not be directly observable.
More generally, we may consider $M$ to be of the form
$M=M_4\times K$, where $M_4$ is approximately
Minkowskian (such as would be the case for our
universe, where $M_4$ is a manifold curved by the
presence of matter).

The low energy approximation to string theory yields
information on the possible solutions for the compact
manifold $K$, provided the size $R_K$ of $K$ is much
larger than the Planck scale:
$$
R_K\gg \ell=\sqrt{\alpha'/2\,\,}\eqno{(6.J.1)}
$$
In this case the approxiamtion in which higher powers of
$\ell$ are neglected is reliable.
The string field equations (6.I.1) restrict the fields
$G$, $B$ and $\Phi$ that are allowed on $K$ and thus
restrict $K$.

If it is assumed that space-time sypersymmetry remains
partially unbroken at distance scales of order
$R_K$, then $K$  must be a K\"ahler manifold, and
$\Phi$ must be constant.
A particular solution corresponds to $H=0$, so that $K$
must be K\"ahler and Ricci flat.
For superstrings, where the critical dimension is
$D=10$, the combination of these conditions amounts to
demanding that $K$ be a Calabi Yau manifold of complex
dimension $3$.

\Proclaim{Appendix A} \rm

The correlation functions in (6.C.1) are defined by
$$
\eqalign{
\left<\phi_1\ldots\phi_n\right>_g &=
\int D\phi\,\phi_1\ldots\phi_n\,e^{-S[\phi,g]}\cr
&={\delta\over\delta J_1}\ldots{\delta\over \delta
J_n}\,e^{W[J,g]}\vrulesub{J=0}\cr}
$$
Here, we consider correlation functions of a canonical
field $\phi$, which in our case is just the map $x$ or
the field $\xi$ and $J$ is some function, just as in
(6.E.1).
We now perform an infinitesimal Weyl rescaling
$\delta_\sigma$, whose support does not contain the
points where the operators $\phi_1\ldots\phi_n$ are
applied.
Thus the Weyl variation becomes:
$$
\delta_\sigma\left<\phi_1\ldots\phi_n\right>_g=
{\delta\over \delta J_1}\ldots{\delta\over \delta
J_n}\,
\delta_\sigma W[J,g]\,e^{-W[J,g]}\vrulesub{J=0}
$$
and using the definition of the effective action by
Legendre transform
$$
\delta_\sigma W[J,g]\vrulesub{J=0}=-\delta_\sigma
\Gamma[\varphi,g]\vrulesub{\varphi}
$$
where $\varphi$ is such that $J=0$.
Equation (6.H.1) follows immediately.



\bye


