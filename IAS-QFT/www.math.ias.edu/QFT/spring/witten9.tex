%Date: Wed, 10 Jun 1998 13:06:55 -0400 (EDT)
%From: Pavel Etingof <etingof@abel.math.harvard.edu>

\input amstex
\documentstyle{amsppt}
\magnification 1200
\NoRunningHeads
\NoBlackBoxes
\document

\def\h{\frak h}
\def\tW{\tilde W}
\def\Aut{\text{Aut}}
\def\tr{{\text{tr}}}
\def\ell{{\text{ell}}}
\def\Ad{\text{Ad}}
\def\u{\bold u}
\def\m{\frak m}
\def\O{\Cal O}
\def\tA{\tilde A}
\def\qdet{\text{qdet}}
\def\k{\kappa}
\def\RR{\Bbb R}
\def\be{\bold e}
\def\bR{\overline{R}}
\def\tR{\tilde{\Cal R}}
\def\hY{\hat Y}
\def\tDY{\widetilde{DY}(\g)}
\def\R{\Bbb R}
\def\h1{\hat{\bold 1}}
\def\hV{\hat V}
\def\deg{\text{deg}}
\def\hz{\hat \z}
\def\hV{\hat V}
\def\Uz{U_h(\g_\z)}
\def\Uzi{U_h(\g_{\z,\infty})}
\def\Uhz{U_h(\g_{\hz_i})}
\def\Uhzi{U_h(\g_{\hz_i,\infty})}
\def\tUz{U_h(\tg_\z)}
\def\tUzi{U_h(\tg_{\z,\infty})}
\def\tUhz{U_h(\tg_{\hz_i})}
\def\tUhzi{U_h(\tg_{\hz_i,\infty})}
\def\hUz{U_h(\hg_\z)}
\def\hUzi{U_h(\hg_{\z,\infty})}
\def\Uoz{U_h(\g^0_\z)}
\def\Uozi{U_h(\g^0_{\z,\infty})}
\def\Uohz{U_h(\g^0_{\hz_i})}
\def\Uohzi{U_h(\g^0_{\hz_i,\infty})}
\def\tUoz{U_h(\tg^0_\z)}
\def\tUozi{U_h(\tg^0_{\z,\infty})}
\def\tUohz{U_h(\tg^0_{\hz_i})}
\def\tUohzi{U_h(\tg^0_{\hz_i,\infty})}
\def\hUoz{U_h(\hg^0_\z)}
\def\hUozi{U_h(\hg^0_{\z,\infty})}
\def\hg{\hat\g}
\def\tg{\tilde\g}
\def\Ind{\text{Ind}}
\def\pF{F^{\prime}}
\def\hR{\hat R}
\def\tF{\tilde F}
\def\tg{\tilde \g}
\def\tG{\tilde G}
\def\hF{\hat F}
\def\bg{\overline{\g}}
\def\bG{\overline{G}}
\def\Spec{\text{Spec}}
\def\tlo{\hat\otimes}
\def\hgr{\hat Gr}
\def\tio{\tilde\otimes}
\def\ho{\hat\otimes}
\def\ad{\text{ad}}
\def\Hom{\text{Hom}}
\def\hh{\hat\h}
\def\a{\frak a}
\def\t{\hat t}
\def\Ua{U_q(\tilde\g)}
\def\U2{{\Ua}_2}
\def\g{\frak g}
\def\n{\frak n}
\def\hh{\frak h}
\def\sltwo{\frak s\frak l _2 }
\def\Z{\Bbb Z}
\def\C{\Bbb C}
\def\d{\partial}
\def\i{\text{i}}
\def\ghat{\hat\frak g}
\def\gtwisted{\hat{\frak g}_{\gamma}}
\def\gtilde{\tilde{\frak g}_{\gamma}}
\def\Tr{\text{\rm Tr}}
\def\l{\lambda}
\def\I{I_{\l,\nu,-g}(V)}
\def\z{\bold z}
\def\Id{\text{Id}}
\def\<{\langle}
\def\>{\rangle}
\def\o{\otimes}
\def\e{\varepsilon}
\def\RE{\text{Re}}
\def\Ug{U_q({\frak g})}
\def\Id{\text{Id}}
\def\End{\text{End}}
\def\gg{\tilde\g}
\def\b{\frak b}
\def\S{\Cal S}
\def\L{\Lambda}

\topmatter
\title Lecture II-9: Wilson loops,'t Hooft's loops, 
and 't Hooft's model of confinement
\endtitle
\author {\rm {\bf Edward Witten} }\endauthor
\endtopmatter

\centerline{Notes by Pavel Etingof and David Kazhdan}

\vskip .1in

\head 9.1. 't Hooft loop operator\endhead

Let us 
recall abelian duality in 4 dimensions, which we discussed two lectures ago. 
Consider a free $U(1)$ gauge theory with a 
$\theta$-angle. Thus we have two dimensionless couplings $e$, $\theta$ 
which combine into a single complex coupling $\tau=\frac{2\pi i}{e^2}+
\frac{\theta}{\pi}$, and the Lagrangian is 
$$
\Cal L=\int_M\biggl(\frac{i\bar\tau}{4\pi}F_+^2-
\frac{i\tau}{4\pi}F_-^2\biggl),\tag 9.1
$$
where $F$ is the curvature of a $U(1)$ connection $A$ 
and $F_+,F_-$ are the selfdual and antiselfdual parts of the curvature.  
We have seen that if the spacetime $M$ is a spin manifold then 
this theory is ``modular 
invariant'' as a function of $\tau$. One modular symmetry $\tau\to\tau+1$ 
is obvious, as it corresponds to shifting the $\theta$-angle by $\pi$,
which does nothing because $c_1^2$ for a spin manifold is even.  (On
a manifold that is not a spin manifold, the symmetry would be only
$\tau\to\tau+2$.)
The symmetry under the 
second generator of the modular group, $\tau\to -1/\tau$,
is more interesting and corresponds to electromagnetic duality discovered by 
Maxwell. More precisely, this means that the theory of a connection 
$A$ with coupling constant $\tau$ is identical
both classically and quantum mechanically to the same theory with coupling
$-1/\tau$ and the connection $B$ such that $dA=const*dB$. 
Now, like two lectures ago, we want to see what happens to operators 
under this duality. In particular, we want to know what happens to 
the Wilson loop operator. 

Recall that the Wilson loop operator has the form 
$W_\gamma(C)=e^{i\gamma Hol_C(A)}$,
where $Hol_C(A)$ denotes the integral of the connection $A$ along 
a closed oriented curve $C$ in the spacetime $M$. 
This operator is      gauge-invariant and well-defined if $\gamma$ is
an integer, or for any real $\gamma$ if the curve $C$ is homotopically
trivial in $M$.  (More generally, there could be several components
$C_i$ with real numbers $\gamma_i$, and the condition is that $\sum_i\gamma_i
C_i$ should be an integral class in $H_1(M)$.)
Matrix elements 
of this operator are computed, 
as usual, by inserting the above exponential into the 
path integral. Similarly to what we found in similar problems in
two and three dimensions, we should get 
that the dual description of the Wilson loop is a recipe which says 
that rather than insert in the path integral an object living 
on $C$, we should integrate over connections having a singularity 
along $C$. 

The precise answer is the following. For any curve $C$ 
the expectation value
$\<W_{\gamma}(C)\O_1...\O_n\>$ equals to 
$\int e^{-\Cal L(B)}\O_1...\O_nDB$, 
where the integral is taken over connections 
on $M\setminus C$ such that the integral of the curvature of $B$
over a small 2-sphere $S$ in a normal 3-space to $C$ at any point equals 
$2\pi \gamma$. 

Let us prove this. We assume that $C$ is a boundary. 
Let $D$ be a 2-chain whose boundary is $C$. 
Recall the calculation from the lecture on Abelian duality:
our fields are $A$ -- the original connection, $G$- the 2-form, and $B$ -- 
the dual connection. We have 
$$
\gather
\int e^{-L(A)}W_\gamma(C)DA=\\
\int DA\,DG\,DB e^{\frac{-i\bar\tau}{4\pi}
\int\Cal F_+^2+\frac{i\tau}{4\pi}
\int\Cal F_-^2}e^{\frac{i}{2\pi}\int G\wedge F_B}
e^{i\gamma\int_D(F_A-G)},\tag 9.1
\endgather
$$  
where $\Cal F=F_A-G$, and the last factor corresponds to 
the insertion of the Wilson loop (recall from the abelian duality discussion 
that the Wilson loop classically 
is $e^{i\gamma\int_DF_A}$, and $F_A$ is to be replaced with $F_A-G$
in the extended theory). Gauging $A$ to $0$, we get 
$$
\gather
\int e^{-L(A)}W_\gamma(C)DA=\\
\int DG\,DB e^{\frac{-i\bar\tau}{4\pi}
\int G_+^2+\frac{i\tau}{4\pi}
\int G_-^2}e^{\frac{i}{2\pi}\int G\wedge F_B}
e^{-i\gamma\int_D G}=\\
\int DG\,DB e^{\frac{-i\bar\tau}{4\pi}
\int G_+^2+\frac{i\tau}{4\pi}
\int G_-^2}e^{i\int G\wedge(\frac{1}{2\pi}F_B-\gamma [D])},
\tag 9.2
\endgather
$$  
where $[D]$ is the delta-function of $D$. Now we define $\tilde F_B:=
F_B-2\pi\gamma [D]$. Then 
after integrating out $G$, (9.2) can be written as
$$
\gather
\int e^{-L(A)}W_\gamma(C)DA=\\
\int DB e^{\frac{i}{4\pi\bar\tau}
\int (\tilde F_B)_+^2+\frac{-i}{4\pi\tau}
\int (\tilde F_B)_-^2},
\tag 9.3
\endgather
$$  
Thus, the effect of the insertion $W_\gamma(C)$ is that $F_B$ is 
replaced in the final answer by $\tilde F_B$. So $B$ is now a connection 
on a line bundle with singularity along $C$, as discussed above.

Note that if $C$ is not a boundary then $\<W_\gamma(C)\>=0$
(even with insertion of any number of local operators). Indeed, we have  
symmetry $A\to A+d\phi$, and $\phi$ does not have to be globally defined
as a map to a circle; in fact, $d\phi$ can be any closed one-form.
So if $C$ is not a boundary then we can choose $d\phi$ in such a way 
that the operator $W_\gamma(C)$ will multiply by $e^{i\alpha}$ for some
nonzero $\alpha$. Hence its expectation value (even with inclusion
of local operators, which are invariant under this transformation) vanishes.
More generally, the correlator 
$\<W_{\gamma_1}(C_1)...W_{\gamma_n}(C_n)\>$ (with any local operators) 
is zero if $\sum \gamma_iC_i\ne 0$ in $H_1(M)$, where $M$ is the spacetime. 
As noted before, for the product of operators in question to be well-defined,
we only need $\sum\gamma_iC_i$ to be an integral class.

 From our construction so far, for any curve $C$ we have two operators: 

1) $W_\gamma(C)=e^{i\gamma\int_CA}$

2) $T_\gamma(C)=e^{i\gamma\int_CB}$.


The second operator, which is dual to the Wilson loop, is called 
the 't Hooft loop operator.  

\head 9.2. Hilbert space interpretation of the 't 
Hooft loop operator
\endhead

Now let us consider this picture from the Hamiltonian point of view. 
Then the spacetime $M$ has the form $M=M^3\times \R$ with Minkowski 
metric. Let $C,C'$ be two nonintersecting closed simple 
 curves in $M^3$. They define operators 
$W_\gamma(C)$ and $T_{\gamma'}(C')$ on the Hilbert space $\Cal H$. 
The following commutation relation for these operators is due to 
't Hooft:
$$
W_\gamma(C)T_{\gamma'}(C')=e^{2\pi i\gamma\gamma'l(C,C')}
T_{\gamma'}(C')W_\gamma(C).\tag 9.4
$$

Let us prove this formula. Let us work in terms of the original connection
$A$. Then the Hilbert space consists of wave functions $\Psi(A)$. 
In this realization, the Wilson loop operator 
 $W_\gamma(A)$ is simply the operator of multiplication by
$e^{i\gamma\int A}$. 

However, the 't Hooft loop operator is 
a bit harder to define. To do this, 
consider the homomorphism $\pi_1(M^3\setminus 
C',x_0)\to\Z$ given by the linking 
number with $C'$. Let $\pi_1^0$ be the kernel of this homomorphism 
and $X$ be the $\Z$-cover of $M_3\setminus C'$ corresponding to 
$\pi_1^0$. Let $\phi$ be a 
function $X\to U(1)$ such that the monodromy corresponding to 
the generator of $\Z$ is $e^{i\gamma}$. Any two such functions 
differ by a gauge transformation, but $\phi$ itself is not an honest gauge 
transformation. Then it is not difficult to check that the 't Hooft loop 
operator $T_{\gamma'}(C')$ is just the ``illegal'' gauge transformation 
by $\phi$: 
$$
(T_{\gamma'}(C')\Psi)(A)=\Psi(A^{\phi}).\tag 9.5
$$ 
Note that this is well-defined since any two such $\phi$'s differ by an 
honest gauge transformation. 

Now formula (9.4) is clear since $TWT^{-1}=e^{i\gamma_{C}\int d\phi}W$,
because of the way that $T$ transforms the connection in the definition of
$W$.

\head 9.3. The 2+1-dimensional analogue of the 3+1-dimensional picture
\endhead

Consider the 2+1-dimensional analogue of this picture. As we saw before, 
in 2+1 dimensions the theory of a scalar field $\phi$ is dual to a gauge 
theory of the dual field $A$. The path integral in $\phi$ with 
insertion of $e^{i\phi(x)}$ is the same as path integral in $A$ where $A$ is 
a connection on $M\setminus x$ which has $\int F=2\pi$, where the 
integral is over a small sphere around $x$. Thus, the operator 
$e^{i\phi}$ corresponds to a magnetic monopole in gauge theory. 

Now consider the 3-dimensional cosine theory, defined by the path integral 
$$
\int D\phi e^{-\int (|d\phi|^2+\e(e^{i\phi}+e^{-i\phi}))}.\tag 9.6
$$
Decomposing this path integral in a power series, and passing to the dual 
variable $A$, we get the sum 
$$
\sum_{m,n}\e^{m+n}\int \frac{dx_1...dx_m}{m!}\int \frac{dy_1...dy_n}{n!}
\int_{\Cal A_{x,y}} e^{-\int F_A^2},\tag 9.7
$$
where $\Cal A_{x,y}$ is the space of connections with monopoles 
at $x_i$ and antimonopoles at $y_j$. Thus the cosine theory maps to 
the theory with monopoles. We saw this more computationally 
when we discussed the Polyakov model two lectures ago. 

\head 9.4. The model of confinement
\endhead

Now we discuss a picture of confinement developed by 't Hooft. 
In general we don't assume that the gauge group is abelian. 
Recall the definition of confinement.
We have a gauge group $G$ and with universal cover $\widehat G$. 
We assume that
 $G$ is the quotient of $\widehat G$ 
which acts faithfully on all fields in the Lagrangian.
We let $R$ be a representation of $\widehat G$. 
As we discussed before, if there is a mass gap, 
there are two usual patterns of decay 
of the expectation value $\<W_R(C)\>$ of the Wilson line operator 
corresponding to the representation $R$ as $C$ gets big:

Pattern 1:
$$
\<W_R(C)\>\sim 
e^{-\l \,\text{Length}(C)}\tag 9.8
$$

Pattern 2:
$$
\<W_R(C)\>\sim 
 e^{-\l\, \text{Area}(C)}\tag 9.9
$$
(here the parameter $\gamma$ is a fixed nonzero number and 
the area of $C$ means the minimal area of the spanning surface).  
The first regime is called the Higgs regime (the length law) 
and the second one is called the 
confinement regime (the area law). 

As we discussed in the previous lecture on confinement, the first regime is 
the case when $R$ is a representation of $G$ itself, and to see confinement 
one needs to consider 
the case when $R$ is a representation of $\widehat G$ but not of $G$. 
Thus interesting $W_R(C)$ correspond to elements of $\pi_1(G)^*$. 

Now let us consider the 't Hooft loop operator $T_\gamma(C)$.  It
is defined for any $G$ by analogy with the definition in the abelian
case.  We fix an element $\gamma\in \pi_1(G)$.  Recall that 
$G$-bundles on a two-sphere $S^2$ are classified by a characteristic class
that takes values in $H^2(S^2,\pi_1(G))$, which is canonically
isomorphic to $\pi_1(G)$.  The choice of $\gamma$ therefore canonically
determines an isomorphism class of $G$-bundles on $S^2$. We can now
define the 't Hooft operator: a path integral
with insertion of $T_\gamma(C)$ is computed by integrating over
connections on $M\setminus C$ which have the property that when restricted
to a small sphere $S$ that links $C$, the bundle has characteristic class
$\gamma$.


't Hooft's idea was to consider $T_\gamma(C)$ instead of $W_R(C)$ 
and find conditions under which there is an
 area law for its expectation value.   This occurs, as he showed,
 for certain Higgs theories.  Then, 't Hooft proposed (following earlier
 ideas of Nambu, Mandelstam, and others) that confinement would be related
 to the Higgs mechanism by a duality that maps 't Hooft loop operators
 into  Wilson loop operators.  This does not explain confinement, but
 it reformulates the problem: to reduce the mysterious phenomenon of 
confinement
 to the much more easily understood Higgs phenomenon, one must understand
 the nonlinear duality that exchanges 't Hooft and Wilson loop operators.

To illustrate the area law for the 't Hooft loop in Higgs theories,
we consider a familiar example:
the $U(1)$ gauge theory with a charged complex scalar $\phi$ 
(of charge 1). The Lagrangian is
$$
\int(\frac{|F|^2}{4e^2}+|D_A\phi|^2+V(\phi\bar\phi)),\tag 9.10 
$$
where $V$ is a (quartic) potential. We will study the 't Hooft loop 
$T_\gamma(C)$, where $\gamma\in\Z$. Thus 
$A$ is a connection and $\phi$ is a section for a hermitian line bundle over 
$M\setminus C$ such that it has first Chern class $\gamma$ when restricted
to a small sphere linking $C$. We will compute 
$\<T_\gamma(C)\>$ for two classes of $V$:

1) $V=\l(\phi\bar\phi+a^2)^2$;

2) $V=\l(\phi\bar\phi-a^2)^2$. 

This theory was considered in Lecture 2. Recall the results of this 
consideration. 

Case 1. In the infrared the theory behaves like the product 
of the theory of a free massive field with a free gauge theory. 
In particular, there is no mass gap. 
Thus, we can calculate $\<T_\gamma(C)\>$ for large $C$ using the free theory. 
But in the free theory this expectation value 
 is the same as $\<W_\gamma(C)\>$  
in the dual theory. It is easy to see that the expectation 
values of both $W_\gamma$ and $T_\gamma$ behave 
according to the length law, because of Coulomb law of charge interaction. 
A theory behaving in this way is said to be in the  ``Coloumb phase.''

Case 2. In the infrared this theory has breaking of gauge symmetry 
and a Higgs mechanism. In particular, speaking classically,
we have a circle of vacua, and at each of 
these vacua the low energy part of the Hamiltonian spectrum 
contains a massive vector and a real massive scalar. So there is a mass gap. 
This theory is not believed to exhibit confinement, i.e. it is believed 
that it exhibits the length law for the Wilson loop.  This is certainly
what one computes in perturbation theory.

In case 2, we will show that there is an area law for the 't Hooft loop 
operator, because of the Higgs mechanism. This happens for topological 
reasons, as explained below.  

As a warmup consider a closed spacetime $M$ and a line bundle 
$\Cal L$ with a nontrivial $c_1$. Let us consider the path integral 
for our theory over sections of this bundle. It turns out that 
the action of all field configurations in this integral has to be very large:
it is bounded below by a constant (which is independent of $M$) times
 the area of the minimal 2-surface which represents a cycle 
Poincare dual to $c_1(\Cal L)$. 

Indeed, if $\phi$ is a section of $\Cal L$ then $\phi$ has to vanish 
on a 2-cycle $\Sigma$ which is dual to $c_1(\Cal L)$. If we 
fix $\Sigma$, we can look at the configuration of minimal action
with such a zero.  For this, we can (if the metric of $M$ is scaled
up) reduce to the case that $M=\R^4=\R^2\times \R^2$,
with $\Sigma$ equal to the first factor.  We can assume that $\phi$ and $A$
are invariant under translations of the first factor in $M$.
In the second factor, we want $\phi$ to vanish at the origin and to
approach the vacuum at infinity (up to gauge transformation), such that
the first Chern class of the bundle, relative to the trivialization at
infinity given by $\phi$, equals 1.
The same problem appeared in lecture 2 (in the guise of finding an instanton
solution of the two-dimensional version of the same model), and we
discussed qualitative properties of the solution.
Anyway, let $I$ be the action of this solution in the two-dimensional
sense (that is, integrated over just the second factor in $M$).   
Going back to a global $\Sigma\subset M$ of smallest area representing
the first Chern class, the minimum action field looks in the normal
directions to $\Sigma$ like the instanton just described; its action
is approximately $I\cdot \text{Area}(\Sigma)$.

Now let us come back to the 't Hooft loop in $\R^4$. 
In this case the bundle is 
over $M\setminus C$, where $M$ is the spacetime. If $D$ is a 
2-chain in $M$ whose boundary is $C$ then $D$ plays 
the role of the $\Sigma$ of the previous discussion. Indeed, if 
$\Cal L$ is a line bundle over $M\setminus C$ with Chern class 1, 
and $\phi$ its section then $\phi$ must vanish on a 2-surface whose boundary 
is $C$. Thus, the same argument as above shows that 
$\<T_1(C)\>\sim e^{-\l\text{ area(D)}}$, where $\text{area}(D)$ is the 
smallest 
area of a disk spanning $C$. This is the area law which we wanted to 
demonstrate. 

This behavior is characteristic of what is called  the Higgs regime, or phase. 

Now let us discuss in more detail the relation of 
the established behavior of the 't Hooft loop operator with confinement. 
It is believed that if $\pi_1(G)\ne 0$ there are at least
three possible phases:

1) Coulomb: no mass gap, gauge bosons in the infrared, 
$W$ and $T$ behave like in the free theory and exhibit the length law. 

2) Higgs: mass gap, length law for $W$, area law for $T$.

3) Confinement: mass gap, area law for $W$, length law for $T$.
   
As already suggested,
't Hooft's idea was that there should be a nonabelian analogue of duality 
which interchanges $W$ with $T$, the Higgs and the confinement regimes, 
and maps the Coloumb phase to itself. Thus, the area law for 
$T$ in a theory implies confinement in the dual theory. This is what happens
for some supersymmetric theories, e.g. the theory relevant to
Donaldson theory. 

In fact, 't Hooft showed that if $R$ and $\gamma$ are such that 
$\gamma|_R\ne 1$ then either $W_R$ on $T_\gamma$ exhibit the area law. 
More specifically, he proved an even stronger statement, namely that 
the set $H$ of all $(c,\gamma)\in\pi_1(G)^*\times \pi_1(G)$ such that 
for some representation $R$ with central character $c$ the operator 
$W_R(C)T_\gamma(C)$ (suitably renormalized) does not exhibit the area law,
is an isotropic subgroup of $\pi_1(G)^*\times \pi_1(G)$ 
with respect to the natural symplectic form. The reason for this, very 
roughly, is the following.  If $A(C)=W_{R_1}T_{\gamma_1}(C)$ and $B(C)=
W_{R_2}T_{\gamma_2}(C)$ 
exhibit the length law then, when acting on the vacuum, they produce
only effects that are localized along $C$ (or there would be an area
law instead of a length law).  So
$$
\<A(C)B(C')\>=\<A(C)\>\<B(C')\>(1+o(1)),d\to\infty\tag 9.11
$$ 
where $d$ 
is the distance between $C$ and $C'$. This shows that if $AB=qBA$ 
where $q$ is a constant then $q$ must be equal to 1. 
By 't Hooft's formula (9.4) (which is clearly valid in the 
nonabelian case as well), this implies that $H$ is isotropic.
By further physical arguments, one shows that if there is a mass gap,
then $H$ is maximal isotropic.  This leads to a more refined classification
of massive phases than was stated above: associated to each massive
phase is a maximal isotropic subgroup of $\pi_1(G)^*\times \pi_1(G)$.
All possibilities can arise, in general.

{\bf Remark.} The  argument showing that loop operators $A$ and $B$ must
commute fails if one of the operators, 
say $A$, exhibits the 
area law.  In  this case, acting on the vacuum with this operator produces
an effect that is not in any way localized 
near $C$; it rather has an effect which is localized near a minimal area disk 
$D$ whose boundary is $C$; such a disk will always intersect $C'$ when the 
linking number is not zero. Formula (9.11) is now valid only if 
$d$ is the distance from $C'$ to $D$, which is always 0, so the formula 
does not tell us anything. 
  
{\bf Remark.} The area law for the 't Hooft operator in a Higgs
phase has many physical and mathematical applications.  For example,
with some small adjustments, what we said above  in analyzing the
behavior of the Higgs phase with a bundle of nonzero first Chern class
could serve as an explanation
of the Meissner effect, the fact that a superconductor (which is described
approximately by the  abelian Higgs model that we examined) expels
magnetic flux.
Perhaps the reader has, at a science fair, seen a demonstration of a
magnet floating above a superconductor; this effect has the same origin.
Mathematically, C. Taubes's analysis of the Seiberg-Witten invariants
of symplectic four-manifolds made use of the same facts: the localization
(in a closely analogous system of equations) of the zeroes of $\phi$
on a surface of smallest area.
\end







