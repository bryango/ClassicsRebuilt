
%From: Lisa C Jeffrey <jeffrey@math.ias.edu>
%Date: Wed, 13 Nov 1996 12:42:34 -0500
%Subject: Faddeev lect 3: please post on web


\documentstyle[12pt]{article}
\input amssym.def
\input amssym.tex
\newcommand{\Tr}{\,{\rm Tr}\,}
\newcommand{\tr}{{\rm tr}\,}

%L. Jeffrey preamble, 5 April 1996

\newcommand{\nc}{\newcommand}

\nc{\isq}{{i}}
\newcommand{\colvec}[2]{\left  ( \begin{array}{cc} #1  \\
     #2  \end{array} \right ) }

%%%%%\newcommand{\Tr}{\,{\rm Tr}\,}
\newcommand{\End}{\,{\rm End}\,}
\newcommand{\Hom}{\,{\rm Hom}\,}

\newcommand{\Ker}{ \,{\rm Ker} \,}

\newcommand{\bla}{\phantom{bbbbb}}
\newcommand{\onebl}{\phantom{a} }
\newcommand{\eqdef}{\;\: {\stackrel{ {\rm def} }{=} } \;\:}
\newcommand{\sign}{\: {\rm sign}\: }
\newcommand{\sgn}{ \:{\rm sgn}\:}
\newcommand{\half}{ {\frac{1}{2} } }
\newcommand{\vol}{ \,{\rm vol}\, }


% define abbreviations for most common commands
%

\newcommand{\beq}{\begin{equation}}
\newcommand{\eeq}{\end{equation}}
\newcommand{\beqst}{\begin{equation*}}
\newcommand{\eeqst}{\end{equation*}}
\newcommand{\barr}{\begin{array}}
\newcommand{\earr}{\end{array}}
\newcommand{\beqar}{\begin{eqnarray}}
\newcommand{\eeqar}{\end{eqnarray}}
\newtheorem{theorem}{Theorem}[section]
%\newtheorem{conjecture}{Conjecture}
\newtheorem{corollary}[theorem]{Corollary}
%\newtheorem{problem}{Problem}
\newtheorem{lemma}[theorem]{Lemma}
\newtheorem{prop}[theorem]{Proposition}
\newtheorem{definition}[theorem]{Definition}
\newtheorem{remit}[theorem]{Remark}
\newtheorem{conjecture}[theorem]{Conjecture}

\newtheorem{example}[theorem]{Example}

\newcommand{\matr}[4]{\left \lbrack \begin{array}{cc} #1 & #2 \\
     #3 & #4 \end{array} \right \rbrack}



\newenvironment{rem}{\begin{remit}\rm}{\end{remit}}




% black board bold face
%note \AA is already defined!
\newcommand{\aff}{{ \Bbb A }}
\newcommand{\RR}{{{\bf  R }}}
\newcommand{\CC}{{{\bf  C }}}
\nc{\FF}{ {\Bbb F} } 
\newcommand{\ZZ}{{{\bf   Z }}}
\newcommand{\PP}{ {\Bbb P } }
\newcommand{\QQ}{{\Bbb Q }}
\newcommand{\UU}{{\Bbb U }}








%***************************




%


%


%Replace greek letters by their roman equivalents with \
%Slightly nonstandard:  theta is \t, tau is \ta, no omicron
\def\a{\alpha}
\def\b{\beta}
\def\g{\gamma}
\def\d{\delta}
\def\e{\epsilon}
\def\z{\zeta}
\def\h{\eta}
\def\t{\theta}
%\def\i{\iota}
\def\k{\kappa}
\def\l{\lambda}
\def\m{\mu}
\def\n{\nu}
\def\x{\xi}
\def\p{\pi}
\def\r{\rho}
\def\s{\sigma}
\def\ta{\tau}
\def\u{\upsilon}
\def\ph{\phi}
\def\c{\chi}
\def\ps{\psi}
\def\o{\omega}

\def\G{\Gamma}
\def\D{\Delta}
\def\T{\Theta}
\def\L{\Lambda}
\def\X{\Xi}
\def\P{\Pi}
\def\S{\Sigma}
\def\U{\Upsilon}
\def\Ph{\Phi}
\def\Ps{\Psi}
\def\O{\Omega}




% calligraphic letters
\newcommand{\calA}{{\mbox{$\cal A$}}}
\newcommand{\calB}{{\mbox{$\cal B$}}}
\newcommand{\calC}{{\mbox{$\cal C$}}}
\newcommand{\calD}{{\mbox{$\cal D$}}}
\newcommand{\calE}{{\mbox{$\cal E$}}}
\newcommand{\calF}{{\mbox{$\cal F$}}}
\newcommand{\calG}{{\mbox{$\cal G$}}}
\newcommand{\calH}{{\mbox{$\cal H$}}}
\newcommand{\calI}{{\mbox{$\cal I$}}}
\newcommand{\calJ}{{\mbox{$\cal J$}}}
\newcommand{\calK}{{\mbox{$\cal K$}}}
\newcommand{\calL}{{\mbox{$\cal L$}}}
\newcommand{\calM}{{\mbox{$\cal M$}}}
\newcommand{\calN}{{\mbox{$\cal N$}}}
\newcommand{\calO}{{\mbox{$\cal O$}}}
\newcommand{\calP}{{\mbox{$\cal P$}}}
\newcommand{\calQ}{{\mbox{$\cal Q$}}}
\newcommand{\calR}{{\mbox{$\cal R$}}}
\newcommand{\calS}{{\mbox{$\cal S$}}}
\newcommand{\calT}{{\mbox{$\cal T$}}}
\newcommand{\calU}{{\mbox{$\cal U$}}}
\newcommand{\calV}{{\mbox{$\cal V$}}}
\newcommand{\calW}{{\mbox{$\cal W$}}}
\newcommand{\calX}{{\mbox{$\cal X$}}}
\newcommand{\calY}{{\mbox{$\cal Y$}}}
\newcommand{\calZ}{{\mbox{$\cal Z$}}}

%% To load script letters:

\font\teneusm=eusm10  \font\seveneusm=eusm7 
\font\fiveeusm=eusm5 
\newfam\eusmfam 
\textfont\eusmfam=\teneusm 
\scriptfont\eusmfam=\seveneusm 
\scriptscriptfont\eusmfam=\fiveeusm 
\def\Scr#1{{\fam\eusmfam\relax#1}}

% Script letters
\newcommand{\ScrA}{{\Scr A}} \newcommand{\ScrB}{{\Scr B}}
\newcommand{\ScrC}{{\Scr C}} \newcommand{\ScrD}{{\Scr D}}
\newcommand{\ScrE}{{\Scr E}} \newcommand{\ScrF}{{\Scr F}}
\newcommand{\ScrG}{{\Scr G}} \newcommand{\ScrH}{{\Scr H}}
\newcommand{\ScrI}{{\Scr I}} \newcommand{\ScrJ}{{\Scr J}}
\newcommand{\ScrK}{{\Scr K}} \newcommand{\ScrL}{{\Scr L}}
\newcommand{\ScrM}{{\Scr M}} \newcommand{\ScrN}{{\Scr N}}
\newcommand{\ScrO}{{\Scr O}} \newcommand{\ScrP}{{\Scr P}}
\newcommand{\ScrQ}{{\Scr Q}} \newcommand{\ScrR}{{\Scr R}}
\newcommand{\ScrS}{{\Scr S}} \newcommand{\ScrT}{{\Scr T}}
\newcommand{\ScrU}{{\Scr U}} \newcommand{\ScrV}{{\Scr V}}
\newcommand{\ScrW}{{\Scr W}} \newcommand{\ScrX}{{\Scr X}}
\newcommand{\ScrY}{{\Scr Y}} \newcommand{\ScrZ}{{\Scr Z}}

%German (Faktur) letters

\newcommand{\grA}{{\frak A}}



\def\eps{\varepsilon}

\setlength{\textwidth}{6.5in}
\setlength{\textheight}{9.1in}
\setlength{\evensidemargin}{0in}
\setlength{\oddsidemargin}{0in}
\setlength{\topmargin}{-.75in}
\setlength{\parskip}{0.3\baselineskip}

%\renewcommand{\theequation}{\thesection.\arabic{equation}}
%\newcommand{\renorm}{{ \setcounter{equation}{0} }}

\newcommand{\Le}{{{\mathchoice{\,{\scriptstyle\le}\,}
  {\,{\scriptstyle\le}\,}
  {\,{\scriptscriptstyle\le}\,}{\,{\scriptscriptstyle\le}\,}}}}
\newcommand{\Ge}{{{\mathchoice{\,{\scriptstyle\ge}\,}
  {\,{\scriptstyle\ge}\,}
  {\,{\scriptscriptstyle\ge}\,}{\,{\scriptscriptstyle\ge}\,}}}}


%\renewcommand{\baselinestretch}{1.5}


%more newcommands
\nc{\bra}{  < }
\nc{\ket}{ > }
%\nc{\isq}{ { \sqrt{-1} } }
%\nc{\isq}{{ i }}
%\nc{\hbar}{{ h}}
\nc{\triang}{ { \bigtriangleup} }

\nc{\astar}{{ a^*}}
\nc{\hata}{ { \hat{a} }}
\nc{\hatastar}{{ \hat{\astar} }}
\nc{\normfac}{{\frac{1}{\sqrt{2 \omega} } }}

\begin{document}

\title{Lecture 3:\\
Comments on Scattering}
\author{Ludwig Faddeev}
\date{21  October 1996}

\maketitle
%\renorm
\nc{\dphi}{\tilde{\phi}} 
\nc{\vac}{{ \sl v}}
\nc{\basvec}[1]{\Phi_{#1} (\astar) } 
\nc{\basvecn}{\basvec{n} } 
\nc{\kn}{(\vec{k})_{n}} 
\nc{\vk}{\vec{k} }

We first describe how to rewrite the 
expression for the $S$-matrix (equation (43) from Lecture 2)

\beq \label{smat}
\calS'(\varphi_0) = 
\exp \left \{ - i \lambda \int \calV (\frac{\delta}{\delta \eta(x)}) dx 
\right \} 
 \exp \left \{ i\int_V \eta(x) \varphi(x) dx + 
\frac{i}{2} \int_{V \times V} G(x-y) \eta(x) \eta (y) dx dy 
\right \} . \eeq 
Here, 
the argument of $\calS'$ is the solution of the free Klein-Gordon equation
\beq \varphi_0 (x) = 
\frac{1}{ (2 \pi)^{3/2} } 
\int  \left ( \astar(\vk)  e^{i k \cdot x} + a (\vk) e^{-i k \cdot x} 
\right )
\frac{d \vk}{ \sqrt{2 k_0} }, \eeq
for $k_0 = \omega(\vk)$. 
We shall rewrite (\ref{smat})
as a path integral of the form
\beq   \label{pathints} \calS'(\varphi_0) = 
\int_{\varphi} \exp \Bigl \{ i \int L(x) dx \Bigr \} 
\prod_{x} d \varphi(x) , \eeq
where we integrate over fields $\varphi$ tending asymptotically 
to certain asymptotic values $\varphi_{\rm in}$ (resp.
$\varphi_{\rm out} $)  as $ t  \to - \infty$ 
(resp. $t \to + \infty$). 
Starting from the formula (\ref{pathints}) one may derive the 
Feynman rules for an  expansion of the S-matrix in terms of Feynman 
diagrams, where we expand
the Lagrangian $L$ about its  stationary 
point  $\varphi_{\rm cl}$.


Our exposition closely follows Section 2.4 of \cite{FS}, which is 
appended. 

We now pass to several examples which illustrate the notion 
of renormalization. First we shall consider an instructive 
example which shows that the mass renormalization is not necessarily
associated with infinities. Rather it is connected with 
the proper definition of the one particle states. As an example
let us consider a Hamiltonian 
\beq \label{ham} H = H_0 + \lambda \sum_{n, m \ge 0 } v_{n,m}.\eeq
Here $H_0$ is the free scalar field Hamiltonian 
\beq \label{ffh} H_0 = \int \omega (\vk) \astar(\vk) a(\vk) d\vk, \eeq
$\lambda$ is a coupling constant 
and $v_{n,m}$ are polynomials in the creation and annihilation 
operators:

\beq \label{vnmdef} 
v_{n,m} = \int {\bf \calV} \Bigl ( (\vk)_n, (\vk')_m \Bigr ) \astar (\vk_1) \dots 
\astar (\vk_n)  a(\vk'_1) \dots a(\vk'_m) \delta (\vk_1 + \dots + \vk_n - 
\vk'_1 - \dots - \vk'_m) (d\vk)_n (d \vk')_m. \eeq
The delta function in (\ref{vnmdef}) implies that the Hamiltonian 
commutes with the momentum operators:
\beq [H, \vec{P}] = 0, \eeq
where 
$$ \vec{P} = \int \vec{k} \astar(\vk) a(\vk) d\vk. $$



We shall make frequent use of the commutation relations
\beq \label{commrel} [a(\vk), \astar (\vk')] = \delta(\vk - \vk'), \eeq
\beq  [a(\vk), a (\vk')] = [\astar(\vk), \astar(\vk')] = 0. \eeq
We can consider that the $v_{11}$ term is absent in (\ref{ham}); otherwise
it is to be included into $H_0$ from the beginning. 

Now if terms $v_{n,0}$ are present then the Fock vacuum $\Omega$
for which 
$$ a(\vk) \Omega = 0 $$
is not an eigenvector of $H$. So the physical vacuum is distinct 
from $\Omega$. Furthermore terms $v_{n,1}$, even in the absence
of terms $v_{n,0}$, preclude the one particle states
$$ \phi_1(\vk) = \astar(\vk) \Omega $$
from being eigenvectors of $H$. On the other hand a Hamiltonian 
of the form (\ref{ham}) but  summing only over $n$ and $m$ such that
$n \ge 2$ and $m \ge 2$ has the same vacuum and one particle
states as $H_0$. One can apply to it the usual formalism of 
scattering theory.

We shall show (in the framework of formal series expansions) that 
it is possible to find a unitary transformation $R = e^{i Q}$, 
where $Q$ is given by an expression 
$$ Q = Q_1 + Q_1^* + Q_2 + Q_2^*$$
and 
$$Q_1 = \sum_n Q_{n,0}, ~~~~ Q_1^* = \sum_{m } \bar{Q}_{0,m} $$
$$Q_2 = \sum_n Q_{n,1}, ~~~~ Q_2^* = \sum_{m } \bar{Q}_{1,m} $$
with terms of the type $(n,0)$, $(0,m)$, $(n, 1)$ and 
$(1, m)$, and corresponding coefficient 
functions $q(\vk)_n)$, 
$\bar{q}((\vk')_m), $ 
$q((\vk)_n, \vk')$,  and $\bar{q}(\vk,(\vk')_m), $ 
such that after the conjugation 
$$ \tilde{H} = R H R^{-1} $$
all unwanted terms can be cancelled. Of course, each $Q$ is a power 
series in $\lambda $:
$$ Q = \sum \lambda^j Q^j$$
Alternatively we can say that in a new representation
of canonical variables
$$ b^{\#} = R a^{\#} R^{-1} $$
the Hamiltonian $H$ has the same vacuum and one particle states
as a free hamiltonian $\tilde{H_0}$, which however is different from 
$H_0$. 




We shall define $\tilde{H}^{(j)} $ $ = R H R^{-1} + O (\lambda^{j+1}), $
solving in such a way that all the $\tilde{H}^{(j)}$ only 
include terms with $n \ge 2 $ and $m \ge 2$. 
We see that 
\beq \label{eiqdef}
 e^{i Q} H e^{-i Q} = H_0 + \lambda ( i [Q^1, H_0] + \sum_{n,m} v_{n,m})
+ \lambda^2 (- [Q^2, H_0 ] + i [Q^1, \sum_{n,m} v_{n,m} ] ) + \dots \eeq
We shall decompose  this equation into its
components at every order in $\lambda$, successively cancelling
the terms $v_{n,0}, $ $v_{0,m}$, $v_{n,1}$ and $v_{1,m}$. 

To leading order in $\lambda$, we need to solve
\beq  \label{qone} 
i [Q^1, H_0] + \sum_{n \ge 0 } \sum_{m \ge 0 }  v_{n,m} = 
\sum_{n = 2}^\infty \sum_{m = 2}^\infty v^{(1)} _{n,m}, \eeq
with some $v_{n,m}^{(1)}$ (in other words we arrange to cancel the terms
$v_{n,m}$ with $n \le 1$ or $m \le 1$). 
We shall show how to choose $Q^1$  to satisfy (\ref{qone}); then 
 to second order in $\lambda$ the equation
 (\ref{eiqdef}) implies 
\beq \label{qtwo}
 - [Q^2, H_0] + i [Q^1, \sum_{n \ge 0} \sum_{m \ge 0 } v_{n,m}] = 
\sum_{n \ge 2 } \sum_{m \ge 2} v^{(2)}_{n,m} \eeq
for some $v_{n,m}^{(2)}$ with $n \ge 2$ and $m \ge 2$. 
It will then be necessary to solve (\ref{qtwo}) for $Q^2$, and so on;
this process enables us successively  to cancel all the terms 
$v_{n,1}, v_{1,m}, v_{n,0}, v_{0,m}$. 

We now describe how to solve equation (\ref{qone})  for $Q^1$:
exactly the same  procedure will then suffice to solve (\ref{qtwo}) for 
$Q^2$, and so on. 
Since (\ref{qone}) expresses
$Q^1$ as a linear function of  $\sum_{n,m} v_{n,m}$, it suffices
to solve the equations
\beq \label{e2} [Q_1^1, H_0] =i v_{n,0}, \eeq
\beq \label{e3} [(Q_1^1)^*, H_0] = i v_{0,m}, \eeq
\beq \label{e4} [Q_2^1, H_0] = iv_{n,1}, \eeq
\beq \label{e5} [(Q_2^1)^*, H_0] = i v_{1,m} \eeq
for individual values of $n$ and $m$.

Notice that for any operator $r_{n,m} $ of type 
$n,m$, 
$$[r_{n,m}, H_0] = \sum_{n,m} 
\int d\vk_1 \dots d\vk_n 
d\vk'_1 \dots d\vk'_m  \delta (\vk_1 + \dots + \vk_n - 
\vk'_1 - \dots - \vk'_m) \times $$
$$
r_{n,m} ((\vk)_n, (\vk')_m) 
\astar(\vk_1) \dots \astar (\vk_n) a(\vk'_1) \dots a(\vk'_m)
  (\sum_j  \omega (\vk_j) - \sum_j \omega (\vk'_j)) 
 , $$
so that the equations (\ref{e2} - \ref{e5}) are solved simply 
by dividing the coefficient functions
$v_{n,0}$, $v_{0,m}$, $v_{n,1}$, $v_{1,m}$ by factors
$\sum_j  \omega(\vk_j ),$ $  - \sum_j (\omega (\vk'_j)$,
 $\sum_j  \omega(\vk_j ) -
 \omega (\vk'_1), $ $\omega (\vk_1) - \sum_j \omega (\vk'_j )$
respectively. Assuming that the one-particle energy is {\em convex}, 
in other words that
$$\omega (\vk_1) + \omega (\vk_2) \ge \omega(\vk_1 + \vk_2) $$
(which means that a particle cannot decay into two
or more particles while conserving momentum) we see that the 
factors above are nonvanishing and there is no difficulty in dividing by 
them. 






However this method cannot be generalized to cancel terms 
$v_{n,m}$ where both $n \ge 2$ and $m \ge 2$, for then the 
factor appearing in the integral for $[Q_1, H_0]$ 
involves $\sum_{j = 1}^n \omega(\vk_j) - \sum_{l = 1}^m 
\omega (\vk'_l)$ which will in general  become
 zero for some values of $\vk_j$
and $\vk'_l$, so it is no longer 
possible to  divide through by this factor in order to solve the 
equation. 


The equations of higher order are solved analogously. As a result we 
achieve our goal of cancelling the unwanted terms. 

In the course of 
this procedure a nonvanishing vacuum energy  (constant terms in the 
expansion) appears, and the one particle energy $\omega (\vk)$ changes. 
In particular, in the relativistic case this amounts to the 
renormalization of mass:
 in the 
new Hamiltonian $\tilde{H}$, the 
function $\omega(\vk) = \sqrt{\vk^2 + m^2} $ which 
multiplies $a (\vk) \astar(\vk)$ in the Hamiltonian 
$H$  is replaced by 
$\omega_{ren}(\vk) = \sqrt{\vk^2 + M^2} $ for an appropriate value of $M$. 
What we have done may thus be described as {\em mass renormalization}.



Our second example of renormalization is ``charge renormalization''. 
Its origin can be illustrated by a simple example.
We begin with a Hamiltonian 
$$H = - \half \triang + \lambda \delta^3 (x), $$
where $\delta^3(x)$ is the Dirac delta distribution. After 
Fourier transforming the equation $H \hat{\psi}(x) = E \hat{\psi}(x)$ (in 
position space) becomes
\nc{\vp}{ { \vec{p} }}
\beq \vp^2  \psi(\vp) +
 \lambda \int \psi(\vp) d^3 \vp = \vk^2 \psi(\vp) \eeq
where the eigenvalue of the energy is $E = \vec{k}^2$. 
One may rewrite this as 
\beq \psi(\vp) = \delta^3(\vec{p} - \vec{k}) - \frac{\lambda}{\vp^2 - \vk^2
 - i 0}
\int \psi(\vec{p}') d\vec{p}'. \eeq
Putting $t = \int \psi(\vp') d\vp', $ we obtain integrating both sides
over $\vp$
\beq t = 1 - \lambda t \int \frac{d \vec{p}}{\vp^2 - \vk^2 - i 0 }. \eeq
The integral diverges; to regularize it we introduce a cutoff 
$L $ and integrate only over $|p| \le L$.  We find
\beq \int_{|p| \le L}  \frac{d\vec{ p} }{|p|^2 - |\vk|^2 - i 0 } = 
4 \pi \int_{0}^L \frac{|p|^2 d|p| }{|p|^2 - |k|^2 - i 0 }. \eeq
This leads to 
\beq t  = 1 - \lambda t 
\int_{|p| \le L }   \frac{d\vec{ p} }{|p|^2 - k^2 - i 0 }  \eeq
$$ = 1 - \lambda t \Bigl ( 4 \pi  L + k^2  \int \frac{d|p|}{
|p|^2 - k^2 - i 0 } \Bigr ). $$ 
This may be rewritten as 
$$ \lambda t = \frac{\lambda}{1 + \lambda (4 \pi  L +  f(\vk))}, $$
where 
$$f(\vk) = 4 \pi (|\vk|^2) \int_0^\infty \frac{d|\vp|}{|\vp|^2 - |\vk|^2 - i 0}
= 2\pi^2 i |\vk|. $$
We may rewrite this equation 
in terms of a ``renormalized'' parameter $\lambda_{ren}$, by
writing
$$ \lambda t 
= \frac{\lambda_{ren} }{1 + 
\lambda_{ren} f(\vk)} $$
where
$$\lambda_{ren} = \frac{\lambda}{1 + 4 \pi \lambda L } $$ 
or 
\beq \label{signren} \frac{1}{\lambda_{ren} } = 
\frac{1}{\lambda} + 4 \pi L. \eeq
We shall now choose $\lambda_{ren}$ (depending on $L$ and $\lambda$) 
in such a way that $\lambda_{ren}$ may remain finite as $L$ tends to 
infinity. 
It is clear from (\ref{signren}) that this is only possible if $\lambda$ is 
{\em negative} (in other words it describes an attracting 
force, not a repelling one):  then one 
may allow  $\lambda$ to depend on 
$L$ in such a way that $\frac{1}{\lambda}$ tends to $- \infty$ as 
$L$ tends to $\infty$, while the 
value of the renormalized constant $\lambda_{\rm ren}$ remains finite.

Renormalization is discussed further in E. Witten's lectures (October 24
and October 31). 












\begin{thebibliography}{99}
\bibitem{FS} L.D. Faddeev, A.A.  Slavnov, {\em Gauge Fields: An Introduction 
to Quantum Theory}, Addison-Wesley (Frontiers in Physics vol. 83), (second 
edition), 1991. 
\end{thebibliography}


\end{document}

