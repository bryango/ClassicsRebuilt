\section{Correlation Functions}
\label{correlators}

A useful statement of the AdS/CFT correspondence is that the partition
function of string theory on $AdS_5 \times S^5$ should coincide with the
partition function of ${\cal N}=4$ super-Yang-Mills theory ``on the
boundary'' of $AdS_5$ \cite{Gubser:1998bc,Witten:1998qj}.  The basic idea
was explained in section~\ref{field_operator}, but before summarizing the
actual calculations of Green's functions, it seems worthwhile to motivate
the methodology from a somewhat different perspective.

Throughout this section, we approximate the string theory partition
function by $e^{-I_{SUGRA}}$, where $I_{SUGRA}$ is the supergravity action
evaluated on $AdS_5 \times S^5$ (or on small deformations of this space).
This approximation amounts to ignoring all the stringy $\alpha'$
corrections that cure the divergences of supergravity, and also all the
loop corrections, which are controlled essentially by the gravitational
coupling $\kappa \sim g_{st} \alpha'^2$.  On the gauge theory side, as
explained in section~\ref{field_operator}, this approximation amounts to
taking both $N$ and $g_{YM}^2 N$ large, and the basic relation becomes
  \eqn{StringGauge}{
   e^{-I_{SUGRA}} \simeq Z_{\rm string} = Z_{\rm gauge} = e^{-W} \ ,
  }
        where $W$ is the generating functional for connected Green's
        functions in the gauge theory.  At finite temperature,
        $W = \beta F$ where $\beta$ is the inverse temperature and
        $F$ is the free energy of the gauge theory.
 When we apply this relation to a Schwarzschild black hole in
$AdS_5$, which is thought to be reflected in the gauge theory by a thermal
state at the Hawking temperature of the black hole, we arrive at the
relation $I_{SUGRA} \simeq \beta F$.  Calculating the free energy of a black
hole from the Euclidean supergravity action has a long tradition in the
supergravity literature \cite{Gibbons:1978ac}, so the main claim that is
being made here is that the dual gauge theory
provides a description of the state of the black hole which is physically
equivalent to the one in string theory. We will discuss the finite
temperature case further in section \ref{FiniteT}, and devote the rest
of this section to the partition function of the field theory on $\IR^4$.

The main technical idea behind the bulk-boundary correspondence is that the
boundary values of string theory fields (in particular, supergravity
fields) act as sources for gauge-invariant operators in the field theory.
From a D-brane perspective, we think of closed string states in the bulk as
sourcing gauge singlet operators on the brane which originate as composite
operators built from open strings.  We will write the bulk fields
generically as $\phi(\vec{x},z)$ (in the coordinate system
(\ref{poinc})), with value $\phi_0(\vec{x})$ for
$z=\epsilon$.  The true boundary of anti-de Sitter space is $z=0$, and
$\epsilon \neq 0$ serves as a cutoff which will eventually be removed.  In
the supergravity approximation, we think of choosing the values $\phi_0$
arbitrarily and then extremizing the action $I_{SUGRA}[\phi]$ in the
region $z > \epsilon$ subject to these boundary conditions.  In short, we
solve the equations of motion in the bulk subject to Dirichlet boundary
conditions on the boundary, and evaluate the action on the solution.  If
there is more than one solution, then we have more than one saddle point
contributing to the string theory partition function, and we must determine
which is most important.  In this section, multiple saddle points will not
be a problem.  So, we can write
  \def\extremum{\mathop{\rm extremum}}
  \eqn{WvsS}{
   W_{\rm gauge}[\phi_0] = -\log \left\langle
    e^{\int d^4 x \, \phi_0(x) {\cal O}(x)} \right\rangle_{CFT} \simeq
    \extremum_{\phi\big|_{z=\epsilon} = \phi_0} I_{SUGRA}[\phi] \ .
  }
 That is, the generator of connected Green's functions in the gauge theory,
in the large $N, g_{YM}^2 N$ limit, is the on-shell supergravity action.

Note that in \WvsS\ we have not attempted to be prescient about inserting
factors of $\epsilon$.  Instead our strategy will be to use \WvsS\ without
modification to compute two-point functions of ${\cal O}$, and then perform
a wave-function renormalization on either ${\cal O}$ or $\phi$ so that the
final answer is independent of the cutoff.  This approach should be
workable even in a space (with boundary) which is not asymptotically
anti-de Sitter, corresponding to a field theory which does not have a
conformal fixed point in the ultraviolet.

A remark is in order regarding the relation of (\ref{WvsS}) to the old
approach of extracting Green's functions from an absorption cross-section
\cite{Gubser:1997se}.  In absorption calculations one is keeping the whole
D3-brane geometry, not just the near-horizon $AdS_5 \times S^5$ throat.
The usual treatment is to split the space into a near region (the throat)
and a far region.  The incoming wave from asymptotically flat infinity can
be regarded as fixing the value of a supergravity field at the outer
boundary of the near region.  As usual, the supergravity description is
valid at large $N$ and large 't~Hooft coupling.  At small 't~Hooft
coupling, there is a different description of the process: a cluster of
D3-branes sits at some location in flat ten-dimensional space, and the
incoming wave impinges upon it.  In the low-energy limit, the value of the
supergravity field which the D3-branes feel is the same as the value in the
curved space description at the boundary of the near horizon region.
Equation~\WvsS\ is just a mathematical expression of the fact that the
throat geometry should respond identically to the perturbed supergravity
fields as the low-energy theory on the D3-branes.

Following \cite{Gubser:1998bc,Witten:1998qj}, a number of papers---notably
\cite{Aref'eva:1998nn, Muck:1998rr, Freedman:1998tz, Liu:1998bu,
Chalmers:1998xr, Muck:1998iz, Solodukhin:1998ec, Lee:1998bx, Liu:1999ty,
D'Hoker:1999tz, Freedman:1998bj, D'Hoker:1998gd, Chalmers:1998wu,
Muck:1998ug, D'Hoker:1998mz, Minces:1999tp, Arutyunov:1999nw}---have 
undertaken the
program of extracting explicit $n$-point correlation functions of gauge
singlet operators by developing both sides of \WvsS\ in a power series in
$\phi_0$.  Because the right hand side is the extremization of a classical
action, the power series has a graphical representation in terms of
tree-level Feynman graphs for fields in the supergravity.  There is one
difference: in ordinary Feynman graphs one assigns the wavefunctions of
asymptotic states to the external legs of the graph, but in the present
case the external leg factors reflect the boundary values $\phi_0$.  They
are special limits of the usual gravity propagators in the bulk, and are
called bulk-to-boundary propagators.  We will encounter their explicit form
in the next two sections.

\subsection{Two-point Functions}
\label{TwoPoint}

For two-point functions, only the part of the action which is
quadratic in the
relevant field perturbation is needed.  For massive scalar fields in
$AdS_5$, this has the generic form
  \eqn{Squad}{
   S = \eta \int d^5 x \, \sqrt{g} 
    \left[ \tf{1}{2} (\partial\phi)^2 + \tf{1}{2} m^2 \phi^2 \right],
  }
 where $\eta$ is some normalization which in principle follows from the
ten-dimensional origin of the action.
The bulk-to-boundary propagator
is a particular 
solution of the equation of motion, $(\square - m^2) \phi = 0$,
which has special asymptotic properties.  We will start by considering the
momentum space propagator, which is useful for computing the two-point
function and also in situations where the bulk geometry loses conformal
invariance; then, we will discuss the position space propagator, which
has proven more convenient for the study of higher point correlators
in the conformal case.  We will always work in Euclidean
space\footnote{The results may be analytically continued to give the
correlation functions of the field theory on Minkowskian $\IR^4$,
which corresponds to the \Poincare\ coordinates of AdS space.}.  A
coordinate system in the bulk of $AdS_5$ such that
  \eqn{AdSCoords}{
   ds^2 = {R^2 \over z^2} \left( d\vec{x}^2 + dz^2 \right) 
  }
 provides manifest Euclidean symmetry on the directions parametrized
by $\vec{x}$.  To avoid divergences associated with the small $z$
region of integration in \Squad, we will employ an explicit cutoff, $z
\geq \epsilon$.

A complete set of solutions for the linearized equation of motion,
$(\square - m^2) \phi = 0$, is given by $\phi = e^{i \vec{p} \cdot
\vec{x}} Z(pz)$, where the function $Z(u)$ satisfies the radial equation
  \eqn{BesselEq}{
   \left[ u^5 \partial_u {1 \over u^3} \partial_u - u^2 - m^2 R^2
    \right] Z(u) = 0 \ .
  }
 There are two independent solutions to \BesselEq, namely $Z(u) = u^2
I_{\Delta-2}(u)$ and $Z(u) = u^2 K_{\Delta-2}(u)$, where $I_\nu$ and
   $K_\nu$ are Bessel functions and
  \eqn{DimMassRel}{
   \Delta = 2 + \sqrt{4 + m^2 R^2} \ .
  }
 The second solution is selected by the requirement of regularity in the
interior: $I_{\Delta-2}(u)$ increases exponentially as $u\to\infty$ and
does not lead to a finite action configuration\footnote{Note that this
solution, when continued to Lorentzian AdS space, generally involves
the non-normalizable mode of the field, with $\lambda_-$ in 
(\ref{hyper}).}.  Imposing the boundary
condition $\phi(\vec{x},z) = \phi_0(\vec{x}) = e^{i \vec{p} \cdot \vec{x}}$
at $z = \epsilon$, we find the bulk-to-boundary propagator 
  \eqn{BoundaryProp}{
   \phi(\vec{x},z) = K_{\vec{p}}(\vec{x},z)
     = {(pz)^2 K_{\Delta-2}(pz) \over 
        (p\epsilon)^2 K_{\Delta-2}(p\epsilon)} e^{i \vec{p} \cdot \vec{x}} \ .
  }
 To compute a two-point function of the operator ${\cal O}$ for which
$\phi_0$ is a source, we write
  \eqn{TwoPtFct}{\eqalign{
   \langle {\cal O}(\vec{p}) {\cal O}(\vec{q}) \rangle
    &= \left. {\partial^2 W\left[ \phi_0 = \lambda_1 e^{i \vec{p} \cdot x} + 
       \lambda_2 e^{i \vec{q} \cdot x} \right] \over
       \partial \lambda_1 \partial \lambda_2} \right|_{\lambda_1 = 
        \lambda_2 = 0}  \cr
    &= \hbox{(leading analytic terms in $(\epsilon p)^2$)}  \cr
    &\qquad {} - \eta \epsilon^{2\Delta-8} (2\Delta-4)
     {\Gamma(3-\Delta) \over \Gamma(\Delta-1)}
     \delta^4(\vec{p}+\vec{q}) \left( {\vec{p} \over 2} 
      \right)^{2\Delta - 4}  \cr
    &\qquad {} + \hbox{(higher order terms in $(\epsilon p)^2$)},  \cr
   \langle {\cal O}(\vec{x}) {\cal O}(\vec{y}) \rangle &=
    \eta \epsilon^{2\Delta-8} {2\Delta-4 \over \Delta} 
     {\Gamma(\Delta+1) \over \pi^2 \Gamma(\Delta-2)}
     {1 \over |\vec{x}-\vec{y}|^{2\Delta}} \ .
  }}
 Several explanatory remarks are in order: 
  \begin{itemize}
   \item
 To establish the second equality in \TwoPtFct\ we have used 
\BoundaryProp , substituted in  \Squad,  performed
the integral  and expanded in
$\epsilon$.  The leading analytic terms give rise to contact terms in
position space, and the higher order terms are unimportant in the
limit where we remove the cutoff.  Only the leading nonanalytic term
is essential.  We have given the expression for generic real values of
$\Delta$.  Expanding around integer $\Delta \geq 2$ one obtains finite
expressions involving $\log \epsilon p$.
   \item{}
 The Fourier transforms used to obtain the last line are singular, but they
can be defined for generic complex $\Delta$ by analytic continuation and
for positive integer $\Delta$ by expanding around a pole and dropping
divergent terms, in the spirit of differential regularization
\cite{Freedman:1992tk}.  The result is a pure power law dependence on the
separation $|\vec{x}-\vec{y}|$, as required by conformal invariance.
   \item{}
 We have assumed a coupling $\int d^4x \, \phi(\vec{x},z=\epsilon) {\cal
O}(\vec{x})$ 
to compute the Green's functions.  The explicit powers of the cutoff in
the final position space answer can be eliminated by absorbing a factor of
$\epsilon^{\Delta-4}$ into the definition of ${\cal O}$.  From here on we
will take that convention, which amounts to inserting a factor of
$\epsilon^{4-\Delta}$ on the right hand side of \BoundaryProp.  
In fact, precise matchings between the normalizations in field theory
and in string theory
%unambiguous normalizations 
for all the chiral primary operators have not
been worked out.  In part this is due to the difficulty of determining the
coupling of bulk fields to field theory operators (or in stringy terms, the
coupling of closed string states to composite open string operators on the
brane).  See \cite{Gubser:1997yh} for an early approach to this problem.
For the dilaton, the graviton, and their superpartners (including gauge
fields in $AdS_5$), the couplings can be worked out explicitly.  In some of
these cases all normalizations have been worked out unambiguously and
checked against field theory predictions (see for example
\cite{Gubser:1998bc,Freedman:1998tz,D'Hoker:1999tz}).
   \item{}
 The mass-dimension relation (\ref{DimMassRel})
 holds even for string states that are not
included in the Kaluza-Klein supergravity reduction: the mass and the
dimension are just different expressions of the second
Casimir of $SO(4,2)$.  For
instance, excited string states, with $m \sim 1/\sqrt{\alpha'}$, are
expected to correspond to operators with dimension $\Delta \sim (g_{YM}^2
N)^{1/4}$.  The remarkable fact is that all the string theory modes with $m
\sim 1/R$ (which is to say, all closed string states which arise from
 massless ten dimensional fields) fall in
short multiplets of the supergroup $SU(2,2|4)$.  All other states have
a much
larger mass.  The operators in short multiplets have algebraically
protected dimensions.  The obvious conclusion is that all operators whose
dimensions are not algebraically protected have large dimension
in the strong 't~Hooft coupling, large $N$ limit to which supergravity
applies.  This is no longer true for theories of reduced supersymmetry: the
supergroup gets smaller, but the Kaluza-Klein states are roughly as
numerous as before, and some of them escape the short multiplets and
live in long multiplets of the smaller supergroups.  They
still have a mass on the order of $1/R$, and typically correspond to dimensions
which are finite (in the large $g_{YM}^2 N$ limit) but irrational.
  \end{itemize}

Correlation functions of non-scalar operators have been widely studied
following \cite{Witten:1998qj}; the literature includes
\cite{Henningson:1998cd, Ghezelbash:1998pf, Arutyunov:1998ve,
Arutyunov:1998xt, l'Yi:1998yt, Volovich:1998tj, l'Yi:1998pi, l'Yi:1998eu,
Koshelev:1998tu, Rashkov:1999ji, Polishchuk:1999nh}.  For ${\cal N}=4$
super-Yang-Mills theory, all correlation functions of fields in chiral
multiplets should follow by application of supersymmetries once those of
the chiral primary fields are known, so in this case it should be enough to
study the scalars.  It is worthwhile to note however that the
mass-dimension formula changes for particles with spin.  In fact the
definition of mass has some convention-dependence.  Conventions seem 
fairly uniform in the literature, and a table of mass-dimension
relations in $AdS_{d+1}$ with unit radius was made in
\cite{Freedman:1999gp} from the various sources cited above (see also
\cite{Ferrara:1998ej}):
  \begin{itemize}
\item scalars:\quad $\Delta_{\pm} = {1 \over 2} (d \pm \sqrt{d^2
+4m^2})$,
\item spinors:\quad  $\Delta = {1 \over 2} (d + 2|m|)$,
\item vectors:\quad
$ \Delta_{\pm} = {1 \over 2} (d \pm \sqrt{(d-2)^2 + 4m^2})$,
\item $p$-forms:\quad
$ \Delta = {1 \over 2} (d \pm \sqrt{(d-2p)^2 + 4m^2})$,
\item first-order $(d/2)$-forms ($d$ even):\quad
$\Delta={1\over 2}(d + 2|m|)$,
\item spin-3/2:\quad
$\Delta = {1 \over 2} (d + 2|m|)$,
\item massless spin-2:\quad
$\Delta = d$.
  \end{itemize}
 In the case of fields with second order lagrangians, we have not attempted
to pick which of $\Delta_\pm$ is the physical dimension.  Usually the
choice $\Delta=\Delta_+$ is clear from the unitarity bound, but in some
cases (notably $m^2 = 15/4$ in $AdS_5$) there is a genuine ambiguity.  In
practice this ambiguity is usually resolved by appealing to some special
algebraic property of the relevant fields, such as transformation under
supersymmetry or a global bosonic symmetry.  
See section~\ref{pfinads} for further discussion.  The scalar case above is 
precisely equation (\ref{branch}) in that section. 

 For brevity we will omit a further discussion of higher spins, and instead
refer the reader to the (extensive) literature. 



\subsection{Three-point Functions}
\label{ThreePoint}

Working with bulk-to-boundary propagators in the momentum representation is
convenient for two-point functions, but for higher point functions position
space is preferred because the full conformal invariance is more obvious.
(However, for non-conformal examples of the bulk-boundary correspondence,
the momentum representation seems uniformly more convenient).  The boundary
behavior of position space bulk-to-boundary propagators is specified in a
slightly more subtle way: following \cite{Freedman:1998tz} we require
  \eqn{BBPlimit}{
   K_\Delta(\vec{x},z;\vec{y}) \to z^{4-\Delta} \delta^4(\vec{x}-\vec{y})
    \quad\hbox{as}\quad z \to 0 \ .
  }
 Here $\vec{y}$ is the point on the boundary where we insert the
operator, and $(\vec{x},z)$ is a point
in the bulk.  The unique regular $K_\Delta$ solving the equation of
motion and satisfying \BBPlimit\ is 
  \eqn{BBPvalue}{
   K_\Delta(\vec{x},z;\vec{y}) = {\Gamma(\Delta) \over \pi^2 
    \Gamma(\Delta-2)} \left( {z \over z^2 + (\vec{x}-\vec{y})^2} 
     \right)^{\Delta} \ .
  }
 At a fixed cutoff, $z=\epsilon$, the bulk-to-boundary propagator
$K_\Delta(\vec{x},\epsilon;\vec{y})$ is a
continuous function which approximates $\epsilon^{4-\Delta}
\delta^4(\vec{x}-\vec{y})$ better and
better as $\epsilon\to 0$.  Thus at any finite $\epsilon$, the Fourier
transform of \BBPvalue\ only approximately coincides with \BoundaryProp\
(modified by the factor of $\epsilon^{4-\Delta}$ as explained after
\TwoPtFct).  This apparently innocuous subtlety turned out to be important
for two-point functions, as discovered in \cite{Freedman:1998tz}.  A correct
prescription is to specify boundary conditions at finite $z = \epsilon$,
cut off all bulk integrals at that boundary, and only afterwards take
$\epsilon\to 0$.  That is what we have done in \TwoPtFct.  Calculating
two-point functions directly using the position-space propagators
\BBPlimit, but cutting the bulk integrals off again at $\epsilon$, and
finally taking the same $\epsilon\to 0$ answer, one arrives at a different
answer.  This is not surprising since the $z = \epsilon$ boundary
conditions were not used consistently.  
The authors of \cite{Freedman:1998tz} checked
that using the cutoff consistently (i.e. with the momentum space propagators)
gave two-point functions $\langle {\cal O}(\vec{x}_1) {\cal O}(\vec{x}_2)
\rangle$ a normalization such that Ward identities involving the
three-point function $\langle {\cal O}(\vec{x}_1) {\cal O}(\vec{x}_2)
J_\mu(\vec{x}_3) \rangle$, where $J_\mu$ is a conserved current, were
obeyed.  Two-point functions are uniquely difficult because of the poor
convergence properties of the integrals over $z$.  The integrals involved
in three-point functions are sufficiently benign that one can ignore the
issue of how to impose the cutoff.

If one has a Euclidean bulk action for three scalar fields $\phi_1$,
$\phi_2$, and $\phi_3$, of the form
  \eqn{IfAction}{
   S = \int d^5 x \, \sqrt{g} \left[
    \sum_i \tf{1}{2} (\partial\phi_i)^2 + \tf{1}{2} m_i^2 \phi_i^2 + 
    \lambda \phi_1 \phi_2 \phi_3 \right] \ ,
  }
 and if the $\phi_i$ couple to operators in the field theory by interaction
terms $\int d^4 x \, \phi_i {\cal O}_i$, then the calculation of $\langle
{\cal O}_1 {\cal O}_2 {\cal O}_3 \rangle$ reduces, via \WvsS, to the
evaluation of the graph shown in figure~\ref{figAssg}.  
  \begin{figure}
      \vskip0cm
   \centerline{\psfig{figure=figAssg.eps,width=1.8in}}
   \vskip0cm
 \caption{The Feynman graph for the three-point function as computed in
supergravity.  The legs correspond to 
factors of $K_{\Delta_i}$, and the cubic vertex to
a factor of $\lambda$.  The position of the vertex is integrated
over $AdS_5$.}\label{figAssg}
  \end{figure}
 That is,
  \eqn{ThreeGraph}{\eqalign{
   &\langle {\cal O}_1(\vec{x}_1) {\cal O}_2(\vec{x}_2)
     {\cal O}_3(\vec{x}_3) \rangle = 
     -\lambda \int d^5 x \, \sqrt{g} 
     K_{\Delta_1}(x;\vec{x}_1) K_{\Delta_2}(x;\vec{x}_2)
     K_{\Delta_3}(x;\vec{x}_3)  \cr
   &\qquad\qquad 
    = {\lambda a_1 \over |\vec{x}_1 - \vec{x}_2|^{\Delta_1+\Delta_2-\Delta_3}
     |\vec{x}_1 - \vec{x}_3|^{\Delta_1+\Delta_3-\Delta_2}
     |\vec{x}_2 - \vec{x}_3|^{\Delta_2+\Delta_3-\Delta_1}} \ ,
  }}
 for some constant $a_1$.  The dependence on the $\vec{x}_i$ is dictated by
the conformal invariance, but the only way to compute $a_1$ is by
performing the integral over $x$.  The result \cite{Freedman:1998tz} is
  \eqn{aOneValue}{\eqalign{
   a_1 &= -{\Gamma\left[ {1 \over 2} (\Delta_1 + \Delta_2 -\Delta_3) \right]
    \Gamma\left[ {1 \over 2} (\Delta_1 + \Delta_3 -\Delta_2) \right]
    \Gamma\left[ {1 \over 2} (\Delta_2 + \Delta_3 -\Delta_1) \right] \over
    2\pi^4 \Gamma(\Delta_1 - 2) \Gamma(\Delta_2 - 2) \Gamma(\Delta_3 -
2)} \cdot  \cr
    &\qquad 
     \Gamma\left[ \tf{1}{2} (\Delta_1 + \Delta_2 + \Delta_3) - 2 \right] \ .
  }}
 In principle one could also have couplings of the form $\phi_1
\partial\phi_2 \partial\phi_3$.  This leads only to a modification of the
constant $a_1$.

The main technical difficulty with three-point functions is that one must
figure out the cubic couplings of supergravity fields.  Because of the
difficulties in writing down a covariant action for type IIB supergravity
in ten dimensions (see however
\cite{Dall'Agata:1997ju,Dall'Agata:1998va,Arutyunov:1998hf}), it is most
straightforward to read off these ``cubic couplings'' from quadratic terms
in the equations of motion.  In flat ten-dimensional space these terms can
be read off directly from the original type~IIB supergravity papers
\cite{Schwarz:1983qr,Howe:1984sr}.  For $AdS_5 \times S^5$, one must
instead expand in fluctuations around the background metric and five-form
field strength.  The old literature \cite{Kim:1985ez} only dealt with the
linearized equations of motion; for 3-point functions it is necessary to go
to one higher order of perturbation theory.  This was done for a restricted
set of fields in \cite{Lee:1998bx}.  The fields considered were those dual
to operators of the form $\tr \phi^{(J_1} \phi^{J_2} \ldots \phi^{J_\ell)}$
in field theory, where the parentheses indicate a symmetrized traceless
product.  These operators are the chiral primaries of the gauge theory: all
other single trace operators of protected dimension descend from these by
commuting with supersymmetry generators.  Only the metric and the five-form
are involved in the dual supergravity fields, and we are interested only in
modes which are scalars in $AdS_5$.  The result of \cite{Lee:1998bx} is
that the equations of motion for the scalar modes $\tilde{s}_I$ dual to
  \eqn{lmrsOp}{
   {\cal O}^I = {\cal C}^I_{J_1 \ldots J_\ell} 
     \tr \phi^{(J_1} \ldots \phi^{J_\ell)}
  }
 follow from an action of the form
  \eqn{lmrsAction}{\eqalign{
   S = {4 N^2 \over (2\pi)^5} \int d^5 x \, \sqrt{g} \Bigg\{ &
    \sum_I {A_I (w^I)^2 \over 2} \left[ -(\nabla \tilde{s}_I)^2 - l(l-4) 
      \tilde{s}_I^2 \right]  \cr
    &\qquad {} + 
     \sum_{I_1,I_2,I_3} {{\cal G}_{I_1 I_2 I_3} w^{I_1} w^{I_2} w^{I_3}
       \over 3} \tilde{s}_{I_1} \tilde{s}_{I_2} \tilde{s}_{I_3} \Bigg\} \ .
  }}
 Derivative couplings of the form $\tilde{s} \partial\tilde{s}
\partial\tilde{s}$ are expected {\it a priori} to enter into \lmrsAction,
but an appropriate field redefinition eliminates them.  The notation in
\lmrsOp\ and \lmrsAction\ requires some explanation.  $I$ is an index which
runs over the weight vectors of all possible representations constructed as
symmetric traceless products of the ${\bf 6}$ of $SU(4)_R$.  These are the
representations whose Young diagrams are
$\oalign{\idget\endrow\idget\endyoung}$,
$\oalign{\idget\idget\endrow\idget\idget\endyoung}$,
$\oalign{\idget\idget\idget\endrow\idget\idget\idget\endyoung}$, $\cdots$.
${\cal C}^I_{J_1 \ldots J_\ell}$ is a basis transformation matrix, chosen
so that ${\cal C}^I_{J_1 \ldots J_\ell} {\cal C}^J_{J_1 \ldots J_\ell} =
\delta^{IJ}$.  As commented in the previous section, there is generally a
normalization ambiguity on how supergravity fields couple to operators in
the gauge theory.  We have taken the coupling to be $\int d^4x \,
\tilde{s}_I {\cal O}^I$, and the normalization ambiguity is represented by
the ``leg factors'' $w^I$.  It is the combination $s^I = w^I \tilde{s}^I$
rather than $\tilde{s}^I$ itself which has a definite relation to
supergravity fields.  We refer the reader to \cite{Lee:1998bx} for explicit
expressions for $A_I$ and the symmetric tensor ${\cal G}_{I_1 I_2 I_3}$.
To get rid of factors of $w^I$, we introduce operators ${\cal O}^I =
\tilde{w}^I {\cal O}^I$.  One can choose $\tilde{w}^I$ so that a two-point
function computation along the lines of section~\ref{TwoPoint} leads to
  \eqn{TwoPointO}{
   \langle {\cal O}^{I_1}(\vec{x}) {\cal O}^{I_2}(0) \rangle = 
    {\delta^{I_1 I_2} \over x^{2\Delta_1}} \ .
  }
 With this choice, the three-point function, as calculated using
\ThreeGraph, is 
  \eqn{ThreePointO}{
   \langle {\cal O}^{I_1}(\vec{x_1}) {\cal O}^{I_2}(\vec{x_2})
    {\cal O}^{I_3}(\vec{x_3}) \rangle = 
    {1 \over N} {\sqrt{\Delta_1 \Delta_2 \Delta_3} 
     \langle {\cal C}^{I_1} {\cal C}^{I_2} {\cal C}^{I_3} \rangle \over
     |\vec{x}_1 - \vec{x}_2|^{\Delta_1+\Delta_2-\Delta_3}
     |\vec{x}_1 - \vec{x}_3|^{\Delta_1+\Delta_3-\Delta_2}
     |\vec{x}_2 - \vec{x}_3|^{\Delta_2+\Delta_3-\Delta_1}} \ ,
  }
 where we have defined
  \eqn{CAngleDef}{
   \langle {\cal C}^{I_1} {\cal C}^{I_2} {\cal C}^{I_3} \rangle = 
    {\cal C}^{I_1}_{J_1 \cdots J_i K_1 \cdots K_j}
    {\cal C}^{I_2}_{J_1 \cdots J_i L_1 \cdots L_k}
    {\cal C}^{I_3}_{K_1 \cdots K_j L_1 \cdots L_k} \ .
  }
 Remarkably, \ThreePointO\ is the same result one obtains from free field
theory by Wick contracting all the $\phi^J$ fields in the three operators.
This suggests that there is a non-renormalization theorem for this
correlation function, but such a theorem has not yet been proven (see however
comments at the end of section~\ref{anomalies}). It
is worth emphasizing that the normalization ambiguity in the bulk-boundary
coupling is circumvented essentially by considering invariant ratios of
three-point functions and two-point functions, into which the ``leg
factors'' $w^I$ do not enter.  This is the same strategy as was pursued in
comparing matrix models of quantum gravity to Liouville
theory.
 


\subsection{Four-point Functions}
\label{FourPointFunctions}

The calculation of four-point functions is difficult because there are
several graphs which contribute, and some of them inevitably involve
bulk-to-bulk propagators of fields with spin.  The computation of
four-point functions of the operators ${\cal O}_\phi$ and ${\cal O}_C$ dual
to the dilaton and the axion was completed in \cite{D'Hoker:1999pj}.  See
also \cite{Muck:1998rr,Liu:1999ty,Freedman:1998bj,D'Hoker:1998gd,
D'Hoker:1999jc,Liu:1998th,D'Hoker:1998mz,Chalmers:1998wu,Chalmers:1999gc,
Gonzalez-Rey:1998tk}
for earlier contributions.  One of the main technical results, further
developed in \cite{D'Hoker:1999ni}, is that diagrams involving an internal
propagator can be reduced by integration over one of the bulk vertices to a
sum of quartic graphs expressible in terms of the functions
  \eqn{DFunction}{\eqalign{
   D_{\Delta_1\Delta_2\Delta_3\Delta_4}(\vec{x}_1,\vec{x}_2,
    \vec{x}_3,\vec{x}_4) &= 
    \int d^5 x \, \sqrt{g} \prod_{i=1}^4 
     \tilde{K}_{\Delta_i}(\vec{x},z;\vec{x}_i),  \cr
   \tilde{K}_\Delta(\vec{x},z;\vec{y}) &= 
    \left( z \over z^2 + (\vec{x} - \vec{y})^2 \right)^\Delta \ .
  }}
 The integration is over the bulk point $(\vec{x},z)$.  There are two
independent conformally invariant combinations of the $\vec{x}_i$:
  \eqn{sAndt}{
   s = {1 \over 2} {\vec{x}_{13}^2 \vec{x}_{24}^2 \over 
    \vec{x}_{12}^2 \vec{x}_{34}^2 + \vec{x}_{14}^2 \vec{x}_{23}^2} \qquad
   t = {\vec{x}_{12}^2 \vec{x}_{34}^2 - \vec{x}_{14}^2 \vec{x}_{23}^2 \over 
        \vec{x}_{12}^2 \vec{x}_{34}^2 + \vec{x}_{14}^2 \vec{x}_{23}^2} \ .
  }
 One can write the connected four-point function as
  \eqn{FourPoint}{\eqalign{
   &\langle {\cal O}_\phi(\vec{x}_1) {\cal O}_C(\vec{x}_2) 
    {\cal O}_\phi(\vec{x}_3) {\cal O}_C(\vec{x}_4) \rangle
    = \left( 6 \over \pi^2 \right)^4 \Bigg[ 
     16 \vec{x}_{24}^2 \left( {1 \over 2s} - 1 \right) D_{4455} + 
     {64 \over 9} {\vec{x}_{24}^2 \over \vec{x}_{13}^2} {1 \over s} D_{3355}  \cr
    &\quad{} + 
     {16 \over 3} {\vec{x}_{24}^2 \over \vec{x}_{13}^2} {1 \over s} D_{2255} -
     14 D_{4444} - {46 \over 9 \vec{x}_{13}^2} D_{3344} - 
     {40 \over 9 \vec{x}_{13}^2} D_{2244} - 
     {8 \over 3 \vec{x}_{13}^6} D_{1144} + 
     64 \vec{x}_{24}^2 D_{4455} \Bigg] \ .
  }}

An interesting limit of \FourPoint\ is to take two pairs of points close
together.  Following \cite{D'Hoker:1999pj}, let us take the pairs
$(\vec{x}_1,\vec{x}_3)$ and $(\vec{x}_2,\vec{x}_4)$ close together while
holding $\vec{x}_1$ and $\vec{x}_2$ a fixed distance apart.  Then the
existence of an OPE expansion implies that
  \eqn{DoubleOPE}{
   \langle {\cal O}_{\Delta_1}(\vec{x}_1) {\cal O}_{\Delta_2}(\vec{x}_2) 
    {\cal O}_{\Delta_3}(\vec{x}_3) {\cal O}_{\Delta_4}(\vec{x}_4) \rangle = 
    \sum_{n,m} {\alpha_n \langle {\cal O}_n(\vec{x}_1) 
      {\cal O}_m(\vec{x}_2) \rangle
     \beta_m \over \vec{x}_{13}^{\Delta_1+\Delta_3-\Delta_m} 
      \vec{x}_{24}^{\Delta_2+\Delta_4-\Delta_n}} ,
  }
 at least as an asymptotic series, and hopefully even with a finite radius
of convergence for $\vec{x}_{13}$ and $\vec{x}_{24}$.  The operators ${\cal
O}_n$ are the ones that appear in the OPE of ${\cal O}_1$ with ${\cal
O}_3$, and the operators ${\cal O}_m$ are the ones that appear in the OPE
of ${\cal O}_2$ with ${\cal O}_4$.  ${\cal O}_\phi$ and ${\cal O}_C$ are
descendants of chiral primaries, and so have protected dimensions.  The
product of descendants of chiral fields is not itself necessarily the descendent
of a chiral field: an appropriately normal ordered product $:{\cal O}_\phi {\cal
O}_\phi:$ is expected to have an unprotected dimension of the form $8 +
O(1/N^2)$.  This is the natural result from the field theory point of view
because there are $O(N^2)$ degrees of freedom contributing to each factor,
and the commutation relations between them are non-trivial only a fraction
$1/N^2$ of the time.  From the supergravity point of view, a composite
operator like $:{\cal O}_\phi {\cal O}_\phi:$ corresponds to a two-particle
bulk state, and the $O(1/N^2) = O(\kappa^2/R^8)$ correction to the mass is
interpreted as the correction to the mass of the two-particle state from
gravitational binding energy.  Roughly one is thinking of graviton exchange
between the legs of figure~\ref{degenerate} that are nearly coincident.
  \begin{figure}
   \vskip0cm
   \centerline{\psfig{figure=degenerate.eps,width=1.8in}}
   \vskip0cm
 \caption{A nearly degenerate quartic graph contributing to the four-point
function in the limit $|\vec{x}_{13}|,|\vec{x}_{24}| \ll
|\vec{x}_{12}|$.}\label{degenerate}
  \end{figure}

If \DoubleOPE\ is expanded in inverse powers of $N$, then the $O(1/N^2)$
correction to $\Delta_n$ and $\Delta_m$ shows up to leading order as a term
proportional to a logarithm of some combination of the separations
$\vec{x}_{ij}$.  Logarithms also appear in the expansion of \FourPoint\ in
the $|\vec{x}_{13}|, |\vec{x}_{24}| \ll |\vec{x}_{12}|$ limit in which
\DoubleOPE\ applies: the leading log in this limit is ${1 \over
(\vec{x}_{12})^{16}} \log\left( {\vec{x}_{13} \vec{x}_{24} \over
\vec{x}_{12}^2} \right)$.  This is the correct form to be interpreted in
terms of the propagation of a two-particle state dual to an operator whose
dimension is slightly different from $8$.

