%From: Dennis Gaitsgory <gaitsgde@math.ias.edu>
%Date: Thu, 3 Oct 1996 16:05:12 -0400


\input amstex
\documentstyle{amsppt}
\magnification=1200
\pagewidth{6.5 true in}
\pageheight{8.9 true in}
\loadeufm
\loadeusm

\catcode`\@=11
\def\logo@{}
\catcode`\@=13

\NoRunningHeads

\font\boldtitlefont=cmb10 scaled\magstep2

\def\dspace{\lineskip=2pt\baselineskip=18pt\lineskiplimit=0pt}

\def\oplusop{\operatornamewithlimits{\oplus}\limits}
\def\w{{\mathchoice{\,{\scriptstyle\wedge}\,}
  {{\scriptstyle\wedge}}
  {{\scriptscriptstyle\wedge}}{{\scriptscriptstyle\wedge}}}}
\def\Le{{\mathchoice{\,{\scriptstyle\le}\,}
  {\,{\scriptstyle\le}\,}
  {\,{\scriptscriptstyle\le}\,}{\,{\scriptscriptstyle\le}\,}}}
\def\Ge{{\mathchoice{\,{\scriptstyle\ge}\,}
  {\,{\scriptstyle\ge}\,}
  {\,{\scriptscriptstyle\ge}\,}{\,{\scriptscriptstyle\ge}\,}}}
\def\vrulesub#1{\hbox{\,\vrule height7pt depth5pt\,}_{#1}}
\def\rightsubsetarrow#1{\subset\kern-6.50pt\lower2.85pt
     \hbox to #1pt{\rightarrowfill}}
\def\mapright#1{\smash{\mathop{\,\longrightarrow\,}\limits^{#1}}}

\def\Lbar{\overline{L}}
\def\Mbar{\overline{M}}
\def\Qbar{\overline{Q}}

\def\Ttil{\widetilde{T}}

\def\Stab{\text{\rm Stab}}
\def\SO{\text{\rm SO}} \def\Spin{\text{\rm Spin}}
\def\SU{\text{\rm SU}} \def\Lie{\text{\rm Lie}}
\def\SL{\text{\rm SL}} \def\Sym{\text{\rm Sym}}

\def\dbC{{\Bbb C}} 
\def\dbR{{\Bbb R}}
\def\dbZ{{\Bbb Z}} 
\def\sbh{\subheading}

\def\scr#1{{\fam\eusmfam\relax#1}}

\def\scrA{{\scr A}}   \def\scrB{{\scr B}}
\def\scrC{{\scr C}}   \def\scrD{{\scr D}}
\def\scrE{{\scr E}}   \def\scrF{{\scr F}}
\def\scrG{{\scr G}}   \def\scrH{{\scr H}}
\def\scrI{{\scr I}}   \def\scrJ{{\scr J}}
\def\scrK{{\scr K}}   \def\scrL{{\scr L}}
\def\scrM{{\scr M}}   \def\scrN{{\scr N}}
\def\scrO{{\scr O}}   \def\scrP{{\scr P}}
\def\scrQ{{\scr Q}}   \def\scrR{{\scr R}}
\def\scrS{{\scr S}}   \def\scrT{{\scr T}}
\def\scrU{{\scr U}}   \def\scrV{{\scr V}}
\def\scrW{{\scr W}}   \def\scrX{{\scr X}}
\def\scrY{{\scr Y}}   \def\scrZ{{\scr Z}}

\def\gr#1{{\fam\eufmfam\relax#1}}

%Euler Fraktur letters (German)
\def\grg{{\frak g}}
\def\grp{{\frak p}}


\topmatter
\title\nofrills
{\boldtitlefont Lectures on supersymmetry}
\endtitle
\author
{\rm Joseph Bernstein}\\ \\
{\rm September 17, 1996}
\endauthor
\endtopmatter



%\NoBlackBoxes
\document
\dspace

\bigskip\bigskip
\head
{\bf Lecture 1. Why sypersymmetry}
\endhead

\bigskip
\subhead
0
\endsubhead
The aim of these lectures is to describe what physicists
have in mind when they talk about supersymmetry (SUSY).

I should say that I am a mathematician, not a physicist, and
these lectures are for mathematicians.

First I thought that these will be  more or less
straightforward course of lectures on supermanifolds.
But then I (and David Kazhdan) realized that the situation
is more interesting.

Physicists have constructed many models with supersymmetry;
for some of them they found nice models formulated in terms
of supermanifolds (super space in physical terminology).
But for many interesting models with supersymmetry their
``supermanifold'' realization is unknown --- and this may be an
interesting problem for mathematicians.

\sbh{1. What are sypersymmetries}


First, what are symmetries?

Physicists have three types of pictures for their theories.

\roster
\item
Classical picture -- fields on a space-time $+$ Lagrangian;

\item
Quantum field theory (QFT)

\item
Scattering theory (theory of $S$-matrix).
\endroster

Usually there are many symmetries of the classical picture, but
many of them give trivial symmetries of the scattering
theory (of the $S$-matrix).
  It turns out that if we want to describe the symmetries of
the $S$-matrix then the answer is very restrictive.
In my lectures I will mostly deal only with classical picture.
But in the first lecture I will describe an important result
from the scattering theory which explains why physicists
are so interested in supersymmetry.
\sbh{2. The group $P$ and its representations}


\sbh{2.1}

Let $V$ be a $4$-dimensional real vector space $(V\simeq
\dbR^4)$ with a quadratic form of signature $\left(
\smallmatrix + &&&\\ &- &&&\\ &&- 
  &&\\ &&&-\endsmallmatrix\right)$.

Let $G$ denote the universal cover of the group $\SO(V)$:
we have
$$
\matrix
1 &\to &\dbZ_2 &\to &G &\to &\SO(V) &\to &1\\
&&&&\wr\wr &&&&\\
&&&&\Spin(V) &&&&
\endmatrix
$$
As a real group, $G$ is isomoprhic to $\SL(2,\dbC)$.
We will list now all finite-dimensional irreducible complex
representations of $G$.

Let $L$ denote a standard $2$-dimensional complex
representation of $G$ (coming from the isomorphism $G\simeq
\SL(2,\dbC)$).
The action of $G$ on $L$ preserves a volume form $\omega$
and we have $L\simeq L^*$, as $G$-representations ($^{\,*}$
denotes the linear dual).

\definition{Claim}
Any irreducible representation of $G$ has a form:

$M_i\otimes\Mbar_j$, where $M_k$ denotes the
$k+1$-dimensional representation $\Sym^k(L)$ and where
$^{^{\overline{ \ \ }}}$ is the complex conjugation.\qquad
$\blacksquare$
\enddefinition

\example{Example}
The space $V$ as a real representation of $G$ identifies
with the space of Hermitian forms on $L$.

We have 
$$
\tau\colon\,V_\dbC\simeq M_1\otimes\Mbar_1=L\otimes
\Lbar
$$
\endexample

\subsubhead
{\bf 2.2}
\endsubsubhead
Let $P$ denote the Poincar\'e group, i.e. the semidirect
product
$$
P=G\ltimes V.
$$

We will be interested in unitary representations of $P$.
Note that the action of the central element $z \in P$
$\left(z=\left(\smallmatrix -1 &\,\,\,\,0\\ \,\,\,\,0
&-1\endsmallmatrix\right)\in \SL(2,\dbC)\simeq G\right)$
defines a $\dbZ_2$-grading on the space of each such
representation:
$$
H=H_+\oplus H_-
$$

Let $H$ be an irreducible unitary representation of $P$.
Then $H\vrulesub{V}$ has a form $\int\limits_{p\in
V^*}H_pd\mu$, where $d\mu$ is a $G$-invariant measure on
$V^*$ concentrated on one $G$-orbit $\scrO\subseteq V^*$ (
in our case each orbit admits a $G$-invariant measure).
The structure of a $P$-representation on $\int\limits_{p\in
V^*}H_pd\mu$ is determined by the action of
$\Stab(p_0)\subseteq G$ ($p_0$ is an arbitrary point of
$\scrO$) on $H_{p_0}$, which must be an irreducible unitary representaion
of this group. 

We say that an irreducible $P$-representation $H$ is
positive if the corresponding $G$-orbit $\scrO\subseteq V^*$
is one of the following three types:

\roster
\item"a)"
$\scrO_m^+$, $m\in\dbR^{+*}\colon\,\{p\in V,\,<p,p>=
   m^2,\,p_0>0\}$;

\item"b)"
$\scrO_0^+\colon\,\{p\in V,\,<p,p>=0,\,p_0>0\}$;

\item"c)"
$\{0\}$.
\endroster

Here we have used the identification $V\simeq V^*$ and an
orientation of the time-coordinate $x_0$ in $V\simeq
\dbR^4$.

We say that a unitary representation $H$ of $P$ is ``good''
if it satisfies the following three properties:

\roster
\item"1)"
It is a sum of finitely many irreducibles $H=\oplusop_{i}H$;

\item"2)"
Each of the $H_i$'s is positive in the above sense;

\item"3)"
For each of the $H_i$'s the representations of stabilizers
$H_{p_i}$ are finite dimensional.
\endroster

\example{Example}
Let $H$ be an irreducible $P$-representation satisfying 2)
and 3).

\roster
\item"a)"
If $\scrO=\scrO_m^+$, choose $p=(m,0,0,0)$;
$\Stab_{p}=\SU(2)$.
Irreducible unitary representations of $\SU(2)$ are
enumerated by positive half-integers and the corresponding
$P$-representations will be denoted as $H_{m,s}$
$s\in\frac12\,\dbZ^+$.

\item"b)"
If $\scrO=\scrO_0^+$, choose $p=(1,1,0,0)$;
$\Stab_{p}=U(1)\ltimes N$, where $N$ is a nilpotent group.
Irreducible $\Stab_{p}$-representations are 
enumerated by half-integers and the corresponding
$P$-representations will be denoted as $H_{0,s}$
$s\in\frac12\,\dbZ$.

\item"c)"
If $\scrO=\{0\}$, $\Stab_0=G$ and $H_{p}$ must be trivial.
Therefore, $H$ is trivial too.
\endroster
\endexample

\subhead
3. Scattering theory and Coleman-Mandula theorem
\endsubhead

\subsubhead
{\bf 3.1}
\endsubsubhead
The scattering picture consists of the following data:

\roster
\item"1)"
$H$ a ``good'' representation of $P$;

\item"2)"
A unitary operator (called $S$-matrix)
$S\colon\,\scrH\to\scrH$, where $\scrH\colon =\Sym(H)$ such
that $S$ commutes with the action of $P$.
\endroster

Here $\Sym(H)$ is understood in the super-sense, i.e.
$$\Sym(H)\simeq\Sym(H_{+})\otimes\Lambda(H_{-}).$$

\definition{Definition}
An even symmetry of the $S$-matrix is an (unbounded)
operator $B\colon\scrH\to\scrH$ such that

\roster
\item"a)"
$B$ is even and preserves the $\dbZ$-grading on $\scrH$;

\item"b)"
$B$ satisfies the Leibnitz rule;

\item"c)"
$[B,S]=0$.
\endroster
\enddefinition

\example{Example}
Consider $\grp=\Lie(P)$.
Then $\grp$ acts n $\scrH$ by symmetry operators.
\endexample

A theorem of Coleman and Mandula classifies all possible
symmetries of the $S$-matrix in a ``non-degenerate''
situation.

\subsubhead
{\bf 3.2}
\endsubsubhead
Assume that the pair $(H,S)$ satisfies the following
additional conditions:

\roster
\item"a)"
$H$ has irreducible summands with strictly positive masses
only, i.e. only the orbits of the type $\scrO_m^+$ are
allowed.

\item"b)"
$S$ is non-degenerate.
\endroster

The last condition is rather obscure; let us, nevertheless,
give a definition.

Let $\Sigma=\operatornamewithlimits{\cup}\limits_{i}\scrO_m^+$
be the support of $H$ in $V^*$ and consider the operator
$S^2\colon\,\Sym^2H\to\Sym^2H$
$$
S^2\colon\,\Sym^2 H\operatornamewithlimits{\hookrightarrow}
\limits^{i}\scrH\mapright{S}\scrH\mapright{pr}
\Sym^2 H.
$$
$\Sym^2 H$ identifies with the space of $L_2$-sections of a
vector bundle with fibers $H_p\otimes H_q$ over the manifold
$\Sym^2\left(\Sigma\right)\smallsetminus\Delta$ ($\Delta$ is the diagonal).

The operator $S^2$ is given by a kernel
$$
T_{p',q'}^{p,q}\,\,(p\not=q;\,p'\not=q')\colon\,\,
H_p\otimes H_q\to H_{p'}\otimes H_{q'}.
$$

{}From the $V\subseteq P$-invariance of $S$ it is easy to
deduce that $T_{p',q'}^{p,q}$ is concentrated on the
submanifold of $(\Sym^2\left(\Sigma\right)\smallsetminus
\Delta)\times(\Sym^2\left(\Sigma\right)\smallsetminus\Delta)$
given by $p+q-p'-q'=0$, i.e. 
$$
T_{p',q'}^{p,q}=\delta(p+q-p'-q')\Ttil{^{p,q}_{p',q'}}.
$$
The non-degeneracy condition says that
$\Ttil{^{p,q}_{p',q'}}$ is a boundary value of an analytic
function and that the matrix $\Ttil{^{p,q}_{p',q'}}$ is
non-degenerate if $p$ is close to $p'$ and $q$ is close to
$q'$.

\proclaim{Theorem {\rm (Coleman-Mandula)}}
Let the pair $(H,S)$ satisfy (a) and (b) as above.
Then

\roster
\item"{\rm a)}"
Any symmetry $B$ of $S$ has a form $B=B_1+B_2$, where
$B_1\in\grp$, $[B_2,\grp]=0$.

\item"{\rm b)}"
The Lie-algebra of symmetries of $S$ has a form
$\scrS=\grp\oplus I$, where $I$ is a finite-dimensional,
reductive Lie algebra.\qquad $\blacksquare$
\endroster
\endproclaim

\subsubhead
{\bf 3.3}
\endsubsubhead

Let us now describe the extension of the Coleman-Mandula
result to the case of non-necessarily even symmetries.

\definition{Definition}
An odd symmetry of $S$ is an operator
$B\colon\,\scrH\to\scrH$ such that

\roster
\item"{\rm a)}"
$B$ is odd and preserves the $\dbZ$-grading on $\scrH$.

\item"{\rm b)}"
$B$ satisfies the super-Leibnitz rule:

\item""
$B(X\otimes Y)=BX\otimes Y+(-1)^{p(X)}X\otimes BY$.

\item"{\rm c)}"
$[B,S]=0$.
\endroster
\enddefinition

Consider the Lie super-algebra $\scrS=\scrS_{\bar{0}}\oplus
\scrS_{\bar{1}}$ of symmetries of $S$.
We have the following result:

\proclaim{Theorem {\rm (Haag, Lopuszansky, Sohnius)}}
Let $(H,S)$ satisfy a) and b) of $3.2$.
Then $\scrS$ has the following structure:

\roster
\item"$1)$"
$\scrS_{\bar{0}}=\grp\oplus I$ (as in Coleman-Mandula
theorem).

\item"$2)$"
$\scrS_{\bar{1}}$ is finite dimensional,
$[\scrS_{\bar{1}},V]=0$ and as a representation of $G$ it
has a form $\oplusop_{i}(L\oplus\Lbar)$.

\item"$3)$"
If $Q\in L_i$, $Q'\in L_j$, we have
$$
[Q,\Qbar{'}]=\delta_{ij}\cdot \tau(Q,\Qbar)\in V_\dbC
\qquad\text{($\tau$ of $2.1$)}
$$
Here  $Q\to\Qbar$ denotes the operation of passing to the 
adjoint operator acting on $H$.
\item"$4)$"
$[Q,Q']=\delta_{ij}\cdot\iota$, where $\iota$ belongs to $I$ and is
central.
\qquad $\blacksquare$
\endroster
\endproclaim


\enddocument



\bigskip\bigskip
%\newpage

\Refs\nofrills{References}

\ref\no 1
\by Cohen A. M. and de Man R., (1996)
\paper Computational evidence for Deligne's conjecture
regarding exceptional Lie groups
\jour C. R. Acad. Sci. Paris
\vol 322, \rom{s\'erie I}
\pages 427--432
\endref

\smallskip
\ref\no 2
\by Deligne P., (1996)
\paper La s\'erie exceptionnelle de groupes de Lie
\jour C. R. Acad. Sci. Paris
\vol 322, \rom{s\'erie I}
\pages 321--326
\endref

\smallskip
\ref\no 3
\by Duflo M., (1977)
\paper Op\'erateurs diff\'erentiels bi-invariants sur un
groupe de Lie
\jour Ann. Sci. E.N.S., $4^{\text{e}}$ s\'erie, t.10
\pages 265--288
\endref

\smallskip
\ref\no 4
\by van Leeuwen M. A. A.,  Cohen A. M. and
  Lisser B. (1992)
\paper LiE, a package for Lie group computations
\inbook CAN
\publaddr Amsterdam
\endref

\smallskip
\ref\no 5
\by Vogel P., (August 1995)
\paper Algebraic structures on modules of diagrams
\paperinfo preprint
\endref

\endRefs







\enddocument








