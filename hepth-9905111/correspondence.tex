
\section{The Correspondence}
\label{correspondence}

In this section we will present an argument connecting type IIB string
theory compactified on $AdS_5\times S^5$ to ${\cal N} =4 $
super-Yang-Mills theory
\cite{Maldacena:1997re}.  Let us start with type IIB string theory in
flat, ten dimensional Minkowski space. Consider $N$ parallel D3 branes
that are sitting together or very close to each other (the precise
meaning of ``very close'' will be defined below). The D3 branes are
extended along a $(3+1)$ dimensional plane in $(9+1)$ dimensional
spacetime.  String theory on this background contains two kinds of
perturbative excitations, closed strings and open strings. The closed
strings are the excitations of empty space and the open strings
end on the D-branes and describe 
excitations of the D-branes.
 If we consider the system at low energies, energies lower
than the string scale $1/l_s$, then only the massless string states
can be excited, and we can write an effective Lagrangian describing
their interactions. The closed string massless states give a gravity
supermultiplet in ten dimensions, and their low-energy effective
Lagrangian is that of type IIB supergravity. The open string massless
states give an $\cn=4$ vector supermultiplet in $(3+1)$ dimensions,
and their low-energy effective Lagrangian is that of $\cn=4$ $U(N)$
super-Yang-Mills theory
\cite{Witten:1996im,joebook}. 

The complete effective action of the massless modes will have
the form
\eqn{lowenergy}{
S = S_{\rm bulk} + S_{\rm brane} + S_{\rm int}. }  $S_{\rm bulk}$ is
the action of ten dimensional supergravity, plus some higher
derivative corrections.  Note that the Lagrangian \lowenergy\ involves
only the massless fields but it takes into account the effects of
integrating out the massive fields. It is not renormalizable (even for
the fields on the brane), and it should only be understood as an
effective description in the Wilsonian sense, i.e. we integrate out
all massive degrees of freedom but we do not integrate out the
massless ones. The brane action $S_{\rm brane}$ is defined on the
$(3+1)$ dimensional brane worldvolume, and it contains the ${\cal N} =
4 $ super-Yang-Mills Lagrangian plus some higher derivative
corrections, for example terms of the form $\alpha'^2 \tr(F^4) $.
Finally, $S_{\rm int}$ describes the interactions between the brane
modes and the bulk modes. The leading terms in this interaction
Lagrangian can be obtained by covariantizing the brane action,
introducing the background metric for the brane
\cite{Leigh:1989jq}.  

We can expand the bulk action as a free quadratic part describing the
propagation of free massless modes (including the graviton), plus some
interactions which are proportional to positive powers of the square
root of the Newton constant.  Schematically we have
\eqn{expans}{
 S_{bulk} \sim {1 \over 2 \kappa^2} \int \sqrt{g} {\cal R} \sim  
\int (\partial h)^2 + \kappa (\partial h)^2 h + \cdots,} 
where we have written the metric as
$g = \eta + \kappa h $. We indicate explicitly
the dependence on the graviton, but the other terms in the Lagrangian,
involving other fields, can be expanded in a similar way.  Similarly,
the interaction Lagrangian $S_{int}$ is proportional to positive powers of
$\kappa $.  If we take the low energy limit, all interaction terms
proportional to $\kappa $ drop out. This is the well known fact that
gravity becomes free at long distances (low energies).  

In order to
see more clearly what happens in this low energy limit it is
convenient to keep the energy fixed and send $l_s \to 0$ ($\alpha' \to
0$) keeping all the dimensionless parameters fixed, including the
string coupling constant and $N$.  In this limit the coupling 
$\kappa \sim g_s
\alpha'^2 \to 0$, so that the interaction Lagrangian relating the bulk
and the brane vanishes. In addition all the higher derivative terms
in the brane action vanish, leaving just the pure ${\cal N} = 4$ $U(N)$
gauge theory in $3+1 $ dimensions, which is known to be a conformal field
theory. And, the supergravity theory in the
bulk becomes free.  So, in this low energy limit we have two decoupled
systems. On the one hand we have free gravity in the bulk and on the
other hand we have the four dimensional gauge theory.  

Next, we consider
the same system from a different point of view.  D-branes are massive
charged objects which act as a source for the various supergravity
fields. As shown in section \ref{black_pbranes} we can find a D3 brane solution
\cite{Horowitz:1991cd} of supergravity, of the form
\eqn{dthree}{ \eqalign{
ds^2 & = f^{-1/2} ( -dt^2 + dx_1^2 + dx_2^2 + dx_3^2 ) + f^{1/2}
(dr^2 + r^2 d\Omega_5^2 )~,
\cr
F_5 &= (1 + * ) dt dx_1 dx_2 dx_3 df^{-1} ~,
\cr
f &= 1 + { R^4 \over r^4 }~,~~~~~~~~ R^4 \equiv  
4 \pi  g_s \alpha'^2 N ~. }}
Note that since $g_{tt}$ is non-constant,
the energy $E_p$ of an object as measured by an observer 
at a constant  position $r$ and the energy $E$ measured by an observer at
infinity are related  by the redshift factor
\eqn{redshift}{
 E = f^{-1/4} E_p ~.
}
 This means that the same object brought closer
and closer to $r =0$ would appear to have lower and lower energy 
for the observer at infinity. 
Now we take the low energy limit in the background described  by equation
\dthree. There are two kinds of low energy excitations (from the point
of view of an observer at infinity). 
We can have  massless particles propagating in the bulk region
with wavelengths that becomes very large, or we can have any kind
of excitation that we bring closer and closer to 
$r=0$. In the low energy limit these two types of excitations 
decouple from each other. The bulk massless particles decouple from
the near horizon region (around $r=0$)
 because the low energy absorption cross section goes like
$ \sigma \sim  \omega^3 R^8 $ \cite{Klebanov:1997kc,Gubser:1997yh},
 where $\omega $ is the 
energy. 
This can be understood from the 
fact that in this limit the wavelength of the particle becomes 
much bigger than  the typical gravitational size of the brane 
(which is of order $R$). 
Similarly, the excitations that live very close to $r = 0$ find it 
harder and harder to climb the gravitational potential and
escape to the asymptotic region. In conclusion, the low energy 
theory consists of two decoupled pieces, one is free bulk supergravity
and the second is the near horizon region of the geometry. 
In the near horizon region, $r \ll R $, we can approximate 
$f \sim R^4/r^4$, and
the geometry becomes 
\eqn{nearhor}{
ds^2 = { r^2 \over R^2 } ( -dt^2 + dx_1^2 + dx_2^2 + dx_3^2 ) + R^2
{ dr^2 \over r^2 } + R^2  d\Omega_5^2,
}
which is the geometry of $AdS_5 \times S^5$. 

We see that both from the point of view of a field theory of open
strings living on the brane, and from the point of view of the
supergravity description, we have two decoupled theories in the
low-energy limit. In both cases one of the decoupled systems is
supergravity in flat space.  So, it is natural to identify the second
system which appears in both descriptions. Thus, we are led to the
conjecture that {\it ${\cal N} =4 $ $U(N)$ super-Yang-Mills theory in
$3+1$ dimensions is the same as (or dual to) type IIB superstring
theory on $AdS_5\times S^5$} \cite{Maldacena:1997re}.

We could be a bit more precise about the near horizon limit and 
how it is being taken. Suppose that we take $\alpha' \to  0$, as 
we did when we discussed the field theory living on the brane. 
We want to keep fixed the energies of the objects in the throat 
(the near-horizon region) in string units,
so that we can consider arbitrary excited
string states there. This implies that
 $\sqrt{\alpha'} E_p \sim {\rm fixed} $. 
For small $\alpha'$  \redshift\  reduces to 
$E \sim E_p r/\sqrt{\alpha'} $. Since we want to keep 
fixed the energy measured from infinity, which is  the
way energies are measured in the field theory, we need to take 
$r \to 0$ keeping $r / \alpha' $ fixed.  
It is then convenient to define a new variable $U \equiv r / \alpha'$, so
that the metric becomes
\eqn{metricu}{
ds^2 = \alpha' \left[ {U^2 \over \sqrt{4 \pi g_s N} }
( - dt^2 + dx_1^2 + dx_2^2 + dx_3^2 ) +
\sqrt{4 \pi g_s N} { d U^2 \over U^2}  + \sqrt{4 \pi g_s N} d \Omega_5^2
\right].
}

This can also be seen by considering a D3 brane sitting at $\vec r$. 
As discussed in section \ref{black_pbranes} this corresponds to 
giving a vacuum expectation value to one of the scalars in the 
Yang-Mills theory. When we take the $\alpha' \to 0$ limit we want
to keep the mass of the ``$W$-boson'' fixed. This mass, which is the
mass of the string stretching between the branes sitting at 
$\vec r =0$ and the one at $\vec r$, is proportional to $  U = 
r/\alpha'$, 
so this quantity should remain fixed in the decoupling limit. 

A $U(N)$ gauge theory is essentially equivalent to a free $U(1)$
vector multiplet times an $SU(N)$ gauge theory, up to some $\IZ_N$
identifications (which affect only global issues).  In the dual string
theory all modes interact with gravity, so there are no decoupled
modes. Therefore, the bulk $AdS$ theory is describing the $SU(N)$ part
of the gauge theory.  In fact we were not precise when we said that
there were two sets of excitations at low energies, the excitations in
the asymptotic flat space and the excitations in the near horizon
region. There are also some zero modes which live in the region
connecting the ``throat'' (the near horizon region) 
with the bulk, which correspond to the $U(1)$
degrees of freedom mentioned above. The $U(1)$ vector supermultiplet
includes six scalars which are related to the center of mass motion of
all the branes
\cite{Gibbons:1993sv}.  From the $AdS$ point of view these zero 
modes live
at the boundary, and it looks like we might or might not decide to
include them in the $AdS$ theory. Depending on this choice we could have
a correspondence to an $SU(N)$ or a $U(N)$ theory. 
The $U(1)$ center of mass degree of freedom
is related to the topological theory of $B$-fields on $AdS$ 
\cite{Witten:1998wy}; if one imposes local boundary conditions for
these $B$-fields at the boundary of $AdS$ one finds a $U(1)$ gauge 
field living at the boundary \cite{seibergprivate},
 as is familiar in Chern-Simons theories
\cite{Witten:1989hf,Elitzur:1989nr}.
These modes living
at the boundary are sometimes called singletons (or doubletons) 
\cite{Fronsdal:1982gq,Freedman:1984na,Pilch:1984xy,Gunaydin:1985wc,%
Gunaydin:1986tc,Gunaydin:1986cs,Bergshoeff:1988jm,Bergshoeff:1988jx,%
Bergshoeff:1989uc}.

As we saw in section \ref{adsgeom}, Anti-de-Sitter space has a large group
of isometries, which is $SO(4,2)$ for the case at hand.  This is the
same group as the conformal group in $3+1$ dimensions.  Thus, the fact
that the low-energy field theory on the brane is conformal is
reflected in the fact that the near horizon geometry is Anti-de-Sitter
space.  We also have some supersymmetries.
The number of supersymmetries is twice that of the full solution 
(\ref{dthree})
containing the asymptotic region \cite{Gibbons:1993sv}.  This doubling
of supersymmetries is viewed in the field theory as a consequence of
superconformal invariance (section \ref{susyinads}),
 since the superconformal
algebra has twice as many fermionic generators as the corresponding
Poincare superalgebra. 
We also have an $SO(6)$
symmetry which rotates the $S^5$. This can be identified with the
$SU(4)_R$ R-symmetry group of the field theory. In fact, the whole
supergroup is the same for the $\cn=4$ field theory and the $AdS_5
\times S^5 $ geometry, so both sides of the conjecture
have the same spacetime symmetries. We will discuss in more detail the
matching between the two sides of the correspondence in section
\ref{tests}.

In the above derivation the field theory is naturally defined on
$\IR^{3,1}$, but we saw in section \ref{confflatspace} that we could also
think of the conformal field theory as defined on $S^3 \times \IR$ by
redefining the Hamiltonian. Since the isometries of $AdS$ are in one
to one correspondence with the generators of the conformal group of the
field theory, we can conclude that this new Hamiltonian 
${1\over 2}(P_0+K_0)$ can be
associated on $AdS$ to the generator of translations in global time.
This formulation of the conjecture is more useful since in the global
coordinates there is no horizon.  When we put the field theory on
$S^3$ the Coulomb branch is lifted and there is a unique ground
state. This is due to the fact that the scalars $\phi^I$ in the field
theory are conformally coupled, so there is a term of the form $\int
d^4 x \tr(\phi^2) {\cal R}$ in the Lagrangian, where $\cal R$ is the
curvature of the four-dimensional space on which the theory is
defined. Due to the positive curvature of $S^3$ this leads to a mass
term for the scalars \cite{Witten:1998qj}, lifting the moduli space.

The parameter $N$ appears on the string theory side as the flux
of the five-form Ramond-Ramond field strength on the $S^5$,
\eqn{flux}{
\int_{S^5}  F_5  = N. 
}
  From the physics of D-branes we know that 
 the Yang-Mills coupling is 
related to the string coupling
through \cite{Polchinski:1995mt,Douglas:1995bn}
\eqn{couplingcon}{
\tau \equiv 
{ 4\pi i  \over g_{YM}^2 } + { \theta \over 2 \pi}   = 
  { i\over g_s} 
+  { \chi \over 2 \pi } ~, } 
where we have also included the relationship of the $\theta$ angle to
the expectation value of the RR scalar $\chi$. We have written the
couplings in this fashion because both the gauge theory and the string
theory have an $SL(2,\IZ)$ self-duality symmetry under which $\tau \to
(a \tau +b) / (c \tau + d)$ (where $a,b,c,d$ are integers with
$ad-bc=1$). In fact, $SL(2,\IZ)$ is a conjectured strong-weak coupling
duality symmetry of type IIB string theory in flat space
\cite{Hull:1995ys}, and it should also be a symmetry in the present
context since all the fields that are being turned on in the $AdS_5 \times
S^5$ background (the metric and the five form field strength) are
invariant under this symmetry.  The connection between the $SL(2,\IZ)$
duality symmetries of type IIB string theory and ${\cal N} =4$ SYM was
noted in
\cite{Tseytlin:1996it,Green:1996qg,Douglas:1996du}. 
The string theory seems to have a parameter that does not appear in
the gauge theory, namely $\alpha'$, which sets the string tension and
all other scales in the string theory. However, this is not really a
parameter in the theory if we do not compare it to other scales in the
theory, since only relative scales are meaningful.  In fact, only the
ratio of the radius of curvature to $\alpha'$ is a parameter, but not
$\alpha'$ and the radius of curvature independently. Thus, $\alpha'$
will disappear from any final physical quantity we compute in this
theory.  It is sometimes
convenient, especially when one is doing gravity calculations, to set
the radius of curvature to one.  This can be achieved by writing the
metric as $ds^2 = R^2 d{\tilde s}^2$, and rewriting everything in terms
of $\tilde g$. With these conventions $ G_N \sim 1/N^2$ and $\alpha'
\sim 1/\sqrt{g_s N} $.  This implies that any quantity calculated purely
in terms of the gravity solution, without including stringy effects,
will be independent of $g_s N$ and will depend only on $N$.  $\alpha'$
corrections to the gravity results give corrections which are
proportional to powers of $1/\sqrt{g_s N}$.

Now, let us address the question of
the validity of various approximations. The analysis of loop diagrams
in the field theory shows that
we can trust the perturbative analysis in the Yang-Mills theory when
\eqn{pert}{
g^2_{YM} N \sim g_s N \sim { R^4 \over l_s^4 } \ll 1 .
}
Note that we need $g_{YM}^2 N$ to be small and not just $g_{YM}^2$.
On the other hand, the classical gravity description becomes reliable when
the radius of curvature $R$ of $AdS$ and of $S^5$ becomes large compared
to the string length,
\eqn{gravity}{
{ R^4 \over l_s^4 } \sim g_s N \sim  g^2_{YM} N \gg 1.
}
We see that the gravity regime \gravity\ and the perturbative 
field theory regime \pert\ are perfectly incompatible. In this fashion
we avoid any obvious contradiction due to  the fact that
the two theories look very different. This is the reason that 
this correspondence is called a ``duality''. The two theories are
conjectured to be exactly the same, but when one side is weakly coupled
the other is strongly coupled and vice versa. This makes the correspondence
both hard to prove and useful,
as we can solve a strongly coupled gauge theory via classical supergravity.
Notice that in \pert \gravity\ we implicitly assumed that 
$g_s <1$. If $g_s > 1$ we can perform an $SL(2,\IZ)$ duality transformation
and get conditions similar to \pert \gravity\ 
but with $g_s \to 1/g_s$. So, we cannot get into the gravity regime
\gravity\ by taking $N$ small  ($N=1,2,..$)
 and $g_s$ very large, since in that case
the D-string becomes light and renders the gravity approximation 
invalid. Another way to see this is to note that the radius of
curvature in Planck units is $R^4/l_p^4 \sim N$.
So, it is always necessary, but not sufficient, to have
large $N$ in order to have a weakly coupled supergravity description. 

One might wonder why the above argument was not a proof rather than
a conjecture. It is not a proof because we did not treat the string
theory non-perturbatively (not even  non-perturbatively in $\alpha'$).
We could also consider different forms of the conjecture. 
In its weakest form the gravity description would be valid
for large $g_s N$, but the full string theory on $AdS$ might not agree with
the field theory. A not so weak form would say that the conjecture 
is valid even for finite $g_s N$, but only in the $N \to \infty $ limit
(so that the $\alpha'$ corrections would agree with the field theory,
but the $g_s$ corrections may not).
The strong form of the conjecture, which is the most interesting one
and which we will assume here, 
is that the two theories are exactly the same for all values 
of $g_s$ and $N$. 
In this conjecture the spacetime is only required to be asymptotic
to $AdS_5\times S^5$ as we approach the boundary. In the interior we
can have all kinds of processes; gravitons, highly excited fundamental
string states, D-branes, black holes, etc. Even the topology of
spacetime can change in the interior. The Yang-Mills theory is
supposed to effectively sum over all spacetimes which are asymptotic 
to $AdS_5\times S^5$. This is  completely analogous to the usual 
conditions of asymptotic flatness. We can have black holes and all kinds
of topology changing processes, as long as spacetime is asymptotically 
flat. In this case asymptotic flatness is replaced by the asymptotic
$AdS$ behavior.

\subsection{Brane Probes and Multicenter Solutions}
\label{multicenter_sols}

The moduli space of vacua of the $\cn=4$ $U(N)$ 
gauge theory is $ (\IR^6)^N/S_N$, parametrizing the positions
of the $N$ branes in the six dimensional transverse space. 
In the supergravity solution one can replace 
\eqn{multicenter}{
f \propto { N \over r^4 } \rightarrow 
\sum_{i=1}^N { 1 \over | \vec r - \vec r_i |^4 },
}
and still have a solution to the supergravity equations.  We see that
if $|\vec r| \gg |\vec r_i |$ then the two solutions are basically the
same, while when we go to $r \sim r_i $ the solution starts looking
like the solution of a single brane. Of course, we cannot trust the
supergravity solution for a single brane (since the curvature in
Planck units is proportional to a negative power of $N$). What we can
do is separate the $N$ branes into groups of $N_i$ branes with
$g_s N_i \gg 1$ for all $i$. 
Then we can trust the gravity solution everywhere.

Another possibility is to separate just one brane (or a small number
of branes) from a group of $N$ branes. Then we can view this brane as
a D3-brane in the $AdS_5$ background which is generated by the other
branes (as described above).  A string stretching between the brane
probe and the $N$ branes appears in the gravity description as a
string stretching between the D3-brane and the horizon of $AdS$.  This
seems a bit surprising at first since the proper distance to the
horizon is infinite. However, we get a finite result for the energy of
this state once we remember to include the redshift factor.  The
D3-branes in $AdS$ (like any D3-branes in string theory) are
described at low energies by the Born-Infeld action, which is the
Yang-Mills action plus 
some higher derivative corrections.
This seems to contradict, at first sight, the fact that
the dual field theory (coming from the original branes) is just the pure
Yang-Mills theory.
%that we just have pure Yang-Mills on the
%branes.  
In order to understand this point more precisely let us write
explicitly the bosonic part of the Born-Infeld action for a D-3 brane
in $AdS$ \cite{Leigh:1989jq},
\eqn{borninf}{
\eqalign{
S = - { 1 \over (2 \pi)^3 g_s \alpha'^2 } & \int d^4 x  f^{-1}
 \left[ \right. 
\cr
& ~~~~
\left. 
\sqrt{- \det( \eta_{\alpha \beta} + f \partial_\alpha
r  \partial_\beta r +  r^2 f  g_{ij}\partial_\alpha
\theta^i\partial_\beta\theta^j + 
  2 \pi \alpha' \sqrt{f} F_{\alpha \beta } )} -1 \right]~,  
\cr          
 ~~~~f = { 4 \pi g_s \alpha'^2 N \over r^4 } & ~,
}}
where $\theta^i$ are angular coordinates on the 5-sphere.           
We can easily check that if we define a new coordinate
 $U= r / \alpha'$, then
all the $\alpha'$ dependence drops out of this action. 
Since $U$ (which has dimensions of energy) 
corresponds to the mass of the W bosons in this configuration,
it is the natural way to express the Higgs 
expectation value that breaks $U(N+1)$ to $U(N)\times U(1)$.
In fact,  the action \borninf\ is precisely the  low-energy 
effective action in the field theory for
the massless $U(1)$ degrees of freedom, that we
obtain after integrating out the massive degrees of freedom (W bosons). 
We can expand \borninf\ in powers of $\partial U$ and
we see that the quadratic term does not have any correction, which
is consistent with the non-renormalization theorem for ${\cal N} =4 $ 
super-Yang-Mills \cite{Seiberg:1994aj}.
 The $(\partial U)^4 $ term has only a one-loop 
correction, 
 and this is also consistent with another non-renormalization
theorem \cite{Dine:1997nq}.
This one-loop correction can be evaluated explicitly
in the gauge theory and the result agrees with the supergravity result
\cite{Douglas:1997yp}. 
It is possible to argue, using broken conformal invariance, 
that all terms in \borninf\  are determined by the $(\partial U)^4 $ 
term \cite{Maldacena:1997re}. 
Since the massive degrees of freedom that we are integrating  out
have a mass proportional to $U$, the action \borninf\ makes sense
as long as the energies involved are much smaller than $U$. 
In particular, we need $\partial U /U \ll U $. Since \borninf\
has the form ${\cal L}( g_s N (\partial U)^2/U^4 )$, the higher order terms
in \borninf\ could become important in the supergravity regime,
when $g_s N \gg 1 $. The Born Infeld action \borninf , as always,
makes sense only when the curvature of the brane is small, but
the deviations from a straight flat brane could be large. In this
regime we can keep the non-linear terms in \borninf\ while
we still neglect the massive string modes and similar effects.
Further gauge theory calculations for effective actions of 
D-brane probes include \cite{Douglas:1998tk,Das:1999ij,Das:1999fx}.


\subsection{The Field $\leftrightarrow$ Operator Correspondence}
\label{field_operator}

A conformal field theory does not have asymptotic states or  an
S-matrix, so the natural objects to consider are operators. 
For example, in ${\cal N} =4 $ super-Yang-Mills we 
have a deformation by a
marginal operator which changes the value of the coupling 
constant. Changing the coupling constant in the field theory is
related by (\ref{couplingcon}) to
changing the coupling constant in the string theory, which is then 
related to the expectation value of the dilaton. 
The expectation value of the dilaton is set by the boundary condition 
for the dilaton at infinity. So, changing the gauge theory coupling
constant  corresponds to changing the boundary 
value of the dilaton. More precisely, let us denote by ${\cal O}$ the
corresponding operator. We can consider adding 
 the term $\int d^4 x \phi_0(\vec x) {\cal O}(\vec x) $ to the 
Lagrangian (for simplicity we assume that such a term was not
present in the original Lagrangian, otherwise we consider 
$\phi_0(\vec x)$ to be the total coefficient of ${\cal O}(\vec x)$ in the
Lagrangian). 
According to the discussion above,
it is natural to assume that this will change the boundary 
condition of the dilaton at the boundary of $AdS$ to (in
the coordinate system (\ref{poincusual}))
$ \phi(\vec x, z )|_{z = 0} = \phi_0(\vec x)$. 
More precisely, as argued in \cite{Gubser:1998bc,Witten:1998qj},
 it is natural to propose that
\eqn{genera}{
\langle e^{\int d^4 x \phi_0(\vec x) {\cal O}(\vec x) } \rangle_{CFT}
= {\cal Z}_{string} 
\Bigg[ \phi(\vec x, z)\Big|_{z = 0 } = \phi_0(\vec x)
\Bigg],
}
where the left hand side is the generating function of correlation 
functions in the field theory, i.e. $\phi_0$ is an arbitrary function
and we can calculate correlation functions of ${\cal O}$ by taking 
functional derivatives with respect to $\phi_0$ and then setting
$\phi_0 =0$. The right hand side is the full partition function
of string theory with the boundary condition that 
the field $\phi$ has the value $\phi_0$ on the boundary of $AdS$. 
Notice that $\phi_0$ is a function of the four variables parametrizing
the boundary of $AdS_5$.

A  formula like \genera\ is valid in general, for any field $\phi$. 
Therefore, each field propagating on AdS space 
is in a one to one correspondence with 
an operator in the field theory.
There is a relation between the mass  of the field $\phi$ and 
the scaling dimension of the operator in the conformal field theory. 
Let us describe this more generally in $AdS_{d+1}$.
The wave 
equation in Euclidean space for  a field of mass $m$ has 
two independent solutions, 
which behave like $z^{d - \Delta } $ and $z^{\Delta}$
for small $z$ (close to the boundary of $AdS$),
where 
\eqn{dimenmass}{
\Delta = {d\over 2} + \sqrt{ {d^2 \over 4} + R^2 m^2 } .
}
Therefore, in order to get consistent behavior for a massive field, 
the boundary condition on the
field in the right hand side of \genera\ should in general be changed to
\eqn{bcond}{
\phi(\vec x , \epsilon) = \epsilon^{d - \Delta } \phi_0(\vec x),
}
and eventually we would take the limit where $\epsilon \to 0$. 
Since $\phi$ is dimensionless, we see that $\phi_0$ has dimensions
of $[{\rm length}]^{\Delta - d}$ which implies, through the 
left hand side of \genera, that the associated operator ${\cal  O}$
has dimension $\Delta$ (\ref{dimenmass}). A more detailed derivation of 
this relation will be given in section \ref{correlators},
where we will verify that the two-point correlation function
of the operator ${\cal O}$ behaves as that of an operator of dimension
$\Delta$ \cite{Gubser:1998bc,Witten:1998qj}. 
A similar relation between fields on AdS and operators in the field
theory exists also for non-scalar fields, including fermions and tensors
on AdS space.

%Field theory correlation functions
%can be computed by calculating the string theory  partition function
%with suitable boundary conditions on the corresponding fields.
%In general \genera\ could be divergent. These divergences are 
%removed by setting up the boundary conditions at some
%some value $z=\epsilon$ as in \bcond .
%Then, one takes $\epsilon \to 0$ and
%renormalizes  the divergences away. These divergences
%correspond to the usual UV divergences in field theory that
%we always need to renormalize. 

%In the gravity approximation the right hand side of \genera\ can
%be approximated by the value of the classical action on a
%classical solution with the prescribed boundary conditions,
%\eqn{classapp}{
%{\cal Z}_{string} \sim e^{ - S(\phi_0)  }. 
%}
%In this limit the calculation of correlation functions 
%can be done in terms of Feynman diagrams. 
%Particles propagate from the points of the boundary where the operators 
%are inserted to the interior of $AdS$, and
%they can interact in the interior
%as shown in figure \ref{dia}.

Correlation functions in the gauge theory can be computed from \genera\ by
differentiating with respect to $\phi_0$.  Each differentiation brings down
an insertion ${\cal O}$, which sends a $\phi$ particle (a closed string
state) into the bulk.  Feynman diagrams can be used to compute the
interactions of particles in the bulk.  In the limit where classical
supergravity is applicable, the only diagrams that contribute are the
tree-level diagrams of the gravity theory (see for instance
figure~\ref{dia}).

\begin{figure}[htb]
\begin{center}
\epsfxsize=3.5in\leavevmode\epsfbox{motivone.eps}
\end{center}
\caption{
 Correlation functions can be calculated (in the large $g_s N$ limit) in
terms of supergravity Feynman diagrams. Here we see the leading
contribution coming from a disconnected diagram plus connected pieces
involving interactions of the supergravity fields in the bulk of $AdS$.  At
tree level, these diagrams and those related to them by crossing are the
only ones that contribute to the four-point function.
}
\label{dia}
\end{figure} 



This method of defining the correlation functions of a field theory
which is dual to a gravity theory in the bulk of AdS space is quite
general, and it applies in principle to any theory of gravity
\cite{Witten:1998qj}.  Any local field theory contains the stress
tensor as an operator. Since the correspondence described above
matches the stress-energy tensor with the graviton, this implies that
the $AdS$ theory includes gravity.  It should be a well defined
quantum theory of gravity since we should be able to compute loop
diagrams.  String theory provides such a theory. But if a new way of
defining quantum gravity theories comes along we could consider those
gravity theories in $AdS$, and they should correspond to some
conformal field theory ``on the boundary''.  In particular, we could
consider backgrounds of string theory of the form $AdS_5 \times M^5$
where $M^5$ is any Einstein manifold
\cite{Kehagias:1998gn,Gubser:1999vd,Romans:1985an}.  Depending on the
choice of $M^5$ we get different dual conformal field theories,
as discussed in section \ref{other_backgrounds}.  Similarly,
this discussion can be extended to any $AdS_{d+1}$ space,
corresponding to a conformal field theory in $d$ spacetime dimensions
(for $d>1$). We will discuss examples of this in section \ref{adsmore}.




\subsection{Holography}
\label{holography}

In this section we will describe how the AdS/CFT correspondence
gives a holographic description of physics in $AdS$ spaces. 

Let us start by explaining the Bekenstein bound, which states that the
maximum entropy in a region of space is $S_{max} = { \rm Area}/4 G_N$
\cite{Bekenstein:1994dz}, where the area is that of the boundary of
the region.  Suppose that we had a state with more entropy than
$S_{max}$, then we show that we could violate the second law of
thermodynamics.  We can throw in some extra matter such that we form a
black hole. The entropy should not decrease. But if a black hole forms
inside the region its entropy is just the area of its horizon, which
is smaller than the area of the boundary of the region (which by our
assumption is smaller than the initial entropy).  So, the second law
has been violated.

Note that this bound implies that the number of degrees of freedom
inside some region grows  as the area of the boundary of a 
region and not like the volume
of the region. In standard quantum field theories this is certainly 
not possible. Attempting to understand this behavior leads to 
the ``holographic
principle'', which states that in a quantum gravity theory 
all physics within some volume
can be described in terms of some theory on the boundary which 
has less than one degree of freedom per Planck area 
\cite{'tHooft:1993gx,Susskind:1995vu} (so that its entropy satisfies the
Bekenstein bound). 

In the AdS/CFT correspondence we are describing physics in the bulk of
$AdS$ space by a field theory of one less dimension (which can be thought
of as living
on the boundary), so it looks like holography. However, it is hard
to check what the number of degrees of freedom per Planck area is, since
the   theory, being 
conformal,  has an infinite number of degrees of freedom, and the 
area of the boundary of AdS space is also infinite. 
Thus, in order to compare things properly we should introduce a cutoff on
the number of degrees of freedom in the field theory and see what
it corresponds to  in the gravity theory. 
For this  purpose let us write the metric of $AdS$ as 
\eqn{adscav}{
ds^2 = R^2 \left[ - \left( { 1 + r^2 \over 1 - r^2 } \right)^2 dt^2 +
{ 4 \over ( 1 - r^2 )^2 }( dr^2 + r^2 d\Omega^2 ) \right].
}
In these coordinates the boundary of $AdS$ is at $r= 1$.
We saw above that when we calculate correlation functions 
we have to specify boundary conditions at $r = 1-\delta $ and then
take the limit of $\delta \to 0$. 
It is clear by studying the action of the conformal group on 
\Poincare coordinates that the radial position plays the role of
some energy scale, since we approach the boundary when we do a 
conformal transformation that localizes objects in the CFT. 
So, the limit $\delta \to 0$ corresponds to going to the UV of the
field theory.
When we are close to the boundary we could also use the \Poincare 
coordinates
\eqn{poinc}{
ds^2 = R^2 { -dt^2 + d{\vec x}^2 + dz^2 \over z^2},
}
in which the boundary is at $z=0$.
 If we consider a particle or wave propagating in 
\poinc\ or \adscav\
 we see that its motion is independent of $R$ in the supergravity
approximation. Furthermore,  if we are in Euclidean space  and
we have a wave that has some spatial extent 
$\lambda$ in the $\vec x$ directions, it will also have an 
extent $\lambda$ in the $z$ direction. This can be seen from \poinc\
by eliminating $\lambda $ through the change of variables 
$ x \to \lambda x $, $ z \to \lambda z$. 
This implies that a cutoff at 
\eqn{iruv}{
z \sim \delta 
} 
corresponds to 
a UV cutoff in the field theory at distances $\delta$, with 
no factors of $R$ ($\delta$ here is dimensionless, in the field theory
it is measured in terms of the radius of the $S^4$ or $S^3$ that the
theory lives on). Equation \iruv\ is called the UV-IR 
relation \cite{Susskind:1998dq}.

 Consider the case of ${\cal N} = 4 $ 
SYM on a three-sphere of radius one.
 We can estimate the number of degrees of freedom in the field theory
with a UV cutoff $\delta$. We get
\eqn{entroft}{
S \sim N^2 \delta^{-3},
}
%where we are considering the field theory on a sphere of radius one. 
since the number of cells into which we divide the three-sphere is of
order $1/\delta^3$.
In the gravity solution  \adscav\ 
the area in Planck units of 
the surface at $r = 1 - \delta $, for $\delta \ll 1$, is 
\eqn{areagrav}{
{ { \rm Area} \over 4 G_N} = { V_{S^5} R^3 \delta^{-3} \over 4 G_N} 
\sim N^2 \delta^{-3}.
}
Thus, we see that the AdS/CFT correspondence saturates the holographic bound
\cite{Susskind:1998dq}. 

One could be a little suspicious of the statement that gravity 
in $AdS$ is holographic, since it does not seem to be saying much
because in $AdS$ space
the volume and the boundary area of a given region scale in the
same fashion as we increase the size of the region. 
In fact, {\it any } field theory in $AdS$ would
be holographic in the sense that the number of degrees of 
freedom within some (large enough) 
volume is proportional to the area (and also to the volume). What makes 
this case different is that we have the additional 
parameter $R$, and then we can take $AdS$ spaces of 
different radii (corresponding to different values of
$N$ in the SYM theory),
and then we can ask whether the number of
degrees of freedom goes like the volume or the area, since 
these have a different dependence on $R$. 

One might  get confused by the fact that the surface $r =1-\delta$
is really nine dimensional as opposed to four dimensional. From the
form of the full metric on $AdS_5\times S^5$ 
we see that as we take $\delta \to 0$ the physical size of
four of the dimensions
of this nine dimensional space grow, while the other five, the $S^5$, 
remain constant. So, we see that the theory on this nine dimensional 
surface  becomes effectively four dimensional, since we need to multiply 
the metric by a factor that goes to zero as we approach the boundary
in order to define a finite  metric
for the four dimensional gauge theory. 

Note that even though it is often said that the field theory 
is defined on the boundary of $AdS$, it actually describes 
all the physics that is going on inside $AdS$.
When we are thinking in the $AdS$ picture it is incorrect to 
consider {\it at the same time} an additional field theory living
at the boundary\footnote{Except possibly for a small number of
singleton fields.}. Different regions of $AdS$ space, which are at
different radial positions, correspond to 
physics at different energy scales in the field theory. 
It is interesting that depending on what boundary we take, $\IR^{3+1}$ 
(in the \Poincare coordinates) or $S^3\times \IR$ (in the global coordinates),
we can either have a horizon or not have one. The 
presence of a horizon in the  $\IR^{3+1}$ case is related to the 
fact that the theory has no mass gap and we can have
excitations at arbitrarily low energies. This will always happen
when we have a horizon, since by bringing a particle close to a horizon
its energy becomes arbitrarily small. We are talking about the
energy  measured with respect to the time associated to the
Killing vector that vanishes at the horizon. In the $S^3 $ case there
is no horizon, and correspondingly the theory has a gap. In this
case the field theory
has a discrete spectrum since it is in finite volume. 

\begin{figure}[htb]
\begin{center}
\epsfxsize=1.5in\leavevmode\epsfbox{geodesic.eps}
\end{center}
\caption{
Derivation of the IR/UV relation by considering a spatial 
geodesic ending at two points on the boundary. 
}
\label{geodesic}
\end{figure} 


Now let us consider the UV/IR correspondence in spaces that are not
$AdS$, like the ones which correspond to the field theories 
living on D-$p$-branes
with $p \not = 3$ (see section \ref{dpbranes}). 
A simple derivation involves  considering
 a classical spatial geodesic
that ends on the boundary at two points separated by a distance $L$ in 
field theory units (see figure \ref{geodesic}). 
 This geodesic goes into the bulk, and it 
has  a point at which the distance to the boundary is maximal.
Let us call this point $r_{max}(L)$. Then, one formulation of 
the UV/IR relation is
\eqn{uvirgen}{
 r = r_{max}(L) \leftrightarrow L.
}
A similar criterion arises if we consider the wave equation 
instead of classical geodesics \cite{Peet:1998wn};
of course both are the 
same since a classical geodesic arises as a limit of the wave equation
for very massive particles. 

Since the radial  direction arises holographically, it is not obvious
at first sight that the theory will be causal in the bulk.
Issues of causality in the holographic description of the spacetime
physics were discussed in 
\cite{Kabat:1999yq,Balasubramanian:1999ri,Horowitz:1999gf,%
Giddings:1999qu}. 

This holographic description has implications for the physics 
of black holes. This description
should therefore explain how the singularity inside black holes
should be treated (see \cite{Horowitz:1998pq}).
Holography also  implies that black hole evolution is 
unitary since the boundary theory is unitary.  
It is not totally clear, from the gravity point of view, how
the information comes back out or where it is stored
 (see \cite{Lowe:1999pk} for a discussion).
Some speculations about  holography and a 
 new uncertainty principle were 
discussed in \cite{Minic:1998nu}.






















