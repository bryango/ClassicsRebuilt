\documentstyle[12pt,epsf]{amsart}

\setlength{\textwidth}{450pt}
\setlength{\textheight}{600pt}
\setlength{\topmargin}{0pt}
\setlength{\oddsidemargin}{0pt}
\setlength{\evensidemargin}{0pt}

% automatic loading of rsfs fonts, if present
\batchmode
  \newfont{\footscrfont}{rsfs10}
  \newfont{\footbbbfont}{msbm10}
\errorstopmode

\newif\ifscrf\scrftrue
\ifx\footscrfont\nullfont
  \scrffalse
\fi

\ifscrf
  \newfont{\scrfont}{rsfs10 scaled\magstep1}  % rsfs12 does not exist
  \newfont{\smallscrfont}{rsfs7}
  \newfont{\tinyscrfont}{rsfs7}
% \newfont{\footscrfont}{rsfs10}
  \newfont{\smallfootscrfont}{rsfs7}
  \newfont{\tinyfootscrfont}{rsfs7}
\fi

\makeatletter

\newif\iffn\fnfalse

\@ifundefined{reset@font}{\let\reset@font\empty}{} %needed for ancient LaTeX
\long\def\@footnotetext#1{\insert\footins{\reset@font\footnotesize
    \interlinepenalty\interfootnotelinepenalty
    \splittopskip\footnotesep
    \splitmaxdepth \dp\strutbox \floatingpenalty \@MM
    \hsize\columnwidth \@parboxrestore
   \edef\@currentlabel{\csname p@footnote\endcsname\@thefnmark}\@makefntext
    {\rule{\z@}{\footnotesep}\ignorespaces
      \fntrue#1\fnfalse\strut}}}
\makeatother

\ifscrf
  \newcommand{\Scr}[1]{\iffn
    \mathchoice{\mbox{\footscrfont #1}}{\mbox{\footscrfont #1}}
    {\mbox{\smallfootscrfont #1}}{\mbox{\tinyfootscrfont #1}}\else
    \mathchoice{\mbox{\scrfont #1}}{\mbox{\scrfont #1}}
    {\mbox{\smallscrfont #1}}{\mbox{\tinyscrfont #1}}\fi}
\else
  \def\Scr{\cal}
\fi


%integral sign with cirlce is \oint

%GIT quotient
\newcommand{\catquot}{\mathchoice
{\mathrel{\mskip-4.5mu/\!/\mskip-4.5mu}}
{\mathrel{\mskip-4.5mu/\!/\mskip-4.5mu}}
{\mathrel{\mskip-3mu/\mskip-4.5mu/\mskip-3mu}}
{\mathrel{\mskip-3mu/\mskip-4.5mu/\mskip-3mu}}
}

%Hodge star (hstar) should probably not be \star
\newcommand{\hstar}{\mathop{*}}

% a wedgeproduct of a more reasonable size and spacing (from Robert Bryant)
\def\wdg{{\mathchoice{\,{\scriptstyle\wedge}\,}{{\scriptstyle\wedge}}
{{\scriptscriptstyle\wedge}}{{\scriptscriptstyle\wedge}}}}
%should it be a mathbin?

%line bundle (lb) and Lagrangian (lg) -- should have distinct fonts
\newcommand{\lb}{{\Scr L}}
\renewcommand{\lg}{{\cal L}}

%same font for circle bundles
\newcommand{\cb}{{\Scr S}}

%coupling constant in gauge theory (\ee) should not be in mathitalic, to avoid
%confusion with exponential
\newcommand{\ee}{\text{e}}
\newcommand{\gee}{\text{g}}

%script A, the space of connections
\newcommand{\cA}{{\cal A}}

%script D, for use in pathspace measures
\newcommand{\cD}{{\cal D}}

%script H, for Hilbert space
\newcommand{\cH}{{\cal H}}

%notation for operators
\newcommand{\cO}{{\cal O}}

%standard blackboard bold items
\newcommand{\IR}{{\Bbb R}}
\newcommand{\IZ}{{\Bbb Z}}

%operatornames
\newcommand{\Hol}{\operatorname{\rm Hol}}
\newcommand{\Maps}{\operatorname{\rm Maps}}
\renewcommand{\Re}{\operatorname{\rm Re}}
\newcommand{\Tr}{\operatorname{\rm Tr}}
\newcommand{\vol}{\operatorname{\rm vol}}

\numberwithin{equation}{section}

%\addtolength{\marginparwidth}{-.32in}
\addtolength{\marginparwidth}{-.4in}
\newcommand{\note}[1]{\marginpar{\scriptsize #1 }} 

\begin{document}

\title[]{Lecture II-2, part I: Infrared Behavior of Quantum Field Theories}
\author[]{Edward Witten$^*$}
\thanks{$^*$Notes by David R. Morrison}
\date{23 \& 29 January 1997}
\maketitle


Given a quantum field theory $X$, we want to solve it, that is, to learn the 
most interesting
things about it.  A big piece of ``solving'' a theory
is determining what it flows to
in the infrared.  Fairly often, the answer is: ``nothing,'' that is, $X$
flows to a trivial theory.  This happens precisely when $X$ has a mass gap,
for then all (Euclidean) correlation functions decay exponentially.
Showing that a given theory flows to a trivial theory may, however,
be a rather deep result.

Very often, the infrared limit is not trivial but is 
a free theory of massless particles, together
with an irrelevant interaction which goes to zero in the infrared.  
In fact, this happens in most of the simplest examples that we will meet.
Note that an irrelevant interaction would, in the ultraviolet, be considered
``unrenormalizable''; the perturbations that are ill-behaved in the
ultraviolet are just the ones that vanish as one flows to the infrared limit.

When a theory is free in the infrared, the question then becomes: {\it
which}\/ massless particles is it a free theory of?  They might not be related
to the ones in the original Lagrangian.  In fact, as we shall see in the second
part of the lecture,
the answer to this
question may depend on the vacuum state we are in.

For the infrared limit to be trivial is a special case of the infrared limit
being free; it is the case that there are no massless particles at all in the
physical spectrum.  


We begin with an example, and then discuss several general features of infrared
limits.

\section{Example from last time}

Consider a theory which breaks $SO(3)$ to $SO(2)=U(1)$.  We have three real
scalars which can be combined to a $3$-component object
$\vec{\phi}$ which transforms in the
$3$-dimensional representation of $SO(3)$.  The Lagrangian is
$$\lg=\frac1{\lambda}\int
d^nx\,\left[\frac12(\partial_\mu\vec{\phi})^2+V(|\vec{\phi}|)\right],$$
where, letting $\rho=|\vec{\phi}|$, $V(\rho)$ is a potential which has a
nondegenerate minimum away from the origin.  

\centerline{\epsfxsize=1in\epsfbox{plot3.eps}}
\centerline{\quad}

\noindent
The simplest such potential is $V(\rho)=\frac18(\rho^2-\rho_0^2)^2$, with a
minimum at $\rho_0$.  If we let
 $\Omega=\phi/\rho\in S^2$, we can rewrite the Lagrangian as
$$\lg=\frac1{\lambda}\int
d^nx\, \left[\frac12(\partial_\mu\rho)^2 +\frac12\rho^2(\partial_\mu\Omega)^2
+V(\rho)\right].$$ 
The term $\frac12\rho^2(\partial\Omega)^2$ represents the round metric on
$S^2$.

When $\lambda$ is small, we can hope that the classical approximation will be
good.  Since $\rho$ is a massive field, we can integrate it out of the theory
by setting it equal to its minimum value $\rho_0$ and studying fluctuations
$$\rho=\rho_0+w,$$
 where $w$ is now
the quantum field which will appear in our path integrals.
To first approximation, we would have an effective Lagrangian
$$\lg_{\text{eff}}(\Omega)=\frac{\rho_0^2}{2\lambda}\int
d^nx\,(\partial\Omega)^2,$$ 
describing
a nonlinear sigma model of maps to $G/H$, which in the present case is $S^2$. 
The massless fields are described by $\Omega$.   

Can this be an answer?  In other words, could {\it any}\/ quantum field theory
flow to this sigma model in the infrared?
In fact, it {\it is}\/ a possible answer for spacetime dimension
$n>2$, because the nonlinear
sigma-model is non-renormalizable above two dimensions, so the interaction we
get is irrelevant.  That is, if $a_i$ are Riemann normal
coordinates on near a point  $P\in S^2$, we can expand
schematically near $P$
$$(\partial\Omega)^2=(da)^2+R a^2 (da)^2 +\dots$$
with $R$ being the Riemann tensor of $S^2$.
{}From this we see that the interaction is irrelevant 
 above two dimensions.  In fact, to give $(da)^2$ dimension $n$, 
 we must assign dimension $(n-2)/2$ to $a$, whence the interaction
 has dimension $2n-2$, which exceeds $n$ for $n>2$.  The fact that the
 sigma model is a possible  answer for $n>2$ but not for $n\leq 2$ is
 an aspect of the fact, already discussed last week, that spontaneous
 breaking of a continuous symmetry is possible for $n>2$ but not for 
 $n\leq 2$.

Is the sigma model the {\it correct}\/ answer for the infrared behavior
of our particular problem, at least for sufficiently small $\lambda$?  
We claim that it is.  To study this point, let us
treat the effects of $w$ perturbatively.  The interactions of $w$
with the Goldstone
boson come from the interaction term in the Lagrangian:
$$\frac1{2\lambda}\rho^2(
\partial\Omega)
^2=\frac1{2\lambda}(\rho_0+w)^2(\partial\Omega)^2= 
\frac1{2\lambda}(\rho_0^2+2\rho_0w+w^2)(\partial\Omega)^2.$$ 
The operator $\partial\Omega$ is highly nonlinear, and can be thought of as  
emitting an arbitrary number of Goldstone bosons.  We need to calculate
Feynman diagrams involving $w$'s, in order to find the effective Lagrangian.  
A typical
diagram is

\centerline{\epsfxsize=2in\epsfbox{feyn1.eps}}
\centerline{\quad}

\noindent
representing $w$ with a solid line and $a$'s with dotted lines.  The 
$w$-propagator is $\frac1{k^2+m_w^2}$; since we are interested in small momenta
$k$, we expand in powers of $k$.  The leading term is
$$\text{const}\,(\partial\Omega)^2\frac1{m_w^2}(\partial\Omega)^2.$$
This has $SO(3)$ symmetry and is an irrelevant interaction.
It is even more irrelevant than the terms, sketched above, that
come by expanding $(\partial \Omega)^2$ in powers of $a$.


In fact, while it is instructive to study these diagrams, just to show
that the sigma model is infrared-stable, we do not need the details
of the diagrams.  All we need to know is that the effective action
has $SO(3)$ symmetry.
So what possible terms could be generated in the effective action?
\begin{itemize}

\item A potential $V(\Omega)$  is not possible, because there is no
$SO(3)$-invariant function on $S^2$.  This is the basic reason that, given
the $SO(3)$ symmetry, the sigma model is infrared stable.  A potential
function would change the picture completely.  For instance, a generic
potential would have an isolated, nondegenerate minimum, giving us
a unique vacuum with an infrared-trivial massive theory, in contrast
to a continuous family of vacua associated with spontaneously broken $SO(3)$.

\item
The only term with only two derivatives that respects all the symmetries
of the problem is $(\partial\Omega)^2$ itself.  So quantum corrections
due to diagrams with $w$ fields should definitely be expected to modify
the coefficient of this term.

\item Other possible terms like $((\partial\Omega^2)^2$ and
$\partial\Omega\,\nabla^2(\partial\Omega)$ have more than two derivatives
and are more and more irrelevant.
\end{itemize}

So the leading infrared behavior is determined by an effective action of 
the form
$\frac1{f^2}(\partial\Omega)^2$ with
$$\frac1{f^2}=\frac{\rho_0^2}{2\lambda}+\text{ loop corrections}.$$
When this is expanded in Riemann normal coordinates about a given
vacuum, that is a given point $P\in G/H$, one gets interactions
(of which the first was sketched above) that involve the Riemann tensor
of $G/H$ and its covariant derivatives.
If one uses these interactions to compute scattering amplitudes
involving Goldstone bosons with small momenta of order $k$, the
tree level amplitudes are all proportional to $k^2$ for $k$ near zero,
as the interaction terms all contain precisely two derivatives.  
Loop contributions are smaller for $k\to 0$, since the interactions
are irrelevant in the infrared.  To be more precise,
loop
amplitudes all either (i) renormalize the constant $f$ in the
$SO(3)$-invariant Lagrangian, or (ii) give corrections to the scattering
amplitudes that vanish faster than $k^2$ for $k\to 0$.

Hence, the terms of order $k^2$ in the Goldstone boson scattering
are all completely determined by the one constant $f$ (or more generally
by the choice of a $G$-invariant metric on the homogeneous space $G/H$).
In the 1960's, it was discovered that the low energy scattering of pions
beautifully fits such a description, with $G=SU(2)\times SU(2)$ and
$H$ a diagonal $SU(2)$.  This is how it was discovered that the strong
interactions have a spontaneously broken approximate chiral symmetry;
the discovery played a very major role in the subsequent development of
physics.

What happens if one wants to compute terms in the Goldstone boson
scattering of higher order than $k^2$?  It is clear that in order $k^4$,
new constants will enter that can only be determined from microscopic
calculations (or experiment), since there are $G$-invariant interactions
with four derivatives (such as the $((\partial\Omega)^2)^2$ term found
above from the explicit tree diagram considered).  However, interestingly,
in four spacetime dimensions, the lowest order correction to the $k^2$
amplitude for Goldstone bosons is not of order $k^4$ but of order
$k^4\ln k$.
It comes from a loop diagram 

\centerline{\epsfxsize=2in\epsfbox{feyn2.eps}}
\centerline{\quad}

\noindent
with vertices drawn from the two-derivative part of the Lagrangian,
and hence is uniquely detemined in terms of the same constant $f$ that
controls the $k^2$ terms in the scattering amplitudes.  The analysis
of low-energy Goldstone boson interactions via the ideas I have explained
is known as ``current algebra.''  In particular, via current algebra
relations, one can deduce from experiment what is the broken symmetry
group $G$, and many of the parameters in the $G$-invariant effective 
Lagrangian.

One final comment about symmetry-breaking examples such as this one: 
if we begin with a $G$-invariant microscopic Lagrangian
$\lg_{\text{micro}}$ which we perturb to
$$\lg_{\text{micro}}+\varepsilon(\delta\lg)$$
with the term $\delta\lg$ not being $G$-invariant, then  in the infrared we
will get
$${1\over f^2}(\partial\Omega)^2+\varepsilon V(\Omega)+O(\varepsilon^2),$$
with $V(\Omega)$ being a non-$G$-invariant operator -- of which the most
relevant part is of course a potential with no derivatives, as suggested
in the notation $V$.
$V(\Omega)$ is highly constrained by the fact that it must transform
under $G$ the same way that $\delta\lg$ does.  For example, in the case
of strong interactions, a small $\delta\lg$ term, breaking $SU(2)\times SU(2)$
to a diagonal $SU(2)$, is actually present; it selects a unique vacuum
from what would otherwise be a continuous family, and gives small masses
to the pions.  In current algebra studies of pions, one really takes the
momentum $k$ to be of order the pion mass.

\section{Which spins?}

Now we consider in a general way
infrared-free theories in $4$ dimensions.
(The considerations that follow generalize above 4 dimensions but
become trivial below dimension 4).  The general discussion seems to suggest
that infrared-free theories might have massless particles of any spin.
But in practice, in all interesting examples I am familiar with, one
can argue {\it a priori}
that any massless particles will have 
spins $0$, $1/2$ or $1$.


Most theories of interest can be formulated not just on flat ${\bf R}^4$,
but on a more general curved $4$-manifold $M^4$ with a general metric $g$.
In fact, any theory that is part of the description of nature has this
property, since general relativity is part of nature and in nature,
space-time is curved!
In quantum field theory, the ability to work on a curved space-time
implies the existence of a very special operator,
called the stress tensor or energy-momentum tensor $T_{\mu\nu}(x)$.
It measures the response to a change in the metric tensor $g$.
We suppose that  a theory is formulated with a general $g$ by a Lagrangian
$\lg(\phi_i;g)$, 
which  is invariant under diffeomorphisms acting both on the $\phi_i$ and on
$g$. $g$ is not one of the fields of the theory -- it is arbitrary
but is held fixed in studying the classical or quantum dynamics of the $\phi_i$
-- and this diffeomorphism invariance means that the theory,
if formulated in a spacetime $(M,g)$, really depends on $g$ only up to
diffeomorphism. In this setup,
the stress tensor is
defined as 
$$T_{\mu\nu}={\delta \lg\over \delta g^{\mu\nu}}.$$
This implies obviously that $T$ is a symmetric tensor
$$ T^{\mu\nu}=T^{\nu\mu}.$$
$T$ can also be shown to obey
$${D_\mu}T^{\mu\nu}=0$$
by virtue of diffeomorphism invariance.
If our theory is actually {\it conformally invariant}, then
$T$ is traceless, that is $g^{\mu\nu}T_{\mu\nu}=0$.

Having such a stress tensor leads to powerful statements even if
one specializes to the case that $M$ is flat Euclidean space. For instance,
last fall, when we axiomatized quantum field theory, we required
Poincar\'e invariance, with conserved charges $Q(K)$ for every Killing
vector field $K$ in spacetime.  The existence of a conserved, symmetric
stress tensor is a local statement that leads to Poincar\'e invariance
globally.  Given any Killing vector field $K$, one uses the Killing
vector equation (which reads $D_\mu K_\nu+D_\nu K_\mu=0$) plus symmetry
and conservation of $T$ to prove that the current
$$J^\nu(K)=K_\mu T^{\mu\nu}$$
is conserved.  Then one obtains the conserved charge
$$Q(K)=\int_{\Sigma^{n-1}}\hstar J(K)$$
where the integral is taken over an initial value  hypersurface
(such as time zero).

I have presented this as if one needs to have a Lagrangian
so as to deduce the existence of a stress tensor $T=\delta\lg/\delta g$.
Though this is  a powerful approach, one can also argue more abstractly.
Consider any theory which can be formulated for any metric on $M$.
To define an operator $T(y)$, we must give a definition,
for any   specified $n$
points $x_1$, \dots, $x_n$ on $M^4$ distinct from each other and from $y$
and operators $\cO_1,\dots, \cO_n$,
the correlation function $\langle\cO_1(x_1)\cdots\cO_n(x_n)T(y)\rangle_g$
(here the subscript serves to emphasize that the correlation function depends 
on 
a metric $g$).
We define this as the derivative of  $\cO_1(x_1)\cdots\cO_n(x_n)\rangle_g$
with respect to $g$:
$$
\frac{\delta}{\delta g(y)}\langle \cO_1(x_1)\cdots\cO_n(x_n)\rangle_g
=\langle\cO_1(x_1)\cdots\cO_n(x_n)T(y)\rangle_g.
$$
The reader should verify that this definition agrees
with the previous one in case a Lagrangian exists. $T$ as defined in this
way is obviously symmetric; it is conserved if the theory depends only
on the diffeomorphism class of the metric $g$.  Many of the properties
of a local quantum field operator follow readily from this definition
of $T$, and it is plausible to believe that they all do in general.


In any event,\footnote{Except in studying quantum gravity, which
has a very different flavor from quantum field theory in 
a fixed spacetime, which is the subject of our lectures this spring.} 
one normally considers in practice theories that have a local,
conserved, symmetric (and of course gauge-invariant)
stress tensor.  As I have essentially already
noted, any theory that appears in nature has this property, since $T$
appears directly in the Einstein equations!

Existence of $T$ leads\footnote{S. Weinberg and E. Witten,
{\it Limits on massless particles}, Phys. Lett. B {\bf 96} (1980), 59--62.}
to sharp restrictions on possible massless particles.
The possible spins of a massless particle in a theory with a stress tensor
are $0,1/2$, and $1$.
A further and analogous restriction is the following.  Let $J^\mu$ be
a conserved current associated with a ``global symmetry.''  Thus,
$J$ transforms as a vector under Poincar\'e, and the conserved charge
$$Q(J)=\int *J$$
is Poincar\'e invariant.  Then $Q$ annihilates any massless particle of spin
1.   

The proofs are so similar that we consider the two cases together.  
In four dimensions, 
denote by $|p,j\rangle$, a massless one-particle
state of momentum $p$ and spin $j$
(in general $j\in {\bf Z}/2$, and let $|p',j\rangle$ be a state of different
momentum in the same Poincar\'e representation.
 Let $|p,j\rangle$  be an eigenstate of $Q$ with
eigenvalue $q$.   Consider the matrix elements
$\langle p',j|J^\nu|j,p\rangle $ and
$\langle p',j|T^{\mu\nu}|j,p\rangle$.  The latter cannot vanish
at all, and the former cannot vanish unless $q=0$.
For  in the limit that $p'\to p$,
 we have by Lorentz invariance
\begin{enumerate}
\item $\langle p',j|J^\nu|j,p\rangle \sim  p^\nu$, and

\item $\langle p',j|T^{\mu\nu}|j,p\rangle \sim p^\mu p^\nu$
\end{enumerate}
The  proportionality constant is $q$ in the first case and 1 in the second, 
since
(as $Q$ and the momentum operators are obtained by integration of
$J$ or $T$) matrix elements of $J$ or $T$ with identical initial
and final states measure the charge or momentum of the state.


On the other hand, one can prove using Lorentz invariance that for all
$p'\not= p$, these matrix elements vanish in the first case for spin greater
than $1/2$, and in the second case for spin greater than 1.
The proof goes by noting first for for $p'\not= p$, the subgroup
of the Lorentz group that leaves fixed both $p$ and $p'$ is a copy
of $SO(2)$ (or $SO(n-2)$ if we are in $n$ spacetime dimensions;
the present considerations degenerate below four dimensions as $SO(n-2)$
is then trivial).  One simply shows that  $SO(2)$ invariance of 
$\langle p',j|J^\nu|j,p\rangle \sim Q p^\nu$ and 
 $\langle p',j|T^{\mu\nu}|j,p\rangle$,
  assuming that these matrix elements are nonzero, implies that 
  the spin is in absolute value $\leq 1/2$ or $\leq 1$, in the two cases.
 A convenient way to perform this computation is to go to a Lorentz
 frame in which (writing the time coordinate first), $p=(1,1,0,0)$
 and $p'=(1,-1,0,0)$.  The $SO(2)$ that leaves fixed both $p$ and $p'$
 is the rotation of the last two coordinates.  Under the generator
 of this $SO(2)$, the states $|j,p\rangle$ and $|j,p'\rangle$ have
 respectively eigenvalue $j$ and $-j$.  The minus sign for the $|j,p'\rangle$
 state,
 which is crucial, arises because it describes a particle
  moving in the opposite direction
 from $|j,p\rangle$; they each have the same spin relative to their
 own directions of motion, but opposite spins if referred to a fixed axis.
 So $\langle p',j|J^\nu|j,p\rangle $ or 
 $\langle p',j|T^{\mu\nu}|j,p\rangle$
 can be nonzero only if, in the $SO(2)$ action on $J$ or $T$, there is
 a term with spin or eigenvalue $-2j$.  As the components of $J$
 transform under $SO(2)$ with
  spin $\leq 1$ in absolute value, while for $T$
 one has components of spin $\leq 2$, we get $|j|\leq 1/2$ or
 $|j|\leq 1$ in the two cases, as was claimed above.

We can actually be somewhat more precise about this result.
  We have so far used only representation
theory, but in quantum field theory one has also a CPT theorem, which
implies in four dimensions
 that every massless particle of spin $j$ is accompanied by one of
spin $-j$.  So spins $\pm 1/2$ will go together, and likewise spins $\pm 1$.

In general dimension $n$, similar reasoning gives the following
result.  The spin of a massless particle is classified by
a representation of the ``little group''  $SO(n-2)$.  If a stress
tensor exists, the allowed
representations  for massless particles are
 the 
spinor representation(s), and exterior powers of the fundamental
$n-2$-dimensional
representation (including the trivial representation).\footnote{
For $n-2$ divisible by four, the middle exterior power can
be decomposed into self-dual or anti-self-dual pieces which are each
{\it real}; one can be present without the other.}
Global charges vanish except for massless particles transforming in the trivial
or spinor representation.
This $n$-dimensional
 formulation is related to the statement in four dimensions
as follows: $j=\pm 1/2$ correspond to the two spinor representations,
while $j=0$ and $j=\pm 1$ come from the exterior powers of the fundamental
representation.


\section{Why are particles massless?}

If the couplings in a theory are generic, massless particles must be massless
for a reason.  One possible reason is supersymmetry, but we won't discuss that
now.  Other possible reasons 
are as follows.

Spin $0$ particles are massless when they are Goldstone bosons, that is, when
there is a broken symmetry.  Spin $1/2$ particles are massless when they are
chiral fermions; their masslessness is due to an {\it unbroken}\/ chiral
symmetry.  This means simply the following:
the CPT theorem says that if one has $n$ massless particles of spin $1/2$,
one also has $n$ such particles of spin $-1/2$.  If there is an
unbroken
symmetry group $G$ and the massless particles of spin $1/2$ transform
in a representation $R$ of $G$, then the massless particles of spin $-1/2$
transform, by the CPT theorem, in the representation $\overline R$ (the
complex  conjugate of $R$).  If $R$ and $\overline R$ are distinct.
this spectrum cannot be perturbed in a $G$-invariant way to a give
masses to the fermions.

The reasons just mentioned are really the only known reasons to have
massless particles of spin 0 or 1/2 without supersymmetry and without
adjusting some parameters to make particles massless.  For spin 1 the
situation is somewhat different.
The Poincar\'e representation
of a  massless spin 1 particle, in four (or more) dimensions, simply cannot
be perturbed to give a mass to the particle, unless there is 
a massless spin 0 particle that can combine with it in a Higgs mechanism
as will be discussed in the second half of
lecture.  If our massless spin 0 particles
are Goldstone bosons, then the broken symmetries that shift  them
ensure that they cannot participate and disappear from the massless spectrum
in a Higgs mechanism.  So as long as the massless spin 0 particles
are Goldstone bosons, and {\it as long as the theory is infrared-free},
the existence of massless spin 1 particles is stable just from Poincar\'e
symmetry.

This
 depends heavily on the theory being infrared-free
since we applied group theory to one-particle states.  If interactions
are important even at low energies, the states with different numbers of
particles can ``mix,'' and we cannot draw a conclusion just by applying
Poincar\'e invariance to the one-particle states.

What interactions can massless particles of these types have?
In the case of spin $0$ particles, which we assume to be Goldstone bosons,
there are no relevant interactions.   We have already seen at the beginning
of this lecture that there are no relevant or marginal
interactions of Goldstone
bosons only.  There are likewise no relevant or marginal 
interactions of fermions
only above two dimensions -- we explored such questions in the fall term --
and with similar arguments and a little more care, one can show that
there are no relevant or marginal couplings of Goldstone bosons to
fermions.



Spin $1$ particles are of a different nature, since they correspond to
gauge fields, and gauge fields {\it can}\/ have relevant interactions in
the infrared.  If $G$ is the gauge group and $A$ is the connection, the
Lagrangian
$$\int\frac1{\gee^2}\,|F_A|^2d^4x$$ is nonlinear.
It contains couplings that in four dimensions
are marginal classically. Whether the interactions
are  relevant or
irrelevant quantum mechanically
depends on the behavior of the $\beta$-function.  An irrelevant
nonlinearity in the infrared will allow the gauge theory to function as an
effective description.  On the other hand, if the nonlinearity is relevant
in the infrared, then the gauge theory is {\it not}\/ the answer.  (In the
intermediate case when $\beta=0$ we would get a non-free theory in the
infrared.)

Goldstone bosons are always invariant under gauge symmetries -- a gauge
group acting on them would violate the global symmetry that leads to
having Goldstone bosons in the first place.
So if we see massless spin $1$ fields in the infrared without adjusting
parameters to make it so, we should expect that  either $G$ must be
abelian, or else there must be enough  fermions in large enough
representations of $G$ so that $\beta>0$.
To explain their masslessness, the 
 fermions are chiral (that is, the states of $j=1/2$ and $j=-1/2$
transform differently) either under the gauge group itself,
or under some unbroken global symmetry group that commutes with the
gauge group.
\end{document}





