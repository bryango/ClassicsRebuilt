%Date: Thu, 31 Oct 1996 17:28:46 EST
%From: Edward Witten <witten@sns.ias.edu>


\input harvmac
{\it ``Problem Set Five''}

(1)  Consider motion of a non-relativistic particle in one dimension
with a delta function potential.

The Hamiltonian is

\eqn\gorry{H={p^2\over 2m}+\lambda \delta(x)}

with $p=-id/dx$ and $m,\lambda$ being constants.

Find a solution of the Schrodinger equation $H\psi = k^2\psi$

which is asymptotic for $x\to -\infty$ to $e^{ikx}+R(k)e^{-ikx}$
and for $x\to +\infty$ to $T(k)e^{ikx}$.  $R$ and $T$ are called
the reflection and transmission amplitudes; show that $|R|^2+|T|^2=1$.

Does this problem have a bound state?  That is, is there an ${\bf L}^2$
solution of the equation $H\psi=-\epsilon\psi$ with $\epsilon>0$?
(The answer may depend on $\lambda$!)

(2) The following is a continuation of a problem from last week.
We considered the simple harmonic oscillator in one dimension
with Lagrangian
\eqn\ibbo{L=\int dt\,\left(\half m\dot x^2-\half m\omega^2x^2\right)}
and Hamiltonian 
\eqn\nibbo{H={p^2\over 2m}+{1\over 2}m\omega^2 x^2}
and we computed the kernel
$K(b,a;t)=(b,e^{-tH}a)$ by a path integral
\eqn\fibbo{K(b,a;t)=\int_{X(0)=a}^{X(t)=b} DX(t')\,\,e^{-L}}
over maps $X$  of $I=[0,t]$ to ${\bf R}$ with $X(0)=a$ and $X(t)=b$.
(In this exercise we will be computing $e^{-tH}$ with real $t$ so $I$
is ``Euclidean'' and we use $e^{-L}$ rather than $e^{iL}$ in the path 
integral.)

Now we'd like to compute $\Tr e^{-tH}=\int da K(a,a;t)$.
To do so, we simply set $b=a$ in \fibbo\ and integrate over $a$.
But a map $X:I\to {\bf R}$ that has the same value $a$ at the endpoints
is the same as a map $X:S\to {\bf R}$, where $S$ is the circle 
(of circumference $t$) obtained  by gluing together the ends of $I$.
Moreover, in the formula \fibbo\ for $K$ we are already once we set $b=a$
integrating over all maps $S\to {\bf R}$ with value $a$ at the former
endpoint; once we also integrate over $a$ to get $\Tr \,e^{-Ht}$, we
are simply integrating over all maps $S\to {\bf R}$.

So the upshot is a formula
\eqn\yoggo{\Tr\,e^{-Ht}=\int dX\,e^{-L}}
where the integral is over all real-valued functions on $S$ and $L$
is the Lagrangian evaluated for such an $X$.

(a)

As $L$ is a homogeneous quadratic function on the linear space 
$Maps(S,{\bf R})$, the integral can be evaluated formally as $1/\sqrt{\det D}$
where $D$ is a differential operator on $S$.  Calculate the determinant
(using the same sort of methods you needed in last week's problem)
and compare to a standard formula for $\Tr \,e^{-Ht}$.

(b)  Now let $J$ be the operation $X\to -X$.  We want to compute
by path integrals $\Tr\,Je^{-Ht}$.  This is $\int da K(-a,a;t)$
so it can be computed exactly as  before except that when
we glue the endpoints of $I$ to make the circle $S$, we have
to require that the boundary values obey $b=-a$.  This means that
$X$ is not a real-valued function on $S$, but a section of
an unorientable real line bundle ${\cal L}$ over $S$.  With ${\cal L}$ 
understood to have a metric (its structure group in particular is
${\bf Z}_2$ and not ${\bf R}^*$)  the Lagrangian $L$ makes sense for
$X$ a section of ${\cal L}$ and we have
\eqn\okoppo{\Tr\,J\,e^{-Ht}=\int' DX\,e^{-L}}
where now the integral runs over the space of sections of ${\cal L}$.
As this space is again a linear space on which $L$ is a quadratic function,
the path integral is again a Gaussian.  Evaluate the determinant
and compare to another evaluation of $\Tr J\,e^{-Ht}$.

(3)  Now we will do an analogous computation for fermions,
to demonstrate by path integrals that quantization of the Lagrangian
\eqn\koko{ L=\int_M dt i\sum_{j=1}^n\psi_j{d\psi_j\over dt} }
leads to the spin representation.  Here $M$ is an oriented one-manifold
and the $\psi_j$ are $n$ real fermions. Actually, to be precise, we will
take $n$ to be even; there are then two spin representations
$S_+$ and $S_-$, and we will see them both appear.

First of all, in \koko\ we might take $M={\bf R}$ in which case we
quantize and get a Hilbert space ${\cal H}$.  On this Hilbert space
there would be in general a Hamiltonian, but in the case at hand
-- since \koko\ is invariant under reparametrizations of $t$ -- the
Hamiltonian is $H=0$.

Rather than trying
to determine by canonical quantization what ${\cal H}$ would be,
here we will study the situation with path integrals.
For this we take $M$ to be $S^1$, as in the last exercise.
To find the dimension of ${\cal H}$, we want $\Tr_{\cal H}e^{-t H}$
which (as $H=0$) as $\Tr_{\cal H}1=\dim\,{\cal H}$.  By naively
imitating the last excerise, you would think we would have
\eqn\hombo{?? \Tr_{cal H}1 =\int D\psi(t) e^{-L}}
which in turn could be computed as a Pfaffian, rather as for
the bosonic harmonic oscillator.

Once we are on a circle, though, we note that $\psi$ is best understood
as a section of the spin bundle of the circle.  There are two
such spin bundles, the trivial and non-trivial $Z_2$ bundles
over $\S^1$ (tensored with ${\bf R}$), so we have two candidate
path integrals.  One spin structure can be obtained from the other by twisting
by an element of $H^1(S^1,{\bf Z}_2)$.  What that means is that
if we are building $S^1$ by gluing together the ends of an interval
$I$, then (given a recipe for the gluing to make one spin structure)
the second can be made by multiplying all fermions by $-1$ prior
to the gluing.  

(a) If then the path integral with one of the two spin structures
computes $\Tr_{\cal H}1$, the path integral with the other spin
structure computes $\Tr_{\cal H}\Gamma$, where $\Gamma$ is an operator
in the quantum theory such that $\Gamma\psi_j=-\psi_j\Gamma$ for
all $j$.  Using your knowledge of canonical quantization and the spin
representation, what is $\Gamma$?  And by considering whether the
Pfaffian is zero or not, can you see which spin structure must go with
$\Tr_{\cal H}1$ and which computes $\Tr_{\cal H}\Gamma$?

(b)  But our main interest here is in the following generalization.
In gluing the ends of the interval $I$ to make $S^1$, we could ``twist''
by an arbitrary element $g$ of the symmetry group $SO(n)$.  Depending
on what spin structure we use, the path 
integral will compute
$\Tr_{\cal H}g$ or $\Tr_{\cal H}g\Gamma$. 

By introducing such a $g$, one builds an $SO(n)$ bundle  over  $S^1$.
 $\psi$ is now a section of $\epsilon\otimes F_g$ where $\epsilon$
is a spin bundle of $S^1$ and $F_g$ is now understood to be the
rank $n$ flat vector bundle with structure group $SO(n)$ made by twisting
by $g$.  The Lagrangian makes sense for such $\psi$'s and should
compute $\Tr_{\cal H}g$ or $\Tr_{\cal H}g\Gamma$ as the case may be.

The problem is now simply to do the computation and get the appropriate
formulas.  First of all, the fermion path integral gives
the Pfaffian of the one-dimensional Dirac operator $D/Dt$ (which
acts on sections of $\epsilon\otimes F_g$),  Mathematically, this
Pfaffian is not a number but a section of a determinant line (better
called a Pfaffian line).
The relevant Pfaffian line bundle is real, but is in fact a non-trivial
line bundle with structure group $Z_2$.
 In other words, if one considers the family of flat
bundles $F_g$ parametrized by $g\in SO(n)$, we get a family of Dirac
operators parametrized by the group $SO(n)$, and the Pfaffian line
bundle is a non-trivial real line bundle over $SO(n)$.

If one takes the double cover of $SO(n)$, replacing it by $Spin(n)$,
then the Pfaffian line becomes trivial and can be trivialized.
Once a trivialization is picked, the Pfaffian of the Dirac operator
becomes a function on $Spin(n)$.  
Of course, we get two such functions, say $f_\epsilon$, depending
on the choice of spin bundle $\epsilon$.

Compute the $f_\epsilon$'s by actually writing down the eigenvalues
of the Dirac operator and taking their product (with a regularization
analogous to what you needed in the bosonic case).  The functions
$f_\epsilon$ are obviously invariant under conjugation in $Spin(n)$
and so are linear combinations of characters of that group.  But in
fact, using your knowledge of spinors, describe precisely
what characters here appear and compare to your identification above
of the operator $\Gamma$.

\end






