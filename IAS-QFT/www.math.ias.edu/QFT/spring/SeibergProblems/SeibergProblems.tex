\input amstex
\documentstyle{qft} 
 
\pageno=37

\def\shortseriestitle{DYNAMICS OF $N=1$ SUSY THEORIES}
\def\shortauthor{N. SEIBERG}
 \leftheadtext{\shortauthor, \shortseriestitle\unskip} 


   \topmatter
   \lecturelabel{Exercises}
   \lecture{}
   \lecturename\endlecturename 
   \endtopmatter
   \rightheadtext{EXERCISES} 

\def\problemlabel{Problem }

We repeat the discussion of $SU(N_c)$ gauge theories for $SP(N_c)$
(Problems~1,3,5) and $SO(N_c)$ (Problems~2,4,6).

	\probsec Consider an $SP(N_c)$ gauge theory with $2N_f$ chiral
superfields in the fundamental ($2N_c$ dimensional) representation.
 \sec Find the moduli space of classical ground states and parametrize them
by gauge invariants.  Write explicitly the constraints when they exist.  What
is the unbroken gauge group at generic points on the moduli space?
 \sec Find the symmetries of the quantum theory.  What is the most general
invariant superpotential which can be generated?  When can we conclude
that no superpotential is generated?
 \sec What happens if the number of chiral superfields is odd?
	\endprobsec

	\problem
 Repeat the analysis of Problem~1 for an $SO(N_c)$ gauge theory with $N_f$
chiral superfields in the fundamental ($N_c$ dimensional) representation.
	\endproblem

	\probsec
 We continue our study of $SP(N_c)$ gauge theory with $2N_f$ chiral
superfields in the fundamental ($2N_c$ dimensional) representation from
Problem~1.
 \sec Find when instantons can generate the superpotential, and check the
necessary conditions we discussed in $SU(N_c)$.  Use the
results for $SU(N_c)$ to show that the coefficient is not zero in this
case.
 \sec Use~(a) to derive the superpotential for smaller values of $N_f$.  Show
that in the pure gauge $SP(N_c)$ theory $\langle \lambda \lambda \rangle
\not=0$.
 \sec Describe the moduli space of quantum vacua for one more flavor than
in~(a).  Find the most symmetric points on the moduli space, and identify the
spectrum of massless particles there.  Check the 't Hooft anomaly matching
conditions there.
 \sec Add one more flavor to~(c).  What is the moduli space of quantum
vacua?  What is the low energy field theory?  Find the most symmetric
point in the moduli space and the spectrum of massless particles
there.  Check the 't Hooft anomaly matching conditions there.
	\endprobsec

	\probsec
 We continue our study of $SO(N_c)$ gauge theory with $N_f$ chiral
superfields in the fundamental ($N_c$ dimensional) representation from
Problem~2.  For $N_c=4$ limit the analysis to equal gauge coupling constants
for the two $SU(2)$ subgroups.
 \sec Use these theories and the information in $SU(N_c)$ theories to show
that $\langle \lambda \lambda \rangle \not=0$ in the pure gauge $SO(N_c)$
theories.  Derive the superpotential for $SO(N_c)$ with $N_f \le N_c-5$.
 \sec Find the effect of gluino condensation in the unbroken gauge group for
$N_f=N_c-4$.  How many branches are there?  Find the spectrum of massless
particles in all supersymmetric vacua.
 \sec Use~(b) to find the superpotential for $N_f=N_c-3$.  What is the
spectrum of massless particles in the supersymmetric vacua?  Check the
't Hooft anomaly matching conditions.  Find the physical origin of the various
terms in the superpotential.
 \sec Use~(c) to determine the coefficients of all instanton effects we
discussed and the value of $\langle \lambda \lambda \rangle$ in the pure
gauge theories using information involving one loop computations only.
 \sec Find the moduli space of vacua for $N_f=N_c-2$.  At the generic point
on this space the spectrum includes a massless photon.  Limit your analysis
to $N_c\ge 4$.  Find the elliptic curve which determines its coupling
constant, $\tau_{eff}$.  Note that this problem has only $N=1$ supersymmetry
but $\tau_{eff}$ is still holomorphic, and therefore can be determined.
Identify the singularities on the moduli space, and find the spectrum of
massless particles there.  Check the 't Hooft anomaly matching conditions.
Show that upon adding a mass to one of the flavors the theory goes over to
the one found in~(c).  Compare with the theory with $N_c=3$, which has $N=2$
supersymmetry.
	\endprobsec


	\probsec
 We continue our study of $SP(N_c)$ gauge theory with $2N_f$ chiral
superfields in the fundamental ($2N_c$ dimensional) representation.
 \sec Find the dual gauge group.  When is it UV free and when is it IR
free?
 \sec Find the map of the chiral operators.  Show that their
dimensions satisfy the unitarity bound.
 \sec Check the 't Hooft anomaly matching conditions.
 \sec Check the flat directions.
 \sec Check the mass deformations.
	\endprobsec

	\problem 
 We continue our study of $SO(N_c)$ gauge theory with $N_f$ chiral
superfields in the fundamental ($N_c$ dimensional) representation.  For
$N_c=4$, limit the analysis to equal gauge coupling constants for the two
$SU(2)$ subgroups.  Repeat all the questions from Problem~5 for this case.
Note the special peculiarities when the dual gauge group is $SO(3)$.
	\endproblem

\enddocument
