%From: Marietta Violi <violi@math.ias.edu>
%Date: Wed, 5 Feb 1997 14:47:02 -0500

\input amstex
\magnification=1200
\documentstyle {amsppt}
\pagewidth{6.5 true in}
\pageheight{8.9 true in}
\nologo

\noindent
{\bf WITTEN'S PROBLEMS} (Spring Term), Set One --
N$^{\text{o}}$. 1\quad(solution by D. Freed)
\smallskip
\hbox to \hsize{\hrulefill}
\bigskip
\noindent
{\bf Problem:}
\medskip
(1)  The strong interactions are described to good
approximation by a theory which has a symmetry group
$G=SU(2)\times SU(2)$ spontaneously broken down to a
diagonal subgroup $H\cong SU(2)$. There are no massless
particles except the Goldstone bosons, which, to the extent
$G$ is a valid symmetry, are exactly massless.  The theory,
to good approximation, has a $G$-invariant Lagrangian $L$.
\medskip
However, $G$ is not an exact symmetry of nature.  It is
broken by perturbations that transform in the (2,2)
representation of $G$ (that is, the tensor product of the
two dimensional representations of the two $SU(2)$'s.)
Among the operators of the $G$-invariant theory are two
``multiplets'' of operators $\Cal O_i$ and $\widetilde{\Cal
O}_j,i,j=1\ldots4$, each transforming as (2,2).  The
Langrangian is, more accurately,
$$
\tilde L=L+\sum\limits^4_{i=1}\,\,\left(\varepsilon_i\Cal
O_i+\tilde\varepsilon_i\widetilde O_i\right).\tag1
$$
The $\varepsilon_i$ and $\tilde\varepsilon_j$ are, of
course, all real.  (That there are precisely two possible
multiplets $\Cal O$ and $\widetilde{\Cal O}$ of
perturbations is explained by the underlying gauge theory,
but for the present exercise we simply assume it.)
\medskip
For a generic choice of the $\varepsilon_i$ and
$\tilde\varepsilon_i$ the exact symmetry group of the model
is only $H'=U(1)$, and this is, in fact, the situation in
nature.
\medskip
Treat the effective theory to first order in $\varepsilon$
and $\tilde\varepsilon$ (which are, in fact, in nature small
and comparable to each other).
\medskip
(a)  Show that, in first order, you would expect the
degeneracy of the vacuum to be lifted, and there to appear
a unique vacuum with mass gap.  In particular, you expect
the $H'$ symmetry to be unbroken.  (Abstractly, $H'$ can be
embedded in $H$; the point that is slightly non-trivial is
that the perturbation picks a vacuum, of the unperturbed
theory, that is invariant under an $H\cong SU(2)$ that
actually does contain $H'$.  $H$ could of course have been
conjugated to a group that does not contain $H'$.)  In
particular, there is no exact symmetry that is
spontaneously broken, and no exact Goldstone boson; the
former Goldstone bosons all get mass.
\medskip
(b)  More surprisingly, show that to first order in
$\varepsilon$ and $\tilde\varepsilon$, the three Goldstone
bosons -- the one that is invariant under $H'$ and the two
that transform in a non-trivial representation of $H'$ --
all get {\it equal} masses.  (This is also observed in
nature, to high precision, the particles being respectively
the neutral and charged pions.)
\medskip
(c)  The three Goldstone bosons will not have {\it exactly}
equal masses.  In what order in $\varepsilon$ and
$\tilde\varepsilon$ will the degeneracy be lifted and how
would you describe this lifting in the effective
Lagrangian?
\bigskip
\noindent
{\bf Solution:}
\medskip
Since the only massless particles in the unbroken
$(\varepsilon=\tilde\varepsilon=0)$ theory are the
Goldstone bosons, in the infrared it flows to a
$\sigma$-model into the homogeneous space $G/H$.
Topologically $G/H$ is a 3-sphere.  Now with
$\varepsilon,\tilde\varepsilon$ nonzero we expect a
perturbation of the $\sigma$-model Langrangian, and we are
asked to contemplate first a perturbation
{\it linear} in $\varepsilon,\tilde\varepsilon$.  The
most relevant perturbation involves no derivative of the
field in the $\sigma$-model, so is simply a potential
function $V$ on $G\slash H$.  Now $G$ double covers $SO_4$,
and we are given that the operators $\Cal O,\,\,\widetilde{\Cal 
O}$ in the problem transform as vectors under $SO_4$.  Thus
if we transform $\varepsilon,\tilde\varepsilon$ as vectors
then the perturbed theory retains the full $G$ invariance.
For fixed $\varepsilon,\tilde\varepsilon$ the group $G$ is
broken to the subgroup which fixes
$\varepsilon,\tilde\varepsilon$, which for generic vectors
$\varepsilon,\tilde\varepsilon$ is isomorphic to $SO_2$.
(This is called ``$H^\prime$'' in the problem.)  
Now the potential
$V$ must be $SO_4$-invariant if we transform
$\varepsilon,\tilde\varepsilon$, and it follows easily that
it has the form
$$
V_{\varepsilon,\tilde\varepsilon}(x)=(\lambda\varepsilon+\tilde
\lambda\tilde\varepsilon)\cdot x.
$$
Here we write $x\in G\slash H\simeq S^3\subset\Bbb R^4$,
the `$\cdot$' denotes the dot product in $\Bbb R^4$, and
$\lambda,\tilde\lambda$ are real constants.   For fixed
$\varepsilon,\tilde\varepsilon$ the potential
$V_{\varepsilon,\tilde\varepsilon}$ is the restriction to
$S^3$ of a linear function in $\Bbb R^4$.  Thus it has a
unique minimum on $S^3$. This is the unique vacuum of the
perturbed theory (question (a)).  Furthermore, it is clear
that the Hessian of $V_{\varepsilon,\tilde\varepsilon}$ at
the minimum is scalar; that is, its eigenvalues -- the
masses of the pseudo Goldstone bosons -- are all equal
(question (b)).  Observe that the potential is invariant
under an $S0_3$ subgroup of $SO_4$ (which fixes the
vector
$\lambda\varepsilon+\tilde\lambda\tilde\varepsilon)$, an
accidental symmetry of the infrared theory.  (We only
expected $SO_2$ invariance.)  Finally, a perturbation
quadratic in $\varepsilon,\tilde\varepsilon$ --- for example
$V_{\varepsilon,\tilde\varepsilon}(x)=(\varepsilon\cdot x)\,(
\tilde\varepsilon\cdot x)$ --- can break this $SO_3$
invariance and so destroy the equality of the Goldstone
boson masses (question (c)).
\vfill\eject

\noindent
{\bf Spring Term QFT Problems, Set One -- N$^{\text o}$. 2}
\medskip
(a)  In this problem we work classically.  Then the minimum
occurs when the gauge fields are zero and $\phi=\phi_0$ is
a constant with $|\phi_0\vert^2=v^2$.
Locally we can identify the $SU_2\times U_1$ in the problem
with $U_2$, which then acts on $\phi$ via the two
dimensional representation. Fixing the gauge fields and
$\phi$ breaks the gauge symmetry down to the global $U_1$
gauge transformations which fix $\phi$.  (A nonzero vector
in $\Bbb C^2$ is fixed by $U_1\subset U_2$.)
\medskip
(b), (c)  We expand the action around the minimum up to
quadratic terms.  Write $A,iB$ for the $SU_2$ and $U_1$
connections and set
$$
A=\pmatrix
iA_3 & A_1+iA_2\\
-A_1+iA_2 & -iA_3
\endpmatrix.
$$
Here $A_1,A_2,A_3,B$ are all real.
Choose $\phi_0=\left(0\atop v\right)$ and write
$$
\phi=\left(0\atop v\right)+\left(\phi_1\atop\phi_2\right).
$$
Then up to quadratic terms we have
$$
\align
|F_B|^2&\sim|dB|^2,\\
|F_A|^2&\sim|dA_1|^2+|dA_2|^2+|dA_3|^2,\\
|D\phi|^2&\sim\biggl\vert\pmatrix d\phi_1+(A_1+iA_2)v\\
d\phi_2+iBv-iA_3v \endpmatrix\biggr\vert^2,\\
&\sim|d\phi_1|^2+|d\phi_2|^2+v^2A_1^2+v_2A^2_2+v^2(B-A_3)^2,\\
\frac{\lambda}{8}\,(|\phi|^2-v^2)&\sim\frac{\lambda}{8}\,|\phi
_1|^2+\frac{\lambda}{8}\,|\phi_2|^2+2v\,\,\text{Re}\,\phi_2.
\endalign
$$
Now rescale $A\rightarrow e_2A,\,\,\,B\rightarrow e_1B$ so
that after introducing a gauge fixing term (as in Problem
Set Six, N$^{\text{o}}$. 2 in the fall), the quadratic
Langrangian is (putting a coefficient 1/2 in front of
$|D\phi|^2$):
$$
\align
\qquad L_{\text{quad}}=\int\,\,d^4_x\quad&\left(\frac{1}{2}\,|d\phi_1|^2+
\frac{\lambda}{8}\,|\phi_1|^2\right)+\left(\frac{1}{2}\,|d\phi_2|^2+2v\,\text{Re}\,\phi
_2+\frac{\lambda}{8}\,|\phi_2|^2\right)\\
+&\left(\frac{1}{2}(\Delta A_1,A_1)+\frac{1}{2}\,v^2e^2_2A^2_1
\right)+\left(\frac{1}{2}\,(\Delta
A_2,A_2)+\frac{1}{2}\,v^2e^2_2A^2_2\right)\\
+&\left(\frac{1}{2}(\Delta A_3,A_3)+\frac{1}{2}(\Delta
B,B)+\frac{1}{2}\,v^2\,(e_1B-e_2A_3)^2\right).
\endalign
$$
{}From this we see that there are two massive complex scalars
$(\phi_1,\phi_2)$; two massive vectors $(A_1,A_2)$ with
mass $ve_2$; a massless vector and a massive vector with
mass $v\sqrt{e^2_1+e^2_2}$.  To prove the last statement
observe that in $L_{\text{quad}}$ the gauge fields 
$A_3,B$ have a mass
matrix with eigenvalues 0 and $v\sqrt{e^2_1+e^2_2}$.
\medskip
The Lie algebra of the unbroken $U_1$ is the subalgebra
where $A_1=A_2=0$ and $e_2A_3=e_1B$.  Thus $\phi_1$ has
charge 2, $\phi_2$ is uncharged, $A_3$ and $B$ are
uncharged, and $A_1,A_2$ have charge 2.
\medskip
\noindent
{\bf Note:}  If we had followed the normalization indicated
in the problem, we would have replaced $A,B$ by
$\frac{A}{2},\,\,\frac{B}{2}$, and then the charge of
$\phi_1,A_1,A_2$ would be 1 rather than 2.  However, we
wanted to avoid the denominators in the computation.

\bye
