%Date: Mon, 20 Apr 1998 15:58:03 -0400 (EDT)
%From: Pavel Etingof <etingof@abel.math.harvard.edu>

\input amstex
\documentstyle{amsppt}
\magnification 1200
\NoRunningHeads
\NoBlackBoxes
\document

\def\tW{\tilde W}
\def\Aut{\text{Aut}}
\def\tr{{\text{tr}}}
\def\ell{{\text{ell}}}
\def\Ad{\text{Ad}}
\def\u{\bold u}
\def\m{\frak m}
\def\O{\Cal O}
\def\tA{\tilde A}
\def\qdet{\text{qdet}}
\def\k{\kappa}
\def\RR{\Bbb R}
\def\be{\bold e}
\def\bR{\overline{R}}
\def\tR{\tilde{\Cal R}}
\def\hY{\hat Y}
\def\tDY{\widetilde{DY}(\g)}
\def\R{\Bbb R}
\def\h1{\hat{\bold 1}}
\def\hV{\hat V}
\def\deg{\text{deg}}
\def\hz{\hat \z}
\def\hV{\hat V}
\def\Uz{U_h(\g_\z)}
\def\Uzi{U_h(\g_{\z,\infty})}
\def\Uhz{U_h(\g_{\hz_i})}
\def\Uhzi{U_h(\g_{\hz_i,\infty})}
\def\tUz{U_h(\tg_\z)}
\def\tUzi{U_h(\tg_{\z,\infty})}
\def\tUhz{U_h(\tg_{\hz_i})}
\def\tUhzi{U_h(\tg_{\hz_i,\infty})}
\def\hUz{U_h(\hg_\z)}
\def\hUzi{U_h(\hg_{\z,\infty})}
\def\Uoz{U_h(\g^0_\z)}
\def\Uozi{U_h(\g^0_{\z,\infty})}
\def\Uohz{U_h(\g^0_{\hz_i})}
\def\Uohzi{U_h(\g^0_{\hz_i,\infty})}
\def\tUoz{U_h(\tg^0_\z)}
\def\tUozi{U_h(\tg^0_{\z,\infty})}
\def\tUohz{U_h(\tg^0_{\hz_i})}
\def\tUohzi{U_h(\tg^0_{\hz_i,\infty})}
\def\hUoz{U_h(\hg^0_\z)}
\def\hUozi{U_h(\hg^0_{\z,\infty})}
\def\hg{\hat\g}
\def\tg{\tilde\g}
\def\Ind{\text{Ind}}
\def\pF{F^{\prime}}
\def\hR{\hat R}
\def\tF{\tilde F}
\def\tg{\tilde \g}
\def\tG{\tilde G}
\def\hF{\hat F}
\def\bg{\overline{\g}}
\def\bG{\overline{G}}
\def\Spec{\text{Spec}}
\def\tlo{\hat\otimes}
\def\hgr{\hat Gr}
\def\tio{\tilde\otimes}
\def\ho{\hat\otimes}
\def\ad{\text{ad}}
\def\Hom{\text{Hom}}
\def\hh{\hat\h}
\def\a{\frak a}
\def\t{\hat t}
\def\Ua{U_q(\tilde\g)}
\def\U2{{\Ua}_2}
\def\g{\frak g}
\def\n{\frak n}
\def\hh{\frak h}
\def\sltwo{\frak s\frak l _2 }
\def\Z{\Bbb Z}
\def\C{\Bbb C}
\def\d{\partial}
\def\i{\text{i}}
\def\ghat{\hat\frak g}
\def\gtwisted{\hat{\frak g}_{\gamma}}
\def\gtilde{\tilde{\frak g}_{\gamma}}
\def\Tr{\text{\rm Tr}}
\def\l{\lambda}
\def\I{I_{\l,\nu,-g}(V)}
\def\z{\bold z}
\def\Id{\text{Id}}
\def\<{\langle}
\def\>{\rangle}
\def\o{\otimes}
\def\e{\varepsilon}
\def\RE{\text{Re}}
\def\Ug{U_q({\frak g})}
\def\Id{\text{Id}}
\def\End{\text{End}}
\def\gg{\tilde\g}
\def\b{\frak b}
\def\S{\Cal S}
\def\L{\Lambda}

\topmatter
\title Lecture II-16: BRST quantization of gauge theories
\endtitle
\author {\rm {\bf Edward Witten} }\endauthor
\endtopmatter

\centerline{Notes by Pavel Etingof and David Kazhdan}

\vskip .1in

In this lecture we will discuss quantization of gauge theory by using BRST 
cohomology. This approach is an improvement of the original Faddeev-Popov 
approach. An advantage of the BRST approach as opposed to the Faddeev-Popov 
method is that BRST makes explicit the independence of quantization of the 
choice of the gauge fixing procedure. 

A similar approach can be used in gravity (see D'Hoker's lectures). 

{\bf 16.1. The general setup.}
 
We start with a general setup, and then consider examples. 
In the general setup, we have a compact gauge group $G$
with Lie algebra $\g$, and 
the group $\hat G$, which is the group of gauge transformations of a 
principal $G$-bundle $E$ over a spacetime $M$.  
Formally, we want to compute the path integral 
$$
\frac{1}{Vol(\hat G)}\int DAD\phi e^{-L(A,\phi)},\tag 16.1
$$
where $L(A,\phi)$ is a gauge invariant Lagrangian
with a gauge field $A$ and matter fields $\phi$. 

The difficulty with a perturbative treatment 
of this path integral is that its kinetic term 
$F_A^2$ for the gauge field is degenerate along orbits of $\hat G$.  
One way to deal with this difficulty 
is to replace the integrand in (16.1) by some expression  
that integrates to $1$ on orbits of $\hat G$ -- then 
(16.1) would equal to the integral of this expression (at least if 
everything were finite dimensional).
For example, this expression could be the delta-function of some gauge, 
i.e. of some submanifold 
in the space of connections and matter fields which is a cross-section for the 
$\hat G$-action (this procedure is called gauge fixing). 
As we know (see Kazhdan's lectures
on gauge theory and Faddeev's lectures), 
this introduces a determinant under the integral 
(the Faddeev-Popov determinant). The determinant is a nonlocal expression, so 
in order to work only with local expressions, one should replace 
this determinant with a Gaussian integral over 
the space of fields times two copies of the odd space of sections of the 
coadjoint bundle of $E$. Thus, one has to introduce additional fermionic
fields $c,\bar c$ with values in $ad(E)$.  
These fermions are called ghosts, 
since they do not correspond to any physical particles and violate 
spin-statistics. After the introduction of ghosts the path integral can be 
treated as usual, e.g. by perturbation theory techniques. 

{\bf 16.2. The BRST differential.} 

Of course, any gauge fixing procedure by definition destroys gauge 
invariance. Therefore, in order to obtain a sensible quantization, we must
make sure that in the final result the gauge symmetry is restored. 
In particular, we must explain what replaces the gauge symmetry 
in the ghost setting of the previous section. 

It turns out that what replaces the gauge symmetry is a certain 
odd derivation of the algebra of local functionals, which 
we will now construct. 

We will first consider the classical theory. 
Let us look what fields our theory has after introduction 
of ghosts. The basic fields are the connection $A$, 
the matter fields $\phi$, 
and the ghosts $c,\bar c$, which are sections of the adjoint bundle to $E$.
We will add an auxiliary scalar bosonic field $h$, whose significance will be 
seen below. 
 
Let $R$ be the algebra of local functionals. 
We want to define an odd derivation 
$\delta:R\to R$ such that $\delta^2=0$
(the BRST differential). 

Recall that the algebra of local functionals $R$ 
is the quotient $\tilde R/I$, where $\tilde R$ 
is the algebra of local expressions in the fields and $I$ is the 
differential ideal generated by field equations.

We first define a derivation $\delta:\tilde R\to\tilde R$,  
and then make sure that the field equations are respected. 

Define $\delta$ on generators by 
$$
\delta c=\frac{1}{2}[c,c], \delta\phi=\delta_c\phi, \delta A=-d_Ac,
\delta\bar c=h,\delta h=0,\tag 16.2
$$
where $\delta_c\phi$ means the variation of $\phi$ along the infinitesimal 
gauge transformation $c$. It is easy to check that $\delta^2=0$. 

Recall from Kazhdan's and D'Hoker's lectures
that the Lagrangian with ghosts for our theory is
$$
\tilde L=L(A,\phi)-
\delta (\bar c(h/2 +\Lambda))=
L(A,\phi)-
(h\L+\frac{h^2}{2}-\bar c\delta\L).\tag 16.3
$$
where $\L=\L(A,\phi)$ is a non-gauge invariant local expression 
(the gauge fixing function).
This Lagrangian is of course equivalent to 
$$
\hat L=L(A,\phi)+(\frac{1}{2}\L^2+\bar c\delta\L).
$$
by eliminating $h$. 

We want the Lagrangian to be invariant
under the derivation $\delta$. Since $L(A,\phi)$ is 
already invariant (because it is gauge invariant), 
it is enough to check that the expression 
$h\L+\frac{h^2}{2}-\bar c\delta\L$ is closed under $\delta$, 
which follows from $\delta^2=0$. 

Since $\delta$ preserves the Lagrangian,
 it preserves the set of its critical points
and hence indeed defines a derivation of the algebra of local functionals. 

As an example consider the case when $E$ is the trivial bundle
(and there is no matter fields). 
One of the possible gauge fixing conditions is the Feynman gauge condition 
$d^*A=0$. So we take $\L=d^*A$. Then one has 
$\delta\L=-d^*d_Ac$, so after elimination of $h$ the Lagrangian 
for pure gauge theory is
$$
L=\frac{1}{4e^2}\int F^2+((d^*A)^2-\bar cd^*d_Ac).\tag 16.4
$$
We see that this Lagrangian is nondegenerate, so one can do perturbation 
theory with it as usual. The derivation $\delta$ on $R$ in this case is 
defined by 
$$
\delta c=\frac{1}{2}[c,c], \delta A=-d_Ac, \delta\bar c=-d^*A.\tag 16.5
$$

{\bf 16.3. The properties of the BRST derivation.}

Thus, we have constructed a derivation $\delta$. The main
properties of $\delta$ are:

1. $\delta^2=0$. 

2. $\delta$ is defined on $\tilde R$ 
apriori, without the use of the Lagrangian $L(A,\phi)$
and the gauge fixing term $\L$. 
It preserves the Lagrangian with ghosts $\tilde L$
and therefore descends to $R$. 

Now let us turn to quantum theory. In this case 
local functionals are replaced by local operators. It can be shown 
that there exists a renormalization procedure under which 
$\delta$ can be defined as above, and properties 1 and 2 hold. 
This is discussed below. 

However, in order to use the BRST method for quantization, 
we will need another, purely quantum, property of $\delta$. 
Namely, denote by $\Cal L_{eff}$ the effective Lagrangian, 
i.e. the Lagrangian for which the classical theory is equivalent 
to the quantum theory for $\Cal L$. This Lagrangian 
is of course nonlocal. 
The third property of 
$\delta$ that we need is

3. $\delta$ preserves the effective action
$\Cal L_{eff}$. That is, $\delta \Cal L_{eff}=0$. 

This property may fail, and if it fails then one says that the theory 
is anomalous. 

{\bf Remark.} Although $\Cal L_{eff}$ is nonlocal, it can be shown 
that $\delta\Cal L_{eff}$ is always the integral of a local 
expression. It can be shown that 
the obstruction to making $\delta\Cal L_{eff}$ zero 
by adding an auxiliary term to the Lagrangian (in a way that does not 
change the physics) is given by a 1-loop calculation. 
Thus, anomalies arise in the 1-loop order of perturbation 
theory, and don't have higher order corrections. 
We will see this at the end of the lecture. 

{\bf 16.4. Operators in gauge theory and BRST cohomology.} 

Assume properties 1-3 hold. Consider the path integral $Z$ given by 
Lagrangian $\tilde L$:  
$$
Z=\int DADcD\bar c Dh D\phi e^{-\tilde L}\tag 16.6
$$
(possibly with some gauge-invariant insertions). 
Properties 1-3 imply that $Z$ is independent 
on the gauge fixing condition $\L$. Indeed, for any local expression $X$ 
we have 
$$
\int 
DADcD\bar c DhD\phi e^{-\tilde L}\delta X=\delta \int DADcD\bar c DhD\phi
e^{-\tilde L}X=0,
\tag 16.7
$$
which implies the independence of $Z$ on $\L$.

{\bf Remark.} In (16.7), we used that $\delta$ preserves the measure 
of integration $DADcD\bar c DhD\phi$. 
It is easy to see that this is equivalent to Property 3 (absence of 
anomalies).
 
The statement that $Z$ is independent of $\L$ holds for 
operators (insertions) 
which are annihilated by $\delta$; for example, for any gauge-invariant 
insertions into $Z$, depending only on $A,\phi$.   
On the other hand, if $\O=\delta\O'$ then the integral of $\O$ is zero 
by (16.7). Thus, the space of ``physical'' 
quantum operators in our theory is the cohomology 
of $\delta$. This cohomology is called the BRST cohomology. 

The BRST cohomology comes with a natural $\Z$-grading. Namely, 
we have a grading in the space of local operators, 
in which gauge and matter fields have degree $0$, $c$ has degree 
$1$ and $\bar c$ degree $-1$. This degree is called the ghost number. 
It is easy to see that $\delta$ raises 
the degree by 1. This allows to introduce a grading in cohomology:
we denote by $H_\delta^q$ the cohomology in degree $q$. 

The  properties of this cohomology, 
which usually hold in this situation are: 

1. $H_\delta^q$ vanishes for $q<0$.

2. $H_\delta^0$ is the space of gauge invariant local operators
depending only on $A$ and $\phi$.

This shows that $\delta$ plays the role of the gauge symmetry 
which was broken when ghosts were introduced. Thus, we have established 
a setting for gauge theory which works well in perturbation theory 
and in which the gauge symmetry does not die but rather 
appears in the form of $\delta$. 

{\bf 16.5. Renormalization and BRST differential.}

Now let us discuss the renormalizability and renormalization group equation 
in the BRST approach. We will restrict ourselves to 4 dimensions and pure 
nonabelian gauge theory. 

Recall that in order for a theory to be renormalizable,
all interactions have to have nonnegative dimension. To find out whether it is 
so for the Lagrangian with ghosts, we will compute the dimensions of fields
(assuming that $\delta$ preserves the scaling dimensions). It is easy to see 
that the dimensions are as follows: 
$[c]=0,[\bar c]=2,[A]=1$. This shows that all interactions 
in the Lagrangian with ghosts are renormalizable. 

{\bf Remark.} In this theory, dimensions of $c$ and $\bar c$ are not uniquely 
determined; the only thing that is determined canonically is $[\bar cc]$,
which equals $2$. This does not lead to a contradiction, since all operators 
of nonzero ghost number have zero expectation value, and so their
scaling dimension has no intrinsic
meaning. This is why we needed to make an additional 
assumption that $\delta$ preserves dimensions to fix precise values 
of the dimensions. If we had assumed that $\delta$ raises dimension by 
$k$ we would get a different answer, which would be equally good for our 
purposes. 

Now let us look for critical couplings which will be renormalized. 
In the setting without ghosts, the usual thing to do is to 
write down renormalizable 
(non-gauge invariant) operators of dimension 4, which correspond to 
critical couplings: $[A,A]^2, [A,A]dA,(dA)^2$, and then argue 
that there is only one gauge invariant combination of these operators, 
so that the only coupling which is to be 
renormalized is the charge $e$. However, in the setting with ghosts, 
we also have to include operators of degree 4 
involving ghosts: $\bar ccA^2,\bar cd^*dc,...$. The gauge invariance condition 
is now replaced by the condition that $\delta$ is a symmetry, so we need 
to renormalize only delta-invariant interactions. This cuts down the number 
of operators to be renormalized, but still leaves us with two 
renormalizable couplings: the charge $e$ and the gauge fixing parameter $a$, 
corresponding to the scaling of the gauge fixing term 
$\frac{\L^2}{2}+\bar c\delta\L$. 
Thus the renormalization group vector field looks like
$$
W=\mu\frac{\d}{\d\mu}+\beta(e)\frac{\d}{\d e}+\tilde\beta(e,a)\frac{\d}{\d a}
.\tag 16.7
$$
Here $\beta$ is the beta-function of the theory, and $\tilde\beta$ 
is the ghost beta-function. The beta-function of the theory 
 depends only on $e$ and is physically meaningful; for example, 
the negativity of its 
leading term insures asymptotic freedom. However, 
the ghost beta-function $\tilde\beta$ has no physical meaning: 
it only matters for renormalization of operators 
and correlators containing $c$ and $\bar c$, which don't 
make sense physically. 

{\bf 16.6. The Hamiltonian approach.}

So far we have considered BRST method from the Lagrangian point of view. 
Now let us consider the connection of the BRST method with the Hamiltonian 
formalism. 

Since ghosts violate spin-statistics (being scalar fermions), 
the ``Hilbert space'' of the theory with ghosts cannot be an actual Hilbert 
space. Namely, it is possible to construct a certain space 
$\tilde\Cal H$ with a Hermitian 
form, which is analogous to the Hilbert space in actual physical theories, 
but the form will not be positive definite. However, on this space 
we have local quantum operators, 
obtained by quantization of classical operators
in the usual way. In particular, we have 
the global charges -- the Hamiltonian $H$ as well as the BRST charge $Q$, 
obtained from ghosts as explained in D'Hokers lecture. 
We also have a grading of $\tilde\Cal H$ by ghost number, obtained 
naturally from the quantization procedure. 

The operator $Q$ has the property $[Q,\O]=\delta\O$ 
for any operator $\O$ (not necessarily $\delta$-closed)
in the theory with ghosts. Also, $Q\Omega=0$, 
where $\Omega$ is the vacuum, and the ghost number of $Q$ is $1$. 

The operator $Q$ has properties 
analogous to those of $\delta$: 

1. $Q^2=0$. This can be confirmed by a direct computation. 

2. $Q$ is defined apriori, without the use of $L$ and $\Lambda$
(by an explicit formula as in D'Hoker's lectures). 
In particular, if $\tilde\Cal H$ is an irreducible representation 
of the operator algebra, then $Q$ is completely determined by $\delta$
and the properties $[Q,\O]=\delta\O$, $Q\Omega=0$. 

3. If there is no anomalies, the element $Q$ commutes 
with the Hamiltonian and with all gauge invariant local operators 
which involve no ghosts. 

{\bf Remark.} As in the Lagrangian setting, here the explicit expression 
for $Q$ is independent on $\L$ only if one uses the operator $h$ 
corresponding to the auxiliary field in the Lagrangian. This operator 
can be 
expressed via other operators in the theory, in a way which depends on $\L$:
$h=-\L$. If one makes this substitution, the obtained formula 
for $Q$ will involve $\L$. Thus, property 2 should be understood as follows: 
there exists a formula for $Q$ in terms of the fields (including $h$!)
which is independent on $\Cal L$ and $\L$ but depends only on field 
configuration. 

Let $H_Q^q$ be the cohomology of $Q$ on $\tilde\Cal H$, graded by ghost 
number.  

The  properties of $Q$ which usually hold are 

1. $H_Q^q$ vanishes if $q<0$.   

2. The Hermitian form is degenerate on 
the kernel of $Q$ in $\tilde\Cal H^0$ (operators of ghost number zero); 
the kernel of this form is the image of $Q$. The induced 
form on $H_Q^0$ is positive definite. 
   
The space $H_Q^0$ plays the role of the physical Hilbert space 
of the theory, so we denote it by $H_{phys}$. 

In the space $H_{phys}$, we have an action of the Hamiltonian $H$ and 
``physical'' local operators $\O\in H_\delta^0$. These operators no longer 
involve ghosts and correspond to actual observables of the theory. 

Let us now compare the BRST and the ``traditional'' approaches to 
quantization of gauge theory. For simplicity, we consider pure gauge theory. 
Traditionally, a scheme of quantization 
would be as follows. Suppose that the space part of the spacetime is compact. 
In this case we have seen that classically the space of solutions 
to the equations of motion can be realized as $T^*\Cal A$, where 
$\Cal A$ is the space of connections on a space cycle
modulo gauge transformations. Therefore we 
would define the Hilbert space as $L^2(\Cal A)$ (with respect 
to some measure). We call this Hilbert space 
the traditional Hilbert space. 

We claim that these approaches give the same result, i.e. $H_{phys}$ 
is isomorphic to $H_{trad}$ as a representation of the operator algebra. 

First of all, $H_{phys}$ does not depend on the gauge fixing term 
$\L$, and the Hamiltonian and the quantum operators in $H_{phys}$ don't 
depend on it either. This follows from the fact that when $\L$ is varied, 
operators in the pseudo-Hilbert space $\tilde\Cal H$ are changed 
by adding a $\delta$-exact expression, so their action on $\delta$-closed 
vectors is changed by a $\delta$-exact expression. 

To identify $H_{phys}$ with $H_{trad}$ we can use a convenient 
gauge fixing term $\L$. It is enough to do it for one such term, but 
we will do it for two -- just for fun. 

Set $\L=ud^*A+vA_0$, where $A_0$ is a time component 
of the connection (this uses the splitting 
of spacetime into space and time). Then we get a sensible  
theory unless both $u=0$ and $v=0$. Even $u=0,v\ne 0$ gives a nice theory -- 
this gauge fixing term is called ``temporal gauge''.

We first consider the case $u=0$. Then the Lagrangian is
$$
\hat L=\int (\frac{1}{4e^2}F^2+\frac{1}{2}v^2A_0^2-
v\bar c\frac{D c}{D t}), 
$$
where $D$ denotes covariant derivative. 
Replacing $\bar c$ with $-v\bar c$, and tending $v$ to infinity
(using the fact that nothing depends on $v$), 
we see that the path integral is localized 
to the hyperplane $A_0=0$, and in the limit we  get a Lagrangian 
$$
\hat L=\int (\frac{1}{4e^2}F^2+
\bar c\frac{d c}{d t}), 
\tag 16.8
$$
 
Since $A_0$ is now zero, we get usual quantum mechanics where 
dynamical variables are a spacial connection, its time derivative, 
and the ghosts. Thus, 
$\tilde\Cal H$ has the form $\O(\tilde\Cal A)\o \L\hat \g^*$, 
where $\tilde\Cal A$ is the space of connections on the space cycle, 
$\O(\tilde\Cal A)$ is the space of functions on $\tilde\Cal A$, 
$\tilde\g$ is the Lie algebra of the group of gauge transformations
on the space cycle, 
and $\L\tilde\g^*$ is the space of functions of $c\in\Pi\tilde\g$.
Moreover, the Hamiltonian for 
the ghosts vanishes since there is no nontrivial evolution 
on the space of classical solutions (the Euler-Lagrange
equations for ghosts are simply $\frac{d c}{d t}=\frac{d \bar c}{d t}=0$). 
Thus the Hamiltonian in our theory is the usual gauge theory Hamiltonian 
$$
H=\frac{1}{2e^2}\int d^3x F_A^2+\frac{e^2}{2}\nabla_A^2, A\in \tilde\Cal A\tag 16.9
$$
acting on the first component of the tensor product. 

It is easy to see that the space $\tilde\Cal H$ with the grading 
by ghost number is nothing but the space of the standard complex of the Lie 
algebra $\tilde\g$ with coefficients in the module $\O(\tilde\Cal A)$. 
Moreover, from the explicit formula for $Q$ 
one gets that in this case $Q$ is 
exactly the differential in the standard complex. Thus, 
the physical Hilbert space $H_{phys}$ which is by definition 
the 0-th cohomology of $Q$, is the 0-th the cohomology of $\tilde\g$ 
with coefficients in $\O(\tilde\Cal A)$. 
This is just the space of $\tilde\g$-invariants 
in $\O(\tilde\Cal A)$, i.e. the space of functions on 
$\tilde\Cal A/G$, which is 
by definition the traditional Hilbert space $H_{trad}$. 

This shows that BRST cohomology is an infinite dimensional generalization of 
Lie algebra cohomology.

Now consider another gauge obtained by setting $v=0,u=1$: $\L=d^*A$. 
Let us try to see the isomorphism between $H_{phys}$ and $H_{trad}$ 
using this gauge. We have, 
$$
\hat L=\frac{1}{4e^2}\int F^2+(\frac{1}{2}(d^*A)^2-\bar cD^*d_Ac).\tag 16.10
$$
In this case the equations of motion for ghosts are nontrivial 
and of second order, so the Hilbert space 
$\tilde\Cal H$ consists of functions of $A_s,c,\bar c,A_0$, where 
$A_s$ is a connection on the space cycle. In this case, one finds 
$$
Q=Q_{\text{Lie algebra cohomology}}+Q',\tag 16.11
$$
where $Q'=\int \pi_{\bar c}\pi_{A_0}$, 
and $\pi_{\bar c}$, $\pi_{A_0}$ are the momentum operators for 
$\bar c$, $A_0$. 

It is easy to check directly that the two summands in (16.11) anticommute, 
and that $(Q')^2=0$. It can also be checked that $Q'$ is acyclic except 
in 0-th degree, where it has a 1-dimensional cohomology. Thus, by Kunneth 
formula, we again get $H_{phys}=H_{trad}$. 

{\bf 16.7. Anomalies.}

Now let us recall conditions 1,2,3 which were necessary for the BRST 
construction, and analyze when they are satisfied. These conditions 
are

1. Lagrangian: $\delta^2=0$.

   Hamiltonian: $Q^2=0$.   

2. Lagrangian: $\delta$ is independent on $\L$ and $\tilde\Cal L$.

   Hamiltonian: $Q$ is independent of $\L,\tilde\Cal L$.   

3. Lagrangian: $\delta \Cal L_{eff}=0$.

   Hamiltonian: $[Q,H]=0$. 

As we mentioned, properties 1 and 2 can always be attained. 

However, as we also mentioned, Property 3 may fail
if anomalies are present. So let us consider anomalies more closely. 

Consider a 4-dimensional gauge theory with a chiral spinor $\psi$
with values in a complex representation $\rho$ of the gauge group $G$ and 
antichiral spinor $\bar \psi$ with values in the dual representation 
$\bar\rho$. The basic gauge-invariant Lagrangian for such fields is
$$
\Cal L=\frac{1}{4e^2}\int F^2+\int \bar\psi D_A\psi.\tag 16.12
$$
In quantum theory we are interested in the path integral $\int e^{-\Cal L}$. 
Integrating out $\psi$ in this path integral, we get 
$$
\int \text{det}(D_A)e^{-L(A)}DA, \tag 16.13
$$
where $D_A:\Gamma(S_+)\to \Gamma(S_-)$ is the covariant 
Dirac operator and $L(A)$ the Lagrangian of the pure gauge theory. 

Integral (16.13) may not have a gauge invariant regularization. What is 
worse, it may not even have a non-gauge-invariant  
regularization for which the gauge invariance
is restored as the cutoff goes to infinity. In this case the gauge theory  
we are considering does not make sense quantum mechanically, even 
in perturbation theory, because gauge symmetry cannot be restored. 
This phenomenon is called an anomaly. 

A geometric reason for an anomaly is that although 
the operator $D_A$ is gauge invariant, 
its determinant $\text{det}(D_A)$, in general, fails to be 
gauge invariant. In other words, this determinant is  
not a function on the space of gauge classes of connections 
but rather a section of some line bundle over this 
space, called the determinant line bundle; this bundle 
comes with a canonical connection. If this canonical 
connection does not trivialize 
the bundle, this ``function'' cannot be sensibly integrated. 

It is useful to distinguish two types of anomalies. 

1. Local anomaly. The canonical connection has a nonzero curvature. 
In this case for suitable spacetime manifolds this curvature  
may represent a nontrivial second cohomology class, so that the determinant 
bundle is not trivial topologically.  

2. Global anomaly. The canonical connection is flat but 
has a nontrivial monodromy (and possibly the bundle is not trivial).

Thus, both local and global anomalies can produce topological anomalies, 
but only the first one can be seen in perturbation theory (by computing 
of the curvature). 

Here we will consider only local anomalies. 

{\bf Remark 1.} To analyze when we can expect local anomalies,
one may consider the situation from the topological point of view.
We assume that our spacetime $M$ is compact and orientable (e.g. $S^d$),
with a specified point $\infty$, and will consider bundles, connections, and 
gauge transformations which are trivial at infinity.      
In this case the space of  gauge classes of connections can be regarded as 
a classifying space $B\hat G$ for the group of gauge transformations $\hat G$. 
Nontrivial line bundles on $B\hat G$ are classified by 
$H^2(B\hat G)$. 
  
Now, if $M$ is compact and orientable, we have the transgression 
 map $\tau:H_2(B\hat G)\to H_{d+2}(BG)$ defined 
as follows: given a two-dimensional homology class, pick 
a surface $S$ in $B\hat G$ which represents it, and take the 
corresponding principal $\hat G$-bundle on $S$. Its transition functions 
can be considered as transition functions of a $G$-bundle on 
the 6-dimensional manifold $S\times M$, which defines 
an element $\tau([S])$ of $H_{d+2}(BG)$. Consider the dual map 
$\tau^*: H^{d+2}(BG)\to H^2(B\hat G)$. It can be shown that the Chern class 
of our line bundle is $\tau^*(C)$, where $C$ is a fixed 
$d+2$-dimensional characteristic class which does not depend on $M$, and 
is computed locally from the curvature. This class is exactly the local 
anomaly.  

Thus for $d=4$ local anomalies live in 
$H^6(BG)$, or $(S^3\g)^\g$, where $\g$ is the Lie algebra of $G$. 

For example, in the standard model the gauge group is 
$SU(3)\times SU(2)\times U(1)$, and thus the space of anomalies
$(S^3\g)^\g$ is 4-dimensional: it equals to the sum 
of four subspaces $S_{inv}^3(su(3)),S_{inv}^3(u(1)),
S_{inv}^2(su(2))\o u(1), S_{inv}^2(su(3))\o u(1)$, which are 1-dimensional
(here ``inv'' denotes that we are taking invariant symmetric polynomials).    

This discussion illustrates why anomalies don't arise in the case when all 
fermions are in a real representation of the gauge group. Indeed, 
in this case, the determinant bundle is real, and thus its 
Chern class must be zero. 
 
{\bf Remark 2.} Although the local anomaly can be considered from 
the above topological point of view, one should remember that it has a purely 
local nature, and has nothing to do with the macrostructure of the spacetime. 
If there is a local anomaly, the quantum theory will not make sense 
on any spacetime, even on $\R^d$. The problem is that    
even if the determinant bundle is topologically 
trivial, it will not have a flat connection
defined in a local way: otherwise  
this flat connection would have been good for any simply connected 
spacetime, and 
no topological anomaly would arise. Thus, 
path integral (16.13) is not sensible even on $\R^d$. 

{\bf Remark 3.} In the standard model, the gauge group is 
$SU(3)\times SU(2)\times U(1)$. In particular,
there is a possibility for local anomalies, and they do appear in reality.
However, one can check that the anomalies 
coming from the different matter fields of the standard model 
miraculously cancel, in all four components of the space of anomalies. 
An explanation of this 
is that the representation of the gauge group in the standard model 
extends (after adding some insignificant summands) to a spinor representation 
of $Spin_{10}$, for which $H^6(BG)$ vanishes. 

Let us show how to analyse anomaly in perturbation theory.
Our goal is to explain why, after possibly enlarging the space of fields,
properties 1. and 2. of section 16.3 can always be assumed to hold
(that is, $\delta^2=0$ and $\delta$ is defined independently of the choice
of a particular Lagrangian) but one cannot assume that the effective Lagrangian
is delta-invariant. 

 First, let us just try to make sense of integral
(16.3) perturbatively.
When we write down Feynman diagrams, we will find divergences 
in the 1-loop order which we cannot remove in a gauge invariant fashion. 
To fix the 1-loop order, we will regularize the path integral 
by adding another, very heavy matter field $\chi$ such that its determinant 
bundle is inverse to that for $\psi$. In favorable cases, 
our original theory should be recovered from this theory in the limit 
when the mass $m$ of $\chi$ goes to infinity. 
In other cases, the procedure will exhibit why there is an anomaly.

To satisfy this condition, the matter field $\chi$ can be taken to 
be a bosonic field $(\chi_+,\chi_-)$
with values in $(S_+\oplus S_-)\o \rho$. In this case  
the complex conjugate field $\bar\chi$ is with values in 
$(S_+\oplus S_-)\otimes \bar\rho$, where $S_+,S_-$ are the spin 
representations of the Poincare
(recall that both $S_+$ and $S_-$ are self-dual and self-complex-conjugate
in Euclidean signature). 
It is of course needless to say that these fields violate spin-statistics and 
therefore, like ghosts, don't make physical sense. 

The natural 
Lagrangian term for the fields $\chi_\pm$  
would be 
$$
L'(A,\chi_\pm)=\int d^4x ((\bar\chi_+, D_A\chi_-)+(\bar\chi_-,D_A\chi_+)+
m(\bar\chi_+,\chi_+)+m(\bar\chi_-,\chi_-))  \tag 16.14
$$
(Here the Dirac operator is skewselfadjoint). 

{\bf Remark.} The $\chi$'s are called
Pauli-Villars regulator fields. 

If we add
expression (16.14) to the Lagrangian, we will get the squared absloute 
value of the determinant rather than the determinant itself, and will not fix 
the anomaly. Thus, we modify (16.14) in a way that breaks the gauge 
invariance:  we let $A_0$ be a fixed connection and set 
$$
L''(A,\chi_\pm)=\int d^4x ((\bar\chi_+,D_A\chi_-)+(\bar\chi_- ,D_{A_0}\chi_+)+
m(\bar\chi_+,\chi_+)+m(\bar\chi_-,\chi_-))  \tag 16.15
$$
Now consider the theory with the Lagrangian $\Cal L+L''$. 
Integrating out the $\chi$ fields, we will get a factor 
$\text{det}(D_AD_{A_0}-m^2)^{-1}$. For $m=0$ this factor
is gauge invariant up to a multiplicative constant, and cancels 
the determinant in the numerator, which is caused by anomaly. 
This shows that in this theory, we don't have a topological anomaly 
for any finite $m$ (i.e. the appropriate determinant bundle is trivial). 
However, for $m>0$, the gauge invariance fails. So we have to 
study the limit $m\to \infty$ (which is supposed to recover our original 
theory) and see whether the gauge symmetry reappears. 

Differentiating the determinant ratio
$\text{det}(D_AD_{A_0}-m^2)/\text{det}(D_A)$
in the direction of a gauge trasformation $t\in \hat \g$, 
we obtain (using the path integral interpretation)
that it is equal to $m\<\int [(\bar\chi_+,t\chi_+)+(\bar\chi_-,t\chi_-)]\>$,
where $\<X\>$ denotes the expectation value of $X$. 
This expectation value has a decomposition in powers of $1/m$.
 
To see whether the failure of gauge invariance 
persists for $m\to\infty$, let us consider 
the two-point function of the curvature operator $F$. 
It is easy to see that the leading contribution (in $1/m$) to the derivative 
of this function in the direction of $t$ is from the 1-loop diagram
with a $\chi$ loop having the t operator inside and two outgoing $F$-edges. 
This contribution is of the 0-th order in $1/m$, and has the form 
$\sum d_{abc}t^aF^bF^c$, where $d_{abc}$ is some tensor. So if $d_{abc}\ne 0$, 
the gauge-noninvariance remains in the limit. 

{\bf Remark.} In case the original fermions were in a real representation,
$d_{abc}$ is zero and 
the regularization in (16.14) is completely satisfactory.
The problem arises when the original representation is complex.
Then a regularization as in (16.14) doesn't work unless one gives up
gauge invariance. 

{\bf Remark.} When $d_{abc}\ne 0$, one can choose a regularization scheme 
to remove all loop contributions to the non-gauge invariance except 1 loop. 

Now let us consider anomalies from the BRST veiwpoint. 
If a local anomaly is present, 
we will have $U=\delta \Cal L_{eff}(A)\ne 0$
(here $\Cal L_{eff}(A)$ is the effective Lagrangian for $A$, with the 
ghosts integrated out). However, 
since the anomaly is local
$U$ must be the integral of a local 
expression of $A$ and $c$ which is linear in $c$. It is also clear that 
$\delta U=0$. Furthermore, one can show that $U$ involves only 
$A$ and its first derivatives (and no matter fields).  

On the other hand, if 
$U$ is $\delta N$ where $N$ is the integral of 
some local expression of $A$ then we can arrange 
$\delta \Cal L_{eff}=0$ by redefining the 
Lagrangian as $\Cal L\to\Cal L+N$. Thus,   
anomalies lie in the cohomology of $\delta$ on local functionals 
of degree $\le 1$ of $A$ and $c$ (linear in $c$) 
modulo complete derivatives. 

Now let us show that such cocycles are in fact related to 
invariant symmetric tensors on the Lie algebra $\g$
(or equivalently, the cohomology of $\g$). 

Let $C$ be a $G$-invariant element in $S^{n+1}\g$. 
To this element there corresponds 
a $2n+2$-dimensional 
characteristic class of $G$-bundles, namely $C(F^{n+1})$, where $F$ 
is the curvature. The Chern-Simons form $CS_C(A)$ corresponding to $C$ is 
the local $2n+1$ form such that $\frac{\tilde\delta C(F^{n+1})}
{\tilde\delta A}=
CS_C(A)\wedge \tilde\delta A$ modulo differentials of local forms
(here $\tilde\delta$ denotes the variation to distinguish from the 
BRST differential $\delta$). 

The main property of the Chern-Simons form is the following. 
Although this form is not gauge invariant, its Lie derivative along 
an infinitesimal gauge transformation is a differential of a local form. 

Now let $M^{2n}$ be our spacetime. Let $A$ be a connection on $M^{2n}$. 
We want to define a functional of the form $U(A)=\int W(A)$, where 
$W(A)$ is a $2n$-form on $M^{2n}$ which is local in $A$ but not gauge 
invariant, and such that $\delta W=0$. 
 
Let $X^{2n+1}$ is a smooth manifold with 
boundary equal to $M^{2n}$. Choose an extension
of the connection $A$ to $X^{2n+1}$
in any way (for simplicity 
we assume that there is no topological obstruction to the choices
of $X$ and the extension of $A$; this assumption is in fact inessential). Now  
set $V(A)=\int_{X} CS_C(A)$. This functional depends 
on the extension of $A$ to $X$. However, by the main property 
of $CS$, the functional $\delta V(A)=W(A,c)$
(where $\delta$ is the BRST differential) does not depend on the extension 
and therefore is a local functional in $A$ and $c$ linear in $c$. 
One can show that it represents a nontrivial cohomology class in the local 
$\delta$-cohomology. Thus, we get an injective map
$S^{n+1}(\g)^\g
\to H^{1,local}_\delta$. 
For 4-dimensional theories $n=2$ and the cocycles come from $(S^3\g)^\g=
H^6(BG)$ as we expected. 

 
\end

Recall (Lecture 2) that in a Hamiltonian approach to classical gauge theory, 
the phase space is obtained as a symplectic quotient of the cotangent bundle. 
The quantum analogue of taking symplectic quotient is taking invariants 
in the Hilbert space. If degeneracies are present,  

In the future it will be convenient to eliminate $h$ and use 
the Lagrangian $\hat L$. After elimination of $h$, we will have 
$\delta\bar c=h=-\L$. However, 
in presence of $h$ the operator $\delta$ is independent on the theory and on 
$\L$ and depends only on the field configuration, which is an important 
property of $\delta$; this is the reason that $h$ is introduced. 



