%Date: Thu, 06 Feb 1997 16:55:08 -0500
%From: Edward Witten <witten@sns.ias.edu>


\input harvmac
Term 2, Problem Set 3

(1)  Let $G(k,n)$ be the Grassmannian of complex $k$ planes in ${\bf C}^n$.
Consider the two-dimensional sigma model with $G(k,n)$ as target space.
Determine the large $n$ behavior for fixed $k$.  Be sure
to include the ``theta angle'' related to $H^2(G(k,n),{\bf Z})={\bf Z}$.
Answer the usual  questions: is there a mass gap? Are symmetries
broken? What are the  quantum numbers of the lightest particles?

(2)  Now do the same for $G_{\bf R}(k,n)$, the Grassmannian of real
$k$ planes in ${\bf R}^n$.  Note that the ``theta angle'' is now
${\bf Z}/2{\bf Z}$-valued as $H^2(G_{\bf R}(k,n),{\bf Z})\cong
{\bf Z}/2{\bf Z}$.

In this case, to understand the case that the theta angle is non-zero,
you might find it helpful to look at my paper ``$\theta$ Vacua
In Two-Dimensional Quantum Chromodynamics,''
Il Nuovo Cimento {\bf 51 A} (1979)    325.   (A few copies will be provided.)


\end


