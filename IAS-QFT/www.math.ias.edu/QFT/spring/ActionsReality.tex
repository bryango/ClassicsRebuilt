\input amstex
\magnification=\magstephalf
\documentstyle{amsppt}
%%%%%%%%%%%% changes to amsppt.sty %%%%%%%%%%%%%%%%%%%%%%%
 \addto\tenpoint{\baselineskip 15pt
  \abovedisplayskip18pt plus4.5pt minus9pt
  \belowdisplayskip\abovedisplayskip
  \abovedisplayshortskip0pt plus4.5pt
  \belowdisplayshortskip10.5pt plus4.5pt minus6pt}\tenpoint
\pagewidth{6.5truein} \pageheight{8.9truein}
\subheadskip\bigskipamount
\belowheadskip\bigskipamount
\aboveheadskip=3\bigskipamount
\catcode`\@=11
\def\output@{\shipout\vbox{%
 \ifrunheads@ \makeheadline \pagebody
       \else \pagebody \fi \makefootline 
 }%
 \advancepageno \ifnum\outputpenalty>-\@MM\else\dosupereject\fi}
\outer\def\subhead#1\endsubhead{\par\penaltyandskip@{-100}\subheadskip
  \noindent{\subheadfont@\ignorespaces#1\unskip\endgraf}\removelastskip
  \nobreak\medskip\noindent}
\outer\def\enddocument{\par% \par will do a runaway check for \endref
  \add@missing\endRefs
  \add@missing\endroster \add@missing\endproclaim
  \add@missing\enddefinition
  \add@missing\enddemo \add@missing\endremark \add@missing\endexample
 \ifmonograph@ % do nothing
 \else
 \vfill
 \nobreak
 \thetranslator@
 \count@\z@ \loop\ifnum\count@<\addresscount@\advance\count@\@ne
 \csname address\number\count@\endcsname
 \csname email\number\count@\endcsname
 \repeat
\fi
 \supereject\end}
\catcode`\@=\active
%%%%%%%%%%%%%%% other macros %%%%%%%%%%%%%%%%%%%%%%%%%%%%
\CenteredTagsOnSplits
\NoBlackBoxes
\nologo
\def\today{\ifcase\month\or
 January\or February\or March\or April\or May\or June\or
 July\or August\or September\or October\or November\or December\fi
 \space\number\day, \number\year}
\define\({\left(}
\define\){\right)}
\define\Ahat{{\hat A}}
\define\Aut{\operatorname{Aut}}
\define\CC{{\Bbb C}}
\define\CP{{\Bbb C\Bbb P}}
\define\Conf{\operatorname{Conf}}
\define\Diff{\operatorname{Diff}}
\define\EE{\Bbb E}
\define\FF{\Bbb F}
\define\End{\operatorname{End}}
\define\Free{\operatorname{Free}}
\define\HH{{\Bbb H}}
\define\Hom{\operatorname{Hom}}
\define\Map{\operatorname{Map}}
\define\Met{\operatorname{Met}}
\define\QQ{{\Bbb Q}}
\define\RP{{\Bbb R\Bbb P}}
\define\RR{{\Bbb R}}
\define\SS{\Bbb S}
\define\Spin{\operatorname{Spin}}
\define\Tor{\operatorname{Tor}}
\define\Trs{\operatorname{Tr\mstrut _s}}
\define\Tr{\operatorname{Tr}}
\define\ZZ{{\Bbb Z}}
\define\[{\left[}
\define\]{\right]}
\define\ch{\operatorname{ch}}
\define\chiup{\raise.5ex\hbox{$\chi$}}
\define\cir{S^1}
\define\coker{\operatorname{coker}}
\define\dbar{{\bar\partial}}
\define\endexer{\bigskip\tenpoint}
%\define\exertag #1#2{\removelastskip\bigskip\medskip\eightpoint\noindent%
%\hbox{\rm\ignorespaces#2\unskip} #1.\ }  
\define\exertag #1#2{#2\ #1}
\define\free{\operatorname{free}}
\define\index{\operatorname{index}}
\define\ind{\operatorname{ind}}
\define\inv{^{-1}}
\define\mstrut{^{\vphantom{1*\prime y}}}
\define\protag#1 #2{#2\ #1}
\define\rank{\operatorname{rank}}
\define\res#1{\negmedspace\bigm|_{#1}}
\define\temsquare{\raise3.5pt\hbox{\boxed{ }}}
\define\theexertag{\theprotag}
\define\theprotag#1 #2{#2~#1}
\define\tor{\operatorname{tor}}
\define\xca#1{\removelastskip\medskip\noindent{\smc%
#1\unskip.}\enspace\ignorespaces }
\define\endxca{\medskip}
\define\zmod#1{\ZZ/#1\ZZ}
\define\zn{\zmod{n}}
\define\zt{\zmod2}
\redefine\Im{\operatorname{Im}}
\redefine\Re{\operatorname{Re}}
\let\germ\frak
%\font\boldtitlefont=cmb10 scaled\magstep2
\NoRunningHeads

\topmatter 
\title\nofrills {Actions and Reality} \endtitle
\author Dan Freed\endauthor 
\endtopmatter
 
\document

A basic constraint on a Minkowski space action is that it
be real. An action $\Cal S_m$ is the integral of a
Lagrangian (density) $L_m$ over Minkowski space $M$:
$$
S_m=\int\limits_ML_m.\tag1
$$
Choose a time $t$ on $M$.  Then we (Wick) rotate to Euclidean space $E$ by
introducing imaginary time $\tau =it$.  By convention the Euclidean action is
$\frac{1}{i}$ times the rotated Minkowski action:
$$
\frac{1}{i}\,S_m=S_e=\int\limits_EL_e.\tag2
$$
None that $e^{iS_m}=e^{-S_e}$.  Also, $S_e$ is not real in
general.
 
We describe the continuation to Euclidean space more precisely for a
$\sigma$-model.  The field is a map $\phi:M\longrightarrow Y$ into some
Riemannian manifold.  The complexification of the space of maps
$M\longrightarrow Y$ is the space of {\it holomorphic} maps $M_{\Bbb
C}\longrightarrow Y_{\Bbb C}$ between the complexified spaces (see Deligne's
notes {\it Real versus complex}).  The Lagrangian extends to a holomorphic
function on this space, and the Euclidean action is the restriction of this
continuation to maps $E\longrightarrow Y$.  (Note that $E_{\Bbb C}=M_{\Bbb
C}$ so $E\subset M_{\Bbb C}$.)  There is a similar picture for other types of
fields.
 
We consider four types of terms which typically occur in an
action:  kinetic terms for bosons, potential terms,
topological terms, and kinetic terms for fermions.  For
simplicity we discuss these terms in mechanics (the
one dimensional case) and then indicate the generalization
to higher dimensions.
 
Let $t,x^1,\ldots,x^{n-1}$ be coordinate on $M$.  We use the
metric
$$
dt^2-(dx^1)^2-\cdots-(dx^{n-1})^2\tag3
$$
on $M$ and the positive definite metric
$$
d\tau^2+(dx^1)^2+\cdots+(dx^{n-1})^2\tag4
$$
on $E$.
 
\subhead Kinetic Terms for Bosons\endsubhead
 
Consider a particle of mass $m$ moving in some Riemannian
manifold $Y$.  It is described by a map $x:\Bbb
R\longrightarrow Y$.  Then the kinetic energy density is
$$
L_m=\frac{m}{2}\,\,\bigg\vert\frac{dx}{dt}\bigg\vert^2dt.\tag5
$$
The continuation to imaginary time -- after dividing by
$i$ -- is 
$$
L_e=\frac{m}{2}\,\,\bigg\vert\frac{dx}{d\tau}
\bigg\vert^2d\tau.\tag6
$$
 
In higher dimensions we might consider a real scalar field
on Minkowski space, which is described by a real function
$\phi:M\longrightarrow\Bbb R$.  The kinetic lagrangian is 
$$
L_m=\frac{1}{2}\,\,|d\phi|^2_M\,\,dtdx^1\cdots
dx^{n-1},\tag7
$$
where $|\cdot|_M$ is the norm (3) on $M$.  The
continuation to $E$ is
$$
L_e=\frac{1}{2}\,|d\phi|^2_E\,\,d\tau dx^1\cdots
dx^{n-1},\tag8
$$
where $|\cdot|_E$ is the Euclidean norm (4).
 
\subhead Potential Terms\endsubhead
 
For the particle $x:\Bbb R\longrightarrow Y$, the potential
energy is described by a function $V:Y\longrightarrow\Bbb
R$.  The corresponding term in the Lagrangian is
$$
L_m=-\,V(x(t))\,dt.\tag9
$$
The continuation to imaginary time is
$$
L_e=V(x(\tau))\,d\tau.\tag10
$$
The extension to higher dimensions is the same:  Potential
terms appear with a $-$ sign in Minkowski actions and
with a $+$ sign in Euclidean actions.
 
\subhead Topological Terms\endsubhead
 
Let $A$ be a {\it real} 1-form on $Y$ and consider the
Lagrangian (for $x\:\RR\to Y$)
$$
L_m=-x^*A.\tag11
$$
The corresponding action is invariant under orientation-preserving
diffeomorphisms of $\RR$, hence the appellation `topological'.  The
continuation to imaginary time is innocuous except for the conventional
division by $i$:
$$
L_e=ix^*A.\tag12
$$
Hence in the Euclidean (imaginary time) Lagrangian the
topological term is imaginary.
 
The topological term (11) appears in the description of a charged particle
moving in an electromagnetic field.  Then $A$ is the ``vector potential''.
(This explains why we write a $-$ sign in (11): the term is part of the
potential energy.)  In a more geometric formulation, we consider the
electromagnetic field to be a connection (on $Y=$ Minkowski space) with gauge
group $U(1)$.  Relative to a trivialization this is an {\it imaginary} 1-form
$\alpha$ on $Y$.  Here is a difference between most physicists and
mathematicians: Physicists write formulas in terms of $A=\pm i\alpha$ whereas
mathematicians write formulas in terms of $\alpha$.  In either case the
reality condition is clear.  The role of the trivialization is a more
interesting story ... for another time.
 
In higher dimensions there is a wide variety of topological
terms which appear. Typically they are $n$--forms
$\omega(a)$ constructed from some field(s) $a$.  In
Minkowski space $\omega(a)$ is real, and the continuation
to Euclidean space is exactly as in (11) and (12).
 
\subhead Kinetic Terms for Fermions\endsubhead
 
By `fermions' here we understand any anticommuting
variables, i.e., elements of an odd vector space.  To
understand reality we begin with a general discussion of
real structures.
 
Let $A$ be an ungraded algebra over $\Bbb C$.  Then a {\it
real structure} on $A$ is a real linear map $a\longmapsto
a^*$ which satisfies
$$
\aligned
(\lambda a)^*&=\overline\lambda
a^*\qquad\qquad\,\,\,\,\,(\lambda\in\Bbb C,\quad a\in A),\\
(ab)^*&=b^*a^*\,\,\,\qquad\qquad(a,b\in A),\\
(a^*)^*&=a\qquad\qquad\qquad(a\in A).
\endaligned\tag13
$$
A familiar example is the quaternion algebra.  The algebra of complex
$n\times n$ matrices is another example, where $*$ is the conjugate
transpose.  Notice that simple conjugation of matrix elements does not
satisfy $(ab)^*=b^*a^*$.  Usually there is only one real structure relevant
to a given problem.  For example, the ``real'' matrices---those that satisfy
$a^*=a$---have the nice property that their eigenvalues are real and in
quantum mechanics they are the operators which correspond to real physical
quantities.  In general, notice that the real elements form a real vector
space $A_{\Bbb R}$ which is {\it not} a subalgebra, though it is closed under
anticommutators.  Similarly, the imaginary elements form a real vector space
closed under brackets, i.e., a real Lie algebra.  The space of derivations
Der$(A)$ inherits a real structure from that on $A$ by the rule
$$
D^*a=(Da^*)^*.\tag14
$$
It satisfies
$$
[D_1,D_2]^*=[D_1^*,D_2^*].\tag15
$$
 
Now suppose $A$ is a super ($\Bbb Z\slash2$--graded) algebra over $\Bbb C$.
Denote the parity of a homogeneous element~$a$ by~$p(a)$.  Then a real
structure satisfies (13) modified by the sign rule:
$$
\align
(\lambda a)^*&=\overline\lambda
a^*\qquad\qquad\qquad\qquad(\lambda\in\Bbb C,\quad a\in A),\\
(ab)^*&=(-1)^{p(a)p(b)}b^*a^*\,\,\,\quad\quad(a,b\in A),\\
(a^*)^*&=a\qquad\qquad\qquad\qquad\quad(a\in A).
\endalign
$$
Notice in the commutative case that
$$
(ab)^*=a^*b^*\qquad\qquad\qquad(\text{$A$ \it commutative})
\tag17
$$
The super Lie algebra of derivations Der$(A)$ inherits a
real structure defined as before by (14), and it again
satisfies (15).
 
Many physicists use a convention which omits the sign in (16).  This leads to
a complication which is explained in a footnote\footnote{More explicitly,
this alternative convention postulates
$$
(ab)^*=b^*a^*,\tag A
$$
and this differs from (16) if both $a$ and $b$ are odd.  As
a consequence (14) must be modified to
$$
D^*a=(a^*\overset\leftarrow\to D)^*,\tag B
$$
where $\overset\leftarrow\to D$ denotes $D$ thought of as acting
on the right, which is accomplished via the formula
$$
b\overset\leftarrow\to D=(-1)^{|D|\,|b|}Db.\tag C
$$
Taken together, (A) and (C) are an inconsistent use of the sign rule.  One
strange consequence is that for $D$ odd, $D=D^*$ if and only if $D$ maps real
even elements to {\it real} odd elements and $D$ maps real odd elements to
{\it imaginary} even elements.\newline
\indent This convention seems to be in force in most texts and
papers on (four dimensional) supersymmetry.  Compare (18)
and (19) below with the usual definitions in those texts:
  $$
 \align D_\alpha&=\frac{\partial}{\partial{\theta
^\alpha}}-i\overline{\theta}^{\dot\alpha} \,\,\frac
{\partial}{\partial x^{\alpha\dot\alpha}},\\
 \overline D_{\dot\alpha}
&=-\,\frac{\partial}{\partial\overline{\theta
}^{\dot\alpha}}+i\theta ^\alpha\,\,\frac{\partial}{\partial
x^{\alpha\dot\alpha}} \endalign $$
} (so as not to confuse the main text with inconsistent formulas).
 
As an example, let $V$ be a real odd vector space and $A=\text{Sym}(V)$
algebra of complex functions on $V$.  (Since $V$ is odd the symmetric algebra
is finite dimensional.)  Let $\phi^1,\ldots,\phi^n$ be a basis of $V^*$.
Then any product $\phi^{\alpha_1}\cdots\phi^{\alpha_k}$ is real, as is the
derivation $\partial\slash\partial\phi^\alpha$.
 
As another example, consider the superspace $\Bbb R^{4| 4}$.  We usually use
a complex basis $\theta^\alpha,\overline{\theta
}^{\dot\alpha}\,(\alpha=1,2;\,\dot\alpha=1,2)$
for the odd coordinates.  The conjugate of $\theta^\alpha$ is
$\overline{\theta}^{\dot\alpha}$.  With our conventions, then,
the operator
$$
D_\alpha=\frac{\partial}{\partial\theta
^\alpha}\,\,-\,\,\overline{\theta}^{\dot\alpha}\,\,\frac
{\partial}{\partial x^{\alpha\dot\alpha}}.\tag 18
$$
has conjugate
$$
\overline
D_{\dot\alpha}=\frac{\partial}{\partial\overline
{\theta}^{\dot\alpha}}-\theta
^\alpha\,\,\frac{\partial}{\partial x^{\alpha\dot\alpha}}.\tag19
$$
Here $x^{\alpha\dot\alpha}$ are the complex even coordinates on $\Bbb
R^{4|4}$ induced from the product of spinors in the usual way.
 
After these preliminaries we return to the particle $x:\Bbb R\longrightarrow
Y$ and add an odd field $\psi$ which is a section of $x^*\Pi TY$, the
parity-reversed pullback of the tangent bundle.  The field $\psi$ should be
thought of as a spinor on $\Bbb R$, but of course the spin bundle on $\Bbb R$
is trivial.  In any case $\psi$ is real and in real time its kinetic term in
the Lagrangian is
$$
L_m=\frac{m}{2}\,\left(\psi,\,\frac{d\psi}{dt}\right)dt.\tag20
$$
Rotating to imaginary time and dividing by $i$, we obtain
$$
L_e=-\,i\,\frac{m}{2}\,\left(\psi\,
\frac{d\psi}{d\tau}\right)d\tau.\tag21
$$
 
In higher dimensions suppose $\psi$ is a complex spinor
field on Minkowski space.  We use a Clifford algebra
$$
\aligned
\text{Minkowski:\;\;}(\gamma^0)^2&=1\\
(\gamma^i)^2&=-1
\endaligned\tag22
$$
where $\gamma^0$ is associated with time and $\gamma^i$
with space.  Let
$$
D\!\!\!\!\!\slash_m=\gamma^0\,\frac{\partial}{\partial
t}+\gamma^i\,\frac{\partial}{\partial x^i}\tag23
$$
be the associated Dirac operator.  Then $D\!\!\!\!\!\slash_m$
is {\it skew-adjoint}.  Let $\overline\psi$ be the
conjugate spinor to $\psi$.  Then the Lagrangian
$$
L_m=\overline\psi\cdot D\!\!\!\!\!\slash_m\psi\,dt\,dx^1\cdots dx^{n-1}\tag24
$$
is real, where we use a bilinear pairing on the spinor fields.  In Euclidean
space we use a Clifford algebra
$$
\text{Euclidean:\;\;}\left(\Gamma^\mu\right)^2=-1\tag25
$$
and {\it self--adjoint} Dirac operator
$$
D\!\!\!\!\!\slash_e=\Gamma^0\,\frac{\partial}{\partial\tau}
+\Gamma^i\,\frac{\partial}{\partial x^i}.\tag26
$$
Then the continuation of (24) to Euclidean space is
$$
L_e=\overline\psi\cdot D\!\!\!\!\!\slash_e\,\psi.\tag27
$$
One should only take (24) and (27) as general guidelines;
the particulars of spinors in each dimension should be
considered.
 
\subhead Summary\endsubhead
 
For reference we collect the various types of terms in the
real and imaginary time mechanics lagrangians:
$$
L_m=\Big\{\frac{m}{2}\,\,\bigg\vert\frac{dx}{dt}\bigg\vert^2+
\frac{m}{2}\left(\psi,\,\frac{d\psi}{dt}\right)-V(x)\Big\}
dt-x^*A,\tag28
$$
$$
L_e=\Big\{\frac{m}{2}\,\,\bigg\vert\frac{dx}{d\tau}\bigg\vert
^2
-i\,\frac{m}{2}\left(\psi,\,\frac{d\psi}{d\tau}\right)+V(x)
\Big\}d\tau+i\,x^*A.\tag29
$$




\enddocument
