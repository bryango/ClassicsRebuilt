%From: Pierre Deligne <deligne@math.ias.edu>
%Date: Mon, 25 Nov 1996 15:08:25 -0500
%Subject: Set 4, No. 2


\documentclass[12pt,leqno]{article}
\usepackage{amsthm,amsbsy,amsfonts,amssymb,amsmath,amscd}
\usepackage{latexsym}
\usepackage{fontenc}
\usepackage[mathscr]{euscript}
\setlength{\textwidth}{6.5in}
\setlength{\textheight}{8.5in}
\setlength{\topmargin}{0pt}
\setlength{\oddsidemargin}{0pt}
\setlength{\evensidemargin}{0pt}
\setlength{\headheight}{0pt}
\setlength{\headsep}{0pt}

%\hoffset=-1.75 true cm
\font\fiverm=cmr5

\input{pictex.tex}

\DeclareMathAlphabet{\scrb}{U}{eus}{b}{n}%for bold script letters


\newcommand{\dbA}{{\mathbb{A}}} %Blackboard Bold
\newcommand{\dbB}{{\mathbb{B}}}
\newcommand{\dbC}{{\mathbb{C}}}
\newcommand{\dbD}{{\mathbb{D}}}
\newcommand{\dbQ}{{\mathbb{Q}}}  
\newcommand{\dbR}{{\mathbb{R}}}

\newcommand{\grA}{{\mathfrak{A}}}  %Gothic or German
\newcommand{\grB}{{\mathfrak{B}}}
\newcommand{\grC}{{\mathfrak{C}}} 
\newcommand{\grD}{{\mathfrak{D}}}
\newcommand{\grG}{{\mathfrak{G}}} 

\newcommand{\bfA}{{\mathbf A}}   %Bold
\newcommand{\bfB}{{\mathbf B}}
\newcommand{\bfC}{{\mathbf C}} 
\newcommand{\bfD}{{\mathbf D}}

\newcommand{\ScrA}{{\mathscr{A}}} %Script 
\newcommand{\ScrB}{{\mathscr{B}}}
\newcommand{\ScrC}{{\mathscr{C}}} 
\newcommand{\ScrL}{{\mathscr{L}}} 
\newcommand{\ScrS}{{\mathscr{S}}}


%\newcommand{\bfscrA}{{\scrb{A}}} %for bold script letters
%\newcommand{\bfscrB}{{\scrb{B}}}
%\newcommand{\bfscrC}{{\scrb{C}}} 
%\newcommand{\bfscrD}{{\scrb{D}}}


%\renewcommand{\baselinestretch}{1.5}


\theoremstyle{plain}
{\theorembodyfont{}\newtheorem{thm}{Theorem}}
\newtheorem{prop}{Proposition}
\newtheorem{lem}{Lemma}
\newtheorem{cor}{Corollary}
\theoremstyle{remark}
{\theorembodyfont{\rmfamily} \newtheorem*{Rem}{\bf Remark}}
%\renewcommand{\theRem}{}


\newcommand{\Le}{{{\mathchoice{\,{\scriptstyle\le}\,}
  {\,{\scriptstyle\le}\,}
  {\,{\scriptscriptstyle\le}\,}{\,{\scriptscriptstyle\le}\,}}}}
\newcommand{\Ge}{{{\mathchoice{\,{\scriptstyle\ge}\,}
  {\,{\scriptstyle\ge}\,}
  {\,{\scriptscriptstyle\ge}\,}{\,{\scriptscriptstyle\ge}\,}}}}
\newcommand{\Ln}{\text{\rm ln}}



\font\boldtitlefont=cmb10 scaled\magstep1

\newcommand{\dspace}{\lineskip=2pt
     \baselineskip=18pt\lineskiplimit=0pt}
\newcommand{\wedgeop}{\mathop{\wedge}\limits}
\newcommand{\w}{{\mathchoice{\,{\scriptstyle\wedge}\,}
  {{\scriptstyle\wedge}}
  {{\scriptscriptstyle\wedge}}{{\scriptscriptstyle\wedge}}}}
\newcommand{\vrulesub}[1]{\hbox{\vrule height7pt depth5pt\,}_{#1}}
\newcommand{\mapright}[1]{\smash{\mathop{\,\longrightarrow\,}%
     \limits^{#1}}}
\newcommand{\plus}{{\sssize +}}
\newcommand{\upvee}{{\sssize\vee}}

\newcommand{\eps}{{\varepsilon}}
\newcommand{\lam}{{\lambda}}
\newcommand{\Lam}{{\Lambda}}
\newcommand{\mynabla}{{\nabla\!}}

\newcommand{\xtil}{\widetilde{x}}

\newcommand{\Adot}{\Dot{A}}
\newcommand{\Bdot}{\Dot{B}}
\newcommand{\Xdot}{\Dot{X}}
\newcommand{\xdot}{\Dot{x}}
\newcommand{\psidot}{\Dot{\psi}}


\newcommand{\GL}{\text{\rm GL}} 
\newcommand{\Hom}{\text{\rm Hom}}
\newcommand{\aff}{\text{\rm aff}} 
\newcommand{\red}{\text{\rm red}}
\newcommand{\Map}{\text{\rm Map}} 
\newcommand{\Lie}{\text{\rm Lie}}
\newcommand{\Diff}{\text{\rm Diff}} 
\newcommand{\Vol}{\text{\rm Vol}}
\newcommand{\Tr}{\text{\rm Tr}}

\title{Witten's Problems, Set Four --- N$^{\text{o}}$. 2}
\author{(solution written by D. Freed)}
\date{}

\overfullrule=5pt
\begin{document}

\maketitle

\hbox to \hsize{\hrulefill}

\bigskip
\dspace
For practice, we solve this problem using dimensional
regularization.
The dimensions of the fields and parameters are
\begin{align*}
[\phi] &=\tfrac{n-2}{2}\\
[m] &=1\\
[g] &=\tfrac{6-n}{2}
\end{align*}
Introduce a parameter $\mu$ with the dimension of mass and
a dimensionless parameter $\lam$.
Then the lagrangian density is
$$
\ScrL=\left\{\tfrac12\left(\vert d\phi\vert^2+m^2\phi^2\right)+
\lam\mu^{\tfrac{6-n}{2}}\,\tfrac{\phi^3}{3!}\right\}d^nx\,\,.
$$
To order $\lam^3$ the $3$-point function is computed by
the diagram shown.
%\begin{figure}[htbp] 
\begin{figure}[h]
$$
\vbox{\beginpicture
\setcoordinatesystem units < .50cm, .50cm>
\setlinear
%
% Fig ELLIPSE
%
\linethickness= 0.500pt
\ellipticalarc axes ratio  1.905:1.905  360 degrees 
	from 12.668 16.542 center at 10.763 16.542
%
% Fig POLYLINE object
%
\linethickness= 0.500pt
\plot 12.351 17.494 13.938 18.605 /
%
% Fig POLYLINE object
%
\linethickness= 0.500pt
\plot  9.144 17.558  7.620 18.542 /
%
% Fig POLYLINE object
%
\linethickness= 0.500pt
\plot  8.255 18.129  8.509 17.939 /
%
% arrow head
%
\plot  8.268 18.040  8.509 17.939  8.344 18.142 /
%
%
% Fig POLYLINE object
%
\linethickness= 0.500pt
\plot 13.399 18.193 13.144 18.002 /
%
% arrow head
%
\plot 13.310 18.205 13.144 18.002 13.386 18.104 /
%
%
% Fig POLYLINE object
%
\linethickness= 0.500pt
\plot 10.446 18.383 10.636 18.415 /
%
% arrow head
%
\plot 10.396 18.311 10.636 18.415 10.375 18.436 /
%
%
% Fig POLYLINE object
%
\linethickness= 0.500pt
\plot 12.446 15.653 12.319 15.494 /
%
% arrow head
%
\plot 12.428 15.732 12.319 15.494 12.527 15.653 /
%
%
% Fig POLYLINE object
%
\linethickness= 0.500pt
\plot  9.366 15.272  9.271 15.431 /
%
% arrow head
%
\plot  9.456 15.245  9.271 15.431  9.347 15.180 /
%
%
% Fig POLYLINE object
%
\linethickness= 0.500pt
\putrule from 10.795 14.637 to 10.795 13.049
%
% Fig POLYLINE object
%
\linethickness= 0.500pt
\putrule from 10.795 13.843 to 10.795 14.002
%
% arrow head
%
\plot 10.859 13.748 10.795 14.002 10.731 13.748 /
%
%
% Fig TEXT object
%
\put{$\scriptstyle p_{_1}$} [lB] at  8.414 18.891
%
% Fig TEXT object
%
\put{$\scriptstyle p_{_2}$} [lB] at 12.795 18.860
%
% Fig TEXT object
%
\put{$\scriptstyle q+p_{_1}$} [lB] at 10.319 18.923
%
% Fig TEXT object
%
\put{$\scriptstyle q$} [lB] at  8.922 14.859
%
% Fig TEXT object
%
\put{$\scriptstyle -(p_{_1}+p_{_2})$} [lB] at 11.366 13.462
%
% Fig TEXT object
%
\put{$\scriptstyle q+p_{_1}+p_{_2}$} [lB] at 12.827 15.399
\linethickness=0pt
\putrectangle corners at  7.620 19.431 and 13.938 13.049
\endpicture}
$$
\end{figure}
Here we are working in momentum space and label the edges
of the diagram with momenta.
Applying the Feynman rules for  Euclidean space, we
evaluate the diagram as
$$
\Gamma=-\lam^3\mu^{\frac32(6-n)}\int\tfrac{d^nq}{(2\pi)^n}
\tfrac{1}{(q^2+m^2)((q+p_1)^2+m^2)((q+p_1+p_2)^2+m^2)}
$$
Now use Feynman parameters to rewrite the integral as
\begin{align}
\Gamma_n
&=-\lam^3\mu^{\frac32(6-n)}\int\tfrac{d^nq}{(2\pi)^n}\int_0^1
d\alpha\int_0^{1-\alpha}d\beta
\tfrac{1}{[(1-\alpha-\beta)q^2+\alpha(q+p_1)^2+\beta
  (q+p_1+p_2)^2+m^2]^3}\label{one}\\
&=-\lam^3\mu^{\frac32(6-n)}\int_0^1d\alpha\int_0^{1-\alpha}d\beta
\int\tfrac{d^nq}{(2\pi)^n}\tfrac{1}{[q^2+A^2]^3},\nonumber
\end{align}
where
\begin{equation}
A^2=m^2+\alpha(1-\alpha)p_1^2+\beta(1-\beta)(p_1+p_2)^2-
2\alpha\beta p_1(p_1+p_2).  \label{two}
\end{equation}
the interchange of integrals is justified for $n$
sufficiently small (here $n<6$).
Note that we made a shift in the integration variable
$q$.
Now the rules for integrals in $n$ dimensions ($n$
complex!) give
\begin{equation}
\begin{aligned}
\int\tfrac{d^nq}{(2\pi)^n}\tfrac{1}{[q^2+a^2]^3}
&=\tfrac{1}{(2\pi)^n}
\tfrac{\pi^{n/2}}{\frac12\Gamma(n/2)}\int_0^\infty
dr\,r^{n-1}\tfrac{1}{[r^2+A^2]^3}\\
&=\tfrac{1}{2^{n-1}\pi^{n/2}\Gamma(n/2)}\cdot
  \tfrac12\tfrac{\Gamma\left(\frac n2\right) \Gamma\left(
3-\frac n2\right)}{\Gamma(3)}A^{n-6}\\
&=\tfrac{\Gamma\left(3-\frac n2\right)}{2^{n+1}\pi^{n/2}}
  A^{n-6}.
\label{three}
\end{aligned}
\end{equation}
Here $\frac{\pi^{n/2}}{\frac12\Gamma(n/2)}$ is the volume
of the unit sphere in $n$ dimensions, and for any $N$, 
\begin{equation*}
\begin{align*}
\int_0^\infty dr\,r^{n-1}\tfrac{1}{[r^2+A^2]^N}
&=\int_0^\infty
dr\,r^{n-1}\tfrac{1}{\Gamma(N)}\int_0^\infty
dt\,e^{-(r^2+A^2)t}t^{N-1}\\
&=\tfrac{A^{n-2N}}{\Gamma(N)}
  \int_0^\infty \tfrac{dt}{t}\int_0^\infty
 dr\,e^{-r^2t}r^{n-l}e^{-t}t^N\\
&=\tfrac{A^{n-2N}}{2\Gamma(N)}\int_0^\infty
   \tfrac{dt}{t} \int_0^\infty \tfrac{du}{u}
  e^{-u}u^{\frac n2}e^{-t}t^{N-n/2}\\
&=\tfrac12 \tfrac{\Gamma\left(\frac n2\right) \Gamma
  (N-n/2)}{\Gamma(N)}A^{n-2N}\,\,.
\end{align*}
\end{equation*}
The interchange of integrations is valid for $n$ small.
Now (\ref{three}) is finite for $n<6$ and has a simple pole
at $n=6$.
Collecting (\ref{one}) and (\ref{three}), we find
\begin{equation*}
\begin{align*}
\Gamma_n
&=-\tfrac{\lam^3}{2^{n+1}\pi^{n/2}}\Gamma\left(\tfrac{6-n}{2}
\right)\mu^{\frac32(6-n)}\int_0^1 d\alpha\int_0^{1-\alpha}
d\beta A^{n-6}\\
&=-\tfrac{\lam^3
B_n}{2^{n+1}\pi^{n/2}}\mu^{\frac32(6-n)}\Gamma
\left(\tfrac{6-n}{2}\right),
\end{align*}
\end{equation*}
where
$$
B_n=\int_0^1 d\alpha\int_0^{1-\alpha}d\beta
\left[m^2+\alpha(1-\alpha)p_1^2+\beta(1-\beta)(p_1+p_2)^2-2\alpha
\beta p_1(p_1+p_2)\right]^{\frac{n-6}{2}}.
$$
Note $B_6=1/2$ and $B_n$ is a holomorphic function of $n$
near $n=6$.
Introduce $\eps=3-n/2$ and write
$$
\Gamma_n=-\lam^3\mu^\eps\left\{\tfrac{1}{2(4\pi)^3}
\Gamma(\eps)(4\pi\mu^2)^\eps B_n\right\}
$$
and then expand around $\eps=0$:
\begin{equation*}
\begin{align*}
&\Gamma_n=\\
&-\lam^3\mu^\eps.\tfrac{1}{2(4\pi)^3}\left\{
  \tfrac{1}{2\eps}-\tfrac{\gamma}{2}-\int_0^1\! d\alpha
  \int_0^{1-\alpha}\!\!\! d\beta\,\Ln
\left(\tfrac{m^2+\alpha(1-\alpha)p_1^2+\beta(1-\beta)(p_1+p_2)^2
-2\alpha\beta
p_1(p_1+p_2)}{4\pi\mu^2}\right)\right\}+O(\eps),
\end{align*}
\end{equation*}
where $\gamma$ is Euler's constant.

To renormalize for $n=6$, we introduce the new lagrangian
density
$$
\ScrL'=\left\{\tfrac12\left(\vert d\phi\vert^2+m^2\phi^2\right)+
\lam\left(1+\tfrac{\lam^2}{2(4\pi)^3}\,\tfrac{1}{6-n}\right)
\mu^{\frac{6-n}{2}}\tfrac{\phi^3}{3!}\right\}d^nx\,\,.
$$
This introduces a diagram which cancels the pole in
$\Gamma_n$ at $n=6$, and the new $3$-point function to one
loop is the finite part of $\Gamma_n$ at $n=6$:
$$
\tfrac{\lam^3}{2(4\pi)^3}\left\{\tfrac{\gamma}{2}+\int_0^1
 d\alpha \int_0^{1-\alpha}d\beta\Ln\left(
\tfrac{m^2+\alpha(1-\alpha)p_1^2+\beta(1-\beta)(p_1+p_2)^2-2
  \alpha\beta p_1(p_1+p_2)}{4\pi\mu^2}\right)\right\}\,\,.
$$


\end{document}




