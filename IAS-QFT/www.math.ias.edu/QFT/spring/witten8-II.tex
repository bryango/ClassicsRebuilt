%revised version of part 2 of lecture 8:



\documentstyle[12pt]{amsart}

\setlength{\textwidth}{450pt}
\setlength{\textheight}{600pt}
\setlength{\topmargin}{0pt}
\setlength{\oddsidemargin}{0pt}
\setlength{\evensidemargin}{0pt}

% automatic loading of rsfs fonts, if present
\batchmode
  \newfont{\footscrfont}{rsfs10}
  \newfont{\footbbbfont}{msbm10}
\errorstopmode

\newif\ifscrf\scrftrue
\ifx\footscrfont\nullfont
  \scrffalse
\fi

\ifscrf
  \newfont{\scrfont}{rsfs10 scaled\magstep1}  % rsfs12 does not exist
  \newfont{\smallscrfont}{rsfs7}
  \newfont{\tinyscrfont}{rsfs7}
% \newfont{\footscrfont}{rsfs10}
  \newfont{\smallfootscrfont}{rsfs7}
  \newfont{\tinyfootscrfont}{rsfs7}
\fi

\makeatletter

\newif\iffn\fnfalse

\@ifundefined{reset@font}{\let\reset@font\empty}{} %needed for ancient LaTeX
\long\def\@footnotetext#1{\insert\footins{\reset@font\footnotesize
    \interlinepenalty\interfootnotelinepenalty
    \splittopskip\footnotesep
    \splitmaxdepth \dp\strutbox \floatingpenalty \@MM
    \hsize\columnwidth \@parboxrestore
   \edef\@currentlabel{\csname p@footnote\endcsname\@thefnmark}\@makefntext
    {\rule{\z@}{\footnotesep}\ignorespaces
      \fntrue#1\fnfalse\strut}}}
\makeatother

\ifscrf
  \newcommand{\Scr}[1]{\iffn
    \mathchoice{\mbox{\footscrfont #1}}{\mbox{\footscrfont #1}}
    {\mbox{\smallfootscrfont #1}}{\mbox{\tinyfootscrfont #1}}\else
    \mathchoice{\mbox{\scrfont #1}}{\mbox{\scrfont #1}}
    {\mbox{\smallscrfont #1}}{\mbox{\tinyscrfont #1}}\fi}
\else
  \def\Scr{\cal}
\fi


%integral sign with cirlce is \oint

%Hodge star (hstar) should probably not be \star
\newcommand{\hstar}{\mathop{*}}

% a wedgeproduct of a more reasonable size and spacing (from Robert Bryant)
\def\wdg{{\mathchoice{\,{\scriptstyle\wedge}\,}{{\scriptstyle\wedge}}
{{\scriptscriptstyle\wedge}}{{\scriptscriptstyle\wedge}}}}
%should it be a mathbin?

% upper half plane
\def\upperhalf{{\frak{h}}}

% Sl(2,Z)
\newcommand{\Sl}{\operatorname{Sl}}

%line bundle (lb) and Lagrangian (lg) -- should have distinct fonts
\newcommand{\lb}{{\Scr L}}
\newcommand{\lbm}{{\Scr M}} % a second line bundle
\newcommand{\lbn}{{\Scr N}} % and a third
\renewcommand{\lg}{{\cal L}}

%same font for circle bundles
\newcommand{\cb}{{\Scr S}}

%coupling constant in gauge theory (\ee) should not be in mathitalic, to avoid
%confusion with exponential

\newcommand{\ee}{\text{e}}

%script D, for use in pathspace measures
\newcommand{\cD}{{\cal D}}

%notation for operators
\newcommand{\cO}{{\cal O}}

%other caligraphic letters
\newcommand{\cF}{{\cal F}}
\newcommand{\cG}{{\cal G}}
\newcommand{\cH}{{\cal H}}

%standard blackboard bold items
\newcommand{\IR}{{\Bbb R}}
\newcommand{\IZ}{{\Bbb Z}}

%operatornames
\newcommand{\vol}{\operatorname{\rm vol}}
\newcommand{\volbar}{\operatorname{\overline{\rm vol}}}
\newcommand{\voltilde}{\operatorname{\widetilde{\rm vol}}}
\newcommand{\Tr}{\operatorname{\rm Tr}}
\newcommand{\Hol}{\operatorname{\rm Hol}}
\newcommand{\Funct}{\operatorname{\rm Funct}}

\numberwithin{equation}{section}

%\addtolength{\marginparwidth}{-.32in}
\addtolength{\marginparwidth}{-.4in}
\newcommand{\note}[1]{\marginpar{\scriptsize #1 }}

\begin{document}

\title[]{Lecture II-8, part II: Abelian Duality in Four Dimensions and
$\Sl(2,\IZ)$}
\author[]{Edward Witten$^*$}
\thanks{$^*$Notes by David R. Morrison}
\date{6 March 1997}
\maketitle

\section{Duality and $\Sl(2,\IZ)$}

In this second part of lecture II-8, we discuss abelian duality in four
dimensions, and give an application to an $\Sl(2,\IZ)$ symmetry of the free
$U(1)$ theory in four dimensions.  We postpone discussion of $\Sl(2,\IZ)$
symmetries of non-free theories to a later lecture, since all
known examples of
that involve supersymmetry.

We work with a $U(1)$ bundle $\lb$ on a $4$-manifold $M$, and a connection
$A$ on $\lb$, whose curvature is $F{=}F_A$.
The gauge theory Lagrangian (in Euclidean signature) including the
topological term is
\begin{equation}\label{lgrone}
\begin{aligned}
\lg(A) &= \frac1{4\ee^2}\int d^4x\,\sqrt{g} F_{mn}F^{mn}
+
\frac{i\theta}{16\pi^2}\int d^4x\,\sqrt{g} \epsilon_{mnpq}F^{mn}F^{pq} \\
 &=\frac1{2\ee^2}\int F_A\wdg\hstar F_A + \frac{i\theta}{4\pi^2}\int
F_A\wdg F_A .
\end{aligned}
\end{equation}
We have used the standard normalization on the kinetic term, and have
normalized the
topological term so that replacing $\theta$ by $\theta+2\pi$ does not change
the physics.  (This property of the topological term derives from the fact
that $c_1(\lb)^2=\int (F_A/2\pi)\wdg (F_A/2\pi)$ is always an integer.  Notice
that on a spin manifold,  $c_1(\lb)^2$ is always an {\it even}\/ integer,
and we gain an additional equivalence under replacement of $\theta$ by
$\theta+\pi$.)

Let $\tau=\frac{\theta}{\pi}+\frac{2\pi i}{e^2}\in \upperhalf$.
As we have just observed, $\tau\mapsto\tau+2$ is a symmetry of this theory,
and $\tau\mapsto \tau+1$ is a symmetry when working on a spin manifold.
To extend this to an $\Sl(2,\IZ)$ action (in the spin manifold case) we
also need
a symmetry which maps $\tau$ to $-1/\tau$; this will be
given by a {\it duality}\/ transformation $F_A\leftrightarrow\hstar F_C$
(with $C$ being a new ``dual'' connection).

The computations for this duality transformation are similar to those in
lecture II-7.  We begin by defining
 $F_\pm=\frac12(F_A\pm\hstar F_A)$, and
rewriting our Lagrangian (\ref{lgrone}) as
\begin{equation}
\begin{aligned}
\lg(A)&=\frac{i\overline{\tau}}{4\pi}\int F_+\wdg F_+ +\frac{i\tau}{4\pi}\int
F_-\wdg F_- \\
&=\frac{i\overline{\tau}}{4\pi}\int \|F_+\|^2-\frac{i\tau}{4\pi}\int
\|F_-\|^2 .
\end{aligned}
\end{equation}
Letting $\cG$ denote the gauge group associated to $A$, the partition
function for this theory can be written as
\begin{equation}
Z(\tau)=\frac1{\vol(\cG)}\sum_{\lb}\int \cD A\,
e^{-\frac{i\overline{\tau}}{4\pi}\int\| F_+\|^2
+\frac{i\tau}{4\pi}\int\| F_-\|^2} .
\end{equation}

Our earlier examples of duality began with a theory of a scalar field
$\phi$ which entered into the Lagrangian only through its derivative
$d\phi$ so that the theory had a symmetry under $\phi\mapsto \phi+c$ (with
$c$ constant); the first step in the duality transformation was to gauge
this symmetry, introducing also an appropriate Lagrange multiplier field.

The present theory is already a gauge theory, being a theory of a
connection $A$ which enters into the Lagrangian only through its curvature
$F_A$, so that there is a symmetry under $A\mapsto A+B$ (with $B$ a flat
connection).  We want to do the analogue of gauging this symmetry, by
allowing $B$ to be an arbitary connection on an arbitrary bundle,
introducing a kind of ``exotic gauge field'' $G$ which is a $2$-form field,
and extending the symmetry to
\begin{equation}
\begin{aligned}
A&\to A+B\\ G&\to G+F_B.
\end{aligned}
\end{equation}
Then $\cF:=F_A-G$ plays the role of the ``gauge-invariant'' quantity,
analogous to the covariant derivative of a scalar field.
It is to be stressed that two $G$ fields will be considered gauge-equivalent
if they differ by $G\to G+F_B$ for $F_B$ the curvature of any connection
on any line bundle.  In our analysis, we will assume
for simplicity that there is no torsion in $H^2(M)$.


We need a ``gauge-invariant'' extension of our Lagrangian.  We might try
\begin{equation}
\lg(A,G)=\frac{i\overline{\tau}}{4\pi}\int
\|\cF_+\|^2-\frac{i\tau}{4\pi}\int\|\cF_-\|^2,
\end{equation}
but this is too simple (because, for example, we could gauge $\cF$ to zero).
To improve this, we introduce a new connection $C$ on a line bundle $\lbn$,
with curvature $F_C$,
and consider the Lagrangian
\begin{equation}
\lg(A,G,C)=\frac{i\overline{\tau}}{4\pi}\int
\|\cF_+\|^2-\frac{i\tau}{4\pi}\int\|\cF_-\|^2 -\frac{i}{2\pi}\int
F_C\wdg G.
\end{equation}
The partition function for this new theory can be represented as a path
integral,
which includes sectors associated to all choices of
bundles $\lb$ and $\lbn$:
\begin{equation}\label{pathbig}
\frac1{\vol(\widetilde{\cG})}\,\frac1{\vol(\cG)}\,\frac1{\vol(\cG_C)}
\sum_{\lb,\lbn} \int\cD A\,\cD G\,\cD C\,
e^{-\frac{i\overline{\tau}}{4\pi}\int
\|\cF_+\|^2+\frac{i\tau}{4\pi}\int\|\cF_-\|^2 +\frac{i}{2\pi}\int
F_C\wdg G},
\end{equation}
where $\cG$ and $\cG_C$ denote the gauge groups associated to $A$ and $C$,
respectively, and $\widetilde{\cG}$ denotes the ``exotic'' gauge group.

To see that this new theory is equivalent to the original one, we first
do the $C$-integral in (\ref{pathbig}): write $C=C_0+C'$, for $C_0$ a fixed
connection on the
line bundle $\lbn$.  Then the $C'$ integral is
\begin{equation}
\frac1{\vol{\cG_C}}\int \cD C'\,e^{\frac{i}{2\pi}\int C'\wdg dG}=\delta(dG).
\end{equation}
Thus, when
we sum over $\lbn$ we find
\begin{equation}
\frac1{\vol{\cG_C}}\sum_{\lbn} \int \cD Ce^{-\frac{i}{2\pi}\int F_C\wdg G}
= \sum_{x\in H^2(M)}
e^{i (x,G)}\delta(dG)=\delta(\left[\frac G{2\pi}\right] \in \IZ)\delta(dG).
\end{equation}
The conditions that $dG=0$ and that $[G/2\pi]$ is an integral class
precisely mean that $G$ is of the form $F_B$ for some connection
on some line bundle and hence that $G$ can be gauged to zero.
After doing this,
it follows that the partition function (\ref{pathbig}) coincides with
$Z(\tau)$, and we recover the original theory.

Alternatively, we can evaluate the  partition function (\ref{pathbig}) by
gauging $A$ to $0$,
using the ``exotic'' gauge invariance (which has an ordinary gauge
invariance as an ambiguity).  This leaves the path integral
\begin{equation}\label{remaining}
\frac1{\vol{\cG_C}} \sum_{\lbn} \int\cD G\int \cD C\,
e^{-\frac{i\overline{\tau}}{4\pi}\int \|G_+\|^2+\frac{i\tau}{4\pi}\int
\|G_-\|^2 +\frac{i}{2\pi}\int F_C\wdg G} .
\end{equation}
To evaluate the $G$ integral, we complete the square,
bearing in mind that
\begin{equation}
\int F_C\wdg G=\int (F_C{}_+\wdg \hstar G_+-F_C{}_-\wdg \hstar G_-).
\end{equation}
In fact, if we define $G'=G - \frac1{\overline\tau} F_C{}_+ +
\frac1\tau F_C{}_-$, then we can write the exponent from
eq.~(\ref{remaining}) as
\begin{equation}
-\frac{i\overline{\tau}}{4\pi}\int \|G'_+\|^2
+\frac{i\tau}{4\pi}\int \|G'_-\|^2
+\frac{i}{4\pi\overline{\tau}}\int\| F_C{}_+\|^2
-\frac{i}{4\pi\tau}\int\| F_C{}_-\|^2 .
\end{equation}
When we carry out the $G'$ integral, the first two terms give a Gaussian
integral which contributes to the overall normalization; integrating out
$G'$ leaves the path integral
\begin{equation}
\frac1{\vol{\cG_C}}\sum_{\lbn}\int \cD C \,
e^{\frac{i}{4\pi\overline{\tau}}\int\| F_C{}_+\|^2
-\frac{i}{4\pi\tau}\int\| F_C{}_-\|^2} .
\end{equation}
This is the same as the original path integral, but with $\tau$ replaced by
$-1/\tau$, precisely what we wanted to show.

As we did in the case of two dimensions, it is possible to analyze the
$\tau$-dependence of the normalization of the path-integral, and obtain
further interesting results.  Some hint of the flavor of the results to be
obtained this way is seen if we evaluate the Gaussian integral indicated
above, which yields
\begin{equation}
\left(\frac{2\pi}{\sqrt{i\overline{\tau}}}\right)^{n_2{}_+}
\left(\frac{2\pi}{\sqrt{-i{\tau}}}\right)^{n_2{}_-}  ,
\end{equation}
where $n_2{}_\pm$ denote the numbers of self-dual and anti-self-dual
$2$-forms.  Of course, these numbers are infinite, so there must be some
cancellation against other normalization factors.  When this is worked out
in detail,\footnote{E. Witten, {\it On $S$-duality in abelian gauge
theory}, Selecta Math (N.S.) {\bf 1} (1995), 383--410.}
the result is found to be
\begin{equation}
Z(\tau)=\tau^{-\frac{\chi+\sigma}4}\overline{\tau}^{-\frac{\chi-\sigma}4}
Z(-1/\tau),
\end{equation}
where $\chi$ and $\sigma$ are the Euler number and signature of $M$,
respectively.   Thus, the partition function $Z(\tau)$ is actually a
modular form for $\Sl(2,\IZ)$ (or for a subgroup, when the manifold is not
spin) of weight
$(\frac{\chi+\sigma}4,\frac{\chi-\sigma}4)$.

We can also follow certain operator insertions through the duality
transformation, as we did in lower dimensions.  An insertion of $F_\pm$ in
the original theory can be realized by inserting $\cF_\pm=F_\pm-G$ in the
extended theory, which can be written
\begin{equation}
\cF_+=F_+-G'_+-\frac1{\overline{\tau}}F_C{}_+, \text{\ or\ }
\cF_-=F_--G'_-+\frac1{\tau}F_C{}_-,
\end{equation}
respectively,
after making the change of variables to $G'$.  Thus, when we gauge $A$ to
zero, and integrate out $G'$, we are left with operator insertions
proportional to $F_C{}_\pm$, namely:
\begin{equation}\label{opmap}
F_+\mapsto(-1/{\overline{\tau}})F_C{}_+, \text{\ and\ }
F_-\mapsto(1/{\tau})F_C{}_- .
\end{equation}
Notice that as a consequence of the $\tau$-dependence of these mappings,
a correlation function involving insertions of $F_+$ and $F_-$ will
have a different modular weight than that of the partition function.



\section{The Hamiltonian formalism}

Returning to the case that the gauge group is $U(1)$,
let us briefly discuss abelian four-dimensional duality in a Hamiltonian
framework.  Take a
$4$-manifold of the form $M_3\times {\Bbb R}$, where ${\Bbb R}$ is a
timelike direction. Note that this is a spin manifold, so we expect
full $Sl(2,\IZ)$ symmetry. For simplicity we suppose that there is no
torsion in $H_1(M_3)$. Each class
$x\in H^2(M_3)$ determines a complex line bundle $\lb_x$ on the
$3$-manifold $M_3$ (satisfying $c_1(\lb_x)=x$).  The Hilbert space for our
theory on the $3$-manifold $M_3$ can be written in the form
\begin{equation}
{\cal H}_\tau(M_3)=\bigoplus_{x\in H^2(M_3,{\Bbb Z})} \cH_x,
\end{equation}
where $\cH_x$ is the Hilbert space which comes from quantizing connections
on $\lb_x$.  (On the left,
we have explicitly indicated the dependence on the coupling
constant $\tau$.)  To construct $\cH_x$,
write an arbitrary connection in the form $A=A_0+\beta$, where
$A_0$ is a harmonic connection (a connection whose curvature is a harmonic
two-form) and
$\beta$ is a $1$-form which is co-closed.
\def\T{{\cal T}}
Let $\T_x$ be the space of harmonic connections on the line bundle ${\cal
L}_x$.
Then the quantization
yields
\begin{equation}
\cH_x=\cH_\beta\otimes L^2(\T_x).
\end{equation}
Here $\cH_\beta$ is a Hilbert space obtained by quantizing the space of
$\beta$'s, and $L^2(\T_x)$ is just the space of $L^2$ functions on $\T_x$.

Note that the factor $\cH_\beta$ is independent of $x$, since the space
of co-closed one-forms is defined with no reference to $x$.  Duality
maps $\cH_\beta$ to itself while  acting separately
on $\cH'=\oplus_x L^2(\T_x)$.   The duality action on $\cH_\beta$
follows from the operator mapping in (\ref{opmap}).

The action of duality on $\cH'$ can be described as follows.  Note that
the $\T_x$'s are all principal homogeneous spaces acted on by the torus
$H^1(M_3,\IR/\IZ)$, which parametrizes flat line bundles on $M_3$; the
action is defined by tensoring any given line bundle with connection
by a flat line bundle determined by a point in $H^1(M_3,\IR/\IZ)$.
Let $y$ denote a character of the abelian group $H^1(M_3,\IR/\IZ)$.
There is a decomposition $L^2(\T_x)=\oplus_y\T_{x,y}$, where $\T_{x,y}$
is the subspace of $L^2(\T_x)$ transforming in the character $y$.  Each
$\T_{x,y}$ is one-dimensional.
Hence
\begin{equation}
\cH'=\oplus_{x,y}\T_{x,y}
\end{equation}
Note that, by Poincar\'e and Pontryagin duality, the character group
of $H^1(M_3,\IR/\IZ)$ is $H^2(M_3,\IZ)$.  Thus, $x$ and $y$ take values
in the same space.  It is hence possible to exchange them, and this
is what the $\tau\to -1/\tau$ transformation does (more precisely,
it acts by $(x,y)\to (-y,x)$).
  Thus duality exchanges a classical
notion -- the decomposition with respect to $x$ -- with a quantum notion
-- the decomposition with respect to $y$.
The claim about how the duality acts will be justified below
where we introduce the operators $Q_E$ and $Q_M$.

Upon quantization---and suppressing $\theta$ for a
moment---one writes the four-dimensional curvature as
$F_A'+e^2\pi_A dt$, where $F_A'$ is a two-form on $M_3$ and $\pi_A$---a
one-form on $M_3$---is the momentum conjugate to the connection $A$.
The Hamiltonian becomes
\begin{equation}
H=\frac1{2\ee^2}\int F_{A_0}^2+{\ee^2\over 2}\nabla_{A_0}+ H(\beta).
\end{equation}
Here $H(\beta)$ is
the part of the Hamiltonian that acts on $\cH_\beta$.  The other
terms act on $\cH'$.  The first term is the magnetic energy
of the harmonic connection $A_0$; it comes from the part of the Lagrangian
quadratic in $F_A'$ and is a multiple of $\int_{M_3}x
\wdg\hstar x$.  The second term, which comes from the part of the Lagrangian
quadratic in $\pi_A$, is the electric energy, the Laplacian
on $\T_x$; it is a multiple of $\int_{M_3}y\wdg\hstar y$.


Including the $\theta$ term  shifts the quantization.  In fact, the
canonical momentum $F_A^\vee=\hstar \pi_A$  as determined from the
original Lagrangian (\ref{lgrone}) is
\begin{equation}\label{fvee}
F_A^\vee
 = 2\pi i \,\frac{\delta S}{\delta F_A}
=\frac{2\pi i}{\ee^2} \hstar F_A - \frac{\theta}{\pi} F_A
\end{equation}
At non-zero $\theta$, one has not $\cH'=\oplus_x L^2(\T_x)$
but $\cH'=\oplus_x\Gamma_{L^2}(\T_x,{\cal S}_\theta)$, where
${\cal S}_\theta$ is a certain flat line bundle over $\T_x$ and
$\Gamma_{L^2}$ is the space of $L^2$ sections.
  I leave it as an exercise to the reader
to identify ${\cal S}_\theta$.  Of course, ${\cal S}_\theta$ is trivial
at $\theta=0$ and only
depends on $\theta$ modulo $2\pi$.


Rewriting the formula for $F_A^\vee$
 in terms of $\tau$, we can determine how this operator transforms
 under $\tau\to -1/\tau$.  Indeed, under the operator mapping
(\ref{opmap}) one gets
\begin{equation}\label{fveeto}
F_A^\vee = -\overline{\tau}F_+ + \tau F_-
\mapsto F_C{}_++F_C{}_-=F_C.
\end{equation}
We will also need the ``dual'' version of this computation:
\begin{equation}\label{dualcomp}
F_A=F_++F_-\mapsto\overline{(-1/\tau)}F_C{}_+-(-1/\tau)F_C{}_- = -F_C^\vee.
\end{equation}

In the Hamiltonian formalism, for any $2$-cycle $\Sigma\subset M_3$
we can define associated ``electric'' and ``magnetic'' operators on the
Hilbert space ${\cal H}_\tau(M_3)$ for any $2$-cycle $\Sigma\subset M_3$,
by
\begin{equation}
\begin{aligned}
Q_E(\Sigma) &= \int_\Sigma \frac{F_A^\vee}{2\pi}
=\int_\Sigma\frac{F_C}{2\pi}  \\
Q_M(\Sigma) &= \int_\Sigma \frac{F_A}{2\pi}=-\int_\Sigma \frac{F_C^\vee}{2\pi}.
\end{aligned}
\end{equation}
(These operators only depend on the class of $\Sigma$ in $H_2(M_3,{\Bbb
Z})$.)
Clearly, under $\tau\to -1/\tau$, that is under $A\to C$, one has $Q_E\to
Q_M$, $Q_M\to -Q_E$.  Since $x$ and $y$ are the eigenvalues of $Q_M$
and $Q_E$, this means that $\tau\to -1/\tau$ acts by $(x,y)\to (-y,x)$.
Moreover,
from the explicit formula (\ref{fvee}) for $F_A^\vee$, one sees that when
$\theta$ is
increased by $2\pi$ (an operation which leaves
 the Hilbert space unchanged), the operator
$Q_M$ is unaltered, but the operator $Q_E$ maps to $Q_E+Q_M$.

The statements made in the last paragraph can be combined to the following:
$Sl(2,\IZ)$ acts on $\cH'$ via the natural action of $Sl(2,\IZ)$ on the
pair $(x,y)$.
\end{document}



