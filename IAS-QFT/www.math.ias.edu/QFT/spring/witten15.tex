%Date: Sat, 4 Apr 1998 19:22:31 -0500 (EST)
%From: Pavel Etingof <etingof@abel.math.harvard.edu>

\input amstex
\documentstyle{amsppt}
\magnification 1200
\NoRunningHeads
\NoBlackBoxes
\document

\def\tW{\tilde W}
\def\Aut{\text{Aut}}
\def\tr{{\text{tr}}}
\def\ell{{\text{ell}}}
\def\Ad{\text{Ad}}
\def\u{\bold u}
\def\m{\frak m}
\def\O{\Cal O}
\def\tA{\tilde A}
\def\qdet{\text{qdet}}
\def\k{\kappa}
\def\RR{\Bbb R}
\def\be{\bold e}
\def\bR{\overline{R}}
\def\tR{\tilde{\Cal R}}
\def\hY{\hat Y}
\def\tDY{\widetilde{DY}(\g)}
\def\R{\Bbb R}
\def\h1{\hat{\bold 1}}
\def\hV{\hat V}
\def\deg{\text{deg}}
\def\hz{\hat \z}
\def\hV{\hat V}
\def\Uz{U_h(\g_\z)}
\def\Uzi{U_h(\g_{\z,\infty})}
\def\Uhz{U_h(\g_{\hz_i})}
\def\Uhzi{U_h(\g_{\hz_i,\infty})}
\def\tUz{U_h(\tg_\z)}
\def\tUzi{U_h(\tg_{\z,\infty})}
\def\tUhz{U_h(\tg_{\hz_i})}
\def\tUhzi{U_h(\tg_{\hz_i,\infty})}
\def\hUz{U_h(\hg_\z)}
\def\hUzi{U_h(\hg_{\z,\infty})}
\def\Uoz{U_h(\g^0_\z)}
\def\Uozi{U_h(\g^0_{\z,\infty})}
\def\Uohz{U_h(\g^0_{\hz_i})}
\def\Uohzi{U_h(\g^0_{\hz_i,\infty})}
\def\tUoz{U_h(\tg^0_\z)}
\def\tUozi{U_h(\tg^0_{\z,\infty})}
\def\tUohz{U_h(\tg^0_{\hz_i})}
\def\tUohzi{U_h(\tg^0_{\hz_i,\infty})}
\def\hUoz{U_h(\hg^0_\z)}
\def\hUozi{U_h(\hg^0_{\z,\infty})}
\def\hg{\hat\g}
\def\tg{\tilde\g}
\def\Ind{\text{Ind}}
\def\pF{F^{\prime}}
\def\hR{\hat R}
\def\tF{\tilde F}
\def\tg{\tilde \g}
\def\tG{\tilde G}
\def\hF{\hat F}
\def\bg{\overline{\g}}
\def\bG{\overline{G}}
\def\Spec{\text{Spec}}
\def\tlo{\hat\otimes}
\def\hgr{\hat Gr}
\def\tio{\tilde\otimes}
\def\ho{\hat\otimes}
\def\ad{\text{ad}}
\def\Hom{\text{Hom}}
\def\hh{\hat\h}
\def\a{\frak a}
\def\t{\hat t}
\def\Ua{U_q(\tilde\g)}
\def\U2{{\Ua}_2}
\def\g{\frak g}
\def\n{\frak n}
\def\hh{\frak h}
\def\sltwo{\frak s\frak l _2 }
\def\Z{\Bbb Z}
\def\C{\Bbb C}
\def\d{\partial}
\def\i{\text{i}}
\def\ghat{\hat\frak g}
\def\gtwisted{\hat{\frak g}_{\gamma}}
\def\gtilde{\tilde{\frak g}_{\gamma}}
\def\Tr{\text{\rm Tr}}
\def\l{\lambda}
\def\I{I_{\l,\nu,-g}(V)}
\def\z{\bold z}
\def\Id{\text{Id}}
\def\<{\langle}
\def\>{\rangle}
\def\o{\otimes}
\def\e{\varepsilon}
\def\RE{\text{Re}}
\def\Ug{U_q({\frak g})}
\def\Id{\text{Id}}
\def\End{\text{End}}
\def\gg{\tilde\g}
\def\b{\frak b}
\def\S{\Cal S}
\def\L{\Lambda}

\topmatter
\title Lecture II-15. Four-dimensional gauge theories 
\endtitle
\author {\rm {\bf Edward Witten} }\endauthor
\endtopmatter

\centerline{Notes by Pavel Etingof and David Kazhdan}

\vskip .1in

In this lecture we will discuss the main known, believed, and 
conjectured results about 4-dimensional gauge theory, with 
or without supersymmetry. 

{\bf 15.1. Gauge theory without supersymmetry.}

We start with no supersymmetry ($N=0$). We consider a pure 
nonabelian gauge theory 
for a simple Lie group $G$, with Lagrangian
$$
L=\frac{1}{4g^2}\int |F|^2.
$$
We could add a topological term with the theta angle:
$\frac{i\vartheta}{16\pi^2}\int F\wedge F$. 
This is a well defined, asymptotically free theory. 
The problem we are interested 
in is how this theory behaves in the infrared. 

It is believed that this theory has a mass gap, and conjectured that it 
exhibits confinement (the area law for the Wilson loop operator, 
see Lecture 6). However, 
to justify the first and especially the second statement is still an
important open problem. 

{\bf Remark.} As we will see, the benefit of supersymmetry 
is that in the supersymmetric gauge theory, both statements can be justified. 

Now let us add matter. We consider the group $G=SU(N)$. 
The matter will be chiral spinors $\psi_i,\psi^*_i$ 
with values in the vector representation $V$ and the dual representation 
$V^*$, and the corresponding antichiral spinors $\bar\psi_i$, $\bar\psi_i^*$
with values in $V^*$ and $V$ (we are in Euclidean signature). 
Let the number of spinors of each kind be $n_f,n_f^*$ (the number of flavors).
We must have $n_f=n_f^*$ to avoid local anomalies (see the next lecture). 

The Lagrangian for the theory with spinors (with zero bare masses) is
$$
\gather
L=\frac{1}{4g^2}\int |F|^2+
\frac{i\vartheta}{16\pi^2}\int F\wedge F+\\
\sum_i\int (\bar\psi_i,D\psi_i)+
(\bar\psi_i^*,D\psi_i^*).\tag 15.2
\endgather
$$

{\bf Remark.} For $G=SU(3)$ and $n_f=2$ or $3$ this is the theory of strong 
interactions. In these cases it is necessary to add small bare masses
to match the physical reality. 

This theory is asymptotically free if $n_f$ is not too large. 
So one is interested in its infrared behavior. 

It turns out that this theory exhibits symmetry breaking. Namely, 
the Lagrangian has a global $U(n_f)\times U(n_f)$ symmetry permuting the 
flavors. However, quantum mechanically, this theory has a chiral 
anomaly (i.e. $D+A$ on chiral spinors
may have a nonzero index). 
Therefore, quantum mechanically the symmetry of the operator 
algebra is broken to the subgroup 
$H=\{(A,B)\in U(n_f)\times U(n_f),det(A)=det(B)\}$
(the rest of the group does not fix the measure of path integration). 

It is believed that 
the group $H$ is spontaneously broken to the diagonal subgroup $H_{diag}$
of $U(n_f)\times U(n_f)$. This breaking produces Goldstone bosons by
Goldstone theorem.    
If $n_f\le 1$, $H$ is already 
diagonal, and we don't get any Goldstone bosons. 
In this case it is believed that the theory has a mass gap. 
However, for $n_f>1$ the group $H$ is different from the diagonal subgroup, 
and there is no mass gap because of Goldstone bosons. 
In this case, if the theory is infrared free and has no other 
massless particles than Goldstone bosons, the low energy effective theory 
should be a sigma-model in the space of vacua $H/H_{diag}$.  

{\bf Remark.} The global symmetry breaking from $H$ to the diagonal 
subgroup is more delicate than the one we considered in Lecture II1:
in the present case this breaking is not seen classically, since the 
classical space of vacua is a point. 

The infrared behavior of this theory 
 is not completely understood, but some aspects are. 
For example, one can sometimes prove the absence of a mass gap, in the
following way. 

Let $J^a$, $a=1,...,dim(H)$ be the currents for $H$. Consider the 
3-point function 
$\<J^a(x)J^b(y)J^c(z)\>$. Since the currents are conserved classically, 
we get 
$$
d_x\<J^a(x)J^b(y)J^c(z)\>=0.\tag 15.1
$$  
if $x,y,z$ are distinct. However, if there is symmetry breaking, 
the l.h.s. of (15.1) 
may be a singular distribution supported on the set of non-distinct $x,y,z$. 
So let us go to the momentum space and compute (15.1) to 1-loop order. 
Let 
$$
F_{\l\mu\nu}^{abc}(k,q,r)\delta(k+q+r)=
\<\hat J_a^\l(k)\hat J_b^\mu(q)
\hat J_c^\nu(r)\>
$$ 
be the 3-point function of $J$ in momentum space. 
Then we get (to 1-loop accuracy)
$$
k_\l F_{\l\mu\nu}^{abc}=\e_{\mu\nu\alpha\beta}q^\alpha r^\beta d^{abc},\tag 
15.2
$$
where $k+q+r=0$ and $d^{abc}$ are some numbers 
($\e$ is a symmetrization tensor). 

It can be proved that this formula is exact, with no higher
order corrections.  This is proved, for instance, by introducing a 
regularization (such as adding higher derivative terms of a certain kind 
to the action ) that guarantees that higher than one-loop contributions
to $\d_x<JJJ>$ are zero.  

It is easy to show that no function
$F$ that is real-analytic near k=q=r=0 can obey (15.2) with nonzero $d$. Hence
if $d$ is not zero, $F$ is singular near zero momentum.  This implies that
the theory has massless particles, so there is no mass gap.

{\bf Remark.} This argument would not work if we considered a two-point 
function instead of the three-point function. 
Namely, it can be shown that in 2n dimensions, it is precisely the
n+1 point function of currents for which there is a formula analogous
to equation (15.2) -- computable from one-loop but exact.  Whenever one has
such a formula, one can use it to prove the absence of a mass gap.
In four dimensions, one uses a three point function; in two dimensions,
one would use a two point function.

It can be seen from this calculation that 
the tensor $d^{abc}$ is of the same nature  
as the tensor $d^{abc}$ in Lecture 16 which corresponds to the  
local gauge anomaly. In particular, its presence has to do with 
the fact that our theory has fields with coefficients in 
the vector representation of the first copy of $U(n_f)$ and no fields 
in the dual representation (i.e. with the fact that the theory 
is chiral with respect to the first copy of $U(n_f)$). 
In fact, if we try to gauge the $H$-symmetry, we will get a 
local gauge anomaly like in Lecture 16.  

Thus, in some cases we can see by a short distance calculation 
that we will have massless particles in the infrared. 

In 4 dimensions, there can be three types of massless particles:
scalars (spin 0), spinors (spin 1/2), and vectors (spin 1). 
Other particles are inconsistent with the existence of the energy-momentum 
tensor. The general principle about massless particles is that they are 
massless for a reason. Namely, it is believed  that in infrared free theories 
spin 0 particles 
are always Goldstone bosons coming from a broken global symmetry, 
spin 1/2 particles are fermions for which mass is prohibited by an unbroken 
chiral symmetry, and spin 1 are gauge bosons. 
This allows us to analyse the origin of a nonzero tensor $d^{abc}$. 

Let us look at the possibilities
in the case when $d^{abc}\ne 0$. 

{\bf Possibility 1.} The theory is not free in the infrared. 

{\bf Possibility 2.} The theory is free in the infrared. 
Then Green's functions can be calculated using Feynman diagrams of light 
fields. So there must be diagrams which contribute to the non-analyticity 
of $F$. There may be two cases. 

Case 1. The field $J(x)$ does not 
create from the vacuum a state of one massless
particle (the two-point function of $J$ has no pole at zero momentum). 
Then the tensor $d^{abc}$ can come only from loop diagrams. 
Such diagrams are expected to be with 
massless fermions inside, maybe not in the same representation as the one
in the classical Lagrangian, but producing a nonzero d tensor
(massless spin 0 and spin 1 particles 
are not expected to contribute to the tensor $d$ 
since, as we mentioned, 
such a contribution arises from chirality which has to do with spin 
$1/2$). 
This possibility was pointed out by t'Hooft in the late seventies, 
but examples were only found in the last few years. 
 
Case 2. There is no massless fermions, and no loop contributions to the 
singularity of $F$.  
Then the tensor $d$ must come from the tree diagram for $J$. 
In this case, 
the two point function of $J$ has a pole at zero momentum,
corresponding to the fact that $J$ creates from the vacuum a state
consisting of one massless particle. 
This state is a
massless boson $\pi$ of spin 0 from the vacuum (see Lecture II1), such that 
$J=*d\pi$ in the infrared limit. This boson $\pi$ 
is the Goldstone boson for the broken global $H$-symmetry. 

{\bf Remark.} Even in case 1, the two point function need not be 
analytic at the origin. 
If there is no mass gap, then the two point function of $J$ is almost
always nonanalytic at zero momentum: it has a branch cut beginning
at zero momentum.

Thus the conclusion is: if the theory has no massless fermions, 
the $H$-symmetry is spontaneously broken to a group $H'\subset H$ 
such that $d|_{H'}=0$. 

This is what happens in nature. Namely, the diagonal subgroup of $H$ has $d=0$ 
since the representation in which the fermions live is real for this group, 
and thus all chiral fields are balanced by antichiral ones. 
This is the most obvious subgroup of $H$ with this property. 

{\bf Remark.} Consider Case 2 from the point of view of 
effective Lagrangians. Consider the microscopic Lagrangian 
$L'=L+\int c^a\wedge J^a$, where $J^a$ are the currents defined above and 
$c^a$ are external sources which are 1-forms on the spacetime. 
For this Lagrangian, 
consider the effective Lagrangian in terms of the fields $c^a$. 
It will have the linear term, the quadratic term, etc. The 
linear term corresponds to one insertion of $J$, so 
it has the form $c^a\wedge *d\pi^a$, where $\pi$ is the Goldstone boson. 
The 3-point function $\<d*J^a(x) *J^b(y)*J^c(z)\>$ considered above 
corresponds to the quadratic term $dc^a\wedge c^b\wedge d\pi^c$
in the effective Lagrangian. The particles $\pi$ and $c$ can actually be 
observed in nature. In fact, it was in seeking
to understand the decay of the $\pi^0$ meson to two photons that
Adler, Bell, and Jackiw discovered anomalies around 1970. 

In conclusion we will say what behavior in the infrared is expected 
in the theories we are considering. Recall that the condition for asymptotic 
freedom is $n_f<11n/3$. If $n_f$ is small compared to $11n/3$, it is believed 
that the theory is infrared free with the behavior described above. 
It is also expected that it exhibits confinement for sufficiently 
small $n_f/n$. However, if $n_f$ is close to $11n/3$, it is expected 
that the theory has nontrivial IR stable fixed points. 
This is expected because the two-loop correction to the
beta function of these theories is positive.  Hence, if the one-loop
beta function is made small by taking $n_f$ close to  but smaller than
11n/3, then the
beta function has an infrared-stable zero close to the origin and within
reach of perturbation theory.

{\bf 15.2. N=1 supersymmetric pure gauge theory.}

Now consider $N=1$ supersymmetric theories. We start with 
``pure gauge theory''. In terms of components, this theory has 
a gauge field $A$ and a chiral spinor $\l$ with values in 
the adjoint representation. 
The Lagrangian is
$$
L=\frac{1}{4g^2}\int |F|^2+
\frac{i\vartheta}{16\pi^2}\int F\wedge F+\int\bar\l iD\l.\tag 15.3 
$$
The symmetry of this theory 
classically is the chiral $U(1)$ symmetry $\l\to e^{i\delta}\l$. 
This chiral symmetry, however, does not exist quantum mechanically since
the operator $D+A$ on spinors in the adjoint bundle 
may have a nontrivial index. However, one can show that this index 
for the sphere $S^4$ is always divisible by $2h$, where $h$ is the dual 
Coxeter number of the Lie algebra (it equals $2hk$ 
for the bundle with second Chern class $k$). This shows that the group 
of symmetries of the measure of integration is $\Z/2h\Z$. 
This group acts on the quantum operator algebra. 

In the infrared limit, the group $\Z/2h\Z$ could possibly be spontaneously 
broken to a subgroup. The smallest subgroup it can break down to 
is $\{-1,1\}$ since $-1$ is the central element of the double cover 
of the Poincare group. 

The standard expectations about this theory are:

1. Mass gap. 

2. Confinement. 

The first statement implies that fermion masses are generated dynamically.
This suggests that $\Z/2h\Z$ is broken down to a subgroup that allows bare
masses for the elementary fermions.  The largest such subgroup is
$\Z/2\Z$. So one expects that the unbroken group is precisely
$\Z/2\Z$. The spontaneous breaking from $\Z/2h\Z$ to $\Z/2\Z$ implies
that the theory has at least $h$ vacuum states.  It is believed that the 
number is precisely $h$.

It is remarkable that the statement about symmetry breaking 
can actually be checked. 
In order to do it, it is enough to show that the one point function 
$\<\l\l\>$ of the operator $\l\l$ is not zero (by $\l\l$ we mean a scalar 
obtained by contraction of the Lie algebra dimensions using the Killing form 
and of the spinor dimensions using the skew form on spinors). 

It is clear that $\<\l\l\>=Ce^{2\pi ik/h}$ in the k-th vacuum, for some 
constant $C$, 
since the generator of the group $\Z/2h\Z$ sends the k-th vacuum to 
$k+1$-th and multiplies $\<\l\l\>$ by $e^{2\pi i/h}$. 

Now let us ask ourselves: how does the function $\<\l\l\>$ depend on the 
theta-angle? Let $J$ be the current of the $U(1)$ symmetry. Because 
of the chiral anomaly, $dJ\ne 0$ but equals to a multiple 
of the Chern-Weil form: 
$$
dJ=\frac{h}{8\pi^2}F\wedge F.\tag 15.4
$$
This implies that the rotation of $\l$ by $e^{i\alpha}$ is equivalent to 
adding $2h\alpha$ to the theta-angle: $\vartheta\to\vartheta+2h\alpha$
(i.e. these two actions of $U(1)$ on the space of theories coincide). 
Thus we should expect that $\<\l\l\>=Ce^{i\vartheta/h}$, where $C$ is real
and independent of the theta-angle. Note that this function is 
multivalued and has $h$ different values, corresponding to $h$ vacua.  

Since $\<\l\l\>$ has $h$ different values, it is convenient 
to consider its $h$-th power $\<\l\l\>^h=C^he^{i\vartheta}$. This is a 
single-valued function of $\vartheta$. 

The constant $C$ depends only on the gauge coupling $g$, and is zero 
in perturbation theory, since the classical chiral symmetry prohibits masses 
in Feynman diagrams of all orders. Thus, if $C\ne 0$, it is a nonperturbative 
phenomenon (like in the Gross-Neveu model, see Gross' lecture 4).

Now let us compute $\<\l\l\>^h$. We have 
$$
\<\l\l\>^h=\lim_{|x_i-x_j|\to \infty}\<\l\l(x_1)...\l\l(x_n)\>\tag 15.5
$$
(cluster decomposition). This statement is true at all the vacua. 

We would like to show that $\<\l\l(x_1)...\l\l(x_n)\>$
 is independent of $x_1,...,x_n$. 
To do this, recall the superspace formulation of our model
(see the superhomework). 

In the superspace formulation, we have the super-spacetime 
$\R^{4,4}$ with coordinates $x_0,...,x_3$ and 
$\theta_+,\theta_-,\bar\theta_+,\bar\theta_-$. 
The Lagrangian of our model is equivalent to the Lagrangian 
$$
L'=\int d^4x (d^2\theta \tau W^2+d^2\bar\theta \bar\tau\bar W^2), 
$$
where $\tau=\frac{\vartheta}{2\pi}+\frac{4\pi i}{g^2}$, and $W$ is the 
supercurvature: $W=\l+\Theta F_+...$,
$\bar W=\bar\l+\Theta F_-+...$. Here $\Theta$ is the canonical 
linear chiral function on $\R^{4,4}$ with values in $S_+$, 
$F_+$ is the self-dual part of the curvature.
The terms $\Theta F_+$ involves the $SU(2)$-map $\C^2\o \C^3\to \C^2$. 

The supercurvature is a spinor-valued chiral superfield. Namely, let $\bar D$ 
be the spinor valued operator which annihilates chiral functions
(we have $[\bar D,\bar D]=0$). 
Then $\bar D W=0$. This implies that $\bar D(WW)=0$ (where $WW$ is a scalar
obtained by contracting spinors). Since $d=[D,\bar D]$, we get 
$d(WW)=\bar D D(WW)$. 

Now recall that for any operator $X$ the 1-point function 
$\<\bar D X\>$ is zero. This implies that 
$\<d(WW(x_1))WW(x_1)...WW(x_n)\>=0$, which yields the independence 
of $\<\l\l(x_1)...\l\l(x_n)\>$ of $x_1,...,x_n$ after setting $\Theta=0$. 

This implies that we can compute 
$\<\l\l\>^h$ at short distances. At short distances, 
$\<\l\l\>^h$ can be computed using asymptotic freedom. 

Our parameters are $\mu$ (the scale of renomalization), $g$ 
(the gauge coupling), and $\vartheta$ (the theta angle). 
Thus $\<\l\l\>=F(\mu,g,\vartheta)$. From dimensional
arguments, $F(\mu,g,\vartheta)=\mu^{3h}f(g,\vartheta)$, where $f$ is dimensionless
(fermions have scaling dimension $3/2$ in 4 dimensions). 
Also, we found that $\<\l\l\>^h$ has the form $Ce^{i\vartheta}$, where 
$C$ is independent of $\vartheta$. This implies 
that $f(g,\vartheta)=\rho(g)e^{i\vartheta}$. 

Now comes the main point. The main property of our theory, which follows 
from supersymmetry, is that if $X$ is an operator such that
$\bar D X=0$ then $\<X\>$ is a holomorphic function of $\tau$, where 
$\tau$ is the ``modular'' parameter in the superspace formulation. 
This follows from the fact that $\frac{\d L'}{\d\bar\tau}=
\int d^4xd^2\bar\theta \bar W^2=\bar D Y$, 
where $Y=\bar D\bar W^2$. Therefore, $F$ is holomorphic in $\tau$. 
This implies that $f(g)=e^{-8\pi^2/g^2}$, i.e. 
$$
F(\mu,g,\vartheta)=C_0e^{i\vartheta-\frac{8\pi^2}{g^2}}.\tag 15.6
$$
where $C_0$ is a constant. 

The constant $C_0$ is tricky to find. 
To find $C_0$, one should notice 
that the only nontrivial 
contributions to $\<\l\l\>^h$ can be from field configurations with 
instanton number (=second Chern class) equal to 1.
The reason that $C_0$ can come only from instanton
number 1 is that $\<\l\l\>^h$ has charge $2h$ under the R symmetry,
so the index of the Dirac operator must be 2h. 
 An attractive further check is that $8\pi^2/g^2$ is the
instanton action.
Thus, the constant $C_0$ can be found by doing perturbation theory 
in the vicinity of a single instanton. In particular, it can be shown 
that $C_0\ne 0$ (dynamical symmetry breaking). 

{\bf Remark 1.}  Since $\bar W=\bar\l+\Theta F_-+...$, 
and since instantons are defined by having 
self-dual curvature and $\l=0$, we have 
$\bar W=0$ for instantons and hence the Lagrangian is holomorphic in $\tau$ 
at the instantons (and only at them). This is an indication
of why the function $F$ should receive its contribution from instantons.

{\bf Remark 2.}  The function $F$ is a
 physically meaningful quantity (it has to do with fermion masses), so 
it obeys the renormalization group equation 
$$
(\mu\frac{\d}{\d \mu}+\beta(g)\frac{\d}{\d g})F=0,\tag 15.7
$$
where $\beta$ is the beta-function. Apriori we know only that 
$\beta(g)=-3hg^3+O(g^5)$. Formula (15.6) implies that  
the 1-loop beta function is in fact exact:
$$
\beta(g)=-3hg^3.\tag 15.8
$$
In other words, the parameter $\tau$ has an additive renormalization:
$\tau(\mu)=\tau(\mu_0)+48\pi^2h\ln(\mu/\mu_0)$. 

{\bf 15.3. N=1 theories with chiral superfields.}

Now let us consider the same theory with an adjoint-valued chiral 
superfield $\Phi$. The new Lagrangian is 
$$
L''=L'+\int d^4xd^4\theta\bar\Phi\Phi+(\e\int d^4xd^2\theta \Phi^2+c.c).
\tag 15.9
$$
(c.c. is ``complex conjugate''.)
Here $\e$ is a complex number which is a mass parameter for $\Phi$. 
If $\e$ is very large, we recover the previous theory as an effective 
theory after integrating out $\Phi$. If $\e=0$ we get Donaldson theory. 

{\bf Remark.} It can be deduced from supersymmetry that the case $\e\ne 0$ 
is the same as the case of large $\e$ in terms of the qualitative behavior 
of the theory in the infrared. This is proved
by redefining $\Phi$ by $\Phi \to \Phi/\sqrt{\e}$.  This operation removes
$\e$ from the superpotential.  The $\e$ dependence appears now
only in the kinetic energy of $\Phi$, which is a term of the form 
$\int d^4 \theta
(...)$.   Such terms are believed to be generally irrelevant in the
infrared behavior of a supersymmetric theory, so one expects the qualitative
behavior of this theory to be independent of $\e$.

For $G=SU(2)$, this  theory has two vacua
(as in the theory without the superfield $\Phi$). Also, we will show 
in later lectures that this theory has a mass gap and confinement. 

Now we will specialize to $G=SU(N)$ and do a superversion of QCD. 
So the Lagrangian will be the same as before
(with $\e=0$), except that the chiral 
superfield $\Phi$ will be in the representation $R=V^{\oplus n_f}
\oplus V^{*\oplus n_f}$, where $V$ is the vector representation of $SU(N)$. 
(As before, the multiplicity of $V$ and $V^*$ must be the same to avoid 
anomalies). Namely, $\Phi=(Q_1,...,Q_k,Q_1^*,...,Q_k^*)$, where $k=n_f$. 
This theory is asymptotically free iff $k<3N$, since $\beta=-\frac{g^3}{\pi}
(3N-k)$. 

The dynamical behavior of this theory is very nontrivial.  
For fuller details see Seiberg's lectures.
The following picture is believed to hold:

1. $k=0$ -- discussed above. There is a mass gap. 

2. $k=1,...,N-1$ -- discussed below. Low energy 
theory is a supersymmetric sigma-model with a nonzero superpotential. 

3. $k=N,N+1$ -- The theory is infrared free 
and has more massless fields than in case 2. 

4. $N+2\le k< 3N/2$ -- The theory is infrared free, is described  
in the infrared 
by a gauge theory, but the IR gauge group is not a subgroup of the 
UV gauge group.

5. $3N/2\le k<3N$ -- the infrared limit is nontrivial and non-free. 

Today we only discuss the case $1\le k\le N-1$. 
We will try to explain why the superpotential is nonzero. 

Let us compute the classical moduli space of vacua. 
According to lecture II2, this moduli space should be
the symplectic quotient: 
$\bold M=R//G$ (since $R$ is the set of zero energy states).  
Here the symplectic structure on $R$ is induced by the Hermitian form 
on $R$. Since $G$ acts holomorphically, we have $R//G=R/G_\C$. 
So we are down to classical invariant theory.

Recall that for $k<n$ 
all invariants of $SL(N)$ in $R$ are functions of the inner 
products $M_{ij}=(Q_i,Q_j^*)$. These invariants are independent, and 
form a coordinate system on $R/G_\C$. Moreover, let us compute 
the stabilizer of a point with $det(M)\ne 0$. It is easy to see that 
this stabilizer is isomorphic to $SL(n-k)$. Thus, we have 
a Higgs phenomenon: the $SU(N)$ gauge symmetry is spontaneously 
broken to $SU(N-k)$ (see lecture II2). In particular, for $k=N-1$ it 
is broken completely. 
This says that for $k<N-1$ the
classical theory has massless gauge fields of $SU(N-k)$ as well as massless
chiral multiplets parametrizing the symplectic quotient.  Quantum
mechanically the story will be more complicated for several reasons.

Now, since the stabilizer of a point in $\bold M$ is $SU(N-k)$, 
we should expect that the low energy effective theory is an 
$N=1$ supersymmetric $SU(N-k)$-gauge theory. As we know, this theory has 
$N-k$ vacua, so we should expect that the quantum vacuum space will be an 
$N-k$-fold cover of $\bold M$. Call this cover 
$\tilde\bold M$. Thus the low energy 
effective theory involves a sigma-model into $\tilde\bold M$ and 
possibly a superpotential. It is shown similarly to the above arguments that 
this superpotential $W$ must be holomorphic on 
$\tilde\bold M$ and as a function 
of the complexified gauge coupling $\tau$. The superpotential should of course 
have the same symmetries as the theory itself. 

Classically our theory has the $U(k)\times U(k)$ global symmetry 
acting on the superfield, as well as the R-symmetry, which 
is the $U(1)$ symmetry rotating $\l$. Some of these symmetries are anomalous 
(chiral anomaly). The anomaly free subgroup which acts on the quantum theory 
is generated by the group $H=\{(A,B)\in U(k)\times U(k); det(A)=det(B)\}$
and the group $U(1)=\{(\zeta,\zeta^{-\frac{N-k}{N}},\zeta^{\frac{N-k}{N}})\in
U(1)\times U(k)\times U(k)\}$. From these symmetries
(and remembering 
from the first part of the lecture 
how the R-symmetry acts on the theta angle), we see that the 
superpotential must equal 
$$
W(M)=c_{N,k}(det M)^{\frac{-1}{N-k}}.\tag 15.10
$$
So we do get a function on a cover $\tilde\bold M$ of $\bold M$.  

Now we need to compute $c_{N,k}$; in particular see if it is zero. 
The key fact for doing this is: If $c_{N,k}\ne 0$ then $c_{N-1,k-1}$ is 
nonzero, and one can be determined from the other. Indeed, if we take $M$ 
which has one very large eigenvalue, we will effectively be down to 
the theory with parameters $N-1,k-1$.

Thus if $k<N-1$, we can descend from $(N,k)$ to $(N-k,0)$ and thus 
reduce the problem to the case of pure $N=1$ gauge theory.
In pure gauge theory, 
the space $\bold M$ is a union of $N-k$ points, 
and the role of the superpotential $W$ is played by the 
function $\<\l\l\>$ that we have computed. 
Thus, $c_{N-k,0}\ne 0$. 

{\bf Remark.} It might seem that the superpotential is defined up to adding 
a locally constant function. However, it is in fact defined up to adding only a
 constant function. This is because in the case when the space is compact, 
there is only one vacuum and so the $h$ points of the moduli space 
``know'' about each other. 

Now let $k=N-1$. Then the descent procedure will stop at $N=2,k=1$.  
In the case $N=2,k=1$ one has  
$W=c_{2,1}(det M)^{-1}$. More precisely, by dimension arguments
$c_{2,1}=\tilde c_{2,1}\mu^5$, where $\tilde c_{2,1}$ is dimensionless. 
It is shown like for the pure gauge theory that instantons which 
contribute to $c_{2,1}$ have instanton number 1. 
Thus, by holomorphicity of $W$ with respect to 
$\tau$, we have 
$$
W=\tilde c_{2,1}\mu^5e^{i\vartheta-8\pi^2/g^2}/det(M).\tag 15.11
$$
  
The computation of $\tilde c_{2,1}$ is tricky. 
The reason is that there are instantons of all sizes, which 
makes it more difficult to do perturbation theory around the instantons.

This was done by Affleck, Dine, and Seiberg.
They showed $c_{2,1}$ is not zero. Thus $c_{N,N-1}$ is not zero. 

\end
 

