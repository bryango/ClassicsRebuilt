%From: Lisa C Jeffrey <jeffrey@IAS.EDU>
%Date: Mon, 25 Nov 1996 08:54:53 -0500
%Subject: Faddeev lecture 4


\documentstyle[12pt]{article}
\input amssym.def
\input amssym.tex
\newcommand{\Tr}{\,{\rm Tr}\,}
\newcommand{\tr}{{\rm tr}\,}

%L. Jeffrey preamble, 5 April 1996

\newcommand{\nc}{\newcommand}


\nc{\isq}{{i}}
\newcommand{\colvec}[2]{\left  ( \begin{array}{cc} #1  \\
     #2  \end{array} \right ) }

%%%%%\newcommand{\Tr}{\,{\rm Tr}\,}
\newcommand{\End}{\,{\rm End}\,}
\newcommand{\Hom}{\,{\rm Hom}\,}

\newcommand{\Ker}{ \,{\rm Ker} \,}

\newcommand{\bla}{\phantom{bbbbb}}
\newcommand{\onebl}{\phantom{a} }
\newcommand{\eqdef}{\;\: {\stackrel{ {\rm def} }{=} } \;\:}
\newcommand{\sign}{\: {\rm sign}\: }
\newcommand{\sgn}{ \:{\rm sgn}\:}
\newcommand{\half}{ {\frac{1}{2} } }
\newcommand{\vol}{ \,{\rm vol}\, }


% define abbreviations for most common commands
%

\newcommand{\beq}{\begin{equation}}
\newcommand{\eeq}{\end{equation}}
\newcommand{\beqst}{\begin{equation*}}
\newcommand{\eeqst}{\end{equation*}}
\newcommand{\barr}{\begin{array}}
\newcommand{\earr}{\end{array}}
\newcommand{\beqar}{\begin{eqnarray}}
\newcommand{\eeqar}{\end{eqnarray}}
\newtheorem{theorem}{Theorem}[section]
%\newtheorem{conjecture}{Conjecture}
\newtheorem{corollary}[theorem]{Corollary}
%\newtheorem{problem}{Problem}
\newtheorem{lemma}[theorem]{Lemma}
\newtheorem{prop}[theorem]{Proposition}
\newtheorem{definition}[theorem]{Definition}
\newtheorem{remit}[theorem]{Remark}
\newtheorem{conjecture}[theorem]{Conjecture}

\newtheorem{example}[theorem]{Example}

\newcommand{\matr}[4]{\left \lbrack \begin{array}{cc} #1 & #2 \\
     #3 & #4 \end{array} \right \rbrack}



\newenvironment{rem}{\begin{remit}\rm}{\end{remit}}




% black board bold face
%note \AA is already defined!
\newcommand{\aff}{{ \Bbb A }}
\newcommand{\RR}{{{\bf  R }}}
\newcommand{\CC}{{{\bf  C }}}
\nc{\FF}{ {\Bbb F} } 
\newcommand{\ZZ}{{{\bf   Z }}}
\newcommand{\PP}{ {\Bbb P } }
\newcommand{\QQ}{{\Bbb Q }}
\newcommand{\UU}{{\Bbb U }}








%***************************




%


%


%Replace greek letters by their roman equivalents with \
%Slightly nonstandard:  theta is \t, tau is \ta, no omicron
\def\a{\alpha}
\def\b{\beta}
\def\g{\gamma}
\def\d{\delta}
\def\e{\epsilon}
\def\z{\zeta}
\def\h{\eta}
\def\t{\theta}
%\def\i{\iota}
\def\k{\kappa}
\def\l{\lambda}
\def\m{\mu}
\def\n{\nu}
\def\x{\xi}
\def\p{\pi}
\def\r{\rho}
\def\s{\sigma}
\def\ta{\tau}
\def\u{\upsilon}
\def\ph{\phi}
\def\c{\chi}
\def\ps{\psi}
\def\o{\omega}

\def\G{\Gamma}
\def\D{\Delta}
\def\T{\Theta}
\def\L{\Lambda}
\def\X{\Xi}
\def\P{\Pi}
\def\S{\Sigma}
\def\U{\Upsilon}
\def\Ph{\Phi}
\def\Ps{\Psi}
\def\O{\Omega}




% calligraphic letters
\newcommand{\calA}{{\mbox{$\cal A$}}}
\newcommand{\calB}{{\mbox{$\cal B$}}}
\newcommand{\calC}{{\mbox{$\cal C$}}}
\newcommand{\calD}{{\mbox{$\cal D$}}}
\newcommand{\calE}{{\mbox{$\cal E$}}}
\newcommand{\calF}{{\mbox{$\cal F$}}}
\newcommand{\calG}{{\mbox{$\cal G$}}}
\newcommand{\calH}{{\mbox{$\cal H$}}}
\newcommand{\calI}{{\mbox{$\cal I$}}}
\newcommand{\calJ}{{\mbox{$\cal J$}}}
\newcommand{\calK}{{\mbox{$\cal K$}}}
\newcommand{\calL}{{\mbox{$\cal L$}}}
\newcommand{\calM}{{\mbox{$\cal M$}}}
\newcommand{\calN}{{\mbox{$\cal N$}}}
\newcommand{\calO}{{\mbox{$\cal O$}}}
\newcommand{\calP}{{\mbox{$\cal P$}}}
\newcommand{\calQ}{{\mbox{$\cal Q$}}}
\newcommand{\calR}{{\mbox{$\cal R$}}}
\newcommand{\calS}{{\mbox{$\cal S$}}}
\newcommand{\calT}{{\mbox{$\cal T$}}}
\newcommand{\calU}{{\mbox{$\cal U$}}}
\newcommand{\calV}{{\mbox{$\cal V$}}}
\newcommand{\calW}{{\mbox{$\cal W$}}}
\newcommand{\calX}{{\mbox{$\cal X$}}}
\newcommand{\calY}{{\mbox{$\cal Y$}}}
\newcommand{\calZ}{{\mbox{$\cal Z$}}}

%% To load script letters:

\font\teneusm=eusm10  \font\seveneusm=eusm7 
\font\fiveeusm=eusm5 
\newfam\eusmfam 
\textfont\eusmfam=\teneusm 
\scriptfont\eusmfam=\seveneusm 
\scriptscriptfont\eusmfam=\fiveeusm 
\def\Scr#1{{\fam\eusmfam\relax#1}}

% Script letters
\newcommand{\ScrA}{{\Scr A}} \newcommand{\ScrB}{{\Scr B}}
\newcommand{\ScrC}{{\Scr C}} \newcommand{\ScrD}{{\Scr D}}
\newcommand{\ScrE}{{\Scr E}} \newcommand{\ScrF}{{\Scr F}}
\newcommand{\ScrG}{{\Scr G}} \newcommand{\ScrH}{{\Scr H}}
\newcommand{\ScrI}{{\Scr I}} \newcommand{\ScrJ}{{\Scr J}}
\newcommand{\ScrK}{{\Scr K}} \newcommand{\ScrL}{{\Scr L}}
\newcommand{\ScrM}{{\Scr M}} \newcommand{\ScrN}{{\Scr N}}
\newcommand{\ScrO}{{\Scr O}} \newcommand{\ScrP}{{\Scr P}}
\newcommand{\ScrQ}{{\Scr Q}} \newcommand{\ScrR}{{\Scr R}}
\newcommand{\ScrS}{{\Scr S}} \newcommand{\ScrT}{{\Scr T}}
\newcommand{\ScrU}{{\Scr U}} \newcommand{\ScrV}{{\Scr V}}
\newcommand{\ScrW}{{\Scr W}} \newcommand{\ScrX}{{\Scr X}}
\newcommand{\ScrY}{{\Scr Y}} \newcommand{\ScrZ}{{\Scr Z}}

%German (Faktur) letters

\newcommand{\grA}{{\frak A}}



\def\eps{\varepsilon}

\setlength{\textwidth}{6.5in}
\setlength{\textheight}{9.1in}
\setlength{\evensidemargin}{0in}
\setlength{\oddsidemargin}{0in}
\setlength{\topmargin}{-.75in}
\setlength{\parskip}{0.3\baselineskip}

%\renewcommand{\theequation}{\thesection.\arabic{equation}}
%\newcommand{\renorm}{{ \setcounter{equation}{0} }}

\newcommand{\Le}{{{\mathchoice{\,{\scriptstyle\le}\,}
  {\,{\scriptstyle\le}\,}
  {\,{\scriptscriptstyle\le}\,}{\,{\scriptscriptstyle\le}\,}}}}
\newcommand{\Ge}{{{\mathchoice{\,{\scriptstyle\ge}\,}
  {\,{\scriptstyle\ge}\,}
  {\,{\scriptscriptstyle\ge}\,}{\,{\scriptscriptstyle\ge}\,}}}}


%\renewcommand{\baselinestretch}{1.5}


%more newcommands
\nc{\bra}{  < }
\nc{\ket}{ > }
%\nc{\isq}{ { \sqrt{-1} } }
%\nc{\isq}{{ i }}
%\nc{\hbar}{{ h}}
\nc{\triang}{ { \bigtriangleup} }

\nc{\astar}{{ a^*}}
\nc{\hata}{ { \hat{a} }}
\nc{\hatastar}{{ \hat{\astar} }}
\nc{\normfac}{{\frac{1}{\sqrt{2 \omega} } }}

\begin{document}

\title{Lecture 4:\\
Singular Lagrangians}
\author{Ludwig Faddeev}
\date{7  November  1996}

\maketitle

%\renorm
\nc{\dphi}{\tilde{\phi}} 
\nc{\vac}{{ \sl v}}
\nc{\basvec}[1]{\Phi_{#1} (\astar) } 
\nc{\basvecn}{\basvec{n} } 
\nc{\kn}{(\vec{k})_{n}} 
\nc{\vk}{\vec{k} }
\nc{\vx}{{ \vec{x} } }
\nc{\vy}{{ \vec{y} } }

In other lectures, considerable attention has been focused on how to construct
the symplectic structure on the space of classical solutions, 
starting from the Lagrangian. 
In this lecture we will instead focus on the
fixed time  Hamiltonian formalism,
and we will see that the transformation which maps the Lagrangian 
into the Hamiltonian is simply a change of variables. The lecture will 
chiefly treat the case of singular Lagrangians. Our aim is to present
the first order formalism, in which the Lagrangian depends linearly 
on time derivatives of the dynamical variables.


\nc{\qdot}{{\dot{q} }}
In the usual Lagrangian formalism, we begin with the configuration space $M$:
coordinates on $M$ will be denoted $q$. Coordinates on the tangent 
bundle $TM $ are then given by $q$ and $\dot{q}$, and the Lagrangian is 
a function $\calL(q, \qdot)$. 
The action is then defined by 
$$ A = \int \calL  (q(t), \qdot(t) ) \, dt. $$
Applying the standard variational  principle we see that the extrema
of the action are given by the {\em Euler-Lagrange equation}
\beq \label{el}  \frac{d}{dt}  \frac{\partial \calL}{\partial \qdot} - 
\frac{\partial \calL}{\partial q} = 0. \eeq
This is ordinarily a system of 
equations of {\em second order} 
since the Euler-Lagrange equation involves the term 
$\ddot{q} {\partial^2 \calL}/{\partial \qdot \partial \qdot}$. 
The Lagrangian is said to be {\em nondegenerate} if the Hessian
${\partial^2 \calL}/{\partial \qdot \partial \qdot}$ is 
invertible. 

In the Hamiltonian formalism, on the other hand, we work with 
$T^* M \times \RR$ (where 
 $T^* M$  is the   cotangent bundle of configuration space, or the 
 phase space).
On this there is a natural 1-form 
\beq \label{oneform} p dq - H(p,q) dt, \eeq
 where we 
have introduced $(q, p )$ as  natural  coordinates on $T^* M$. 
The corresponding Lagrangian is written as 
\beq \label{lagstd} l = p \qdot - H(p,q), \eeq
and the action is 
$$ A = \int  (p \qdot - H(p,q)) dt. $$
The Euler-Lagrange equation applied to $A$  then gives Hamilton's 
equations:
\beq \dot{p} = - \frac{\partial H}{\partial q}, 
~~~~\dot{q} = \frac{\partial H}{\partial p}. \eeq
Note that 
 the Lagrangian $l$ is linear  in time derivatives, and the Hessian 
of $l$ is identically zero. Thus our system is of first order.
In the standard fashion, we have converted a system of second order 
differential equations into a system of first order differential 
equations, at the expense of increasing the number of variables.
To explicitly see how the Hamiltonian formalism
is derived from the Lagrangian formalism, we introduce 
$v = \qdot$ and treat $q$ and $v$ as independent variables. 
The Lagrangian then depends on $q$, $v$, 
$\qdot$ and $\dot{v}$, and has the following form: 
\beq l = \frac{\partial \calL}{\partial v} (\qdot - v) + \calL(q, v). \eeq
Comparing with (\ref{lagstd})
one finds that the Hamiltonian   is given by 
$$ H = v \frac{\partial \calL}{\partial v} - \calL, $$
and we obtain the Hamiltonian formalism by performing the Legendre 
transformation 
$$ (q, v) \mapsto (q,p) $$
where we have defined 
$$ p = \frac{\partial \calL}{\partial v}. $$


On a general manifold $\Gamma$ with coordinates $\xi$, we would 
write the Lagrangian as 
\beq l = \sum_a f_a(\xi) \dot{\xi}^a - \phi(\xi). \eeq
The one-form on $\Gamma \times \RR$ corresponding to (\ref{oneform}) 
would be written as 
$$  \sum_a f_a(\xi) d \xi^a  - \phi dt = \omega - 
\phi dt, $$
and it gives rise to a two-form $ \Omega = d \omega$. (Here $d$ denotes
the de Rham differential on $\Gamma$, rather than   on $\Gamma \times \RR$.) 
The equation of motion becomes 
$$ \Omega \dot{\xi} = \frac{\partial \phi}{\partial \xi}. $$
The condition that the Lagrangian is nonsingular is equivalent to the 
condition that the 
two-form $\Omega$ is invertible: for this to hold, it is 
necessary that $\Gamma$ be even dimensional.    For   important Lagrangians
in physics (for instance the Yang-Mills Lagrangian and the 
Lagrangian for electrodynamics) the two-form $\Omega$ is in fact not
invertible. Let us investigate this situation in more detail.

Invoking the Darboux theorem, 
on  $\Gamma $ we write the coordinates
$\xi$ as $(p, q, z)$ in such a way that the one-form $\omega$ becomes
$$ \omega = p dq + d \theta.$$
The total derivative $d \theta$ may be discarded, and after this the 
one-form corresponding to the 
Lagrangian becomes
$$ l = p dq - \phi(p,q, z) dt, $$
where
$\frac{\partial \phi}{\partial z} = 0 . $ 
The variables $z$ may be divided into ``excludable'' variables (which will
ultimately be discarded) and others which enter linearly. After 
eliminating the excludable variables, the Lagrangian becomes
\beq \label{elimexcl} l = p dq - \phi(p, q) dt - \sum_\alpha \lambda^\alpha 
\varphi_\alpha (p,q). \eeq
Here, the $\lambda^\alpha $ play the role of {\em Lagrange multipliers}
while the $\varphi_\alpha(p,q)$ are constraints. Among the 
equations of motion we now obtain the 
constraints
$$\varphi_\alpha(p,q) = 0, $$
so we may substitute $p = p(\eta), $ $ q = q(\eta)$ where
$\eta$ are coordinates on the manifold 
$$ \Gamma^l = \{ (p,q):  \varphi_\alpha(p,q) = 0 \} . $$

We thus obtain a new phase space $\Gamma^l $ on which the 
excludable variables and the Lagrange multipliers have been eliminated,
so that the variables $z$ no longer appear and we are left with
a subspace parametrized by   the variables $(p,q)$  where the 
constraints $\varphi_\alpha(p,q) = 0 $ have been imposed. 
The new phase space $\Gamma^l$ may have a nondegenerate symplectic 
form, and then our
Hamiltonian reduction is done.  
If not, we repeat the above process until Lagrange multipliers
 no longer appear, 
in which case we have reduced to  a phase space with a 
nondegenerate symplectic form. 

Explicit solution of the constraints could be complicated, 
so it is instructive to understand what properties of the 
singular Lagrangian (\ref{elimexcl}) guarantee that it will 
become nonsingular after the constraints are imposed. There are two 
important instances of this. 
We can distinguish two different types of constraints. A collection 
of constraints $\{ \varphi_a \}$ are called 
{\em first class} if the Poisson brackets of all constraints
vanish on the zero locus of the constraints: in other words, if 
\beq \{ \varphi_a, \varphi_b \}|_{\varphi = 0 } = 0 . \eeq
We claim that if $\{ \phi, \varphi_a\}|_{\varphi = 0 }  = 0  $ for 
all $a$, then in the next step of the procedure one obtains a proper
nondegenerate Lagrangian. More precisely, the submanifold 
$\Gamma|_{\varphi = 0 } $ may be fibered so that the base is symplectic 
and the fibers are isotropic with respect to the Poisson bracket 
$\{ \cdot, \cdot \}$. One may summarize the procedure by saying that
``first class constraints kill twice'': when one restricts to the 
zero locus of the constraints $\varphi_a$,
 one must also set to zero the variables
which are canonically conjugate to the $\varphi_a$, in 
other words one must set to zero all Poisson brackets with 
the $\varphi_a$.  If there are 
$N$ constraint equations $\varphi_a, $ $ a = 1, \dots, N$, 
then imposing the constraints reduces
the  dimension of the phase space by $2N$. 

The  second type of constraints $\varphi_a$  are those  for which
$\det \{ \varphi_a, \varphi_b \} \ne 0 $: a system of constraints
for which this condition holds is called a system of {\em second
class constraints}. If a Lagrangian 
gives rise to  a system of second class constraints, the next
step of reduction leads to a nondegenerate Lagrangian. The number of
$\phi_a$ must be  $2N$ and the dimension of the phase space is then 
reduced by $N$ through imposition of the constraints. 
Constrained quantization was originally considered by 
Dirac \cite{Dirac}. 



\noindent{\bf Example 1: the free scalar field} 

Given a field $\varphi(t, \vx) $, and the free field
Lagrangian \beq \label{lagr1} 
\calL = \frac{1}{2}(\partial_\mu \varphi)^2, \eeq
canonical quantization normally
specifies that one should introduce the canonically conjugate
momentum $\pi(t, \vx)  = \partial_0 \varphi (t, \vx)$, where
$x_0 = t $ and $(x_1, x_2, x_3) = \vx$. The points $\vx$ in space
should be thought of as labelling the field. Notice that the momentum
$\pi$ is not Lorentz invariant.

 We may alternatively replace the
 Lagrangian (\ref{lagr1}) (which is second order in 
time derivatives) by a first order Lagrangian: we do this by 
introducing a vector field $\varphi_\mu$ (in other words a 
collection of four fields $(\varphi_0, \varphi_1, \varphi_2, \varphi_3)$
 which transform according to the standard action of the Lorentz group on 
$\RR^4$) and write
\beq \label{lagr2}
l = (\partial_\mu \varphi ) \varphi^\mu - \frac{1}{2} (\varphi^\mu)^2. \eeq
Of course under the substitution 
$\varphi_\mu = \partial \varphi/\partial x^\mu $ the Lagrangian in 
(\ref{lagr2}) reduces to that in (\ref{lagr1}): however if
we regard $\varphi_\mu$ as independent of $\varphi$,  (\ref{lagr2}) is 
first order in time derivatives. 
(We have introduced the Einstein summation convention, summing over
 pairs of repeated indices $\mu$, where 
$\mu = \{ 0, k \} $ and $k = 1, 2, 3$ correspond to space coordinates
while $\mu = 0 $ corresponds to  the time coordinate.)
We may rewrite (\ref{lagr2}) as 
\beq l = (\partial_0 \varphi ) \varphi^0 - (\partial_k \varphi ) \varphi^k
 - \frac{1}{2} \varphi_0^2 +\frac{1}{2} \varphi_k^2. \eeq
Under the substitution 
$\pi = \varphi_0  $, and solving for $\varphi_k, $ 
$\varphi_k = \partial_k \varphi$, we obtain 
\beq \calL 
= (\partial^0 \varphi) \pi - \frac{1}{2} (\pi^2 + (\partial_k \varphi)^2 ),
\eeq 
which indeed shows that $\pi$ and $\varphi$ are 
canonically conjugate and that the Hamiltonian $H$ has the usual 
form 
$$ H = \half (\pi^2 + (\partial_k \varphi)^2. ) $$
The variables $\varphi_k$ can be eliminated because they enter
(\ref{lagr2}) quadratically and without time derivatives.

\noindent{\bf Example 2: electromagnetism}

In this example it is important that spacetime has dimension 4. 
We start with a vector field $A_\mu(x) $ (the photon field),
 which should be thought of as 
a 1-form or a connection:
$$ A = \sum_\mu A_\mu dx^\mu. $$
The associated curvature is
$$F = dA,$$ or as a two-form $F_{\mu \nu} dx^\mu \wedge dx^\nu, $ where
\beq \label{curv} 
F_{\mu \nu} = \partial_\mu A_\nu - \partial_\nu A_\mu. \eeq
The Lagrangian is 
\beq \label{lagel1} \calL = \int |F|^2, \eeq
which is second  order in time derivatives. A trick due to 
J. Schwinger enables us to replace it with a Lagrangian which 
is first order in time derivatives: this is to regard
$A$ and $F$ as independent variables. By analogy with our procedure
 in Example 1, we write
\beq  \label{lagel2} 
\calL = (\partial_\mu A_\nu - \partial_\nu A_\mu) F^{\mu \nu} 
- \frac{1}{2} F_{\mu \nu}^2. \eeq
Of course (\ref{lagel2}) reduces to (\ref{lagel1}) by restoring 
the functional dependence (\ref{curv}) of $F$ on $A$. However 
the Lagrangian in (\ref{lagel2})  is linear in time derivatives of the 
fields, and also introduces some constraints. 

We rewrite (\ref{lagel2}) as 
\beq \label{lagel3} 
\calL = (\partial_0 A_k) F^{0k} + A_0 (\partial_k F^{0k}) - 
F^{ik} (\partial_i A_k - \partial_k A_i) - \frac{1}{2} (F^{0k})^2 
+ \frac{1}{2} (F^{ik})^2. \eeq
Here, $F^{0k} = E^k$ are the components of 
the {\em electric field } $\vec{E}$, while
$A_k$ are the components of 
 the {\em vector potential}.
 Up to the Hodge star operator $*$ 
which converts two-forms  on $\RR^3$ into one-forms,
the quantity $F^{ik}$ is the 
{\em magnetic field} $\vec{H}$: we substitute 
\beq \label{magdef}  \vec{H} = \nabla \times \vec{A}, \eeq
or 
$H_i = \epsilon_{ijk} \nabla_j A_k $
in terms of the totally antisymmetric 
 tensor $\epsilon$ in three indices which
represents the Hodge star operator in $\RR^3$. 
Thus the Lagrangian becomes
\beq \label{lagel4}
\calL = (\partial_0 A_k) E^k - \frac{1}{2} (\vec{E}^2 - \vec{H}^2) + 
A_0 \partial_k E^k, \eeq
where we already excluded the independent variables $\vec{H}$, so that
$\vec{H}$  in (\ref{lagel4}) is given by (\ref{magdef}). 
On the other hand, $A_k$  and $E^k$ are canonically conjugate variables 
(appearing in a term which is first order in time derivatives), and 
$A_0$ is a Lagrange multiplier multiplying the 
constraint
\beq G = \nabla \cdot E = \partial_k E^k . \eeq
The constraint $G = 0 $ is called {\em Gauss's law}: it tells us
that the divergence of the electric field is zero. This is a first 
class constraint, $ \{ G(\vx), G(\vy) \} = 0 $, and we have also 
\beq \{ G, \vec{E} \} = \{ G, \vec{H} \} = 0 . \eeq
For any function $\lambda$ on $\RR^3$ 
we have 
$$ \{ A_k, \int G(\vx) \lambda (\vx) d \vx \} = \partial_k \lambda(\vx); $$
equivalently $G$ is a generator of $U(1)$ gauge transformations: 
the gauge group $\calG$ is the group of maps from $\RR^3$ to $U(1)$, and 
its Lie algebra is the group of maps from $\RR^3$ to $\RR$. An element
$\lambda \in {\rm Lie} (\calG) $ 
sends $A_k $ to $A_k  + \partial_k 
 \lambda$. The imposition of Gauss's law may be viewed
as symplectic reduction  with respect to the action of the $U(1)$ gauge
group $\calG$.

Since the constraint $G$ is  of first class, it reduces the degrees
of freedom from three functions $A_k(\vx)$ to two. Thus
(in physical language) light has two polarizations. 



The above treats the situation of the electromagnetic field in the 
absence of sources. If on the other hand the electric field 
interacts with a charged field $J^\mu$, Gauss's law becomes  
\beq \label{gaussmod}  G = \nabla \cdot E + J_0(\vx) \eeq
where $J_0 $ is the {\em charge density}. If we define
$$ Q = \int_{\RR^3}  J_0 dx $$ 
if we express $Q$ as a flux of the electromagnetic field 
via a  surface in $\RR^3$ (for instance a sphere $S^2(R)$ of 
radius $R$) 
we obtain 
$$ Q = \lim_{R \to \infty} \int_{S^2(R)} \epsilon_{ikj }
E^k dS_{ij}. $$
Thus the equation 
(\ref{gaussmod})   leads to {\em Coulomb's law}, 
according to which the electromagnetic force from a point
charge  at distance $r$ is proportional to 
$1/r^2$. 



\noindent{\bf Example 4: Gravitation}

In this example we introduce a metric $g^{\mu \nu}$ with the associated
{\em affine connection} $\Gamma_{ij}^k$ and the scalar curvature 
(we omit indices for clarity)
$$ R = \partial  \Gamma + \Gamma^2. $$
To write  a Lagrangian that is  first order in time derivatives, 
we take $g$ and $\Gamma$ as independent variables: more precisely we write
$$ h ^{\mu \nu} = \sqrt{g} g^{\mu \nu},$$  and write the Lagrangian as 
\beq \label{laggrav1} \calL = h (\partial \Gamma + \Gamma^2). \eeq
The connection $\Gamma$ contains first order time derivatives of $h$, so 
$\partial \Gamma$ contains second order time derivatives of $h$. 
This may be altered by integrating by parts, so that we obtain
\beq \label{laggrav2}
\calL = - \Gamma \partial h + h \Gamma^2. \eeq
This Lagrangian is not manifestly  covariant, however it is  this
form which is to be 
used in the so called {\em asymptotically flat} case. 

The definition of asymptotic flatness can be given on different levels of 
sophistication; we shall use the most na\"{\i}ve but practical 
way of introducing the admissible coordinates. In such 
coordinates we have in  the vicinity of space infinity 
$$ g = g_0 + O (\frac{1}{r} ), $$ 
$$ \partial g = O (\frac{1}{r^2} )$$ 
$$\Gamma  = O (\frac{1}{r^2}) $$
where $g_0$ is the Minkowski metric. 
%Dottie: please insert the  text here. -- Lisa Jeffrey
An admissible change of variables is of the form
\beq \label{coords}
X_\mu\to \Lambda_{\mu v}X_r+a_\mu+
b_\mu(x),
\eeq
where $\Lambda_{\mu v}$ is a Lorentz rotation, $a_\mu$
defines  a translation  and
$$
b_\mu(X)= O\left(\frac{1}{r} \right),\quad
\partial b=O\left(\frac{1}{r^2}\right).
$$
In these coordinates the density (\ref{laggrav2}) is of
order $O\left(\frac{1}{r^4}\right)$ and the action
$$
A=\int\calL\,dx
$$
is invariant with respect to the change of coordinates
(\ref{coords}).

Inspection of the density (\ref{laggrav2}) cubic in the 
variables $(h,\Gamma)$, shows that out of
$ 10 + 40 = 50$  variables, the  $\Gamma_{00}^\mu$ enter
linearly and 26 other components $\Gamma$ enter
quadratically without time derivatives.
So the Lagrangian (\ref{laggrav2}) takes the form $(\ref{elimexcl})$, 
however the 
constraints do not commute with the corresponding $\phi$.
So we must solve them explicitly.
Fortunately this is possible and altogether solving the
constraints and eliminating the excludable variables
allows one to express 30 components of $\Gamma$ via the metric
$h^{\mu \nu}$ and 6 components $\Gamma_{ik}^0$.
Let 
\begin{eqnarray*}
q^{ik} &=& h^{00}h^{ik}-h^{in}h^{nk}\\
\Pi_{ik} &=& \frac{\Gamma_{ik}^0}{h^{00}}
\end{eqnarray*}
be a (density of weight $+2$) of the first
quadratic form and (a density of weight $-1$) of the
second quadratic form of the surface ${x}_0=0$ embedded
into space-time.
After solving for $\Gamma$, mentioned above, the
Lagrangian (\ref{laggrav2}) reduces to the form
$$
\Pi_{ik}\partial_0 q^{ik}+\frac{1}{h^{00}}C_0(\Pi,q)+
\frac{h^{0k}}{h^{00}}C_k(\Pi,q)+\partial_i\partial_k
q^{ik}
$$
where $C_0$ and $C_k$ are some functionals of $\Pi$,
$q$ and  derivatives.
The term $\partial_i\partial_k q^{ik}$ cannot be
discarded in spite of the fact that it is a pure divergence as its
flux through infinity does not vanish.
Indeed this flux
$$
\calM=\lim_{R \to \infty}\int\limits_{S_R}
\epsilon_{k\ell m}\partial_{ik}q^{ik}dS^{\ell n}
$$
defines the full energy $\calM$ of the gravitational
field and a very nontrivial theorem states  that
$\calM$ is positive unless the metric is not flat.

Four constraints $C_0$, $C_k$ are of the first class;
let $\eta_k(x)$, $\eta_0(x)$ be the vector field and the 
function on the space slice.
We put
\begin{eqnarray*}
C_0(\eta_0) &=& \int C_0(\Pi(x),q(x))\eta_0(x)d^3 x\\
C(\eta) &=& \int C_k(\Pi(x),q(x))\eta^k(x)d^3x
\end{eqnarray*}
to get the brackets
$$
\{C(\eta''),C(\eta')\}=
C([\eta^n,\eta'])
$$
where $[\quad,\quad]$ is the commutator of vector
fields.
This means that $C(\eta)$ are generators of
diffeomorphisms of our space slice.
Furthermore
$$
\{C(\eta),C_0(\eta_0)\}=
C_0(\eta\eta_0)
$$
where $\eta\eta_n$ is the  result of the action of the vector
field $\eta$ on the function $\eta_0$.
Finally
$$
\{C_0(\eta''_0),C_0(\eta'_0)\}=
C(\hat{\eta}),
$$
where the vector field $\hat{\eta}$ is given by
$$
\hat{\eta}^n=q^{in}(\partial_i 
\eta''_0\eta'_0-\eta''_0\partial_i \,
\eta'_0)
$$
What about the Hamiltonian?
Taking into account the asymptotic condition, it is
more natural to consider
$\left(\frac{1}{h^{0n}}+1\right)$ rather than
$1/h^{nm}$ to be a lagrangian multiplier; thus, the
Hamiltonian
$$
H=C_0 + \partial_i \partial_n q^{in}
$$
differs from the constraint $C_{0}$ only by  a term  which is a 
total divergence.
The explicit expression for $C_0$ shows that $H$ is
quadratic in $\Pi$ and first derivatives of $q$, like  all 
other Hamiltonians in previous examples.
The $4$ constraints kill $4$ degrees of freedom from the 6
in $q^{ik}$, so gravitons (like photons above) have two
polarizations.

The reduction above is associated with the names
of Dirac  and Arnowitt-Deser-Misner [ADM].
The explicit forms of all expressions above in the
coordinates $q^{ik}$, $\Pi_{ik}$ can be found in my
survey \cite{F:grav}.


%% end of inserted text
\noindent{\bf Example 4: the Dirac equation} 

In this lecture, lack of time prevents us from 
treating   the case of the Dirac equation,
in which the field is  a spin $1/2$ fermion, and the Lagrangian is 
first order in time derivatives from the beginning and is nondegenerate.







\begin{thebibliography}{99}
\bibitem{Dirac} P.A.M.  Dirac, {\em Proc. Roy. Soc.} {\bf A 246}, 
326, 1958; P.A.M. Dirac, {\em Lectures on Quantum Mechanics}, New York
(Yeshiva University), 1964. 
\bibitem{F:grav} L.D. Faddeev, Sov. Phys. Uspekhi {\bf 25} (1982) %poss 92?
130.
\bibitem{FS} L.D. Faddeev, A.A.  Slavnov, {\em Gauge Fields: An Introduction 
to Quantum Theory}, Addison-Wesley (Frontiers in Physics vol. 83), (second 
edition), 1991. 
\end{thebibliography}


\end{document}





