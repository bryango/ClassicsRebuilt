%From: Pierre Deligne <deligne@math.ias.edu>
%Date: Fri, 11 Oct 1996 15:49:17 -0400

%From: Pierre Deligne <deligne@math.ias.edu>
%Date: Fri, 4 Oct 1996 16:18:43 -0400
%Subject: spinors.tex

\input amstex
\documentstyle{amsppt}
\magnification=1200
\input pictex
\loadeufm

\font\dotless=cmr10 %for the roman i or j to be
                    %used with accents on top.
                    %(\dotless\char'020=i)
                    %(\dotless\char'021=j)
\font\itdotless=cmti10
\def\itumi{{\"{\itdotless\char'020}}}
\def\itumj{{\"{\itdotless\char'021}}}
\def\umi{{\"{\dotless\char'020}}}
\def\umj{{\"{\dotless\char'021}}}
\font\thinlinefont=cmr5
\font\smaller=cmr5
\font\boldtitlefont=cmb10 scaled\magstep2

\NoRunningHeads
\pagewidth{6.5 true in}
\pageheight{8.9 true in}
\loadeusm

\catcode`\@=11
\def\logo@{}
\catcode`\@=13

\def\eps{{\varepsilon}}

\def\undertext#1{$\underline{\vphantom{y}\hbox{#1}}$}
\def\nspace{\lineskip=1pt\baselineskip=12pt%
     \lineskiplimit=0pt}
\def\dspace{\lineskip=2pt\baselineskip=18pt%
     \lineskiplimit=0pt}

\def\wedgeop{\operatornamewithlimits{\wedge}\limits}
\def\oplusop{\operatornamewithlimits{\oplus}\limits}
\def\simover#1{\overset\sim\to#1}
\def\w{{\mathchoice{\,{\scriptstyle\wedge}\,}
  {{\scriptstyle\wedge}}
  {{\scriptscriptstyle\wedge}}{{\scriptscriptstyle\wedge}}}}
\def\Le{{\mathchoice{\,{\scriptstyle\le}\,}
{\,{\scriptstyle\le}\,}
{\,{\scriptscriptstyle\le}\,}{\,{\scriptscriptstyle\le}\,}}}
\def\Ge{{\mathchoice{\,{\scriptstyle\ge}\,}
{\,{\scriptstyle\ge}\,}
{\,{\scriptscriptstyle\ge}\,}{\,{\scriptscriptstyle\ge}\,}}}
\def\doubleheaddownarrow#1{\hbox{$\Big\downarrow%
     \kern-5.96pt\lower2pt\hbox{$\downarrow$}$}%
     \rlap{$\vcenter{$\kern-18pt\scriptstyle#1$}$}}
\def\rmapdown#1{\Big\downarrow\kern-1.0pt\vcenter{
     \hbox{$\scriptstyle#1$}}}

\def\Gal{\text{\rm Gal}}  \def\sign{\text{\rm sign}}
\def\Br{\text{\rm Br}}  \def\im{\text{\rm Im}}
\def\sBr{\text{\rm sBr}} \def\Spin{\text{\rm Spin}}
\def\sgn{\text{\rm sgn}} \def\pr{\text{\rm pr}}
\def\Id{\text{\rm Id}} \def\Sp{\text{\rm Sp}}
\def\Gr{\text{Gr}} \def\SO{\text{\rm SO}}
\def\Sym{\text{\rm Sym}} \def\SL{\text{\rm SL}}
\def\End{\text{\rm End}}
\def\Hom{\text{\rm Hom}} 
\def\Ker{\text{\rm Ker}}
\def\Tr{\text{\rm Tr}}
\def\Lie{\text{\rm Lie}} 

\def\fbar{\bar{f}}
\def\kbar{\bar{k}}
\def\zbar{\bar{z}}
\def\Abar{\bar{A}}
\def\Sbar{\bar{S}}

\def\dbC{{\Bbb C}}
\def\dbG{{\Bbb G}}
\def\dbH{{\Bbb H}}
\def\dbR{{\Bbb R}}
\def\dbZ{{\Bbb Z}}


\def\scr#1{{\fam\eusmfam\relax#1}}

\def\scrD{{\scr D}}
\def\scrL{{\scr L}}
\def\scrS{{\scr S}}   
\def\scrU{{\scr U}}   

\def\gr#1{{\fam\eufmfam\relax#1}}

%Euler Fraktur letters (German)
\def\gro{{\gr o}}
\def\grs{{\gr s}}

\def\grso{{\grs\gro}}


\topmatter
\title\nofrills
{\boldtitlefont Notes on spinors}
\endtitle
\author
P. Deligne
\endauthor
\endtopmatter

%\bigskip
\centerline{School of Mathematics}
\centerline{Institute for Advanced Study}
\centerline{Princeton, NJ \ 08540}

\smallskip
\centerline{e-mail: deligne\@math.ias.edu}

\smallskip
\centerline{(corrected October 11, 1996)}

\NoBlackBoxes
\parindent=20pt
\frenchspacing
\document
\dspace
\bigskip
The present notes owe a lot to Bourbaki's treatment in
Alg., Ch. 9.
I have learned of the analogy between spinorial and
oscillator representations (see 3.5, 3.8, 3.10) from a
lecture R. Howe gave in 1978.
The \S5 has been inspired by a conversation with
Witten.

\bigskip\bigskip
\subhead
1. Super Brauer group
\endsubhead

Our aim is to explain and investigate the super
analogues of the notions of division algebra,
semi-simple algebra, central simple algebra and Brauer
group.

The super Brauer group has been introduced by
C.~T.~C.~Wall: Graded Brauer Groups, J. Reine Angew.
Math. 213 (1963-64), p. 187--199.
In loc. cit., Theorem 3 p. 194, a description of its
structure more precise than our 1.21 is given.

\subhead
1.1 
\endsubhead
The sign rule forces the following definitions.
The {\it opposite} $A^0$ of a super ring $A$ has the
same underlying additive group, and its multiplication
law is given by
$$
(x,y)\longmapsto (-1)^{p(x)p(y)}y\,x
$$
for $x$, $y$ homogeneous, with parity
$p(x),p(y)\in\dbZ/2$.
A left super $A$-module is turned into a right super
$A^0$-module by
$$
m\,a=(-1)^{p(m)p(a)}a\,m.
$$

If $M$ an $N$ are two left super modules,
$\underline{\Hom}_A(M,N)$ is $\mod{2}$ graded, with even
(resp. odd) part the additive group of even (resp odd)
additive morphisms $f\colon\,M\to N$ such that
$f(a\,m)=(-1)^{p(f)p(a)}a,f(m)$.
The even part of $\underline{\Hom}_A(M,N)$ is the group
$\Hom_A(M,N)$ of morphisms from $M$ to $N$ in the
category of left super modules .

If $f\colon\,A\to B$ is a morphism of super rings, the
{\it commutant} of $A$  in $B$ is the super ring of
$b\in B$ for which the super commutator
$ab-(-1)^{p(a)p(b)}ba$ vanishes for $a\in f(A)$.
The {\it commutant} $A'$ of a left super $A$-module $M$
is the commutant
$\underline{\End}_A(M)=\underline{\Hom}_A(M,M)$ of $A$
in $\underline{\End}_\dbZ(M)$.
Example:
for $M$ the left module $A$, $A'=\underline{\End}_A(A)$
is identified with $A^0$ by $f\mapsto f(1)$.
If we view $M$ as a right $A'{^0}$-module, the ring of
right multiplications by elements of $A'{^0}$ is the
commutant in the ordinary, ungraded, sense.
The bicommutant, i.e. the commutant of $M$ viewed as an
$A'$-module, coincide with its ungraded analogue.

The {\it center} of $A$ is the commutant of $A$ in $A$.

If $Z$ is super commutative and $A$, $B$ are super
algebras over $Z$, i.e. are provided with
$a\colon\,Z\to A$ and $b\colon\,Z\to B$, the tensor
product algebra $A\otimes_Z B$ is $A\otimes_Z B$ with
the product law $(a\otimes b)$.
$(a'\otimes b')=(-1)^{p(b)p(a')}(aa')\otimes(bb')$:
the image of $A$ and $B$ in $A\otimes_Z B$ super
commute.

A divisor ring ($=$ field, possibly non commutative) is
a ring $D$ such that the left $D$-module $D$ is simple.
Adding ``super'' everywhere, one obtains the notion of
super division ring.
Equivalent definition: $D\not=0$ and every homogeneous
element $x\not=0$ is invertible.
The usual proof gives:

\proclaim{Proposition 1.2}
The commutant of a simple super module is a super
division ring.
\endproclaim

\demo{Proof}
If $a'\in A'$ is homogeneous, its kernel and image are
graded submodules$\ldots$
\enddemo

A left super module $M$ is {\it semi-simple} if it is a
sum of simple super modules.

\proclaim{Proposition 1.3}
Assume that $2$ is invertible in $A$.
Then, a left super-module $M$ is semi-simple if and
only if the underlying ungraded module $\vert M\vert$
is.
\endproclaim

\demo{Proof}
Only if: it suffices to see that if  $M$ is simple, the
underlying ungraded module $\vert M\vert$ is
semi-simple.
Let $M_1$ be the sum of the simple submodules of $\vert
M\vert$.
The parity map $\alpha\colon\,x\mapsto(-1)^{p(x)}x$ is
an automorphism of $(A,M)$.
By transport of structures, $M_1$ is stable by $\alpha$.
It is hence a graded submodule (we use here that $2$ is
invertible).
The simplicity of $M$ forces $M=M_1$ and $\vert
M\vert=M_1$ is semi-simple.
\enddemo

If: It suffices to see that any super submodule $N$ of $M$
is a direct factor.
By assumption, $\vert N\vert$ is a direct factor of
$\vert M\vert$, meaning that there is a retraction
$f\colon\,\vert M\vert\to\vert N\vert$.
The map $(f+\alpha(f))/2$ is a graded retraction; it
makes $N$ a direct factor of $M$.

{\it From now on in this {\rm \S}, $k$ is a field of
characteristic $\not=2$.
We say \undertext{algebra} and \undertext{module} for
super algebra and super left module.
Non super objects will be called ungraded.
Algebras and modules are assumed finite dimensional
over the ground field {\rm (}usually $k${\rm )}.}

\subhead
1.4
\endsubhead
An algebra $A$ is {\it semi-simple} if the category of
its modules is semi-simple.
Equivalent conditions: the $A$ module $A$ is
semi-simple; $A$ admits a faithful semi-simple module
$M$.

\demo{Proof}
We write $\prod$ for the parity change functor
$X\mapsto X\otimes k$, for $k$ viewed as an odd
$k$-vector space, with basis noted $1^-$.
The equivalence of the conditions follows from:
any module is a quotient of some $A^n\oplus\left(\prod
A\right)^m$; if $M$ is faithful, $A$ is a submodule of
some $M^n\oplus\left(\prod M\right)^m$.
\enddemo

Corollaries of 1.3:

\proclaim{Corollary 1.5}
An algebra is semi-simple if and only if the underlying
ungraded algebra is.
\endproclaim

\proclaim{Corollary 1.6}
If $M$ is a semi-simple module, $A$ maps onto its
bicommutant $A''$.
\endproclaim

By 1.3 and the identity 1.1 of the bicommutant with its
ungraded analog, we are reduced to the ungraded case
(Bourbaki Alg. Ch. 8, \S4 n$^o$ 2, Th. 1)

\proclaim{Corollary 1.7}
If $k'$ is a separable extension of $k$, $A$ is
semi-simple if and only if the $k'$-algebra $A\otimes_k
k'$ is.
\endproclaim

By 1.5, we are reduced to the ungraded case (Bourbaki
Alg. Ch. 8, \S7 n$^o$ 3, cor. 2 to prop. 3).

\subhead
1.8
\endsubhead
An algebra $A$ is {\it central simple} if it is
semi-simple with center $k$.
The condition ``center $k$'' implies that $A\not=0$.
It does not imply that the underlying ungraded algebra
$\vert A\vert$ has a center reduced to $k$.
Example: the algebra $k[\eps]$ with $\eps$ odd,
$\eps^2=1$ is a (super) division algebra, hence is
semi-simple.
The non vanishing of the super commutator
$[\eps,\eps]=2\eps^2$ shows that its center is reduced
to $k$: it is central simple, but $\vert A\vert$ is not.

\proclaim{Corollary 1.9}
If $k'$ is a separable extension of $k$, $A$ is central
simple if and only if the $k'$-algebra $A\otimes_k k'$
is.
\endproclaim

\demo{Proof}
Results from 1.7, and from the fact that taking the center
is compatible with extension of scalars.
Indeed, the center is the intersection of the kernel of
the super commutators maps $adx$, for $x$ running over a
system of generators of the algebra.
\enddemo

\remark{\bf Remark}
The conclusion of 1.9 holds for any field extension
$k'/k$.
This results from 1.14.
\endremark

\subhead
1.10
\endsubhead
Let $D$ be a division algebra.
If $D$ is purely even, $D$ has two isomorphism classes
of simple modules: $D$, viewed as purely even, or as
purely odd.
If $D$ is not purely even, $D$ has only one isomorphism
class of simple modules.
In the purely even case, $M_{r,s}(D)$ is
correspondingly isomorphic to $M_{r+s}(D)$.
By ``matrix algebra over $D$'', we mean in all cases an
algebra $M_{r,s}(D)$, with $r+s>0$.

\proclaim{Proposition 1.11}
An algebra $A$ is central simple if and only if it is
isomorphic to a matrix algebra over some division
algebra $D$ with center $k$.
\endproclaim

\demo{Proof}
Let $A$ be a $k$-algebra.
If $F$ is a composition series of the $A$-module $A$,
any simple $A$-module occurs, up to parity change, in
the successive quotients of $F$.
It follows that they are only a finite number of
isomorphism classes of simple modules.
Let $S_1,\dotsc,S_n$ be a system of representatives of
them, up to parity change.
Let $D_i$ be the commutant of $S_i$.
If $D_i$ is not purely even, the $D_i$-module $S_i$ is
isomorphic to $D_i^n$ for some $n\not=0$.
if $D_i$ is purely even, $D_i^{n\vert m}$ occurs as
well.
By 1.6,
$$
A\longrightarrow \prod\,\underline{\End}_{D_i}(S_i)
$$
is onto.
It is an isomorphism if $A$ is semi-simple.
\enddemo

If $A$ is central simple, we have $n=1$:
up to parity change, $A$ has exactly one isomorphism
class of simple modules $S$.
If $D$ is its commutant, we have
$$
\align
A\simover{\longrightarrow}\underline{\End}_D(S)=
&M_n(D^0)\quad\text{or}\\
&M_{n,m}(D^0)
\endalign
$$
for $D^0$ the opposite algebra.
Indeed, $\End_D(D)=D^0$.
By 1.6, the center $A\cap D$ of $A$ is also the center
of $D$: if $A$ is central simple, the center of $D$ is
reduced to $k$ and $A$ has the required form.
The converse is left to the reader.
The proof gave the

\remark{\bf Complement 1.12}
The algebra $D$ of 1.11 is $(\End_A(S))^0$ for $S$ the
unique simple $A$-module (up to isomorphism and parity
change).
It is hence determined, up to isomorphism, by $A$.
\endremark

\proclaim{Proposition 1.13}
If $k$ is separably closed, the only division algebras
$D$ with center $k$ are, up to isomorphism, $k$ and
$k[\eps]$ with $\eps$ odd and $\eps^2=1$.
\endproclaim

\demo{Proof}
The even part $D^+$ of $D$ is an ordinary division
algebra.
Let $Z$ be its center.
The center of $D$ being reduced to $k$, it follows that
the fixed point set of the automorphisms $xzx^{-1}$
($x\in D^-$, $x\not=0$) of $Z$ is $k$.
We see that $Z$ is a Galois extension of $k$.
As $k$ is separably closed, $Z=k$ and $D^+=Z=k$, because
$\Br(k)=0$.

If $D$ is purely even, we get $D=k$.
If not, pick $\eps\not=0$ in $D^-$.
It is invertible.
It follows that $D^-=\eps D^+$ and that $\eps^2\not=0$.
As $k$ is separably closed of characteristic $\not=2$,
$\eps^2$ has a square root $\lambda$ in $k$.
Replacing $\eps$ by $\lambda^{-1}\eps$, we obtain that
$D$ is isomorphic to the $k[\eps]$ of 1.13.
\enddemo

\proclaim{Corollary 1.14}
An algebra $A\not=0$ is central simple if and only if 
$$
A\otimes A^0 \simover{\longrightarrow} \End_k(A).
\tag1.14.1
$$
\endproclaim

The map is defined by the commuting $A$ and
$A^0$-module structures of $A$, the $A^0$-module
structure coming from $A^0=\End_A(A)$.

If $A$ is not semi-simple (resp. with center $k$), the
image of $A\otimes A^0$ in $\End_k(A)$ respects the
radical of $A$ (resp. commutes with the center), hence
cannot be all of $\End_k(A)$.

We now assume $A$ central simple and prove that
$A\otimes A^0\simover{\longrightarrow} \End_k(A)$.
Extending the scalars, we may assume that $k$ is
separably closed.
By 1.11 and 1.13, one has then $A\sim M_{r,s}\otimes D$,
with $D$ as in 1.13.
this reduces us to the cases of $M_{r,s}$, left to the
reader, and $D$.
If $D=k$, the statement is clear.
For $K=k[\eps]$, in the basis $(1,\eps)$, the left
multiplication by $\eps\in D$ (resp. $\eps\in D^0$) is
$$
\pmatrix\format \c &\quad\c\\
0 &1\\
1 &0
\endpmatrix
\qquad\qquad \text{(resp. $\pmatrix\format \c &\quad\r\\
0 &-1\\
1 &0
\endpmatrix\,\,$)}
$$
which suffices to generate the $2\times 2$ matrix
algebra.

\proclaim{Corollary 1.15}
The tensor product of two non zero algebras is central
simple if and only if each is.
\endproclaim

\demo{Proof}
The map (1.14.1) for a tensor product is the tensor
product of the maps (1.14.1) for each factor.
\enddemo

\proclaim{Definition 1.16}
\roster
\runinitem"(i)"
Two central simple algebras are {\rm similar} if they
are matrix algebras over isomorphic division algebras
(cf. 1.11, 1.12).
\endroster

\noindent
{\rm (ii)}\enspace
The {\rm super Brauer group} $\sBr(k)$ is the set of 
similarity classes of central simple algebras over $k$,
with composition law induced by $\otimes$.
\endproclaim

The composition law is associative and commutative, by
associativity and commutativity of the tensor product.
It has a unit: the class of matrix algebras.
By 1.14, the class of $A^0$ is inverse to that of $A$:
$\sBr(k)$ is a commutative group.

\subhead
1.17 Example
\endsubhead
By 1.13, if $k$ is separably closed, then
$\sBr(k)=\dbZ/2$.

The trivial element in $\sBr(k)$ corresponds to the
central simple algebras $A=\underline{\End}_k(S)$, for
$S$ a (super) vector space.
The commutant $Z$ of $A^+$ in $A$ is contained in
$A^+$:
it is the center $Z$ of $A^+$ and consists of those
endomorphisms of $S$ which act as scalars on 
$S^+$ and on $S^-$.
If $A$ is not purely even, i.e. if $S$ is not
homogeneous, $Z\simeq k\times k$, and the two
homomorphisms $\chi\colon\,Z\to k$ correspond to the
two isomorphism classes of simple $A$-modules $T$ (permuted
by parity change) by $T\mapsto \chi$ such that $zv=\chi(z)v$
for $v\in T^+$.

The non trivial element in $\sBr(k)$ corresponds to the
central simple algebras $A=M_n(D)$, $D=k[\eps]$ as
above.
Here, $A^+$ is central simple and its commutant $Z$ in
$A$ is of dimension $(1,1)$ (isomorphic to $D$).
One has $A^+\otimes Z\simover{\rightarrow} A$.

\proclaim{Proposition 1.18}
the super Brauer group of $\dbR$ is $\dbZ/8$.
It is generated by the super division algebra
$\dbR[\eps]$ with $\eps$ odd and $\eps^2=1$.
\endproclaim

\demo{Proof}
We first enumerate all super division algebras $D$ over
$\dbR$.
If, after extending scalars to $\dbC$, $D$ becomes
isomorphic to a matrix algebra over $\dbC[\eps]$ (non
trivial class in $\sBr(\dbC)$), then $D^+$ is a
division algebra, its commutant $Z$ is a division
algebra of dimension $(1,1)$ and $D^+\otimes
Z\simover{\rightarrow}D$.
Indeed, all those assertions hold after extension of
scalars to $\dbC$, and are invariant by extension of
scalars.
This gives four possibilities: $D^+=\dbR$ or $\dbH$,
and $Z=\dbR[\eps]$ with $\eps$ odd and $\eps^2=\pm 1$.
It follows already that $\sBr(\dbR)$ is of order $8$.

For $D$ purely even, we get the usual Brauer group:
$D=\dbR$ or $\dbH$.

If $D$ is not purely even and becomes a matrix algebra
over $\dbC$, the center of $D^+$ is a non trivial
extension $Z$ of $\dbR$, as one checks after extension
of scalars to $\dbC$.
It follows that one has $C\simeq \dbC$ and $D^+$,
central simple over $Z$, is reduced to $\dbC$ as well.
Fix $\eps\not=0$ in $D^-$.
One has $D^-=D^+.\eps$.
the center of $D$ being reduced to $\dbR$, the
automorphism $\eps z\eps^{-1}$ of $\dbC$ must be non
trivial: $\eps z\eps^{-1}=\zbar$.
It fixes  $\eps^2\colon\,\eps^2\in\dbR$.
Changing $\eps$ to $\eps z$ changes $\eps^2$ to
$\eps^2 z\zbar$.
This gives the two last cases
$$
D=\dbC[\eps]\quad\text{with $\eps$ odd, $\eps
z\eps^{-1}=\zbar$ and $\eps^2=\pm1$}.
$$
\enddemo

We now check that the map from $\dbZ/8$ to the group
$\sBr(\dbR)$ defined by $1\mapsto$ (class of
$\dbR[\eps]$, $\eps$ odd, $\eps^2=1$), is as follows:
$$
\alignat4
&0 &&\longmapsto \dbR &\qquad &4 &&\longmapsto \dbH\\
&1 &&\longmapsto \dbR[\eps], \eps^2=1
      && 5 &&\longmapsto\dbH[\eps], \eps h=h\eps, \eps^2=1\\
&2 &&\longmapsto \dbC[\eps], \eps z=\zbar\eps,
     \eps^2=1 &&6 &&\longmapsto\dbC[\eps],
     \eps z=\zbar\eps, \eps^2=-1\\
&3 &&\longmapsto \dbH[\eps], \eps h=h\eps, \eps^2=-1
  &&7 &&\longmapsto\dbR[\eps], \eps^2=-1
\endalignat
$$

The image of $2$ is the class of $\dbR[\eps_1,\eps_2]$
with $\eps_i^2=1$, $\eps_1\eps_2=-\eps_2\eps_1$.
One has $(\eps_1\eps_2)^2=-1$ and the claim for $2$
 follows.
Passing to the opposite algebra, we get the image of
$6$.
It is not the same as the image of $2$, and it follows
that $\sBr(\dbR)$ is indeed cyclic of order $8$.
As the class of $\dbH$ is of order $2$, it must be the
image of $4$.
One concludes by using that $5=4+1$ and using passages
to the opposite algebra.

\remark{\bf Remark 1.19}
Let $A$ be a central simple algebra over $\dbR$ with a
class $n$ in $2.\dbZ/8$, i.e. such that $D_\dbC$,
deduced from $A$ by extension of scalars, is a matrix
algebra.
The algebra $A_\dbC$ has two isomorphism classes of
simple modules, permuted by parity change.

Each class is defined over $\dbR$, i.e. is fixed by
$\Gal(\dbC/\dbR)$, if and only if the center $Z$ of
$A^+$ is $\dbR$ (if $A$ is purely even) or
$\dbR\times\dbR$.
This happens if and only if $n\equiv 0(4)$.
if $n\equiv 0(8)$, each isomorphism class can be
realized over $\dbR$: $A$ is a matrix algebra.
If $n\equiv 4(8)$, it cannot: $A$ admits only a module
$M$ which by extension of scalars becomes twice a
simple module over $A_\dbC$.
Its commutant is $\dbH$.

If $n=2(8)$, the ungraded algebra $A$ admits an
ungraded module $M$ which, after extension of scalars,
becomes the underlying ungraded module to a simple
$A_\dbC$-module.
The even and odd components of $M_\dbC$ are permuted by
complex conjugation.

If $n\equiv 6(8)$, one can only find $M$ which, after
extension of scalars, becomes twice such a module.
Again, the even and odd part are permuted by complex
conjugation.
The commutant of $M$ (as ungraded module) is $\dbH$.
\endremark

If $A$ is central simple with a class $n$ in $\dbZ/8$
which is odd, $A^+$ is a matrix algebra if $n=1$ or
$7\mod 8$, and a matrix algebra over $\dbH$ otherwise.
In the first case, the unique simple $A_\dbC^+$-module
can be realized over $\dbR$.
In the second, it cannot.
The centralizer $Z$ of $A^+$ in $A$ is isomorphic to
$\dbR\times\dbR$ if $n\equiv 1$ or $5(8)$, to $\dbC$
otherwise.

\remark{\bf 1.20 Remark}
For $A$ a central simple algebra over $\dbR$, let $t$
be the trace for the underlying ungraded algebra,
$\sgn^+$ and $\sgn^-$ be the signatures of the
quadratic forms $t(xy)$ on $A^+$ and $A^-$,
respectively, and define
$$
s(A):= \sgn^+ + \sgn^- i\in\dbZ[i]\subset\dbC
$$
For a tensor product $A_1\otimes A_2$, $t$ vanishes on
the $A_1^{\pm}\otimes A_2^{\pm}$ other than
$A_1^+\otimes A_2^+$, where it is the tensor product of
$t$ for $A_1$ and $A_2$.
On $A_1^{\pm}\otimes A_2^{\pm}$, $t(xy)$ is related to
the tensor product of the bilinear forms $t_1(x_1y_1)$
and $t_2(x_2y_2)$ by
$$
\alignat3
&A_1^+ &&\otimes A_2^+:\,\,\,\, &&t(xy)=t_1(x_1y_1)\otimes
     t_2(x_2y_2)\\
&A_1^+ &&\otimes A_2^-: &&\text{same}\\
&A_1^- &&\otimes A_2^+: &&\text{same}\\
&A_1^- &&\otimes A_2^-: &&t(xy)=-t_1(x_1y_1)
     \otimes t_2(x_2y_2),
\endalignat
$$
the latter because $(x_1\otimes y_1)\cdot(x_2\otimes
y_2)=-(x_1x_2)\otimes(y_1y_2)$.
\endremark

The signature being multiplicative, and changing sign
when the bilinear form is replaced by its opposite,
we get
$$
s(A_1\otimes A_2)=s(A_1)s(A_2).
$$

For a matrix algebra $M_{r,s}$, off diagonals
contribute an isotropic form, and
$$
s(M_{r,s})=r+s
$$

For $\dbR[\eps]$ with $\eps$ odd and $\eps^2=1$, one
has $s=1+i$, of order $8$ in $\dbC^*/(\dbR^+)^*$.
This shows again that the group $\sBr(\dbR)$, being of
order $8$, is cyclic with generator the division
algebra $\dbR[\eps]$ with $\eps$ odd and $\eps^2=1$.

\proclaim{1.21 Construction}
The super Brauer group $\sBr(k)$ is an extension of
$H^0(\Gal(\kbar/k), \dbZ/2)=\dbZ/2$ by
$H^1(\Gal(\kbar/k),\dbZ/2)=k^*/k^{*2}$ by
$H^2(\Gal(\kbar/k)\kbar{^*})=\Br(k)$.
\endproclaim

In 1.21, $\kbar$ is a separable closure of $k$.

The map to $\dbZ/2$ is extension of scalars to $\kbar$,
see 1.17.
We now consider the kernel.

Let $A$ be a central simple algebra whose class in
$\sBr(k)$ is in the kernel of the map to $\dbZ/2$.
Extending scalars to $\kbar$, we obtain $\Abar$,
isomorphic to some $M_{r,s}$.
As such, it has two isomorphism classes of simple
modules, permuted by parity change.
Let $\scrS(\Abar)$ be the set of those isomorphisms
classes.
Having two elements, it can be viewed as a torsor over
$\dbZ/2$.
The Galois group $\Gal(\kbar/k)$ acts on $\scrS(\Abar)$
by transport of structures, defining an homomorphism
$$
h_A\colon\,\Gal(\kbar/k)\longrightarrow \dbZ/2,
$$
i.e. an element $h_A\in H^1(\Gal(\kbar/k),\dbZ/2)$:
the class of the equivariant $\dbZ/2$-torsor
$\scrS(\Abar)$.

For a tensor product $A_1\otimes A_2$, simple
$(A_1\otimes A_2\otimes \kbar)$-modules are the tensor
products over $\kbar$ of simple
$A_1\otimes\kbar$-modules and simple
$A_2\otimes\kbar$-modules, giving
$$
\scrS(A_1\otimes A_2^-)=\scrS(\Abar_1)+\scrS(\Abar_2)
$$
(addition of torsors) and $h_A=h_{A_1}+h_{A_2}$.

For $A$ a matrix algebra $M_{r,s}$, $\Gal(\kbar/k)$
acts trivially on $\scrS(\Abar)$.
it follows that $A\mapsto h_A$ factors through an
homomorphism
$$
h\colon\,\Ker(\sBr(k)\to\dbZ/2)\longrightarrow
H^1(\Gal(\kbar/k),\dbZ/2).
$$

Let us show that $h$ is onto.
By 1.17, if $A$ is not purely even, $\scrS(\Abar)$ is 
identified with the set of homomorphisms
$$
\Hom(Z,\kbar)
$$
for $Z$ the center of $A^+$.
We need to show that any quadratic extension of $k$ can
be obtained as a $Z$.
Indeed, for
$$
(k[\eps]\text{ with $\eps^2=a$})\otimes
(k[\eps]\text{ with $\eps^2=-1$}),
$$
$Z=A^+$ is $k\left(\sqrt{a}\,\right)$.

It remains to compute the kernel of $h$.
It corresponds to algebras $A$, with $\Abar\simeq
M_{r,s}$, for which an isomorphism class of simple
$\Abar$-module $\Sbar$  can be defined over $k$.
The automorphism group of $\Sbar$ is $\kbar{^*}$.
The obstruction to finding $S$ giving $\Sbar$ by
extension of scalars is hence in
$H^2(\Gal(\kbar/k)\kbar^*)$.

It is additive in $A$, and vanishes only if $A\simeq
M_{r,s}$.
This gives an injective morphism
$$
\Ker(h)\longrightarrow H^2(\Gal(\kbar/k),\kbar^*).
$$
It is onto, as shown by taking $A$ to be purely even
(usual Brauer group).

\newpage

\subhead
2. Variant
\endsubhead

The case of Clifford algebras suggests to consider a
variant of the super Brauer group.

\subhead
2.1
\endsubhead
One considers central simple super algebras $A$,
together with an even involution $\beta\colon\,A\to A$
which is an antiautomorphism of the underlying ungraded
algebra:
$$
\beta(xy)=\beta(y)\beta(x).\tag2.1.1
$$

The tensor product is defined as follows:
take the tensor product of super algebras, and define
$\beta$ as extending the involutions of the factors:
$$
\beta(a\otimes b)=(-1)^{p(a)p(b)}\beta(a)
\otimes\beta(b).
\tag2.1.2
$$
Tensor product is associative and commutative.

\example{2.2 Example}
Let $S$ be a super vector space, and
$A:=\underline{\End}(S)$.
Antiautomorphisms $\beta$ of the underlying ungraded
algebra correspond to non degenerate bilinear forms on
$S$ taken up to a factor by
$$
(ax,y)=(x,\beta(a)y).
$$
Involutive $\beta$ correspond to symmetric or
antisymmetric forms.
Even $\beta$ correspond to even or odd forms.
\endexample

 Suppose given two super vector spaces $S_i$ ($i=1,2$)
provided with even or odd, symmetric or antisymmetric
bilinear forms.
Let $(A,\beta)$ be the tensor product of the
corresponding $(A_i,\beta)$.
One has
$$
A=A_1\otimes A_2=\underline{\End}(S_1)\otimes
\underline{\End}(S_2)\simover{\longrightarrow}
\underline{\End}(S_1\otimes S_2).
$$
We leave it to the reader to check that the involution
$\beta$ of $A$ correspond to the following bilinear
form on $S_1\otimes S_2$:
$$
(s_1\otimes s_2,t_1\otimes t_2)+(-1)^{(p(s_1)+p(t_1))p(s_2)}
(s_1,s_2)(t_1,t_2).
\tag2.2.1
$$
Its parity is the sum of the parities of the forms
$(\quad,\quad)_i$ on the $S_i$.
Its sign ($+$ for symmetric, $-$ for antisymmetric) is
the product of the signs of the $(\quad,\quad)_i$, times
$(-1)$ at the power the product of the parities.

\subhead
2.3
\endsubhead
Let us call $(A,\beta)$ neutral if $A$ is of the form
$\End(S)$ and if $\beta$ is given by an even symmetric
form $(\quad,\quad)$ on $S$: the sum of symmetric forms
on $S^+$ and $S^-$.

Define $(A,\beta)$ and $(B,\beta)$ to be {\it similar}
if for suitable neutrals $(M,\beta)$ and $(N,\beta)$,
one has
$$
(A,\beta)\otimes (M,\beta)\simeq
(B,\beta)\otimes(N,\beta).
$$
Similarity is an equivalence relation, and is stable
by tensor product.
let $C(k)$ be the set of similarity classes.
Tensor product induces on $C(k)$ a composition law
associative, commutative and with unit.
The following proposition shows it is group.

\proclaim{Proposition 2.4}
Let $(A,\beta)$ be as in 2.1, and let $\Tr$ be the
trace of the underlying ungraded algebra.
Let $\alpha$ be the parity automorphism $a\mapsto
(-1)^{p(a)}a$.
Then, $(A,\beta)\otimes(A^0,\alpha\beta)$ is neutral.
More precisely,
$$
A\otimes A^0\simover{\longrightarrow}
\underline{\End}(A)
$$
and its involution corresponds to the even symmetric
form $\Tr(x\beta(y))$ on $A$.
\endproclaim

\demo{Proof}
We have
$$
\Tr((a\otimes b\cdot x)\beta(y))=(-1)^{p(b)p(x)}
\Tr(axb\beta(y));
$$
as
$$
\split
\Tr(axb&\beta(y))=\Tr(xb\beta(y)a)\quad\text{and that}\\
 &b\beta(y)a=\beta(\beta(a)y\beta(b))=
  (-1)^{p(b)p(y)}\beta(\beta(a)\otimes\beta(b).y),
\endsplit
$$
this equals
$$
=(-1)^{p(b)(p(x)+p(y))}\Tr(x.\beta(\beta(a)\otimes\beta
(b)y).
$$
Both sides are zero if $p(a)+p(b)+p(x)+p(y)$ is odd.
We hence may replace $p(b)(p(x)+p(y))$ by
$p(b)(p(x)+p(b))$, giving
$$
=\Tr(x.\beta((-1)^{p(a)p(b)}\beta(a)\otimes
\alpha\beta(b).y),
$$
$a\otimes b\mapsto
(-1)^{pa\,pb}\beta(a)\otimes\alpha\beta(b)$ is the
tensor product of $\beta$ and $\alpha\beta$.
\enddemo

\proclaim{Proposition 2.5}
The kernel of the ``forgetting $\beta$'' map from $C(k)$
to $\sBr(k)$ is cyclic of order $4$.
\endproclaim

\demo{Proof}
The $(A,\beta)$ whose class is in the kernel are of the
form $\End(S)$ of 2.2, with $\beta$ given by a non
degenerate bilinear form on $S$ which is even or odd,
as well as symmetric or antisymmetric.
We label those cases by $2.\dbZ/8$ as follows:
$$
\align
0 &\,\colon\,\,\text{ even symmetric}\\
2 &\,\colon\,\,\text{ odd symmetric}\\
4 &\,\colon\,\,\text{ even antisymmetric}\\
6 &\,\colon\,\,\text{ odd antisymmetric.}
\endalign
$$

By 2.2, the tensor product of algebras of type $i$ and
$j$ in $2.\dbZ/8$ is of type $i+j$.
It follows that each type is a similarity class, and
2.5 follows.
\enddemo

\proclaim{Proposition 2.6}
If $k$ is separably closed, $C(k)=\dbZ/8$, with
generator the class of $(k[\eps],\Id)$ with $\eps$ odd
and $\eps^2=1$.
\endproclaim

The proposed generator maps to the non trivial element
of $\sBr(k)$.
It remains to see that its square is a generator of the
kernel of $C(k)\to \sBr(k)$, computed in 2.5.
More precisely,

\proclaim{Lemma 2.7}
For $k$ separably closed,
$(k[\eps],\Id)\otimes(k[\eps],\Id)$ is of the type $2$ of
2.5.
\endproclaim

The algebra $k[\eps]$ is isomorphic to the opposite
algebra $k[\eta]$, with $\eta^2=-1$, and 2.6 follows
from the more general lemma, valid over any $k$

\proclaim{Lemma 2.8}
The tensor product of $k[\eps],\Id)$ and
$(k[\eta],\Id)$, with $\beta$ the identity, is of the
type $2$ of 2.5.
\endproclaim

We identify the tensor product with the $2\times 2$
matrix algebra, by taking $\eps=\left(\smallmatrix
0 &1\\ 1 &0\endsmallmatrix\right)$ and 
$\eta=\left(\smallmatrix 0 &-1\\ 1
&\,\,\,\,\,0\endsmallmatrix\right)$.
Those matrices are fixed by the transposition rel. the
quadratic form $xy$, which corresponds to an odd
symmetric form.

\remark{\bf 2.9 Remark}
Suppose $(A,\beta)$ over $k$ separably closed is non
trivial in $\sBr(k)$.
The simple $A$-modules are then obtained as follows:
start from a $A^+$-simple module $S^+$, and extend
scalars to $A$.
The involution $\beta$, restricted to $A$, corresponds
to a non degenerate bilinear form $(\quad,\quad)$ on
$S^+$, symmetric or antisymmetric, 
unique up to a scalar factor.
It is even or odd depending on the class of $A$ in
$\dbZ/8$.
The class of $(A,\beta)$ in $\dbZ/8$ detect if it is
even or odd, and if the restriction of $\beta$ to the
($2$-dimensional) centralizer $Z$ of $A^+$ is the
identity or not:
$$
\alignat4
1 &\,\colon\,\, &&\text{ symmetric} &\,\,,&  
     &\quad &\beta\vert Z=\Id\\
3 &\,\colon\,\, &&\text{ antisymmetric}  &\,\,,&  
     &\quad &\beta\vert Z\not=\Id\\
5 &\,\colon\,\, &&\text{ antisymmetric}  &\,\,, & 
     &\quad &\beta\vert Z=\Id\\
7 &\,\colon\,\, &&\text{ symmetric} &\,\,,& 
     &\quad &\beta\vert Z\not=\Id
\endalignat
$$
\endremark

\noindent
Determination of $\beta\vert Z$:\enspace
As $(Z,\beta\vert Z)$ is unchanged by tensoring with a
$(B,\beta)$ with $B$ a matrix algebra, if suffices to
consider the case of the tensor product of $n$
$k[\eps]$ with $\beta=\Id$.
In this case, $Z$ is generated by the product $\eta$ of
the $\eps$, and $\beta(\eta)=(-1)^{n(n-1)/2}\eta$.

\medskip\noindent
Parity of $(\quad,\quad)$:\enspace
Let $(B,\beta)$ be the tensor product of $(A,\beta)$
with $(k[\eps],\Id)$ (resp. $(k[\eps],\beta)$ with
$\beta(\eps)=-\eps$).
Its invariant is respectively $2$, $4$, $6$, $0$ (resp.
$0$, $2$, $6$, $6$).
If $S$ is a simple $B$-module, its restriction to $A$
continues to be simple.
If the restriction to $S^+$ of the form $(\quad,\quad)$
on $S$ is non degenerate, it is the form
$(\quad,\quad)$ we are looking for, and the table above
results from that of 2.5

\subhead
2.10
\endsubhead
For a general field $k$, extension of scalars to a
separable closure and 2.6 gives a commutative diagram
with exact rows
$$
\spreadmatrixlines{2\jot}
\matrix
0 &\longrightarrow &\dbZ/4 &\longrightarrow &C(k)
     &\longrightarrow &\sBr(k) & &\\
&&\Big\Vert && \doubleheaddownarrow{} &&\Big\downarrow & &\\
0 &\longrightarrow &\dbZ/4 &\longrightarrow  &\dbZ/8
  &\longrightarrow &\dbZ/2 &\longrightarrow &0
\endmatrix
\tag2.10.1
$$
For $k=\dbR$, (1.18) shows that $C(\dbR)$ maps onto
$\sBr(k)$ and (2.10.1) shows that the maps
$$
\align
C(\dbR) &\longrightarrow \dbZ/8=C(\dbC)\qquad\text{and}\\
C(\dbR) &\longrightarrow \dbZ/8=\sBr(\dbR)
\endalign
$$
induce an isomorphism from $C(\dbR)$ to the fiber
product of $\dbZ/8$ and $\dbZ/8$ over $\dbZ/2$:
$$
0\to C(\dbR)\to\dbZ/8\times\dbZ/8\to\dbZ/2\to 0.
$$

\newpage

\subhead
3. The Clifford algebra and spin-modules
\endsubhead

\subhead
3.1
\endsubhead
Let $k$ be a commutative ring, $V$ be a $k$-module
and $Q$ be a quadratic form on $V$.
The {\it Clifford algebra} $C(Q)$ is the $k$-algebra
generated by the $k$-module $V$ with the relations
$$
x^2=Q(x).1
\tag3.1.1
$$
for $x\in V$.
This is Bourbaki's definition (Alg. Ch. 9 \S9,
n$^o$ 1).
Some authors prefer to use as defining relations
$x^2=-Q(x).1$.
We will sometimes write $C(V)$, or $C(V,Q)$, instead
of $C(Q)$.

It results immediately from this definition that 

\smallskip\noindent
{\bf (A)}\enspace
$C(Q)$ is $\mod 2$-graded, the image of $V$ being
odd.
In other words: $C(Q)$ is a super algebra.
Indeed, the defining relations are in the even part
of the tensor algebra on $V$.

\smallskip\noindent
{\bf (B)}\enspace
The algebra $C(Q)$ admits a unique antiinvolution
$\beta$ which is the identity on the image of $V$.
Indeed, the opposite algebra $C(Q)^0$ is a solution
to the same universal problem as $C(Q)$ is.
Bourbaki's terminology: $\beta$ is the principal
antiautomorphism of $C(Q)$.
By definition, $\beta(xy)=\beta(y)\beta(x)$.
That there is no sign $(-1)^{p(x)p(y)}$ is not a
misprint.

If we apply (3.1.1) to $x+y$, $x$ and $y$ and take a
difference, we obtain the polarized form of (3.1.1):
$$
xy+yx=\phi(x,y).1
\tag3.1.2
$$
for $\phi$ the bilinear form $Q(x+y)-Q(x)-Q(y)$
associated to $Q$.

When $2$ is invertible in $k$, (3.1.1) is equivalent
to (3.1.2): take $x=y$ in (3.1.2).
In general, if $X\subset V$ generates $V$, (3.1.1)
is implied by (3.1.1) for $x\in X$, and (3.1.2) for
$x\not=y$ in $X$.
This makes it clear that

\smallskip\noindent
{\bf (C)}\enspace
The formation of $C(Q)$ is compatible with extension
of scalars.

\smallskip\noindent
{\bf (D)}\enspace
If $(V,Q)$ is the orthogonal direct sum of $(V',Q')$
and $(V'',Q'')$, the Clifford algebra $C(Q)$ is the
tensor product, in the sense of super algebras, of
$C(Q')$ and $C(Q'')$.
the principal antiautomorphism $\beta$ of $C(Q)$ is
the tensor product, in the sense of 2.1, of those of
$C(Q')$ and $C(Q'')$.

Indeed, $C(Q)$ is generated by the $k$-modules $V'$
and $V''$, with the relations (3.1.1) for $x\in V'$
or $x\in V''$ and the relations (3.1.2) for $x\in
V'$ and $y\in V''$.
For $x\in V'$ and $y\in V''$, (3.1.2) reduces to a
commutation relation (in the super sense)
$$
xy=(-1)^{\deg(x)\,\deg(y)}yx
$$
and (D) follows.

\smallskip\noindent
{\bf (E)}\enspace
The identity of $V$ extends to an isomorphism from
the opposite of the super algebra $C(Q)$ to the
super algebra $C(-Q)$.

\smallskip
Indeed, each defining relation $x.x=Q(x).1$ is
replaced by its opposite $x.x=-Q(x).1$.

\proclaim{Proposition 3.2}
Assume that $V$ is finite dimensional over a field
$k$ of characteristic $\not=2$.
If the form $Q$ is non degenerate, the super algebra
$C(Q)$ is central simple (in the super-sense).
\endproclaim

\demo{Proof}
$V$ is the orthogonal direct sum of $1$-dimensional
subspace.
By 3.1 (D) and 1.15, we are reduced to the case
where $\dim(V)=1$.
In that case, $C(Q)$ is a super division algebra
with center $k$.
\enddemo

\proclaim{Corollary 3.3}
The construction $(V,Q)\mapsto C(Q)$ induces a
morphism from the Witt group of $k$ to its super
Brauer group.
\endproclaim

\demo{Proof}
By 3.2 and 3.1 (D), it remains only to check that if
$V$ is hyperbolic, then $C(Q)$ is a matrix algebra.
Indeed, an hyperbolic quadratic space of
characteristic $\not=2$ can be presented as an
orthogonal direct sum $(W,Q)\oplus(W,-Q)$ and one
uses 3.1 (E) and 1.14.
Another proof will be given in 3.7.
\enddemo

\proclaim{Corollary 3.4}
The class of $(C(Q),\beta)$ in the variant $C(k)$
(2.3) of $\sBr(k)$ depends only on $\dim(V)\mod 8$
and the class of $(V,Q)$ in the Witt group $W(k)$.
\endproclaim

This results from 2.10.1, as $\dim(V)\mod 8$
determines the similarity class of $(C(Q),\beta)$ on
the separable closure.
The class in $W(k)$ determines the dimension of
$V\mod 2$, and 3.3, 3.4 give rise to a commutative
diagram with exact rows
$$
\minCDarrowwidth{20pt}
\CD
0 @>>> 2.\dbZ/8 @>>>\Ker(W(k)\times
     \dbZ/8 \to \dbZ/2) @>>> W(k) @>>> 0\\
@. @\vert @VVV  @VVV\\
0 @>>> 2.\dbZ/8 
@>>> C(k) @>>> \sBr(k) @.\\
@. @\vert @VVV @VVV\\
0 @>>> 2.\dbZ/8 @>>> C(\kbar)=\dbZ/8 @>>>
     \dbZ/2 @>>> 0\,\,.
\endCD
$$

\subhead
3.5
\endsubhead
Let $F$ be the filtration of $C(Q)$ defined by
$$
F_n=\text{image of }\bigoplus\limits_{0\Le i\Le n}
V^{\otimes i}.
$$
If $V$ is a free $k$-module, one has (Bourbaki Alg.
Ch. 9 \S9, n$^o$ 3, Th. 1)
$$
\wedgeop^{*} V\simover{\longrightarrow} \Gr^F C(Q).
\tag3.5.1
$$

We now give an ``explanation'' of (3.5.1), assuming
that $k$ is a field of characteristic $\not=2$.
As $2$ is invertible, $C(Q)$ is generated by $V$
with the relations (3.1.2) (instead of (3.1.1)).

If $\scrL$ is a super Lie algebra, the
Poincar\'e-Birkhoff-Witt theorem says that if $F$ is
the filtration of the universal enveloping algebra
$\scrU(\scrL)$ given by
$$
\gather
F_n=\text{image of }\bigoplus\limits_{0\Le i\Le n}
\scrL^{\otimes i},\\
\intertext{then}
\Sym^*(\scrL)\simover{\longrightarrow}
\Gr^F\scrU(\scrL)
\endgather
$$
where $\Sym^*$ is taken in its super sense: it is
$\Sym^*(\scrL^+)\otimes\wedgeop^{*}(\scrL^-)$.
If $i\colon\,k\hookrightarrow\scrL$ is central, we
have a variant of $\scrU(\scrL)$:
$$
\scrU_0(\scrL):=\scrU(\scrL)/\text{relation
$i(1)=1$}.
$$
A $\scrU_0(\scrL)$-module can be identified with a
super Lie algebra module on which $i(1)$ acts as the
identity.
A variant of Poincar\'e-Birkhoff-Witt says that for
the quotient filtration $F$ of $\scrU_0(\scrL)$,
$$
\Sym^*(\scrL/k)\simover{\longrightarrow}\Gr^F
\scrU_0(\scrL)\tag3.5.2
$$

The Theorem (3.5.1) is a particular case of (3.5.2):
take $\scrL:=k\oplus V$, with $V$ odd, $k$ even and
central, and the super commutator
$$
[x,y]=\phi(x,y).1
$$
for $x,y\in V$.
The definition of $C(Q)$ using (3.1.2) amounts to
$$
C(Q)=\scrU_0(\scrL),
$$
making (3.5.1) a particular case of (3.5.2).

\subhead
3.6
\endsubhead
Suppose $(V,Q)$ is isotropic: $V$is a direct sum
$L^*\oplus L$ with
$$
Q(\omega+\ell)=\left<\omega,\ell\right>.
$$
The exterior algebra $\wedgeop^{*}L$ is then turned
into a $C(Q)$-module by requiring $\ell\in L$ (resp.
$\omega\in L^*$) to act by exterior multiplication 
$\ell\w$ (resp. contraction by $\omega$).
It is a super module, if we take the parity of
$\wedgeop^{i}L$ to be that of $i$.

\proclaim{Proposition 3.7}
With the notations of 3.4, if $L$ is a free module
of finite rank, one has
$$
C(Q)\simover{\longrightarrow}\underline{\End}
\left(\wedgeop^{*}L\right).\tag3.7.1
$$
\endproclaim

\demo{Proof}
Let $L=\oplus L_i$ be a decomposition of $L$ as a
direct sum of free modules of rank one.
It defines a corresponding decomposition of $V$ in
the orthogonal direct sum of the $(L_i^*\oplus L_i)$.
Both sides of (3.7.1) are the tensor product of the
corresponding $C(L_i^*\oplus L_i)$ and
$\underline{\End}\,(\wedgeop^{*}L_i)$.
This reduces (3.7.1) to the case $L=k$, which is
left to the reader.
\enddemo

\subhead
3.8
\endsubhead
One can view 3.7 as an odd analogue of the
Schr\"odinger representation of Heisenberg groups.
Indeed, let $(V,\psi)$ be a symplectic, rather than
orthogonal, finite dimensional vector space over
$k$, assumed of characteristic zero.
Provide $k\oplus V$ with the Lie algebra structure
for which $k$ is central and
$[v_1,v_2]=\psi(v_1,v_2)$ for $v_1$, $v_2$ in $V$:
the Heisenberg Lie algebra attached to $V$.
If $V$ is a direct sum $L^*\oplus L$, with
$\psi((\omega',\ell'),(\omega'',\ell''))=\omega'(\ell'')
-\omega''(\ell')$, the symmetric algebra $\Sym^*(L)$
is turned into a $\scrU_0(k\oplus V)$-module by
requiring $\ell$ in $L$ to act by multiplication, and
$\omega\in L^*$ to act by contraction.
If $L$ is the dual of $E$, $\Sym^*(L)$ is the ring
of polynomial functions on $E$, the action of
$\ell\in L$ is multiplication by a linear function,
and that of $\omega\in L^*$ --- identified with a
constant vector field on $E$ --- is $\partial_\omega$.
This representation identifies $\scrU_0(k\oplus V)$
with the ring $\scrD_E$ of all linear differential
operators on $E$ with polynomial coefficients.

In 3.6, $\wedgeop^{*}L$ is similarly the ring of
function on the odd super affine space defined by
$E$, with $\ell\w$ and $\omega{\sssize\bot}$ multiplication
and derivation operators, respectively.

\subhead
3.9
\endsubhead
Under the assumptions of 3.7, with $k$ a field, and
$V\not=0$, the algebra $C(Q)$ is a matrix algebra by
3.7, hence has up to isomorphism a unique simple
module $S$.
The automorphism group of $S$ is $k^*$.
This module is even a super module over the super
algebra $C(Q)$.
Its grading is, however, unique only up to parity
change.

It follows that any automorphism $g$ of $C(Q)$ can
be extended to an automorphism of ($C(Q)$, module
$S$).
the extension is unique up to an automorphism
$\lambda\in k^*$ of $S$.
If $g$ respects the grading of $C(Q)$, the extension
respects or permutes the homogeneous components
$S^{\pm}$ of $S$, depending on the action of $g$ on
the center of $C^+(Q)$.

In particular, $O(Q)$ acts projectively on $S$.
The parity respecting subgroup is $\SO(Q)$.

Similar arguments can be made for $k$ any local
ring.
Taking the dual numbers $k[x]/(x^2)$, and interpreting
the Lie algebra $\grso(Q)$ as $\ker(O(Q)(k[x]/(x^2))
\to O(Q)(k))$, one gets a
projective action of the Lie algebra $\grso(Q)$ on $S$.
As $C(Q)\simover{\longrightarrow}\End_k(S)$, this
action is given by a Lie algebra morphism
$$
\fbar\colon\,\grso(Q)\longrightarrow
C(Q)/k
$$
with $[\fbar(x),v]$ being the Lie algebra action of
$\grso(Q)$ on $V$.
As $C(Q)$ is the Lie algebra product of $k$ and
$[C(Q),C(Q)]$ (usual brackets), $\fbar$ lifts 
uniquely to a morphism $f$ from $\grso(Q)$ to
$[C(Q),C(Q)]\subset C(Q)$.
Here is a direct construction of this lifting.
Identify $\grso(Q)$ to $\wedgeop^{2}V$ by
$$
x\w y\longmapsto\text{ endomorphism }
\phi(y,v)x-\phi(x,v)y.
$$
Map $x\w y$ to $\tfrac12\,(xy-yx)$.

\noindent
Check:\enspace
as $xy+yx$ is scalar,
$$
\left[\tfrac12\,(xy-yx),v\right]=
[xy,v]=x\{y,v\}-\{x,v\}y
$$
for $\{\qquad\}$ the super bracket
$$
=\phi(y,v)x-\phi(x,v)y.
$$

The argument we used to see the existence and
unicity of $f$ used that $V$ is split and even
dimensional.
The actual construction used neither, and holds for
any non degenerate quadratic form.
It turns any $C(Q)$-module into a $\grso(Q)$-module.

\subhead
3.10
\endsubhead
The argument used in 3.9 is analogue to the one by
which, using the unicity of an irreducible
representation of the Heisenberg commutation
relations, one gets on the representation space a
projective action of a real symplectic group --- or
an actual representation of its double covering, the
metaplectic group.
This is, however, an analytic story, involving
infinite dimensional Hilbert space, as evidenced by
the fact that the metaplectic double covering of the
real symplectic group is not algebraic.
The Lie algebra story, however, has a purely
algebraic analogue.
With the notations of 3.8, the symmetrized product
gives a vector space isomorphism
$$
\Sym^*(V)\simover{\longrightarrow}\scrU_0(k\oplus V)
\tag3.10.1
$$
The image of $\Sym^2(V)$ by this map is a Lie
algebra normalizing $V$, identified by its action on
$V$ with the Lie algebra of the symplectic group.

Assume $V$ is the dual of $E$, so that $\Sym^*(V)$
is the polynomial functions on $E$.
If we replace $\psi$ by $t\psi$, and transport the
product on the corresponding $\scrU_0(k\oplus V)$ to
$\Sym^*(V)$ by (3.10.1), we obtain on $\Sym^*(V)$ a
product $*_t$ depending on $t$.
It comes from a $k[t]$-algebra structure on
$\Sym^*(V)\otimes k[t]$.

One has
$$
f*_t g=fg+\tfrac12\,\{f,g\}t+O(t^2),
$$
for $\{\qquad\}$ the Poisson bracket on the
symplectic manifold $E$.
A weakened version of the fact that
$\Sym^*(V)\subset\scrU_0(k\oplus V)$ acts on $V$ as
Lie $\Sp(V)$ is the fact that the hamiltonian vector
fields on $E$ given by quadratic functions are the
infinitesimal symplectic transformations.

{\it We now assume that $k$ is a field of
characteristic $\not=2$, that $V$ is a finite
dimensional vector space, and $Q$ a non degenerate
quadratic form on $V$}.

\proclaim{Proposition 3.11}
The sub Lie algebra of $C^+(Q)$ normalizing $V\subset
C(Q)$ is $k\oplus f(\grso(Q))$, and $f(\grso(Q))$
the subspace on which the principal antiautomorphism
of $C(Q)$ acts by $-1$.
\endproclaim

\demo{Proof}
If $x$ in $C^+(Q)$ normalizes $V$, $adx\vert V$ is
an infinitesimal orthogonal transformation:
$$
\phi([xv],v)=[xv]v+v[xv]=[x,v^2]=
[x,Q(v)]=0.
$$
Let us write $x=y+f(z)$ with $z\in \grso(V)$.
By construction, $y$ centralizes $V$, hence is in the
center of $C(V)$.
Being even, it is in the center of the super algebra
$C(V)$, hence a scalar by 2.2.
This proves the first assertion.

The principal antiautomorphism $\beta$ transform
$\tfrac12\,(vw-wv)$ ($v,w\in V$) into $\tfrac12\,(wv-vw)$.
This checks the second assertion.
\enddemo

\proclaim{Corollary 3.12}
Let $S$ be a $C^+(Q)$-module and $(\quad,\quad)$ a
bilinear form on $S$ for which
$$
(xs,t)=(s,\beta(x)t)\tag3.12.1
$$
for $x$ in $C^+(Q)$ and $s$, $t$ in $S$.
For the induced structure of $\grso(Q)$-module, the
bilinear form $(\quad,\quad)$ is invariant.
\endproclaim

\proclaim{Proposition 3.13}
Let $S$ be an absolutely simple super
$C(Q)$-modules.

\smallskip\noindent
{\rm (i)}\enspace
If $\dim(V)$ is even, the representations $S^+$ and
$S^-$ of $\grso(Q)\subset C^+(V)$ are non isomorphic
irreducible representations.

\smallskip\noindent
{\rm (ii)}\enspace
If $\dim(V)$ is odd, the representations $S^+$ and
$S^-$ are irreducible and isomorphic.
\endproclaim

Isomorphy in (ii) result from the decomposition
$$
C(Q)=C^+(Q)\otimes(\text{centralizer $Z$ of
$C^+(Q)$ in $C(Q)$}):
$$
an odd generator of $Z$ provides an isomorphism.

As $\grso(Q)=\wedgeop^{2}V\subset C^+(Q)$ generates
the algebra $C^+(Q)$, (i) and (ii) results from the
structure of $C^+(Q)$ when $k$ is separably closed:
a matrix algebra over $k\times k$ in case (i), a
matrix algebra over $k$ in case (ii).

\remark{\bf 3.14}
Whether there exist an absolutely simple super
$C(Q)$-module $S$ (i.e. $S$ remains simple after
extension of scalar to $\kbar$) depends on the class
of $C(Q)$ in $\sBr(k)$, itself depending only on the
class of $Q$ in the Witt group.
The kind of bilinear forms on $S$ for which $\beta$
is the transposition depends on the class of
$(C(Q),\beta)$ in the variant $C(k)$ (2.3) of
$\sBr(k)$.
\endremark

\medskip\noindent
{\it Until the end of this {\rm \S}, we assume $k$
separably closed}.

\medskip
For $\dim(V)$ even (resp. odd), $C(Q)$ (resp. $C^+(Q)$)
is a matrix algebra, with involution $\beta$.
If $S$ is a simple $C(Q)$- (resp. $C^+(Q)$-) module,
there is up to a scalar factor a unique bilinear form
$(\quad,\quad)$ on $S$ such that
$$
(as,t)=(s,\beta(a)t).
\tag3.14.1
$$
For $\dim\,V$ even, there is on $S$ a $\dbZ/2$-grading,
unique up to parity change, turning $S$ into a super
module.

A space of {\it spinors} is a super $C(Q)$-module
(resp. a $C^+(Q)$-module) $S$ provided with
$(\quad,\quad)$ obeying (3.14.1).
it is unique up to isomorphism, with $\pm 1$ as the
only automorphisms.
The $\Lie$ algebra of infinitesimal automorphisms of
$(V,S)$ provided  with $Q$, the module structure of $S$
and $(\quad,\quad)$ is $\grso(Q)$.

For $\dim(V)$ even, $S^+$ and $S^-$ are the spaces of
{\it semi-spinors}.
By 3.7, if $L$ is a maximal isotropic subspace of $V$,
there is up to a scalar factor a unique semi-spinor
$s(L)$ such that $L.s(L)=0$ for the $C(Q)$-module
structure of $S$.
The two types of maximal isotropic subspaces are
distinguished by whether $s(L)$ is in $S^+$ or in $S^-$.
For the semi-spinors $s(L)$, one can check that the
vanishing of $(\quad,\quad)$ has a simple interpretation
$$
(s(L'),s(L''))=0 \Leftrightarrow L'\cap L''\not=0
\tag3.14.2
$$
(C. Chevalley, the algebraic theory of spinors,
Columbia Univ. Press 1954, III 2.4, p. 79).

The class of $(C(Q),\beta)$ in the group $C(k)=\dbZ/8$
of 2.6 is the dimension of $V$ modulo $8$: write $V$ as
an orthogonal direct sum of lines, and apply 3.1 (D).
By the tables of 2.5 and 2.7, the form $(\quad,\quad)$
is hence as follows, according to the vaue of
$\dim(V)\mod 8$:
$$
\alignat3
&0 &\quad &\colon\quad \text{ even} &&\text{ symmetric} \\
&1 &&\colon\quad && \text{ symmetric}\\
&2 &&\colon\quad \text{ odd}  && \text{ symmetric}\\
&3 &&\colon\quad &&\text{ antisymmetric}\\
&4 &&\colon\quad \text{ even} &&\text{ antisymmetric}\\
&5 &&\colon\quad  &&\text{ antisymmetric}\\
&6 &&\colon\quad \text{ odd} &&\text{ antisymmetric}\\
&7 &&\colon\quad  &&\text{ symmetric}\,.
\endalignat
$$

If $\dim(V)$ is even, the even or odd character of
$(\quad,\quad)$ can be read from (3.14.2), using that
$L
$ and $L''$ are of the same type if and only if $L'\cap
L''$ is of even codimension in $L'$ (and $L''$): if
$\dim(V)/2$ is even (resp. odd), and $L'$, $L''$ of
different (resp. same) type, $L'\cap L''\not=0$ and
$(s(L'),s(L''))=0$.

If $e_1,\dotsc,e_n$ is an orthogonal basis of $V$, the
product $z=e_1\ldots e_n$ generates the commutant $Z$
of $C^+(Q)$.
One has
$$
\beta(z)=(-1)^{n(n-1)/2}z\tag3.14.3
$$
If $\dim(V)$ is even, $\beta\vert Z$ is trivial if and
only if $(\quad,\quad)$ is even.
By (3.14.3),
this shows again that the parity of $(\quad,\quad)$ is
that of $\dim(V)/2$.

\subhead
3.15
\endsubhead
Consider $(k[\eps],\beta)$ with $\eps$ odd, $\eps^2=1$ and
$\beta(\eps)=\pm\eps:\,\,\sign\,\pm$.
The class of $(A,\eta):=(C(V),\beta)\otimes(k[\eps],\beta)$
is $\dim\,V\pm 1\mod 8$.
If $V$ is odd dimensional, and $S$ a simple super
module over $A$, $S$ remains a simple super module over
$C(V)$, and $S^+$ is a simple $C^+(V)$-module.
If the form $(\quad,\quad)$ on the $A$-module $S$
obeying (3.14.1) is even (resp. odd), a form
$(\quad,\quad)$ on the $C^+(V)$-module $S^+$ obeying
(3.14.1) is
$$
\alignat3
&(s,t) &\,\,&=&\,\,&(s,t), \text{ respectively}\\
& &&=&&(s,\eps t)
\endalignat
$$
with sign that of $(\quad,\quad)$ (resp. $\pm$ that of
$(\quad,\quad)$).
This shows again that if $2n\in\dbZ/8$ corresponds to
even forms, then the forms for $2n$ and $2n\pm1$ have
the same sign.
If $2n$ corresponds to odd forms, the sign is the same
for $2n-1$, opposite for $2n+1$.

\subhead
3.16
\endsubhead
Let $S$ be a space of spinors.
If $\dim\,V$ is even, the morphism of
$\grso(Q)$-modules
$$
V\otimes S\longrightarrow S
$$
given by the $C(V)$-module structures transposes to a
morphism
$$
\{\quad\}:\,\,S\otimes S\longrightarrow V
$$
characterized by the equality of inner products
$$
(w,\{s,t\})=(ws,t)\tag3.16.1
$$
in $V$, resp. $S$.
The parity of $\{\quad\}$ is opposite to that of
$(\quad,\quad)$ on $S$.
The sign is the same.
Depending on $\dim\,V\mod 8$, $\{\quad\}$ is
$$
\alignat2
&0 &\qquad &\text{odd symmetric}\\
&2 &&\text{even symmetric}\\
&4 &&\text{odd antisymmetric}\\
&6 &&\text{even antisymmetric}
\endalignat
$$

If $\dim(V)$ is odd, choosing an odd generator $z$ of
the centralizer $Z$ of $C^+(Q)$, one similarly defines
$\{\quad\}:\,\,S\otimes S\to V$ by
$$
(w,\{s,t\})=(zws,t).\tag3.16.2
$$
The sign of $\{\quad\}$ is that of $(\quad,\quad)$ on
$S$, times that of $\beta$ acting on $z$: by 2.9, it is
depending on $\dim(V)\mod 8$:
$$
\alignat2
&1 &\qquad &\text{symmetric}\\
&3 &&\text{symmetric}\\
&5 &&\text{antisymmetric}\\
&7 &&\text{antisymmetric}.
\endalignat
$$

\subhead
3.17
\endsubhead
If $V_1$ and $V_2$ are even dimensional, and if $S_1$
and $S_2$ are spaces of spinors for $V_1$ and $V_2$,
then $S_1\otimes S_2$ is a space of spinors for the
orthogonal direct sum of $V_1$ and $V_2$.
its module structure is given by the tensor product, in
the super sense, of those of $S_1$ and $S_2$, and the
form $(\quad,\quad)$ is given by (2.2.1).

Suppose now that $v\not=0$ is an isotropic vector in
$V$, supposed to be even dimensional.
Define
$$
V_1:= v^\bot/kv,
$$
of dimensional two less.
If $S$ is a space of spinors for $V$, $v$ acting on $S$
supercommutes with $v^\bot$.
The kernel $\Ker(v)$ is hence stable by $v^\bot$.
The action of $v^\bot$ factors through $V_1$ and turns
$\Ker(v)$ into a $C(V_1)$-module.

\proclaim{Proposition 3.18}
With the notations of 3.17

\smallskip
\roster
\item"{\rm (i)}"
$\Ker(v)=\im(v)$ is an irreducible $C(V_1)$-module.

\item"{\rm (ii)}"
The form $(\quad,\quad)$ vanishes on $\Ker(v)$.

\item"{\rm (iii)}"
The form $(\quad,\quad)_1$ on $\Ker(v)$ defined by
$(s,vt)_1=(-1)^{p(s)}(s,t)$ turns $\Ker(v)$ into a
space of spinors for $V_1$.
\endroster
\endproclaim

\demo{Proof}
(i)\enspace
Lift $V_1$ in $v^\bot$ and let $H$ be the orthogonal
complement of the lifting: $V$ becomes the orthogonal
direct sum of $V_1$ and $H$, with $\dim(H)=2$ and $v\in H$.
The space of spinors for $V_1$ becomes the tensor
product of one for $V_1$ and one for  $H$, with the
action of $v$ coming from its action on spinors for
$H$.
This reduces us to the two dimensional case, where $v$
acts as $\left(\smallmatrix 0 &1\\ 0
&0\endsmallmatrix\right)$ in a suitable basis.
The claim follows.

\noindent
(ii)\enspace
results from $\Ker(v)=\im(v)$:
$$
(vs,vt)=(s,vvt)=0
$$
as $v^2=Q(v)=0$.
In (iii), $t\mod \Ker(v)$ is determined by $vt$, so
that, by (i) and (ii), $(\quad,\quad)$ is well defined.

\noindent
(iii)\enspace
It remains to check that $(\quad,\quad)_1$ obeys
(3.14.1).
Indeed, if $t_1=vt$ and $w\in v^\bot$, one has
$wt_1=-vwt$ and
$$
(ws,t_1)_1=(-1)^{p(ws)}(ws,t)=(-1)^{p(ws)}(s,wt)
=(s,wt_1)_1
$$
\enddemo

\proclaim{Proposition 3.19}
Under the same assumptions, if $s,t\in\Ker(v)$, one has
in $V$
$$
\{s,t\}=(-1)^{p(s)}(s,t)_1v.\tag3.19.1
$$
\endproclaim

\demo{Proof}
We check that both sides have the same inner product
with any $w\in V$.
If $t=vx$, the left side indeed gives
$$
\align
(w,\{s,x\}) &=(ws,vx)=(vws,x)=((vw+wv)s,x)\\
&=(v,w)(s,x)=(-1)^{p(s)}(v,w)(s,t)_1
\endalign
$$
\enddemo

\newpage

\subhead
4. The group $\Spin$
\endsubhead

\subhead
4.1
\endsubhead
Let $V,Q)$ be a finite dimensional quadratic vector
space over $k$ of characteristic $\not=2$.
Let $G$ be the group of invertible elements in $C^+(Q)$
or $C^-(Q)$ which normalize $V$.
We let it act on $V$ by
$$
\rho(g):\,\,v\longmapsto (-1)^{p(g)}gvg^{-1}.
$$
This action respects the quadratic form $Q(v)=v^2$ on
$V$.
If $g$ fixes $V$, $g$ is in the center (in the super
sense) of $C(Q)$, hence is a scalar: we have an exact
sequence
$$
1\longrightarrow k^*\longrightarrow G\longrightarrow 
O(V).\tag4.1.1
$$

Elements $w\in V$ with $Q(w)\not=0$ are in $G^-:= G\cap
C^-(Q)$, and $\rho(w)$ is the reflection relative to
the hyperplane orthogonal to $w$.
As each element of $O(V)$ is a product of reflections,
the sequence (4.1.1) is exact ot the right.
As a product of  $k$ reflections has determinant
$(-1)^k$, it follows from (4.1.1) that $G^+:=G\cap
C^+(Q)$ maps onto $\SO(V)$, while $G^-$ maps onto
$O(V)-\SO(V)$.
We have
$$
1\longrightarrow k^*\longrightarrow G^+\longrightarrow
\SO(V)\longrightarrow 1.\tag4.1.2
$$

\subhead
4.2
\endsubhead
If $g\in G$, applying $\beta$ to the defining relation
$$
gv=(-1)^{p(g)}vg,
$$
we see that $\beta(g)\in G$ with
$\rho(\beta(g))=\rho(g)^{-1}$:
$g\beta(g)$ is in $k^*$.
The {\it Spin group} is the kernel of
$$
g\beta(g):\,\, G^+\longrightarrow k^*.\tag4.2.1
$$
On $k^*\subset G^+$, $g\beta(g)$ is squaring.
The sequence (4.1.2) is an exact sequence of algebraic
groups, and it follows that it induces a commutative
diagram
$$
\spreadmatrixlines{1\jot}
\matrix
1 &\to &\mu_2 &\to &\Spin &\to &\SO &\to &1\\
&&\Big\downarrow &&\Big\downarrow &&\Big\Vert &&\\
1 &\to &\dbG_m &\to &G^+ &\to &\SO &\to &1\\
\\
&&&\vbox{\beginpicture
\setcoordinatesystem units < .50cm, .50cm>
\linethickness=1pt
\setlinear
%
% Fig POLYLINE object
%
\linethickness= 0.500pt
\plot  7.620 21.590  9.335 20.574 /
%
% arrow head
%
\plot  9.084 20.649  9.335 20.574  9.148 20.758 /
%
%
% Fig TEXT object
%
\put{\lower2pt\hbox{$\ssize \lambda^2$}} [lB] at  7.683 20.733
\linethickness=0pt
\putrectangle corners at  7.620 21.590 and  9.335 20.574
\endpicture}
&\null\quad\rmapdown{g\beta(g)} &&&&\\
&&&&\dbG_m &&&&
\endmatrix
\tag4.2.2
$$
with $\dbG_m=$ multiplicative group, $\mu_2=\{\pm1\}$.
In (4.2.2), the first two rows are exact sequences of
algebraic groups.
On the level of points, the first row may not be exact:
it gives rise to a coboundary with values in
$H^1(\Gal(\kbar/k),\mu_2)$.

\subhead
4.3
\endsubhead
If $S$ is a $C^+(Q)$-module, and $(\quad,\quad)$ a
bilinear form on $S$ obeying (3.14.1), the action of
$\Spin\subset C^+(Q)^*$ on $S$ respects the form
$(\quad,\quad)$:
$$
(gs,gt)=(s,\beta(g)gt)=(s,t).
$$
This is the counterpart of 3.12 for groups.

For $k$ separably closed, and $S$ a space of spinors,
the group $\Spin$ is the group of automorphisms of
$(V,S)$ provided with the following structure:

\smallskip\noindent
For $\dim(V)$ even: \enspace
the map $V\otimes S\to S$, the $\mod 2$ grading of $S$
and $(\quad,\quad)$ on $S$.

Indeed, from $V\otimes S\to S$ one recovers $Q(v)=v^2$
and the $C(V)$-module structure, the $\mod 2$ grading
of $S$ cuts $O(V)$ down to $\SO$, and the group of
automorphisms, contained in $C(Q)^{+*}$, is a double
covering of $\SO(Q)$.

\smallskip\noindent
For $\dim(V)$ odd:\enspace
$Q$, a volume element, the $C^+(Q)$-module structure on
$S$ and $(\quad,\quad)$.
The proof is similar.

This is a counterpart of 3.14 for groups.

\newpage

\subhead
5. The Minkowski case
\endsubhead

\proclaim{Theorem}
Let $(V,Q)$ over $\dbR$ be of signature
$(+,-,\dotsc,-)$.
Let $S_\dbR$ be a real representation of $\Spin(V,Q)$,
irreducible as a real representation.
Assume that after extension of scalars to $\dbC$, it
becomes a sum of spinorial or semi-spinorial
representations.
The commutant $Z$ of $S_\dbR$ is $\dbR$, $\dbC$ or
$\dbH$.
Let $\overline{\phantom{x}}$ be the standard
anti-involution of $Z$.

\medskip\noindent
{\rm (i)}\enspace
Up to a real factor, there exists a unique symmetric
morphism $\{\quad,\quad\}
\colon\,S_\dbR\otimes S_\dbR\to V$ such that
$\{zs,t\}=\{x,\zbar t\}$, i.e. which is
invariant by the group of elements of norm $1$ of $Z$.

\smallskip\noindent
{\rm (ii)}\enspace
For $v\in V$, if $Q(v)>0$, the form
$\left<v,\{s,t\}\right>$ on $S_\dbR$ is positive or
negative defined.
\endproclaim

The set of $v$ for which $Q(v)>0$ has two connected
components.
It follows from (ii) that for $v$ in one of them, call
it $C$, the form $<v,\{s,t\}>$ is positive defined and
that for $v$ in the other, i.e. for $-v$ in $C$, it is
negative defined.

The class of $C(V)$ in $\sBr(\dbR)=\dbZ/8$ (1.18) is
the signature modulo $8$ (3.3), i.e. is $2-d$ if $V$ is
of dimension $d$.
The representation $S_\dbR$ is a simple $C^+(V)$-module
and, by the table 1.18, depending on $d$ modulo $8$, 
$Z$ and the
complexification of $S_\dbR$ are given as follows.
For $d$ even, $S^+$ and $S^-$ denote the semi-spinorial
representations.
For $d$ odd, $S$ denotes the spin representation.
$$
\alignat4
&0 &\,\,&\colon &\quad &\dbC\,\,, &\quad &S^++S^-\\
&1 &&\colon &&\dbR\,\,, &&S\\
&2 &&\colon &&\dbR\,\,, &&S^+\text{ or }S^-\\
&3 &&\colon &&\dbR\,\,, &&S\\
&4 &&\colon &&\dbC\,\,, &&S^++S^-\\
&5 &&\colon &&\dbH\,\,, &&2S\\
&6 &&\colon &&\dbH\,\,, &&2S^+\text{ or }2S^-\\
&7 &&\colon &&\dbH\,\,, &&2S
\endalignat
$$

\demo{Proof}
We first show that if $d$ is congruent
to $2$ modulo $8$, a morphism $\{\quad\}$ with the
listed properties exists.
In this case, the signature is $0$ modulo $8$, so that
$C(V)$ is a matrix algebra.
Let $S=S^+\oplus S^-$ be a simple $C(V)$-module.
The possible $S_\dbR$ are $S^+$ and $S^-$.
The pairing $(\quad,\quad)$ on $S$ (relative to which
the principal antiautomorphism $\beta$ is transposition)
is odd (3.14).
The corresponding $\{\quad\}\colon\,S\otimes S\to V$ is
even and symmetric (3.16).
\enddemo

Fix $v$ with $Q(v)>0$ and its orthogonal $V_1$.
The graded module $S$ remains simple as a graded
$C(V_1)$-module, and $S^+$ remains absolutely
irreducible as a representation of the group
$\Spin(V_1)$.
The bilinear form $\left<v,\{s,t\}\right>$ on $S^+$  is not
identically zero: if it were zero for one $v$, it would be
zero for all $v$
 by $\Spin(V)$-invariance, and $\{\quad,\quad\}$
would be zero.
This form is invariant by the compact group
$\Spin(V_1)$.
As $S^+$ is an irreducible representation of
$\Spin(V_1)$, it must be definite.
The same applies to $S^-$.

We now show, in all dimensions, the existence of a
morphism $S_\dbR\otimes S_\dbR\to V$ with the listed
properties.
Embed $V$ in a Minkowski space $V_0$ of dimension 
congruent to $2$ modulo $8$:
$V$ is the orthogonal direct sum of $V$ and of a
negative defined quadratic space of a suitable
dimension.
Let $\pr$ be the orthogonal projection of $V_0$ onto
$V$.

Let $S_0=S_0^+\oplus S_0^-$ be a simple
$C(V_0)$-module.
We dispose of $\{\quad\}_0\colon\,S_0^\pm\otimes
S_0^\pm\to V_0$, with the required properties for
$V_0$.
As $\Spin(V)\subset\Spin(V_0)$, we can also by
restriction consider $S_0$ as a
$\Spin(V)$-representation.
The action of $\Spin(V)$ comes from the structure of
$C(V)$-module of $S_0$, deduced from its structure of 
$C(V)$-module by $C(V_0)\hookrightarrow C(V)$.
It follows that the irreducible constituents of $S_0$,
as a $\Spin(V)$-representation, are of the type
considered in the theorem.
Further, $S$ being a faithful $C(V_0)$-module, and
hence a faithful $C(V_0)$-module, all representations of
the type considered occur as a direct factor.
There exists a morphism of $\Spin(V_0)$-representations
$S_\dbR\hookrightarrow S_0^+$ or $S_\dbR\hookrightarrow
S_0^-$.
On $S_\dbR$, we now define $\{\quad\}_1$ as the
orthogonal projection
$$
\{s,t\}_1=\pr\{s,t\}_0
$$
of $\{\quad\}_0$ on $V$.
{}From the same properties of $\{\quad\}_0$, the form
$\{\quad\}_1$ is symmetric and if $Q(v)>0$ with $v$ in
$V$, the form $\left<v,\{s,t\}_1\right>=
\left<v,\{s,t\}_0\right>$ on $S_\dbR$
is positive or negative defined.
It remains to average $\{\quad\}_1$ over the compact
group $K$ of elements of norm $1$ in $Z$ to obtain the
required
$$
\{s,t\}=\int_{K}\{ks,kt\}_1 dk
$$

We now prove the unicity, up to a real factor, of
$\{\quad\}$.

\subhead
{\rm Case 1}
\endsubhead
$Z=\dbR$,i.e. the complexification $S_\dbC:=
S_\dbR\otimes\dbC$ of $S_\dbR$ is an irreducible
representation.
We note it $S_0$.
It is spinorial for $d$ odd, semi-spinorial for $d$
even.
If we exclude the case $d=2$, $V$ is an absolutely irreducible
representation, and it suffices to see that, over
$\dbC$, $V_\dbC$ occurs at most once in $S_0\otimes S_0$.

The case $d=2$ is easy to treat directly:
$V$ decomposes as $D_1\oplus D_2$, for $D_1$ and $D_2$
the two isotropic lines, and the $S^\pm$ are of
dimension one with tensor square isomorphic,
respectively, to $D_1$ and $D_2$.

\subhead
{\rm Case 2}
\endsubhead
$Z=\dbH$, i.e. $S_\dbC$ is twice an irreducible
representation $S_0$: $S_\dbC=S_0\otimes_\dbC W$ with
$\dim(W)=2$.
The invariance condition on $\{\quad\}$ amounts, after
complexification, to $\SL(2,W)$-invariance:
$\{\quad\}$ is to be the tensor product of
$\{\quad\}_0\colon\,S_0\otimes S_0\to V_\dbC$ and of the
unique (up to a factor) antisymmetric $\psi\colon\,
W\otimes W\to\dbC$.
Symmetry of $\{\quad\}$ amounts to antisymmetry of
$\{\quad\}_0$.
Again, it suffices to see that $V_\dbC$ occurs at most once
in $S_0\otimes S_0$.

\subhead
{\rm Case 3}
\endsubhead
$Z=\dbC$, i.e. $S_\dbC$ is the sum of two inequivalent
(complex conjugate) representations.
This can happen only for $d$ even, with
$S_\dbC=S^+\oplus S^-$ the sum of the two
semi-spinorial representations.
The invariance condition on $\{\quad\}$ amounts to
say that, after complexification, $\{\quad\}$ is the
sum of $S^+\otimes S^-\to V$ and of its symmetric.
It suffices to see that $V$ occurs at most twice
in$(S^+\oplus S^-)\otimes(S^+\oplus S^-)$.

Let us check those multiplicity statements, over
$\dbC$.
If $d$ is odd, and for $S$ the spinorial
representation, $S\otimes S\simeq
\underline{\End}(S)\simeq C^+(V)\simeq \wedgeop^{2*} V$.
As $\wedgeop^{i} V\simeq \wedgeop^{d-i} V$ as a
representation of the special orthogonal group, this can be
rewritten
$$
S\otimes S\simeq \bigoplus \wedgeop^{i} V
\qquad\text{(sum for $0\Le i\Le (d-1)/2$) }.
$$
The $\wedgeop^{i} V$ for $i\le (d-1)/2$ are
irreducible, non isomorphic (fundamental)
representations, and $V$ occurs only once.

If $d$ is even, let $S=S^+\oplus S^-$ be the sum of the
semi-spinorial representations.
We have now
$$
S\otimes S\simeq \bigoplus \wedgeop^{i} V\qquad
(0\Le i\Le d)
$$
and $V$ occurs exactly twice.
The unicity claim follows.


\enddocument



