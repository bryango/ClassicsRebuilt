\documentstyle{amsppt}
\magnification=1200
\input epsf.tex
\loadeufm

\font\boldtitlefont=cmb10 scaled\magstep2

\NoRunningHeads
\pagewidth{6.5 true in}
\pageheight{8.9 true in}
\loadeusm

\catcode`\@=11
\def\logo@{}
\catcode`\@=13

\def\eps{{\varepsilon}}

\def\nspace{\lineskip=1pt\baselineskip=12pt%
     \lineskiplimit=0pt}
\def\dspace{\lineskip=2pt\baselineskip=18pt%
     \lineskiplimit=0pt}

\def\plus{{\sssize +}}
\def\wedgeop{\operatornamewithlimits{\wedge}\limits}
\def\w{{\mathchoice{\,{\scriptstyle\wedge}\,}
  {{\scriptstyle\wedge}}
  {{\scriptscriptstyle\wedge}}{{\scriptscriptstyle\wedge}}}}
\def\bulletsatop{{\overset{\sssize\bullet}\to{\sssize\bullet}}}

\def\Spin{\text{\rm Spin}}
\def\Sp{\text{\rm Sp}}
\def\SO{\text{\rm SO}}
\def\Sym{\text{\rm Sym}} 
\def\SL{\text{\rm SL}}
\def\End{\text{\rm End}}
\def\Tr{\text{\rm Tr}}


\def\dbC{{\Bbb C}}
\def\dbG{{\Bbb G}}
\def\dbH{{\Bbb H}}
\def\dbR{{\Bbb R}}
\def\dbZ{{\Bbb Z}}


\topmatter
\title\nofrills
{\boldtitlefont Spinors: tables}
\endtitle
\author
P. Deligne
\endauthor
\endtopmatter


\NoBlackBoxes
\parindent=20pt
\frenchspacing
\document
\dspace

\bigskip
\subhead
1. Complex story
\endsubhead

The Dynkin diagram of the group $\Spin(n)$ is:
$$
\vbox{\epsfxsize=3.0in\epsfbox{fig1.eps}}
$$
where $r$ is the rank.
The spin or semi-spin representations are the fundamental
representations corresponding to the vertex, or vertices,
at the right of the diagram.
If one restricts the spin representation of $\Spin(2r+1)$
to $\Spin(2r)$, one obtains the sum of the two
semi-spin representations of $\Spin(2r)$.
Each of the two semi-spin representations of $\Spin(2r)$
restricts as the spin representation of $\Spin(2r-1)$.

The Dynkin diagram makes clear the low rank exceptional
isomorphisms.
We list in the next table $n=2,3,4,5$ or $6$, the Dynkin
diagram, a group $G$ to which $\Spin(n)$ is isomorphic,
the description of spin or semi-spin representations as
representations of $G$, and the description of the
defining representation of $\SO(n)=\Spin(n)/(\dbZ/(2))$
as representation of $G$.
\def\bulletsatop{{\overset{\sssize\bullet}\to{\sssize\bullet}}}
$$
\vbox{
\tabskip=14pt
\halign to 6.5 true in{#\hfill &\hfill #\hfill  &\hfill #\hfill  
     &#\hfill &\hfill #\hfill \cr
$n$ &diagram &$G$  &(semi)spin &defining orthogonal\cr
\noalign{\bigskip}
$2$ &none &$\dbG_m$ &characters $z$, $z^{-1}$ 
  &$z^2\oplus z^{-2}$\cr
\noalign{\medskip}
$3$ &${\sssize\bullet}$ &$\SL(2)$ &defining $V$ &adjoint
$=\Sym^2(V)$\cr
\noalign{\medskip}
$4$ &$\bulletsatop$ &$\SL(2)\times \SL(2)$
     &defining $V_1$, $V_2$ &$V_1\otimes V_2$\cr
\noalign{\medskip}
$5$ &\vbox{\epsfxsize=.75in\epsfbox{table-figa.eps}}
  &$\Sp(4)$ &defining $V$ &$\wedgeop^{2}\,V$\cr
\noalign{\medskip}
6  &\vbox{\epsfxsize=.75in\epsfbox{table-figb.eps}}
  &$\SL(4)$ &defining $V$, and $V^*$
  &$\wedgeop^{2}\,V$\cr}}
$$
The symmetric bilinear form on the defining orthogonal
representation is given by

\medskip\noindent
$n=3$: \ Killing form $\Tr(adx\,ady)$, or

$\left<v^2,w^2\right>=\psi(v,w)^2$, for $\psi$ symplectic
 ($=$ volume) form of $V$.

\medskip\noindent
$n=4$: \ $\left<v_1\otimes v_2,\,w_1\otimes
w_2\right>=\psi(v_1,w_1)\psi_2(v_2,w_2)$.

\medskip\noindent
$n=5$: \ $\left<v_1\w v_2,\,w_1\w w_2\right>=
 \psi(v_1,w_1)\psi(v_2,w_2)-\psi(v_1,w_2)\psi(v_2,w_1)$.

\medskip\noindent
$n=6$: \ $\left<v_1\w v_2,\,w_1\w w_2\right>=
  v_1\w v_2\w w_1\w w_2\in\wedgeop^{4}\,V=\dbC$.

\medskip
For $n$ odd, the spin representation $S$ of $\Spin(n)$ is
orhtogonal or symplectic.
There is a morphism of representations from $S\otimes S$
to the defining representation $V$; it is symmetric or
antisymmetric.

For $n$ even, the semi-spin representations $S^{\plus}$
and $S^-$ are orthogonal, symplectic, or dual of each
other.
There is either a morphism of representations
$S^{\plus}\otimes S^-\to V$, or morphisms
$S^{\plus}\otimes S^{\plus}\to V$ and $S^-\otimes S^-\to
V$, symmetric or antisymmetric.
Which is the case depends only on $n\bmod{8}$.
The next table gives $n\bmod{8}$, the kind of bilinear
form on spinors, and the kind of bilinear forms (spinors)
$\otimes$ (spinors) $\to V$.
$$
\vbox{
\tabskip=20pt
\halign{\hfill #\hfill  &#\hfill &#\hfill\cr
$n\bmod{8}$ &forms on spinors &symmetry of spinors,
  spinors $\to V$\cr
\noalign{\bigskip}
1 &orthogonal &symmetric\cr
2 &$S^{\plus}$ dual to $S^-$ &symmetric (on $S^{\plus}$, 
  and on $S^-$)\cr
3 &symplectic &symmetric\cr
4 &$S^{\plus}$ and $S^-$ sympletic 
  &$S^{\plus}\otimes S^-\to V$\cr
5 &symplectic &antisymmetric\cr
6 &$S^{\plus}$ dual to $S^-$ &antisymmetric (on
  $S^{\plus}$, and on $S^-$)\cr
7 &orthogonal &antisymmetric\cr
8 &$S^{\plus}$ and $S^-$ orthogonal 
     &$S^{\plus}\otimes S^-\to V$\cr}}
$$

\bigskip
\subhead
2. Real story
\endsubhead

An irreducible representation (over $\dbR)$ of
$\Spin(p,q)$ is said to be spinorial if, after extensioin
of scalars to $\dbC$, it becomes a sum of spinorial or
semi-spinorial representations.
If $p-q$ is odd (equivalent to $p+q$ odd), there is a
unique spinorial irreducible representation.
It is either real or quaternionic.
if $p+q$ is even, either there is only one such
representation, in which case it is complex, giving over
$\dbC$ the sum of the two semi-spinorial representations,
or there are two of them, both real or both quaternionic.
If real, they remain irreducible over $\dbC$: \
$S^{\plus}$ and $S^-$ have a real form.
If they are quaternionic, after extension of scalars to
$\dbC$, one becomes $2S^{\plus}$, the other becomes
$2S^-$.
In which case we are depends only on the signature
$p-q\bmod{8}$.
Table:
$$
\vbox{\tabskip=50pt
\halign{\hfill #\hfill &\hfill #\hfill\cr
$p-q$ &real, complex or quaternionic\cr
\noalign{\bigskip}
$1$ or $7$ &$\dbR$\cr
$2$ or $6$ &$\dbC$\cr
$3$ or $5$ &$\dbH$\cr
$4$ &$\dbH$, $\dbH$\cr
$8$ &$\dbR$, $\dbR$\cr}}
$$

\bigskip
\subhead
3. Minkowski signature
\endsubhead

In Minkowski signature, $(+, -, -, \ldots)$, and with a
positive light cone chosen on $V$, if $S$ is an
irreducible spinorial representation, there is up to a
real factor a unique symmetric bilinear maps $B\colon\,
S\otimes S\to V$.
It can be normalized so that $Q(s):= \frac12\,B(s,s)$ is
with values in the closed positive cone.
Such a $Q$  is now unique up to a positive real factor.

If $S$ is complex, $B$ is the real part of an hermitian
bilinear form with values in $V_{\dbC}$: \ if $J$ is the
complex structure, $Q$ is invariant by the $J$, as well
as by the circle group of $\exp(\theta J)$.
After extension of scalars to $\dbC$, $S$ becomes the sum
of the two semi-spinorial representations, and $B$
corresponds to a pairing $S^{\plus}\otimes S^-\to V$.

If $S$ is quaternionic, the quaternions with absolute
value one preserve $Q$.
After extension of scalors to $\dbC$, $S$ becomes the sum
$S_0\otimes T$ of two copies of a spinorial or
semi-spinorial representation $S_0$, and $B$ the tensor
product of an antisymmetric pairing $S_0\otimes S_0\to
V_{\dbC}$ with a scalar alternating form $T\otimes
T\to\dbC$.
The quaternions become $\End(T)$.

The next table gives the nature of irreducible spinorial
representations, as a function of $n\bmod{8}$, for
$\Spin(1,n-1)$.
If $\Spin(1,n-1)$ has two irreducible spinorial
representations, they have isomorphic restrictions to
$\Spin(1,n-2)$.
The last column of the table describes the restriction of
an irreducible representation of $\Spin(1,n-1)$ to
$\Spin(1,n-2)$.
It is either an irreducible spinorial representation, or
twice it, or the sum of two distinct spinorial
representations.
Noted: \ $S$, $2S$, $S^{\plus}+S^-$.
$$
\vbox{\offinterlineskip\tabskip=15pt
\halign{\hfill #\hfill &\hfill #\hfill
  &\hfill #\hfill\cr
$n\bmod{8}$ &nature &restriction to $\SO(1,n-2)$\cr
\noalign{\bigskip}
$1$ &$\dbR$ &$S$\cr
\noalign{\medskip}
$2$ &$\dbR$, $\dbR$ &$S$\cr
    &($S^{\plus}$ and $S^-$, in duality) &\cr
\noalign{\medskip}
$3$ &$\dbR$ &$S^{\plus}+S^-$\cr
\noalign{\medskip}
$4$ &$\dbC$ &$2S$\cr
\noalign{\medskip}
$5$ &$\dbH$ &$2S$\cr
\noalign{\medskip}
$6$ &$\dbH$, $\dbH$ &$S$\cr
    &($S^{\plus}$ and $S^-$, in duality) &\cr
\noalign{\medskip}
$7$ &$\dbH$ &$S^{\plus}+S^-$\cr
\noalign{\medskip}
$8$ &$\dbC$ &$S$\cr}}
$$

\enddocument

