%From: John W Morgan <jmorgan@math.ias.edu>
%Date: Mon, 19 May 1997 16:51:58 -0400
%Subject: Lecture I

\documentstyle[11pt]{article}


%These are the macros which are in common with all of the
% sections in the paper mmr
% Each section, for now, should begin with \documentstyle[11pt,cd]{article}
% and then have \input{mmrmacros} followed by \begin{document}
% The only exception is that the \Label macro is slightly different
% in each file and should be put in separately.
%New CD macros
\newcommand{\cdrl}{\cd\rightleftarrows}
\newcommand{\cdlr}{\cd\leftrightarrows}
\newcommand{\cdr}{\cd\rightarrow}
\newcommand{\cdl}{\cd\leftarrow}
\newcommand{\cdu}{\cd\uparrow}
\newcommand{\cdd}{\cd\downarrow}
\newcommand{\cdud}{\cd\updownarrows}
\newcommand{\cddu}{\cd\downuparrows}
% (S) Proofs.
% (S-1) Head is automatically supplied by \proof.

\def\proof{\vspace{2ex}\noindent{\bf Proof.} }
\def\tproof#1{\vspace{2ex}\noindent{\bf Proof of Theorem #1.} }
\def\pproof#1{\vspace{2ex}\noindent{\bf Proof of Proposition #1.} }
\def\lproof#1{\vspace{2ex}\noindent{\bf Proof of Lemma #1.} }
\def\cproof#1{\vspace{2ex}\noindent{\bf Proof of Corollary #1.} }
\def\clproof#1{\vspace{2ex}\noindent{\bf Proof of Claim #1.} }
% End of Proof Symbol at the end of an equation must precede $$.

\def\endproof{\relax\ifmmode\expandafter\endproofmath\else
  \unskip\nobreak\hfil\penalty50\hskip.75em\hbox{}\nobreak\hfil\bull
  {\parfillskip=0pt \finalhyphendemerits=0 \bigbreak}\fi}
\def\endproofmath$${\eqno\bull$$\bigbreak}
\def\bull{\vbox{\hrule\hbox{\vrule\kern3pt\vbox{\kern6pt}\kern3pt\vrule}\hrule}}
\addtolength{\textwidth}{1in}                  % Margin-setting commands
\addtolength{\oddsidemargin}{-.5in}
\addtolength{\evensidemargin}{.5in}
\addtolength{\textheight}{.5in}
\addtolength{\topmargin}{-.3in}
\addtolength{\marginparwidth}{-.32in}
\renewcommand{\baselinestretch}{1.6}
\def\hu#1#2#3{\hbox{$H^{#1}(#2;{\bf #3})$}}          % #1-Cohomology of #2
\def\hl#1#2#3{\hbox{$H_{#1}(#2;{\bf #3})$}}          % #1-Homology of #2
\def\md#1{\ifmmode{\cal M}_\delta(#1)\else  % moduli space, delta decay of #1
{${\cal M}_\delta(#1)$}\fi}
\def\mb#1{\ifmmode{\cal M}_\delta^0(#1)\else  %moduli space, based, delta
					      %decay of #1
{${\cal M}_\delta^0(#1)$}\fi}
\def\mdc#1#2{\ifmmode{\cal M}_{\delta,#1}(#2)\else    %moduli space, delta
						      %decay, chern class #1
						      %of #2
{${\cal M}_{\delta,#1}(#2)$}\fi}
\def\mbc#1#2{\ifmmode{\cal M}_{\delta,#1}^0(#2)\else   %as before, based
{${\cal M}_{\delta,#1}^0(#2)$}\fi}
\def\mm{\ifmmode{\cal M}\else {${\cal M}$}\fi}
\def\ad{{\rm ad}}
\def\msigma{\ifmmode{\cal M}^\sigma\else {${\cal M}^\sigma$}\fi}
\def\cancel#1#2{\ooalign{$\hfil#1\mkern1mu/\hfil$\crcr$#1#2$}}
\def\dirac{D\hskip-.67em\slash}
\newtheorem{thm}{Theorem}
\newtheorem{theorem}{Theorem}[subsection]
\newtheorem{proposition}[theorem]{Proposition}
\newtheorem{lemma}[theorem]{Lemma}
\newtheorem{claim}[theorem]{Claim}
\newtheorem{example}[theorem]{Example}
\newtheorem{corollary}[theorem]{Corollary}
\newtheorem{D}[theorem]{Definition}
\newenvironment{defn}{\begin{D} \rm }{\end{D}}
\newtheorem{addendum}[theorem]{Addendum}
\newtheorem{R}[theorem]{Remark}
\newenvironment{remark}{\begin{R}\rm }{\end{R}}
\newcommand{\note}[1]{\marginpar{\scriptsize #1 }} 
\newenvironment{comments}{\smallskip\noindent{\bf Comments:}\begin{enumerate}}{\end{enumerate}\smallskip}

\renewcommand{\thesection}{\Roman{section}}
\def\eqlabel#1{\addtocounter{theorem}{1}
\write1{\string\newlabel{#1}{{\thetheorem}{\thepage}}}
\leqno(\rm\thetheorem)}
\def\cS{{\cal S}}
\def\ov{\overline}












\title{Kaluza-Klein compactifications, supersymmetry, and Calabi-Yau
spaces: I}  
\author{Andrew Strominger\thanks{Notes by John Morgan}}
\date{}
\begin{document}
\maketitle

\section{Introduction}

As we have seen in d'Hoker's lectures,
study of the ten-dimensional 
heterotic string dynamics leads us to a low energy effective action
of the form:
$$S_{\rm eff}[g,A,\phi,\ldots]
=\frac{1}{2\kappa^2}\int_{X^{10}}d^{10}x\sqrt{-{\rm 
det}(g)}e^{-2\phi}\left(R(g)+\frac{\alpha'}{30}{\rm Tr}F_A^2+\cdots\right)$$ 
for a metric $g$ of signature $(-,+,+,\cdots,+)$ on a ten-dimensional
manifold $X^{10}$,  a connection
$A$ on a principal $G$-bundle over $X^{10}$,  a  scalar field
$\phi$, called the {\sl dilaton field}, and other fields including the
supersymmetric partners of these that we have not written here.  Here
$R(g)$ is the  
scalar curvature of $g$, $G$ is compact, semisimple, $F_A$ is the
curvature of $A$ and $Tr F_A^2$ represents a $G$-invariant metric on the
Lie algebra of $G$.  The study of the heterotic string shows
that it must be the case that the Cartan matrix
of the Dynkin diagram for $G$ is an even unimodular lattice in
dimension $16$.  The only two possibilities for $G$ are $E_8\times E_8$
or $Spin(32)/({\bf Z}/2{\bf Z})$ for a central ${\bf Z}/2{\bf Z}$ which
is not in the kernel of the projection to $SO(32)$.
At the level we are working in these lectures, the delicate
distinction between various groups with Lie algebra $SO(32)$ cannot be
seen. 
For this reason, and also to simplify the notation,
we shall consider $SO(32)$ instead of the more exotic form. 
Lastly, the expression $d^{10}x\sqrt{-{\rm det}(g)}$ is the volume
element associated to the metric. The minus sign arises from the fact
that the metric is of signature $(1,9)$ so that the sign is necessary
to have a positive quantity under the square root.

Our goal in these lectures is to study and analyse this Lagranian in
its own right, not considering the small coreections
arising from string theory.
It turns out that, because of the topological nature of the analysis,
our conclusions remain largely valid in the exact string theory. 
In fact, this Lagrangian was written down as a Lagrangian for
supergravity before string theory was  invented, and so is of interest
in its own right.
There are  two basic problems that any such analysis should
resolve if this theory is to be a reasonable model of nature.

\smallskip
\noindent{\bf Problem 1.} This action  exists for metrics,
connections, etc. on a ten-dimensional manifold $X^{10}$, yet 
it seems fairly apparent that any theory of nature must have  to do
with four-dimensional space-time of our universe.  There must be an
explanation of the `collapsing' of ten space-time dimensions to four. 

\smallskip
\noindent{\bf Problem 2.} The two gauge groups that occur for this
action, $E_8\times E_8$ and $SO(32)$, are not
appropriate ones for the 
observed matter content of the universe. Somehow  by a
mechanism of spontaneous symmetry-breaking, we must get down to a
smaller group which is suitable for describing observed matter. 


Our approach will be to first look for classical ground states
(solutions) minimizing the action and
study their features. 
At a later stage one should consider quantum corrections.

Motivated by the two problems we listed above, here are the properties
that we want our ground states (or classical solutions) to have:

\smallskip
\noindent
1.  The effective theory looks four-dimensional instead of
ten-dimensional. 

\smallskip
\noindent
2. The gauge group $G$ should be broken down to a `good' gauge group,
one that fits with the GUT (grand unification theory) scheme, so that
it produces matter content observed in nature.  The gauge group of the
standard model is $SU(3)\times SU(2)\times U(1)$ and accounts for all
the matter that we know of, in the sense that all the `elementary'
particles are grouped together into a representation of this reductive
group. 
The GUT schemes are to
embed this group in a larger  group, $SU(3)\times SU(2)\times
U(1)\subset G$, so that the  
representations that occur in the standard model occur among
those arising in the restriction of an irreducible representation of
$G$ to the subgroup $SU(3)\times SU(2)\times U(1)$. This scheme
(among others) 
allows the following set of groups for $G$:
$$SU(3)\times SU(2)\times U(1)\subset SU(5)\subset SO(10)\subset
E_6.$$
Of course, on the level of groups on can continue the series with
$$E_6\subset E_7\subset E_8,$$
but the irreducible representations that we must use have to be chiral
and hence 
not isomorphic to their duals.  This rules out $E_7$ and $E_8$ where
the relevant representations are real.
Somehow we must break either $E_8\times E_8$ or $SO(32)$ down to one
of these groups, most likely $E_6$. 


\smallskip
\noindent
3. Our solution should have $N=1$  supersymmetry in dimension $4$ (the
$N=1$  super Poincar\'e group).
We do not observe in supersymmetry in nature, so whatever supersymmetry
exists in our effective four-dimensional gravity theory must be capable
of being entirely broken to $N=0$, hopefully by small quantum
corrections, which we shall not discuss here.  There are no
mechanisms for breaking $N\ge 2$  
supersymmetry down to a chiral theory. This implies that we can have
at most $N=1$ 
supersymmetry in our model.  On the other hand, we desire nontrivial
supersymmetry in our model because it is the best  mechanism we
know which is capable of accounting for the very small ratios of
masses of the observed low energy particles, i.e., for solving the
`hierarchy problem' as it is known to physicists.

\smallskip
\noindent
4. Our solution should yield a realistic matter content.

\bigskip
The story that we shall describe was worked out in the mid 1980's (see
[Candelas, Horowitz, Strominger, and Witten, Nucl. Phys. B
{\bf 258} (1985)] and [Green, Schwarz, Witten: Chapters 12 --
16]). More recently, with the  
advent of dualities, there are new wrinkles, but the basic story as it
was understood ten years ago remains relevant.


\section{Kaluza-Klein Model}

Let us begin with a much simpler and much older example, which goes
back to the work of Kaluza-Klein in 1926. We consider the 
five-dimensional Einstein-Hilbert action
$$S[\widehat{g}]=\frac{1}{2\pi\kappa^2}\int_Yd^5z\sqrt{-{\rm
det}(\widehat{g})}R(\widehat{g}).$$
(The variable $z$ is real.)
The action is for  a metric  $\widehat{g}$ on a
five-dimensional space, and $R(\widehat{g})$ is the scalar curvature
of $g$. 
Lastly, $\kappa$ is Newton's constant.  (The hats are simply to
distinguish this from other metrics which will arise later.)
The classical equations of motion for this action are Einstein's
equations 
$$R_{MN}(\widehat{g})=0,$$
the vanishing of the Ricci curvature tensor of $\widehat{g}$.

One solution $\widehat{\overline{g}}$ is on the manifold $Y={\bf R}^4\times S^1$ where the
metric is
$$\widehat{\overline{g}}_{MN}(z)dz^M dz^N=\eta_{\mu\nu}dx^\mu
dx^\nu+r^2d\theta^2;\ \ \  
\ {\rm where}\ \ z=(x,\theta).$$
Here we have introduced several conventions which we shall use
consistently. The indices
$\mu,\nu$ range from $0$ to $3$ and are indices for coordinates
$x^\mu$ in Minkowski four-dimensional space time.  The metric tensor
$\eta_{\mu\nu}$ is the standard diagonal form $(-,+,+,+)$. 
The indices $M,N$ range from $0$ to $4$ and index the  coordinates
$z^M$ on $Y$. Here, $r$ is a constant which determines the radius of
the internal circle.

We begin with this basic solution and study the situation near this
solution. 
It is convenient to write metrics (or symmetric two-tensors) in a
slightly different basis: 
$$\widehat{g}_{MN}(z)dz^M
dz^N=e^{-\rho/3}\left(e^\rho\left(d\theta+\kappa A_\mu 
dx^\mu\right)^2+g_{\mu\nu} dx^\mu dx^\nu\right).$$
In this description the metric on the five-dimensional space has
become three fields: a four-dimensional metric tensor $g_{\mu\nu}$, a
one form on four-space $A_\mu dx^\mu$, and a scalar field $\rho$.  Of 
course, at this point all these fields depend not only on the point in
the four-space but also on the auxiliary (or {\sl internal}) variable
$\theta$. 


Now let us consider the expansion in Fourier modes with respect to the
variable $\theta$ on the circle:
$$g_{\mu\nu}(z)=\sum_{n=-\infty}^\infty g_{\mu\nu;n}(x)e^{in\theta},$$ 
$$A_\mu(z)=\sum_{n=-\infty}^\infty A_{\mu;n}(x)e^{in\theta},$$
$$\rho(z)=\sum_{n=-\infty}^\infty\rho_n(x)e^{in\theta}.$$
At this point what we have done is to rewrite the five-dimensional
metric as an infinite collection of purely four-dimensional fields
(metrics, one-forms, scalars). As we shall see shortly, when taking
the low energy effective limit we can ignore all but finitely many of
these fields. (We keep only the zero modes in the above  Fourier
expansions.) 

Now we are ready to consider excitations (or linearized
fluctuations) 
around our solution $\widehat{\overline{g}}$.  Consider a general metric
$\widehat{g}_{MN}(z)dz^Mdz^N$  near $\widehat{\overline{g}}$. The
linearized fluctuation is 
$$\tilde h_{MN}(z)=\widehat{g}_{MN}(z)-\widehat{\overline{g}}_{MN}.$$ 
It is convenient to define
$$h_{MN}=\tilde h_{MN}-\frac{1}{2} \widehat{\overline{g}}^{MN}\tilde
h_{MN} \widehat{\overline{g}}_{MN}.$$

We view the symmetric two-tensor $h=h_{MN}(z)dz^M dz^N$ as
parametrizing the fluctuations of the metric. 
We must take some care here for there are gauge symmetries,
the group of diffeomorphisms of $Y$. Metrics lying on the same orbit
of this group action must
be considered as equivalent since they differ simply
reparametrizing the space.
The standard way around this problem is to pick a gauge (a slice
perpendicular to the orbit of the gauge symmetries). In the current
context the simplest gauge to pick is called the {\sl transverse
gauge}.  It consists of considering metrics 
satisfying
$$
\nabla^Mh_{MN}   =  0.
$$
Now, as we observed above, the equations of motion are the
Einstein equations
$$R_{MN}(\widehat{g})=0.$$
Expanding $R_{MN}(\widehat{\overline{g}}+h)=0$ to first order in $h$ yields
$$\Box^5_{\widehat{\overline{g}}}h_{MN}(z)=
\left(\eta^{\mu\nu}\partial_\mu\partial_\nu
+\frac{1}{r^2}\partial_\theta^2\right)h_{MN}(x,\theta)=0.$$  

Combining this equation with the Fourier mode expansion we see that
$$\Box^4_{\widehat{\overline{g}}}h_{MN;n}(x)-\frac{n^2}{r^2}h_{MN;n}(x)=0.$$
That is to say $h_{MN;n}(x)$ satisfies the wave equation
in four-space for a field of
mass $|n|/r$.  Invoking a basic result from quantum mechanics we
know that the lowest energy excitation of a field of mass $|n|/r$ has
energy at least $|n|/r$.
Now it is time to say more about the size of the circle factor in $Y$.
So far it has been a free parameter in the theory. We shall take $r$
to be our roughly the Planck length, which is $10^{-33}$ cm. At this
length scale, all the excitations of $h_{MN;n}(x)$ have huge energy
provided that $n\not= 0$. Since we are interested in the low energy
effective theory, we can safely ignore all the modes $h_{MN;n}(x)$
except the zero modes.
Thus, at  energy scales below $1/r$, our effective theory is purely
four-dimensional.  It has three fields on four-space which are now   
independent of the internal variable $\theta$. The fields are: a
metric $g_0=g_{\mu\nu;0}(x)dx^\mu dx^\nu$ on ${\bf R}^4$, a one-form
$A_{0}=A_{\mu;0}(x) dx^\mu$ on four-space and a scalar field $\rho_0(x)$ on
four-space.    

Next, let us plug these fields into the Kaluza-Klein action  and write
the resulting  effective four-dimensional action. 
 It is actually a somewhat complicated, if routine, computation to
carry out the substitution; what comes out is:
\begin{eqnarray*}
\lefteqn{S_{\rm eff}[g_{0},A_{0},\rho_0] = } &  & \\
\quad & & =r\int_{{\bf
R}^4}d^4x\sqrt{-g_{0}}\left(
\frac{1}{\kappa^2}R(g_{0})-\frac{1}{4}e^{\rho_0}
F_{\mu\nu}^0F^{\mu\nu}_{0}
-\frac{1}{6\kappa^2}(\nabla\rho_0)^2
+\left((\nabla\rho_1)^2+ \frac{1}{r^2}\rho_1^2+\cdots
\right)\right), 
\end{eqnarray*}
where $F^0_{\mu\nu}dx^\mu dx^\nu$ is the curvature of the connection
$A_{0}$.  
All the terms enclosed in the last pair of parentheses represent massive
particles and can be ignored in our low energy effective action.
Thus, we are left with
$$S_{\rm eff}[g_{0},A_{0},\rho_0]=r\int_{{\bf
R}^4}d^4x\sqrt{-g_{0}}\left(
\frac{1}{\kappa^2}R(g_{0})-\frac{1}{4}e^{\rho_0}
F_{\mu\nu}^0F^{\mu\nu}_{0}
-\frac{1}{6\kappa^2}(\nabla\rho_0)^2\right).$$
 


We began with a five-dimensional gravity theory and for sufficiently
small radius $r$ we have found the low energy effective four-dimensional
action and written this action in terms of three four-dimensional
fields: a metric, a connection which is related to the Killing vector 
field on the five-manifold tangent to the circle directions, and a
scalar field $\rho$ which is related 
to the free parameter $r$ in the five-dimensional theory.




Let us now consider the symmetries of this effective four-dimensional
action. 
We claim that these are simply the symmetries of the original
five-manifold $Y={\bf R}^4\times S^1$ which 
commute with the 
natural $S^1$-action coming from this product structure of $Y$.
This group is generated by two types of diffeomorphisms. The first are
simply diffeomorphisms of ${\bf R}^4$. 
These are 
generated by vector fields $\zeta$ on ${\bf R}^4$, independent of
$\theta$.  It is easy to see 
that the action of these on our fields  are given by:
$$\delta_\zeta g_{\mu\nu}={\cal L}_\zeta(g_{\mu\nu}),$$
$$\delta_\zeta A_\mu={\cal L}_\zeta(A_\mu),$$
$$\delta_\zeta\rho={\cal L}_\zeta(\rho).$$
Here, in all cases ${\cal L}_\zeta$ denotes the Lie derivative in the
$\zeta$-direction. 
The second type of diffeomorphisms are those generated by vector
fields of the form
$$x^\mu\mapsto x^\mu;\ \ \ \theta\mapsto \theta+\zeta^5(x).$$
Under these vector fields we have
$$\delta_\zeta(A_\mu)=-\frac{1}{\kappa}\partial_\mu\zeta^5,$$ 
$$\delta_\zeta g=\delta_\zeta\rho=0.$$
Notice, in particular, that $A=A_\mu dx^\mu$ transforms like an
abelian (i.e., $U(1)$) gauge field.


Of course, there are other diffeomorphisms of the  five-manifold $Y$,
which do not preserve the $S^1$-action. But these diffeomorphism mix
the massive and low energy modes.  Since we have thrown away the
massive modes in forming our effective four-dimensional Lagrangian,
these symmetries are broken in the limiting process, leaving the
unbroken group for the four-dimensional effective action exactly the
one described in the previous paragraph. 

\section{Compactifying Einstein's Equation from ten dimensions to four
dimensions} 

Now let us consider  ten-dimensional pure gravity.
The action is
$$S[\widehat{g}]=\int_{X^{10}}d^{10}x\sqrt{-{\rm
det}(\widehat{g})}R(\widehat{g}),$$ 
where $\widehat{g}$ is a metric on $X^{10}$ with signature
$(-,+,\cdots,+)$. 
The classical solutions extremizing this Einstein-Hilbert action are
Ricci-flat metrics 
$$R_{MN}(\widehat{g})=0.$$
In analogy with the Kaluza-Klein model, let us study ground states
(classical solutions) of the following form:
The manifold $X^{10}$ is a product ${\bf R}^4\times K^6$ where $K^6$ is
compact, and the metric is of the form
$\widehat{g}_0=(\widehat{g}_0)_{MN}dz^Mdz^N$ where
$$(\widehat{g}_0)_{MN}=\pmatrix{\eta_{\mu\nu}  & 0 \cr 0 & 
(g_0)_{mn}(y)},$$ 
where $g_0$ is a positive definite metric on $K^6$.
Our notation is similar to the five-dimensional case.  The indices
$\mu,\nu$ run 
from $0$ to $3$ and index the coordinates $x^\mu$ on Minkowski space. The
indices $m,n$ index the internal coordinates $y^m$ which are local
coordinates on $K$. The indices $M,N$ run from $0$ to $9$ and index
local coordinates $z=(x,y)$ on the product ${\bf R}^4\times K^6$.
The split metric $\widehat{g}_0$ is Ricci-flat if and only if $g_0$ is
a Ricci flat metric on 
$K^6$. The existence of such a metric, of course, imposes a topological
condition on $Y$. We fix now a classical solution of this form.

As in the Kaluza-Klein model, let us study small fluctuations about
this classical solution:
$$h_{MN}(z)=\widehat{g}_{MN}(z)-(\widehat{g}_0)_{MN}.$$
Again, there is the gauge symmetry of the diffeomorphism of $X^{10}$
to deal with, and again we use the transverse
gauge.  The Ricci-flat condition to first order in $h_{MN}$ is
$$\left(\Box^4+\Box^6\right)h_{MN}(z)=0,$$
where
$$\Box^4h_{\mu\nu}=\eta^{\mu\nu}\partial_\mu\partial_\nu h_{\mu\nu}$$
and
$$\Box^6(h_{mn}) =
(g_0)^{ab}\nabla_a\nabla_bh_{mn}+2{{{R^a}_m}^b}_nh_{ab} .$$
Following the same line of reasoning as before, we expand in Fourier
modes in harmonics of $K^6$.
We write
$$h_{\mu\nu}(z)=\sum_kh_{\mu\nu;k}(x)Y_k(y)$$
where
$$\Box^6Y_k(y)=-\lambda_kY_k(y).$$
Here since the metric on $Y$ is positive definite, this D'Alembertian
on scalar functions on $Y$ and is the (negative of the) usual 
Laplacian. It is a non-positive operator, so that $\lambda_k\ge 0$ for
all $k$.
The equation satisfied by $h_{\mu\nu;k}$ is
$$(\Box^4-\lambda_k)h_{\mu\nu;k}(x)=0.$$
Once again we find that $h_{\mu\nu;k}$ is a solution to the wave
equation in four-space for a field of mass $\lambda_k$. If we scale
the metric on $K^6$ by multiplying it by $1/r$, then $\lambda_k$
scales by $1/r^2$. In particular, if we scale by $1/r$ for $r$ very
small, then for all $k$ for which $\lambda_k\not=0$, all
excitations of the field $h_{\mu\nu;k}(x)$ become extremely massive,
and hence can be ignored in our low energy effective action.

There is a similar analysis for the fields $h_{mn}(z)$.  We impose
the transverse gauge condition on the metric in the $K^6$-direction
and we expand the equations of motion in
the harmonics of $K^6$:
$$h_{mn}(z)=\sum_k\varphi_k(x)Y_{mn;k}(y),$$
where this time $Y_{mn;k}(y)dy^m dy^n$ are sections of the
symmetric square of the cotangent bundle of $Y$ which are eigenvectors with
eigenvalue $\lambda'_k$ for the D'Alembertian:
$$\Box^6Y_{mn;k}(y)=g^{ab}\nabla_a\nabla_bY_{mn;k}(y)
+2{{{R^a}_m}^b}_nY_{ab;k}(y)=-\lambda'_kY_{mn;k}(y).$$
Thus, we obtain
$$\Box^4\varphi_k(x)-\lambda'_k\varphi_k(x)=0.$$
As before, this is  the wave equation for a wave of mass $\lambda'_k$. 
 

Here, though, it is not necessarily the case that $\lambda'_k\ge 0$
for all 
$k$. There can be a finite number of negative eigenvalues for the
operator on $Y$. A negative eigenvalue means that our solution $g_0$
is a critical point for the Einstein-Hilbert action for positive
definite metrics on $Y$ which is not a local minimum for this action,
i.e., an unstable critical point.
We wish to avoid this possibility, so for now we simply assume that
our operator $\Box^6$ on $Y$ has no negative eigenvalues.
(This will be automatic by supersymmetry in the Calabi-Yau case.)
As long as $\lambda'_k>0$, all excitations of this field will have mass
on the order of $1/r^2$ and can be ignored in our low energy limit.
Thus, we keep only those $\varphi_k(x)$ for which $\lambda'_k=0$, i.e.,
we keep only those $\varphi_k(x)$ for which the corresponding
$Y_{mn;k}(y)dy^m dy^n$ is a harmonic section of the symmetric square
of the co-tangent bundle of $K^6$.  This space of harmonic forms is
the formal tangent space at $g_0$ to the moduli space of Ricci-flat
metrics on $K^6$.
So, for each parameter in the moduli space of solutions to the Einstein
equations on the internal space $K^6$ we have a massless scalar for
the effective action on four-space. 
These fields are the analogues of the massless field $\rho$ we found
in the Kaluza-Klein model, a field which came from the one parameter,
the radius, describing the metric on the internal space in that
theory. 


\section{Adding Matter to the Mix}

We have made a fairly complete analysis of pure gravity compactified
from ten dimensions to four dimensions by using a compact
six-dimensional manifold $K^6$ with a Ricci-flat metric. It is time
now to add other terms to the action so as to create a theory that has
matter as well as gravity.
Our goal here is to give the general nature of the terms that will add
matter, both bosons and fermions, to the action that we have been
studying. In the next section we will be more explicit as to the exact
nature of the terms we are adding. Here the analysis is quite general
and will be  sketched only since much of it is similar to
what we have done in great detail above for the metric.

First of all we can add boson fields by adding to the action a term of
the form $S(\varphi)$ where $\varphi$ is a scalar or vector-valued
function on ten-dimensional space-time.
Once again, we expand $\varphi$ in Fourier modes using the harmonics
of $Y$, and we can disregard all but the zero modes. In this way,
after compactification $\varphi$  becomes a boson field on four-space.

Let us consider fermions.  We fix a spin structure on ${\bf R}^4\times
Y^6$ leading to spinor fields $\psi$ which can be sections of the spin
bundles tensored with other auxiliary bundles with connections. 
The Clifford algebra for an inner product space of type $(1,9)$ is
${\bf R}[32]$, the $32\times 32$ real matrix algebra.  The even
component of the 
Clifford algebra is the block $16\times 16$ real matrices. 
This means that the spin bundle over $Y$ has a real structure and
decomposes as $S^+_{\bf R}(Y)\oplus S^-_{\bf R}(Y)$, each factor being a
$16$-dimensional real bundle.  A section of $S^+_{\bf
R}(Y)$ or $S^-_{\bf R}(Y)$ is called a Majorana-Weyl spinor.
Its chirality is by definition which of these two spin bundles it is a
section of. 
The
metric $\widehat{g}_0$ leads to a spinor connection and thus to the
real Dirac operator 
$$\dirac\colon \Gamma(S^\pm_{\bf R}(Y))\to \Gamma(S^\mp_{\bf R}(Y)).$$
In terms of the gamma-matrices for Clifford multiplication
we can write
$$\dirac=\gamma^mD_m$$
where $D_m$ is the spin covariant derivative in the direction of the
$m^{th}$ coordinate.  The Dirac equation on the ten-dimensional
manifold is then written as
$$i\dirac^{(10)}\psi=0.$$
Since the metric $\widehat{g}_0$ is split, so is the Dirac operator,
and in fact the Dirac equation is
$$i(\dirac^4+\dirac^6)\psi=0,$$
where $\dirac^4$ and $\dirac^6$ are the Dirac operators on Minkowski
four-space and $K^6$ respectively.
The solutions to this equation can be written as a sum
$$\psi(z)=\sum_k\psi_k(x)\psi'_k(y)$$
with
$$i\dirac^4\psi_k(x)=\lambda'_k\psi_k(x)$$ and
$$i\dirac^6\psi_k'(y)=-\lambda'_k\psi'_k(y).$$
Again, throwing away massive fluctuations, means keeping only the zero
modes of the Dirac equation on the internal space $K^6$.

\section{The Effective action from ten-dimensional heterotic string
theory} 

The effective action that heterotic string theory in ten dimensions
gives us is
\begin{eqnarray*}
\lefteqn{S_{\rm eff}[g,\varphi,\psi,\lambda,\chi,A,H] =} & &   \\
\quad & &= \frac{1}{2\kappa^2}\int_Y d^{10}x\sqrt{-{\rm
det}(g)}e^{-2\varphi}\left(R(g)+4(\nabla\varphi)^2-\frac{1}{3}H^2\right.  \\ 
& & \left. +\frac{\alpha'}{30}Tr\,
F_A^2-\overline\psi_M\gamma^{MNP}D_N\psi_P -\overline{\lambda}\dirac
\lambda -Tr\,\overline{\chi}\dirac\chi+\left(\cdots\right)\right).
\end{eqnarray*}
The fields in question are as follows:
There is a metric $g$ of signature $(-,+,\cdots+)$ on $Y$.
There is the dilaton field $\varphi$ which is a real scalar field.
There is a connection $A$
on a principal $E_8\times E_8$ or $SO(32)$ bundle
$P$ over $Y$. Its curvature is denoted $F_A$.
The axion form $H$ is a three-form with the property that
$$dH=\alpha'(tr\,R\wedge R-\frac{1}{30}Tr\, F_A\wedge F_A).$$
This means that we can write $H$ at least locally in the
form
$$H=dB+\alpha'\left(CS(\nabla_g)-\frac{1}{30}CS(A)\right),$$
where $B$ is a two-form and $CS(A)$ is the Chern-Simons three-form of
the connection $A$.
The boson fields $g,B,\varphi$ are grouped together in a multiplet (i.e.,
they are various components of an irreducible representation).  This
multiplet has a supersymmetric partner which is a multiplet given by a
pair of spinors $(\psi,\lambda)$. 
The spinor $\psi$  is a Majorana-Weyl spinor with values in the cotangent
bundle of $Y$; that is to say, it is  a section of
$S^+_{\bf R}(Y)\otimes T^*(Y)$.  It  is called the
{\sl gravitino}.
The spinor $\lambda$ is a Majorana-Weyl spinor of opposite
chirality to that of $\psi$.  That is to say, it is a section of
$S^-_{\bf R}(Y)$. It is  called the {\sl dilatino}.
The gauge field $A$ has its 
supersymmetric partner $\chi$, called the {\sl
gaugino}.  This field  is a Majorana-Weyl
spinor with values in the adjoint bundle of $P$
and is of the same chirality as $\psi$. That is to say, it is a
section of $S^+_{\bf R}(Y)\otimes {\rm ad} P$.



Having discussed the fields that occur,  let us turn to
the functions of them that appear in the action.
As usual, $R(g)$ is the scalar curvature of $g$ and $TrF_A^2$ 
is the norm of $F_A$ in the $G$-invariant metric in the adjoint
bundle. 
Lastly, $\gamma^{MNP}$ is obtained from the gamma-matrices by
skew-symmetrizing: 
$$\gamma^{MNP}=\gamma^{[M}\gamma^N\gamma^{P]}.$$
These are the gamma-matrices that give the representation of
three-forms on spinors.  
In language more familiar to mathematicians, all the spinor pairings in
the action 
arise from the two basic pairings: the nondegenerate inner product
between spinors of the opposite chirality:
$$\langle\ \ \rangle\colon S^+_{\bf R}(Y)\otimes S^-_{\bf R}(Y)\to {\bf R}$$
and the adjoint of Clifford multiplication
\begin{equation}\label{cm}
S^+_{\bf R}(Y)\otimes S^+_{\bf R}(Y)\to T(Y).
\end{equation}
(Here we are viewing Clifford multiplication as a pairing
$T(Y)\otimes S^+_{\bf R}(Y)\to S^-_{\bf R}(Y)$. To write its adjoint
as we have above, we use the isomorphisms $T(Y)\cong T^*(Y)$, given by
the metric, and $S^+_{\bf R}(Y)\cong S^-_{\bf R}(Y)^*$, given by the
inner product.)
The term $\overline\lambda\dirac\lambda$ in the action is simply the
inner product
$\langle \lambda,\dirac\lambda\rangle$. 
Likewise, $Tr\overline\chi\dirac\chi$ is $Tr\langle
\chi,\dirac\chi\rangle$ defined from the inner product on spinors and
the trace in the adjoint representation.  Lastly, the pairing
$\overline\psi_M\gamma^{MNP}D_N\psi_P$ in the action means 
the pairing $\langle \psi,\nabla\psi\rangle$, where the pairing is
between a one-form with values in the positive spin bundle and a
two-form with values in the same spin bundle. The pairing is given by wedging
together the forms to produce a three-form and then letting that
three-form  operate by
Clifford multiplication on one of the spinors to produce a spinor of
minus chirality which can then be paired with the plus chirality
spinor to produce a real number.


In addition, there are other terms (indicated by $(\cdots)$) that we
have not explicitly written 
down and which are not important for the analysis that we shall make.
These extra terms come in two types.  Some are required by
supersymmetry.  There are, for example, terms which are 
fourth power in the spinor fields such as
$(\overline{\psi}\psi)^2$. There are also terms which are higher order
in $\alpha'$, which is a dimensionful parameter.  These can be ignored
if we assume that, even though $K^6$ is on the Planck scale, it is 
much larger that the string length.


This action clearly has a group of symmetries: the group of
orientation-preserving diffeomorphisms of the underlying manifold. 
But in addition, we have gone to great lengths to build it so that it
has supersymmetry, $N=1$ local supersymmetry to be more precise.  For
example,  our boson fields form two multiplets 
$(g,B,\varphi)$ and $A$, and each multiplet  has its fermionic partner
$(\psi,\lambda)$ and $\chi$,  respectively. 
This of course, is only a necessary condition for the existence of
supersymmetry. But in fact the $N=1$ local supersymmetry algebra is
generated by sections $\epsilon$ of the Majorana-Weyl spin bundle
$S^+_{\bf R}(Y)$. These generators  are odd elements  in the super Lie
algebra. 
The infinitessimal transformation laws are partially given by:
\begin{eqnarray*}
\delta_\epsilon\chi & = & F_{MN}\gamma^{MN}\epsilon+({\rm fermions}^2 \\
\delta_\epsilon\psi_M  &  =  &  \nabla_M\epsilon -\frac{1}{4}
H_{MAB}\gamma^{AB}\epsilon +({\rm fermions})^2 \\
\delta_\epsilon\lambda  &  =  & (\gamma^M \nabla_M\varphi)\epsilon
+\frac{1}{24}H_{MNP}\gamma^{MNP}\epsilon +({\rm fermions})^2 \\
\delta_\epsilon g_{MN}  &  =  &  \overline\epsilon
\gamma_N\psi_M+\overline\epsilon\gamma_M \psi_N+({\rm fermions})^2 \\
\delta_\epsilon A_M  &  =  &  \overline\epsilon \gamma_M\chi +({\rm
fermions})^2 \\
\delta_\epsilon B & = &  \cdots \\
\delta_\epsilon \phi & = & \cdots 
\end{eqnarray*}
The complete laws can be written down, but we shall not do that here
as we will not need them. 

The space of classical solutions is invariant under the supersymmetry
algebra generated by these infinitessimal (odd) symmetries, and on the
space of solutions the algebra generated by these infinitessimal
symmetries closes to yield the $N=1$ local supersymmetry algebra.
In fact, one has the following formula for the supercommutator: 
$$[\delta_\epsilon,\delta_{\epsilon'}]=\delta_v+ T$$
where $v$ is the vector field obtained by pairing $\epsilon$ and
$\epsilon'$ under the pairing given in Equation~\ref{cm}.  Written in
terms of Clifford matrices we have
$$v^m=\overline{\epsilon'}\gamma^m\epsilon.$$
The term  $T$ is an expression which includes gauge transformations
and terms which vanish when one imposes the
equations of motion. 


All of this was done without worrying about the compactification.
The next step in the study is to show under what conditions  we can
compactify down to four-dimensions. In doing this compactification
much of  the $N=1$ local supersymmetry will be broken, but in special
circumstances we will be able to retain $N=1$ super Poincar\'e
symmetry in dimension four.
This is where the next lecture will begin.
\end{document}







