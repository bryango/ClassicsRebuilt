% use amslatex v1.1

\documentstyle[11pt]{amsart}
%\NoPageNumbers
\input epsf

\def\epsfsize#1#2{0.5#1}

%\setlength{\textwidth}{425pt}
%\setlength{\textheight}{550pt}
\setlength{\textwidth}{450pt}
\setlength{\textheight}{600pt}
\setlength{\topmargin}{0pt}
\setlength{\oddsidemargin}{0pt}
\setlength{\evensidemargin}{0pt}

\pagestyle{empty}

\begin{document}

\centerline{\bf MAPPING CLASS GROUP ACTION ON SPIN STRUCTURES}
\medskip
\centerline{Rob Kusner}
%\medskip
%\centerline{March, 1997}
\bigskip

A spin structure on a (punctured, marked) orientable 
surface is induced by an
immersion into $S^2={\Bbb CP}^1={\Bbb C}\cup\{\infty\}$ from the unique
spin structure `$\sqrt{dz}$' on $S^2$.
In the case of an annulus, the two spin structures are:

\centerline{\epsfbox{anneven.eps} \hskip1.5in
\epsfbox{ann2.eps}}

The one on the left (EVEN) extends to an immersion of the disk,
whereas the one on the right (ODD) does not extend.  Regularly
homotopic immersions correspond to equivalent spin structures,
while regularly homotopic images correspond to spin structures
which are equivalent up to change of marking.  The mapping class
group of the surface, by definition, effects this change of
marking.

The mapping class group is generated by 
Dehn twists.  Using a ``disk with
bands'' picture, these twists can be visualized as ``band slides''.
The three even spin structures on a torus, and their equivalence 
up to change of marking, can be represented as:



\centerline{\epsfbox{tor3.eps}}

\noindent
and the odd spin structure on the torus as:   


%\noindent\hfill\epsfbox{toroddodd.eps}\hskip 1.5in 
%\raisebox{48pt}{\parbox{3in}{
\centerline{\epsfbox{toroddodd.eps}}
\noindent
Note that the rules of (mod 2) multiplication\hfill\break
\medskip
\centerline{ODD$\,\cdot\,$EVEN=EVEN$\,\cdot\,$EVEN=EVEN$\,
\cdot\,$ODD=even}
\smallskip
\centerline{ODD$\,\cdot\,$ODD=odd}
\smallskip\noindent
correspond to the plumbing 
together of the
annular bands (sliding an ODD band over another ODD band
yields an ODD band!)
For surfaces of higher genus, 
decompose as a connected sum of tori and use (mod 2) addition:

\smallskip

\centerline{odd + odd = even = even + even}

\centerline{even + odd = odd = odd + even}

\smallskip

\noindent
(Check by using band slides that these 
are the only two classes of
spin structure (up to change of marking), 
and furthermore, that the
mapping class group transitively permutes all the even spin
structures -- and similarly, 
all the odd spin structures -- amongst
themselves!)

\smallskip

\noindent
Algebraically, these pictures show how to associate a (mod 2)
quadratic form with each spin structure on a (marked) orientable
surface.  The mapping class group acts on the first (mod 2)
homology group (by intersection-preserving change of basis)
and the (mod 2) Arf-invariant (EVEN or ODD) of this
quadratic form, distinguishes the two
orbits.  For non-orientable surfaces, 
there is an analogous picture
for ``pin structures'' using immersions to $RP^2$; at the
algrebraic level, each pin structure corresponds to a (mod 4)
quadratic form and the mapping class 
group orbits are determined by
its (mod 8) Arf-invariant.
%\hfil}}

\end{document}




\noindent
and the odd spin structure on the torus as

\centerline{\epsfbox{toroddodd.eps}}

\noindent
(Sliding an ODD band over another ODD band yields an ODD band!)

For surfaces of higher genus, 
decompose as connected sum and use (mod 2) arithmetic:

\centerline{ODD + ODD = EVEN = EVEN + EVEN}

\centerline{EVEN + ODD = ODD = ODD + EVEN}

\noindent
(Check by using band slides!)


\end{document}




