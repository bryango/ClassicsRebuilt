%From: Pierre Deligne <deligne@IAS.EDU>
%Date: Fri, 3 Jan 1997 11:16:44 -0500

\input amstex
\documentstyle{amsppt}
\magnification=1200
\loadbold
\loadeusm

\font\boldtitlefont=cmb10 scaled\magstep2

\NoRunningHeads
\pagewidth{6.5 true in}
\pageheight{8.9 true in}

\catcode`\@=11
\def\logo@{}
\catcode`\@=13

\def\eps{{\varepsilon}}

\def\dbR{{\Bbb R}}

\def\undertext#1{$\underline{\vphantom{y}\hbox{#1}}$}
\def\nspace{\lineskip=1pt\baselineskip=12pt%
     \lineskiplimit=0pt}
\def\dspace{\lineskip=2pt\baselineskip=18pt%
     \lineskiplimit=0pt}

\def\w{{\mathchoice{\,{\scriptstyle\wedge}\,}
  {{\scriptstyle\wedge}}
  {{\scriptscriptstyle\wedge}}{{\scriptscriptstyle\wedge}}}}
\def\Le{{\mathchoice{\,{\scriptstyle\le}\,}
{\,{\scriptstyle\le}\,}
{\,{\scriptscriptstyle\le}\,}{\,{\scriptscriptstyle\le}\,}}}
\def\Ge{{\mathchoice{\,{\scriptstyle\ge}\,}
{\,{\scriptstyle\ge}\,}
{\,{\scriptscriptstyle\ge}\,}{\,{\scriptscriptstyle\ge}\,}}}

\def\red{\text{\rm red}}

\def\bfs{\bold{s}}

%Script letters:
\def\scr#1{{\fam\eusmfam\relax#1}}
\def\scrE{{\scr E}} 
\def\scrL{{\scr L}} 
\def\scrM{{\scr M}} 

\def\scrLbar{\overline{\scrL}}


\topmatter
\title\nofrills
{\boldtitlefont Appendix: the 2-form on a space of extremals}
\endtitle
\author
P. Deligne
\endauthor
\endtopmatter

\NoBlackBoxes
\parindent=20pt
\frenchspacing
\document
\bigskip
\dspace
In his lectures, Bernstein considered lagrangian
densities $\scrL(\phi)$ on $V$, for $\phi$ in the space
$\scrM$ of maps $V\to M$, or more generally for $\phi$ a
section of a fiber bundle $M_V$ over $V$.
The case of maps to $M$ correspond to $M_V=M\times V$.
He considered the case where the density $\scrL(\phi)$
at $v\in V$ depends only on $\phi$ and $d\phi$ at $v$,
and explained how to construct a closed $2$-form
$\omega$ on the space $\scrE$ of extremals.
The construction depended on a cohomology class of
hypersurfaces $\Gamma$ with an oriented normal bundle:
$\Gamma$ has a ``past'' side and a ``future'' side.
In practice, $\Gamma$ is a ``space-like'' hypersurface.
The construction uses integration on $\Gamma$, hence
requires some control at the infinity of $\Gamma$, and a
similar control on homologies among $\Gamma$'s.
We will ignore this by pretending that $\Gamma$ is
compact and that cohomology is taken with compact
support.

In this appendix, our aim is to explain a small
modification of the construction which works as well
when $\scrL(\phi)$ depends on higher derivatives of
$\phi$.
One can also take $\phi$ to be for instance a bundle
with connection, rather than a map to some $M$.
If $V$ is a supermanifold, the new construction makes
clear that the $2$-form obtained is the same whether one
works in term of superfields $\phi\colon\,V\to M$ and of
superdensities $\scrL(\phi)$ on $V$, or in term of
``components'' of $\phi$ and of the Lagrangian density
$\scrLbar(\phi)$ on $V_{\red}$ deduced from
$\scrL(\phi)$ by ``integrating out'' the odd variables,
for some projection $V\to V_{\red}$.

What we don't get are $2$-forms $\omega_\Gamma$,
depending on $\Gamma$, on the space $\scrM$ of all
$\phi$, and inducing $\omega$ on the space $\scrE$ of
extremals.

The basic construction is local around a space-like
hypersurface $\Gamma$ and, to explain it, we will assume
that $V=\Gamma\times\dbR$, the projection to $\dbR$
being called ``time''.

For a $2$-parameter family $\phi[a_1,a_2]$ of extremals,
we now define the value at $(a_1,a_2)=(0,0)$ of the pull
back of $\omega$ by $(a_1,a_2)\mapsto\phi[a_1,a_2]$.
We choose a family $\phi_1[a_,a_2]$ of maps to $M$,
deforming $\phi[a_1,0]$, agreeing with $\phi[a_1,a_2]$
in the future, and agreeing with $\phi[a_1,0]$ in the
past:
$$
\alignat2
\phi_1[a_1,0] &=\phi[a_1,0], &\qquad &\\
\phi_1[a_1,a_2] &=\phi[a_1,a_2] &&\text{in the (far) future},\\
\phi_1[a_1,a_2] &=\phi[a_1,0] &&\text{in the (far)
past}.
\endalignat
$$

We similarly choose $\phi_2$, having the same properties
with the roles $a_1$ and $a_2$ permuted.

\noindent
Definition:
$$
\omega(\partial_1,\partial_2)=\int_V
(\partial_1\partial_2\scrL(\phi_1)-\partial_2\partial_1
\scrL(\phi_2))\quad\text{at}\quad (a_1,a_2)= (0,0).
$$
The integrand vanishes in the future, where
$\phi_1=\phi_2$, as well as in the past, where both
terms vanish.
This makes the integral defined.

Using $\Gamma$, with equation $t=0$, we now express
$\omega$ as an exterior derivative $d\alpha_\Gamma$ and
show independence of the choices of $\phi_1$ and
$\phi_2$.
Let $Y$ be the function on $\dbR$ which is $1$ for $x<0$
and $0$ for $x>0$: \ $\frac{d}{dx}\,Y=-\delta(x)$.
We will have to integrate on $V$ up to $t=0$ some smooth
density $\mu$ vanishing in the past.
In the supervariety case, this means integrating
$Y(t)\mu$ (integration on a supervariety with boundary
$\Gamma$).
This depends on $\Gamma$, not just on $\Gamma_{\red}$.

The $1$-form $\alpha_\Gamma$ is defined only at
extremals, but makes sense on tangent vectors to the
space $\scrM$ of all $\phi$.
Fix an extremal $\phi[0]$ and consider a family
$\phi[a]$ deforming it.
We now define $\alpha_\Gamma$ at the corresponding
tangent vector.
Let $\phi_1$ be a deformation of $\phi[0]$ agreeing with
$\phi[a]$ for $t>-\eps$, and with $\phi[0]$ in the past.

\noindent
Definition:
$$
\alpha_\Gamma(\partial_a)=\int
Y(t)\partial_a\scrL(\phi_1[a])\quad\text{at}\quad
a=0
$$
The integrand vanishes in the future, because $Y(t)$
does, and in the past, where $\phi_1[a]$ is constant.
This makes the integral defined.
Because $\phi[0]$ is an extremal, it does not depend on
the choice of $\phi_1$: \ the density
$\partial_a\scrL(\phi_1[a])$ is the variation of the
density $\scrL$ for a variation $\delta_1\phi$ of
$\phi[0]$, and if we consider another deformation
$\phi_2$, $\delta_1\phi-\delta_2\phi$ has support in a
time interval $[-A,-\eps]$.
It is the tangent vector for a deformation $\phi_3$ with
a support in $t<0$ and
$$
\split
\int Y(t) &[\partial_a\scrL(\phi_1[a])-\partial_a\scrL(\phi_2
  [a])]\quad\text{(at $a=0$)}\\
&=\int Y(t)\partial_a\scrL(\phi_3[a])\quad
\text{(at $a=0$)}=\int\partial_a\scrL(\phi_3[a])
\quad\text{(at $a=0$)}=0
\endsplit
$$

To see that $\omega=d\alpha_\Gamma$, one uses the
formula that for commuting vector fields $X$ and $Y$,
one has
$$
(d\alpha)(X,Y)=X.\alpha(Y)-Y.\alpha(X)\,\,.
$$
For the pull back of $\alpha_\Gamma$ by
$(a_1,a_2)\mapsto\phi[a_1,a_2]$, this gives, provided
that $\phi_1[a_1,a_2]=\phi[a_1 a_2]=\phi[a_1 a_2]$ for
$t>-\eps$:
$$
\align
d\alpha_\Gamma(\partial_1,\partial_2) &=\partial_1\int
  Y(t)\partial_2\scrL(\phi_1)-\partial_2\int Y(t)
  \partial_1\scrL(\phi_2)\\
&=\int Y(t)(\partial_1\partial_2\scrL(\phi_q)-
  \partial_2\partial_1\scrL(\phi_2))\\
&=\int \partial_1\partial_2\scrL(\phi_1)-\partial_2\partial_1
  \scrL(\phi_2)
\endalign
$$
at $(a_1,a_2)=(0,0)$: \ the definition of $\omega$.

In the case treated by Bernstein, integration by parts
shows that the form $\alpha_\Gamma$ is the restriction
to $\scrE$ of the one he considers, on $\scrM$.

The modification we presented does not require the
tangent space of $\scrM$ at $\phi$ to be the space of
sections of a vector bundle on $V$.
Basically, it requires only that it be the space of
sections of a soft sheave of $\dbR$-vector spaces on
$V$.
One has to be able to deform $\phi$ in a prescribed way
in different regions of $V$.
For $V$ a supermanifold, this does not preclude imposing
suitable constraints on $\phi$.





\enddocument





