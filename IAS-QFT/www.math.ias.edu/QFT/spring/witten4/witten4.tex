%**start of header
\input amstex
\documentstyle{amsppt}
\magnification=\magstephalf
%%%%%%%%%%%% changes to amsppt.sty %%%%%%%%%%%%%%%%%%%%%%%
 \addto\tenpoint{\baselineskip 15pt
  \abovedisplayskip18pt plus4.5pt minus9pt
  \belowdisplayskip\abovedisplayskip
  \abovedisplayshortskip0pt plus4.5pt
  \belowdisplayshortskip10.5pt plus4.5pt minus6pt}\tenpoint
\pagewidth{6.5truein} \pageheight{8.9truein}
\subheadskip\bigskipamount
\belowheadskip\bigskipamount
\aboveheadskip=3\bigskipamount
\catcode`\@=11
\def\output@{\shipout\vbox{%
 \ifrunheads@ \makeheadline \pagebody
       \else \pagebody \fi \makefootline 
 }%
 \advancepageno \ifnum\outputpenalty>-\@MM\else\dosupereject\fi}
\outer\def\subhead#1\endsubhead{\par\penaltyandskip@{-100}\subheadskip
  \noindent{\subheadfont@\ignorespaces#1\unskip\endgraf}\removelastskip
  \nobreak\medskip\noindent}
\outer\def\enddocument{\par% \par will do a runaway check for \endref
  \add@missing\endRefs
  \add@missing\endroster \add@missing\endproclaim
  \add@missing\enddefinition
  \add@missing\enddemo \add@missing\endremark \add@missing\endexample
 \ifmonograph@ % do nothing
 \else
 \vfill
 \nobreak
 \thetranslator@
 \count@\z@ \loop\ifnum\count@<\addresscount@\advance\count@\@ne
 \csname address\number\count@\endcsname
 \csname email\number\count@\endcsname
 \repeat
\fi
 \supereject\end}
\catcode`\@=\active
%%%%%%%%%%%%%%% other macros %%%%%%%%%%%%%%%%%%%%%%%%%%%%
\CenteredTagsOnSplits
\NoBlackBoxes
\nologo
\def\today{\ifcase\month\or
 January\or February\or March\or April\or May\or June\or
 July\or August\or September\or October\or November\or December\fi
 \space\number\day, \number\year}
\define\({\left(}
\define\){\right)}
\define\Ahat{{\hat A}}
\define\Aut{\operatorname{Aut}}
\define\CC{{\Bbb C}}
\define\CP{{\Bbb C\Bbb P}}
\define\Conf{\operatorname{Conf}}
\define\Diff{\operatorname{Diff}}
\define\EE{\Bbb E}
\define\FF{\Bbb F}
\define\End{\operatorname{End}}
\define\Free{\operatorname{Free}}
\define\HH{{\Bbb H}}
\define\Hom{\operatorname{Hom}}
\define\Map{\operatorname{Map}}
\define\Met{\operatorname{Met}}
\define\QQ{{\Bbb Q}}
\define\RP{{\Bbb R\Bbb P}}
\define\RR{{\Bbb R}}
\define\SS{\Bbb S}
\define\Spin{\operatorname{Spin}}
\define\Tor{\operatorname{Tor}}
\define\Trs{\operatorname{Tr\mstrut _s}}
\define\Tr{\operatorname{Tr}}
\define\ZZ{{\Bbb Z}}
\define\[{\left[}
\define\]{\right]}
\define\ch{\operatorname{ch}}
\define\chiup{\raise.5ex\hbox{$\chi$}}
\define\cir{S^1}
\define\coker{\operatorname{coker}}
\define\dbar{{\bar\partial}}
\define\endexer{\bigskip\tenpoint}
%\define\exertag #1#2{\removelastskip\bigskip\medskip\eightpoint\noindent%
%\hbox{\rm\ignorespaces#2\unskip} #1.\ }  
\define\exertag #1#2{#2\ #1}
\define\free{\operatorname{free}}
\define\index{\operatorname{index}}
\define\ind{\operatorname{ind}}
\define\inv{^{-1}}
\define\mstrut{^{\vphantom{1*\prime y}}}
\define\protag#1 #2{#2\ #1}
\define\rank{\operatorname{rank}}
\define\res#1{\negmedspace\bigm|_{#1}}
\define\temsquare{\raise3.5pt\hbox{\boxed{ }}}
\define\theexertag{\theprotag}
\define\theprotag#1 #2{#2~#1}
\define\tor{\operatorname{tor}}
\define\xca#1{\removelastskip\medskip\noindent{\smc%
#1\unskip.}\enspace\ignorespaces }
\define\endxca{\medskip}
\define\zmod#1{\ZZ/#1\ZZ}
\define\zn{\zmod{n}}
\define\zt{\zmod2}
\redefine\Im{\operatorname{Im}}
\redefine\Re{\operatorname{Re}}
\let\germ\frak


\catcode`\@=11
\def\Head#1#2{\add@missing\endroster \add@missing\enddefinition
  \add@missing\enddemo \add@missing\endexample
  \add@missing\endproclaim
  \par\penaltyandskip@{-100}\subheadskip
  \noindent{\subheadfont@\ignorespaces \S#1. #2\unskip\endgraf}\removelastskip
  \nobreak\medskip\noindent}
\catcode`\@=\active

\NoRunningHeads % USE IN FINAL VERSION; THEN COMMENT OUT NEXT LINE
% \headline{\eightpoint PRELIMINARY VERSION \hfil \today}

\define\Grk{Gr(k,N)}
\define\Gtil{\tilde{G}}
\define\Herm{\operatorname{HermitianEnd}}
\define\Hf{\scrH_\phi }
\define\Hgauge{\Cal H_{\text{gauge}}}
\define\Ne{N_e}
\define\Qf{Q_{\phi }}
\define\Seff{S_{\text{eff}}}
\define\Sym{\operatorname{Sym}}
\define\TT{\Bbb T}
\define\Vol{\operatorname{Vol}}
\define\Wbar{\overline{W}}
\define\bphi{\bar{\phi }}
\define\geff{g_{\text{eff}}}
\define\gtil{\tilde{g}}
\define\hol{\operatorname{hol}}
\define\id{\operatorname{id}}
\define\modn{\Cal M_n}
\define\pf{\pi _\phi }
\define\phih{\hat{\phi}}
\define\scrA{\Cal{A}}
\define\scrG{\Cal{G}}
\define\scrH{\Cal{H}}
\define\sigtil{\tilde{\sigma}}
\define\voll{\operatorname{vol}}
\define\wbar{\bar{w}}

\input epsf

\document
%**end of header


	\topmatter
 \title\nofrills Lecture II-4: The large~$N$ limit of the $\CP^{N-1}$ model
\endtitle 
 \author Edward Witten  \endauthor
	\endtopmatter

\centerline{Notes by Dan Freed}
\bigskip 
\bigskip

\document


 \comment
 lasteqno @ 34
 \endcomment

        \proclaim{Remark}
 The lecture treated the $\sigma $-model into projective space, but these
notes cover the generalization to a Grassmannian, as requested in Problem
Set~3.
        \endproclaim

\bigskip 
 
In this lecture we discuss the large~$N$ behavior of the two dimensional
$\sigma $-model into the Grassmannian~$Gr(k,N)$ of $k$~dimensional subspaces
of~$\CC^N$.  Here $k$~is fixed as~$N\to\infty $.  We also consider the real
Grassmannian.  Since Grassmannians have positive Ricci curvature these field
theories are asymptotically free, but in any case our task is to investigate
the infrared behavior. 
 
The Euclidean action of the $\sigma $-model is 
  $$ S[\phi ] = \frac{1}{g^2}\int_{\Sigma }d^2x\,|d\phi |^2 - i\theta
     \int_{\Sigma} \phi ^*(\alpha ), \tag{1} $$
where $\Sigma$~is a Riemann surface, $\phi \:\Sigma \to Gr(k,N)$ a map into
the Grassmannian, $\alpha \in H^2\bigl(Gr(k,N),\ZZ \bigr)$ a generator of the
cohomology, and $g,\theta $~are parameters of the theory.  We specify the
metric on~$Gr(k,N)$ shortly.  The second term is a topological
term;\footnote{The factor of~$i$ is present in the {\it Euclidean\/} action
so that the action conjugates under orientation reversal.  In this way its
continuation to Minkowski space is real.  See Freed's notes {\it Actions and
Reality\/} for more details.} for $\Sigma $~closed the integral is
integer-valued.  Thus a shift $\theta \to\theta +2\pi $ does not affect the
model.  The parameter~$\theta $ is also a parameter of the quantum theory,
but renormalization exchanges the dimensionless coupling constant~$g$ with a
mass parameter~$\mu $.

The rescaled coupling constant~$\gtil$, defined by 
  $$ g^2 = \gtil^2/N,  \tag{2} $$
is more natural in the large~$N$ limit, as we will see.


 \Head{4.1}{The Questions}
We ask specific questions about the behavior of the model. 

 \roster
 \item"1." Is there a mass gap? 
 \item"2." What is the $\theta $ dependence of the partition function?

\noindent {\it Remark.\/}
 For $\Sigma $~small the partition function 
  $$ Z(\theta ) = \int_{} D\phi \;\;e^{-\frac{1}{\geff^2}\int_{\Sigma
     }d^2x\,|d\phi |^2} \,e^{i\theta \int_{\Sigma }\phi ^*(\alpha )}
      $$
can be studied using perturbation theory.  Here $\geff$~is the effective
coupling, which varies with the distance scale in the theory set by the size
of the surface~$\Sigma $.  (Classically the model is conformally invariant,
so does not depend on the size of~$\Sigma $, but quantum mechanically this is
no longer true.)  By asymptotic freedom this coupling is small at small
distances, hence the assertion that perturbation theory applies in this
regime.  Now the classical solutions to~\thetag{1} are harmonic maps $\Sigma
\to Gr(k,N)$.  Note that the only $\theta $~dependence is through the degree
of~$\phi $, and so we split the integral as a sum over maps~$\phi $ of
varying degrees.  In degree~0 we obtain constant maps and in general for
degree~$n$ some moduli space~$\modn$ of harmonic maps, which are the {\it
instantons\/} of this model.  Degree~1 instantons are (anti)holomorphic maps.
The perturbation expansion around these solutions has the rough form
  $$ Z(\theta ) \sim \frac{\Vol\bigl(Gr(k,N) \bigr)}{\sqrt{\det(\cdot
     )}}(1+\cdots ) \,+\, \sum\limits_{\pm}e^{\pm i\theta }e^{-cN/\geff^2}
     \frac{\Vol{\Cal{M}_1}}{\sqrt{\det(\cdot )}}(1+\cdots ) + \text{higher
     instantons} \tag{3} $$
The constant~$c$ in the exponential is the action of a $1$-instanton, which
is independent of~$N$.  The factor of~$N$ comes from~\thetag{2}.  If $\geff
<< 1$ we see that the $\theta $~dependent term of~$Z(\theta )$ vanishes
exponentially as~$N\to\infty $.  As the area of~$\Sigma $ increases the
effective coupling~$\geff$ also increases.  Equation~\thetag{3} is the answer
for $\Sigma $~compact and of small area, but we will find a vastly different
result for~$\Sigma =\RR^2$.

 \item"3." {\it Symmetry Breaking.\/}  Symmetry breaking in two dimensions
is possible for a discrete symmetry, and in this model we have the parity
symmetry
  $$ \aligned
      P\:\Sigma &\longmapsto \overline{\Sigma}\qquad \qquad
     \text{($\overline{\Sigma}$~is $\Sigma$ with the opposite orientation)}
     \\
      \theta &\longmapsto -\theta \endaligned \tag{4} $$
For $\Sigma =\RR^2$ we implement the orientation reversal by an
orientation-reversing isometry, i.e., a reflection.  Then for $\theta =0$~
and $\theta =\pi$ ~the parity symmetry~$P$ acts on a fixed theory and we ask
if it is broken in the quantum theory.
 
 \item"4." The group~$PSU(N)$ of isometries of the Grassmannian acts in the
classical theory.  Since continuous symmetry groups are unbroken in two
dimensions, this symmetry acts in the quantum theory as well.  However, it is
possible that a realization of the theory has a symmetry group which is a
cover of~$PSU(N)$, the latter being the group which acts on the operator
algebra.  Does that happen here?
 \endroster


 \Head{4.2}{An Equivalent Formulation}
To study the model we rewrite it, that is, we construct an action with the
same classical and quantum physics.  As a preliminary we recall some basic
geometry of the Grassmannian.  Over~$Gr(k,N)$ lies a canonical sequence of
vector bundles
  $$ 0 @>>> S @>s>> Gr(k,N)\times \CC^N @>>> Q @>>> 0,  \tag{5} $$
where the fiber of~$S$ at a $k$-plane~$\pi $ is simply $\pi $~viewed as a
subspace of~$\CC^N$.  Fix the standard metric on~$\CC^N$. It induces a metric
on~$S$ and identifies~$Q\cong S^\perp$.  There is a canonical
connection~$\nabla $ on~$S$, obtained by projecting the natural connection on
the trivial bundle~$\Grk\times \CC^N$.  We easily compute
  $$ \nabla =d-s^*ds, \tag{6} $$
where $s$~is the inclusion $S@>s>>\Grk\times \CC^N$.  Then 
  $$ \nabla s\:T\bigl(\Grk\bigr)\longrightarrow \Hom(S,S^\perp) \tag{7} $$
is an isomorphism.  We use it to induce a metric on~$\Grk$, the metric needed
to write down the $\sigma $-model action~\thetag{1}. 
 
Now if $\phi \:\Sigma \to\Grk$ we pullback~\thetag{5} to obtain a sequence of
bundles over~$\Sigma $, and by~\thetag{7} the lagrangian density of~$\phi$ is
  $$ |d\phi |^2 = |(\phi ^*\nabla )\phi ^*s|^2. \tag{8} $$
Note $\phi ^*s\:\phi ^*S\to\Sigma \times \CC^N$ determines~$\phi $.  The idea
is to replace~$\phi $ by such a bundle map, and so first to replace~$\phi
^*S$ by a fixed bundle.  Note $\deg(\phi ^*S)=\deg(\phi )$ so that the
topology of~$\phi ^*S$ determines the cohomology class~$\phi ^*(\alpha )$,
which appears in the second term of the action~\thetag{1}.  Hence fix a
vector bundle $E\to\Sigma $ of rank~$k$ and degree~$d$.  Also fix a hermitian
metric on~$E$.  We introduce a new field 
  $$ \phih\:E\longrightarrow \Sigma \times \CC^N  $$
which we constrain to be an isometric immersion: 
  $$ \phih^*\phih=\id_E. \tag{9} $$ 
The image of~$\phih$ determines a map $\phi \:\Sigma \to\Grk$ which is
unchanged if we shift~$\phih$ by a unitary gauge transformation of~$E$.  To
rewrite~\thetag{8} in terms of~$\phih$ we need a connection on~$E$, and as
there is no natural choice we introduce a variable unitary connection~$A$.
Using~$\phih$ we identify~$E$ with a subbundle of the trivial bundle $\Sigma
\times \CC^N$, so can differentiate~$\phih$ using the usual derivative~$d$.
Writing~$A$ as a 1-form plus this trivial connection we find
  $$ \aligned
      |d_A\phih|^2 &= |d\phih + \phih A|^2 \\ 
      &= |d\phih|^2 + 2\Re(d\phih,\phih A) + |A|^2,\endaligned \tag{10} $$
since~$\phih^*\phih=\id_E$.  This expression is quadratic in~$A$, so if
\thetag{10}~is a classical lagrangian for~$A$ we can use the equations of
motion to obtain
  $$ A_0 = - \phih^*d\phih. \tag{11} $$
Comparing with~\thetag{6} we see that $A_0$~is the pullback of the canonical
connection on~$S$, and so by~\thetag{8} 
  $$ |d\phi |^2 = |d_{A_0}\phih|^2.  $$
In other words, the lagrangian~\thetag{10} is equivalent to~$|d\phi |^2$ for
fields which satisfy the constraint~\thetag{9}.  We impose the constraint
via a lagrange multiplier field
  $$ \sigma \:\Sigma \longrightarrow \Herm(E). $$ 
The $\theta $~term in the original action~\thetag{1} can be computed using
the (skew-Hermitian) curvature~$F_A$ via Chern-Weil theory.  Altogether we
obtain for our new action\footnote{Some explanation about the lagrange
multiplier term is in order.  It is based on the general formula
  $$ \int_{}e^{i(\sigma ,x)}d\sigma  = \delta (x), $$ 
where $\sigma $~lies in the dual space to~$x$, the measure~$d\sigma $ is
suitable normalized, and $\delta (x)$~is the $\delta $-distribution
supported at the origin.  In the Minkowski space lagrangian the lagrange
multiplier term is 
  $$ \int_{\Sigma }\Tr \sigma (\phih^*\phih - \id_E), $$ 
and $\sigma $~should be interpreted as a 2-form (or density);  it lies in the
dual space to the function $\phih^*\phih-\id_E$.  Rotation to Euclidean space
yields the second term of~\thetag{12}.} 
  $$ S[\phih,A,\sigma ] = \frac{1}{g^2}\int_{\Sigma }|d_A\phih^2| -
     i\int_{\Sigma} \Tr\sigma (\phih^*\phih-\id_E) + \frac{\theta }{2\pi
     }\int_{\Sigma }\Tr F_A. \tag{12} $$
As we have explained, the classical equations of motion (and other classical
constructs) computed from~\thetag{12} are equivalent to those computed from
the original action~\thetag{1}.  The classical computation which led
to~\thetag{11} is valid quantum mechanically since the dependence
of~\thetag{12} on~$A$ is quadratic and the Hessian is the identity operator
(see~\thetag{10}).  (We ignore the constant determinant factor which we
obtain from the $A$ integral.)  Thus the quantum physics is the same as well.


 \Head{4.3}{The Large~$N$ Effective Theory}
The argument here is almost identical to that for the large~$N$ $\sigma
$-model into a sphere (see lecture~II@-3).  So we will be brief. 
 
First, the $\phih$~integral is Gaussian, so the partition function is 
  $$ \aligned
      Z &= \sum\limits_{E}\int_{}\frac{DA}{\voll}D\sigma
     D\phih\,e^{-S[\phih,A,\sigma ]} \\
      &= \sum\limits_{E}\int_{}\frac{DA}{\voll}D\sigma \exp -\left[ \Tr \ln
     \left( \frac{d_A^*d_A}{g^2}-i\sigma \right) + i\int_{\Sigma }\Tr\sigma +
     \frac{\theta }{2\pi }\int_{\Sigma }\Tr F_A \right]. \endaligned
     \tag{13} $$
Here $\sum\limits_{E}$~is the sum over bundles~$E$.  We take~$\Sigma =\RR^2$.
Then the sum over~$E$ is irrelevant, as is the topological term.  In a first
approach to this problem, the topology of the bundle will not be important,
since it is spread over an infinite volume.  Once we get a basic
understanding of what the quantum vacuum looks like, it will not be hard to
go back and see how the topology enters.  We evaluate the leading behavior in
large~$N$ by evaluating at a stationary point of the exponential in the
integrand, i.e., a classical solution of the effective action.
 
The only Poincar\'e invariant (that is, Euclidean invariant) possibility for
the gauge field is~$A=0$.  In that case we rewrite minus the integrand in the
last line of~\thetag{13} as
  $$ \Seff[A,\sigma ] = N\Tr\ln\left( \frac{d^*d}{g^2} - i\sigma  \right) +
     i\int_{}\;\Tr\sigma ,   $$
where the operator in the first term acts on sections of~$E^*$.  Now the only
Poincar\'e invariant possibility is $\sigma $~constant, and so we
diagonalize~$\sigma $ and pass to $k$~one dimensional problems for an
eigenvalue.  At this point we rescale 
  $$ \aligned
      \sigma &=N\sigtil \\ 
      g^2 &= \tilde{g}^2/N,\endaligned \tag{14} $$
and so for each eigenvalue we have exactly the problem we had for the
large~$N$ $\sigma $-model in to a sphere.  Thus the solution~$\sigtil _0$ has
all eigenvalues equal and is specified by
  $$ -i\tilde{g}^2\sigtil_0 = \Lambda ^2e^{-4\pi /\tilde{g}^2} = M^2 
     $$
for $\Lambda $~an ultraviolet cutoff and now $\tilde{g}^2$~the running
coupling constant.  We define~$M^2$ to be this dynamically generated mass
squared. 
 
So in the large~$N$ effective action $\sigtil $~acquires a mass.  Note that
the $A$~field has no transverse degrees of freedom---it is a gauge field in
two dimensions---so does not enlarge the spectrum of the model (though, of
course, it affects the Hamiltonian as we will see, and in fact diminishes the
spectrum).  Thus the large~$N$ limit
has a mass gap.  This is the answer to the first in our list of questions.
 
To answer the other questions we need to compute something more precise,
namely the leading approximation to the large~$N$ effective action.  This
means that we do perturbation theory for the action~\thetag{12} about the
point $\phih_0=0$, $A_0=0$, $\sigtil_0= iM^2/\tilde{g}^2$.  So
shifting~$\sigtil$ by~$\sigtil_0$ and rescaling~$\phih$ we have
from~\thetag{12}
  $$ S'[\phih,A,\sigtil] = \int_{\Sigma} |d_A\phih|^2 \;+\; M^2\!\int_{\Sigma}
     |\phih|^2 \;-\;i\tilde{g}^2\!\int_{\Sigma } \Tr\sigtil\phih^*\phih \;+\;
     \frac{\theta }{2\pi }\int_{\Sigma }\Tr F_A. \tag{15} $$
Here we omit a constant term (which only shifts the partition function by a
constant) and a linear term in~$\sigtil$ (since we are expanding around a
solution of the effective action).  

 \midinsert
 \bigskip
 \centerline{
  \epsfxsize= 200pt
  \epsfysize= 66pt
 \epsffile{Figure1.eps}}
 \nobreak
 \botcaption{Figure~1: The inverse $\sigtil$ propagator}  
 \endcaption
 \bigskip
 \endinsert

The effective action is computed in perturbation theory using one particle
irreducible Feynman diagrams with external lines for~$A$ and~$\sigma $ and
with internal $\phih$~lines.  So the inverse propagator for~$\sigtil$ is
computed from the diagram shown in Figure~1, where the solid line
represents~$\phih$ and the dotted line represents~$\sigma $.  (Note that
\thetag{15}~has no quadratic term in~$\sigtil$, else Figure~1 would be a
correction to such a term.)  We evaluate the diagram in momentum space as
  $$ \aligned
      -\gtil^4Nk\int_{}\frac{d^2q}{(2\pi)^2}\,
     \frac{1}{q^2+M^2}\,\frac{1}{(p-q)^2 + M^2}
      &= -\frac{\gtil^4Nk}{4\pi ^2}\int_{0}^1 \!\!d\alpha
     \int_{}d^2q\,\frac{1}{[q^2 + (M^2 + \alpha (1-\alpha )p^2)]^2} \\
      &= -\frac{\gtil^4Nk}{\pi }\int_{0}^1\frac{d\alpha }{M^2 + \alpha
     (1-\alpha )p^2} \\
      &= - \frac{\gtil^4Nk}{\pi M^2} + O(p^2)\qquad
     \text{as~$p\to0$}.\endaligned \tag{16} $$
This corresponds to a term
  $$ C|\sigtil|^2 $$ 
in the effective action, with~$C>0$.  (The minus sign comes since in the
Euclidean theory diagrams compute negative contributions to the effective
action.)  This is the dominant term in the infrared, which means that
$\sigtil$~is massive and does not affect the long range behavior of the
theory.

 \midinsert
 \bigskip
 \centerline{
  \epsfxsize=300pt
  \epsfysize=80pt
 \epsffile{Figure2.eps}}
 \nobreak
 \botcaption{Figure~2: The inverse $A$~propagator}  
 \endcaption
 \bigskip
 \endinsert


 \midinsert
 \bigskip
 \centerline{
  \epsfxsize=250pt
  \epsfysize=130pt
 \epsffile{Figure3.eps}}
 \nobreak
 \botcaption{Figure~3: The interaction vertices for the $A$~field}  
 \endcaption
 \bigskip
 \endinsert


The inverse propagator for~$A$ is computed by the diagrams in Figure~2, which
come from the first term in~\thetag{15}.  Here the wavy line represents~$A$.
To compute these diagrams we need the Feynman rules for the vertices
indicated in Figure~3, which correspond to the terms  
  $$ \gathered
      2\Re(d\phih,\phih A) \\ 
      |\phih A|^2\endgathered  $$
in the action~\thetag{15}.  The indices refer to a standard orthonormal basis
for~$\RR^2$.  The Feynman rule (in momentum space) for the second vertex is
easy:
  $$ II=-\delta _{\mu \nu }.  $$
(There is a minus sign since the Euclidean functional integral
involves~$e^{-S}$.)  For the vertex~$I$ we must remember that $\phih$~is
complex and that $A$~is skew-Hermitian: 
  $$ 2\Re(\partial _\mu \phih,\phih A_\mu ) = -\partial _\mu \phih A_{\mu
     }\phih^* + \phih A_\mu \partial _\mu \phih^*. $$ 
Thus the vertex is 
  $$ I = -i(k_1 - k_2)_\mu .  $$
Note that one of the solid lines in the vertex represents~$\phih$ and the
other solid line represents~$\phih^*$.  So the sum of the diagrams in
Figure~2 is 
  $$ \int_{}\frac{d^2q}{(2\pi)^2 } \frac{(p-2q)_\mu (p-2q)_{\nu }}{(q^2 +
     M^2)((p-q)^2+M^2)} \;-\;2\delta _{\mu \nu } \int_{}\frac{d^2q}{(2\pi)^2
     } \frac{1}{(q^2 + M^2)}. \tag{17} $$
The factor of~2 in the second term is from the two ways of attaching the
$A$~lines to the external vertices; the corresponding factor of~2 in the
first term is canceled by the symmetry which exchanges the two internal
vertices in the first diagram.  (That is, there is a factor of~$1/2$ from the
expansion of the exponential, since we have two triple vertices.)  Each term
in~\thetag{17} is divergent, but the divergences cancel in the difference,
and after some computation similar to~\thetag{16} the answer to leading order
in~$p$ is
  $$ \frac{N}{12\pi M^2}(p_{\mu} p_\nu - p^2\delta _{\mu \nu }). \tag{18} $$
This corresponds to a term 
  $$ \frac{N}{24\pi M^2} |F_A|^2 \tag{19} $$
in the effective action.  (Again we must recall that diagrams contribute
negatively to the effective action.)  In fact, \thetag{18}~and
\thetag{19}~correspond precisely if~$k=1$ (the $\sigma $-model into
projective space), since then $A$~is an abelian connection.  In the
nonabelian case there are cubic and quartic terms, but by gauge invariance
their leading contribution must be as in~\thetag{19}.  We remark that
\thetag{19}~is the lowest order gauge invariant term we can write, and so its
appearance can be predicted from gauge invariance alone, but of course we
must do a computation to determine the coefficient.
 
So, finally, the long distance behavior of the two dimensional $\sigma
$-model into the Grassmannian~$\Grk$, in the large~$N$ limit, is equivalent
to the long distance behavior of a two dimensional gauge theory with gauge
group~$U(k)$ and charged massive scalars.  The action is 
  $$ \frac{N}{2e^2}\int_{\Sigma }|F_A|^2 + \frac{\theta }{2\pi }\int_{\Sigma
     }\Tr F_A + \int_{\Sigma }|d_A\phi |^2 + M^2\int_{\Sigma }|\phi
     |^2. \tag{20} $$
The first term in~\thetag{20} was just computed, where the coupling~$e^2$
summarizes the numerical factor in~\thetag{19}.  The second term is the
$\theta $~term in~\thetag{15}.  The last two terms are the first two terms
in~\thetag{15}, except we now drop the `$\hat{\ }$' for convenience.  So
$\phi $~is a section of~$(E^*)^{\oplus N}$.


 \Head{4.4}{Real Grassmannians}
From the beginning we can replace the complex Grassmannian with the real
Grassmannian.  In that case the bundle~$E$ is real, $\phih\:E\to\Sigma \times
\RR^N$, and there is no other change except in the representation of the
$\theta $~term.  For simplicity we consider the $\sigma $-model into the
Grassmannian~$Gr_\RR^0(k,N)$ of {\it oriented\/} $k$-planes in~$\RR^N$.  (It
is a double cover of the Grassmannian of unoriented $k$-planes.)  Now
$H^2\bigl(Gr_\RR^0(k,N);\ZZ \bigr)\cong \ZZ/2\ZZ$ and the $\theta $~term is
meant to detect this class.  Thus the second term of~\thetag{1} is replaced
by
  $$ i\theta \deg_2\alpha ,  $$
where $\deg_2$~is the mod~2 degree (0 or~1) and $\theta =0$ or~$\theta =\pi
$.  In the reformulation of the problem $E$~is an oriented real $k$-plane
bundle over~$\Sigma$, and the topological term in~\thetag{12} and subsequent
formulas is
  $$ i\theta w_2(E)[\Sigma ], \tag{21} $$
where $w_2$~is the second Stiefel-Whitney class.  For $\Sigma =\RR^2$~we can
rewrite~\thetag{21} in terms of a {\it Wilson line\/} operator.  Namely, a
field configuration with finite action~\thetag{12} has $A$~essentially flat
at infinity.  Since $\RR^2$~is contractible we can lift the
$SO(k)$~connection A to a $\Spin(k)$ connection~$\tilde{A}$, and the holonomy
on a large loop~$C\subset \RR^2$ is approximately~$\pm1$ depending on the
Stiefel-Whitney class of the induced bundle on~$S^2$.  (We let `$-1$'~denote
the nontrivial element of~$\Spin(k)$ covering $1\in SO(k)$.)  Choose a
representation~$R$ of~$\Spin(k)$ and consider
  $$ W_R(C) = \frac{\Tr_R\hol_C(A)}{\dim R}. \tag{22} $$
In the limit where the loop~$C$ becomes large, this computes the exponential
of~\thetag{21}, where $\theta =0$ if $R$~is a representation of~$SO(k)$ and
$\theta =\pi $ if $R$~is a representation of ~$\Spin(k)$ but not of~$SO(k)$.
 
The generalization to an arbitrary connected compact Lie group~$G$ is clear.
A $G$~bundle over~$\Sigma $ has a characteristic class in~$H^2(\Sigma ,\pi
_1G)$.  It pairs with a homomorphism $e^{i\theta }\:\pi _1G\to\TT$ to give a
term in the exponentiated action.  Here $\TT=U(1)$~is the circle group.  On
the other hand a representation of the simply connected covering group
induces a homomorphism $e^{i\theta }\:\pi _1G\to\TT$, so we can use the
Wilson operator~\thetag{22} to write the $\theta $~term on~$\RR^2$. 

 \Head{4.5}{Pure Gauge Theory}
 We still must determine the quantum behavior of the theory with effective
action~\thetag{20}.  For this, we will first practice by analyzing the pure
gauge theory in two dimensions, whose action is the sum of the first two
terms of~\thetag{20}.  We may as well consider an arbitrary connected compact
Lie group~$G$.  The norm in the first term of~\thetag{20} is defined via a
bi-invariant metric on~$G$.  We quantize the theory on the circle~$\cir_V$ of
circumference (=volume)~$V$.  In general, to quantize a gauge theory in
$n$~dimensions on an $n$-manifold~$Y$, we consider connections on~$\RR\times
Y$ in {\it temporal gauge\/} and take solutions to the equations of motion up
to time-independent gauge transformations.  (See Kazhdan's lectures on the
quantization of gauge theories.)  For pure gauge theory without the $\theta
$~term the resulting space is the (co)tangent bundle of the space of
connections on~$Y$ modulo gauge transformations.  For $Y=\cir_V$ we first fix
a basepoint; then a connection up to gauge equivalence is specified by the
holonomy, an element of~$G$.  A change of basepoint conjugates the holonomy,
and so the quantum Hilbert space is
  $$ \Hgauge(\theta \!=\!0) \;=\; L^2(G)^G,  $$
the space of conjugacy invariant functions on~$G$.  A basis for this space is
the set of characters of irreducible representations. 
 
Next we compute the Hamiltonian.  Let $A_t=A_t(x)dx$ be a connection
on~$\RR\times \cir_V$ (with coordinates~$t,x$) in temporal gauge, relative to
some trivialization, and let $g_t\in G$ be the corresponding path of
holonomies.  In a gauge where $A_t$~is constant, we have $g_t=e^{VA_t}$.
Then the first term in the action~\thetag{20} is
  $$ \frac{N}{2e^2}\int_{}\int_{}\;\left| \frac{dA}{dt} \right| \,dtdx =
     \frac{N}{2e^2V}\int_{}\;|\dot{g}|^2\,dt. \tag{23} $$
This is the lagrangian for a classical particle of mass~$N/e^2V$ moving on
the Riemannian manifold~$G$; the corresponding quantum Hamiltonian is
  $$ H_{\text{gauge}} = \frac{e^2V}{2N}\Delta_G, \tag{24} $$
where $\Delta_G$~is the laplacian on~$G$.  The eigenfunctions are the
characters of the irreducible representations with eigenvalues proportional
to the Casimir.   
 
Now consider the $\theta $~term in~\thetag{20}.  For $G=U(k)$ we have a
natural closed imaginary 1-form~$\alpha \in i\Omega ^1_{G}$ which is the
trace of the Maurer-Cartan form.  Then in terms of the path~$g_t$ of
holonomies the $\theta $~term in the action is 
  $$ \frac{\theta }{2\pi }\int_{}\;g^*\alpha . \tag{25} $$
This is a topological term---it is invariant under reparametrizations of the
path~$g_t$.  More geometrically, $\theta \alpha $~is a flat connection on a
topologically trivial hermitian line bundle~$L_\theta $ over~$G$, and up to
equivalence it is given by an element in~$H^1(G;\TT)$.  Then \thetag{25}~is
parallel transport in this flat bundle.\footnote{In general, the classical
field theory action on a manifold with boundary---here the interval---is not
a number, but we do not stress that point here.}  More generally, we can
``twist'' our mechanical system by any hermitian line bundle~$L$ with
connection.  Physically this describes a particle moving in an
electromagnetic field.  The quantum Hilbert space is the the space of
sections of~$L$ with Hamiltonian the laplacian for such sections.  In our
case we obtain the space
  $$ \Hgauge(\theta ) = L^2(G,L_\theta )^G  $$
of invariant sections with Hamiltonian~\thetag{24}.  For arbitrary~$G$ an
element $e^{i\theta} \in H^1(G;\TT)$ corresponds to a homomorphism
$e^{i\theta }\:\pi _1G\to\TT.$  Recall that $\pi _1G$~is a subgroup of the
center of~$G$.  Then the eigenfunctions of the laplacian acting
on~$\Hgauge(\theta )$ are the characters of representations of the simply
connected cover of~$G$ whose restriction to~$\pi _1G$ is~$e^{i\theta }$; the
eigenvalue is again proportional to the Casimir.
 
For the unitary group~$G=U(k)$ we identify~$\theta \in \RR/2\pi \ZZ$ as
before.  The smallest Casimir occurs for a
representation~$\det^{\tilde{\theta}/2\pi }$, where $\tilde{\theta }\in \RR$
is a representative of~$\theta $ of smallest absolute value.  If $\theta
\not= \pi $~there is a unique such~$\tilde{\theta }$, but for~$\theta =\pi $
there are two possibilities: $\tilde{\theta }=\pi $ and~$\tilde{\theta }=-\pi
$.  For $G=SO(k)$ the simply connected cover is~$\tilde{G}=\Spin(k)$.
For~$\theta =0$ there is a unique lowest representation, the trivial
representation of~$SO(k)$.  For~$\theta =\pi$ the lowest representation is
the spin representation if $k$~is odd, and the two half spin representations
if $k$~is even.  In the large volume limit only the lowest eigenvalue
survives, so we have two vacuua for~$\theta =\pi $ in the complex~$\bigl(U(k)
\bigr)$ case and for $k$~even in the real $\bigl(SO(k) \bigr)$~case.  Observe
that the two vacuua correspond under the involution $g\mapsto -1\cdot g$ in
these cases.  (Notice also that the bundle~$L_{\theta =0}$ is real.)  We will
see that this vacuum structure persists when we add matter.  Thus parity
symmetry~\thetag{4} is spontaneously broken at~$\theta =\pi $ for $G=U(k)$
and~$G=SO(2\ell )$.  This answers the third of our questions.
 
As the volume $V\to\infty $ the eigenvalues of~\thetag{24} become widely
separated.  In particular, in the infinite volume limit there is no state of
finite energy above the vacuum (or vacuua).  So the physical Hilbert space in
infinite volume consists only of the vacuum (or vacuua)---pure gauge theory
on~$\RR^2$ is trivial.

Specialize to~$G=U(1)$.  Then the {\it vacuum energy density\/}, which is the
minimum eigenvalue of the Hamiltonian~\thetag{24} divided by the volume~$V$,
is 
  $$ E_{\text{vac}}(\theta ) = \frac{e^2}{2N}\min\limits_{n\in
     \ZZ}(n-\frac{\theta }{2\pi })^2. \tag{26} $$
Notice that the derivative of~$E_{\text{vac}}$ has a discontinuity at~$\theta
=\pi $.   
 
For $G=U(k)$ formula~\thetag{26} is simply multiplied by~$k$. 
 
The partition function~$Z_{\Sigma }(\theta )$ of the pure gauge theory
for~$\Sigma =[0,T]\times S^1_V$ has the Hilbert space interpretation
  $$ Z_\Sigma(\theta ) \sim \langle \Omega |e^{-TH(\theta )}|\Omega
     \rangle\quad \text{as $T\to\infty$},  $$
where $H(\theta )$~is the Hamiltonian and $\Omega $~the vacuum.  For the pure
gauge theory, we obtain from~\thetag{26} for $G=U(1)$
  $$ Z_\Sigma(\theta ) \sim \exp\bigl(cTV - \frac{e^2TV}{2N}\min\limits_{n\in
     \ZZ}(n-\frac{\theta }{2\pi })^2 \bigr)\quad \text{as
     $T\to\infty$}. \tag{27} $$
Here $c$~is a constant which represents the indeterminacy of the path
integral, or equivalently the fact that we are free to add a constant
(independent of~$\theta $) to the Hamiltonian.  This is quite different than
the prediction~\thetag{3} from the instanton sum.  Note that due to the
classical conformal invariance of the $\sigma $-model, the instantons used in
deriving~\thetag{3} do not have a definite size; they can be rescaled.  For
that reason the instanton sum is not reliable at large distances.  In any
case \thetag{27}~answers question~2 for pure gauge theory.


 \Head{4.6}{Classical Electromagnetism in Two Dimensions}
Before analyzing the quantum gauge theory with bosonic matter---the $\phi
$~field in the action~\thetag{20}---we discuss the classical physics.  For
the classical analysis we work in the $G=U(1)$~theory.  The classical
equations---Maxwell's equations---for pure gauge theory are
  $$ \frac{N}{e^2}d^*_AF_A=0. \tag{28} $$
Since we are in two spacetime dimensions, this implies that the electric
field~$f=f_A=*F_A$ is a constant.  Let `$x$'~denote the coordinate on space,
which is simply a copy of~$\RR$.  Add a point charge of charge~1 at~$x=x_0$.
Then the electric field~$f$ as a function of~$x$ satisfies~\thetag{28} with a
right hand side due to the charge:
  $$ \frac{N}{e^2}\frac{df}{dx}=-\delta (x-x_0).  $$
Thus the value of the electric field jumps by~$-e^2/N$ across a charge.
(See Figure~4.)  Allowing for many charges, and assuming all of them are
multiples of the basic charge (which is a conclusion of the quantum theory),
we see that
  $$ \theta =\frac{2\pi Nf}{e^2}\in \RR/2\pi \ZZ \tag{29} $$
is a constant.  So there is an angle in the classical theory (assuming charge
quantization).

 \midinsert
 \bigskip
  $$ \hfil f \ \ \uparrow\ \  f - \frac{e^2}{N} \qquad\qquad \qquad   
   f \ \ \downarrow\ \  f + \frac{e^2}{N}\hfil 
  $$
 \nobreak
 \botcaption{Figure~4: Change in electric field across a positive or negative
charge}   
 \endcaption
 \bigskip
 \endinsert


The classical energy density of an electric field~$f$ is computed from the
action~\thetag{23} (where $f=dA/dt$) to be~$\frac{N}{2e^2}f^2$.  As in the
quantum theory, for a fixed value of~$\theta $ in~\thetag{29} there is a
unique minimum obtained at some~$f=f_0$ as long as $\theta \not=\pi $.  Any
valid configuration must have finite energy compared to the vacuum energy.
Thus if~$\theta \not= \pi $ we must have $f(x)\to f_0$ as $x\to\pm\infty $.
Using the formula above for the jump of the electric field across a charge,
we see that the total charge of a finite energy configuration is zero.  This
means that there is {\it confinement\/}---it is impossible to have a single
charged particle or any other isolated collection of charges with nonzero
total charge.  On the other hand, for~$\theta =\pi $ we have minimum energy
density at~$f_0=\pm e^2/2N$, and so for any finite energy configuration
$f(x)\to\pm e^2/2N$ as~$x\to\pm\infty $.  Thus there is a finite energy
configuration with a single particle: the electric field satisfies~$f(-\infty
)=e^2/2N$, $f(\infty )=-e^2/2N$.  (See Figure~5.)  In this case there is no
confinement.  Also, in this case there are four components of finite energy
configurations, depending on the value of~$f$ at~$\pm\infty $.

 \midinsert
 \bigskip
  $$   \frac{e^2}{2N}\ \ \uparrow \ \ \frac{-e^2}{2N}$$

 \nobreak
 \botcaption{Figure~5:  A one particle state for~$\theta =\pi $}  
 \endcaption
 \bigskip
 \endinsert


Notice that whereas in three space dimensions the Coulomb potential between
point charges separated by distance~$r$ is proportional to~$1/r$, the Coulomb
potential in one space dimension is proportional to~$r$.  This means the
potential energy grows as the charges separate, which is another way of
understanding that confinement occurs. 


 \Head{4.7}{Quantum theory with matter}
Now we want to show that there is confinement in the quantum theory of the
lagrangian~\thetag{20} as long as~$\theta \not= \pi $.  This is the assertion
that every finite energy configuration in the quantum theory has total charge
zero. 
 
As a preliminary, consider the theory of matter only (no gauge field).  In
the simplest case~$k=N=1$ there is a single free complex scalar field~$\phi $
with mass~$M$.  The Hilbert space~$\Hf$ of this theory is the completed
symmetric algebra of~$W\oplus \Wbar$, where $W$~is the scalar representation
of Poincar\'e with mass~$M$.  There is a global $U(1)$~symmetry which
rotates~$\phi $.  The corresponding quantum operator~$\Qf$, the Noether
charge, has value~1 on~$W$ and~$-1$ on~$\Wbar$, so value~$p-q$
on~$\Sym^pW\otimes \Sym^q\Wbar$.  A state in this subspace represents
$p$~positively charged particles and $q$~negatively charged particles.  For
arbitrary~$k,N$ we have $kN$~copies of this picture.  In particular,
for~$k=1$ there is a global $SU(N)$~symmetry (beyond the $U(1)$~symmetry
already discussed.)
 
Now add the gauge field.  For~$k=N=1$ we have a $U(1)$~gauge theory with a
single charged scalar field.  The global $U(1)$~symmetry of the preceding
paragraph is now a local symmetry.  Consider first the case~$\theta =0$.  For
small coupling~$e^2$ we construct the quantum Hilbert space of the joint
system by quantizing the symplectic manifold of classical solutions
to~\thetag{20}.  Thus take~$\Sigma $ to be the cylinder~$\RR\times S^1_V$.
Then the space of classical solutions is a vector bundle over the cotangent
bundle~$T^*\scrA$ to the space~$\scrA$ of connections on~$S^1_V$; the fiber
of this vector bundle at the trivial connection~$A=0$ is the real symplectic
vector space underlying~$W\otimes \Wbar$.  (Note we quantize by a complex
polarization, which leads to the Hilbert space~$\Hf$ described above.)  To
implement gauge symmetry we must take the symplectic quotient by the
group~$\scrG$ of gauge transformations.  Fix a basepoint (infinity)
on~$S^1_V$.  Then the subgroup~$\scrG_0$ of gauge transformations which equal
the identity at the basepoint acts freely on~$\scrA$, and the quotient is
identified with~$U(1)$ via holonomy.  In the pure gauge theory the symplectic
quotient of~$T^*\scrA$ by~$\scrG_0$ is diffeomorphic to~$T^*U(1)$; its
quantization~$L^2\bigl(U(1) \bigr)$ was discussed previously.  As the
length~$V\to\infty $ recall that the only state which remains of finite
energy is the vacuum state.  Classically, the vacuum corresponds to the zero
section, a lagrangian submanifold of~$T^*U(1)$.  The subgroup~$U(1)\subset
\scrG$ of constant gauge transformations acts trivially on~$\scrA$---hence
trivially on~$T^*\scrA$---so does not enter into pure gauge theory.  In the
theory with matter the symplectic quotient by~$\scrG_0$ is a vector bundle
over~$T^*U(1)$ with fiber~$W\oplus \Wbar$ at the identity.  Now the constant
gauge transformations act nontrivially in the fibers by scalar
multiplication.  To implement the symmetry we have two choices: we can
quantize the symplectic quotient or we can consider the subspace of Hilbert
space annihilated by the corresponding quantum operator~$\Ne$.  Pursuing
first the latter, we see that in the infinite volume limit the quantum
Hilbert space before implementing the~$U(1)$ is simply~$\Hf$, since the gauge
field contributes only the vacuum state.  The operator~$N_e$ is simply equal
to~$\Qf$.  Thus the Hilbert space of the theory is the subspace of states
with total charge~$\Qf=0$.  Therefore, just as in the classical theory we
have confinement.  There is a mass gap, and the smallest mass is~$2M$.  In
the theory with small coupling, the qualitative picture is the same.
 
If instead we take the symplectic quotient of~$W\oplus \Wbar$ by~$U(1)$, we
are led to a singular space.  The moment map is $\mu (w,\wbar') = |w|^2 -
|\wbar'|^2$, and $\mu \inv (0)$~is singular at~$(0,0)$.  In the quantization
this singular point corresponds to the vacuum, and it is not hard to see
heuristically that we are led to the same description as before. 
 
Now allow~$\theta \not= 0$.  In the language of geometric quantization the
prequantum line bundle over~$T^*U(1)$ is now twisted by the pullback
of~$L_\theta \to U(1)$.  Quantum states correspond to ``Bohr-Sommerfeld''
leaves of the given polarization, which are equally spaced parallel circles
in the cylinder~$T^*U(1)$.  More precisely, if $p$~is the (momentum)
coordinate in the cotangent space, the circles occur at $p=n-\theta /2\pi $
for integral~$n$.  The energy of the corresponding quantum state (in pure
gauge theory) is given in~\thetag{26}, and as $V\to\infty $ we only keep the
closest circle(s) to~$p=0$, which corresponds to the vacuum state.
If~$\theta \not= \pi $ there is a unique such circle, and we have the same
picture of confinement as above.  For~$\theta =\pi $ there are two such
circles, corresponding to the two vacuum states.  Thus in the theory with
matter, before dividing out by the constant gauge transformations, we must
quantize two disjoint copies of~$W\oplus \Wbar$ to obtain~$\Hf\oplus \Hf$.
If, as before, we assign zero charge to each of the vacuum states, then the
operator~$N_e$ which corresponds to the $U(1)$~symmetry is~$\Qf\oplus \Qf$.
The subspace of~$\Hf\oplus \Hf$ annihilated by~$\Qf\oplus \Qf$ is simply two
copies of the system seen previously, each copy with a vacuum.  These are two
``realizations'' of the quantum theory, each with a mass gap of size~$2M$.
It turns out---and this is hard to explain from this viewpoint---that we can
also assign {\it different\/} charges to the two vacuua.  In that case the
operator~$N_e$ is $(\Qf-1)\oplus \Qf$ or $\Qf\oplus (\Qf+1)$.  The kernel in
each case has one particle states of mass~$M$, and there is no confinement.
There is no vacuum state in either realization.  The four realizations
correspond to the classical picture of the previous section.

To justify these heuristic pictures we compute.  From~\thetag{20} we see that
the Hamiltonian is 
  $$ H = \int_{-\infty }^{\infty}dx\;\bigl(\frac 12|\pi _\phi |^2 + |d_A\phi
     |^2 + M^2|\phi |^2 + \frac{N}{2e^2}|f_A|^2 \bigr), \tag{30} $$
where $\phi $~is a field on~$\RR$.  Here $\pi _\phi $~is the conjugate
momentum to~$\phi $, and $f=f_A$ is the Hodge star of the curvature as
before.  We derive an effective Hamiltonian for~$\phi $ by plugging in the
equation of motion of~$A$.  The latter is obtained by varying the
lagrangian~\thetag{20} with respect to~$A$: 
  $$ \frac{N}{e^2}\frac{df}{dx} = j, \tag{31} $$
where the current is 
  $$ j = \phi \pf - \overline{\phi\pf}.  $$
We integrate~\thetag{31} to obtain 
  $$ f(x) = \frac{e^2}{N} \left\{ \int_{-\infty }^{\infty}dy\;G(x-y)j(y) + c
     \right\}  \tag{32} $$
for some constant~$c$, where 
  $$ G(x-y) = \cases 1 ,&x\ge y;\\0,&x<y.\endcases  $$
Note that 
  $$ \lim\limits_{x\to-\infty } f(x) = \frac{e^2c}{N}. \tag{33} $$
Plugging into~\thetag{30} we see 
  $$ H = H_0 + \Delta H,  $$
where $H_0$~is the Hamiltonian for the free scalar particle (if we compute
at~$A=0$), and 
  $$ \Delta H = \frac{e^2}{2N}\int_{-\infty }^{\infty}dx\;\left\{
     \int_{-\infty }^{\infty}dy\;G(x-y)j(y) + c \right\}^2.   $$
The perturbation term~$\Delta H$ is nonlocal and singular. 
 
Now we must determine the subspace of states~$\Psi $ on which $\Delta H$~is
finite.  Consider first~$c=0$.  Then $\langle \Psi |\Delta H | \Psi
\rangle$ is finite if and only if 
  $$ \lim\limits_{x\to\infty }\langle \Psi |f(x)|\Psi   \rangle=0, 
     $$
where $f(x)$ ~is defined by~\thetag{32}.  Now 
  $$ \lim\limits_{x\to\infty }\int_{-\infty }^{\infty}dy\;G(x-y)j(y) =
     \int_{-\infty } ^{\infty}dy\; j(y) = \Qf  $$
is the charge operator in the Hilbert space ~$\Hf $.  So if~$c=0$ we have
confinement: 
  $$ \langle \Psi |\Qf|\Psi   \rangle = 0. $$ 
In general, $c$~is related to~$\theta $ through~\thetag{33} and~\thetag{29}:
  $$ |c| = \min\limits_{n} \left( \frac{\theta }{2\pi }+n \right).  
     $$
This is the value of the electric field at~$-\infty $.  Also, we subtract an
(infinite) constant from~$\Delta H$ to account for the nonzero energy
at~$\infty $:  
  $$ (\Delta H)_{\text{normalized}} = \frac{e^2}{2N}\int_{-\infty
     }^{\infty}dx\;\Biggl[\left\{ \int_{-\infty }^{\infty}dy\;G(x-y)j(y) + c
     \right\}^2 - c^2\Biggr].  $$
This is finite on states~$\Psi$ which satisfy
  $$ \Qf(\Qf+2c)|\Psi\rangle=0, \tag{34} $$
i.e., $\Qf|\Psi \rangle=0$ or~$\Qf|\Psi\rangle=-2c$.  For $\theta \not= \pi $
the only possibility is~$\Qf|\Psi\rangle=0$ and we have confinement.
For~$\theta =\pi $ we have either~$c = \pm1/2$, and \thetag{34}~is satisfied
by states with~$\Qf|\Psi\rangle=0$ or~$\Qf|\Psi\rangle=\mp1$.  As in the
classical theory, this gives four sectors.  We denote them $\scrH_{++},
\scrH_{+-}, \scrH_{-+}, \scrH_{--}$ according to the value of the electric
field at~$-\infty $ and~$+\infty $.  There is a vacuum state and confinement
in~$\scrH_{++}, \scrH_{--}$; there is neither a vacuum nor confinement
in~$\scrH_{+-}, \scrH_{-+}$.
 
In the confining cases the symmetry group~$PSU(N)$ acts.  In the sectors of
the $\theta =\pi $~theory with no confinement the covering group
$SU(N)$~acts.   
 
The story for $k>1$ is similar.

Finally, we make a remark about the electric charge.  In the quantum picture
it is an operator~$Q_e$, and from Noether's theorem applied to~\thetag{20} we
compute the relation to~$\Ne$: 
  $$ Q_e = \frac{e^2}{N}(\Ne+\frac{\theta }{2\pi }).  $$
The eigenvalues of~$\Ne$ are integral, but those of~$Q_e$ are shifted
from~$\frac{e^2}{N}\ZZ$ if~$\theta \not= 0$.  Note the flow (monodromy) in
the eigenvalues of~$Q_e$ as $\theta $~runs around the circle from~$\theta =0$
to~$\theta =2\pi $.  We will encounter this phenomenon again in four
dimensional gauge theory.


\enddocument
