\documentclass[lecture]{qft-l}
\usepackage[dvips]{epsfig}
\usepackage{amssymb}
 
\catcode`\@=11
\renewcommand{\over}{\@@over}
\renewcommand{\atop}{\@@atop}
\renewcommand{\above}{\@@above}
\renewcommand{\overwithdelims}{\@@overwithdelims}
\renewcommand{\atopwithdelims}{\@@atopwithdelims}
\renewcommand{\abovewithdelims}{\@@abovewithdelims}
\catcode`\@=13

%%%%    Greek Letters
	\newcommand{\al}{\alpha}
	\newcommand{\del}{\delta}
	\newcommand{\eps}{\epsilon}
	\newcommand{\lam}{\lambda}
	\newcommand{\tht}{\theta}
	\newcommand{\om}{\omega}
	\newcommand{\Lam}{\varLambda}
	\newcommand{\PHI}{\varPhi}

%%%%    Fonts
        \font\tt=cmtt10
	\newcommand{\bm}[1]{{\mbox{\boldmath ${#1}$}}}
	\font\bbb=msbm7 %at 11 pt
	\font\BBB=msbm10 %at 11 pt
	\newcommand{\ZZ}{{\mbox{\BBB{Z}}}}
	\newcommand{\zz}{{\mbox{\bbb{Z}}}}
	\newcommand{\re}{{\mbox{\bbb{R}}}}
	\newcommand{\RE}{{\mbox{\BBB{R}}}}
	\newcommand{\co}{{\mbox{\bbb{C}}}}
	\newcommand{\CO}{{\mbox{\BBB{C}}}}
	\newcommand{\HA}{{\mbox{\BBB{H}}}}
	\font\frak=eufm10 at 11 pt
        \newcommand{\g}{\mbox{\frak{g}}}
        \newcommand{\gu}{\mbox{\frak{u}}}
        \newcommand{\gsp}{\mbox{\frak{sp}}}
        \newcommand{\gsu}{\mbox{\frak{su}}}
        \newcommand{\gso}{\mbox{\frak{so}}}
\newcommand{\medwedge}{\mbox{\fontsize{12pt}{0pt}\selectfont $\wedge$}}
	\newcommand{\Hom}{{\rm Hom}}

%%%%    Format
%	\newcounter{sect}\setcounter{sect}{0}
%	\newcounter{subsect}
%	\newcommand{\sect}[1]{\vspace{4ex}
%		\addtocounter{sect}{1}\setcounter{subsect}{0}
%		\begin{flushleft}
%		{{\large\bf \arabic{sect}. {#1}}}
%		\end{flushleft}
%		\setcounter{thm}{0}
%		\setcounter{equation}{0}
%		\def\theequation{\arabic{sect}.\arabic{equation}}}
%	\newcommand{\subsect}[1]{\vspace{.5ex}\addtocounter{subsect}{1}
%		\begin{flushleft}
%		{{\bf \arabic{sect}.\arabic{subsect} {#1}}}
%		\end{flushleft}}
%	\newtheorem{thm}{Theorem}[sect]
%	\newtheorem{prop}[thm]{Proposition}
%	\newtheorem{lemma}[thm]{Lemma}
%	\newtheorem{cor}[thm]{Corollary}
%	\newtheorem{defn}[thm]{Definition}
%	\newtheorem{rmk}[thm]{Remark}
%	\newtheorem{ex}[thm]{Example}
%\newcommand{\proof}[1]{\noindent{\em Proof.}$\quad${#1}$\hfill\Box$
				%\vspace{2ex}}
	\newcommand{\be}{\begin{equation}}
	\newcommand{\ee}{\end{equation}}
	\newcommand{\bea}{\begin{eqnarray}}
	\newcommand{\eea}{\end{eqnarray}}
	\newcommand{\nno}{\nonumber \\}
	\newcommand{\sep}[1]{\!\!\!\! &{#1}& \!\!\!\! }
	\newcommand{\eq}{\sep{=}}
	\newcommand{\vc}{\sep{ }}

%%%%	Math Symbols
	\newcommand{\bra}{\langle}
	\newcommand{\ket}{\rangle}
	\newcommand{\m}{\backslash}
	\newcommand{\inv}[1]{\frac{1}{#1}}
	\newcommand{\hf}{{\textstyle \inv{2}}}
	\newcommand{\e}[1]{e^{{#1}}}
	\newcommand{\ii}{i}
	\newcommand{\dr}{d}
	\newcommand{\pdr}{\partial}
	\newcommand{\sss}{^{\mathrm{ss}}}
	\newcommand{\Det}{\mathrm{Det}}
	\newcommand{\set}[2]{\{{#1}\,|\,{#2}\}}
	\newcommand{\rank}{\mathrm{\,rank\,}}
	\newcommand{\im}{\mathrm{im}}
	\newcommand{\Pf}{\mathrm{Pf\,}}
	\newcommand{\tr}{\mathrm{tr}}
	\newcommand{\tree}{_\mathrm{tree}}
	\newcommand{\one}{{\mathbf 1}}
	\newcommand{\cc}{\mathrm{c.c.}}

%%%%	Others
        \newcommand{\AAA}{{\mathcal A}}
        \newcommand{\FF}{{\mathcal F}}
        \newcommand{\GG}{{\mathcal G}}
        \newcommand{\LL}{{\mathcal L}}
        \newcommand{\MM}{{\mathcal M}}
        \newcommand{\MC}{\MM_{{\rm c}}}
        \newcommand{\MQ}{\MM_{{\rm q}}}
        \newcommand{\MS}{\MM_{{\rm smooth}}}
        \newcommand{\UU}{{\mathcal U}}
        \newcommand{\OO}{{\mathcal O}}
	\newcommand{\hLam}{\,\hat{\!\Lam}{}}
	\newcommand{\tLam}{\,\tilde{\!\Lam}{}}
	\newcommand{\htLam}{\,\hat{\tilde{\!\Lam}}{}}
	\newcommand{\hC}{\hat{C}{}}
	\newcommand{\hF}{\hat{F}{}}
	\newcommand{\tC}{\tilde{C}{}}
	\newcommand{\tF}{\tilde{F}{}}
	\newcommand{\tM}{\tilde{M}{}}
	\newcommand{\tQ}{\tilde{Q}{}}
	\newcommand{\tN}{\tilde{N}_c}
	\newcommand{\htC}{\hat{\tilde{C}}{}}
	\newcommand{\htF}{\hat{\tilde{F}}{}}
	\newcommand{\htM}{\hat{\tilde{M}}{}}
	\newcommand{\htQ}{\hat{\tilde{Q}}{}}
	\newcommand{\hM}{\hat{M}{}}
	\newcommand{\hQ}{\hat{Q}{}}
	\newcommand{\Mperp}{\bra M^\perp\ket}
	\newcommand{\tom}{\tilde{\om}_c}
        \newcommand{\eff}{_\mathrm{eff}}
	\newcommand{\dirac}{\not\kern-3pt D}
	\newcommand{\two}[4]{\left\{    \begin{array}{ll}
					{#1}, & {\mbox{if }} {#2}, \\
					{#3}, & {\mbox{if }} {#4}
					\end{array}     \right.}
		\newcommand{\four}[4]{\left(	\begin{array}{cc}
				{#1}	&	{#2}	\\
				{#3}	&	{#4}
				\end{array}   \right)}
 
\LogoOnfalse
\def\thelectureseries{Dynamics of $N=1$ SUSY Theories}
\catcode`\@=11
\def\@author{N. Seiberg}

%%%\def\exname{SOLUTIONS TO EXERCISES:\\GENERAL REMARKS}
%%%\def\exmark{%\let\@secnumber\@empty
%%%    \@secmark\markright{}}%

%%%\def\theexercises{%
%%%  \addtocounter{lecture}1\relax
%%%  \refstepcounter{chapter}%
%%%\chapter*{}
%%%  {\Large\bfseries
%%%   \raggedleft
%%%   \@xp\uppercase\@xp{\exname} \\
%%%   \endgraf}%
%%%  \@xp\exmark\@xp{\rm\exname}%
%%%  \relax\vspace{34\p@}\noindent
%%%  \normalsize\labelsep .5em\relax
%%%  \leftmargin\labelwidth
%%%  \advance\leftmargin\labelsep
%%%	\usecounter{enumi}\sloppy
%%%  \clubpenalty9999 \widowpenalty\clubpenalty  \sfcode`\.\@m}

  \def\exmark{%\let\@secnumber\@empty
%    \@secmark\markright\sectionrunhead\sectionname}%
    \@secmark\markright\exrunhead\chaptername}%

  \def\exrunhead#1#2#3{%
    \let\@tempa\chaptername
    \uppercasenonmath{\@tempa}%
    \def\@tempb{SOLUTIONS TO EXERCISES (BY S. WU): #3\unskip}%
    \uppercasenonmath{\@tempb}%
    \textmd{\@tempb}
    }

\def\@lecture[#1]#2{%
  \addtocounter{lecture}1\relax
  \refstepcounter{chapter}%
\gdef\thelecturename{#1\unskip}\firstlecturefalse
  {\Large\bfseries
   \raggedleft
   \@xp\uppercase\@xp{\thelecturelabel} \\
   \vspace*{3pt}%
    #2\unskip
   \endgraf}%
  \let\@secnumber=\thelecturenum
  \@xp\exmark\@xp{#1}%
%  \vspace{34\p@}\noindent}
  \vspace{6\p@}\noindent}

\gdef\chaptername{SOLUTIONS TO EXERCISES (BY SIYE WU)}
\def\thelecturelabel{SOLUTIONS TO EXERCISES (BY SIYE WU)}

\def\Subhead#1{\subsection*{#1}}

\catcode`\@=13

\overfullrule=5pt

\begin{document}

\setcounter{chapter}3
\setcounter{lecture}3
\setcounter{page}{39}

\lecture{General Remarks}
 \Headnn{The classical moduli space}

\catcode`\@=11
%{\let\@makefnmark\relax  \let\@thefnmark\relax
%   \footnote{\bf Solutions by Siye Wu\\ }\addtocounter{footnote}{-1}}
\catcode`\@=13

We recall that the quark superfield $Q$ in an $N=1$ supersymmetric gauge
theory is valued in a representation $R$ of the gauge group $G$.
The bosonic potential is $V=|D|^2$, where $D\colon R\to\g^*\cong\g$
is the moment map of the Hamiltonian $G$-action on $R$.
Therefore the space $\MC$ of classical ground states (modulo gauge 
transformations), called the moduli space of classical vacua,
is the symplectic quotient $D^{-1}(0)/G$.

The space $\MC=D^{-1}(0)/G$ is also the categorical quotient of $R$ by
$G^\co$.\footnote{G.\ Kempf, L.\ Ness, in: Lecture Notes of Math.\ 732,
(1979) pp.\ 233-244; G.\ W.\ Schwarz, in: Progress in Math.\ 80, (1989)
pp.\ 135-151; P.\ Heinzner, F.\ Loose, Geom.\ Funct.\ Anal.\ 4 (1994) 288-297}
That is to say, $\MC$ is a complex space whose coordinate ring is isomorphic
to $\CO[R\,]^G$.
In fact, for any $Q\in R$, the orbit $G^\co\cdot Q$ is closed if and
only if $G^\co\cdot Q\cap D^{-1}(0)\ne\emptyset$.
If $Q\in D^{-1}(0)$, then $G^\co\cdot Q\cap D^{-1}(0)=G\cdot Q$.
Let the semi-stable part of $R$ be
$R\sss=\set{Q\in R}{G^\co\cdot Q\cap D^{-1}(0)\ne\emptyset}$.
Let $\pi\colon R\sss\to R\sss/G^\co$ be the quotient map.
We have\\
1) the the inclusion $D^{-1}(0)\to R\sss$ induces a homeomorphism
$D^{-1}(0)/G\to R\sss/G^\co$;\\
2) for any point $O\in R\sss/G^\co$, $\pi^{-1}(O)$ contains exactly
one closed $G^\co$-orbit in $R$;\\
3) for any open set $U\subset R\sss/G^\co$, we have an isomorphism
$\OO(U)\cong\OO(\pi^{-1}(U))^G$.\\
This justifies the identification of the structure sheaf $\OO_{D^{-1}(0)/G}$
with $(\OO_R)^G$.
(However the invariants in $\CO[R\,]^G$ do not distinguish the
$G^\co$-orbits in $R$.)

In what follows we will consider the case $R=\Hom(F,C)$.
Here $C$, called the space of colors, is an irreducible unitary 
representation of $G$. 
$F$, the space of flavors, is an Hermitian vector space on which $G$
acts trivially.
At the classical level, the global symmetry group is $U(F)\times U(1)_X$,
where $U(1)_X$ is the rotation on the fermionic coordinates in the $N=1$
superspace.
We write $U(F)=SU(F)\times_{Z(SU(F))}U(1)_A$, 
where $Z(SU(F))$ is the center of $SU(F)$ and
$U(1)_A$ is the scalar multiplication of $U(1)$ on $F$.

\Headnn{Axial anomaly and the global symmetry}

The classical action depends on the gauge group $G$, the matter representation
$R$, the coupling constant $g$ and the $\theta$-term.
In the quantum theory, the coupling constant $g$ depends on the 
renormalization scale $\mu$: $g(\mu)$ runs according to
$\mu\frac{\dr g}{\dr\mu}=\beta(g)$.
Here $\beta(g)$ is the $\beta$-function; at one-loop it is given by
\begin{equation}
\beta(g)=-\frac{b_0}{(4\pi)^2}g^3,
\end{equation}
where $b_0={\textstyle \frac{3}{2}}c_2(G)-\hf c_2(R)$.
The quantum theory thus has a parameter of mass dimension $1$:
the (complex) scale $\Lam$ defined by
	\begin{equation}\label{running}
\Lam^{b_0}=\mu^{b_0}\e{-\frac{8\pi^2}{g^2(\mu)}}\e{\ii\theta}
=\mu^{b_0}\e{2\pi\ii\tau(\mu)},
	\end{equation}
where $\tht$ is the theta angle and 
$\tau(\mu)=\frac{\theta}{2\pi}+\frac{4\pi\ii}{g^2(\mu)}$.

Recall that if $R=\Hom(F,C)$, then the 
classical theory has a global symmetry
$U(F)\times U(1)_X$.
Upon quantization, 
both $U(1)_A$ and $U(1)_X$ shift the theta angle $\theta$,
and hence $\Lam^{b_0}$, whereas $SU(F)$ does not.
Consequently neither of these $U(1)$ groups is a true symmetry of 
a fixed quantum theory; this is called the axial anomaly.
Rather, the group $U(1)_A\times U(1)_X$ 
acts on a family of quantum theories.
The $U(1)_A$ and $U(1)_X$ charges of $Q$, 
its fermionic component $\psi_Q$,
the gluino $\lam$ and the scale $\Lam$ are summarized as follows:
\begin{equation}
\renewcommand{\arraystretch}{1.3}
\begin{tabular}{c|ccc|c}
& $U(1)_A$ & $\;\times\!\!\!\!\!$ & $U(1)_X$ & $U(1)_R$	\\
\hline
$Q$  &$-1$   &&$0$ &$\frac{c_2(R)-c_2(G)}{c_2(R)}$\\
$\psi_Q$&   $-1$     &&   $-1$	&$-\frac{c_2(G)}{c_2(R)}$\\
$\lam$		&   $0$	     &	&   $1$	&	$1$\\
$\Lam^{b_0}$	&   $c_2(R)$  &	&$c_2(R)-c_2(G)$&	$0$
\end{tabular}
\end{equation}

\medskip\noindent
Here and below, representations of $U(1)$ are labeled by their weights.

The anomaly-free subgroup of $U(1)_A\times U(1)_X$ is a product of 
a $U(1)$ subgroup, denoted by $U(1)_R$, and a finite group containing
the center of $SU(F)$.
We normalize the generator of the Lie algebra $\gu(1)_R$ so that it projects
to that of $\gu(1)_X$.
The weights of $U(1)_R$, which are also shown in the above table, 
are related to those of $U(1)_A$ and $U(1)_X$ by
	\begin{equation}\label{RXA}
R=X+\frac{c_2(G)-c_2(R)}{c_2(R)}A.
	\end{equation}
The true quantum symmetry is $SU(F)\times U(1)_R$ up 
to a discrete factor.

\Headnn{Global gauge anomalies}

Consider a principal $G$-bundle $P$ over the (Euclidean) space-time.
Let $\AAA$ be the space of connections and $\GG$, the group of gauge
transformations.
Let $\dirac^+_R$ be the Dirac operator acting on Weyl spinors twisted by
a representation $R$.
A gauge theory with Weyl fermions has gauge anomalies if the determinant
line bundle $\Det\dirac^+_R$ over $\AAA/\GG$ is not trivial.
When the gauge group $G$ is semi-simple, the curvature $\FF$ of 
$\Det\dirac^+_R$ is proportional to a multi-linear form $d_R$ on 
the Lie algebra $\g$ defined by
	\begin{equation}\label{cubic}
d_R(x,y,z)=\tr_R x\{y,z\}\quad\quad(x,y,z\in\g).
	\end{equation}
Local gauge anomalies come from the non-vanishing of $\FF$.
For $G=Sp(n)$ or $G=SO(n)$, we have $d_R=0$ for any representation $R$ of $G$.
Consequently the bundle $\Det\dirac^+_R$ is flat and there are no local gauge
anomalies.

However if there is a non-contractible loop in $\AAA/\GG$, the holonomy
along it may be non-trivial even if $\FF=0$.
This is called global gauge anomaly.
Since the space $\AAA$ is contractible, $\pi_1(\AAA/\GG)\cong\pi_0(\GG)$.
When the (Euclidean) space-time is $S^4$, we have $\pi_0(\GG)\cong\pi_4(G)$.
When $G$ is a simple Lie group, $\pi_4(G)\ne1$ only if only if the universal
covering group is $Sp(n)$ for some $n\ge1$, in which case $\pi_4(G)\cong\ZZ_2$.
Thus the holonomy of $\Det\dirac^+_R$ for any $R$ is $\pm1$, and consequently,
the bundle $(\Det\dirac^+_R)^{\otimes 2}$ is always trivial.
In particular, consider $R=\Hom(F,C)$, where $G$ acts on $F$ trivially.
When $G=Sp(n)$ and $C\cong\CO^{2n}$ is its defining representation, 
the bundle $\Det\dirac^+_R=(\Det\dirac^+_C)^{\otimes\dim F}$ is trivial
if $\dim F$ is even.
This condition will guarantee the absence of global gauge anomalies.
That the bundle $\Det\dirac^+_C$ itself is non-trivial is a deeper fact
related to the Bott periodicity.

For $G=SO(n)$, the curvature $\FF$ of $\Det\dirac^+_R$ vanishes for any
representation $R$.
If $n\ge6$, since $\pi_4(G)=1$, the bundle $\Det\dirac^+_R$ is trivial for
any representation $R$ (including the defining representation).
If $G$ is simple and the representation $R$ is non-trivial and of real type,
then up to a finite subgroup of the center, $G$ can be embedded into $SO(d)$,
where $d=\dim R$.
The $G$-bundle $P$ induces a principal $SO(d)$-bundle $P_0$.
Let $\AAA_0$ be the space of connections on $P_0$, and $\GG_0$, the group of
gauge transformations.
Then there exist maps $\AAA\to\AAA_0$, $\GG\to\GG_0$ and
$\AAA/\GG\to\AAA_0/\GG_0$.
The determinant line bundle $\Det\dirac^+_R$ over $\AAA/\GG$ is
the pull-back of corresponding bundle over $\AAA_0/\GG_0$.
If $d\ge6$, the latter is trivial.
If $2\le d\le5$, the case can be reduced to $d\ge6$ since 
$\Det\dirac^+_{R\oplus R\oplus R}=(\Det\dirac^+_R)^{\otimes3}
\cong\Det\dirac^+_R$.
Therefore  $\Det\dirac^+_R$ is trivial for any real representation
$R$ of any compact Lie group $G$.

We list a few consequences of this result.
First, $R$ and $R'$ are both real or quarternionic representations of $G$,
then the representation $R\otimes R'$ is real.
So the bundle $\Det\dirac^+_{R\otimes R'}$ is trivial.
Second, if $G=SO(n)$ for any $n\ge3$, and $C\cong\CO^n$, the defining
representation, then the bundle $\Det\dirac^+_C$ is trivial.
Finally, since the adjoint representation is always of real type, 
the bundle $\Det\dirac^+_\mathrm{adj}$ is trivial.
Thus the gluino $\lam$ does not contribute to the global gauge anomaly.


\lecture{Problems 1--2}

\catcode`\@=11
%{\let\@makefnmark\relax  \let\@thefnmark\relax
%   \footnote{\bf Solutions by Siye Wu\\}\addtocounter{footnote}{-1}}
\catcode`\@=13

\Headnn{Solution to Problem 1}

Let $C\cong\CO^{2N_c}$ be an Hermitian vector space with a compatible
symplectic form $\om_c$.
$C$ is the defining representation of the compact group $G=Sp(C)\cong Sp(N_c)$.
We write the complexification $G^\co=Sp(C,\CO)$.
The representation on $C$ is quarternionic.
We have $c_2(C)=1$.
The dual Coxeter number of $G$ is $h=N_c+1$.
Therefore $c_2(G)=2(N_c+1)$.
Let the flavor space $F$ be of dimension $2N_f$.
(It is even to ensure the absence of the global gauge anomaly.)
Then $c_2(R)=2N_f$ and $b_0=3(N_c+1)-N_f$.

\Subhead{Classical theory}

The moduli space of classical vacua is $\MC\cong\Hom(F,C)\sss/Sp(C,\CO)$.
We define gauge invariant quantities
	\begin{equation}
M=Q^*\om_c\in\medwedge^2F^*.
	\end{equation}
Clearly, $M$ satisfies the constraint
	\be\label{sp-constraint}
\wedge^{N_c+1}M=0.
	\end{equation}
It is a classical result of Weyl that the ring of invariants 
$\CO[\Hom(F,C)]^G$ is generated by $M\in\medwedge^2F^*$ with the (only)
relations (\ref{sp-constraint}).\footnote{H.\ Weyl, {\it Classical groups}, 
\S VI.1}
So the moduli space $\MC$ is parametrized by $M$ subject to
the constraint (\ref{sp-constraint}).

Recall that $C$ has a quarternionic structure.
We have an orthogonal decomposition of the Lie algebra
	\begin{equation}
\gu(C)=\gu(C,\HA)\oplus\ii\{\HA{\mbox{-self-adjoint operators on }}C\}.
	\end{equation}
Since $\gu(C,\HA)=\gsp(C)$, the moment map is
	\begin{equation}
D(Q)={\mbox{orthogonal projection of }}\ii QQ^\dagger{\mbox{ onto }}\gu(C,\HA).
	\end{equation}
So $D(Q)=0$ if and only if $QQ^\dagger$ is $\HA$-self-adjoint on $C$.
In this case, there is an $\HA$-unitary basis of $C$ such that $QQ^\dagger$
has the form
	\begin{equation}
QQ^\dagger\sim
\begin{pmatrix}
a_1^2 &&&\\
&\ddots&&\\
&&a_r^2&\\
&&&
\end{pmatrix}
\otimes
\begin{pmatrix}
1 &0\\
0 &1\end{pmatrix},
\end{equation}
where $a_1,\dots,a_r>0$, $r=\hf\rank Q$.
Hence there is a unitary basis in $F$ such that $Q$ is of the form 
	\begin{equation}\label{sp-normal}
Q\sim
\begin{pmatrix}
a_1 &&&\\
&\ddots&&\\
&&a_r&\\
&&&
\end{pmatrix}
\otimes
\begin{pmatrix}
1 &0\\
0 &1\end{pmatrix}.
\end{equation}
In the complex language, any $Q\in\Hom(F,C)\sss$ can be put into the standard
form (\ref{sp-normal}) by a complexified gauge transformation and a global
symmetry.
The stabilizer of $Q$ in $Sp(C,\CO)$ is isomorphic to
$Sp(N_c-r,\CO)$ and that in $Sp(C)$ is $Sp(N_c-r)$.
At a generic point in $\MC$, the gauge group is broken to $Sp(N_c-N_f)$
if $N_c>N_f$; it is completely broken if $N_c\le N_f$.


\Subhead{Quantum theory}

Let $\Lam$ be the (complex) scale of the quantum theory.
Since $\Lam^{b_0}$ has $U(1)_A$-charge $c_2(R)=2N_f$, it is appropriate to
write
	\begin{equation}\label{sp-Lam}
\Lam^{3(N_c+1)-N_f}\in\medwedge^{2N_f}F^*.
	\end{equation}
The anomaly-free subgroup in $U(1)_A\times U(1)_X$ is $U(1)_R\times\ZZ_{2N_f}$,
where the $R$-charge can be computed by, following (\ref{RXA}),
	\begin{equation}
R=X+\frac{N_c+1-N_f}{N_f}A.
	\end{equation}
Since $\ZZ_{2N_f}=Z(SU(F))$, the global symmetry of the quantum theory is 
$SU(F)\times U(1)_R$.
The superfields $Q$ and $M$ transform as
	\begin{equation}
\renewcommand{\arraystretch}{1.3}
	\begin{tabular}{c|ccc}
&   $SU(F)$	&$\times$&	$U(1)_R$		\\
\hline
$Q$	&   $F^*$	&	&  $1-\frac{N_c+1}{N_f}$	\\
$M$	& $\medwedge^2F^*$ &	&  $2(1-\frac{N_c+1}{N_f})$
\end{tabular}
\end{equation}

\medskip\noindent
The vacuum degeneracy in the classical moduli space $\MC$ may be lifted by
a dynamically generated superpotential $W$ for the light fields $M$.
$W$ must be invariant under the gauge and global symmetries.
The fields in $M$ are already gauge invariant; the only combination 
invariant under $SU(F)$ is the Pfaffian $\Pf M$.
Also, the superpotential is holomorphic on $\MC$ hence in $M$.
These restrictions imply that $W$ is of the form
	\begin{equation}
W\sim(\Lam^{b_0})^x(\Pf M)^y.
	\end{equation}
The exponents $x$, $y$ can be fixed as follows:\\
1) Since $\Lam^{b_0},\Pf M\in\medwedge^{2N_f}F^*$ and $W$ is 
a number, we have $x+y=0$.\\
2) As a superpotential, $W$ has $U(1)_X$ charge $2(N_c+1-N_f)x=2$.\\
3) $W$ has mass dimension $b_0x+2N_fy=3$.\\
Notice that 1) and 2) together imply that $W$ has $U(1)_R$-charge $2$.
The above three equations for two unknowns $x$ and $y$ are consistent
and we get $x=-y=\inv{N_c+1-N_f}$.
Hence the superpotential is
	\begin{equation}\label{sp-W}
W=A\,\left(\frac{\Lam^{3(N_c+1)-N_f}}{\Pf M}\right)^{\inv{N_c+1-N_f}},
	\end{equation}
where $A$ is a constant that depends only on $N_c$, $N_f$ and the
renormalization scheme.
Notice that if $N_f\ge N_c+1$, then the superpotential $W$ in (\ref{sp-W})
does not exist because $\Pf M=0$.
In this case, the classical vacuum degeneracy is not lifted by quantum effects.
When $N_f\le N_c$, the decay of $W$ at $M\to\infty$ is consistent with
asymptotic freedom, which holds when $N_f<3(N_c+1)$.

\Headnn{Solution to Problem 2}

Let $C\cong\CO^{N_c}$ be a complex vector space with a real structure,
i.e., $C$ is the complexification of $C_\re\cong\RE^{N_c}$.
Suppose $C_\re$ is equipped with a metric $g_c$ and a compatible volume form
$v_c$, which provides a trivialization of $\det C_\re^*$.
(In the symplectic case, $v_c$ is fixed by the symplectic form $\om_c$.)
$C$ is the defining representation of the compact Lie group 
$G=SO(C)\cong SO(N_c)$.
(The complexification is denoted by $G^\co=SO(C,\CO)$.)
Let the flavor space $F$ have dimension $N_f$.
For $N_c\ge5$, the dual Coxeter number $h=N_c-2$ and $c_2(C)=2$.
So $c_2(G)=2(N_c-2)$, $c_2(R)=2N_f$ and $b_0=3(N_c-2)-N_f$.
For $N_c=4$, $G=(G_+\times G_-)/\ZZ_2$, where $G_\pm\cong SU(2)$.
For each factor $G_\pm$, we have $c_2(G_\pm)=4$, $c_2(C)=2$, $c_2(R)=2N_f$.
So $b_0=6-N_f$ for the coupling constant of each $G_\pm$.
These are formally the continuation of the cases $N_c\ge5$.
For $N_c=3$, $G\cong SU(2)$ up to a $\ZZ_2$ factor.
We have $c_2(G)=c_2(C)=4$, $c_2(R)=4N_f$ and $b_0=6-2N_f$.


\Subhead{Classical theory}

The moduli space of classical vacua is $\MC\cong\Hom(F,C)\sss/SO(C,\CO)$.
The gauge invariants are the mesons $M=Q^*g_c\in S^2F^*$ and,
if $\rank Q=N_c$, the baryons $B=Q^*v_c\in\medwedge^{N_c}F^*$.
They satisfy the constraints 
	\begin{equation}\label{so-constraint}
\rank M\le N_c,\quad\quad B\otimes B=\wedge^{N_c}M.
	\end{equation}
Here $\wedge^{N_c}M\in S^2(\medwedge^{N_c}F^*)$ corresponds to the product
of non-zero eigenvalues of $M$.
The ring of invariants $\CO[\Hom(F,C)]^G$ is generated by $M\in S^2F^*$ and,
if $N_f\ge N_c$, $B\in\medwedge^{N_c}F^*$ with the relations
(\ref{so-constraint}).\footnote{H.\ Weyl, {\it Classical groups}, \S II.9}
So $M$ and $B$ subject to the constraints in (\ref{so-constraint})
parametrize the moduli space $\MC$.

Consider the orthogonal decomposition
	\begin{equation}
\gu(C)=\gso(C)\oplus\ii\{\RE{\mbox{-self-adjoint operators on }}C_\re\}.
	\end{equation}
Note that $\gso(C)$ is the set of skew-self-adjoint operators on $C_\re$.
The moment map of the $G$-action on $\Hom(F,C)$ is
	\begin{equation}
D(Q)={\mbox{orthogonal projection of }}\ii QQ^\dagger{\mbox{ onto }}\gso(C).
	\end{equation}
So $D(Q)=0$ if and only if $QQ^\dagger$ is $\RE$-self-adjoint on $C_\re$.
In this case, there is an orthogonal basis of $C$ (i.e., up to a gauge 
transformation) such that $QQ^\dagger$ has the form
	\begin{equation}
QQ^\dagger\sim
\begin{pmatrix}
a_1^2 &&&\\
&\ddots &&\\
&&a_r^2&\\
&&&
\end{pmatrix},
	\end{equation}
where $a_1,\dots,a_r>0$ if $r=\rank Q<N_c$
and $a_1,\dots,a_{N_c-1}>0$, $a_{N_c}\ne0$ if $r=N_c$.
In the latter case, $B^{1,\dots,N_c}=a_1\dots a_{N_c}$
is the only non-zero component of $B$.
Hence up to gauge and global symmetries $Q$ is of the form
	\begin{equation}\label{so-normal}
Q\sim
\begin{pmatrix}
a_1 &&&\\
&\ddots&&\\
&& a_r&\\
&&&
\end{pmatrix}.
	\end{equation}
The stabilizer of $Q$ in $SO(C,\CO)$ is isomorphic to
$SO(N_c-r,\CO)$, and that in $SO(C_\re)$, $SO(N_c-r)$.
At a generic point in $\MC$, the gauge group is broken to $SO(N_c-N_f)$
if $N_c\ge N_f+2$; it is totally broken if $N_c<N_f+2$.


\Subhead{Quantum theory}

Let $\Lam$ be the (complex) scale of the quantum theory.
We have 
	\begin{equation}\label{so-scale}
\Lam^{3(N_c-2)-N_f}\in(\medwedge^{N_f}F^*)^{\otimes2}
	\end{equation}
if $N_c\ge4$.
(For $N_c=4$, this holds for the scale of each $SU(2)$ theory.) 
If $N_c=3$, then
	\begin{equation}\label{so3-scale}
\Lam^{6-2N_f}\in(\medwedge^{N_f}F^*)^{\otimes4}.
	\end{equation}
The anomaly-free subgroup of $U(1)_A\times U(1)_X$ is
$U(1)_R\times\ZZ_{2N_f}$ if $N_c\ge4$ and
$U(1)_R\times\ZZ_{4N_f}$ if $N_c=3$.
Here the $R$-charges are computed by
	\begin{equation}
R=X+\frac{N_c-2-N_f}{N_f}A.
	\end{equation}
The $\ZZ_{N_f}$ subgroup in $\ZZ_{2N_f}$ or $\ZZ_{4N_f}$ 
is the center of $SU(F)\cong SU(N_f)$.
Therefore the global symmetry of the quantum theory is
$(SU(F)\rtimes\ZZ_2)\times U(1)_R$ if $N_c\ge4$ 
and $(SU(F)\rtimes\ZZ_4)\times U(1)_R$ if $N_c=3$.
Here $SU(F)\rtimes\ZZ_n=\set{U\in U(F)}{U^n\in SU(F)}$.
For all $N_c\ge3$, the superfields $Q$ and $M$ transform as
\begin{equation}
\renewcommand{\arraystretch}{1.3}
\begin{tabular}{c|ccc}
&   $SU(F)$	&$\times$&	$U(1)_R$		\\
\hline
$Q$	&   $F^*$	&	&  $1-\frac{N_c-2}{N_f}$	\\
$M$	& $S^2F^*$ &	&  $2(1-\frac{N_c-2}{N_f})$
\end{tabular}
\end{equation}

\medskip\noindent
Invariance under the gauge and global symmetries and holomorphicity
imply that the superpotential, if it exists, is of the form
	\begin{equation}
W\sim(\Lam^{b_0})^x(\det M)^y
	\end{equation}
for some exponents $x$, $y$.
When $N_c\ge4$, considerations exactly the same as in the symplectic case 
yield three equations $x+y=0$, $2=2(N_c-N_f-2)x$ and $b_0x+2N_fy=3$
for the two unknowns $x$ and $y$.
Again, they have a solution $x=-y=\inv{N_c-2-N_f}$.
Hence
	\begin{equation}\label{so-W}
W=A\,\left(\frac{\Lam^{3(N_c-2)-N_f}}{\det M}\right)^{\inv{N_c-2-N_f}},
	\end{equation}
where $A$ is a constant that depends only on $N_c$, $N_f$ and the
renormalization scheme.
Clearly $W$ does not exist if $N_f=N_c-2$.
If $N_f>N_c$, then $\rank M<N_f$ and $\det M=0$.
So $W$ does not exist either.
If $N_c-2<N_f\le N_c$, $W\to\infty$ as $Q\to\infty$;
this is in contradiction with asymptotic freedom ($b_0>0$ in this range).
So the superpotential is not generated when $N_f\ge N_c-2$.
When $N_c=4$, (\ref{so-W}) is the superpotential generated by each of 
the $SU(2)$ factors.
We limit the analysis to the case of equal scales for the two $SU(2)$ factors.
When $N_c=3$, the three equations $2x+y=0$, $2=(4-4N_f)x$ and $b_0x+2N_fy=3$
are also consistent and have a solution $2x=-y=\inv{1-N_f}$.
So,
	\begin{equation}
W=A\,\left(\frac{\Lam^{3-N_f}}{\det M}\right)^{\inv{1-N_f}}.
	\end{equation}
Similarly, $W$ does not exist if $N_f=1$ or $N_f>3$.
When $N_f=2$, the form of $W$ is not consistent with asymptotic freedom
($b_0=2>0$ in this case).
When $N_f=3$, we have $b_0=0$, but the two-loop $\beta$-function is positive.
The theory is infrared free and does not have a superpotential.
In summary, no superpotential can be generated when $N_c=3$.


\lecture{Problems 3--4}

\catcode`\@=11
%{\let\@makefnmark\relax  \let\@thefnmark\relax
%   \footnote{\bf Solutions by Siye Wu}\addtocounter{footnote}{-1}}
\catcode`\@=13

	\Headnn{Solution to Problem 3}

\Subhead{{\boldmath $N_f=N_c$}: instanton generated superpotential}

Recall that when $N_f\le N_c$, it is possible to have a dynamically
generated superpotential (\ref{sp-W}).
This superpotential is generated by instantons when $N_f=N_c$,
because the following three requirements are independently met:\\
1) The result of any calculation in instanton perturbation theory is 
proportional to $\e{2\pi\ii\tau(\mu)}\sim\Lam^{b_0}$.
The superpotential $W$ in (\ref{sp-W}) has the same power of $\Lam$ 
when $N_f=N_c$.\\
2) An instanton vertex absorbs $2(N_c+1)$ of the $\lam$ and $2N_f$ of
the $\psi_Q$.
The usual vertex is given by the term $\lam\psi_QQ$.
When $N_f=N_c$, it is possible to form a diagram with $2$ external
lines of $\psi_Q$.\\
3) The $Sp(N_c)$ gauge symmetry is completely broken when the 
renormalization scale $\mu$ is small compared to $\bra Q\ket$.
This means the coupling constant stays small for all $\mu$.
Hence the instanton calculation is trustworthy.\\
Thus we expect $A\ne0$ when $N_f=N_c$.

We study how the constant $A$ depends on $N_c$ and $N_f$
using deformations on the moduli space.
Suppose $Q$ acquires an expectation value $\bra Q\ket\in\Hom(F,C)$
with $\hf\rank\bra Q\ket=r$.
Then $C=\hC\oplus\hC^\perp$, $F=\hF\oplus\hF^\perp$,
where $\hC=(\im\bra Q\ket)^\perp$, $\hF=\ker\bra Q\ket$
have dimensions $2(N_c-r)$, $2(N_f-r)$, respectively.
The expectation value $\bra Q\ket$ breaks the gauge group to $Sp(\hC)$.
Let $\Mperp\in\medwedge^2\hF^{\perp *}$ be the restriction
of $\bra M\ket$ to $\hF^\perp$; $\Mperp$ is non-degenerate.
Because of the superpotential (\ref{sp-W}), the excitations of $M$ in
the $\hF^\perp$ directions become massive.
In the low energy theory, $M$ is of the form
	\begin{equation}\label{sp-M}
M=\four{\hM}{}{}{\Mperp},
	\end{equation}
where $\Mperp$ is fixed and $\hM\in\medwedge^2\hF^2$ contains 
the light modes.
The space of light flavors is $\hF$.

When the renormalization scale $\mu>>(\Pf(\Mperp/2))^{1/2r}$,
the coupling constant $g(\mu)$ runs according to
	\begin{equation}\label{sp-high}
(\Lam/\mu)^{3(N_c+1)-N_f}=\e{-\frac{8\pi^2}{g^2(\mu)}}.
	\end{equation}
When $\mu<<(\Pf(\Mperp/2))^{1/2r}$, the low energy theory has $N_c-r$ colors
and $N_f-r$ flavors.
Let $\hLam$ be its scale.
Then
	\begin{equation}\label{sp-low}
(\hLam/\mu)^{3(N_c-r+1)-(N_f-r)}=\e{-\frac{8\pi^2}{g^2(\mu)}}.
	\end{equation}
We use the $\overline{{\rm DR}}$ subtraction scheme.
Matching the coupling constant $g(\mu)$ in (\ref{sp-high}) and (\ref{sp-low})
at $\mu=(\Pf(\Mperp/2))^{1/2r}$, we get
	\begin{equation}\label{sp-eflat}
\Lam^{3(N_c+1)-N_f}=\Pf(\Mperp/2)\,\hLam^{3(N_c-r+1)-(N_f-r)}.
	\end{equation}
This relation is valid for any $N_c$, $N_f$ and $r\le N_c,N_f$.
Using (\ref{sp-eflat}), the superpotential of the low energy
theory has the same form (\ref{sp-W}) but with a different coefficient 
$\hat{A}$ related to $A$ by
	\begin{equation}
A=2^{\frac{r}{N_c+1-N_f}}\hat{A}.
	\end{equation}

When $N_f=N_c$, if we choose $r=N_c-1$, then $\hf\dim\hC=\hf\dim\hF=1$.
Due to the accidental isomorphism $Sp(1)\cong SU(2)$ and their defining 
representations, the low energy theory is identical to the $SU(2)$
supersymmetric gauge theory with a single flavor of quark.
We know from the $SU(2)$ theory that $\hat{A}=1$ 
under the $\overline{{\rm DR}}$ scheme.
So $A=2^{N_c-1}$ when $N_f=N_c$.


\Subhead{{\boldmath $N_f<N_c$} and the pure gauge theory}

To derive the superpotential for smaller values of $N_f$ but with a fixed
$N_c$, we need to apply a gauge invariant mass deformation.
We perturb the theory by adding at the tree level a superpotential
$W\tree=\bra M,m\ket$, where $m\in\medwedge^2F$.
Let $r=\hf\rank m$.
We write $F=\hF\oplus\hF^\perp$ such that $\hf\dim\hF=N_f-r$ and 
the projection $m^\perp\in\medwedge^2\hF^\perp$ of $m$ is non-degenerate.
The gauge group of the low energy theory is still $Sp(C)$,
but the flavor space is $\hF$.
Let $\hLam$ be the scale.
When $\mu>>(\Pf m^\perp)^{1/r}$, $g(\mu)$ runs according to (\ref{sp-high}).
However when $\mu<<(\Pf m^\perp)^{1/r}$, there are $N_f-r$ flavors of 
light quarks, and we get
	\begin{equation}\label{sp-low'}
(\hLam/\mu)^{3(N_c+1)-(N_f-r)}=\e{-\frac{8\pi^2}{g^2(\mu)}}.
	\end{equation}
Matching the coupling constant $g(\mu)$ in (\ref{sp-high}) and (\ref{sp-low'})
at $\mu=(\Pf m^\perp)^{1/r}$, we get
	\begin{equation}\label{sp-emass}
\hLam^{3(N_c+1)-(N_f-r)}=\Pf m^\perp\Lam^{3(N_c+1)-N_f}.
	\end{equation}
This relation is valid for any $N_c$, $N_f$ and $r\le N_f$.

If $N_f\le N_c$ in the ultraviolet theory, the total superpotential after
the mass deformation is
	\begin{equation}\label{sp-Wm}
W=A\,\left(\frac{\Lam^{3(N_c+1)-N_f}}{\Pf M}\right)^{\inv{N_c+1-N_f}}
+\bra M,m\ket.
	\end{equation}
We eliminate the heavy fields in $M$ by minimizing bosonic potential.
At the extremum of the superpotential (\ref{sp-Wm}), $M$ is of the 
block-diagonal form (\ref{sp-M}), where
	\begin{equation}\label{sp-perp}
\Mperp=\left(\frac{A}{N_c+1-N_f}\right)^{\frac{N_c+1-N_f}{N_c+1-(N_f-r)}}
\left(\frac{\Pf m^\perp\Lam^{3(N_c+1)-N_f}}{\Pf\hM}\right)
^{\inv{N_c+1-(N_f-r)}}(m^\perp){}^{-1}.
	\end{equation}
Here $(m^\perp){}^{-1}\in\medwedge^2\hF^{\perp *}$ is induced by 
$m^\perp\in\medwedge^2\hF^\perp$.
We compute the superpotential of the light mesons $\hM$ by
substitute in (\ref{sp-Wm}) the solution of $\Mperp$.
Using $\bra(m^\perp){}^{-1},m^\perp\ket=r$, 
$\dr\,\Pf M=\Pf M\bra\dr M,M^{-1}\ket$ and (\ref{sp-emass}), 
we deduce that the superpotential has precisely the same form as (\ref{sp-W}), 
with the coefficient 
	\begin{equation}\label{sp-decouple}
\hat{A}=(N_c+1-(N_f-r))
\left(\frac{A}{N_c+1-N_f}\right)^{\frac{N_c+1-N_f}{N_c+1-(N_f-r)}}.
	\end{equation}
If we have $N_f=N_c$ to begin with, then $A=2^{N_c-1}$ and
$\hat{A}=(N_c+1-(N_f-r))2^{\frac{N_c-1}{N_c+1-(N_f-r)}}$.
Equivalently, for a theory with $N_f\le N_c$, the coefficient $A$ 
in (\ref{sp-W}) is given by
	\begin{equation}\label{sp-A}
A=(N_c+1-N_f)2^{\frac{N_c-1}{N_c+1-N_f}}.
	\end{equation}
In all cases of $N_f\le N_c$, the superpotential is non-zero.
When $1\le N_f\le N_c$, the classical vacuum degeneracy is lifted by $W$
and the quantum theory has no vacuum.

In the pure $Sp(C)\cong Sp(N_c)$ gauge theory, $N_f=0$, and the superpotential
	\begin{equation}\label{sp-pure}
W=(N_c+1)2^{\frac{N_c-1}{N_c+1}}\Lam^3
=(N_c+1)\mu^32^{\frac{N_c-1}{N_c+1}}\e{\frac{2\pi\ii}{N_c+1}\tau}.
	\end{equation}
This is $(N_c+1)$-valued, corresponding to the fact
that there are $N_c+1$ vacua in quantum theory.
In particular, when $N_c=1$, (\ref{sp-pure}) agrees with the result for
$SU(2)$ gauge theory.
Notice also that the superpotential $W$ is a constant:
although non-zero, it does not contribute to the effective Lagrangian.
Nevertheless, the result (\ref{sp-pure}) implies gaugino condensation.
We consider an extended theory in which the parameter $\tau$ becomes 
a superfield and assume that supersymmetry is unbroken after quantization.
The partition function
	\begin{equation}
Z(\tau)=\int DA\,D\lam\,\e{\ii S(A,\lam,\tau)}
=\e{\ii\int\dr^4x\dr^2\tht\,W(\tau)}
	\end{equation}
is holomorphic in $\tau$ and $W(\tau)$ has the same form as (\ref{sp-pure}).
Since the Lagrangian density of the pure gauge theory is
	\begin{equation}
\LL=\inv{4\pi\ii}\int\dr^2\tht\;\tau\;\tr W_\al W^\al+\cc,
	\end{equation}
where $W_\al=\lam_\al+\cdots$ is the superfield strength, we get
\begin{equation}
\begin{aligned}
\bra\lam\lam\ket &=4\pi\frac{\del}{\del F_\tau}\log Z(\tau)
=4\pi\frac{\pdr}{\pdr F_\tau}\int\dr^2\tht\,W(\tau)\\
&=4\pi\frac{\pdr}{\pdr\tau}W(\tau)
=2^{4-\frac{2}{N_c+1}}\pi^2\ii\Lam^3
=2^{4-\frac{2}{N_c+1}}\pi^2\ii\mu^3\e{\frac{2\pi\ii}{N_c+1}\tau}.
\end{aligned}
\end{equation}
Again, this is $(N_c+1)$-valued.


\Subhead{{\boldmath $N_f=N_c+1$}}

In the above mass deformation of theories with $N_f\le N_c$, if $r=N_f$,
then $\hF^\perp=F$, $m^\perp=m$ and $M^\perp=M$.
The extremum of the total superpotential (\ref{sp-Wm}) is reached at
	\begin{equation}\label{sp-exp}
\bra M\ket=2^{\frac{N_c-1}{N_c+1}}(\Lam^{3(N_c+1)-N_f}\Pf m)^{\inv{N_c+1}}
\,m^{-1}.
	\end{equation}
Just as in the supersymmetric $SU(N_c)$ gauge theories,
the above result is valid in the range $N_f\ge N_c+1$.
For $N_f=N_c+1$, the solution (\ref{sp-exp}) satisfies
	\begin{equation}\label{sp-q}
\Pf M=2^{N_c-1}\Lam^{2(N_c+1)}.
	\end{equation}
This is independent of $m$, and shall remain valid in the absence of
the mass term.
We claim that in the low energies the quantum theory is described
by $M$ subject to the constraint (\ref{sp-q}).
To check that the theory reduces to $N_f=N_c$ via mass deformation,
we add a mass term $\bra M,m\ket$ with $\hf\rank m=1$.
Then $\hf\dim\hF=N_c$.
Because of the constraint (\ref{sp-q}), we have
	\begin{equation}
\bra M,m\ket=\frac{2^{N_c-1}m^\perp\Lam^{2(N_c+1)}}{\Pf\hat{M}}
=2^{N_c-1}\frac{\hLam^{2N_c+3}}{\Pf\hM},
	\end{equation}
where the last equality follows from (\ref{sp-emass}) when $N_f=N_c+1$, $r=1$.
This is the superpotential (\ref{sp-W}) with the correct coefficient
(\ref{sp-A}) when $N_f=N_c$.

The quantum moduli space $\MQ$ is smooth.
There are no additional massless fields other than $M$ subject to the 
constraint (\ref{sp-q}).
The absence of massless gluons is known as the confinement of color charges.
We check the consistency of these postulates using 't Hooft's anomaly 
matching conditions.
Each point $M_0\in\medwedge^2F^*$ on $\MQ$ is a symplectic form on $F$.
Let $Sp(F,\CO)$ be the (complex) symplectic group preserving $M_0$.
The most symmetric points on $\MQ$ are those compatible with
the Hermitian structure on $F$.
There the global symmetry $SU(F)\times U(1)_R$ is broken to
$Sp(F)\times U(1)_R$, where the (compact) symplectic group
$Sp(F)=Sp(F,\CO)\cap U(F)$ depends on the choice of $M_0$.
Notice that when $N_f=N_c+1$, $U(1)_X=U(1)_R$ is anomaly free.
In the ultraviolet theory, the massless fermions are $\psi_Q$ and $\lam$.
In the infrared theory, they are the fermionic partners $\psi_M$ of $M$ 
in $T_{M_0}\MQ$.
Under the unbroken global symmetry, 
these light fermions transform according to
\begin{equation}
\begin{tabular}{cc|ccc}
\renewcommand{\arraystretch}{1.3}
&&$Sp(F)$ &$\times$ &$U(1)_R$\\
\hline
ultraviolet & $\psi_Q$	&   $F^*$  &&  $-\frac{N_c+1}{N_f}=-1$\\
&$\lam$	&   $\one$	&  &	$1$\\
\hline
infrared   &$\psi_M$& $\medwedge^2F^*\ominus\one$ 
& &  $1-2\frac{N_c+1}{N_f}=-1$ 
\end{tabular}
\end{equation}

\medskip\noindent
Here and below, $\one$ denotes the trivial representation.
The non-zero combinations to 't Hooft's condition are given by
\begin{equation}
\renewcommand{\arraystretch}{1.3}
\begin{tabular}{c|lll}
& ultraviolet	&& infrared	\\
\hline
$\gsp(F)^2\gu(1)_R$ & $2N_cc_2(F)\cdot(-1)$
&& $c_2(\medwedge^2F\ominus\one)\cdot(-1)$\\
$\gu(1)_R$ & $4N_cN_f\;(-1)+N_c(2N_c+1)\;1$ && $(N_f(2N_f-1)-1)\;(-1)$ \\
$\gu(1)_R^3$ &$4N_cN_f\;(-1)^3+N_c(2N_c+1)\;1^3$ && $(N_f(2N_f-1)-1)\;(-1)^3$
\end{tabular}
\end{equation}

\medskip\noindent
To calculate $c_2(\medwedge^2F\ominus\one)$, we note that the 
root system of $\gsp(N_f)$ is\break
$\set{\eps_i\pm\eps_j,2\eps_i}{1\le i\ne j\le N_f}$
and the positive roots are $\eps_i\pm\eps_j$ ($1\le i\le j\le N_f$).
The highest weights of the representations of $F$ and
$\medwedge^2F\ominus\one$ 
are $\lam=\eps_1$ and $\lam_2=\eps_1+\eps_2$, 
respectively.
Therefore $c_2(\medwedge^2F\ominus\one)
=\frac{(\lam_2+2\rho,\lam_2)\,\dim(\wedge^2F\ominus\one)}
{(\lam_1+2\rho,\lam_1)\,\dim F}c_2(F)=2(N_f-1)c_2(F)$, where
$\rho=\sum_{i=1}^{N_f}(N_f+1-i)\eps_i$ is half of the sum of positive roots.
The two columns match when $N_f=N_c+1$.

Finally, we propose a global anomaly matching condition and show that it is
satisfied in the above situation.
Consider two theories with the same (anomaly free) global symmetry group $H$
but with different contents of light fermionic fields, which form 
representations $R$ and $R'$ of $H$, respectively.
If the two theories describe the same low energy physics, then if $H$ is
hypothetically gauged, the determinant line bundles $\Det\dirac^+_R$
and $\Det\dirac^+_{R'}$ of the Dirac operators twisted by $R$ and $R'$
shall be isomorphic.
't Hooft's anomaly matching condition checks whether the curvatures of the two
bundles are the same.
This is a stringent condition for the two theories to describe the same
low energy physics.
However the two bundles can still differ by a flat bundle, which can be
non-trivial if $\pi_4(H)\ne1$.
If the representation $R^*\oplus R'$ of $H$ is real,
then the determinant line bundle $\Det\dirac^+_{R^*\oplus R'}\cong
(\Det\dirac^+_R)^{-1}\otimes\Det\dirac^+_{R'}$ is trivial.
Therefore the reality of the representation $R^*\oplus R'$ is a sufficient
condition of global anomaly matching.
In the above problem, suppose only the $Sp(F)$ factor of the global symmetry
is gauged.
In the ultraviolet theory, the representation is an even number ($2N_c$)
of copies of $F^*$ and the trivial representations.
In the infrared theory, the representation 
$\medwedge^2F^*\ominus\one\subset F^*\otimes F^*$
is real because $F^*$ is quarternionic. 
So the global anomalies match.


\Subhead{{\boldmath $N_f=N_c+2$}}

When $N_f=N_c+2$, we obtain from (\ref{sp-exp}) that $\medwedge^{N_c+1}M\to0$
as $m\to 0$. 
So there is no quantum correction to the classical moduli space $\MC$.
We claim that in the low energy theory, all components of $M$ become dynamical
and that there is a superpotential
	\begin{equation}\label{sp-W+2}
W=-\frac{\Pf M}{2^{N_c-1}\Lam^{2N_c+1}}.
	\end{equation}
Indeed, varying $W$ with respect to $M$, we get the constraint 
(\ref{sp-constraint}).
If we add a mass term $W\tree=\bra M,m\ket$ with $\hf\rank m=1$, 
then $\hf\dim\hF=N_c+1$.
At the extremum of $W+W\tree$, $M$ is of the block-diagonal form
(\ref{sp-M}).
{}From varying $M^\perp$, we get
	\begin{equation}
\Pf\hM=2^{N_c-1}m^\perp\Lam^{2N_c+1}.
	\end{equation}
By (\ref{sp-emass}), this is exactly the constraint (\ref{sp-q}).

The most symmetric point on $\MC$ is $M=0$,
where the full global symmetry $SU(F)\times U(1)_R$ is unbroken.
This is also the most singular point.
The singularity is the result of additional massless fields.
In fact all the light fields in $M$ are massless at the origin.
We check this using 't Hooft's anomaly matching condition.
Under the global symmetry, 
the light fermions in the ultraviolet and the
infrared theories transform as
	\begin{equation}
	\renewcommand{\arraystretch}{1.3}
	\begin{tabular}{cc|ccc}
&	&   $SU(F)$	&$\times$&	$U(1)_R$		\\
	\hline
ultraviolet & $\psi_Q$	&   $F^*$  &&  $-\frac{N_c+1}{N_f}$	\\
&$\lam$	&   $\one$	&	&	$1$	\\
	\hline
infrared   &$\psi_M$& $\medwedge^2F^*$ & &  $1-2\frac{N_c+1}{N_f}$ 
	\end{tabular}
	\end{equation}

\medskip\noindent
The non-zero combinations to be matched are
	\begin{equation}\label{sp-tH}
	\renewcommand{\arraystretch}{1.3}
	\begin{tabular}{c|ll}
& ultraviolet		& infrared\\
\hline
$\gsu(F)^3$ & $2N_c\,(-d_F)$	& $-d_{\wedge^2F}$\\
$\gsu(F)^2\gu(1)_R$	& $2N_cc_2(F)\cdot(-\frac{N_c+1}{N_f})$
& $c_2(\medwedge^2F)\cdot(1-2\frac{N_c+1}{N_f})$	\\
$\gu(1)_R$ & $4N_cN_f\;(-\frac{N_c+1}{N_f})+N_c(2N_c+1)\;1$
& $N_f(2N_f-1)\;(1-2\frac{N_c+1}{N_f})$	\\
$\gu(1)_R^3$ & $4N_cN_f\;(-\frac{N_c+1}{N_f})^3+N_c(2N_c+1)\;1^3$
& $N_f(2N_f-1)\;(1-2\frac{N_c+1}{N_f})^3$
	\end{tabular}
	\end{equation}

\medskip\noindent
We claim that $c_2(\medwedge^2F)=2(N_f-1)c_2(F)$ and 
$d_{\wedge^2F}=2(N_f-2)d_F$.
Indeed, 
let $\eps_1,\dots,\eps_{2N_f}$ be the weights of the fundamental
representation of $SU(F)$.
Since $\sum_{i=1}^{2N_f}\eps_i=0$, we have
$\sum_{i<j}(\eps_i+\eps_j)^2=2(N_f-1)\sum_{i=1}^{2N_f}\eps_i^2$ and
$\sum_{i<j}(\eps_i+\eps_j)^3=2(N_f-2)\sum_{i=1}^{2N_f}\eps_i^3$.
Therefore the two columns match when $N_f=N_c+2$.

	\Headnn{Solution to Problem 4}


\Subhead{{\boldmath $N_f\le N_c-5$} and pure gauge theory}

When $N_f\le N_c-5$, it is possible to have a dynamically generated
superpotential of the form (\ref{so-W}).
Since most of the analysis is parallel to the $Sp(N_c)$ gauge theories,
we only give an outline here.


Suppose $M$ acquires a large expectation value $\bra M\ket$ of rank $r$.
Let $\hF=\ker\bra M\ket$.
Then $F=\hF\oplus\hF^\perp$ and the restriction $\Mperp\in S^2\hF^{\perp *}$
of $\bra M\ket$ to $\hF^\perp$ is non-degenerate.
The low energy theory has $N_c-r$ colors and $\dim\hF=N_f-r$ flavors.
So $b_0=3(N_c-r-2)-(N_f-r)$.
Matching the coupling constant at $\mu=(\det\Mperp)^{1/2r}$, we obtain
        \begin{equation}\label{so-eflat}
\Lam^{3(N_c-2)-N_f}=\det\Mperp\hLam^{3(N_c-r-2)-(N_f-r)}.
        \end{equation}
In fact, this relation remains valid for all values of $N_c$, $N_f$ as long as
$N_c-r\ge5$.
For future needs, we also consider here the cases $N_c-r\le4$.
For $N_c-r=4$, in the low energy $SO(4)$ theory, both scales $\hLam_\pm$ of
the $SU(2)_\pm$ factors are given by (\ref{so-eflat}).
If $N_c-r=3$, we derive the relation in two stages:
first from $N_c\ge5$ to $N_c=4$, then from $N_c=4$ to $N_c=3$. 
In the second stage, $N_c=4$, $r=1$.
The low energy theory has $3$ colors and $N_f-1$ flavors.
Above the scale $\hLam$, the gauge group is $SU(2)_+\times SU(2)_-$;
the coupling constants $g_\pm(\mu)$, or $\tau_\pm(\mu)$ run according
(\ref{running}) with $b_0=6-N_f$.
Since $SU(2)$ is diagonally embedded, its coupling constant is given by
$\tau(\mu)=\tau_+(\mu)+\tau_-(\mu)$.
Therefore the scale is
$(\hLam_+^{6-N_f}\hLam_-^{6-N_f})^{1/2(6-N_f)}=(\hLam_+\hLam_-)^{1/2}$.
At low energies, the $SU(2)$ gauge theory has $b_0=6-2(N_f-1)$.
Matching the coupling constant $g(\mu)$ at $\mu=\hf\Mperp$, we get
	\begin{equation}\label{so43-eflat}
4\hLam_+^{6-N_f}\hLam_-^{6-N_f}=\Mperp^2\hLam^{6-2(N_f-1)}.
	\end{equation}
Combining (\ref{so-eflat}) and (\ref{so43-eflat}), we get,
for any $N_c\ge4$, $N_f$ and $r=N_c-3$,
	\begin{equation}\label{so3-eflat}
4(\Lam^{3(N_c-2)-N_f})^2=(\det\Mperp)^2\hLam^{6-2(N_f-N_c+3)}.
	\end{equation}

When $N_f\le N_c-5$ and $N_c-r\ge5$, a comparison of the superpotential
(\ref{so-W}) above and below $\mu=(\det\Mperp)^{1/2r}$ yields $A=\hat{A}$.
If $N_f=N_c-5$ and $r=N_f$, the low energy theory is a pure 
$SO(5)\cong Sp(2)/\ZZ_2$ gauge theory.
Using (\ref{sp-A}), we get $\hat{A}=3\cdot2^{\inv{3}}$.
So $A=3\cdot2^{\inv{3}}$ when $N_f=N_c-5$.

Now we add a mass term $W\tree=\hf\bra M,m\ket$ with $\rank m=r$.
Again $F=\hF\oplus\hF^\perp$.
Here $\dim\hF=N_c-r$ and the projection $m^\perp\in S^2\hF^\perp$ of $m$
is non-degenerate.
The low energy theory has the same number of colors and $N_f-r$ flavors.
So $b_0=3(N_c-2)-(N_f-r)$.
Matching the coupling constant at $\mu=(\det m^\perp)^{1/r}$, we obtain
        \begin{equation}\label{so-emass}
\hLam^{3(N_c-2)-(N_f-r)}=\det m^\perp\Lam^{3(N_c-2)-N_f}.
        \end{equation}
This is valid for $N_c\ge5$ and all $N_f$.
When $N_c=4$, the scales $\hLam_\pm$ of the $SU(2)_\pm$ factors are related
to the scales $\Lam_\pm$ of the ultraviolet theory by the same formula.
When $N_c=3$, (\ref{so-emass}) should be replaced by
	\begin{equation}\label{so3-emass}
\hLam^{6-2(N_f-r)}=(\det m^\perp)^2\Lam^{6-2N_f}.
        \end{equation}

When $N_f\le N_c-5$, the superpotential $W$ before mass deformation is 
given by (\ref{so-W}).
Varying the heavy modes in $M$, the extremum of the total superpotential
$W+W\tree$ is reached when $M$ is of the diagonal form (\ref{sp-M}) with
	\begin{equation}\label{so-perp}
\Mperp=\left(\frac{2A}{N_c-2-N_f}\right)^{\frac{N_c-2-N_f}{N_c-2-(N_f-r)}}
\left(\frac{\det m^\perp\Lam^{3(N_c-2)-N_f}}{\det\hM}\right)
^{\inv{N_c-2-(N_f-r)}}(m^\perp){}^{-1}.
	\end{equation}
In terms of the of the light fields $\hM$, this extremum has the same form 
as (\ref{so-W}) with
        \begin{equation}
\hat{A}=\frac{N_c-2-(N_f-r)}{2}
\left(\frac{2A}{N_c-2-N_f}\right)^{\frac{N_c-2-N_f}{N_c-2-(N_f-r)}}.
        \end{equation}
If we start from $N_f=N_c-5$, then $A=3\cdot 2^{\inv{3}}$ 
and $\hat{A}=\hf(N_c-2-(N_f-r))2^{\frac{4}{N_c-2-(N_f-r)}}$.
Equivalently, for a theory with $N_f\le N_c-5$, the coefficient
$A$ in (\ref{so-W}) is given by
	\begin{equation}\label{so-A}
A=\hf(N_c-2-N_f)2^{\frac{4}{N_c-2-N_f}}.
	\end{equation}

For pure gauge theories, $N_f=0$ and $A=\hf(N_c-2)2^{\frac{4}{N_c-2}}$.
As in (\ref{so-ll}), for the pure $SO(N_c)$ gauge theory with $N_c\ge5$,
	\begin{equation}\label{so-ll}
\bra\lam\lam\ket=2^{2+\frac{4}{N_c-2}}\pi^2\ii\Lam^3
=2^{2+\frac{4}{N_c-2}}\pi^2\ii\mu^3\e{\frac{2\pi\ii}{N_c-2}\tau};
	\end{equation}
this is $(N_c-2)$-valued, corresponding to the $N_c-2$ vacua.
Finally, the result is consistent with another accidental isomorphism
$SO(6)\cong SU(4)/\ZZ_2$.
Indeed, when $N_c=6$, $A=4$;
this agrees with the coefficient of the pure $SU(4)$ gauge theory.


\Subhead{{\boldmath $N_f=N_c-4$}}

A generic flat deformation on the moduli space breaks the gauge group to 
$SO(4)\cong(SU(2)_+\times SU(2)_-)/\ZZ_2$.
{}From (\ref{so-eflat}), the scales $\hLam_\pm$ of the two $SU(2)$ gauge
theories (without matter) are related to the scale $\Lam$ of 
the ultraviolet theory by
	\begin{equation}
\hLam_\pm^6=\frac{\Lam^{2(N_c-1)}}{\det M}.
	\end{equation}
In the infrared, the superpotential of the two $SU(2)_\pm$ gauge theories are
$W_\pm=2\hLam_\pm^3$, respectively.
But both of them are two-valued, corresponding to the two vacua of
each $SU(2)$ theory.
We fix a square root.
Then the superpotential of the original theory is
	\begin{equation}\label{so4-W}
W=2(\eps_++\eps_-)\left(\frac{\Lam^{2(N_c-1)}}{\det M}\right)^{1/2},
	\end{equation}
where $\eps_+,\eps_-=\pm1$.
There are two physically inequivalent branches $\eps_+=\eps_-$ and
$\eps_+=-\eps_-$.

The superpotential (\ref{so4-W}) is compatible with mass deformation.
In the branch $\eps_+=\eps_-$, it is the continuation to $N_f=N_c-4$ of
(\ref{so-W}).
As before, adding a mass term $\hf\bra M,m\ket$ of $\rank m=r$
reduces the theory to $N_f=N_c-4-r$ with the same superpotential
(\ref{so-W}) in the low energies.
In the other branch $\eps_+=-\eps_-$, the total superpotential is 
$W\tree=\hf\bra M,m\ket$.
The theory has no vacuum and hence does not contribute to the low energy
theory with a smaller $N_f$.

For $\eps_+=\eps_-$, the superpotential (\ref{so4-W}) lifts the classical 
vacuum degeneracy and the quantum theory has no vacuum.
In the branch $\eps_+=-\eps_-$, the vacuum degeneracy is not lifted by 
quantum effects.
Classically there is a singularity at $M=0$, where the full gauge symmetry
is restored and all the gluons become massless.
We claim that in quantum theory, the singularity is smoothed out and the
color charges are confined.
That is, there are no massless gluons and
all the massless fermions are contained in $M$, even at the origin of $\MM$.
We check this claim by the 't Hooft and the global anomaly matching conditions.
At $M=0$, the identity component of the unbroken global symmetry is
$SU(F)\times U(1)_R$.
The light fermions in high and low energy theories transform according to
	\begin{equation}
	\renewcommand{\arraystretch}{1.3}
	\begin{tabular}{cc|ccc}
&	&   $SU(F)$	&$\times$&	$U(1)_R$\\
\hline
ultraviolet & $\psi_Q$	&   $F^*$& & $-\frac{N_c-2}{N_f}$\\
&$\lam$	&   $\one$	&	&	$1$			\\
\hline
infrared   &$\psi_M$& $S^2F^*$	&&$1-2\frac{N_c-2}{N_f}$ 
	\end{tabular}
	\end{equation}

\medskip\noindent
The non-zero combinations to be matched are given by
	\begin{equation}\label{so4-tH}
	\renewcommand{\arraystretch}{1.3}
	\begin{tabular}{c|ll}
& ultraviolet	& infrared	\\
\hline
$\gsu(F)^3$	& $-N_cd_F$	& $-d_{S^2F}$			\\
$\gsu(F)^2\gu(1)_R$	& $N_cc_2(F)\cdot(-\frac{N_c-2}{N_f})$
& $c_2(S^2F)\cdot(1-2\frac{N_c-2}{N_f})$	\\
$\gu(1)_R$	& $N_cN_f\;(-\frac{N_c-2}{N_f})+\hf N_c(N_c-1)\;1$
& $\hf N_f(N_f+1)\;(1-2\frac{N_c-2}{N_f})$ \\
$\gu(1)_R^3$  & $N_cN_f\;(-\frac{N_c-2}{N_f})^3+\hf N_c(N_c-1)\;1^3$
	& $\hf N_f(N_f+1)\;(1-2\frac{N_c-2}{N_f})^3$
	\end{tabular}
	\end{equation}

\medskip\noindent
Let $\eps_i$ ($i=1,\dots,N_f$) be the weights in the defining representation
$F$.
Using $\sum_{i=1}^{N_f}\eps_i=0$, it is easy to check
$\sum_{i\le j}(\eps_i+\eps_j)^2=(N_f+2)\sum_{i=1}^{N_f}\eps_i^2$ and
$\sum_{i\le j}(\eps_i+\eps_j)^3=(N_f+4)\sum_{i=1}^{N_f}\eps_i^3$.
So $c_2(S^2F)=(N_f+2)c_2(F)$ and $d_{S^2F}=(N_f+4)d_F$.
It is now straightforward to check that the two columns of (\ref{so4-tH})
match when $N_f=N_c-4$, fulfilling 't Hooft's criterion.
For $N_f=2$, $SU(F)\cong Sp(1)$.
In the ultraviolet theory, the number of copies of $F^*$ is even ($N_c=6$).
In the infrared, the representation $S^2F^*\subset F^*\otimes F^*$ is real.
So the global anomalies also match.


\Subhead{{\boldmath $N_f=N_c-3$}}

We first find the superpotential by flat deformations on the moduli space.
A generic expectation value of $Q$ breaks the gauge group to $SO(3)$.
Let $\hF\subset F$ a subspace of dimension $1$.
We consider the limit where a large expectation value 
$\Mperp\in S^2\hF^{\perp*}$ with rank $N_f-1$ breaks the theory to 
$SO(4)\cong(SU(2)_+\times SU(2)_-)/\ZZ_2$ 
with one quark $\hQ\in\Hom(\hF,C)$, which further break the gauge group 
to a diagonal subgroup $SU(2)/\ZZ_2$.\footnote{The decomposition 
of $SO(4)$ into $SU(2)_+\times SU(2)_-$ is canonical.
The isotropy group of $\hQ\in\Hom(\hF,C)$ (in $SO(4)$) and its isomorphisms 
onto $SU(2)_\pm$ depend on $\hQ$.}
In the first stage of deformation, from (\ref{so-eflat}), 
the scales $\hLam_\pm$ of the $SU(2)_+\times SU(2)_-$ theory are given by
	\begin{equation}
\hLam_\pm^5=\frac{\Lam^{2N_c-3}}{\det\Mperp}.
	\end{equation}
The effective $SU(2)_\pm$ theories have instanton generated superpotentials
	\begin{equation}\label{so3-W1}
\hat{W}_\pm=2\frac{\hLam_\pm^5}{\det\hM}=2\frac{\Lam^{2N_c-3}}{\det M}.
	\end{equation}
In the second stage, we have the low energy pure $SU(2)$ gauge theory whose
scale $\hLam$ is given by (\ref{so3-eflat}).
It has a superpotential
	\begin{equation}\label{so3-W2}
\hat{W}=2\hLam^3=4\eps\frac{\Lam^{2N_c-3}}{\det M},
	\end{equation}
where $\eps=\pm1$ comes from taking the square root of (\ref{so3-eflat})
and labels the two vacua of the pure $SU(2)$ theory.
Combining (\ref{so3-W1}) and (\ref{so3-W2}), we find that the superpotential 
of the $N_f=N_c-3$ theory is
	\begin{equation}\label{so3-W}
W=\hat{W}_++\hat{W}_-+\hat{W}=4(1+\eps)\frac{\Lam^{2N_c-3}}{\det M}.
	\end{equation}
The first term in (\ref{so3-W}) is due to the instanton effect
whereas the second term comes from gaugino condensation in the unbroken 
$SU(2)$ theory.
As in $SU(N_c)$ and $Sp(N_c)$ gauge theories, the superpotential (\ref{so3-W})
is proportional to the exponential of the instanton action, and it is possible
to form a diagram with two external fermion lines using the instanton vertex.
The novelty here is that the unbroken gauge group $SU(2)$ is non-Abelian.
However, when the gauge group is broken to the diagonal SU(2), an instanton
in one $SU(2)_\pm$ cannot be rotated into the unbroken group.
Therefore the instanton computation is still reliable because
there is no infrared divergence.

We check that the result (\ref{so3-W}) is consistent with mass deformation.
The theory has two branches: $\eps=1$ and $\eps=-1$.
After adding a mass term $W\tree=\hf\bra M,m\ket$ with $\rank m=1$,
the two branches should correspond to those of $N_f=N_c-4$.
In fact, when $\eps=1$, the total superpotential (\ref{so3-W}) is the
continuation of (\ref{so-W}), just as (\ref{so4-W}) when $\eps_+=\eps_-$.
So the extremum of $W+W\tree$ matches (\ref{so4-W}) in the branch
$\eps_+=\eps_-$.
In particular, from (\ref{so-perp}), the extremum is reached when
	\begin{equation}\label{so3-sol}
\Mperp=4\left(\frac{\Lam^{2N_c-3}}{m^\perp\det\hM}\right)^{1/2}.
	\end{equation}
The ambiguity in the sign of the square root corresponds to the two
possibilities $\eps_+=\eps_-=\pm1$.
In the other branch $\eps=-1$, we claim that the theory has additional light
fields $q\in F\otimes\medwedge^{N_f}F$ coupling to $M$ and that
the superpotential near $M=0$ behaves as
	\begin{equation}\label{so4-Wq}
W\sim\inv{2\mu}M(q,q),
	\end{equation}
where $\mu\in(\medwedge^{N_f}F)^{\otimes2}$ 
is a parameter of mass dimension $1$.
The $R$-charges of $M$, $q$ are $-\frac{2}{N_f}$, $1+\inv{N_f}$, respectively.
This is consistent with requirement that $W$ has $R$-charge $2$.
Upon mass deformation, we write $F=\hF\oplus\hF^\perp$ as before.
The extremum of $W+W\tree$ is reached at
$q^\perp=\pm(-m\mu)^{1/2}$, $\hat{q}=0$, $M(q,\cdot)=0$.
The light fields are in $M=\hM\in S^2\hF$.
This matches the low energy behavior of $N_f=N_c-4$.
The signs of $q^\perp$ correspond to the choices of $\eps_+$, $\eps_-$
in $\eps_+=-\eps_-=\pm1$.

We check the postulate on the existence of $q$ using 't Hooft's and global
anomaly matching conditions.
In the low energy theory, the global symmetry $SU(F)\times U(1)_R$
is unbroken at $M=0$.
Also at $M=0$, the fields $q$ are massless.
The $R$-charge of $\psi_q$ is $\inv{N_f}$.
So their contribution to the left column of (\ref{so4-tH}) is
	\begin{equation}\label{anomaly-q}
	\renewcommand{\arraystretch}{1.3}
	\begin{tabular}{c|c}
& infrared: additional contribution from $q$	\\
\hline
$\gsu(F)^3$		&	$d_F$	\\
$\gsu(F)^2\gu(1)_R$	&	$c_2(F)\cdot\inv{N_f}$	\\
$\gu(1)_R$		&	$N_f(\inv{N_f})$ \\
$\gu(1)_R^3$  		&	$N_f(\inv{N_f})^3$
	\end{tabular}
	\end{equation}

\medskip\noindent
It is straightforward to check that for $N_f=N_c-3$, each term in the left
column of (\ref{so4-tH}) is equal to the sum of the corresponding one in the
right column and that in (\ref{anomaly-q}).
When $N_f=2$, the number of copies of $F^*$ in the ultraviolet theory
is odd ($N_c=5$).
In the infrared theory, the representation $S^2F^*$ is real, but there is
an additional quark $q$ in the representation $F$.
Therefore the global anomalies match.


\Subhead{{\boldmath $N_f=N_c-2$}}

Classically, a generic expectation value $\bra M\ket$ breaks the gauge group
to the Abelian group $SO(2)\cong U(1)$.
At the singularities, where $\bra M\ket$ is of a lower rank, the unbroken
gauge group is larger and hence there are more massless gauge bosons.
At the quantum level, since no superpotential is generated, the vacuum
degeneracy is not lifted.
In fact when $N_f=N_c-2$, $U(1)_X=U(1)_R$ is anomaly free and 
$M$ is neutral, whereas a superpotential should have $R$-charge $2$.
We expect the classical picture to be valid when $\bra M\ket$ is large.
There the theory is in the Higgs phase.
Modification is needed near the singularities due to strong coupling effects.
In particular, the quantum moduli space may have different singularities.

Over a smooth point $M$ in the moduli space, we have a $U(1)$ gauge theory
whose scale is set by $M$.
By $N=1$ supersymmetry, the effective gauge coupling $\tau\eff(M)$ is 
holomorphic in $M$.
However, due to the periodicity in the (effective) $\theta$-angle and
the electric-magnetic duality, $\tau\eff$ needs not be single-valued in $M$;
it is defined up to fractional linear transformations of $SL(2,\ZZ)$.
At the singularities, $\tau\eff$ is not defined because there are extra
massless fields.
Consequently, there are possible non-trivial $SL(2,\ZZ)$ monodromies around
the singularities.
These information can be summarized by a family of elliptic curves over
the smooth part $\MS$ of the moduli space whose fibre over $M$ has modular
parameter $\tau\eff(M)$.
The $j$-invariant of the family is a well-defined holomorphic function on
$\MS$.
In our situation, the $j$-invariant is a function of $U=\det M$ because of
the (anomaly free) $SU(F)$ symmetry.
We construct a family of elliptic curves over the smooth part of
$\UU=(\medwedge^{N_f}F^*)^{\otimes2}$;
the family over $\MS$ is then the pull back by the determinant map.

We consider the range where $M$ is of the form (\ref{sp-M}):
the large expectation value $\Mperp$ of rank $N_c-3$ breaks the gauge group
to $SO(3)\cong SU(2)/\ZZ_2$, which is further broken to $U(1)$ by $\hM$.
The $SO(3)$ theory in the intermediate stage has a single flavor.
It is the $N=2$ theory which gives rise to Donaldson and Seiberg-Witten
invariants.
Using (\ref{so3-eflat}), the scale $\hLam$ is given by
	\begin{equation}
\hLam^4=4\left(\frac{\Lam^{2N_c-4}}{\det\Mperp}\right)^2.
	\end{equation}
The family of elliptic curve is
	\begin{equation}\label{SWcurve}
y^2=(x-\hLam^2)(x+\hLam^2)(x-u),
	\end{equation}
where the parameter $u$ is related to our theory by $u=\hM/4$.
After making a change of variables $x\mapsto x/4\det\Mperp$,
$y\mapsto y/(4\det\Mperp)^{3/2}$, (\ref{SWcurve}) becomes
	\begin{equation}
y^2=(x-8\Lam^{2N_c-4})(x+8\Lam^{2N_c-4})(x-U).
	\end{equation}
This is a good approximation for large $U$.
When $U$ goes around $\infty$, the monodromy is $PT^{-2}$.
For large $U$, we have
	\begin{equation}\label{approx-curve}
\e{2\pi\ii\tau\eff(M)}=\left(\frac{\Lam}{U^{1/2(N_c-2)}}\right)^{4N_c-8}.
	\end{equation}

The exact family of curves which agrees with (\ref{approx-curve}) for
large $U$ must be of the form
	\begin{equation}
y^2=(x-8\Lam^{2N_c-4})(x+8\Lam^{2N_c-4})(x-U+\al\Lam^{2N_c-4}).
	\end{equation}
The curve is singular at $U=(\al\pm8)\Lam^{2N_c-4}$.
To determine $\al$, we add a mass term $W\tree=\hf\bra M,m\ket$ with 
$\rank m=1$.
Then the vacuum degeneracy is lifted except at two singularities, which 
shall correspond to the two branches $\eps=\pm1$ of the $N_f=N_c-3$ theory.
The $\eps=1$ branch has a superpotential
	\begin{equation}\label{so2-Wm}
\hf M^\perp m^\perp=8\frac{\hLam^{2N_c-3}}{\det\hM}.
	\end{equation}
So $U=16\hLam^{2N_c-3}/m^\perp=16\Lam^{2N_c-4}$.
The $\eps=-1$ branch has $\det\hM=0$, hence $U=0$.
Therefore we have a unique solution $\al=8$.
After making another change of variable $x\mapsto x-8\Lam^{2N_c-4}$, we get
the family of elliptic curves of our theory
	\begin{equation}
y^2=x(x-16\Lam^{2N_c-4})(x-U).
	\end{equation}
The monodromies when $U$ goes around $16\Lam^{2N_c-4}$ and $0$ are
$(T^2S)T(T^2S)^{-1}$ and $ST^2S^{-1}$, 
and the fields that become massless at $U=16\Lam^{2N_c-4}$ and $U=0$
are dyons and magnectic monopoles, respectively.
Unlike the $N=2$ gauge theory, the theory here does not have the $\ZZ_2$
symmetry that relates the two singular components in the modular space.
We will see that the physics in these branches are inequivalent.

Taking $M$ around the surface $\det M=16\Lam^{2N_c-4}$ in $\MM$,
$U$ goes around\break
 $16\Lam^{2N_c-4}$ once, so the monodromy is 
$(T^2S)T(T^2S)^{-1}$.
{}From the one-loop $\beta$-function, we conclude that a pair of dyons
$E^\pm$ of charges $\pm1$ become massless at $U=16\Lam^{2N_c-4}$.
The effective action when $U$ is near $16\Lam^{2N_c-4}$ is of the form
	\begin{equation}
W\sim(U-16\Lam^{2N_c-4})\,E^+\!\cdot E^-,
	\end{equation}
where $\cdot$ is the pairing between fields of opposite $U(1)$ charges.
After adding the mass term $W\tree=\hf\bra M,m\ket$ with $\rank m=1$,
the $M^\perp$ equation of motion implies that $E^+\cdot E^-=-m^\perp/2\det\hM$.
The light fields are in $\hM$ and the superpotential (\ref{so2-Wm})
agrees with the $\eps=1$ branch of $N_f=N_c-3$.
The $U(1)$ gauge bosons becomes heavy due to the expectation values of
$E^\pm$.
But since $E^\pm$ are dyons, this is called oblique confinement.

We check 't Hooft's anomaly condition at a generic point $M_0$ on the moduli 
space.
The light fermions in the infrared theory are the mesons $M$, the photino
$\lam_0$ and, if $\det M=16\Lam^{2N_c-4}$, the dyons $E^\pm$.
When $M_0$ is non-degenerate and is compatible with the Hermitian form on $F$,
the global symmetry is broken to $SO(F)\times U(1)_R$.
The light fermions in the high and low energy theories transform as
	\begin{equation}
	\renewcommand{\arraystretch}{1.3}
	\begin{tabular}{cc|ccc}
&&$SO(F)$ &$\times$ &$U(1)_R$\\
\hline
ultraviolet & $\psi_Q$	&   $F^*$  &&  $-\frac{N_c-2}{N_f}=-1$\\
&$\lam$	&   $\one$	&  &	$1$\\
\hline
infrared   &$\psi_M$& $S^2F^*$ 
& &  $1-2\frac{N_c-2}{N_f}=-1$\\
&$\lam_0$	& $\one$	&&	$1$\\
&($E^\pm$) 	& $\one$	&&	$0$
	\end{tabular}
	\end{equation}

\medskip\noindent
With or without $E^\pm$, the non-zero combinations to be matched are
	\begin{equation}
	\renewcommand{\arraystretch}{1.3}
	\begin{tabular}{c|lll}
& ultraviolet	&& infrared	\\
\hline
$\gso(F)^2\gu(1)_R$ & $N_cc_2(F)\cdot(-1)$ &&   $c_2(S^2F)\cdot(-1)$\\
$\gu(1)_R$ & $N_cN_f\;(-1)+\hf N_c(N_c-1)\;1$ && $\hf N_f(N_f+1)\;(-1)+1$ \\
$\gu(1)_R^3$ &$N_cN_f\;(-1)^3+\hf N_c(N_c-1)\;1^3$ 
&& $\hf N_f(N_f+1)\;(-1)^3+1^3$
	\end{tabular}
	\end{equation}

\medskip\noindent
It is easy to check that the two columns match when $N_f=N_c-2$.
The global anomalies also match because all the representations of $SO(F)$
are of real type here. 
We compare this and the $Sp(N_c)$ theory with $N_f=N_c+1$.
There, the meason $M$ is constrained on the surface (\ref{sp-q}) and
there is no massless photon.
In the low energies, the fermion $\psi_M$ satisfies a traceless conditon.
In the present theory, $M$ can be deformed away from the singular branch
and hence there is no traceless condition on $\psi_M$.

At the other singularity $U=0$, a non-trivial loop around a point $M_0$ with 
$\rank M_0=r$ is mapped one that goes $N_f-r$ times around $0$.
So the monodromy is $ST^{2(N_f-r)}S^{-1}$.
Therefore the number of pairs of monopoles becoming massless at $M_0$
is $N_f-r$.
This suggest a superpotential
	\begin{equation}\label{so2-W0}
W\sim\inv{2\mu}\bra M,\tQ^+\!\cdot\tQ^-\ket,
	\end{equation}
where $\mu$ is of mass dimension $1$ and $\tQ^\pm$ represent magnetic 
monopoles that become massless at the singularity.
After adding a mass term $W=\hf\bra M,m\ket$ with $\rank m=1$,
the fermionic content reduces to the $\eps=-1$ branch of $N_f=N_c-3$.
Again, the expectation values of $\tQ^\perp$ lift the $U(1)$ gauge boson.
Consequently we see confinement in the original description.
The light superfield $q$ in the $N_f=N_c-3$ theory can be constructed by
	\begin{equation}
q=\inv{2\sqrt{m^\perp\mu}}
(\htQ^+\!\cdot(\tQ^-){}^\perp-(\tQ^+){}^\perp\!\cdot\htQ^-).
	\end{equation}

We check 't Hooft's anomaly matching condition at $M=0$.
In addition to $M$, the fermionic component $\psi_{\tQ^\pm}$ of $\tQ^\pm$
and the photino $\lam_0$ transform as
	\begin{equation}
	\renewcommand{\arraystretch}{1.3}
	\begin{tabular}{c|ccc}
&   $SU(F)$	&$\times$&	$U(1)_R$		\\
		\hline
$\psi_{\tQ^\pm}$&   $F$		&	&  $0$		\\
$\lam_0$	& $\one$	&	&  $1$
	\end{tabular}
	\end{equation}

\medskip\noindent
Their contribution to anomaly is
	\begin{equation}\label{so2-tH}
	\renewcommand{\arraystretch}{1.3}
	\begin{tabular}{c|c}
& infrared: additional contribution from $\lam_0$, $\tQ^\pm$\\
\hline
$\gsu(F)^3$		&	$2d_F$	\\
$\gsu(F)^2\gu(1)_R$	&	$0$	\\
$\gu(1)_R$		&	$1$	\\
$\gu(1)_R^3$  		&	$1^3$
	\end{tabular}
	\end{equation}

\medskip\noindent
It is easy to check that the left column of (\ref{so4-tH}) is the sum of
the right column and (\ref{so2-tH}).
The global anomalies also match.

 
\lecture{Problems 5--6}

\catcode`\@=11
%{\let\@makefnmark\relax  \let\@thefnmark\relax
%   \footnote{\bf Solutions by Siye Wu}\addtocounter{footnote}{-1}}
\catcode`\@=13

	\Headnn{Solution to Problem 5}

The supersymmetric gauge theory with gauge group $G=Sp(C)$ and
quarks $Q\in\Hom(F,C)$ is called the electric theory.
When $N_f\ge N_c+3$, we define a dual theory, called the magnetic theory,
with gauge group $\tilde{G}=Sp(\tC)$.
The color space $\tC$ is an Hermitian vector space of dimension
$2\tN=2(N_f-N_c-2)$ equipped with a compatible symplectic form $\tom$.
The flavor space is
	\begin{equation}
\tF=F^*\otimes(\medwedge^{2N_f}F)^{\inv{2(\tN+1)}}.
	\end{equation}
In addition to the dual quarks $\tQ\in\Hom(\tF,\tC)$, the magnetic theory has
elementary mesons $M\in\medwedge^2F^*$.
The fields $\tQ$, $M$, their fermionic components $\psi_{\tQ}$, $\psi_M$
and the gluino $\tilde{\lam}$ of the magnetic theory transform under
the group $SU(F)\times U(1)_R$ according to
	\begin{equation}\label{sp-transD}
	\renewcommand{\arraystretch}{1.3}
	\begin{tabular}{c|ccc}
& $SU(F)$	&$\times$   &	$U(1)_R$			\\
	\hline
$\tQ$	& $F$	&	&   $\frac{N_c+1}{N_f}$		\\
$\psi_{\tQ}$	& $F$		&&   $\frac{N_c+1}{N_f}-1$\\
$M$	  & $\medwedge^2F^*$	&&$2(1-\frac{N_c+1}{N_f})$	\\
$\psi_M$  & $\medwedge^2F^*$	&&$1-2\frac{N_c+1}{N_f}$	\\
$\tilde{\lam}$	& $\one$	&&   $1$
	\end{tabular}
	\end{equation}

\medskip\noindent
The global symmetry $SU(F)\times U(1)_R$ is anomaly free in the dual theory;
this can be checked by a simple calculation
$(\frac{N_c+1}{N_f}-1)\cdot 2N_fc_2(\tC)+c_2(\tilde{G})=0$.
The fields $\tQ$ and $M$ couple through a superpotential
	\begin{equation}\label{sp-WD}
W=\inv{2\mu}\bra M,\tQ^*\tom\ket,
	\end{equation}
where
	\begin{equation}\label{sp-mu}
\mu\in(\medwedge^{2N_f}F^*)^{\inv{\tN+1}}
	\end{equation}
has mass dimension $1$.
The superpotential has $R$-charge $2$ and is invariant under $SU(F)$.
Therefore the dual theory has the same global symmetry $SU(F)\times U(1)_R$.

The $\beta$-functions of the electric and magnetic theories are given by
$b_0=3(N_c+1)-N_f$ and $\tilde{b}_0=3(\tN+1)-N_f=2N_f-3(N_c+1)$, respectively.
The scale $\tLam$ of the dual theory should be related to $\Lam$ and $\mu$.
Because $\Lam$, $\tLam$ and $\mu$ all have mass dimension $1$ and
because of (\ref{sp-mu}), (\ref{sp-Lam}) and its magnetic counterpart,
the only possible relation among them is
	\begin{equation}\label{sp-scaleD}
\Lam^{3(N_c+1)-N_f}\tLam^{3(\tN+1)-N_f}=C(-1)^{\tN+1}\mu^{N_f},
	\end{equation}
where $C$ is a constant to be determined later.

We check that duality is an involution.
In the double dual theory, the number of colors is $N_f-\tN-2=N_c$, and the
flavor space is $\tF^*\otimes(\medwedge^{2N_F}\tF)^{\inv{2(N_c+1)}}\cong F$.
There are new mesons $\tM\in\medwedge^2\tF^*$ and quarks $Q\in\Hom(F,C)$.
The total superpotential is
	\begin{equation}\label{sp-WDD}
W=\inv{2\mu}\bra M,\tM\ket+\inv{2\tilde{\mu}}\bra\tM,Q^*\om_c\ket,
	\end{equation}
where $\tilde{\mu}\in(\medwedge^{2N_f}\tF^*)^{\inv{N_c+1}}\cong
(\medwedge^{2N_f}F^*)^{\inv{\tN+1}}$.
We propose that $\tilde{\mu}=-\mu$.
Then the double dual theory has the same scale $\Lam$ as the electric theory,
and the equation of motion of $\tM$ is $M=Q^*\om_c$.
Therefore we recover the electric theory.

The electric theory is infrared free when $b_0\le0$, i.e., 
when $N_f\ge3(N_c+1)$.
(When $b_0=0$, the two-loop $\beta$-function is positive.)
It is ultraviolet free when the dual theory is infrared free, i.e.,
when $\tilde{b}_0\le0$ or $N_f\le\frac{3}{2}(N_c+1)$.
We claim that for $\frac{3}{2}(N_c+1)<N_f<3(N_c+1)$, the theory flows
to an interacting superconformal field theory in the infrared.
The infrared behavior of the $Sp(N_c)$ theories with $N_f\ge N_c+3$ 
and their duals is summarized in the following table.

\begin{center}
\renewcommand{\arraystretch}{1.3}
\begin{tabular}{c|c|c}
&   electric theory	&     magnetic theory	\\
\hline
$N_c\ge1$, $N_f\ge3(N_c+1)$ &infrared free &strongly coupled\\
($\tN+3\le N_f\le\frac{3}{2}(\tN+1)$) &  &\\
\hline	
$\frac{3}{2}(N_c+1)<N_f<3(N_c+1)$ &    non-trivial &non-trivial	\\
($\frac{3}{2}(\tN+1)<N_f<3(\tN+1)$)& 
infrared fixed point& infrared fixed point\\  
\hline
$N_c+3\le N_f\le\frac{3}{2}(N_c+1)$ & strongly coupled 
  & infrared free\\
($\tN\ge1$, $N_f\ge3(\tN+1)$) &		&
\end{tabular}
\end{center}
\medskip

That the electric and the magnetic theories describe the same low energy 
physics will be established through the following highly nontrivial 
consistency tests.


\Subhead{Chiral operators and the unitarity bound}

Though their field contents seem rather different, the electric and 
the magnetic theories have the same gauge invariant operators.
We construct the mapping of gauge invariant chiral operators.
The composite meson $Q^*\om_c$ of the electric theory corresponds to the
elementary meson $M$ of the magnetic theory, 
whereas the composite meson $\tM=\tQ^*\tom$ of the magnetic theory vanishes
by the equation of motion of $M$.

At the non-trivial infrared fixed-points, the dimensions of the chiral fields
can be calculated by their $R$-charges.
In fact,
	\begin{equation}
D(M)=\frac{3}{2}R(M)=3\left(1-\frac{N_c+1}{N_f}\right).
	\end{equation}
The unitarity bound $D(M)\ge1$ is satisfied if $N_f\ge\frac{3}{2}(N_c+1)$.


\Subhead{'t Hooft anomaly matching condition}

Under $SU(F)\times U(1)_R$, the fermions $\psi_{\tQ}$, $\psi_M$, 
$\tilde{\lam}$ transform as (\ref{sp-transD}).
At $M=0$, all matter fields are massless and the 
whole global symmetry group is unbroken. 
The non-zero contributions to the 't Hooft condition are
	\begin{equation}
	\renewcommand{\arraystretch}{1.3}
	\begin{tabular}{c|c}
&magnetic\\
\hline
$\gsu(F)^3$ &$2\tN\;d_F\!-\!d_{\wedge^2F}$\\
$\gsu(F)^2\gu(1)_R$ &$2\tN\;c_2(F)\cdot(\frac{N_c+1}{N_f}\!-\!1)+
  c_2(\medwedge^2F)\cdot(1\!-\!2\frac{N_c+1}{N_f})$	\\
$\gu(1)_R$ & $4\tN N_f\;(\frac{N_c+1}{N_f}\!-\!1)
+N_f(2N_f\!-\!1)\;(1\!-\!2\frac{N_c+1}{N_f})+\tN(2\tN+1)\;1$\\
$\gu(1)_R^3$  &$4\tN N_f\;(\frac{N_c+1}{N_f}\!-\!1)^3
+N_f(2N_f\!-\!1)\; (1\!-\!2\frac{N_c+1}{N_f})^3+\tN(2\tN+1)\;1^3$
	\end{tabular}
	\end{equation}

\medskip\noindent
Using $d_{\wedge^2F}=2(N_f-2)d_F$ and $c_2(\medwedge^2F)=2(N_f-1)$,
the above matches the left column of (\ref{sp-tH}).


\Subhead{Deformation along the flat directions}

The magnetic theory has the same space of ground states as the electric theory.
In the electric theory, the classical moduli space $\MC$ consists of $M$
satisfying (\ref{sp-constraint}) as a classical constraint.
Since the superpotential is absent, the quantum moduli $\MQ$ space is 
the same as $\MC$; we denote both by $\MM$.
In the magnetic theory, the equations of motion of $M$ set $\tQ=0$. 
The same constraint (\ref{sp-constraint}) are enforced by non-perturbative
quantum effects.
In fact, if $\hf\rank M>N_c$, then the number of flavors in the low energy 
magnetic theory is $N_f-\hf\rank M<\tN+2$.
When $N_f-\hf\rank M\le\tN$, a superpotential of the form (\ref{sp-W}) is
generated, and consequently there is no vacuum.
When $N_f-\hf\rank M=\tN+1$, the light components of $\tM=\tQ^*\tom$ are
constrained by an equation like (\ref{sp-q}) which implies $\tQ\ne0$.
Therefore the superpotential (\ref{sp-WDD}) does not have an extremum
on the constraint surface and there is no vacuum in this case either.
Thus $\hf\rank M\le N_c$ in the low energy magnetic theory.

We show that the procedure of deforming $M$ in $\MM$ commutes with taking
the dual.
Suppose in the electric theory, $M$ has an expectation value
$\bra M\ket$ with $\hf\rank\bra M\ket=r$.
At low energies, the gauge group is broken to $Sp(N_c-r)$,
and there are $N_f-r$ light flavors.
Using the previous notations,
the mass scale $\hLam$ of the effective theory is given by (\ref{sp-eflat}).
The original magnetic theory has $\tN$ colors and $N_f$ flavors,
and the scale $\tLam$ is related to $\Lam$ by (\ref{sp-scaleD}).
Now the expectation value $\bra M\ket$ gives mass to $r$ components of 
the dual quark $\tQ$.
The low energy magnetic theory has $\tN$ colors and $N_f-r$ flavors.
Its scale $\htLam$ is given by
	\begin{equation}\label{sp-mflat}
\htLam^{3(\tN+1)-(N_f-r)}=\Pf(\Mperp/2\mu)\tLam^{3(\tN+1)-N_f}.
	\end{equation}
It follows from (\ref{sp-eflat}) and (\ref{sp-mflat}) that
the relation (\ref{sp-scaleD}) is preserved at low energies.
Thus the low energy magnetic theory is dual to the low energy electric theory.


\Subhead{Mass deformation}

We now check that mass deformation commutes with taking the dual.
Suppose a mass term $\hf\bra M,m\ket$ with $\hf\rank m=r$
is added to the electric theory.
We follow the previous notations.
The low energy theory has quarks $\hQ\in\Hom(\hF,C)$.
Its scale $\hLam$ is given by (\ref{sp-emass}).
When $r\le N_f-N_c-2$, it has a dual description as well.
In the magnetic theory dual to the original electric theory, let 
$\htF=\hF\otimes(\medwedge^{2N_f}F)^{\inv{2(\tN+1)}}$.
The equations of motion of the heavy components of $M$ imply that
	\begin{equation}
\tQ^*\tom=\four{\htM}{}{}{-2\mu m^\perp},
	\end{equation}
where $\htM\in S^2\htF$ contains the light components.
This breaks the $Sp(\tN)$ gauge symmetry to $Sp(\tN-r)$.
Integrating out the heavy components in $M$, $\tQ$, we obtain
a low energy effective superpotential
	\begin{equation}\label{sp-Wmass}
\hat{W}=\inv{2\mu}\bra\hat{M},\hat{\tilde{M}}\ket.
	\end{equation}
This is exactly a magnetic theory of $\tN-r$ flavors and $N_f-r$ quarks.
Its mass scale $\htLam$ is given by the magnetic counterpart of 
(\ref{sp-eflat}), i.e.,
	\begin{equation}\label{sp-mmass}
\tLam^{3(\tN+1)-N_f}=(-\mu)^r\Pf m^\perp\htLam^{3(\tN-r+1)-(N_f-r)}.
	\end{equation}
{}From (\ref{sp-emass}) and (\ref{sp-mmass}),
we see that the relation (\ref{sp-scaleD}) is preserved at low energies.
So the low energy theory of the magnetic theory after mass deformation is
dual to that of the electric theory.

If $r=N_f-N_c-2$, then in the low energy electric theory, the number of 
flavors is $N_f-r=N_c+2$.
In the low energy magnetic theory, the gauge symmetry is completely 
broken since $\tN-r=0$.
Therefore, the instanton contribution should be included in the magnetic
theory just as in the case $N_f=N_c$ for the electric theory.
Now the number of zero modes of $\psi_{\tQ}$ is $2N_f>2\tN$;
these $2(N_c+2)$ extra ones are absorbed by the interaction (\ref{sp-WD}),
generating a term proportional to the $(N_c+2)$-th power
of $(M/2\mu)_{\rm light}$ in the effective superpotential.
So the instanton contribution is
\begin{align}\label{part2}
2^{\tN-1}\frac{\tLam^{3(\tN+1)-N_f}\Pf(M/2\mu)_{\mathrm{light}}}
{\Pf(\tM)_{\mathrm{heavy}}}
&=\frac{(-1)^r\tLam^{3(\tN+1)-N_f}
  \Pf\hM}{2^{N_c+3}\mu^{N_f}\Pf m^\perp}\\
&\null\kern-65pt
=-\frac{C\;\Pf\hM}{2^{N_c+3}\Pf m^\perp\Lam^{3(N_c+1)-N_f}}
=-\frac{C\;\Pf\hM}{2^{N_c+3}\hLam^{2N_c+1}},\nonumber
\end{align}
where (\ref{sp-scaleD}), (\ref{sp-emass}) have been used.
This is exactly (\ref{sp-W+2}) if we choose $C=16$.

	\Headnn{Solution to Problem 6}

Recall that the electric theory has a gauge group $G=SO(C)$
and quarks $Q\in\Hom(F,C)$.
The cases with $N_c\ge4$ and $N_f\le N_c-2$ have been treated earlier.
When $N_f\ge N_c-1$, we define a magnetic theory whose gauge group
is $\tilde{G}=SO(\tC)$.
Here the space of dual colors $\tC$ is a complex vector space of
dimension $\tN=N_f-N_c+4$.
It has a linear metric $\tilde{g}_c$ and a compatible volume form
$\tilde{v}_c$.
The space of flavors is
	\begin{equation}
\tF=F^*\otimes(\medwedge^{N_f}F)^\inv{\tN-2}.
	\end{equation}
The matter fields of the magnetic theory are quarks $\tQ\in\Hom(\tF,\tC)$
and (elementary) mesons $M\in\medwedge^2F^*$.

We first treat the cases $N_c\ge4$, $N_f\ge N_c$.
We concentrate on the identity component $SU(F)\times U(1)_R$ of the global
symmetry.
The fields $\tQ$, $M$, their fermionic components $\psi_{\tQ}$, $\psi_M$
and the gluino $\tilde{\lam}$ transform as
	\begin{equation}\label{so-transD}
	\renewcommand{\arraystretch}{1.3}
	\begin{tabular}{c|ccc}
& $SU(F)$	&$\times$   &	$U(1)_R$		\\
\hline
$\tQ$	& $F$	&	&   $\frac{N_c-2}{N_f}$		\\
$\psi_{\tQ}$  & $F$	&	&   $\frac{N_c-2}{N_f}-1$	\\
$M$	& $S^2F^*$	&	&   $2(1-\frac{N_c-2}{N_f})$	\\
$\psi_M$& $S^2F^*$	&	&   $1-2\frac{N_c-2}{N_f}$	\\
$\tilde{\lam}$	& $\one$	&	&   $1$
	\end{tabular}
	\end{equation}

\medskip\noindent
It is easy to check that $SU(F)\times U(1)_R$ is anomaly free 
in the dual theory.
The fields $\tQ$ and $M$ couple through a superpotential
	\begin{equation}\label{so-WD}
W=\inv{2\mu}\bra M,\tQ^*\tilde{g}_c\ket,
	\end{equation}
where
	\begin{equation}\label{so-mu}
\mu\in(\medwedge^{N_f}F^*)^{\frac{2}{\tN-2}}
	\end{equation}
has mass dimension $1$.
The superpotential has $R$-charge $2$ and is invariant under $SU(F)$.

When $N_f\ge N_c$, the $\beta$-functions of the electric and magnetic theories
are given by $b_0=3(N_c-2)-N_f$ and $\tilde{b}_0=3(\tN-2)-N_f=2N_f-3(N_c-2)$,
respectively.
The scale $\tLam$ of the magnetic theory is given by
	\begin{equation}\label{so-scaleD}
\Lam^{3(N_c-2)-N_f}\tLam^{3(\tN-2)-N_f}=C(-1)^{\tN}\mu^{N_f},
	\end{equation}
where $C$ is a constant to be determined.
It is easy to check that duality is an involution just as the symplectic case.
The electric theory is infrared free when $N_f\ge3(N_c-2)$.
It is ultraviolet free when the dual theory is infrared free, i.e., 
when $N_f\le\frac{3}{2}(N_c-2)$.
We claim that when $\frac{3}{2}(N_c-2)<N_f<3(N_c-2)$, the theory flows
to an interacting superconformal field theory in the infrared.
The infrared behavior of the $SO(N_c)$ theories with $N_f\ge N_c+3$ 
and their duals is summarized in the following table.

\begin{center}
\renewcommand{\arraystretch}{1.3}
\begin{tabular}{c|c|c}
&  electric theory &   magnetic theory	\\
\hline
$N_c\ge4$, $N_f\ge3(N_c-2)$ &infrared free &strongly coupled\\
($\tN\le N_f\le\frac{3}{2}(\tN-2)$) & 	 &	\\
\hline	
$\frac{3}{2}(N_c-2)<N_f<3(N_c-2)$ &non-trivial 	&non-trivial\\
($\frac{3}{2}(\tN-2)<N_f<3(\tN-2)$)& 
infrared fixed point& infrared fixed point\\  
\hline
$N_c+3\le N_f\le\frac{3}{2}(N_c-2)$ & strongly coupled 
  & infrared free\\
($\tN\ge4$, $N_f\ge3(\tN-2)$) &		&
\end{tabular}
\end{center}

\Subhead{Chiral operators and the unitarity bound}

We construct the mapping of gauge invariant chiral operators.
Obviously, the composite mesons $Q^*\om_c$ of 
the electric theory correspond
to the elementary meson $M$ of the magnetic theory.
Recall the baryons $B=Q^*v_c\in\medwedge^{N_c}F^*$ 
in the electric theory.
Since the (fermionic) superfield strength $W_\al$ is valued in
the Lie algebra $\gso(C)\cong\medwedge^2C$, we can define additional
gauge invariant fields
$b=Q^*v_c(W_\al,W^\al,\cdot\,)\in\medwedge^{N_c-4}F^*$
and $w_\al=Q^*v_c(W_\al,\cdot\,)\in\medwedge^{N_c-2}F^*$.
The field $q$ in the $N_f=N_c-3$ theory is proportional to $b$.

In the dual theory, the baryons 
$\tilde{B}\in\medwedge^{\tN}\tF^*\cong\medwedge^{N_f-N_c+4}F\otimes
(\medwedge^{N_f}F^*)\otimes(\medwedge^{N_f}F^*)^{\frac{2}{\tN-2}}$
and 
$\tilde{b}\in\medwedge^{\tN-4}\tF^*\cong\medwedge^{N_f-N_c}F\otimes
(\medwedge^{N_f}F^*)\otimes(\medwedge^{N_f}F^*)^{-\frac{2}{\tN-2}}$,
$\tilde{w}_\al\in\medwedge^{\tN-2}\tF^*
\cong\medwedge^{N_f-N_c+2}F\otimes
(\medwedge^{N_f}F^*)$.
It is easy to check that
$R(B)=R(\tilde{b})=\frac{N_c(N_f-N_c+2)}{N_f}$,
$R(b)=R(\tilde{B})=\frac{(N_c-2)(N_f-N_c+4)}{N_f}$ and
$R(w_\al)=R(\tilde{w}_\al)=1+\frac{(N_c-2)(N_f-N_c+2)}{N_f}$.
So $B$ and $\tilde{b}$, $b$ and $\tilde{B}$, $w_\al$ and 
$\tilde{w}_\al$ have the same quantum numbers, respectively.
The mapping of chiral operators in the electric and magnetic theories
are given by $B\mapsto\tilde{b}$, $b\mapsto\tilde{B}$ and
$w_\al\mapsto\tilde{w}_\al$, upon choosing a trivialization of 
$\medwedge^{N_f}F$.

At non-trivial infrared fixed-points, the dimension of any chiral field is
given by $\PHI$ is $D(\PHI)=\frac{3}{2}R(\PHI)$.
Since $R(Q)=1-\frac{N_c-2}{N_f}$ and $R(W_\al)=1$, we have 
$D(B),D(b),D(w_\al)\ge1$ and $D(M)=3(1-\frac{N_c-2}{N_f})\ge1$ 
if $N_f\ge\frac{3}{2}(N_c-2)$.
The unitarity bounds are satisfied.


\Subhead{'t Hooft anomaly matching condition}

Under the global symmetry, 
the light fermions $\psi_{\tQ}$, $\psi_M$,
$\tilde{\lam}$ transform according to (\ref{so-transD}).
At $M=0$ in $\MQ$, the whole group is unbroken. 
The non-zero combinations in the 't Hooft condition are
	\begin{equation}
	\renewcommand{\arraystretch}{1.3}
	\begin{tabular}{c|c}
&magnetic\\
\hline
$\gsu(F)^3$ &$\tN\;d_F-d_{S^2F}$\\
$\gsu(F)^2\gu(1)_R$ & $\tN\;c_2(F)\cdot(\frac{N_c-2}{N_f}-1)\!+\!
  c_2(S^2F)\cdot(1-2\frac{N_c-2}{N_f})$	\\
$\gu(1)_R$ &$\tN N_f\;(\frac{N_c-2}{N_f}-1)
\!+\!\hf N_f(N_f\!+\!1)\;(1-2\frac{N_c-2}{N_f})\!+\!\hf\tN(\tN-1)\;1$\\
$\gu(1)_R^3$ & $\tN N_f\;(\frac{N_c-2}{N_f}-1)^3
  \!+\!\hf N_f(N_f\!+\!1)\;(1-2\frac{N_c-2}{N_f})^3\!+\!\hf\tN(\tN-1)\;1^3$
	\end{tabular}
	\end{equation}

\medskip\noindent
Using $d_{S^2F}=(N_f+4)d_F$ and $c_2(S^2F)=(N_f+2)c_2(F)$,
the above matches the left column of (\ref{so4-tH}).


\Subhead{Deformation along the flat directions}

First, the dual theory has the same moduli space of vacua.
In the electric theory, the condition $\rank M\le N_c$ is a classical 
constraint.
If $\rank M>N_c$ in the magnetic theory, then the number of light flavors is
$N_f-\rank M<\tN-4$.
So a superpotential of the form (\ref{so-W}) is generated and consequently
the quantum theory has no vacuum.
If $\rank M=N_c$, then according to (\ref{so-constraint}), there are two 
choices of sign of $B=\pm(\wedge^{N_c}M)^{1/2}$ in the electric theory.
In the dual theory, the number of light flavors is $N_f-\rank M=\tN-4$.
So there are two branches.
One branch has a superpotential (\ref{so4-W}) with $\eps_+=\eps_-$
and hence does not have a vacuum.
The other branch has two vacua, labeled by $\eps_+=-\eps_-=\pm1$.
Therefore the magnetic theory also has two vacua for each $M$ with 
$\rank M=N_c$.

We check that flat deformations on the moduli space commutes with taking the
dual.
If $M$ acquires an expectation value $\bra M\ket$ with $\rank \bra M\ket=r$,
then the low energy electric theory has $N_c$ colors and $N_f-r$ flavors.
In the magnetic theory, the expectation value $\bra M\ket$ gives mass to some
of the dual quarks.
The low energy magnetic theory has $\tN-r$ colors and $N_f-r$ flavors.
This is dual to the low energy electric theory.
Further, using (\ref{so-eflat}) and the magnetic analog of (\ref{so-emass}),
one can show that the relation (\ref{so-scaleD}) is preserved in the
low energy theories.


\Subhead{Mass deformation}

If we add a mass term $W\tree=\hf\bra M,m\ket$ with $\rank m=r$ in the
electric theory, then the low energy theory has $N_c$ colors and $N_f-r$
flavors.
Its scale $\hLam$ is given by (\ref{so-emass}).
When $r\le N_f-N_c+1$, this low energy theory has a dual theory.
The latter has $\tN-r$ colors and $N_f-r$ flavors.
Its scale $\htLam$ is given by the magnetic analog of (\ref{so-eflat}).
It is easy to show that the relation (\ref{so-scaleD}) is preserved in
low energy theories and so is the form of the superpotential (\ref{so-WD})
after integrating out the heavy fields.

When $r=N_f-N_c+1$, the number of colors in the low energy magnetic theory is
$\tN-r=3$.
Just as the electric theory when $N_f=N_c-3$, we consider an intermediate
stage in which the gauge group is broken to
$SO(4)\cong(SU(2)_+\times SU(2)_-)/\ZZ_2$.
When the groups $SU(2)_\pm$ are further broken, there are instanton
generated superpotentials.
Now the number of zero modes of $\psi_{\tQ}$ is $2N_f>2(\tN-3)$.
The extra $2(N_c-1)$ ones are absorbed by the interaction (\ref{so-WD}),
generating a term proportional to the $(N_c-1)$-th power of 
$(\mu^{-1}M)_\mathrm{light}$.
Thus the superpotential contains an extra term
\begin{align}
4\frac{\tLam^{3(\tN-2)-N_f}\det(\mu^{-1}M)_{\mathrm{light}}}
{(\det\tM)_{\mathrm{heavy}}}
&=\frac{4(-1)^r\tLam^{3(\tN-2)-N_f}\det\hM}{\mu^{N_f}\det m^\perp}\\
&\null\kern-25pt
=-\frac{4C\,\det\hM}{\det m^\perp\Lam^{3(N_c-2)-N_f}}
=-\frac{4C\,\det\hM}{\hLam^{2N_c-5}}.\nonumber
\end{align}
Here we have used (\ref{so-scaleD}) and (\ref{so-emass}).
So the total superpotential in the low energy magnetic theory is
	\begin{equation}\label{so3m-WD}
\hat{\tilde{W}}=\inv{2\mu}\bra\hM,\htQ^*\tilde{g}_c\ket
-\frac{4C\det\hM}{\hLam^{2N_c-5}}.
	\end{equation}

Finally, we derive the relation between the scales of the low energy theories.
The scale $\hLam$ of the low energy electric theory is given by 
(\ref{so-emass}).
The scale $\htLam$ of the low energy magnetic theory
is given by the magnetic analog of (\ref{so3-eflat}), i.e.,
	\begin{equation}\label{so3-mmass}
4(\tLam^{3(\tN-2)-N_f})^2=\det(-\mu m^\perp)^2\htLam^{6-2(N_f-\tN+3)}.
	\end{equation}
Using (\ref{so-emass}) and (\ref{so3-mmass}), the relation (\ref{so-scaleD})
reduces at low energies to
	\begin{equation}\label{so3m-scaleD}
(\hLam^{2N_c-5})^2\htLam^{6-2(N_c-1)}=4C^2\mu^{2N_c-2}.
	\end{equation}


\Subhead{{\boldmath $N_f=N_c-1$}}

We consider the theory with $N_f=N_c-1$ and scale $\Lam$.
The electric theory is strongly coupled at low energies.
{}From (\ref{so3m-WD}), the superpotential of the magnetic theory is
	\begin{equation}\label{so3-WD}
W=\inv{2\mu}\bra M,\tQ^*\tilde{g}_c\ket-\frac{4C\det M}{\Lam^{2N_c-5}}.
	\end{equation}
{}From (\ref{so3m-scaleD}), the scale of the dual theory is given by
	\begin{equation}\label{so3-scaleD}
(\Lam^{2N_c-5})^2\tLam^{6-2(N_c-1)}=4C^2\mu^{2N_c-2}.
	\end{equation}
In fact, up to a constant, (\ref{so3-scaleD}) is the only possible relation
consistent with (\ref{so-mu}), (\ref{so-scale}), the magnetic analog of 
(\ref{so3-scale}), and the fact that $\Lam$, $\tLam$, $\mu$ all have
mass dimension $1$.
Under a flat deformation, (\ref{so3-scaleD}) is satisfied by the low
energy electric and magnetic theories.

We show that adding the mass term $W\tree=\hf\bra M,m\ket$ with $\rank m=1$
reduces the theory to $N_f=N_c-2$.
The total superpotential is $W+W\tree$.
Following previous notations, we have $\tF=\htF\oplus\htF^\perp$
with $\dim\htF^\perp=1$.
The dual quarks have the decomposition
$\tQ=(\htQ,\tQ^\perp)\in\Hom(\htF,\tC)\oplus\Hom(\htF^\perp,\tC)$.
The $M^\perp$ equations of motion imply
	\begin{equation}
(\tQ^\perp){}^*\tilde{g}_c=\frac{8C\mu\det\hM}{\Lam^{2N_c-5}}-\mu m^\perp.
	\end{equation}
In particular, $\im(\htQ^\perp)$ is not isotropic in $\tC$.
We write $\tC=\htC\oplus\htC^\perp$, where $\htC^\perp=\im(\htQ^\perp)$.
The value of $\tQ^\perp$ breaks the dual gauge group $SO(\tC)$ to
$SO(\htC)\cong SO(2)$.
The equations of motion imply that $M$ is block-diagonal and that 
$\tilde{g}_c(\htQ,\htQ^\perp)=0$, i.e., $\htQ\in\Hom(\htF,\htC)$.
Let $\htC=\htC^+\oplus\htC^-$, where the weights of $SO(\htC)$ in $\htC^\pm$
are $\pm1$, respectively.
We write $\htQ=(\htQ^+,\htQ^-)$, where $\htQ^\pm\in\Hom(\htF,\htC^\pm)$.
Then the superpotential $W+W\tree$ reduces in the low energies to
	\begin{equation}
\hat{W}\sim\inv{2\mu}\bra\hM,\htQ^+\!\cdot\htQ^-\ket,
	\end{equation}
where $\cdot$ is the pairing between $\htC^+$ and $\htC^-$.
This matches (\ref{so2-W0}).
Due to the instanton effects in symmetry breaking,
it is only an approximate result near $\hM=0$.
When $\det\hM$ is large, the quarks $\htQ$ become heavy.
After integrating out the heavy fields, the effective superpotential is
	\begin{equation}\label{so3-Wm}
W=\inv{2\mu}M^\perp
\left(4u-\frac{8C\mu\det\hM}{\Lam^{2N_c-5}}+\mu m^\perp\right),
	\end{equation}
where $u=\tilde{g}_c(\tQ^\perp,\tQ^\perp)/4$.
The effective theory, which is the $SO(\tC)\cong SO(3)$ gauge theory with
a single flavor $\tQ^\perp$, has $N=2$ supersymmetry.
Its scale $\tLam'$ is given by
	\begin{equation}
\tLam'{}^4=\det(\mu^{-1}\hM)^2\tLam^{6-2(N_c-1)}
=\left(\frac{2C\mu\det\hM}{\Lam^{2N_c-5}}\right)^2.
	\end{equation}
The superpotential (\ref{so3-Wm}) lifts the vacuum degeneracy of the $u$-plane
except at the two singularities
$u=\pm\tLam'{}^2=\pm8C\mu\det\hM/\Lam^{2N_c-5}$.
There is still no vacuum with the plus sign.
With the minus sign, there is a vacuum if $\hM$ satisfies
	\begin{equation}
\det\hM=\frac{m^\perp\Lam^{2N_c-5}}{16C}=\frac{\hLam^{2N_c-4}}{16C}.
	\end{equation}
This matches the non-zero singular branch of the $N_f=N_c-2$ theory
when $C=2^{-8}$.

\begin{thebibliography}{100}

	\item
I.\ Affleck, M.\ Dine and N.\ Seiberg, 
{\em Dynamical supersymmetric breaking in supersymmetric QCD},
Nucl.\ Phys.\ B241 (1984) 493-534.
	\item
N.\ Seiberg, {\em Exact results on the space of vacua of 
four-dimensional SUSY gauge theories},
Phys.\ Rev.\ D49 (1994) 6857-6863.
	\item
N.\ Seiberg and E.\ Witten, {\em Electric-magnetic duality, monopole
condensation, and confinement in $N=2$ supersymmetric Yang-Mills theory},
Nucl.\ Phys.\ B 431 (1994) 484-550.
	\item
N.\ Seiberg and E.\ Witten, {\em Monopoles, duality and chiral symmetry 
breaking in $N=2$ supersymmetric QCD}, Nucl.\ Phys.\ B 431 (1994) 484-550.
        \item
K.\ Intriligator and N.\ Seiberg,
{\em Phases of $N=1$ supersymmetric gauge theories in four dimensions},
Nucl.\ Phys.\ B 431 (1994) 551-565.
	\item
N.\ Seiberg,
{\em Electric-magnetic duality in supersymmetric non-Abelian gauge theories},
Nucl.\ Phys.\ B435 (1995) 129-146.
        \item
K.\ Intriligator and N.\ Seiberg,
{\em Duality, monopoles, dyons, confinement and oblique confinement in
supersymmetric $SO(N_c)$ gauge theories},
Nucl.\ Phys.\ B 444 (1995) 125-160.
	\item
K.\ Intriligator and P.\ Pouliot,
{\em Exact superpotential, quantum vacua and duality in supersymmetric 
$SP(N_c)$ gauge theories},
Phys.\ Lett.\ B 353 (1995) 471-476.
	\item
K.\ Intriligator and N.\ Seiberg, 
{\em Lectures on supersymmetric theories and electric-magnetic duality},
in: String theory, gauge theory and quantum gravity, Proc. of the Trieste 
Spring School and Workshop, Eds.\ R.\ Dijkgraaf et al.,
Nucl.\ Phys.\ B (Proc.\ Suppl.) 45B, C (1996) 1-28.
	\item
M.\ E.\ Peskin,
{\em Duality in supersymmetric Yang-Mills theory}, in:
Fields, strings and duality, Proceedings of the 1996 TASI (Boulder, CO),
pp.\ 729-809, Eds.\ C.\ Efthimiou and B.\ Greene, World Sci.\ Publishing,
River Edge, NJ, 1997.
	\end{thebibliography}

	\end{document}

