%% This is a plain TeX file
%%
\magnification=1200
\hsize=6.5 true in
\vsize=8.7 true in
\input epsf.tex

\input amssym.def
\input amssym.tex
%\input eight.tex
% beginning of file eight.tex

\font\eightrm=cmr8
\font\sevenrm=cmr7
\font\sixrm=cmr6
\font\fiverm=cmr5
\font\eighti=cmmi8
\font\sixi=cmmi6
\font\fivei=cmmi5
\font\eightsy=cmsy8
\font\sixsy=cmsy6
\font\fivesy=cmsy5
\font\tenex=cmex10
\font\eightit=cmti8
\font\eightsl=cmsl8
\font\eighttt=cmtt8
\font\eightbf=cmbx8
\font\sixbf=cmbx6
\font\fivebf=cmbx5
% Cindy's attempt:
\font\eightmsb=msbm8
\font\sixmsb=msbm6

\def\eightpoint{\def\rm{\fam0\eightrm}% switch to 8-point type
  \textfont0=\eightrm \scriptfont0=\sixrm \scriptscriptfont0=\fiverm
  \textfont1=\eighti \scriptfont1=\sixi \scriptscriptfont1=\fivei
  \textfont2=\eightsy \scriptfont2=\sixsy \scriptscriptfont2=\fivesy
  \textfont3=\tenex \scriptfont3=\tenex \scriptscriptfont3=\tenex
  \textfont\itfam=\eightit  \def\it{\fam\itfam\eightit}%
  \textfont\slfam=\eightsl  \def\sl{\fam\slfam\eightsl}%
  \textfont\ttfam=\eighttt  \def\tt{\fam\ttfam\eighttt}%
  \textfont\bffam=\eightbf  \scriptfont\bffam=\sixbf
   \scriptscriptfont\bffam=\fivebf  \def\bf{\fam\bffam\eightbf}%
% Cindy's attempt:%
%\textfont\msbfam=\eightmsb%
%\scriptfont\msbfam=\sixmsb%
%  \textfont4=\eightmsb%         %=\msbfam =\eightmsb%
%  \scriptfont4=\sixmsb%         %=\msbfam=\sixmsb%
%  \textfont\msbfam=\eightmsb%
%  \scriptfont\msbfam=\sixmsb%
%  \def\msb{\fam\msbfam\eightmsb}%
% end of Cindy's attempt%
  \normalbaselineskip=9pt
  \setbox\strutbox=\hbox{\vrule height7pt depth2pt width0pt}%
  \let\sc=\sixrm  \normalbaselines\rm}

% end of file eight.tex

\font\dotless=cmr10 %for the roman i or j to be
                    %used with accents on top.
                    %(\dotless\char'020=i)
                    %(\dotless\char'021=j)
\font\itdotless=cmti10
\def\itumi{{\"{\itdotless\char'020}}}
\def\itumj{{\"{\itdotless\char'021}}}
\def\umi{{\"{\dotless\char'020}}}
\def\umj{{\"{\dotless\char'021}}}
\font\smaller=cmr5
\font\boldtitlefont=cmb10 scaled\magstep2
\font\smallboldtitle=cmb10 scaled \magstep1
\font\ninerm=cmr9
\font\dun=cmdunh10 %scaled\magstep1
\font\Rfont=cmss10

\footline={\hfil {\tenrm IX.\folio}\hfil}

\def\eps{{\varepsilon}}
\def\Eps{{\epsilon}}
\def\kap{{\kappa}}
\def\lam{{\lambda}}
\def\Lam{{\Lambda}}
\def\mynabla{{\nabla\!}}

\def\underNS{\underline{\NS}}
\def\underR{\underline{\R}}

\def\Bmu{{B_{\mu\nu}}}
\def\Gmu{{G_{\mu\nu}}}

\def\xdot{{\dot x}}
\def\xddot{{\ddot x}}

\def\undertext#1{$\underline{\vphantom{y}\hbox{#1}}$}
\def\nspace{\lineskip=1pt\baselineskip=12pt%
     \lineskiplimit=0pt}
\def\dspace{\lineskip=2pt\baselineskip=18pt%
     \lineskiplimit=0pt}

\def\Half{\raise4.5pt\hbox{{\vtop{\ialign{##\crcr
\hfil\rm ${\scriptstyle 1}$\hfil\crcr
  \noalign{\nointerlineskip\vskip1.5pt}%
   \hbox to 4pt{\hrulefill}\crcr
\noalign{\nointerlineskip\vskip1.5pt}%
   ${\scriptstyle 2}$\crcr}}}}}
%\def\Half{{{\scriptstyle 1}\over{\scriptstyle 2}}}

\def\half{\raise4.5pt\hbox{{\vtop{\ialign{##\crcr
  \hfil\rm $1$\hfil\crcr
   \noalign{\nointerlineskip}--\crcr
   \noalign{\nointerlineskip\vskip-1pt}$2$\crcr}}}}}
\def\third{\raise4.5pt\hbox{{\vtop{\ialign{##\crcr
  \hfil\rm $1$\hfil\crcr
  \noalign{\nointerlineskip}--\crcr
  \noalign{\nointerlineskip\vskip-1pt}$3$\crcr}}}}}
\def\fourth{\raise4.5pt\hbox{{\vtop{\ialign{##\crcr
  \hfil\rm $1$\hfil\crcr
  \noalign{\nointerlineskip}--\crcr
  \noalign{\nointerlineskip\vskip-1pt}$4$\crcr}}}}}
\def\sixth{\raise4.5pt\hbox{{\vtop{\ialign{##\crcr
  \hfil\rm $1$\hfil\crcr
  \noalign{\nointerlineskip}--\crcr
  \noalign{\nointerlineskip\vskip-1pt}$6$\crcr}}}}}
\def\eighth{\raise4.5pt\hbox{{\vtop{\ialign{##\crcr
  \hfil\rm $1$\hfil\crcr
  \noalign{\nointerlineskip}--\crcr
  \noalign{\nointerlineskip\vskip-1pt}$8$\crcr}}}}}

\def\oplusop{\mathop{\oplus}\limits}
\def\w{{\mathchoice{\,{\scriptstyle\wedge}\,}
  {{\scriptstyle\wedge}}
  {{\scriptscriptstyle\wedge}}{{\scriptscriptstyle\wedge}}}}
\def\Le{{\mathchoice{\,{\scriptstyle\le}\,}
{\,{\scriptstyle\le}\,}
{\,{\scriptscriptstyle\le}\,}{\,{\scriptscriptstyle\le}\,}}}
\def\Ge{{\mathchoice{\,{\scriptstyle\ge}\,}
{\,{\scriptstyle\ge}\,}
{\,{\scriptscriptstyle\ge}\,}{\,{\scriptscriptstyle\ge}\,}}}
\def\plus{{\hbox{$\scriptscriptstyle +$}}}
\def\xdot{\dot{x}}
\def\Condition#1{\item{#1}}
\def\Firstcondition#1{\hangindent\parindent{#1}\enspace
     \ignorespaces}
\def\Proclaim#1{\medbreak
  \medskip\noindent{\bf#1\enspace}\it\ignorespaces}
  %the way to use this is:
  %"\Proclaim{Theorem 1.1.}" for instance.
\def\finishproclaim{\par\rm
     \ifdim\lastskip<\smallskipamount\removelastskip
     \penalty55\medskip\fi}
\def\Item#1{\par\smallskip\hang\indent%
  \llap{\hbox to\parindent {#1\hfill\enspace}}\ignorespaces}
\def\ItemItem#1{\par\indent\hangindent2\parindent
     \hbox to \parindent{#1\hfill\enspace}\ignorespaces}
\def\vrulesub#1{{\,\vrule height7pt depth5pt}_{\,#1}}
\def\underbrake#1#2{\mathop{#1}\limits_{\raise3pt
  \hbox{%
\vrule height 3pt depth 0pt
  %\kern.1pt
  \hbox to #2{\hrulefill}
  \kern-3.4pt
  \vrule height 3pt depth 0pt}}}
\def\notD{{\slash\kern-7pt D}}

\def\ominus{{$-$\kern-9pt $\bigcirc$}}
\def\Oplus{{+\kern-9pt $\bigcirc$}}
\def\ssbullet{{\scriptstyle\bullet\,\,\,}}

\def\im{{\rm Im}}  
\def\A{{\rm A}}
\def\P{{\rm P}}
\def\EL{{\rm L}}
\def\Open{{\rm open}}
\def\osc{{\rm osc}}
%\def\Pic{{\rm Pic}} 
\def\Sp{{\rm Sp}}
\def\R{{\rm R}}  \def\NS{{\rm NS}}
\def\RNS{{\rm RNS}}
\def\Diff{{\rm Diff}}  \def\expt{{\rm expt}}
\def\cylinder{{\rm cylinder}}
\def\Closed{{\rm closed}}
\def\Map{{\rm Map}}  
%\def\spurious{{\rm spurious}}
\def\Met{{\rm Met}} 
\def\Spin{{\rm Spin}}
\def\spin{{\rm spin}} 
\def\phys{{\rm phys}}
\def\diag{{\rm diag}}  
%\def\Vir{{\rm Vir}}
%\def\Res{{\rm Res}}
\def\Null{{\rm null}} 
\def\mass{{\rm mass}}
\def\SO{{\rm SO}} 
\def\GSO{{\rm GSO}}
\def\Tr{{\rm Tr\,}}
%\def\SU{{\rm SU}} 
\def\tr{{\rm tr}}
\def\Weyl{{\rm Weyl}} 
\def\Lorentz{{\rm Lorentz}}
\def\Ker{{\rm Ker}}
\def\Range{{\rm Range}} 
%\def\SL{{\rm SL}} \def\prim{{\rm primitive}}
\def\Det{{\rm Det}} 
\def\re{{\rm Re}}
%\def\dist{{\rm dist}} \def\PSL{{\rm PSL}}
\def\Vol{{\rm Vol}} 
%\def\ghosts{{\rm ghosts}}
%\def\Fock{{\rm Fock}} 
\def\BRST{{\rm BRST}}
%\def\Weil-Peterson{{\rm Weil-Peterson}}
\def\IIA{{\rm II~A}}
\def\IIB{{\rm II~B}}

\def\1bar{\bar{1}}
\def\2bar{\bar{2}}
\def\3bar{\bar{3}}
\def\4bar{\bar{4}}
%\def\fbar{\bar{f}}  
\def\mubar{\bar{\mu}}
\def\nubar{{\bar{\nu}}}
\def\taubar{{\bar{\tau}}}
%\def\barh{\bar{h}}  
\def\gammabar{\bar{\gamma}}
\def\betabar{\bar{\beta}}
%\def\kbar{\bar{k}}   
\def\psibar{\bar{\psi}}
\def\lambar{\bar{\lambda}} \def\xibar{{\bar{\xi}}}
\def\mbar{\bar{m}}   \def\Lambar{\bar{\Lambda}}
\def\thetabar{{\bar{\theta}}}
\def\zetabar{{\bar{\zeta}}}
%\def\phibar{\bar{\phi}}
%\def\wbar{\bar{w}}  \def\etabar{\bar{\eta}}
%\def\vbar{\bar{v}}  \def\partialbar{\bar{\partial}}
%\def\xbar{\bar{x}}  
\def\cbar{\bar{c}}
\def\bbar{\bar{b}}
%\def\epsbar{\bar{\epsilon}}
\def\Gambar{\bar{\Gamma}}
\def\rbar{\bar{r}}
\def\zbar{\bar{z}}  
%\def\Abar{\bar{A}}
\def\Bbar{{\bar{B}}}  \def\Cbar{{\bar{C}}}
\def\Fbar{{\bar{F}}}
\def\Gbar{\bar{G}}
\def\Kbar{\bar{K}}
\def\Mbar{\bar{M}}
%\def\Pbar{\bar{P}}
\def\Sbar{\bar{S}}
\def\Tbar{\bar{T}}
\def\scrFbar{\bar{\scrF}}
\def\scrCbar{\bar{\scrC}}

%\def\scrFbc{{\scrF^{(bc)}}}
%\def\scrFbcbar{{\scrF^{(\bbar\cbar)}}}

\def\chat{\hat{c}}
%\def\ghat{\hat{g}}
%\def\muhat{\hat{\mu}}
\def\Rhat{{\widehat{R}}}
\def\scrDhat{{\widehat{\scrD}}}

\def\Atil{\tilde{A}}
\def\Btil{\widetilde{B}}
\def\atil{\tilde{a}}
\def\btil{\tilde{b}}
\def\dtil{\tilde{d}}
\def\xtil{\tilde{x}}
%\def\htil{\tilde{h}}
%\def\Ctil{\widetilde{C}}
%\def\Dtil{\widetilde{D}}
\def\Ltil{\tilde{L}}
\def\Ntil{\widetilde{N}}
\def\Ptil{\widetilde{P}}
\def\Wtil{\widetilde{W}}
\def\betatil{\tilde{\beta}}
%\def\Ttil{\widetilde{T}}
\def\scrFtil{\widetilde{\scrF}}
%\def\epstil{\tilde{\eps}}
%\def\psitil{\tilde{\psi}}

\def\dbR{{\Bbb R}}
%\def\dbZ{{\Bbb Z}}

%These two files (in this order!!) are necessary
%in order to use AMS Fonts 2.0 with Plain TeX

\input amssym.def
\input amssym.tex

%Capital roman double letters(Blackboard bold)
\def\db#1{{\fam\msbfam\relax#1}}

\def\dbA{{\db A}} \def\dbB{{\db B}}
\def\dbC{{\db C}} \def\dbD{{\db D}}
\def\dbE{{\db E}} \def\dbF{{\db F}}
\def\dbG{{\db G}} \def\dbH{{\db H}}
\def\dbI{{\db I}} \def\dbJ{{\db J}}
\def\dbK{{\db K}} \def\dbL{{\db L}}
\def\dbM{{\db M}} \def\dbN{{\db N}}
\def\dbO{{\db O}} \def\dbP{{\db P}}
\def\dbQ{{\db Q}} \def\dbR{{\db R}}
\def\dbS{{\db S}} \def\dbT{{\db T}}
\def\dbU{{\db U}} \def\dbV{{\db V}}
\def\dbW{{\db W}} \def\dbX{{\db X}}
\def\dbY{{\db Y}} \def\dbZ{{\db Z}}

\font\teneusm=eusm10  \font\seveneusm=eusm7 
\font\fiveeusm=eusm5 
\newfam\eusmfam 
\textfont\eusmfam=\teneusm 
\scriptfont\eusmfam=\seveneusm 
\scriptscriptfont\eusmfam=\fiveeufm 
\def\scr#1{{\fam\eusmfam\relax#1}}


%Upper-case Script Letters:

\def\scrA{{\scr A}}   \def\scrB{{\scr B}}
\def\scrC{{\scr C}}   \def\scrD{{\scr D}}
\def\scrE{{\scr E}}   \def\scrF{{\scr F}}
\def\scrG{{\scr G}}   \def\scrH{{\scr H}}
\def\scrI{{\scr I}}   \def\scrJ{{\scr J}}
\def\scrK{{\scr K}}   \def\scrL{{\scr L}}
\def\scrM{{\scr M}}   \def\scrN{{\scr N}}
\def\scrO{{\scr O}}   \def\scrP{{\scr P}}
\def\scrQ{{\scr Q}}   \def\scrR{{\scr R}}
\def\scrS{{\scr S}}   \def\scrT{{\scr T}}
\def\scrU{{\scr U}}   \def\scrV{{\scr V}}
\def\scrW{{\scr W}}   \def\scrX{{\scr X}}
\def\scrY{{\scr Y}}   \def\scrZ{{\scr Z}}

\def\gr#1{{\fam\eufmfam\relax#1}}

%Euler Fraktur letters (German)
\def\grA{{\gr A}}	\def\gra{{\gr a}}
\def\grB{{\gr B}}	\def\grb{{\gr b}}
\def\grC{{\gr C}}	\def\grc{{\gr c}}
\def\grD{{\gr D}}	\def\grd{{\gr d}}
\def\grE{{\gr E}}	\def\gre{{\gr e}}
\def\grF{{\gr F}}	\def\grf{{\gr f}}
\def\grG{{\gr G}}	\def\grg{{\gr g}}
\def\grH{{\gr H}}	\def\grh{{\gr h}}
\def\grI{{\gr I}}	\def\gri{{\gr i}}
\def\grJ{{\gr J}}	\def\grj{{\gr j}}
\def\grK{{\gr K}}	\def\grk{{\gr k}}
\def\grL{{\gr L}}	\def\grl{{\gr l}}
\def\grM{{\gr M}}	\def\grm{{\gr m}}
\def\grN{{\gr N}}	\def\grn{{\gr n}}
\def\grO{{\gr O}}	\def\gro{{\gr o}}
\def\grP{{\gr P}}	\def\grp{{\gr p}}
\def\grQ{{\gr Q}}	\def\grq{{\gr q}}
\def\grR{{\gr R}}	\def\grr{{\gr r}}
\def\grS{{\gr S}}	\def\grs{{\gr s}}
\def\grT{{\gr T}}	\def\grt{{\gr t}}
\def\grU{{\gr U}}	\def\gru{{\gr u}}
\def\grV{{\gr V}}	\def\grv{{\gr v}}
\def\grW{{\gr W}}	\def\grw{{\gr w}}
\def\grX{{\gr X}}	\def\grx{{\gr x}}
\def\grY{{\gr Y}}	\def\gry{{\gr y}}
\def\grZ{{\gr Z}}	\def\grz{{\gr z}}

%\overfullrule=0pt

\parindent=25pt
\line{\dun --- DRAFT --- \hfill{\rm IASSNS-HEP-97/72}}

\bigskip\bigskip
\centerline{\boldtitlefont Lecture 14}
\bigskip
\centerline{\smallboldtitle IX. Superstring 
Perturbation Theory}

\medskip
\centerline{Eric D'Hoker}

\frenchspacing

%$$
%\vbox{\epsfxsize=2.5in\epsfbox{fig1.eps}}
%$$
\dspace
\bigskip
We shall now develop a systematic treatment of
superstring perturbation theory in the $\RNS$
formulation.
Local $N=1$ supersymmetry (SUSY) on the worldsheet plays a
fundamental r\^{o}le, but this symmetry is not manifest
in the component formalism, set up in \S{VII}.
The $N=1$ superfield formalism, formulated in the
language of $N=1$
supergeometry, treats diffeomorphisms and local
supersymmetry transformations on an equal footing, and
renders both manifest.
Their combined action makes up the group of
super-diffeomorphisms of the worldsheet super-manifold, 
and local supersymmetry invariants can be
constructed from super-diffeomorphism tensors.
We first present the different ingredients needed and
then apply these methods to the $\RNS$ superstrings.
A general reference for this subject is E. D'Hhoker and
D. H. Phong, ``The Geometry of String Perturbation
Theory'', Rev. Mod. Phys. 60 (1988), p. 917.

\bigskip\noindent
\Item{\bf A)} {\bf $N=1$ supergeometry {\rm (also
called $N=1$, $D=2$ supergravity)}}

We consider a supermanifold $\Sigma$ of dimension
$(2\vert2)$, and use local coordinates
$\xi^M=(\xi^m,\theta^\mu)$, $m=1,2$, $\mu=1,2$
with the grading
$$
\xi^M\xi^N=(-)^{MN}\xi^N\xi^M
\eqno{(9.A.1)}
$$
where $(-)^{MN}$ equals $1$ unless $M=\mu$, $N=\nu$, in
which case it equals $-1$.
We introduce a local orthonormal {\it frame}
$$
E^A\equiv d\xi^M E_M{}^A\qquad\qquad
A=(a,\alpha)\quad a=z,\zbar;\,\,\alpha=+,-\,\,,
\eqno{(9.A.2)}
$$
where 
it is understood that $a$ has the same grading as $m$, 
while $\alpha$ has the same grading as $\mu$.
We denote the inverse frame by $E_A{}^M$, and we have
$E_A{}^M\,E_M{}^B=\delta_A{}^B$ and $E_M{}^A\,E_A{}^N=
\delta_M{}^N$.

Next, we consider frame rotations by a group of $U(1)$
gauge transformations, under which the fields $E^z$,
$E^{\zbar}$, $E^+$, $E^-$ have weights $-1$, $+1$,
$-\Half$, $+\Half$, respectively.
A general superfield $V$ has $U(1)$ weight $-n$ if it
transforms as $(E^z)^{\otimes n}$.
We introduce a $U(1)$ gauge field or {\it connection}
$$
\Omega=d\xi^M\Omega_M
\eqno{(9.A.3)}
$$
with associated covariant derivatives
$$
\scrD_A^{(n)}V\equiv
E_A{}^M(\partial_M+in\Omega_M)V\,\,.
\eqno{(9.A.4)}
$$
Here $\partial_M\equiv\partial/\partial\xi^M$ and
$i=\sqrt{-1\,\,}$.
Torsion $T_{AB}{}^C$ and curvature $R_{AB}$ superfield
tensors are defined by the graded commutators
$$
[\scrD_A,\scrD_B]V=T_{AB}{}^C\scrD_C V+in\,R_{AB}V\,\,.
\eqno{(9.A.5)}
$$

The data of $N=1$ supergeometry is the frame $E{}^A$ and
the $U(1)$ connection $\Omega$, together with the following
{\it constraints} on the torsion and curvature, arising
in $[\scrD_\alpha,\scrD_\beta]$:
$$
T_{\alpha\beta}^\gamma=0;\quad
T_{\alpha\beta}^c=2\gamma_{\alpha\beta}^c;\quad
R_{++}=R_{--}=0
\eqno{(9.A.6)}
$$
Here, $\gamma^c$ are the $2$-dimensional Dirac
matrices, satisfying
$\{\gamma^a,\gamma^b\}=-\eta^{ab}$, with a convenient
representation given by 
$\gamma_{++}^z=\gamma_{--}^{\zbar}=1$,
all other components $=0$.
The constraints of (9.A.6) are equivalent to the following
set of constraints
$$
\eqalign{
&\scrD_+^2=\scrD_z,\quad
\scrD_-^2=\scrD_{\zbar}\cr
&\{\scrD_+,\scrD_-\}V=in\,R_{+-}V\,\,,\cr}
\eqno{(9.A.7)}
$$
where $V$ is any superfield of weight $n$.
The structure relations of the remaining covariant
derivatives are defined by using the (graded) Jacobi
identities.
The only free quantity in this supergeometry is the {\it
curvature superfield} $R_{+-}=R_{-+}\equiv R$.

The symmetry transformations on the data $E^A$ and
$\Omega$ that preserve the constraints (9.A.6) or
equivalently (9.A.7) are

\vfill\eject

\parindent=75pt
\Item{1) $s\Diff(\Sigma)$:} super-diffeomorphisms
$\xi^M\to\xi'{^{M}}(\xi^N)$

\Item{} contains $\Diff(\Sigma)$, local $N=1$
susy, and further {\it algebraic} transformatons.

\Item{2) a) $s\Weyl(\Sigma)$:} generated by a
superfield $\Sigma$, contains $\Weyl(\Sigma)$ and further
algebraic transformations.

\Item{\quad b) $sU(1)$:} generated by a superfield
$L$, contains $U(1)$ gauge transformations and further
algebraic transformations.

\parindent=25pt
\smallskip
It is convenient to combine transformations
a) and b) into complexified Weyl
transformations, given in terms of a complex superfield
$\Lam$, defined by
$$
\Lam\equiv\Sigma-iL,\quad \Lam^*\equiv\Sigma+iL\,\,.
\eqno{(9.A.8)}
$$
The superderivatives transform  under $s\Weyl\times sU(1)$ as
$$
\eqalign{
\scrD_+^{(n)} &=e^{n\Lam-\Half\Lam^*}\scrDhat_+^{(n)}
  e^{-n\Lam}\cr
\scrD_-^{(n)} &=e^{-n\Lam^*-\Half\Lam}
  \scrDhat_-^{(n)}e^{n\Lam^*}\cr}
\eqno{(9.A.9)}
$$
and the curvature transforms as 
$$
R_{+-}=e^{-\Sigma}\left(\Rhat_{+-}-2i\scrDhat_+\scrDhat_-
\Sigma\right)\,\,.
\eqno{(9.A.10)}
$$
From (9.A.9), the transformation rules for $E_{\pm}$ and 
$\Omega_{\pm}$ may be deduced, and from these combined
with the torsion constraints (9.A.6), the transformation
laws of $E_z$, $E_{\zbar}$, $\Omega_z$ and
$\Omega_{\zbar}$ follow.


$N=1$ supergeometry admits an almost complex structure
$$
J_M{}^N=E_M{}^a \Eps_a{}^b E_b{}^N+E_M{}^\alpha
\gammabar_\alpha{}^\beta E_\beta{}^N
\eqno{(9.A.11)}
$$
where $\Eps^2=\gammabar^2=-1$.
In fact, $J$ is a {\it complex structure} provided 
the constraints given in (9.A.6) are satisfied.
(This may be viewed as one of the motivations for
choosing those constraints.)
We shall denote local complex coordinates by
$(\xi,\xibar,\theta,\thetabar)$ with $\xi=1/\sqrt{2\,\,}\,
(\xi^1+i\xi^2)$,
$\theta=1/\sqrt{2\,\,}\,(\theta^1+i\theta^2)$, and $\xibar$
and $\thetabar$ their complex conjugates.

By a combined $s\Diff$, $s\Weyl$ and $sU(1)$
transformation, any $N=1$ supergeometry may be mapped
on a locally flat (Euclidean) supergeometry, characterized
by the superderivatives
$$
\eqalign{
\scrD_+
&={\partial\over\partial\theta}
+\theta{\partial\over\partial\xi}\cr
\noalign{\medskip}
\scrD_- &={\partial\over\partial\thetabar}+
  \thetabar{\partial\over\partial\xibar}\cr}
\qquad\qquad
\eqalign{
\scrD_z &=\scrD_+^2={\partial\over\partial\xi}\cr
\noalign{\medskip}
\scrD_{\zbar} &=\scrD_-^2
={\partial\over\partial\xibar}\,\,.\cr}
\eqno{(9.A.12)}
$$
The residual symmetry group of $s\Diff\times
sU(1)\times s\Weyl$ that leaves this supergeometry
invariant is the group of {\it superconformal}
transformations, which consists of superanalytic
diffeomorphisms, $\xi\to\xi'(\xi,\theta)$,
$\xibar\to\xibar'(\xibar,\thetabar)$,
$\theta\to\theta'(\xi,\theta)$;
$\thetabar\to\thetabar'(\xibar,\thetabar)$, combined
with $s\Weyl$ and $sU(1)$ transformations.

{\it Wess-Zumino gauge} is obtained by fixing a gauge
for the action of the {\it algebraic} parts of
$s\Diff(\Sigma)$, $sU(1)$ and $s\Weyl$.
The remaining independent fields are $e_m{}^a$,
$\chi_m{}^\alpha$ and an auxiliary field $A$, and we
have
$$
\eqalign{
E_m{}^a &=e_m{}^a+\theta\gamma^a\chi_m-{\Half}
\theta\thetabar e_m{}^a A\cr
E_m{}^\alpha &=-{\Half}\chi_m{}^\alpha-
{\scriptstyle i\over\scriptstyle 2}
\theta^\beta(\gamma^a)_\beta{}^\alpha e_m{}^a
A+\cdots\cr
E_\mu{}^a &=(\gamma^a)_\mu{}^\beta\theta_\beta\cr
E_\mu{}^\alpha &=\delta_\mu{}^\alpha\Bigl(1+{i\over 4}
\theta\thetabar A\Bigr)\cr}
\eqno{(9.A.13)}
$$
This presentation of the supergeometry is so-called {\it in
components}.

Integration over $\Sigma$ is performed with the help of
a super-measure $d\mu_E$, defined as follows:
$$
d\mu_E\equiv s\det\,E_M{}^A d\xi\,d\xibar\,d\theta\, 
d\thetabar
\eqno{(9.A.14)}
$$
where the superdeterminant (also sometimes referred to
as the Berezinian) is defined by
$$
s\det\,E_M{}^A\equiv\det(E_m{}^a-E_M{}^\alpha
(E_\mu{}^\alpha)^{-1}E_\mu{}^a)\cdot
(\det\,E_\mu{}^\alpha)^{-1}
\eqno{(9.A.15)}
$$
Of course, $d\mu_E$ is $s\Diff$ invariant.
Inner products on superfields $\Psi_1$ and $\Phi_2$ of
weight $n$ are defined by
$$
\left<\Phi_1,\Phi_2\right>=\int\nolimits_{\Sigma}
d\mu_E\,\Phi_1^*\Phi_2\,\,.
\eqno{(9.A.16)}
$$

\bigskip\noindent
\Item{\bf B)} {\bf Functional Integral Representation
of Transition Amplitudes}

The $\RNS$ worldsheet action, introduced in \S{VII} in
components, admits a simple expression in superfield
form.
We introduce a string coordinate superfield of $sU(1)$
weight $0$; $X=\Sigma\to\dbR^{10}$
$$
X^\mu=x^\mu+\theta^\mu\psi_+{}^\mu+\thetabar\psi_-^\mu+\theta
\thetabar\,F^\mu\qquad \mu=0,1,\ldots,9\,\,.
\eqno{(9.B.1)}
$$
The action
$$
S_X={1\over 4\pi}\int\nolimits_{\Sigma}d\mu_E
\scrD_-X\cdot\scrD_+X
\eqno{(9.B.2)}
$$
reduces to the $\RNS$ action,{\baselineskip=9pt\footnote{$^*$}%
{\eightpoint In locally flat coordinates,
this action equals in components
$$
S_X={1\over 4\pi}\int\nolimits_\Sigma d^2 z(\partial_z
x\cdot\partial_{\zbar}x-\psi_+\cdot\partial_{\zbar}
\psi_+-\psi_-\cdot\partial_z\psi_- -F^2)
\eqno{(9.B.3)}
$$
so that the sign of the $\psi$-kinetic term is opposite
to the convention we used previously.
Notice that this reverses the sign of the $\psi$ stress
tensor.
Apologies!
(Conventions are to be uniformized in the final write-up.)

}}
except for the fact that
it has an entire term of the form $F^2$.
This term is non-dynamical and the $F$ field may be
integrated out without any consequences.

The next action we need is that of the super $(B,C)$
system:
let $B$ and $C$ be superfields of $sU(1)$ weight $n$
and $\Half-n$, respectively.
The $(B,C)$ action is then
$$
S_{BC}={1\over 2\pi}\int\nolimits_{\Sigma}
d\mu_E(B\scrD_-^{(n)}C+\Bbar\scrD_+^{(-n)}\Cbar)\,\,.
\eqno{(9.B.4)}
$$
To make contact with the $(b,c)$ action, we decompose the
superfields $B$, $C$ in components
$$
\eqalign{B &=\beta+\theta b+\thetabar
b'+\theta\thetabar\beta'\cr
C &=c+\theta\gamma+\thetabar\gamma'+
  \theta\thetabar c'\cr}
\eqno{(9.B.5)}
$$
The fields $b'$, $\beta'$, $\gamma'$ and $c'$ are
auxiliary, just as $F$ was an auxiliary field
in $S_X$, and we shall drop
their contribution to $S_{BC}$.
The $U(1)$ weight assignments in Wess-Zumino gauge are
now obtained by using the fact that $\theta$ (resp.
$\thetabar$) have weight $-\Half$ (resp. $\Half$).
Hence, the weights are, for $n=-1$:
$$
\def\bigstruta{\vrule height 15pt depth 3pt width 0pt}
\def\bigstrutb{\vrule height 20pt depth 4pt width 0pt}
\vbox{\offinterlineskip
\halign{\hfil#\hfil\bigstruta &\hfil#\hfil \qquad
  \vrule\quad &\hfil #\hfil \bigstrutb
 &\hfil \quad #\hfil  &\hfil \quad #\hfil  
  &\hfil \quad #\hfil \cr
&field\hfill &$\beta$ &$b$ &$\gamma$ &$c$\cr
\noalign{\hrule}
&$U(1)$ weight\hfill &$3/2$ &$2$ &$-1/2$ &$-1$\cr}}
\eqno{(9.B.6)}
$$
We recognize the $(b,c)$ system of anti-commuting ghost
fields, and we now also find a $(\beta,\gamma)$ system
of commuting ghost fields (called superghost) which has
precisely the correct weight assignments to be the
ghost system for local supersymmetry.
The $(\beta\gamma)$ fields are worldsheet spinors, and
their spin structure must be the same as that of
$\psi_+{}^\mu$, in order for $(\beta\gamma)$ to be $N=1$
local supersymmetry ghosts.
The $BC$ action may be worked out in components and
becomes in $WZ$ gauge:
$$
S_{BC}={1\over 2\pi}\int\nolimits_{\Sigma} d^2 z
[b\nabla_{(-1)}
^zc+\beta\nabla_{(-1/2)}^z\gamma-\chi_{\zbar}{}^+
S_+^{\zbar}+c.c.]
\eqno{(9.B.7)}
$$
where the $BC$ supercurrent is
$$
S_{z+}\equiv +{\Half}\,b\gamma-
{\scriptstyle 3\over\scriptstyle  2}\,\beta
\nabla_z c-\nabla_z\beta c
\eqno{(9.B.8)}
$$
This current is the  $N=1$ superpartner of the stress
tensor and it is conserved; in fact it is analytic.

The combined $X$, $(B,C)$ system for $D=10$ components of
$X$ is superconformal invariant, and so the
super-conformal field theory associated with this
system depends upon $N=1$ supergeometry only through
its {\it supermoduli}.
Recall that supermoduli space is defined by
$$
s\scrM_h\equiv {\{E_M{}^A,\Omega_M\hbox{ satisfying
constraints of (9.A.6)}\}\over
s\Diff\times s\Weyl\times sU(1)}
\eqno{(9.B.9)}
$$
and is a space of dimension
$$
\dim\,s\scrM_h=\cases{
(0\vert 0) &\null\kern-2.83 true cm $h=0$\cr
\noalign{\medskip}
\left\{\matrix{
(1\vert 0) &\qquad\qquad h=1 &\hbox{even spin structure}\hfill\cr
(1\vert 1) &\qquad\qquad h=1 &\hbox{odd spin structure}\hfill\cr}
\right. &\cr
\noalign{\medskip}
(3h-3\vert 2h-2) &\null\kern-2.83 true cm$n\Ge 2$ \cr}
\eqno{(9.B.10)}
$$
The holomorphic cotangent space to $s\scrM_h$ is
spanned by super-holomorphic $3/2$ differentials
$\Phi_K$, which satisfy
$$
\scrD_-^{(3/2}\Phi_K=0\qquad\qquad
{{\Phi_K\hbox{ odd graded}}\atop
{K=1,\ldots,\dim\,s\scrM_h}}
\eqno{(9.B.11)}
$$
Super-Beltrami differentials are defined as the
duals to $\Phi_K$, and may be parametrized by the super-moduli
variations of $E_M{}^A$:
$$
\mu_K\equiv E_-{}^M {\partial E_M{}^z\over
\partial m_K}\qquad K=1,\ldots,\dim\,s\scrM_h\,\,.
\eqno{(9.B.12)}
$$
It is instructive to work out the $\mu_K$ in
Wess-Zumino gauge:
$$
\mu_K=\thetabar\left(e_{\zbar}{}^n
{\partial e_n{}^z\over\partial m_K}-\theta
{\partial\chi_{\zbar}{}^+\over \partial
m_K}\right)\,\,.
\eqno{(9.B.13)}
$$
The first term on the r.h.s. is recognized as the
ordinary Beltrami differential, for $m_K$ an even
modulus, while the second term corresponds to a novel
odd Beltrami differential for $m_K$ an odd modulus.

We are now ready to state a preliminary
{\it Definition} of transition amplitudes for $\RNS$
superstring amplitudes.
This preliminary definition is the natural generalization
of the ``summation over surfaces'' definition adopted for
bosonic string amplitudes in \S{I}.
It is preliminary in the sense that it treats left and
right chirality worldsheet spinors with the same spin
structure.
In the final definition of transition amplitudes, these
spin structures will have to be summed over independently
for left and right movers, and this will be presented in
\S{IX.F}, after the Chiral Splitting Theorem has been
established.
The preliminary definition will be useful, however, in
order to introduce superghosts, and it is needed in order
to set up the Chiral Splitting Theorem.
Transition amplitudes are defined by
$$
A_h=\int DE_M{}^A D\Omega_M\delta\hbox{ (constraints) }
\int DX^\mu V_1\ldots V_N e^{-S_X}
\eqno{(9.B.14)}
$$
for certain vertex operators $V_i$, to be specified
shortly.

In the critical dimension, $D=10$, and for
super-conformally invariant vertex operators $V_i$, we
have the following {\it Theorem}
$$
\eqalign{
A_h=\int\nolimits_{s\scrM_h}\prod\limits_{K}
dm_K\int &DX^u\int D(B\Bbar)\int D(C\Cbar)V_1\ldots
V_N e^{-S_X-S_{BC}}\cr
&\cdot\prod\limits_{K}\delta(\left<\mu_\kap,B\right>)
\delta(\left<\mubar_\kap,\Bbar\right>)\cr}
\eqno{(9.B.15)}
$$
The proof of this result is much more involved than in
the case or bosonic strings, because the inner products
$\left<\Phi_1,\Phi_2\right>$ are no longer positive on
superfields, and extra care must be used in dealing
with the $\scrD_+$ and $\scrD_-$ operators.
We shall not prove it here; we refer the reader to Rev.
Mod. Phys. {\bf 60} (1988) p. 917.

\vfill\eject

%\bigskip\noindent
\Item{\bf C)} {\bf Superconformal Field Theory {\rm
(some basics)}}

A conformal field theory with an additional $N=1$ local
supersymmetry invariance is automatically invariant
under superconformal transformations, and is an
 $N=1$ {\it superconformal field theory}.
Each conformal field has a superpartner (which is also
a conformal field), and together both fields fit into a
superconformal field; $X^\mu$, $B$, $C$ are examples of
such fields.
Conserved currents are other examples of conformal
fields that generalize to supercurrents. 
Conservation of a supercurrent $J$ is given by
$$
\scrD_- J=0
\eqno{(9.C.1)}
$$
In local super-complex coordinates, this means that
$$
J=J(z,\theta)=J_{1/2}(z)+\theta J_1(z)
\eqno{(9.C.2)}
$$
with $J_{1/2}$ and $J_1$ holomorphic in $z$.
The associated charge is defined by
$$
Q_J=\oint dz\,d\theta\,J=\oint dz\,J_1(z)
\eqno{(9.C.3)}
$$

Important examples of supercurrents include the stress
tensor, which is built up out of the ordinary stress
tensor $T_{zz}$ and the (worldsheet) supercurrent $S_{z+}$:
$$
T(z,\theta)\equiv S_{z+} (z)+\theta T_{zz}(z)\,\,.
\eqno{(9.C.4)}
$$
For the $X$-system, $T$ is given by
$$
\eqalign{
T^{(X)} &=-{\Half}\scrD_+X\cdot\scrD_+^2X\cr
&=-{\Half}\,\psi_+\cdot\partial_z x+\theta
\left(-{\Half}\,\partial_z x\cdot\partial_z x-{\Half}\,
\partial_z \psi_+\cdot\psi_+\right)\cr}
\eqno{(9.C.5)}
$$
We recognize indeed the super-current and the $x\psi$
stress tensor.
For the $(B,C)$ system, $T$ is given by
$$
\eqalign{
T^{(BC)} &=-C\scrD_+^2 B+{1\over2}\scrD_+
C\scrD_+B-{3\over2}\scrD_+^2C B\cr
&=-c\partial_z\beta+{1\over2}\gamma b-{3\over2}\partial_z
c\beta+\theta(T^{(bc)}+T^{(\beta\gamma)})\cr}
\eqno{(9.C.6)}
$$

The OPE of $T$ with superfields of weight $(h,0)$ is
determined by Ward identities, and we have the asymptotic
expansions
$$
\eqalign{
T(z,\theta)\Phi(w,\zeta)\sim n
{\theta-\zeta\over (z-w-\theta\zeta)^2}\Phi(w,\zeta)
 &+{\Half}{1\over
z-w-\theta\zeta}\scrD_+\Phi(w,\zeta)\cr
&+{\theta-\zeta\over
z-w-\theta\zeta}\scrD_+^2\Phi(w,\zeta)\cr}
\eqno{(9.C.7)}
$$
For example $\scrD_+ X$, $B$ and $C$ have weights
$n=1/2,3/2,-1$ respectively.
The super-Virasoro algebra is similarly expressed.
Denote $s\equiv z-w-\theta\zeta$
$$
T(z,\theta)T(w,\zeta)\sim{{1\over4}\chat\over s^3}
+{\scriptstyle 3\over\scriptstyle 2}
{\theta-\zeta\over s^2}T(w,\zeta)+
{\Half}\,{1\over s}\scrD_+T(w,\zeta)+{\theta-\zeta\over s}
\scrD_+^2T(w,\zeta)
\eqno{(9.C.8)}
$$
where $\chat={2\over 3}c$ represents the central
charge.

%\vfill\eject

\bigskip\noindent
\Item{\bf D)} {\bf {\rm BRST} Quantization}

The combined action $S_X+S_{BC}$ is BRST invariant.
For completeness, we record the BRST transformations on
the fields
$$
\eqalign{
\delta X^\mu &=\lam C\scrD_+^2 X^\mu-{\Half}\,\lam\,\scrD_+
C\scrD_+ X^\mu+\hbox{ complex conjugate}\cr
\delta C &=\lam C\scrD_+^2 C-{1\over 4}\lam\scrD_+
  C\scrD_+ C\cr
\delta B &=-\lam T\cr}
\eqno{(9.D.1)}
$$
where $\lam$ is a constant Grassmann-valued parameter
and $T$ is the stress tensor superfield, defined in C).
The associated BRST supercurrent is
$$
J_{\BRST}=C(T^{(X)}+{\Half}T^{(BC)})-{3\over
4}\scrD_+(C\scrD_+ CB)
\eqno{(9.D.2)}
$$
The BRST charge
$$
Q_{\BRST}=\oint dz\,d\theta\,J_{\BRST}
\eqno{(9.D.3)}
$$
is conserved and nilpotent precisely in $D=10$, in
analogy with the bosonic string.
Vertex operators for physical states commute with $Q_{\BRST}$,
while vertex operators of spurious states are
$Q_{\BRST}$-commutators of some operator.
We shall not carry out BRST quantization any further
here.

\bigskip\noindent
\Item{\bf E)} {\bf Vertex operators for physical states}

We distinguish two kinds of vertex operators: those for
the $\NS$ sector, creating space-time bosons, and those
for the $\R$ sector, creating space-time fermions (also
sometimes called the fermion vertex operator).
For closed strings, this produces four different kinds
of vertex operators: $\NS$-$\NS$ and $\R$-$\R$ creating
bosons, and $\NS$-$\R$ and $\R$-$\NS$ creating
fermions.
The $\R$-vertex cannot be constructed directly 
out of the fields
$X^\mu$, since it must produce a branch cut starting at
its insertion points.

\bigskip\noindent
\Item{\bf (1)} {\bf The $\NS$ vertex operator}

For simplicity, we specialize to the closed $\RNS$
string, and consider $\NS$-$\NS$ vertex operators
first; the
left-moving part may be constructed using the chiral
splitting theorem for the $\RNS$ string, to be derived
in the next section, \S{F}.  We begin by constructing $\RNS$
vertex operators for $\NS$-$\NS$ states 
without ghost dependence.

Vertex operators in the $\NS$-$\NS$ sector are given in
a fashion completely analogous to the bosonic string
$$
V(\eps,k)=\int\nolimits_{\Sigma}d\mu_E
P_n(\eps,\scrD_+X,\scrD_-X,\scrD_+^2X,\ldots\,)
e^{ik\cdot X}
\eqno{(9.E.1)}
$$
Here, $P_n$ is a polynomial involving a sum over terms,
each of which has a total number  $n$ of $\scrD_+$
derivatives.
By $sU(1)$ invariance of $V$, $P_n$ must then also have
a total number $n$ of $\scrD_-$ derivatives.
(On non-flat $N=1$ supergeometries, the curvature
superfield $R_{+-}$ may also enter, just as the Gaussian
curvature $R_g$ entered for the bosonic string in the
dilaton vertex.
$R_{+-}$ effectively counts for one $\scrD_+$ and for one
$\scrD_-$.)
By construction, $V(\eps,k)$ is invariant under
$s\Diff$ transformations.
It remains to guarantee that $V(\eps,k)$ is $s\Weyl$
invariant.

Under constant $s\Weyl$ transformations, $\scrD_+$ and
$\scrD_-$ have weight $1/2$, $d\mu_E$ has weight $-1$
and the exponential has weight $k^2$.
Given that $\#\scrD_+\,=n_\scrD=\,\#\scrD_-$, we have
$$
M^2=-k^2=n_\scrD-1
\eqno{(9.E.2)}
$$
The $\GSO$ projection forces $n$ to be odd only, since
the number of $\NS$ fields applied to the ground state
in each sector (left and right moving) must be odd.
Thus, we have $n_\scrD=2n+1$, with $n=0,1,2,\ldots$
and the masses of $\NS$-$\NS$ states are given by
$$
M^2=2n\qquad\qquad n=0,1,2,\ldots
\eqno{(9.E.3)}
$$
Non-constant $s\Weyl$ transformations will require in
addition that certain conditions be fulfilled on the
polarization tensors $\eps$, such as transversality.

The massless string $\NS$-$\NS$ vertex operator is
$$
V(\eps,k)=\eps_{\mu\nu}(k)\int\nolimits_{\Sigma}
d\mu_E\scrD_+X^\mu\scrD_-X^\nu e^{ik\cdot X}
\eqno{(9.E.4)}
$$
$s\Weyl$ invariance requires
$\eps_{\mu\nu}(k)k^\mu=\eps_{\mu\nu}(k)k^\nu=0$, and we
recover the vertex operator for dilaton, graviton and
$B_{\mu\nu}$ fields.
It is useful to work out this vertex operator in
components, dropping the auxiliary field $F$, and other
terms that vanish by the field equations, such as
$\partial_{\zbar}\psi_+$.
(The full justification for dropping terms that vanish
by the field equations may be given with the help of
the Chiral Splitting Theorem in the next section,
\S{F}.)
Restricting to left-movers only, we use the fact that
$$
X^\mu=X_L{}^\mu+X_R{}^\mu
\eqno{(9.E.5)}
$$
and obtain the left-moving $\NS$ part of the vertex
operator (9.E.?) as follows:

$$
\int d\theta\,\scrD_+X_L^\mu e^{ik\cdot X_L}
=(\partial_z x_L^\mu-i\psi_+^\mu\psi_+^\nu k_\nu)
e^{ik\cdot x_L}
\eqno{(9.E.6)}
$$
This vertex contains the bosonic vertex
operator, as well as an operator that allows for an
application of an even number of $\NS$ fermions only.

%\beginsection{(2) The $\R$-vertex operator}
\bigskip\noindent
\Item{\bf (2)} {\bf The $\R$ vertex operator}

Ramond states transform under spinor representations of
the (double cover of the) Lorentz group, $\Spin(1,9)$.
This group has two inequivalent Majorana-Weyl spinors
$S_\alpha$ and $S'{^\alpha}$, each of dimension $16$ so
that we let $\alpha=1,\ldots,16$.
To describe the representations $S_\alpha$ and
$S'{^{\alpha}}$, we introduce the
Clifford algebra of $\Spin(1,9)$
$$
\{\Gamma^\mu,\Gamma^\nu\}=2\eta^{\mu\nu}I_{(32)}
\eqno{(9.E.7)}
$$
where $I_{(32)}$ is the identity matrix of dimension
$32$.
The chirality matrix $\Gambar$ is defined by
$$
\Gambar\equiv\Gamma^0\Gamma^1\cdots\Gamma^9
$$
and satisfies
$$
\Gambar^2=I_{(32)}\quad\hbox{and}\quad
\{\Gamma^\mu,\Gambar\}=0
\eqno{(9.E.8)}
$$
Weyl spinors of left (resp. right) chirality are
eigenvectors of $\Gambar$ with eigenvalues $+1$ (resp.
$-1$).
The charge conjugation matrix $C$ is defined by the
relation
$$
(\Gamma^\mu)^T=-C\Gamma^\mu C^{-1}
\eqno{(9.E.9)}
$$
and anti-commutes with $\Gambar$:
$$
\{\Gambar,C\}=0\,\,.
\eqno{(9.E.10)}
$$
The charge conjugate $\psi^c$ of $\psi$ is defined by
$$
\psi^c\equiv C\psibar^T\qquad\qquad
\psibar=\psi^\dagger\Gamma^0
\eqno{(9.E.11)}
$$
A Majorana spinor is a spinor whose charge conjugate is
equal to itself, $\psi^c=\psi$.
It is consistent (and convenient) to choose
$C=\Gamma^0$, so that a Majorana spinor is {\it real}:
$\psi^c=\psi^*=\psi$.
We shall choose a basis in which
$$
\Gambar=\pmatrix{
I_{(16)} &0\cr
&\cr
0 &-I_{(16)}\cr}\,\,,\qquad
C=\Gamma^0=\pmatrix{
0 &I_{(16)}\cr
&\cr
-I_{(16)} &0\cr}
\eqno{(9.E.12)}
$$
A 32 component Majorana spinor $\psi$ now decomposes
into two 16 component Majorana-Weyl spinors $S_\alpha$
and $S'{^\alpha}$.
It will be convenient to work with
$\Gamma$-matrices in this basis: 
$$
\psi=
\pmatrix{
S_\alpha\cr
\cr
S'{^\alpha}\cr}
\qquad\qquad\qquad
\Gamma^\mu=
\pmatrix{
0 &(\gamma^\mu)_{\alpha\beta}\cr
\cr
(\gamma^\mu)^{\alpha\beta} &0\cr}
\eqno{(9.E.13)}
$$
with $\alpha,\beta=1,\ldots,16$. 
It then follows from (9.E.9), (9.E.12) that
$\gamma_{\alpha\beta}^\mu=\gamma_{\beta\alpha}^\mu$
and $\gamma^{\mu\alpha\beta}=\gamma^{\mu\beta\alpha}$.
The generator of $\Spin(1,9)$ transformations in this
representation is ${\Half}\,\Gamma^{\mu\nu}$, with
$$
\eqalign{
\Gamma^{\mu\nu} &\equiv{\Half}\,[\Gamma^\mu,
  \Gamma^\nu]=\pmatrix{\gamma^{\mu\nu} &0\cr
0 &\gamma^{\mu\nu}\cr}\cr
\gamma^{\mu\nu} &\equiv {1\over
2}\,[\gamma^\mu,\gamma^\nu]\,\,.\cr}
\eqno{(9.E.14)}
$$
Clearly, $\Gamma^{\mu\nu}$ commutes with $\Gambar$, and
${\Half}\,\gamma^{\mu\nu}$ are the representation
generators of $\Spin(1,9)$ in one of the Weyl
representations.

The $\SO(1,9)$ transformation properties in the Ramond
sector are governed by the $\SO(1,9)$ currents (we omit
the contribution from the $x$-field)
$$
j^{\mu\nu}(z)=\psi_+^\mu\,\psi_+^\nu(z)
\eqno{(9.E.15)}
$$
Clearly, the field $\psi_+^\mu$ itself transforms under
the vector representation of $\SO(1,9)$:
$$
j^{\mu\nu}(z)\psi_+^\rho(w)\sim{1\over z-w}
(\eta^{\mu\rho}\psi_+^\nu(w)-\eta^{\nu\rho}
\psi_+^\mu(w))\,\,.
\eqno{(9.E.16)}
$$
We seek to construct fields $S_\alpha(z)$ and
$S'{^\alpha}(z)$ such that transform as spinors under
$\Spin(1,9)$ with representation matrices
$\gamma^{\mu\nu}$:
$$
\eqalignno{
j^{\mu\nu}(z) S_\alpha(w) &\sim{1\over z-w}
\left(-{\Half}\right)
(\gamma^{\mu\nu})_\alpha{}^\beta S_\beta(w) &(9.E.17a)\cr
\noalign{\vskip10pt}
j^{\mu\nu}(z) S'{^\alpha}(w) &\sim{1\over z-w}
\left(-{\Half}\right)(\gamma^{\mu\nu})^\alpha{}_\beta
S'{^\beta}(w)\,\,. &(9.E.17b)\cr}
$$
From these OPE's, and from the fact that $j^{\mu\nu}$
is bilinear in $\psi_+^\mu$, we may infer the OPE of
$\psi_+$ with $S$:
$$
\eqalignno{
\psi_+^\mu(z) S_\alpha(w) &\sim{1\over (z-w)^{{1\over
2}}}\,
{1\over\sqrt{2\,\,}}\,(\gamma^\mu)_{\alpha\beta}
S'{^\alpha}(w)&(9.E.18a)\cr
\noalign{\vskip10pt}
\psi_+^\mu(z) S'{^\alpha}(w) &\sim{1\over(z-w)^{{1\over
2}}}\,{1\over\sqrt{2\,\,}}\,
(\gamma^\mu)^{\alpha\beta}S_\beta(w)\,\,.&(9.E.18b)\cr}
$$
The relative normalization between $S$ and $S'$ is
arbitrary, and has been fixed above on symmetry
grounds.
The conformal weights of $S_\alpha(z)$ and
$S'{^\alpha}(z)$ are computed from their OPE with the
stress tensor $T^{(\psi)}=-{\half}\,
\partial_z\psi_+\cdot\psi_+$:
$$
\eqalign{
T^{(\psi)}(z)S_\alpha(w) &\sim\left(-{\Half}\right)
\left(-{\Half}\right){1\over\sqrt{2\,\,}}\,{1\over
  (z-w)^{3/2}}\gamma_{\alpha\beta}^\mu\,\psi_+^\mu(z)
S'{^\beta}(w)\cr
\noalign{\vskip10pt}
&\sim{1\over 16}(\gamma^\mu\gamma_\mu)_\alpha^\beta
S_\beta(w){1\over (z-w)^2}\,\,.\cr}
\eqno{(9.E.19)}
$$
Since $\gamma^\mu\,\gamma_\mu=10\,I_{(16)}$, we find
that the conformal weight of $S_\alpha$ (and similarly
of $S'{^{\alpha}}$) is $5/8$. 

To obtain the spin fields $S_\alpha$ and $S'{^\alpha}$,
we make use first of the action of the currents $j^{\mu\nu}$
in the Cartan subalgebra of $\Spin(1,9)$ only.
These components mutually
commute and may be bosonized in terms
of $5$ independent free scalar Bose fields:
$$
j^{2a,2a+1}(z)=-i\partial_z\phi^a(z)\qquad
a=0,1,2,3,4\,\,.
\eqno{(9.E.20)}
$$
The $R$-fields $\psi_+{}^\mu$ are exponentials of
$\phi^a$ with $\pm 1$ weights:
$$
{1\over\sqrt{2\,\,}}\,(\psi_+^{2a}\pm
i\psi_+^{2a+1})=e^{\pm i\phi^a}\,\,.
\eqno{(9.E.21)}
$$
The spin fields are exponentials of $\phi^a$ with
$\pm\,{\Half}$ weights:
$$
S_\alpha(z)=e^{i\alpha\cdot\phi}\,\,.
\eqno{(9.E.22)}
$$
where we identify the space-time indices with the
spinor representation weights, given by
$\alpha=(\pm\,{\Half},\, \pm\,{\Half},\, \pm\,{\Half}, 
\pm\,{\Half},\, \pm\,{\Half})$.
The spinors $S_\alpha$ are obtained for $\alpha$ with
an even number of $-$ signs, while $S'{^\alpha}$ is
obtained with an odd number of $-$ signs.
The conformal dimension of this field is easily
verified to be ${\Half}\,\alpha^2=5/8$, with the help
of the standard formula for the conformal weight of an
exponential of a $c=1$ conformal field.

One may complete the analysis by showing that the
currents $j^{\mu\nu}$ corresponding to roots may also be
constructed in terms of exponentials (multiplied by
certain cocycle factors) and that $j^{\mu\nu}$ indeed
satisfies the correct OPE with $S_\alpha$ and
$S'{^\alpha}$.

With the above normalization, one also obtains the
OPE's of $S$ with $S$ and with $S'$:
$$
\eqalign{
S'{^{\alpha}}(z)S_\beta(w) &\sim {\delta_\beta{}^\alpha\over
(z-w)^{5/4}}+{1\over (z-w)^{1/4}}\,{\half}\,
\gamma^{\mu\nu}j_{\mu\nu}(w)\cr
S_\alpha(z)S_\beta(w) &\sim
{1\over(z-w)^{3/4}}\,\gamma_{\alpha\beta}^\mu\,\psi_\mu\cr}
\eqno{(9.E.23)}
$$
(Notice that the construction of spinor fields that we
have just given for $\Spin(1,9)$ may be carried over in
a straightforward way to $\Spin(2n)$.
In particular, this type of construction is needed to
obtain the spinor of $\SO(16)$ necessary to build
the full current algebra of $E_8\times E_8$ in the
heterotic string.
There, the spin fields have precisely conformal
dimension $(1,0)$, since the weight vectors all have length
$\alpha^2=2$.)

It remains to construct the Ramond vertex operator.
The spin field by itself cannot produce a vertex
operator for massless fermions, since its conformal
weight is $5/8\not=1$.
The resolution of this issue is found in a contribution
to the $R$ vertex operator from the $(\beta,\gamma)$ and
possibly $(b,c)$ ghost system.
The relevant conformal primary field is obtained as
follows.
(We consider only the left-moving part of the vertex
operator.)
$$
W_-(z,u,k)=u^\alpha(k)\scrO(z)S_\alpha(z)e^{ik\cdot
x_L(z)}
\eqno{(9.E.24)}
$$
Here, $u^\alpha$ is a spinor analogue of the
polarization tensors $\eps$ that entered the construction of
$\NS$ states and vertex operators.
The operator $\scrO(z)$ depends on the $\beta$ and
$\gamma$ ghosts, and is a primary field of weight
$3/8$, so that $W_-$ is of weight $1$ for massless
states (with $k^2=0$).
$\scrO(z)$ is constructed in such a way that $W_-$ is
$\BRST$ invariant; it is an analogue of the spin field,
but now for the superghost system.

The $(\beta,\gamma)$ ghost system may be represented in
terms of two free boson fields $\phi$ and $\chi$, such
that
$$
\gamma=e^{\phi-\chi}\qquad\qquad\beta=e^{-\phi+\chi}\partial_z
\chi
\eqno{(9.E.25)}
$$
with
$$
\matrix{
\phi(z)\,\phi(w)\hfill 
&\sim\hfill &-\ln(z-w)\hfill &c_\phi=13\hfill\cr
\noalign{\bigskip}
\chi(z)\chi(w)\hfill &\sim\hfill 
&\null\quad\! \ln(z-w)\hfill &c_\chi=-2\hfill\cr}
\eqno{(9.E.26)}
$$
The ghost number current is $j_z=-\partial_z\phi$.
The operator $\scrO(z)$ is again an exponential with
weight ${\Half}\,$:
$$
\scrO(z)=e^{-{\Half}\phi(z)}\,\,.
\eqno{(9.E.27)}
$$
Because the central charge of the $\phi$ system is
$13$, the dimension of $\scrO(z)$ is $3/8$, instead of
$1/8$, as would be the case for a field $\phi$ of
central charge $1$.

The above construction of $W_-$ implies, though, that
$W_-$ has ghost number $-1/2$!
But overall ghost number is given in terms of the index
theorem, and this would suggest that the number of
Ramond vertex operators that can be inserted on any
Riemann surface is determined in terms of the topology
of the surface, which is clearly absurd.

The resolution of this problem lies in the subtleties
of the quantization of the $(\beta,\gamma)$ system, which
we shall not discuss here.
The final result is that the quantization of the
$(\beta,\gamma)$ system yields vertex operators in
different possible pictures, labeled by their ghost
number.
One shows, for example, that there is an equivalenT
vertex operator for $W_-$, which had ghost number $-1/2$.
This operator is $W_+$ with ghost number $+\Half$ and is
given by
$$
W_+(z,u,k)=e^{\phi/2}u^\alpha(\gamma_\mu)_{\alpha\beta}
S'{^\beta}(z)(\partial_z x^\mu-i\psi_+^\mu k\cdot\psi_+)
e^{ik\cdot x}
\eqno{(9.E.28)}
$$
Vertex operators for $\R$ states  may now be inserted
on any worldsheets in any arbitrary (even) numbers.

\bigskip\noindent
\Item{\bf F)} {\bf The Chiral Splitting Theorem}

In critical $\RNS$ string theory, we have powerful
results on the superholomorphicity properties of
correlation functions for (factorized) vertex
operators, at fixed internal momenta.
This result is contained in the {\it Chiral Splitting
Theorem} for the $\RNS$ string.

We begin by defining {\it factorized vertex operators}.
A vertex operator
$$
V_i=\int d^2z_i\,d^2\theta_i\,W_i(z_i,\theta_i,\zbar_i,\thetabar_i)
$$
is factorized provided $W_i$ is the product of holomorphic
and anti-holomorphic factors
$$
W_i(z_i,\theta_i,\zbar_i,\thetabar_i)=W_i(z_i,\theta_i)\Wtil_i
(\zbar_i,\thetabar_i)
$$
Any vertex operator may be written as a linear
combination of factorized vertex operators.
For example, for $\NS$-$\NS$ massless states, we have
$$
W_i(z_i,\theta_i;\zbar_i,\thetabar_i)
=\left(\xi_{i\mu}\scrD_+ X^\mu e^{ik\cdot X_L}\right)
\left(\xibar_{i\mubar}\scrD_- X^{\mubar}e^{ik\cdot
X_R}\right)
\eqno{(9.F.1)}
$$

Next, let $A_I$ and $B_I$, $I=1,\ldots,h$ be any
canonical basis of homology $1$-cycles with canonical
intersection form.
We define $h$ independent 
internal loop momenta $p_I\in\dbR^{10}$,
associated with each $A_I$-cycle, by
$$
p_I=\oint_{A_I}dz\,d\theta\,\scrD_+ X
\eqno{(9.F.2)}
$$
Let the spin structures for left and right chirality
spinors be equal, and equal to $\nu$.

The correlation function of the unintegrated vertex
operators $W_i$ at fixed internal loop momenta $p_I$ and fixed
supermoduli are defined by%
{\baselineskip=9pt\footnote{$^*$}{\eightpoint It is understood
that for genus $h=0$ (resp. $h=1$) three (resp. $1$) of
the vertex operators should not be integrated over, and
should be multiplied by a factor of
$\delta(\!(C,\psi)\!)$, where $\psi$ is a
superconformal Killing vector.

}}
$$
\eqalign{
&\left<W_1\ldots W_N\right>_E(p_I)\cr
&\null\qquad \equiv\int DX\int DBDC\,\,W_1\ldots
W_N\prod\limits_{I=1}^h\delta\left(p_I^\mu-\int\nolimits_{A_I}
dz\,d\theta\,\,\scrD_+X^\mu\right)\cr
&\null\qquad\qquad\qquad \prod\limits_{K=1}^{\dim\,s\scrM_h}
\vert\left<\mu_K,B\right>\vert^2 e^{-S_X-S_{BC}}\cr}
\eqno{(9.F.3)}
$$
(I) The first part of the {\it Chiral Splitting Theorem} states
that the $\RNS$ amplitudes factorize at fixed internal
loop momenta:
$$
\left<W_1\ldots W_N\right>_E(p_I)=\delta(k)
\scrC_\nu^F\scrCbar_\nu^F
\eqno{(9.F.4)}
$$
Here, $\scrC_\nu{}^F=\scrC_\nu{}^{\Fbar}
(z_i,\theta_i;\zeta_i,;m_i;
p_I,k_i)$ is a complex analytic function of
supermoduli $m_K$, $K=1,\ldots,\dim\,s\scrM_h$, of
the insertion points $(z_i,\theta_i)$, and of the
left-moving factors $\zeta_{i\mu}$ of the polarization
tensors. 
$\scrCbar_\nu^F$ is the complex conjugate of
$\scrC_\nu{}^F$, with the same spin structure
$$
\scrCbar_\nu^F(\zbar_i,\,\thetabar_i;\zetabar_{i};\,
\mbar_K;\,p_I,\,k_i)=
\scrC_\nu{}^F(z_i,\,\theta_i;\zeta_i;\,m_K\;\,p_I,\,k_i)^*
\eqno{(9.F.5)}
$$
The functions $C_\nu^F$ may be evaluated explicitly
using Green functions and determinants, using the prime
form and the Szeg\"o kernel introduced in \S{V}.

\medskip\noindent
(II)\enspace
The second part of the Chiral Splitting Theorem
prescribes the transition amplitudes for Type II and
heterotic strings.
We shall denote the analogous complex functions 
$\scrC^B=\scrC^B(z_i;m_k;p_I,k_i)$
for the bosonic string, compactified on the $16$-$\dim$
lattices of $E_8\times E_8$ or $\Spin(32)/Z_2$.

\medskip\noindent
\undertext{Type II A, B}:
$$
A_h=\delta(k)\sum\limits_{\nu,\nubar}Q_{\nu\nubar}
\int\nolimits_{\dbR^{10h}}d^{10}p_I\int\nolimits_{s\scrM_h}
dm_K\,d\mbar_K \prod\limits_{i=1}^N
\int\nolimits_{\Sigma}d^2z_i\,
d^2\theta_i\,C_\nu^F\Cbar_{\nubar}^F
\eqno{(9.F.6)}
$$

\medskip\noindent
\undertext{Heterotic $E_8\times E_8$, $\Spin(32)/Z_2$}
$$
A_h=\delta(k)\sum\limits_{\nu}Q_\nu\int\nolimits_{\dbR^{10\,h}}
d^{10}p_I\int\nolimits_{s\scrM_h}dm_K
\int\nolimits_{\scrM_h}d\mbar_k\prod\limits_{i=1}^N
\int\nolimits_{\Sigma}
dz_i\,d\theta_i\int\nolimits_{\Sigma}d\zbar_i\,C_\nu^F
\Cbar^B
\eqno{(9.F.7)}
$$
Here $Q_{\nu\nubar}$ and $Q_\nu$ are spin structure
dependent weight factors which realize the $\GSO$
projection.
If carried out in the supermoduli picture, $\nu$ and
$\nubar$ should run only over even or odd.
But, if the integrals are understood in components,
then one should sum over all individual spin structures
in each class as well.
In general $Q_{\nu\nubar}=\pm1$, and $Q_\nu=\pm1$.

\bigskip\noindent
\Item{\bf G)} {\bf Tree-Level Amplitudes for $\NS$-$\NS$
states}

To tree level, the worldsheet has the topology of the
sphere, or stereographically projected plane, and there
are neither moduli, nor odd moduli.
Thus, we let $g=2\vert dz\vert^2$ and $\chi=0$.
We concentrate on the scattering of massless $\NS$-$\NS$
states, which can be represented by the vertex operator
$$
V(\eps,k)=\eps_{\mu\nu}(k)\int\nolimits_{\Sigma}d^2 
z\,d^2\theta\,\,
\scrD_+X^\mu\scrD_- X^\nu e^{ik\cdot x}\,\,.
\eqno{(9.G.1)}
$$
By factorizing $\eps_{\mu\nu}(k)=\zeta_\mu(k)\zetabar_\nu(k)$,
we may start from a more convenient vertex operator
$$
V(\zeta,\zetabar;k)=\int\nolimits_{\Sigma} d^2 z\,d^2\theta\,
e^{ik\cdot X+\xi\cdot\scrD_+X+\xibar\cdot\scrD_- X}
\eqno{(9.G.2)}
$$
and restrict to its contributions linear in $\zeta$ and linear
in $\zetabar$. 
It is understood that $\zeta$ and
$\zetabar$ are Grassmann parameters.
The $X$-propagator is easily calculated:
$$
\left<X^\mu(z,\theta)X^\nu(z',\theta')\right>=-\ln\,\vert
z-z'-\theta\theta'\vert^2\,\eta^{\mu\nu}
\eqno{(9.G.3)}
$$
The determinants of $x$, $\psi$, $b$, $c$ and $\beta$,
$\gamma$ fields all combine into an overall constant,
which we denote by $\grz$.

The chiral splitting theorem is easily used here:
there are no internal loop momenta, so that the
correlation function of the unintegrated vertex
operators 
$$
W(z_i,\theta_i;\zbar_i,\thetabar_i)
=W_L(z_i,\theta_i)W_R(\zbar_i,\thetabar_i)
\eqno{(9.G.4)}
$$
will yield a product of a function $C^F$, complex
analytic in $z_i$, $\theta_i$ times its complex
conjugate.
Here, the chiral vertex operators are
$$
W_L(z_i,\theta_i)=e^{ik_i\cdot X_L(z_i,\theta_i)+
\zeta_i\cdot\scrD_+ X_L(z_i,\theta_i)}
\eqno{(9.G.5)}
$$
and analogously for $W_R$, where the chiral fields
$X_L$ are contracted with the effective propagator
$$
\left<
X_L^\mu(z,\theta)X_L^\nu(z',\theta')\right>=-\ln(z-z'-
\theta\theta')\eta^{\mu\nu}\,\,.
\eqno{(9.G.6)}
$$
One readily finds
$$
\eqalign{
&\left<W_L(z_1,\theta_1)\cdots W_L(z_N,\theta_N)\right>=
\scrC^F(z_i,\theta_i)\cr
&\null\qquad
\scrC^F=\exp\sum\limits_{i\not=j}^N\left\{
ik_i\cdot\zeta_j\,{\theta_{ij}\over z_{ij}}+{\Half}
\,\zeta_i\cdot\zeta_j\,{1\over z_{ij}}+{\half}\,
k_i\cdot k_j\ln\,z_{ij}\right\}\cr}
\eqno{(9.G.7)}
$$
where we have used the standard notation
$$
\left\{
\eqalign{
\theta_{ij} &=\theta_i-\theta_j\,\,;\cr
z_{ij} &=z_i-z_j-\theta_i\theta_j\,\,.\cr}
\right.
\eqno{(9.G.8)}
$$

At tree level, we have invariance under superconformal
automorphisms in $O\Sp(1,2)$; the integration measure
must be properly treated to account for this
invariance.
This group of transformations is defiend as follows;
consider the matrices
$$
K=\pmatrix{\null\kern8pt 0 &+1 &0\cr
-1 &\kern8pt0 &0\cr
\null\kern8pt0 &\kern8pt0 &1\cr}
\qquad\qquad 
T=\pmatrix{a &b &\alpha\cr
c &d &\beta\cr
\gamma &\delta &A\cr}\eqno{(9.G.9)}
$$
$$
z\to {az+b+\alpha\theta\over cz+d+\beta\theta}\qquad\qquad\qquad
\theta\to {\gamma z+\delta+A\theta\over cz+d+\beta\theta}
\eqno{(9.G.10)}
$$
where Latin entries are commuting and Greek entries are
anticommuting.
To obtain a {\it superconformal} transformation $T$, the
line element $dz=dz+\theta\,d\theta$ must be transformed
into itself up to a conformal scaling.
Equivalently the form
$$
z_{12}=z_1-z_2-\theta_1\theta_2={v_1w_2-v_2w_1-\psi_1\psi_2
\over w_1w_2}\eqno{(9.G.11)}
$$
should transform into itself up to a conformal scaling.
This is uniquely achieved when the orthosymplectic form is
left invariant under $T$:
$$
T^T KT=D\eqno{(9.G.12)}
$$
The weight under which the difference transforms is easily
derived, and we have
$$
T\colon\, z_{12}\to \zbar_{12}={z_{12}\over
(cz_1+d+\beta\theta_1)(cz_2+d+\beta\theta_2)}\eqno{(9.G.13)}
$$
Similarly the line element transforms as
$$
dz\to d\zbar={dz\over (cz+d+\beta\theta)^2}\eqno{(9.G.14)}
$$
and the volume element as
$$
dz\w d\theta\to{dz\w d\theta\over (cz+d+\beta\theta)}\,\,.
\eqno{(9.G.15)}
$$
Elements in $O\Sp(1,2)$ are in unique correspondence with
a triplet of points in the superplane $(z_1,\theta_1)$,
$(z_2,\theta_2)$, $(z_3,\theta_3)$ obeying one single
(Grassmann valued) constraint.
The counting works out because $O\Sp(1,2)$ has $3$ commuting
and $2$ anticommuting parameters.
The constraint is an $O\Sp(1,2)$ invariant Grassmann
valued function dependent on three points, given by
$$
\Delta={z_{12}\theta_3+z_{31}\theta_2+z_{23}\theta_1
+\theta_1\theta_2\theta_3\over(z_{12}z_{23}z_{31})^{1/2}}
\eqno{(9.G.16)}
$$
The natural value for $\Delta$ is of course $0$, which
implies that one $\theta$ is dependent.
With this value for $\Delta$, it is easy to see that there
is a unique correspondence between triplets of points
satisfying $\Delta=0$, and elements of $O\Sp(1,2)$, so
that the latter may be accordingly parametrized.
The invariant volume element induced on $O\Sp(1,2)$ is
then
$$
d\mu={dz_1dz_2dz_3d\theta_1 d\theta_2 d\theta_3\over
(z_{12}z_{23}z_{31})^{1/2}}\,\Delta\eqno{(9.G.17)}
$$
As for the bosonic string, the volume of the
superconformal transformation group must be extracted from
the amplitudes.
As a result of $O\Sp(1,2)$ invariance, the $0$, $1$,
$2$ point functions vanish.

We now work out the three point amplitude.
In evaluating $\scrC^F$, one retains terms proportional to
$\zeta_1\zeta_2\zeta_3$; however, the term with three
$\theta$'s vanishes because
$\theta_{12}\theta_{23}\theta_{31}=0$.
Defining the tensors
$$
K_{\mu_1\mu_2\mu_3}=
\eta_{\mu_1\mu_2}k_{1\mu_3}+\eta_{\mu_2\mu_3}
k_{2\mu_1}+\eta_{\mu_3\mu_1}k_{3\mu_2}\eqno{(9.G.18)}
$$
we find for the three point function
$$
\left<V(\eps_1,k_1)V(\eps_2k_2)V(\eps_3k_3)\right>=
4(2\pi)^{10}\delta(k)
\eps_1^{\mu_1\mubar_1}\eps_2^{\mu_2\mubar_2}
\eps_3^{\mu_3\mubar_3}K_{\mu_1\mu_2\mu_3}
K_{\mubar_1\mubar_2\mubar_3}\eqno{(9.G.19)}
$$
As a result of transversality of the polarization
tensors, the $3$ point function for massless external
particles also vanishes.

To compute the four point amplitude, we shall fix
superconformal invariance as follows: $z=z_1$, $z_2=0$,
$z_3=1$, $z_4=\infty$,
$\theta_1,\theta_2,\theta_3=\theta_4=0$, and then have
$$
\eqalign{
\exp\scrG_4^\zeta &=\left\{\zeta_1\cdot\zeta_2\zeta_3\cdot
\zeta_4{1\over z_{12}z_{34}}+\zeta_1\cdot\zeta_3\zeta_2
\cdot\zeta_4{1\over
z_{13}z_{24}}+\zeta_1\cdot\zeta_4\zeta_2\cdot\zeta_3
{1\over z_{14}z_{23}}\right\}\cr
\noalign{\medskip}
&+\left\{\zeta_1\cdot\zeta_2\left(k_1\cdot\zeta_3k_2\cdot
\zeta_4{\theta_2\theta_1\over z_{12}z_{13}z_{24}}+k_1
\cdot\zeta_4k_2\cdot\zeta_3{\theta_2\theta_1\over
z_{12}z_{14}z_{23}}\right)+\hbox{perm.}\right\}\cr}
\eqno{(9.G.20)}
$$
In principle, one should now multiply this whole
expression by the one involving the $\zetabar$'s, perform
the integrals over $z$ and $\theta$ and regroup terms,
clearly a feudal task.
The calculation is enormously simplified by the
factorization properties of the Veneziano-integrals.
Recall that we have the ordinary integrals
$$
\int{d^2z\over\pi}z^A\zbar^{\Atil}(1-z)^B(1-\zbar)^{\Btil}=
{\Gamma(-1-\Atil-\Btil\over \Gamma(-\Atil)\Gamma(-\Btil)}\quad
{\Gamma(1+A)\Gamma(1+B)\over \Gamma(A+B+2)}
\eqno{(9.G.21)}
$$
provided $A-\Atil$ and $B-\Btil$ are integers, which is
always the case in string theory.
Using the reciprocity formula for $\Gamma$-functions and
the fact that $A-\Atil$ and $B-\Btil$ are integers, this
expression is actually symmetric under
$(A,B)\leftrightarrow (\Atil,\Btil)$, as one might expect
from complex conjugation.
More importantly, the answer factorizes into a product of
factors each only dependent either on the parameters for
the $z$ or $\zbar$ coordinates.
This product property implies that one must only consider
say the $z$ coordinates to find the full amplitude, which
by the same token will also completely factorize as a
function of $\zeta$'s and $\zetabar$'s.
An analogous formula is derived for the super-integrals we
need:
$$
\eqalign{
\int{d^2z_1\over\pi}d^2\theta_2[\theta_1\theta_2]^a
&[\thetabar_1\thetabar_2]^{\atil} z_{12}^A\,\zbar_{12}^{\Atil}
(1-z_1)^B(1-\zbar_1)^{\Btil}\cr
&=(-2i)^{1-a}(+2i)^{1-\atil}
{\Gamma(-\atil-\Atil-\Btil)\over
\Gamma(-\Atil)\Gamma(-\Btil)}{\Gamma(1+A)\Gamma(1+B)\over
\Gamma(A+B+1+a)}\cr}
\eqno{(9.G.22)}
$$
Here $a$ and $\atil$ are either $0$ or $1$, and the
integrals are symmetric under $(aAB)\leftrightarrow
(\atil\Atil\Btil)$.
With the help of (9.G.22), it is now straightforward to
evaluate the four point function
$$
\eqalign{
\bigl<V(\eps_1,k_1)
  &V(\eps_2,k_2)V(\eps_3,k_3)V(\eps_4,k_4)\bigr>\cr
&=(2\pi)^{10}\delta(k)g^4\int d^2z_1d^2\theta_2\vert
z_{12}\vert^{-s}\vert z_1-1\vert^{-u}
e^{\scrG_4^\zeta+\scrG_4^{\zetabar}}\cr
&=\pi(2\pi)^{10}\delta(k)g^4{\Gamma(-s/2)\Gamma(-t/2)
  \Gamma(-u/2)\over
\Gamma\left(1+{s\over 2}\right)\Gamma\left(1+{t\over 2}\right)
\gamma\left(+{u\over 2}\right)}\eps^{1\1bar}\eps^{2\2bar}
\eps^{3\3bar}\eps^{4\4bar}K_{1234}
K_{\1bar\2bar\3bar\4bar}\cr}
\eqno{(9.G.23)}
$$
Using the abbreviation $i$ for $\mu_i$ to save some
writing, we have
$$
\eqalign{
K_{1234}
&=(st\eta_{13}\eta_{24}
-su\eta_{14}\eta_{23}-tu\eta_{12}\eta_{34})\cr
&-s(k_1^4k_3^2\eta_{24}+k_2^3k_4^1\eta_{13}-k_1^3k_4^2
  \eta_{23}-k_2^4k_3^1\eta_{14})\cr
&+t(k_2^1k_4^3\eta_{13}+k_3^4k_1^2\eta_{24}-
  k_2^4k_1^3\eta_{34}-k_3^1k_4^2\eta_{12})\cr
&-u(k_1^2k_4^3\eta_{23}+k_3^4k_2^1\eta_{14}-k_1^4k_2^3\eta_{34}
  -k_3^2k_4^1\eta_{12})\cr}
\eqno{(9.G.24)}
$$

\bigskip\noindent
\Item{\bf H)} {\bf One loop Amplitudes for $\NS$-$\NS$
states in Type II A, B}

We begin by using part (I) of the Chiral Splitting
Theorem, and we compute the Chiral Amplitudes
$\scrC_\nu^F$.
The spin structure for left and right movers is the
same here.

For {\it even spin structures} $\nu$, there are no odd
moduli, and there are no Dirac zero modes.
The metric on the torus is taken to be $g=2\vert
dz\vert^2$, and the torus is represented as usual by a
parallelogram with sides $0$, $1$, $\tau$, $1+\tau$,
with $\tau\in H$ in the upper half complex plane $H$.
The $S_X$ and $S_{BC}$ actions are free actions in
components:
$$
\eqalign{
S_X+S_{BC}={1\over 4\pi}\int\nolimits_{\Sigma}
d^2 z &\Bigl(\partial_z x\cdot\partial_{\zbar}x-
\psi_+\cdot\partial_{\zbar}\psi_+-\psi_-\cdot
\partial_z\psi_-\cr
&+2b\partial_{\zbar}c+2\bbar\partial_z\cbar+
2\beta\partial_{\zbar}\gamma+2\betabar\partial_z
\gammabar\Bigr)\,\,.\cr}
\eqno{(9.H.1)}
$$
The ghost and anti-ghost insertions are reduced to
$$
b\bbar\,c\cbar\,\,.
\eqno{(9.H.2)}
$$
No anti-superghost insertions occur for even spin
structures, since there are no zero modes there.
In the $\NS$-$\NS$ sector, vertex operators do not
involve ghosts, so we may integrate out $b$, $c$,
$\bbar$, $\cbar$, $\beta$, $\gamma$, $\betabar$ and
$\gammabar$.
We shall also retain the determinants of $x$ and
$\psi_{\pm}$ integrations.
Thus
$$
\delta(k)\scrC_\nu\scrC_\nu=M_\nu\Mbar_\nu\,\scrF_\nu
\scrFbar_\nu
\eqno{(9.H.3)}
$$
where $M_\nu\,\Mbar_\nu$ arises from determinants and
$\scrF_\nu\,\scrFbar_\nu$ arises from correlation
functions of vertex operators.
We find
$$
\eqalign{
M_\nu\Mbar_\nu = &\left({\det'\Delta_0\over
  (\im\,\tau)^2}\right)^{-5}
\pmatrix{
\det\,\notD_+\cr
\phantom{XXX}\cr}_\nu^5
\pmatrix{
\det\,\notD_-\cr
\phantom{XXX}\cr}_\nu^5\cr
&\left({\det'\Delta_{-1}\over (\im\,\tau)^2}\right)^{+1}
  (\det\,\Delta_{-1/2}^-)_\nu^{-1}\cr}
\eqno{(9.H.4)}
$$
As usual, on the torus, these determinants are related
to one another, and we find
$$
\eqalign{
{\det'\Delta_0\over(\im\,\tau)^2} &={\det'\Delta_{-1}
\over(\im\,\tau)^2}=\vert\eta(\tau)\vert^4\cr
(\det\,\notD_+\,\notD_-)_\nu
&=(\det\,\Delta_{1/2}^-)_\nu=
\left\vert{\vartheta[\nu](0\vert\tau)\over
\eta(\tau)}\right\vert^2\cr}
\eqno{(9.H.5)}
$$
Putting all together:
$$
M_\nu={\vartheta[\nu](0\vert\tau)^4\over
\eta(\tau)^{12}}\qquad\qquad\qquad
\Mbar_\nu={\overline{\vartheta[\nu](0\vert\tau)^4}\over
\overline{\eta(\tau)}^{12}}
\eqno{(9.H.6)}
$$
up to an overall phase which is independent of $\tau$,
but which may depend upon $\nu$.
Clearly such a phase can be absorbed into the spin
structure weights $Q_{\nu\nubar}$.
The vertex contributions $\scrF_\nu$ are easily
evaluated:
$$
\eqalign{
\scrF_\nu(z_i, \theta_i,\zeta,k,p)
&=\exp\{i\pi p^2\tau+2\pi
p\cdot\sum\limits_{i}(-\zeta_i\theta_i+ik_iz_i)\cr
&\qquad\qquad -{\Half}\sum\limits_{i\not=j}(k_i\cdot
  k_jG_\nu(z_i,\theta_i;z_j,\theta_j)+\zeta_i\cdot\zeta_j
  \scrD_+^i\scrD_+^jG_\nu\cr
&\qquad\qquad\qquad
  +2ik_i\cdot\zeta_j\scrD_+^jG_\nu)\}\,\,.\cr}
\eqno{(9.H.7)}
$$
Here, the correlation function $G_\nu$ is given by
$$
\eqalign{
G_\nu(z_i,\theta_i;z_j,\theta_j)
&=\left<X_L(z_i,\theta_i)X_L(z_j,\theta_j)\right>\cr
&=-\ln\,E(z_i,z_j)+\theta_i\theta_j S_\nu(z_i,z_j)\cr}
\eqno{(9.H.8)}
$$
where $E(z,w)$ is the prime form, defined in \S{V}, and
$S_\nu$ is the Szeg\"o kernel.
For the torus, these functions are particularly simple:
$$
\eqalign{
E(z,w) &={\vartheta_1(z-w\vert\tau)\over
  \vartheta'_1(0\vert\tau)}\,\,;\cr
S_\nu(z-w) &={\vartheta[\nu](z-w\vert\tau)
  \vartheta'_1(0\vert\tau)\over
\vartheta[\nu](0\vert\tau)\vartheta_1(z-w\vert\tau)}\,\,.\cr}
\eqno{(9.H.9)}
$$

For {\it odd spin structures}, $\nu$, there is now one
complex odd modulus $\chi$ and there are $10$ Dirac
zero modes, one for each space-time direction.
The action $S_X+S_{BC}$ is the one given on in (9.H.1)
but there is now also an insertion of the worldsheet
supercurrent involving
$$
\chi S_{z+}\,\,\, S_{z+}=\left(\psi_+\cdot \partial_z
x-\Half\,b\gamma+{3\over 2}\,\beta\partial_z c+
\partial_z\beta\,c\right)+c.c\,\,.
\eqno{(9.H.10)}
$$
Integrating out $\chi$ will bring down an insertion of
$\psi_+\cdot
\partial_z x$ and of $\psi_-\cdot\partial_{\zbar}x$.
Combining this with the ghost and anti-ghost
insertions, we have
$$
b\bbar\,c\cbar\,\delta(\beta)\delta(\betabar)\delta(\gamma)
\delta(\gammabar)\int S_{z+}\int S_{\zbar-}\,\,.
\eqno{(9.H.11)}
$$

Now, we specialize to correlation functions of massless
$\NS$-$\NS$ states.
As shown in (9.E.5), these operators involve at most
$2$ factors of $\psi_+$.
To obtain a non-zero contribution from odd spin
structures, we need to have at least $5$ external
vertex operators.
For example, the anomaly diagrams with $6$ external
vertex operators receives contributions solely from odd
spin structure.
We shall concentrate on amplitudes with $4$ or fewer
external vertex operators, and odd spin structures will
never contribute to those.
Notice that this means that for these amplitudes, there
is no difference between Type II A, B.
For completeness, we record that for $\nu$ odd, we have
$M_\nu=M_{\nubar}=1$.

It remains to work out the amplitudes with $0$, $1$,
$2$, $3$, $4$ external vertex operators.
We begin with determining the factors $Q_{\nu\nubar}$
of spin structure weights.
Following Mumford, we label $\vartheta$-functions as
follows:
$$
\eqalign{
\vartheta_{00}(z,\tau) &=\vartheta_3(z,\tau)
  =\vartheta(z,\tau)\cr
\vartheta_{01}(z,\tau) &=\vartheta_4(z,\tau)
  =\vartheta\left(z+\Half,\tau\right)\cr
\vartheta_{10}(z,\tau) &=\vartheta_2(z,\tau)
  =e^{i\pi\tau/4+\pi iz}\vartheta\left(z+{\tau\over 2},
  \tau\right)\cr
\vartheta_{11}(z,\tau) &=\vartheta_1(z,\tau)
  =e^{i\pi\tau/4+\pi iz}
  \vartheta\left(z+\Half+{\tau\over 2},\tau\right)\cr
\vartheta(z,\tau) &\equiv\sum\limits_{n=-\infty}^\infty
   e^{i\pi n^2\tau+2\pi inz}\cr}
\eqno{(9.H.12)}
$$
Modular transformations of $\vartheta_2$,
$\vartheta_3$, $\vartheta_4$ obey
$$
\eqalign{
\vartheta_{ab}(0,\tau+1) &=e^{{i\pi\over 4}\,a}
  \vartheta_{a(b+a)}(0,\tau)\cr
\vartheta_{ab}\left(0,-{1\over\tau}\right) &=(-)^{ab}
  \sqrt{-i\,\tau\,\,}\vartheta_{ba}(0,\tau)\cr}
\eqno{(9.H.13)}
$$
The Dedekind $\eta$-function satisfies
$$
\eqalign{
\eta(\tau+1)^{12} &=-\eta(\tau)^{12}\cr
\eta\left(-{1\over\tau}\right)^{12} &=\tau^6
  \eta(\tau)^{12}\cr}
\eqno{(9.H.14)}
$$
Thus
$$
\eqalign{
M_{ab}(\tau+1) &=-e^{i\pi a}M_{a(b+a+1)}(\tau)\cr
M_{ab}\left(-{1\over\tau}\right) &=-{1\over \tau^4}
  M_{ba}(\tau)\,\,.\cr}
\eqno{(9.H.15)}
$$
We must now require modular invariance of the
amplitude:
$$
\eqalignno{
&\sum\limits_{(a,b)}Q_{(ab)\nubar}M_{(ab)}
(\tau+1)=
\sum\limits_{(a,b)} Q_{(a,b)\nubar}M_{(ab)}
(\tau) &(9.H.16)\cr
&\sum\limits_{(a,b)}Q_{(ab)\nubar}M_{(ab)}(-1/\tau)=-
{1\over\tau^4}\sum\limits_{(a,b)}
Q_{(a,b)\nubar}M_{ab}(\tau)\,\,.\cr}
$$
(The factor of $-1/\tau^4$ is compensated for by the
$p$-integral.)
Clearly from $\tau\to -1/\tau$, we get
$$
Q_{(01)\nubar}=Q_{(10)\nubar}
\eqno{(9.H.17)}
$$
while from $\tau\to\tau+1$, we get
$$
Q_{(01)\nubar}=-Q_{(00)\nubar}
\eqno{(9.H.18)}
$$
Setting 
$$
Q_{(01)\nubar}=Q_{(10)\nubar}=-Q_{(00)\nubar}
\eqno{(9.H.19)}
$$
we fix the overall phase of the amplitude, which is
unobservable anyway.
One proceeds similarly for $\nubar$.

It is now straightforward to compute the {\it $0$-point
function}.
(Recall that in the bosonic string this object
produced the first divergence we had encountered.)
Since there are no vertex operator insertions, we simply
have
$$
\eqalignno{
&A_1(0-pt)=\Vol(M)\int_{\scrM_1}\,
{d^2\tau\over \tau_2^4}\,\sum\limits_{\nu\nubar}
Q_{\nu\nubar}M_\nu M_{\nubar}&(9.H.20)\cr
\noalign{\hbox{It suffices to examine the combination}}
&\sum\limits_{\nu}Q_{\nu\nubar}M_\nu=
  Q_{(01)\nubar}M_{(01)}+Q_{(10)\nubar}
M_{(10)}+Q_{(00)\nubar}M_{(00)}& (9.H.21)\cr}
$$
Notice that since $M_{(11)}=0$ the corresponding
contribution has been omitted in (9.H.21).
Now, using the expressions for $M_\nu$, and the equality for
$Q$'s, we find
$$
\sum\limits_{\nu}Q_{\nu\nubar}
={Q_{(01)\nubar}\over\eta(\tau)^{12}}\left[
\vartheta_4(0\vert\tau)^4+\vartheta_2(0\vert\tau)^4-
\vartheta_3(0\vert\tau)^4\right]=0
\eqno{(9.H.22)}
$$
by a famous Jacobi $\vartheta$-function identity.
This identity arose once before in
matching bosonic and fermionic multiplicities at all
mass levels.
In fact, this cancellation is the result of space-time
supersymmetry.
The $0$-point function, and thus the $1$-point
functions vanish.
Recall from the discussion of the bosonic string that
the vanishing of the $1$-point function means that
Minkowski flat space-time is a one-loop solution to the
superstring equations!

Notice that our argument for cancellation was effected
only in the left-moving sector; thus, the same argument
holds true in heterotic string theory as well.

The above summation identity on $\vartheta$-functions is
only one of a series, usually called the
Riemann identities on theta functions.
It is most useful here to express two more of these
identities in
terms of the Dirac propagator (i.e. the Szeg\"o
kernel):
$$
\eqalign{
&\sum\limits_{\nu}Q_{\nu\nubar}M_\nu
S_\nu(z_1-z_2)S_\nu(z_2-z_1)=0\cr
&\sum\limits_{\nu}Q_{\nu\nubar}M_\nu
S_\nu(z_1-z_2)S_\nu(z_2-z_3)S_\nu
  (z_3-z_1)=0\cr}
\eqno{(9.H.23)}
$$
Without giving details here, one similarly shows with the help 
of these identities that 
the two- and three-point functions vanish.
In physics terms, this means that to one loop order,
there is neither mass renormalization (massless
particles remain massless) nor coupling constant
renormalization.
These results again arise from space-time
supersymmetry, and are usually referred to as 
{\it non-renormalization theorems}. 

\bigskip\noindent
\Item{\bf I)} {\bf One-loop amplitudes in the heterotic
string}

Let us consider the free fermion representation of the
heterotic strings, starting from $32$ Majorana-Weyl
fermions $\lam^{\alpha}$, represented by $32$
Euclidean Weyl spinors $\lam_-^\alpha(\zbar)$.
For simplicity, consider amplitudes with external
states that are neutral under the gauge group.
Then the one loop contribution of the internal degree of
freedom will result entirely from the determinantal
factors arising from integrating out $\lam_-^\alpha$.
These depend upon the spin structures, just as the ones
for space-time degrees of freedom $\psi_-^\mu$.

For the gauge group $\Spin(32)/Z_2$, all
$\lam_-^\alpha$ have the same spin structure, and the
determinant for given spin structure contributes a
factor
$$
\Mbar_\nubar^4={\overline{\vartheta[\nubar](0\vert\tau)}^{16}
\over \overline{\eta(\tau)}^{48}}
\eqno{(9.I.1)}
$$
The only modular invariant we can form is with all
coefficients $Q_{\nubar}$ equal for all $3$ even spin
structures.

For the gauge group $E_8\times E_8$, $\lam_-^\alpha$
are divided into two classes of $16$.
For each class, the determinant is
$$
\Mbar_\nubar^2={\overline{\vartheta[\nu](0\vert\tau)}^8\over
\overline{\eta(\tau)}^{24}}\,\,.
\eqno{(9.I.2)}
$$
The only modular invariant is
$$
\sum\limits_{\nubar}\Mbar_\nu^2
={\overline{\vartheta_2(0\vert\tau)}^8+
\overline{\vartheta_3(0\vert\tau)}^8+
\overline{\vartheta_4(0\vert\tau)}^8\over
\overline{\eta(\tau)}^{24}}\,\,.
\eqno{(9.I.3)}
$$
One can show, using relations on the number of
representations of an integer as a sum of squares, that
this function equals the $\vartheta$-function for the
$E_8$ lattice: $\vartheta_{E_8}(0\vert\tau)$.
Furthermore, the partition functions for $\Spin(32)/Z_2$
and $E_8\times E_8$ are the same in view of
$$
\sum\limits_{\nubar}\Mbar_\nubar^4=\biggl(\sum\limits_{\nubar}
\Mbar_\nubar^2\biggr)^2\,\,.
\eqno{(9.I.4)}
$$

We see that no other modular invariant combinations are
allowed.
You may think that $\sum\limits_{\nu}Q_\nu M_\nu=0$ is
allowed, but this would mean that some states enter
with negative weight, and this is impossible since
there are no fermions here!
This argument shows now that $E_8\times E_8$ and
$\Spin(32)Z_2$ are indeed the only possibilities, as we
had promised.

\bigskip\noindent
\Item{\bf J)} {\bf The $\NS$-$\NS$ $4$ point function}

It remains to discuss the $4$-point functions, again
for massless $\NS$-$\NS$ states.
Clearly, the $4$-point function cannot vanish, lest the
theory be non-interacting.
It is evaluated by making use of a Riemann identity on
$\vartheta$-functions with $4$ Szeg\"o kernels:
$$
\sum\limits_{\nu}Q_{\nu\nubar}M_\nu
S_\nu(z_1-z_{\sigma(1)})S_\nu(z_2-z_{\sigma(2)})
S_\nu(z_3-z_{\sigma(3)})S_\nu(z_1-z_{\sigma(4)})
=16\pi^4 Q_{(00)\nubar}
\eqno{(9.J.1)}
$$
where $\sigma(i)$ is a permutation of $i$, 
{\it without fixed points}.

The final result for the $4$-point function can be
conveniently expressed as follows:
$$
\eqalignno{
&A_1(k_i,\eps_i)=\delta(k)\eps^{1\1bar}\eps^{2\2bar}
\eps^{3\3bar}\eps^{4\4bar}K^{1234}K^{\1bar\2bar\3bar\4bar}
A_1(s,t)&(9.J.2)\cr
\noalign{\hbox{where the single Lorentz-invariant
function $A_1(s,t)$ is given by}}
&A_1(s,t)=\half \int_{\scrM_1}{d^2\tau\over \tau_2^2}
\int_{\Sigma}{d^2 z_1\over \tau_2}\int_{\Sigma}
{d^2 z_2\over\tau_2}\int_{\Sigma}
{d^2 z_3\over\tau_2}e^{_{s\over
2}(G_{12}+G_{34}-G_{13}-G_{24})}\cr
&\null\kern8.40 true cm e^{+{t\over 2}
(G_{23}+G_{14}-G_{13}-G_{24})}&(9.J.3)\cr}
$$
where $G_{ij}$ stands for the $1$-loop propagator of
the $x$-field:
$$
G_{ij}=G(z_i-z_j)\vert\tau)
\equiv -\ln\,\left\vert{\vartheta_1(z_i-z_j\vert\tau)\over
\theta'_1(0\vert\tau)}\right\vert^2+{2\pi
\over \tau_2}\,\im\,(z_i-z_j)^2
\eqno{(9.J.4)}
$$
and $K$ is the kinematical factor that also arose in
the tree-level amplitudes $A_0(k_i,\Eps_i)$ for
$\NS$-$\NS$ transition amplitudes.
Recall that $K$ is polynomial in moments.
By translation invariance on the torus, the point $z_4$
may be chosen at will, say at $z_4=0$.

It remains to study the function $A_1(s,t)$!
We begin by examining the convergence structure of the
above functional representation as a function of $s$
and $t$.

\medskip
\Item{(1)}
For fixed $\tau$, and $z_i\to z_j$, $i\not=j$, we have
$$
G(z_i-z_j\vert\tau)\to -\ln\,\vert z_i-z_j\vert^2\to
+\infty
\eqno{(9.J.5)}
$$
\Item{(2)} 
while for $z_i$'s well separated, but
$\im\,\tau\to\infty$, we have
$$
G(z_i-z_j\vert\tau)\to-\tau_2\qquad\qquad
\to-\infty\,\,.
\eqno{(9.J.6)}
$$

\noindent
The latter limit imposes the strictest convergence
condition:
$$
\re(s),\,\,
\re(t),\,\,
\re(u)\Ge 0
\eqno{(9.J.7)}
$$
But, since we also have
$s+t+u=0$ for on-shell massless external states, the
convergence condition (9.J.7) requires 
$\re(s)=\re(t)=\re(u)=0$.
It is easily seen that this condition is sufficient for
convergence, since the integral is bounded by the value
at $s=t=0$.
This means that the integral is absolutely convergent
for at least some values of the external momenta, in
contrast with the case of the bosonic string, where the
integral representation is never convergent for any
momenta.
Yet, the integral representation is defined only for
unphysical momentum configurations where $s$, $t$, $u$
are purely imaginary.

A second shortcoming of the integral representation is
that formally, when $s$, $t$, $u$ are close to the
physical region corresponding to real values of $s$ and
$t$, $A_1(s,t)$ would be real.
This would be inconsistent in an interacting theory,
where the $1$-loop $4$-point  function cannot be real,
since its imaginary part is related by unitarity to the
square of the tree level amplitude, which cannot vanish
in any intracting theory.

Our task is thus to properly define the amplitude
throughout $(s,t)\in\dbC\times\dbC$, including around
the physical region where $(s,t)\in\dbR\times\dbR$.
Naturally, one seeks an analytic continuation of the
amplitude in both $s$ and $t$.
(Recall that even at tree level, the integral
representation $A_0(s,t)$ was absolutely convergent
only in a small region, and the full amplitude had to
be analytically continued in $s$ and $t$ as well.)

To appreciate the physical meaning of such an analytic
continuation, we compare with the situation in quantum
field theory.
There, the physical principle of causality is
equivalent to the requirement of locality of observable
fields, and this property, combined with Lorentz
invariance, implies analytic dependence of the Green
functions on the external momenta.
Often, this analytic behavior is conveniently
summarized in the form of a spectral representation.
For example, the $2$-point function (off-shell)
may be expressed as

$$
\lower10pt\hbox{\vbox{\epsfxsize=1.5in\epsfbox{fig1.eps}}}
\qquad H(s)=\int_0^\infty dM^2\,{\rho(M^2)\over
  s-M^2+i\eps}\quad\eps>0\,\
\eqno{(9.J.8)}
$$

\noindent
where the real and positive function $\rho(M^2)$ is the
density of states at mass $M$, also called the spectral
density.
Provided the integral over $M^2$ converges, it follows
that $H(s)$ is analytic throughout $\dbC$, except
possibly on the positive real axis, where it may have
poles and branch cuts.

In string theory, we do not have available at present a
formulation of the amplitudes in terms of local field
observables (i.e. commuting at space-like separations).
It is thus not known \`a priori whether analyticity of
the amplitudes can be derived solely from Lorentz
invariance and causality.
(Probably, in a good second quantized formulation, one
should be able to derive this.)
Thus, analyticity of string amplitudes must be
established, and the questions that must be answered
are

\medskip
\item{(1)} does an analytic continuation exist?
\item{(2)} is the analytic continuation unique?
\item{(3)} is the analytic continuation physically
acceptable?

\medskip\noindent
The present problem of analytic continuation of
$A_1(s,t)$ is a difficult matter of pure
analysis. 
Yet, all three questions above have been answered in
the affirmative, so that one has a full analytically
continued amplitude $A_1(s,t)$ available, defined
throughout $(s,t)\in\dbC\times\dbC$, with cuts and
poles on $\dbR\times\dbR$, but nowhere else.

What we have achieved here is the construction of a
{\it finite} and {\it unitary} 
amplitude, at least to one loop order, that
defines $\NS$-$\NS$ string scattering!
We shall not reproduce the details of the analytic
continuatin procedure here.
We shall limit ourselves to describing the physical
singularities that appear in $A_1(s,t)$ upon analytic
continuation.
To do so it is most convenient to represent the
different singularities that appear as a function of
$s$ and $t$ in terms of QFT Feynman diagrams.
The Feynman diagrams indicate the region in moduli
space (including the positions of vertex operators
$z_i$) that produce the singularities.
The singularities essentially arise from combinations
of the singular limits of $G_{ij}$ in (9.J.5) and
(9.J.6), and fall into five categories, schematically
indicated by five different types of Feynman diagrams
below.
Whenever two external $z_i$ and $z_j$ approach one
another the Feynman diagram representation will show
that their external lines merge
$$
\vbox{\epsfxsize=2.5in\epsfbox{merge.eps}}
$$
This singularity may be treated at the operator level
by considering the operator product expansion (OPE) of
the two external vertex operators.
The internal  line then represents all the operators
that arise in the OPE.
This type of singularity produces a pole in the total
momentum traversing the internal line, i.e. in
$-(k_i+k_j)^2$, and the pole turns out to be at the
value of the $\mass^2$ of the physical string states,
i.e. $2n=M^2$, $n\in\dbZ^+$.
The five different categories represented by diagrams
(a), (b), (c), (d) and (e) arise, respectively, from
$4$, $3$, $2$, $1$ and $0$ external lines coming close
together.
Within each category, there are additional
singularities (represented by a dotted line aross the
loop) that arise from the asymptotic behavior in
$\tau$, given by (9.J.6).
$$
\halign{#\hfill &#\hfill &#\hfill\cr
(a)&\qquad\qquad
    \lower20pt\hbox{\epsfxsize=2.00in\epsfbox{fig-a.eps}}&\cr
\noalign{\bigskip}
(b)&\qquad\qquad
    \lower20pt\hbox{\epsfxsize=2.00in\epsfbox{fig-b.eps}}&\cr
\noalign{\bigskip}
(c)&\qquad\qquad
    \lower30pt\hbox{\epsfxsize=2.00in\epsfbox{fig2.eps}} 
    &\item{$\bullet$}double poles in $s$ at $s=2n$\cr
\noalign{\vskip-25pt}
    &&\item{$\bullet$}branch cuts in $s$, starting at\cr
    && \item{}\qquad $s=2n$, \quad $n\in\dbZ^+$.\cr
%\noalign{\vfill\eject}
(d)&\qquad\qquad
    \lower30pt\hbox{\epsfxsize=1.5in\epsfbox{fig3.eps}}
  &\item{$\bullet$}single poles in $s$ at $s=2n$\cr
\noalign{\vskip-25pt}
  &&\item{$\bullet$}branch cuts in $s$, starting at\cr
  &&\item{}\qquad $s=2n$,\quad $n\in\dbZ^+$.\cr
\noalign{\medskip}
(e)&\qquad\qquad
    \lower40pt\hbox{\epsfxsize=1.50in\epsfbox{fig4.eps}} 
  &\item{$\bullet$}no poles\cr
\noalign{\vskip-35pt}
  &&\item{$\bullet$}branch cuts in $s$, starting at \cr
\noalign{\vskip-5pt}
 &&\item{}\qquad $s=2n$, \quad $n\in\dbZ^+$\cr
\noalign{\vskip-5pt}
 &&\item{}with spectral density $\rho(2n,t)$\cr
\noalign{\vskip-5pt}
 &&\item{$\bullet$}$\rho(2n,t)$ has branch cuts in $t$,\cr
\noalign{\vskip-5pt}
 &&\item{}starting at $t=2m$, $m\in\dbN^0$.\cr}
$$
It is understood that within each category, we also
include permutation of the diagrams shown, for example
by adding diagrams on which the external legs have been
permuted.

Now, for superstrings, diagrams in categories (a) and
(b) vanish identically.
This follows from the fact that the on-shell $1$- and
$2$-point functions for massless states vanish
identically.
The singularities of the remaining diagrams
are precisely the ones that one
would expect on physical grounds, and their structure
confirms the consistency of the string theory up to this order.
A complete proof is in E. D'Hoker and D. H. Phong, ``The
Box Graph in Superstring Theory'', Nucl. Phys. B440
(1995), 24.











\bye



