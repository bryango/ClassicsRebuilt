%From: Lisa C Jeffrey <jeffrey@IAS.EDU>
%Date: Wed, 27 Nov 1996 14:34:43 -0500
%Subject: Faddeev Lecture 5

\documentstyle[12pt]{article}
\input amssym.def
\input amssym.tex
\newcommand{\Tr}{\,{\rm Tr}\,}
\newcommand{\tr}{{\rm tr}\,}

%L. Jeffrey preamble, 5 April 1996

\newcommand{\nc}{\newcommand}

\nc{\isq}{{i}}
\newcommand{\colvec}[2]{\left  ( \begin{array}{cc} #1  \\
     #2  \end{array} \right ) }

%%%%%\newcommand{\Tr}{\,{\rm Tr}\,}
\newcommand{\End}{\,{\rm End}\,}
\newcommand{\Hom}{\,{\rm Hom}\,}

\newcommand{\Ker}{ \,{\rm Ker} \,}

\newcommand{\bla}{\phantom{bbbbb}}
\newcommand{\onebl}{\phantom{a} }
\newcommand{\eqdef}{\;\: {\stackrel{ {\rm def} }{=} } \;\:}
\newcommand{\sign}{\: {\rm sign}\: }
\newcommand{\sgn}{ \:{\rm sgn}\:}
\newcommand{\half}{ {\frac{1}{2} } }
\newcommand{\vol}{ \,{\rm vol}\, }


% define abbreviations for most common commands
%

\newcommand{\beq}{\begin{equation}}
\newcommand{\eeq}{\end{equation}}
\newcommand{\beqst}{\begin{equation*}}
\newcommand{\eeqst}{\end{equation*}}
\newcommand{\barr}{\begin{array}}
\newcommand{\earr}{\end{array}}
\newcommand{\beqar}{\begin{eqnarray}}
\newcommand{\eeqar}{\end{eqnarray}}
\newtheorem{theorem}{Theorem}[section]
%\newtheorem{conjecture}{Conjecture}
\newtheorem{corollary}[theorem]{Corollary}
%\newtheorem{problem}{Problem}
\newtheorem{lemma}[theorem]{Lemma}
\newtheorem{prop}[theorem]{Proposition}
\newtheorem{definition}[theorem]{Definition}
\newtheorem{remit}[theorem]{Remark}
\newtheorem{conjecture}[theorem]{Conjecture}

\newtheorem{example}[theorem]{Example}

\newcommand{\matr}[4]{\left \lbrack \begin{array}{cc} #1 & #2 \\
     #3 & #4 \end{array} \right \rbrack}



\newenvironment{rem}{\begin{remit}\rm}{\end{remit}}




% black board bold face
%note \AA is already defined!
\newcommand{\aff}{{ \Bbb A }}
\newcommand{\RR}{{{\bf  R }}}
\newcommand{\CC}{{{\bf  C }}}
\nc{\FF}{ {\Bbb F} } 
\newcommand{\ZZ}{{{\bf   Z }}}
\newcommand{\PP}{ {\Bbb P } }
\newcommand{\QQ}{{\Bbb Q }}
\newcommand{\UU}{{\Bbb U }}








%***************************




%


%


%Replace greek letters by their roman equivalents with \
%Slightly nonstandard:  theta is \t, tau is \ta, no omicron
\def\a{\alpha}
\def\b{\beta}
\def\g{\gamma}
\def\d{\delta}
\def\e{\epsilon}
\def\z{\zeta}
\def\h{\eta}
\def\t{\theta}
%\def\i{\iota}
\def\k{\kappa}
\def\l{\lambda}
\def\m{\mu}
\def\n{\nu}
\def\x{\xi}
\def\p{\pi}
\def\r{\rho}
\def\s{\sigma}
\def\ta{\tau}
\def\u{\upsilon}
\def\ph{\phi}
\def\c{\chi}
\def\ps{\psi}
\def\o{\omega}

\def\G{\Gamma}
\def\D{\Delta}
\def\T{\Theta}
\def\L{\Lambda}
\def\X{\Xi}
\def\P{\Pi}
\def\S{\Sigma}
\def\U{\Upsilon}
\def\Ph{\Phi}
\def\Ps{\Psi}
\def\O{\Omega}




% calligraphic letters
\newcommand{\calA}{{\mbox{$\cal A$}}}
\newcommand{\calB}{{\mbox{$\cal B$}}}
\newcommand{\calC}{{\mbox{$\cal C$}}}
\newcommand{\calD}{{\mbox{$\cal D$}}}
\newcommand{\calE}{{\mbox{$\cal E$}}}
\newcommand{\calF}{{\mbox{$\cal F$}}}
\newcommand{\calG}{{\mbox{$\cal G$}}}
\newcommand{\calH}{{\mbox{$\cal H$}}}
\newcommand{\calI}{{\mbox{$\cal I$}}}
\newcommand{\calJ}{{\mbox{$\cal J$}}}
\newcommand{\calK}{{\mbox{$\cal K$}}}
\newcommand{\calL}{{\mbox{$\cal L$}}}
\newcommand{\calM}{{\mbox{$\cal M$}}}
\newcommand{\calN}{{\mbox{$\cal N$}}}
\newcommand{\calO}{{\mbox{$\cal O$}}}
\newcommand{\calP}{{\mbox{$\cal P$}}}
\newcommand{\calQ}{{\mbox{$\cal Q$}}}
\newcommand{\calR}{{\mbox{$\cal R$}}}
\newcommand{\calS}{{\mbox{$\cal S$}}}
\newcommand{\calT}{{\mbox{$\cal T$}}}
\newcommand{\calU}{{\mbox{$\cal U$}}}
\newcommand{\calV}{{\mbox{$\cal V$}}}
\newcommand{\calW}{{\mbox{$\cal W$}}}
\newcommand{\calX}{{\mbox{$\cal X$}}}
\newcommand{\calY}{{\mbox{$\cal Y$}}}
\newcommand{\calZ}{{\mbox{$\cal Z$}}}

%% To load script letters:

\font\teneusm=eusm10  \font\seveneusm=eusm7 
\font\fiveeusm=eusm5 
\newfam\eusmfam 
\textfont\eusmfam=\teneusm 
\scriptfont\eusmfam=\seveneusm 
\scriptscriptfont\eusmfam=\fiveeusm 
\def\Scr#1{{\fam\eusmfam\relax#1}}

% Script letters
\newcommand{\ScrA}{{\Scr A}} \newcommand{\ScrB}{{\Scr B}}
\newcommand{\ScrC}{{\Scr C}} \newcommand{\ScrD}{{\Scr D}}
\newcommand{\ScrE}{{\Scr E}} \newcommand{\ScrF}{{\Scr F}}
\newcommand{\ScrG}{{\Scr G}} \newcommand{\ScrH}{{\Scr H}}
\newcommand{\ScrI}{{\Scr I}} \newcommand{\ScrJ}{{\Scr J}}
\newcommand{\ScrK}{{\Scr K}} \newcommand{\ScrL}{{\Scr L}}
\newcommand{\ScrM}{{\Scr M}} \newcommand{\ScrN}{{\Scr N}}
\newcommand{\ScrO}{{\Scr O}} \newcommand{\ScrP}{{\Scr P}}
\newcommand{\ScrQ}{{\Scr Q}} \newcommand{\ScrR}{{\Scr R}}
\newcommand{\ScrS}{{\Scr S}} \newcommand{\ScrT}{{\Scr T}}
\newcommand{\ScrU}{{\Scr U}} \newcommand{\ScrV}{{\Scr V}}
\newcommand{\ScrW}{{\Scr W}} \newcommand{\ScrX}{{\Scr X}}
\newcommand{\ScrY}{{\Scr Y}} \newcommand{\ScrZ}{{\Scr Z}}

%German (Faktur) letters

\newcommand{\grA}{{\frak A}}



\def\eps{\varepsilon}

\setlength{\textwidth}{6.5in}
\setlength{\textheight}{9.1in}
\setlength{\evensidemargin}{0in}
\setlength{\oddsidemargin}{0in}
\setlength{\topmargin}{-.75in}
\setlength{\parskip}{0.3\baselineskip}

%\renewcommand{\theequation}{\thesection.\arabic{equation}}
%\newcommand{\renorm}{{ \setcounter{equation}{0} }}

\newcommand{\Le}{{{\mathchoice{\,{\scriptstyle\le}\,}
  {\,{\scriptstyle\le}\,}
  {\,{\scriptscriptstyle\le}\,}{\,{\scriptscriptstyle\le}\,}}}}
\newcommand{\Ge}{{{\mathchoice{\,{\scriptstyle\ge}\,}
  {\,{\scriptstyle\ge}\,}
  {\,{\scriptscriptstyle\ge}\,}{\,{\scriptscriptstyle\ge}\,}}}}


%\renewcommand{\baselinestretch}{1.5}


%more newcommands
\nc{\bra}{  < }
\nc{\ket}{ > }
%\nc{\isq}{ { \sqrt{-1} } }
%\nc{\isq}{{ i }}
%\nc{\hbar}{{ h}}
\nc{\triang}{ { \bigtriangleup} }

\nc{\astar}{{ a^*}}
\nc{\hata}{ { \hat{a} }}
\nc{\hatastar}{{ \hat{\astar} }}
\nc{\normfac}{{\frac{1}{\sqrt{2 \omega} } }}

\begin{document}

\title{Lecture 5:\\
Quantization of Yang-Mills fields}
\author{Ludwig Faddeev}
\date{12  November  1996}

\maketitle

%\renorm
\nc{\dphi}{\tilde{\phi}} 
\nc{\vac}{{ \sl v}}
\nc{\basvec}[1]{\Phi_{#1} (\astar) } 
\nc{\basvecn}{\basvec{n} } 
\nc{\kn}{(\vec{k})_{n}} 
\nc{\vk}{\vec{k} }

\nc{\vx}{\vec{x} }

This lecture treats Yang-Mills theory, an analogue 
of the field theory for electromagnetism (Example 2 of Lecture 4) in which
the gauge  group  $C^\infty (\RR^4, U(1))$ is replaced by $\calG = C^\infty
(\RR^4, G)$ where $G$ is a nonabelian compact Lie group. 
In this case Feynman discovered the disturbing
fact that if one attempts to treat  this theory 
following the approaches which worked in the case of the electromagnetic
field, one obtains an   S-matrix which is 
 not unitary. 
Here we shall see that the reason for this is a difference of the 
geometry of the gauge orbits, which are linear in the electromagnetic
case but nonlinear for the case of the nonabelian gauge field. 

\nc{\lieg}{ {\rm Lie} (G) }

We begin much as in our treatment of electromagnetism in Lecture 4.
As in electromagnetism, we have a field $A_\mu(\vx, t) $
which is a vector representation of the Lorentz group (in other words, 
it is a one-form on $V = \RR^4$); however each component $A_0, $ $A_1, 
$ $A_2, $ $A_3$ now takes values in the Lie algebra  ${\rm Lie}(G)$, 
so the one-form $A = A_\mu dx^\mu$ 
 takes values in ${\rm Lie}(G)$. (More generally it could
be taken to have values in an arbitrary unitary  representation of $G$, 
but for simplicity we take the representation to be ${\rm Lie}(G)$ endowed
with the adjoint representation.)
An element $g$ of the gauge group $\calG = C^\infty (\RR^4, G)$ acts on $A$ 
by the transformation 
\beq A \mapsto A^g \eqdef g A g^{-1} + (\partial_\mu g) g^{-1} \eeq
The curvature is given by 
\beq \label{curv} F = d A + A \wedge A, \eeq
where if $A = \sum_a X_a \otimes \alpha_a$ for a basis 
$X_a$ of $\lieg$ and a collection of one-forms $\alpha_a$, we 
define
 $$A \wedge A = \sum_{a,b} [X_a, X_b] \alpha_a \wedge \alpha_b. $$
The curvature (which is a $\lieg$-valued two-form on $\RR^4$) 
transforms under the gauge transformation $g$ as 
\beq F \mapsto F^g = g F g^{-1}. \eeq
As in the abelian case, the Lagrangian is
\beq \calL = \frac{1}{\epsilon^2} \Tr (F^2), \eeq
where if $F = \sum_a X_a \otimes \beta_a$ for a collection of two-forms
$\beta_a$, we define
\beq \Tr (F^2) =\sum_{a,b} < X_a, X_b> \beta_a \wedge * \beta_b. \eeq
(A symmetric quadratic form on the space of two-forms on $\RR^4$ is 
defined in the same way, and is also denoted by $\Tr$.) 
Here, we have introduced an ${\rm Ad}$-invariant inner product $ < \cdot, 
\cdot > $ on 
$\lieg$, which in the case of matrix groups is given by the trace
$\Tr$. We have also introduced a coupling constant $\epsilon$. 

This Lagrangian contains terms of the form $(d A)^2$ as well 
as    terms cubic in $A$ (of order $1$ in derivatives) and terms 
quartic in $A$ (with no derivatives). There are thus two types of vertices,
a 3-valent  vertex (corresponding to the term cubic in $A$) and a 4-valent
vertex (corresponding to the term quartic in $A$). By studying the 
corresponding perturbation expansion, Feynman found that the S-matrix 
was not unitary. 

To proceed further, we need to find the physical variables. We write the 
Lagrangian in a form that is first order in derivatives, by the device
introduced in Lecture 4 where we regarded $A$ and $F$ as independent
variables. Thus we see that the Lagrangian is (suppressing the overall
multiplicative constant $1/\epsilon^2$ for the moment)
\beq \calL = \Tr (dA + A^2, F) - \half \Tr F^2. \eeq
As in Lecture 4, we have ten variables
$$ A_0, ~~A_k, ~~F_{0k} = E_k, ~~F_{ik} = \epsilon_{ikl} H_l. $$
Exactly as in Lecture 4, the variables $A_k$ and $E_k$ are canonically 
conjugate, while the $F_{ik}$ (or equivalently the $H_l$) are excludable
variables and $A_0$ is a Lagrange multiplier multiplying a constraint
\nc{\gauss}{\frak{G} }
\beq \label{gauss}
\gauss = \nabla_k E^k = \partial_k E^k + [A_k, E_k]. \eeq
As in Lecture 4 we obtain an  equation of motion  ({\em Gauss's Law}) 
$$ \gauss = 0;  $$
the difference from the case of abelian $G$ (electromagnetism) is the
presence in (\ref{gauss}) of the quadratic term 
$[A_k, E_k]$. 


\nc{\vy}{\vec{y} } 



The canonical commutation relations are
\beq \left \{  A_k^a (\vx), E_b^k(\vy)  \right \} = 
\delta_l^a \delta_b^l \delta^{(3) } (\vx - \vy). \eeq
Here, $a$ and $b$ index a basis $\{ X_a\} $ of the Lie algebra $\lieg$ and 
we have decomposed 
$A_k = \sum_a A_k^a \otimes X_a $ for functions $A_k^a$ which form a 
one-form $A^a$ on $\RR^3$. 


We introduce a source term $\gauss(\eta) = \int \Tr
 \gauss(\vx) \eta(\vx) d\vx $
(where $\eta : \RR^3 \to \lieg$ is an element of the Lie algebra of the
gauge group). 
We find that the Poisson brackets of the source terms are 
\beq \left \{ \gauss(\eta_1), \gauss(\eta_2)  \right \} 
= \gauss ([\eta_1, \eta_2]), \eeq
in other words the map $\eta \mapsto 
\gauss(\eta)$  (which may be regarded as the 
momentum map for the gauge group action) is a homomorphism of Lie algebras 
from 
the Lie algebra of the gauge group (equipped with the Lie bracket
induced by $[, ]$) to the Lie algebra of functions (equipped with the 
Poisson bracket). Equivalently, the constraints are a representation
of the Lie algebra of the gauge group. The constraint
$\gauss$ generates gauge transformations through the Poisson bracket:
\beq \left 
\{ \gauss(\eta), A_k(x) \right \} = 
\partial_k \eta + [A_k, \eta ]. \eeq
The Hamiltonian is 
$$ \calH = \half (E^2 + H^2). $$
As in the case of electromagnetism, the variables $F_{ik}$ are excludable. 
Likewise as in the case of electromagnetism, the constraint equations are
first class: we must set $\gauss = 0 $ and also the 
Poisson brackets $\{ \gauss, f \}  $ (for arbitrary
functions $f$) must all be set to zero.

The observables  the reduced system are thus functionals
\beq \phi(A, E)|_{\gauss = 0 } \eeq 
subject to the condition 
\beq \{ \gauss, \phi \} = 0 . \eeq

Reduction and solving constraints break the manifest Lorentz 
invariance. So if we wish to retain this invariance, we must learn
how to describe dynamics (the functional integral) without explicitly
solving the constraints. 
To understand how to treat a simpler case,  
let us return to a finite dimensional phase space $\Gamma $,
 and recall
the treatment of this situation as at the beginning of Lecture 4. 
We assume we have a Lagrangian 
\beq l = \sum_{i = 1}^N p_i dq_i - H + \sum_{\alpha = 1}^M 
\lambda_{\alpha} \varphi^\alpha;  \eeq
here we assume that $M  < N $ (so that 
the dimension $2(N-M)$ of the reduced phase
space is positive).  To parametrize the physical phase 
space, we take 
\beq \tilde{\Gamma} = \{(p, q) \in \Gamma: \varphi^\alpha(p,q) = 0 
~\mbox{for all} ~\alpha \}/\calG \eeq
where $\calG$ is the gauge group generated by the constraints
$\varphi^\alpha$ through the Poisson bracket. (In the Yang-Mills example
the action of the constraints via
Poisson bracket  actually  does exponentiate to  a group action,
though this is not always true more generally.) 
To parametrize the physical degrees of freedom we introduce a slice 
\beq \label{slice} \chi_\beta (p,q) = 0 , ~~\alpha = 1, \dots, M \eeq
which intersects each orbit of $\calG$ exactly once: 
the slice (\ref{slice}) is called a {\em subsidiary condition}. 
We choose the subsidiary conditions $\chi_\alpha$ to satisfy
\beq \{ \chi_\alpha, \chi_\beta \} = 0 . \eeq
In order to obtain a nondegenerate symplectic structure, we require 
also that 
\beq \det (\{ \varphi_\alpha, \chi^\beta \} ) \ne 0. \eeq
Imposing the subsidiary conditions (\ref{slice}) and also 
the constraint equations $$ \varphi_\alpha (p, q) = 0 $$ 
we find a smaller phase space $\Gamma^*$. We may 
take local coordinates on the big
phase space $\Gamma $ given by 
$$ q = (\chi^\alpha, q^*) $$ and 
$$ p = (p^\alpha, p^*); $$ 
here $p^*$ and $q^*$ are the 
physical variables, while 
$p^\alpha $ and $\chi^\alpha$ must be excluded. 
Let us write the Liouville measure as 
\beq \label{meas} 
dp^* dq^* = \delta(\chi) \delta (\varphi) \det (\{ \varphi, 
\chi \} ) dp dq;  \eeq
if we were able to solve 
\beq\varphi(p,q) = 0
 \Longleftrightarrow p^\alpha = p^\alpha (p^*, q^*) \eeq
this would correspond to the transformation of the delta function 
under a change of variables, 
\beq \delta (\varphi) = (\partial \varphi/\partial p )
\delta (p - p (p^*, q^*) )
\eeq
which would give rise to the determinant in  (\ref{meas}). 

We additionally make use of the fact that 
$$ \delta (\varphi) =  \int e^{i \lambda \varphi} d\lambda,  $$
so that we can omit the delta function $\delta(\varphi)$ if the 
action already contains a term $\sum_\alpha\lambda_\alpha  \varphi^\alpha$. 
Thus the functional integral giving rise to the evolution operator is 
\beq \int \exp \frac{i}{\hbar} \int (p \dot{q} - H - \lambda \varphi) dt 
\prod_t \frac{ dp(t) dq(t) d\lambda(t)}
{ 2 \pi \hbar}  \delta (\chi) \det (\{ \chi, \varphi \} ) . \eeq
To pass to the $S$-matrix only the {\em physical} degrees of freedom are
fixed at the asymptotic values of $t$, while the Lagrange multiplier and
the values of the constraints are allowed to vary freely. 
Returning to the Yang-Mills case, we get the path integral 
\beq \label{pint} \int \exp \left \{ \frac{i}{\hbar} 
\int_{\RR^4}  (  (\partial_0 A) E - \half (E^2 + H^2) 
- A_0 (\nabla\cdot E) ) dx  \right \}   \prod_t \prod_{\vx} 
\delta (\partial_k A_k) (\det M_C)  d A_\mu d E_k ,  \eeq
where 
as our subsidiary condition we have taken the {\em Coulomb gauge} condition,
\beq \label{subcon} \partial_k A_k = 0 , \eeq
and have introduced the operator $M_C$ on $\lieg$-valued functions
 on $\RR^3$, defined by 
\beq \label{mcoul} M_C  \eta = - \partial_k \partial_k \eta  
- [A_k, \eta]. \eeq

The only objects appearing in   (\ref{pint}) which are not Lorentz invariant
are $\det M_C$ and $\partial_k A_k$. 


We can explicitly integrate in (\ref{pint}) over the variables 
$E$ to get 
\beq  \int \exp \left \{ \frac{i}{\hbar} \int_{\RR^4}
 (F_{\mu \nu})^2  ) dx 
\right \} 
\prod_x d A_\mu(x) \prod_t \prod_{\vx}   \delta (\partial_k A_k) 
\det M_C (t)  . \eeq
We interpret this expression as an integral over $\calA/\calG$, where
$\calA$ is a space of connections and $\calG$ is a gauge group. 
 Our subsidiary condition 
(\ref{subcon}) is the choice of a point on the orbit of $\calG$: 
the determinant $\det M_C$ gives the correct measure. 

To check that we have indeed obtained the correct measure on the space
of $\calG$ orbits reduced from the measure $\prod_{x} \prod_\mu dA_\mu(x)$, 
it suffices to check that
\beq \label{meas2} 
\int \delta (\partial_k A_k^g)  \prod_x \prod_{g \in \calG}  
d g_x|_{\partial_k A_k = 0 } 
= \frac{1}{\det M_C}; \eeq
here $dg_x $ is the Haar measure on $G$ at the point $g_x $. 
We can linearize the action of $\calG$ so that infinitesimally
we obtain 
$$ \int \delta  (M_C \eta) 
\prod_x  d\eta_x = \frac{1}{\det M_C} $$
where we have defined the operator $M_C $ by (\ref{mcoul}).
After this understanding we can proceed to choose a different 
subsidiary condition, in particular a manifestly Lorentz invariant
one 
\beq\partial_\mu A_\mu  = c,\eeq
where $c$ is an arbitrary scalar function. The reduced measure is 
given by 
$$ \det M_L \prod_x 
 \delta (\partial_\mu A_\mu - c(x) ) \prod dA_\mu(x)  $$ 
where $\det M_L$ is obtained by averaging the 
$\delta$-function 
$$ \frac{1}{\det  M_L   } = \int 
 \delta (\partial_\mu A_\mu^g  - c(x) ) \prod_x  dg(x). $$ 
Here,  we have  defined the operator $M_L$ (compare the 
definition  (\ref{mcoul}) of $M_C$) by 
\beq \label{mlor}
 M_L \eta = \square \eta   + \partial_\mu [A_\mu, \eta ].\eeq

The path integral  now becomes 
\beq \label{astnotes} \int \exp  \left \{ \frac{i}{\hbar} 
\int (F_{\mu \nu}^2 dx ) \right \} 
\prod_x \delta (\partial_\mu A_\mu - c) (\det M_L) 
\prod_x d A_\mu(x). \eeq

The final step is to average over $c(x)$. 
Ultimately the expression (\ref{astnotes}) becomes
\beq \label{pathint2}
\int \exp \left \{  \frac{i}{\hbar}  \int_{\RR^4} (F_{\mu \nu})^2  d x 
\right \} 
\prod_x \delta (\partial_\mu A_\mu(x)  - c) 
(\det M_L) \prod d A_\mu(x)  \prod_c e^{i \alpha \int c^2 d x} \eeq
\beq = \int \exp\left \{  \frac{i}{\hbar} \int_{\RR^4}
  (F_{\mu \nu})^2  + \alpha 
(\partial_\mu A_\mu)^2  )dx \right \}  \prod_x  (\det M_C) \prod d A_\mu(x).
 \eeq
Here, $\alpha$ is an arbitrary constant. 
The integrand in  (\ref{pathint2}) is not local (because the determinant
of the operator $M_C$ is not local): to transform it into a local 
expression we introduce Grassmann variables $c = 
(c_1, \dots, c_r)  $ and $\bar{c} = (\bar{c}_1, \dots, 
\bar{c}_r),  $ 
in other words
odd variables in a Grassmann algebra, satisfying the anticommutation 
relations
$$ c_i \bar{c}_j +  \bar{c}_j c_i = 0,  $$
$$ c_i^2 = \bar{c_i}^2  = 0 $$ 
 and recall
the identity
\beq \label{grassdet} \det A = \int e^{\bar{c}  \frak{A}  c} d c^* dc \eeq
where $A$ is an endomorphism of a vector space of dimension $r$ 
 and we have introduced the corresponding
endomorphism of a vector space of dimension $2r$,
\beq \frak{A} = \matr{0}{A}{A}{0}. \eeq
We now replace the Grassmann variables $c$ and $\bar{c}$ (which took 
values in a finite dimensional vector space) by Grassmann variables 
$c$ and $\bar{c}$ taking values in the infinite
dimensional space of  functions on $\RR^4$. 
Rewriting the determinant in 
(\ref{pathint2}) by formally applying
(\ref{grassdet}),  (\ref{pathint2}) is transformed into the form
\beq \label{pathint3} \int \exp \left \{ \frac{i}{\hbar \epsilon^2} 
\Tr \int_{\RR^4} \Bigl (  (F_{\mu \nu})^2 + \alpha (\partial_\mu A^\mu)^2 + 
(\partial_\mu \bar{c}, \partial_\mu c + [A_\mu, c] \Bigr )dx  \right \}
\prod dA_\mu d c d \bar{c}. \eeq
We have now  transformed the path integral into an integral of the 
form $\int \exp (i {\rm Action} ) $, where the action is 
local and has a nondegenerate quadratic form, and we integrate over all
fields (which are assumed to satisfy appropriate boundary conditions). 
Here $c$ and $\bar{c} $ are  fields which should be regarded as carrying
charge (because of the terms mixing $A_\mu$,  $c$ and $\bar{c}$ ) and which 
should be regarded as fermionic (because  they are  odd elements in a 
superalgebra or Grassmann algebra).
The expression (\ref{pathint3}) is the generating functional 
from which the $S$-matrix of Yang-Mills theory may be extracted.
There are still the 3-valent and 4-valent vertices in which 
all the lines emanating from the vertex correspond to the propagator
of $A$. Additionally we now have another
 3-valent vertex with one $A$ propagator, one $c$ propagator 
and one $\bar{c}$ propagator: the $c$ and $\bar{c} $ propagators
may be distinguished by writing an arrow on the 
$c$ propagator pointing toward the vertex and an arrow on the 
$\bar{c}$ propagator pointing away from the vertex. 

Near a classical solution of the equations of motion for the 
Yang-Mills fields, the correction terms entering the perturbation
expansion are proportional to ${F_{\mu \nu}^2}$ 
(recall that a term of the form ${F_{\mu \nu}^2}$  appeared in our
original Lagrangian).
The term $\frac{1}{\epsilon^2} (F_{\mu \nu}^2 ) $ in the Lagrangian
is renormalized to 
$$(\frac{1}{\epsilon^2} + a {\rm ln} (\Lambda)  )
 F_{\mu \nu}^2   $$ for an appropriate cutoff parameter
$\Lambda$ and a suitable constant $a$. We notice that $a$ must be negative
in order for the renormalization procedure to succeed:
this property is referred to in the literature as ``asymptotic freedom''.






 

%%%%%\begin{thebibliography}{99}
%%%%%\bibitem{FP} L.D. Faddeev, V.N. Popov, {\em Phys. Lett.}
%%{\bf B25}, 30, 1967.

%%%%%\bibitem{FS} 
%L.D. Faddeev, A.A.  Slavnov, {\em Gauge Fields: An Introduction 
%to Quantum Theory}, Addison-Wesley (Frontiers in Physics vol. 83), (second 
%edition), 1991. 
%\end{thebibliography}


\end{document}

