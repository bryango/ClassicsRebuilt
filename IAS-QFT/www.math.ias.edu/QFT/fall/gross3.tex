%Date: Mon, 20 Jan 1997 13:55:46 -0500 (EST)
%From: Pavel Etingof <etingof@abel.math.harvard.edu>

\input amstex
\documentstyle{amsppt}
\magnification 1200
\NoRunningHeads
\NoBlackBoxes
\document

\def\tW{\tilde W}
\def\Aut{\text{Aut}}
\def\tr{{\text{tr}}}
\def\ell{{\text{ell}}}
\def\Ad{\text{Ad}}
\def\u{\bold u}
\def\m{\frak m}
\def\O{\Cal O}
\def\tA{\tilde A}
\def\qdet{\text{qdet}}
\def\k{\kappa}
\def\RR{\Bbb R}
\def\be{\bold e}
\def\bR{\overline{R}}
\def\tR{\tilde{\Cal R}}
\def\hY{\hat Y}
\def\tDY{\widetilde{DY}(\g)}
\def\R{\Bbb R}
\def\h1{\hat{\bold 1}}
\def\hV{\hat V}
\def\deg{\text{deg}}
\def\hz{\hat \z}
\def\hV{\hat V}
\def\Uz{U_h(\g_\z)}
\def\Uzi{U_h(\g_{\z,\infty})}
\def\Uhz{U_h(\g_{\hz_i})}
\def\Uhzi{U_h(\g_{\hz_i,\infty})}
\def\tUz{U_h(\tg_\z)}
\def\tUzi{U_h(\tg_{\z,\infty})}
\def\tUhz{U_h(\tg_{\hz_i})}
\def\tUhzi{U_h(\tg_{\hz_i,\infty})}
\def\hUz{U_h(\hg_\z)}
\def\hUzi{U_h(\hg_{\z,\infty})}
\def\Uoz{U_h(\g^0_\z)}
\def\Uozi{U_h(\g^0_{\z,\infty})}
\def\Uohz{U_h(\g^0_{\hz_i})}
\def\Uohzi{U_h(\g^0_{\hz_i,\infty})}
\def\tUoz{U_h(\tg^0_\z)}
\def\tUozi{U_h(\tg^0_{\z,\infty})}
\def\tUohz{U_h(\tg^0_{\hz_i})}
\def\tUohzi{U_h(\tg^0_{\hz_i,\infty})}
\def\hUoz{U_h(\hg^0_\z)}
\def\hUozi{U_h(\hg^0_{\z,\infty})}
\def\hg{\hat\g}
\def\tg{\tilde\g}
\def\Ind{\text{Ind}}
\def\pF{F^{\prime}}
\def\hR{\hat R}
\def\tF{\tilde F}
\def\tg{\tilde \g}
\def\tG{\tilde G}
\def\hF{\hat F}
\def\bg{\overline{\g}}
\def\bG{\overline{G}}
\def\Spec{\text{Spec}}
\def\tlo{\hat\otimes}
\def\hgr{\hat Gr}
\def\tio{\tilde\otimes}
\def\ho{\hat\otimes}
\def\ad{\text{ad}}
\def\Hom{\text{Hom}}
\def\hh{\hat\h}
\def\a{\frak a}
\def\t{\hat t}
\def\Ua{U_q(\tilde\g)}
\def\U2{{\Ua}_2}
\def\g{\frak g}
\def\n{\frak n}
\def\hh{\frak h}
\def\sltwo{\frak s\frak l _2 }
\def\Z{\Bbb Z}
\def\C{\Bbb C}
\def\d{\partial}
\def\i{\text{i}}
\def\ghat{\hat\frak g}
\def\gtwisted{\hat{\frak g}_{\gamma}}
\def\gtilde{\tilde{\frak g}_{\gamma}}
\def\Tr{\text{\rm Tr}}
\def\l{\lambda}
\def\I{I_{\l,\nu,-g}(V)}
\def\z{\bold z}
\def\Id{\text{Id}}
\def\<{\langle}
\def\>{\rangle}
\def\o{\otimes}
\def\e{\varepsilon}
\def\RE{\text{Re}}
\def\Ug{U_q({\frak g})}
\def\Id{\text{Id}}
\def\End{\text{End}}
\def\gg{\tilde\g}
\def\b{\frak b}
\def\S{\Cal S}
\def\L{\Lambda}

\topmatter
\title Lecture 3: Renormalization groups (continued)
\endtitle
\author {\rm {\bf David Gross} }\endauthor
\endtopmatter

\centerline{Notes by P.Etingof and D.Kazhdan}
\vskip .1in

{\bf 3.1. Dynamical patterns of the renormalization group flow.} 

In the last lecture we saw that the renormalization group equation
for the $\phi^4$-theory in the one-loop approximation had the form
$g_*'=Ag_*^2$. This equation has a fixed point at $g_*=0$, which is attracting
for $g_*<0$ and repelling at $g_*>0$, so it looks like
$$
>>>>>>*>>>>>> 
$$
The fact that this fixed point is repelling for $g_*>0$ allowed us to 
show that $\phi^4$ theory is not asymptotically free in the region $g_*>0$. 

Now consider  what happens for an arbitrary renormalizable field theory
with a single renormalizable coupling $\l$. 
Let us write down the renormalization group equation for $\l$ based on
some mass-independent finite renormalization prescription:
$$
\mu\frac{d\l}{d\mu}=\beta(\l).
$$
This equation always has a fixed point at $\l=0$ (the free theory). 
The behavior of solutions of this equation near the fixed point can be 
determined by looking at the sign of 
the first nontrivial term of the Taylor expansion 
of $\beta(\l)$. There can be two patterns of behaviour. 

1. An attracting, or
ultraviolet (UV) stable fixed point ($\beta(\l)$ and $\l$ have 
opposite signs for small $\l$):
$$
*<<<<<<<
$$
In this case, 
$\l(\mu)$ is driven to the fixed point $\l=0$ 
in the UV limit $\mu\to \infty$, which implies 
asymptotic freedom. More precisely, if $\beta(\l)\sim A\l^{k+1}$, $\l\to 0$, 
then, as one finds by solving the renormalization group equation, 
$$
|\l(\mu)|\sim (|A|k)^{-1/k}|\ln(\mu/\mu_0)|^{-1/k}, \mu\to\infty.\tag 3.1
$$
However, in the limit $\mu\to 0$ 
$\l(\mu)$ is driven away from the fixed point. 

2. A repelling, or infrared (IR) stable 
fixed point ($\beta(\l)$ has the same sign as $\l$ for small $\l$).
$$
*>>>>>>>
$$
In this case $\l(\mu)$ is driven away from the fixed point $\l=0$
in the UV limit $\mu\to \infty$, so there is no asymptotic freedom. 
However, in the IR limit $\mu\to 0$ the coupling $\l(\mu)$ approaches the 
point $\l=0$, at the rate given by (3.1).  

Now we face a question: what happens if $\l(\mu)$ is 
driven away from the fixed point? What will be its behavior 
when $\l$ is not very small? 

The answer to this question cannot be given within the framework 
of perturbation theory. If the field theory which we are considering 
actually exists non-perturbatively, then there could be two options:

Option 1. $\l(\mu)$ is driven to a fixed point $\l_0\ne 0$:
$$
*>>>>>*
$$

Option 2. $\l(\mu)$ is driven to infinity. 

{\bf Remark.} In general, it could happen that the differential
equation $\l'=\beta(\l)$ has no nonzero solutions which are 
defined globally for all $\mu$. (Example: $\beta(\l)=A\l^2$). 
This means, the coupling $\l(\mu)$ becomes infinite
at some finite value of $\mu$ (e.g. $\l=\frac{\l_0}{1-A\l_0\ln(\mu/\mu_0)}$).
This kind of singularity is called a renormalon. We will discuss this topic 
later in more detail. For now let us just say that the appearance of
a renormalon does not necessarily 
mean that the theory does not exist but may just mean 
that we are doing perturbation expansion around a wrong vacuum state
(see the next lecture). 

Let us consider some examples. 

{\bf Example 1: $\phi^4$ theory}. In this case, as we saw above, 
the fixed point $g=0$ is UV stable for $g<0$ and IR stable for $g>0$, so 
we have asymptotic freedom in the unphysical region $g<0$. 
For $g>0$, there is no asymptotic freedom, and
in the 1-loop approximation $\l(\mu)$ becomes infinite
at finite $\mu$: $\mu=\mu_0e^{\frac{1}{Ag_*^0}}$. 
Whether this singularity survives non-perturbatively, 
provided that $\phi^4$-theory exists non-perturbatively -- 
cannot be determined in perturbation theory. 

{\bf Example 2: QED}. In QED, the beta-function has the form
$$
\beta(e)=\frac{e^3}{12\pi^2}+O(e^5),\tag 3.2
$$
where $e$ is the dimensionless electric charge. 
This can be inferred from computing the 1-loop correction to 
the 3-point function $\<\psi(x)A(y)\psi(z)\>$ 
where $A$ is the electromagnetic field and $\psi$ represents a charged 
particle, e.g. an electron
(see also the solution of problem 2 in Witten's problem set 6). 
Therefore, the free fixed point 
$e=0$ is IR stable in both regions $e<0$, $e>0$ 
(which are both physically meaningful). 
Thus QED is not asymptotically free. 

So what happens 
in the UV limit? Solving the renormalization group equation
in the 1-loop approximation, we get
$$
e^2=\frac{e_0^2}{1-\frac{e_0^2}{6\pi^2}\ln(\mu/\mu_0)}.\tag 3.3
$$
Thus, if we trust the 1-loop approximation, the electric charge 
grows as $\mu\to\infty$, and becomes infinite at 
$\mu=\mu_0\exp(6\pi^2/e_0^2)$ (the so-called Landau singularity). 

One can wonder whether this singularity exists nonperturbatively.
This is a rather academic question, since QED is not proved and 
generally not believed to exist nonperturbatively. However, 
let us estimate how big $\mu$ is. We have $e_0^2/4\pi\sim 1/137$, and
$\mu_0\sim 1 \text{MeV}$. So we get
$\mu\sim e^{3\pi\times 137/2}\mu_0\sim 10^{250}\text{MeV}$. 
Such energy is not contained in the whole observable part of the universe. 
So even if the singularity occured at $\mu$, it does not concern us, since we
are studying physics at much lower energy scales. 
This is the reason why, although QED is not asymptotically free, and its 
perturbation expansion is not valid at high energies, it gives a 
good approximation to reality in a certain range of energies, and 
can be used to make predictions which can be checked experimentally. 

In formula (3.3), $e_0$ has the meaning of the 
observed (effective) charge when the measurements are made at an 
experimentally accessible scale of energies $\mu_0$, 
while $e$ has the meaning of the actual charge (i.e. the charge computed 
on the basis of a very precise theory which describes physics at 
a very high scale of energies $\mu$). Passage from $e$ to $e_0$ by moving down
with the renormalization group flow from $\mu$ to $\mu_0$ has the  
meaning of averaging out degrees of freedom between $\mu_0$ and $\mu$. 
The inequality $e_0<e$ for $\mu_0<\mu$, which follows from (3.3),
has the meaning of {\it screening} of charges: the observed charge 
is less than the actual charge. 

{\bf Remark.} This screening phenomenon is known 
already in classical physics (screening of charges by medium): 
$e_0=e/\e$, where $\e>1$ is the dielectric permeability of the medium. 
To explain this, one should model the medium as a collection 
of very small dipoles. In the presence of 
a charged particle with charge $e$ (say $e>0$) 
they will align in such a way that the negative parts will face
the particle. If we draw a sphere $S$ centered at the particle, it will cut 
some of the dipoles in half, so that negative parts will be inside of the 
sphere and the positive parts outside. Now we can use Newton's theorem:
the field of a uniformly charged sphere with charge $q$ is zero
inside of the sphere, and is the same as the field of a point charge $q$ 
situated at the center,  outside of the sphere.  
This theorem implies that the total contribution 
 of the outside parts of the dipoles 
to the electric field at a point of $S$ is zero, while the total contribution 
of the negative parts is the same as that of 
a negative point charge $q$ at the center. So the effective
value of the charge is $e-q$. 
(We should note that $q$ depends on $e$, since the 
number of dipoles which will align depends on the strength of the electric 
field. In fact, $q$ is proportional to $e$.)

{\bf Example 3: Nonabelian gauge theory in 4 dimensions}. 
Before 1970-s, it was believed that all field theories 
are IR stable but not asymptotically free. However, in the 70-s it was found 
that pure non-abelian gauge theories are asymptotically free. 
Moreover, they remain asymptotically free even if 
fermions are added, as long as the number of fermions
does not exceed a certain limit. 

Namely, one can show that the $\beta$-function
for a non-abelian gauge theory with gauge group $G$
and fermions $\psi_i$.  
(a compact simple Lie group) has the following form
in the 1-loop approximation:
$$
\beta(g)=-\frac{1}{48\pi^2}(11C_{\g}-4\sum_i C_{V_i})g^3,\tag 3.4
$$
where $\g$ is the Lie algebra of $G$, $V_i$ are the representations of $G$ 
where the fermions $\psi_i$ 
live, and $C_{V}$ is the Casimir $C$ of $\g$ acting 
in a representation $V$. Here the Casimir is normalized in such a way that
$C_{\g}=h^\vee$, where $h^\vee$ is the dual Coxeter number of $\g$
($h^\vee=N$ for $SU(N)$).  
The proof of formula (3.4) can be found it textbooks on QFT, for example
in P.Ramond's textbook. 

Formula (3.4) tells us that the pure gauge theory is asymptotically free,
so that $g(\mu)$ is a decreasing function for small $g$ 
(so-called ``antiscreening''), 
and gives us the bound on the number of fermions which can be added
without destruction of asymptotic freedom. 

For example, consider QCD (quantum chromodynamics), which is believed to
describe strong interactions. In QCD, $G=SU(3)$, and $V_i=\C^3$, 
so $C_{\g}=3$ and $C_{V_i}=1/2$, so we get
$$
\beta(g)=-\frac{1}{48\pi^2}(33-2N_f)g^3,\tag 3.4
$$
where $N_f$ is the number of fermions. So, in order for this theory 
to be asymptotically free, we have to have $N_f\le 16$. 

Asymptotic freedom of QCD can be checked experimentally. 
One of the possible experiments is described in a simplified way 
in the appendix to Witten's Lecture 4.  

{\bf 3.2. Are there any asymptotically free theories 
without nonabelian gauge fields?}

Example 3 in the previous section raises the question whether 
there can be an asymptotically free theory 
without nonabelian gauge fields. 
The answer is the following. 

In two dimensions, there are many asymptotically free theories without  
nonabelian gauge fields. For example, the 2-dimensional theory with 
$N$ fermions and interaction $\sum_j(\bar\psi_j\psi_j)^2$, which we will 
consider in the next lecture, is asymptotically free. 

In four dimensions, this is not the case. Namely, there is
the following result.

\proclaim{Coleman-Gross Theorem} Let $\Cal L(\phi_i,\psi_j)$ be a 
Poincare invariant, renormalizable Lagrangian 
of scalar fields $\phi_k$ and spinor fields $\psi_j$:
$$
\Cal L=\int d^4x\biggl(
\frac{1}{2}\sum_k (\nabla\phi_k)^2+
\sum_j(\psi_j,(iD+M_j)\psi_j) 
+V(\phi)+Y(\phi,\psi)\biggr),\tag 3.5
$$
where $V(\phi)$ is a quartic polynomial, and 
$Y(\phi,\psi)$ are Yukawa interaction terms of the form 
$\phi\psi\psi$. Suppose that the function 
$V(\phi)$ is bounded from below. Then the quantum field theory 
defined by $\Cal L$ is not asymptotically free. 
\endproclaim

It is clear that this theorem generalizes to the case where $\Cal L$ 
is allowed to contain Abelian gauge fields: in example 2 we saw that 
this would make the theory even ``less'' asymptotically free. 

The condition that $V(\phi)$ is bounded from below 
cannot be dropped: we have seen 
that the $\phi^4$ theory with $g<0$ is asymptotically 
free. However, one can show that 
such a theory cannot exist nonperturbatively. 
Namely, one can show that if such a theory had existed, 
its spectrum of energy would not be bounded from below, 
which contradicts the existence of the theory
(the energy positivity axiom would be violated).

Let us show this for the case of $\phi^4$ theory. 
We consider the $\phi^4$ theory with negative coupling constant $g$. 
In this case, the classical potentail $V_0(\phi)=\frac{g\phi^4}{4!}
+\frac{m^2}{2}\phi^2$ is not bounded from below. Apriori, it does not 
imply that the theory makes no sense, since 
this problem could be cured in the quantum theory. 
In order to see what happens in the quantum theory, we should 
compute the effective potential, $V_{\text{eff}}(\phi)$,
which represents the effective classical theory for the quantum theory 
defined by $V_0(\phi)$. If this potential is not bounded from below, 
we can conclude that the theory does not exist quantum-mechanically.

As we know, the 
effective classical action is computed as 
a sum of amplitudes of 1-particle 
irreducible Feynman diagrams (see the end of 
Kazhdan's lecture on Feynman graphs):
$$
S_{\text{eff}}(\phi)=
\sum_{k\ge 1}\int_{\sum p_i=0}\Gamma_{2k}(p_1,...,p_{2k})
\tilde\phi(p_1)...\tilde\phi(p_{2k})dp,\tag 3.6
$$
where $\Gamma_{2k}$ are 1-particle irreducible correlations functions,
and $\tilde\phi$ denotes the Fourier transform of $\phi$. 
By the definition, the effective potential $V_{\text{eff}}(\phi)$ is 
the function of one variable defined by the formula
$$
V_{\text{eff}}(\phi)=S_{\text{eff}}(\phi),\tag 3.7 
$$
where $\phi$ is a constant function.
That is,
$$
V_{\text{eff}}(\phi)=\sum_{k\ge 2} \Gamma_{2k}(0)\phi^{2k}.\tag 3.8
$$
In general, it is not easy to see whether this function is bounded 
from below or not. However, the fact that the $\phi^4$ theory
with $g<0$ is asymptotically free allows to figure this out. 
The demonstration that $V_{eff}(\phi)$ is not bounded
from below is contained in Weinberg's textbook in Chapter 18, 
Section 18.2. 
This unboundedness implies that
at large scales of $\phi$ there exist states of arbitrarily low energy,
i.e. that the corresponding quantum theory in fact does not exist. 

Thus, we have seen that a sensible 
4-dimensional quantum field theory which is asymptotically free 
must contain nonabelian gauge fields. But theories which are 
not asymptotically free are usually not believed to exist
nonperturbatively. Does this mean that theories without nonabelian 
gauge fields, like QED, do not make any physical sense? 
(This would really be very 
strange, since QED was used to make some predictions, 
which were in a very good agreement with experiment). 

The answer of course is no. Even though QED may not exist nonperturbatively
as a separate theory, its Lagrangian may be a part of a bigger Lagrangian, 
which contains other fields (in particular, nonabelian gauge fields),
and define a theory which is asymptotically free and perhaps exists 
nonperturbatively. However, at low energies, when considering electromagnetic 
interactions, we can ignore all the additional terms in the Lagrangian, and
use the Lagrangian of QED. 

In fact, the real situation is the following. There is the so called
``Standard model'', which is a field theory with many different 
fields, and includes QED as a subtheory. This theory is believed to 
exist nonperturbatively, and to describe all interactions except 
gravity in the real world. Now, if we want to compute the scattering 
data for two electrons in this theory, the answer will be of the order
$e^2$, where $e$ is the charge of the electron, and 
corrections due to terms in the Lagrangian not contained in the 
Lagrangian of QED are of order $e^4$ and higher
(see appendix to Witten's lecture 4). Therefore, at
low energies, where $e$ is small, QED in the 1-loop approximation is still 
valid.  

{\bf 3.3. Renormalization group equations with many couplings.}

So far we have considered the renormalization group flow only for 
theories with one critical coupling. Similarly one can define 
the renormalization group flow in the case when there are many such couplings,
$g_1,...,g_n$. We will assume that the renormalization group flow
is defined by a mass independent family of renormalization prescriptions
$\pi_\mu$ parametrized by $\mu\in M$ (for example, dimensional regularization 
-- minimal subtraction). In this case, subcritical couplings 
do not enter in the renormalization group equation for $g_i$, so it has the 
form
$$
\mu\frac{d g_i}{\d \mu}=\beta_i(g_1,...,g_n).\tag 3.9
$$
In general, we know nothing about the flow defined by (3.9), except that 
the point $g_i=0$ (the free theory) is a fixed point of this flow. 
However, there is the following conjecture, which says that 
this flow behaves as a gradient flow. 

\proclaim{Conjecture 3.1} There exists a function $F(g_1,...,g_n)$ such that
its total derivative with respect to $\mu$ is positive at any nonsingular 
point of the flow (3.9). 
\endproclaim

{\bf Remark.} Here we, of course, assume that the theory we are considering 
exists nonperturbatively. 

Conjecture 3.1 prohibits various complicated dynamical patterns which could 
occur in a multidimensional dynamical system (limiting cycles, 
ergodicity, attractors, etc.), and implies that any trajectory is 
either driven to a fixed point or to infinity. 

The intuition behind conjecture 3.1 is the Wilsonian point of view 
that the (backward) renormalization group flow comes from erasing 
degrees of freedom. The meaning of the function $F$ is ``the measure 
of the number of degrees of freedom''. Unfortunately, it is unknown 
how make mathematical sense of this idea in general, so conjecture 3.1 remains 
open. It is proved, however, in many special cases, 
for example, in 2-dimensional field theories. 

{\bf Remark.} In 2-dimensional theories with bosonic fields 
the space of couplings is infinite-dimensional, since the classical 
dimension of elementary bosonic fields vanishes
(couplings are arbitrary functions). Therefore, the renormalization group
equations are equations in the space of functions. 

{\bf 3.4. The renormalization group equation for composite operators.} 

Suppose we are given a renormalizable field theory $T$ 
given by some Lagrangian 
$\Cal L$.  
Recall from Witten's lecture 3 that every local operator $\O(x)$ in the 
classical field theory defined by $\Cal L$ can be quantized (non-canonically), 
and that the space $A_l^0(T)$ of classical operators of 
scaling dimension $\le l$ admits a canonical quantization,
which we will denote by $A_l(T)$. 

Recall that for
 any operator $\O\in A_l(T)$, the correlation functions of $\Cal O$
are defined to be the $\e$-coefficients of the correlation functions
of the Lagrangian
$$
\Cal L_\e=\Cal L+\e\O(y),\tag 3.10
$$
where $\e^2=0$. 

Let $Z=\cup_{l\ge 0}Z_l$ be a subspace in the space of differential 
polynomials of fields (with complex coefficients), 
invariant under the orthogonal group, 
and the groups $G_1$, $G_2$ of scaling the values of
fields and scalings of spacetime, and such 
that the natural map $Z_l\to A_l^0(T)$ is an isomorphism 
for all theories $T$ of a given type. 
We will call $V$ the set of representative differential 
polynomials, and will always represent a classical local functional 
by its unique preimage in $Z_l$. Abusing language, we will call elements of 
$Z$ ``operators''. 

{\bf Remark.} The need to introduce $Z$ is caused by the fact that in any
field theory, there are nontrivial relations between differential monomials
(field equations), and therefore each local functional has many
representations as a differential polynomial of elementary fields. 

Now we will define the renormalization group flow on the space of operators. 
We will use the notations of Lecture 2 and Witten's lecture 3. 

Let $Y$ be the space of renormalizable theories of the corresponding type, 
and $Y_l$ be the space of Lagrangians of the form 
(3.10), where $\O$ is of dimension $\le l$. The space $Y_l$ is 
a vector bundle over $Y$: $Y_l\to Y$, whose fiber at a point $T\in Y$
is the space $A_l(T)$ of operators of dimension $\le l$
for the theory $T$. 

Let $\pi_\mu: Y\to P$ be a renormalization prescription. 
Let  
$P_l=P\times Z_l$ be the total space of the trivial bundle
over $P$ with fiber $Z_l$, and 
and $\pi_\mu^l: Y_l\to P_l$ 
be an isomophism of vector bundles which descends to $\pi_\mu:Y\to P$. 
$\pi_\mu^l$ is called an extension of the renormalization prescription
$\pi_\mu$ to operators of dimension $\le l$ (this extension is 
not, in general, 
defined by $\pi_\mu$, and has to be defined additionally;
below we will give one of such definitions).
Let $R_{\mu_1\mu_2}^l=\pi_{\mu_2}\circ \pi_{\mu_1}^{-1}: P_l\to P_l$.
The transformation $R_{\mu_1\mu_2}^l$ is called the remormalization 
group transformation for composite operators
for the renormalization prescription $\pi_\mu^l$. It defines a flow
on $P_l\times M$ by $t(z,\mu)=
(R_{\mu,t\mu}z,\mu)$. 

As before, it is convenient to represent the renormalization group flow 
as a vector field. Denote this vector field on $P_l$ by $W_l$. 
It is clear that $W_l$ projects to $W$ under the projection $P_l\to P$, 
and preserves the structure of a vector bundle on $P_l$.
Therefore, 
$$
W_l(z,\mu)=W(z,\mu)+\tilde\gamma_l(z,\mu),\tag 3.11
$$
where $\tilde\gamma_l: P\times M\to \End Z_l$ is an operator-valued function
(regarded as a vertical vector field on $P_l$). 

If $\O_i$ is a basis of $Z_l$, homogeneous with respect to both $G_1,G_2$,
and $\O=\sum f_i\O_i$,
then the renormalization group equations
can be written explicitly as
$$
\mu\frac{d f_i(z,\mu)}{d \mu}=\sum_j\tilde\gamma_l^{ij}(z,\mu)f_j(z,\mu),
\tag 3.12
$$
where $\tilde\gamma_l^{ij}$ is the matrix of the operator $\tilde\gamma_l$ in the basis 
$\O_i$, and $z=z(\mu)$ is a solution of the renormalization 
group equation for parameters of the theory.   

Now we should explain how to extend the map $\pi_\mu$ to $\pi_\mu^l$. 
As usual, there are many ways to do it. However, if $\pi_\mu$ is 
the dimensional regularization prescription, there is the following 
natural way of extending $\pi_\mu$ to $\pi_\mu^l$.

We will consider the $\phi^4$ theory. 
Let $T\in Y$, $\pi_\mu(T)=\Cal L$ be the Lagrangian which
describes $T$ at scale $\mu$ (in the sense of dimensional 
regularization) and  
$\O\in A_l(T)$. 

\proclaim{Proposition 3.2} There exists a unique 
differential polynomial  
$F_\O\in Z_l$ such that the correlation functions of the operator 
$\O$ in the theory $T$ are equal to the $\e$-coefficients of the correlation 
functions of the Lagrangian $\Cal L_\e=\Cal L+\e F_\O$ computed using
dimensional regularization and minimal subtraction at scale $\mu$ 
(see Section 2.2 of Lecture 2).
\endproclaim

\demo{Proof} The statement follows from the fact that
classical field equations survive quantization, which was explained 
(for $\phi^4$ theory),
and checked in the simplest case in Witten's lecture 3, Section 3.7.
\enddemo
 
Now we define the map $\pi_\mu^l$ by the formula $\pi_\mu^l(T,\O)=
(\pi_\mu(T),F_\O)$. As was explained above, this defines a renormalization
group flow on the space of operators. Moreover, since our renormalization 
prescription is mass-independent, the function $\tilde\gamma_l(z,\mu)$ does not
explicitly depend on $\mu$.

{\bf Remark.} 
The mass independence of the dimensional regularization prescription
implies that 
the map $\tilde\gamma_l(z)$ respects not only
the filtration of the space $Z_l$ of operators by classical 
scaling dimension, but also 
the grading by scaling dimension. That is, operators mix only with operators 
of the same dimension. This may be false for a mass-dependent prescription.  

Let us now consider the example of the $\phi^4$ theory. 
In this case, 
equation (3.12) has the form
$$
\mu\frac{d f_i}{d \mu}=\sum_j\tilde\gamma_l^{ij}(g_*(\mu),a(\mu))f_j
\tag 3.13
$$
(there is no dependence on $m_*$ due to mass independence of the 
renormalization
prescription).
Let $\Delta_i$ be the degree of $\O_i$ with respect to scaling of $\phi$. 
It is clear from the scale-of-$\phi$-invariance of (3.13) that
$\gamma_l^{ij}(g_*,a)=\hat\gamma_{l}^{ij}(g_*)a^{(\Delta_i-\Delta_j)/2}$.
Therefore, if we conjugate equation (3.13) by 
$a^{-D_\phi/2}$, where $D_\phi$ is the scale-of-$\phi$-degree 
operator, and introduce functions $h_i=f_ia^{-\Delta_i/2}$, we will get
the following equations:
$$
\mu\frac{d h_i}{d \mu}=\sum_j\gamma_l^{ij}(g_*(\mu))h_j,
\tag 3.14
$$
where $$
\gamma_l=\hat\gamma_l-\frac{D_\phi\delta}{2}.\tag 3.15
$$ 
The operator $\gamma_l$ is already a function only of one variable
$g_*$. 

{\bf 3.5. Anomalous dimension.}

{\bf Definition.} The operator $\gamma_l$ is called the 
operator of anomalous dimensions at $g_*$,
for operators of degree $\le l$. The matrix $\gamma_l^{ij}(g_*)$
is called the matrix of anomalous dimensions at $g_*$. 

For the simplest operators one can define scalar-valued
anomalous dimension. Namely, Let $F\in Z_l[a^{\pm 1/2}]$  
be a generalized eigenvector of $\gamma_l(g_*)$ with eigenvalue 
$\gamma_F(g_*)$ for all $g_*$. In this case we will say 
that $\gamma_F(g_*)$ is the anomalous dimension of 
the operator $F$ at $g_*$. In general, any quantity $X$ 
depending on $\mu$ and obeying the equation
$\mu\frac{d X}{d\mu}=\gamma_X(g_*(\mu))X$, where
$\gamma_X$ is a scalar function, will be said to have 
anomalous dimension $\gamma_X$. 

Now let us explain the name ``anomalous dimension''.
Let $F$ be a (real) operator (with $a$-dependent coefficients) of 
engineering dimension $d_F$,
scale-of-$\phi$ degree $0$, 
 and scalar anomalous dimension $\gamma_F(g_*)$.
For simplicity we assume that it is a true (not only generalized) eigenvector
of $\gamma_l$.   
Let 
$$
\<F(x)F(0)\>=G_2^F(x^2,\mu,m_*,a,g_*)\tag 3.16
$$ 
be the 2-point function 
of $F$ in position space. Using the renormalization group flow, we conclude
that
$$
\mu\frac{d G_2^F(x^2,\mu,m_*(\mu),a(\mu),g_*(\mu))}{d\mu}=
2\gamma_F(g_*(\mu))G_2^F(x^2,\mu,m_*(\mu),a(\mu),g_*(\mu)).\tag 3.17
$$
But the engineering dimension of $G_2^F$ is $2d_F$, so (3.17) can be rewritten
in the form
$$
\gather
s\frac{d}{ds}G_2^F((x/s)^2,\mu_0,m_*(s\mu_0),a(s\mu_0),g_*(s\mu_0))=\\
2(d_F+\gamma_F(g_*(s\mu_0)))G_2^F((x/s)^2,\mu_0,m_*(s\mu_0),a(s\mu_0),
g_*(s\mu_0)).\tag 3.18
\endgather
$$
Integrating this equation, we get
$$
\gather
G_2^F((x/s)^2,\mu_0,m_*(s\mu_0),a(s\mu_0),g_*(s\mu_0))=\\
s^{2(d_F+\int_{\mu_0}^{s\mu_0}\gamma_F(g_*(r\mu_0))d\ln r)}
G_2^F(x^2,\mu_0,m_*(\mu_0),a(\mu_0),g_*(\mu_0)).\tag 3.19
\endgather
$$
If $m_*,a,g_*$ did not depend on $\mu$ for large $\mu$, 
we would get 
$$
G_2^F(x^2)\sim |x|^{-2(d_F+\gamma_F)},\tag 3.20 
$$
which shows that the operator 
$F$ has ``quantum scaling dimension'' $d_F+\gamma_F$. 
Thus, anomalous dimension is just the difference between the
quantum and the classical scaling dimension. 

In general, formula (3.20) does not hold, because there is a nontrivial 
renormalization group flow on the space of parameters. If 
this flow does not have an UV stable fixed point, there is usually no hope
to derive a scaling law of the form (3.20). However, if
the flow has an UV stable fixed point $g_*^f$, and 
$\gamma_F(g_*^f)=\gamma_F$, then up to logarithmic factors formula (3.20) 
is true. 

Furthermore, if there is an UV stable fixed point $g_*^f$, 
one can replace matrix anomalous dimensions by scalar ones. 
Namely, one should bring the matrix $\gamma_l(g_*^f)$ to Jordan normal form.
The generalized eigenvectors of $\gamma_l(g_*^f)$ with eigenvalue
$\nu$ are then called ``operators of pure anomalous dimension $\nu$''.

Unfortunately, the only type of an UV stable fixed point that we 
can study successfully in perturbation theory is the fixed point
$g_*^f=0$ for an asymptotically free theory. In this case, the anomalous 
dimensions $\gamma_F(g_*^f)$ in the above sense vanish, because
$\gamma_l(0)=0$.  
However, for $g_*$ close to zero anomalous dimensions are nonzero, 
and the knowledge of them helps get a very sharp UV asymptotics 
of operators and correlation functions. The most remarkable thing is 
that in order to compute the exact asymptotics of 
operators, up to multiplication by a constant, it is enough to know
anomalous dimensions only in the first order approximation, which 
are obtained by an easy explicit calculation. 
Let us show how this works. 

We will consider $\phi^4$ theory with $g<0$ (never mind that 
it does not exist nonperturbatively; the results will apply to any 
asymptotically free theory).

Let $F$ be a
scale-of-$\phi$-independent (i.e. $G_1$-nvariant)
 eigenvector of $\gamma_l$ with eigenvalue $\gamma_F(g_*)$, and
$$
\gamma_F(g_*)=\gamma_F^1g_*+O(g_*^2),\tag 3.21
$$
We will assume that $F$ is scale of $\phi$-independent
We have seen above that because of asymptotic freedom for large $\mu$ 
one has $g_*(e^t\mu_0)=-\frac{16\pi^2}{3t}
+o(t^{-2+\alpha})$, for any $\alpha>0$. 
Therefore, using the fact that $a$ is constant modulo $g_*^2$, 
we obtain from formula (3.19):
$$
G_2^F(x^2)\sim \text{const} |x|^{-2d_F}|\ln|x||^{32\pi^2\gamma_F^1/3}.
\tag 3.22
$$
This remarkable formula is actually the exact asymptotics (!!!), except that
in order to compute the constant on the right hand side we of 
course have to know the function $\gamma_F$ to all orders. 
However, if we are not interested in the constant, we only need to know 
the first order approximation. 

Formula (3.22) can be generalized to the case of an arbitrary 
correlation function. Namely, if $F_1,...,F_n$ are any 
scale-of-$\phi$-independent operators 
which are eigenvectors of the operator $\gamma_l^1$, 
with eigenvalues $\gamma_{F_1}^1,...,\gamma_{F_1}^n$, then 
$$
\<F_1(e^{-t}x_1)...F_n(e^{-t}x_n)\>\sim 
C(x_1,...,x_n) e^{-t(\sum d_{F_i})}t^{16\pi^2(\sum \gamma_{F_i}^1)/3}.\tag
3.23
$$

In order to compute the anomalous dimension of an operator, 
it is enough to study the dynamics of its 2-point function. 
Let us show how to do it in examples. 

{\bf Examples.} 1. $F=\phi$. The anomalous dimension of $\phi$ is zero in
the 1-loop approximation, since there is no 1-loop 
contributions to the 2-point function at $x\ne 0$. 
However, in the 2-loop approximation, computing the only 
nontrivial 2-loop diagram, one gets a nontrivial correction. 
Of course, this correction does not enter in formula (3.23), so that 
correlation functions scale asymptotically by the naive classical rule. 

In fact, it is easy to see that 
$$
\gamma_\phi=-\delta/2\tag 3.24
$$ 
This is true because
 by the definition $\hat\gamma_\phi=0$, since Schwinger functions
do not change along the renormalization group flow. 

2. $F=\phi^2$. In this case, analyzing the function 
$\<\phi^2(x)\phi^2(0)\>$, which was computed in Witten's lecture 3, 
it is easy to find that 
in the 1-loop approximation
$$
\gamma_{\phi^2}(g_*)=-\frac{g_*}{16\pi^2}+O(g_*^2).\tag 3.25
$$

In fact, it is easy to see that $\gamma_{\phi^2}=-2\gamma-\delta$. 
This follows from the fact that the anomalous dimension of
the Lagrangian by definition vanishes. Therefore, 
the anomalous dimension of $m^2\phi^2/2$ vanishes, because it occurs 
in the Lagrangian, and cannot mix with other terms. Thus, 
$\gamma_{\phi^2}=-2\gamma_m$. Since $m=m_*a^{1/2}\mu$, 
$\gamma_m=\gamma_{m_*}+\gamma_a/2=\gamma+\delta/2$, as desired. 

Since we know that $\gamma=\frac{g_*}{32\pi^2}+O(g_*^2)$, we obtain another 
proof of (3.25).

Formula (3.25) allows us to write down the asymptotics of the 2-point function
of $\phi^2$ in the asymptotically free case. Indeed, using (3.22), we get
$$
G_2^{\phi^2}(x^2)\sim \text{const} |x|^{-4}|\ln |x||^{-2/3}.\tag 3.26
$$

Finally, let us discuss to what extent the notion of anomalous dimension 
is canonical. The problem is, we have made a lot of choices (renormalization 
prescription, the space $Z$), and of course the anomalous dimension operator
$\gamma_l$ will depend on them. So what is choice independent?
The answer: 1) if $g_*^f$ is a nonzero UV stable fixed point, then 
the eigenvalues of $\gamma_*^f(g_*)$ are invariantly defined, and 
they are anomalous dimensions of fields in the quantum theory 
corresponding to $g_*^f$; 2) the eigenvalues of the first order coeffients 
of $\gamma_l$ at $g_*=0$ are also invariantly defined; in the asymptotically 
free case they give the logarithmic part of the short-distance asymptotics of
the correlation functions. 

{\bf 3.6. The canonical part of the $\beta$-function.}

Suppose we have a quantum field theory with a single renormalizable 
coupling $\l$. In this case, we saw that the renormalization group 
flow is defined by the equation
$$
\mu\frac{d\l}{d\mu}=\beta(\l), \tag 3.27
$$
where $\beta(\l)=A\l^{k+1}+...$ is the beta-function. 
Clearly, we can redefine the coupling $\l$ by using 
a change of variable $\l_1=f(\l)$. 
This redefinition should be trivial for small 
values of the coupling in order to make physical sense, 
so we should impose the condition 
$f'(0)=1$; that is, $f(\l)=\l+a_2\l^2+...$
Of course, under this transformation the beta-function will change. 
So to what extent is the beta-function defined canonically?

This question reduces to the following mathematical problem
from the theory of ``normal forms''.

{\bf Problem.} Let $S_A^k$ be the set of all formal vector fields 
of the form $A\l^{k+1}\frac{d}{d\l}+...$ with real coefficients $(k>0$), and 
$G$ be the group of formal diffeomorphisms of the form 
$\l\to f(\l)$, $f'(0)=1$. Compute the quotient $S_A^k/G$, where
$G$ acts by conjugation.
In other words, find the normal form of a vector field 
$\beta(\l)\frac{d}{d\l}$, where $\beta$ is as above. 

This problem is easily solved by recursion, and the answer is the following.

{\bf Answer.} $S_A^k/G=\{(A\l^{k+1}+B\l^{2k+1})\frac{d}{d\l},B\in \Bbb R\}$.
In particular, any vector field of the form $\beta(\l)\frac{d}{d\l}$ 
as above can be reduced by conjugation to the normal form
$(A\l^{k+1}+B\l^{2k+1})\frac{d}{d\l}$, while different normal forms 
are not equivalent. 

Thus, we see that 
it is not only the coefficient $A$ that is defined canonically 
in the beta-function, but also the coefficient $B$. 
In particular, in $\phi^4$-theory, the beta-function  
has the form $A\l^2+B\l^3+...$, and 
in QED the form $A\l^3+B\l^5+...$, where $A,B$ are defined canonically. 

Let us explain the meaning of the coefficient $B$ on the example of $\phi^4$ 
theory. Consider this equation in the region $\l<0$,
where we have asymptotic freedom. 
Solving the equation $\l'=A\l^2+B\l^3+...$, we get
$$
\l(t)=\frac{1}{At}+\frac{B}{A^2t^2}\ln t+O(1/t^2),\tag 3.28
$$
where $t=\ln(\mu/\mu_0)$. Thus, the coefficient $B$ encodes the second term
of the asymptotic expansion of $\l(t)$, which characterizes 
the singularity of $\l$, as a function of $z=1/t$, at $z=0$. 

\end

We will split $A^l_0$ in a direct sum of invariant subspace of $\gamma_l$ 
with eigenvalues $\gamma^i(g_*)$: $Z_l=\oplus Z_l[\gamma^i]$. 
The operators from $A^l_0(\gamma_i)$ are said to have anomalous domension 
$\gamma_i$. 

\end

