%From: John W Morgan <jmorgan@math.ias.edu>
%Date: Mon, 19 May 1997 16:52:17 -0400
%Subject: Lecture II

\documentstyle[11pt]{article}


%These are the macros which are in common with all of the
% sections in the paper mmr
% Each section, for now, should begin with \documentstyle[11pt,cd]{article}
% and then have \input{mmrmacros} followed by \begin{document}
% The only exception is that the \Label macro is slightly different
% in each file and should be put in separately.
%New CD macros
\newcommand{\cdrl}{\cd\rightleftarrows}
\newcommand{\cdlr}{\cd\leftrightarrows}
\newcommand{\cdr}{\cd\rightarrow}
\newcommand{\cdl}{\cd\leftarrow}
\newcommand{\cdu}{\cd\uparrow}
\newcommand{\cdd}{\cd\downarrow}
\newcommand{\cdud}{\cd\updownarrows}
\newcommand{\cddu}{\cd\downuparrows}
% (S) Proofs.
% (S-1) Head is automatically supplied by \proof.

\def\proof{\vspace{2ex}\noindent{\bf Proof.} }
\def\tproof#1{\vspace{2ex}\noindent{\bf Proof of Theorem #1.} }
\def\pproof#1{\vspace{2ex}\noindent{\bf Proof of Proposition #1.} }
\def\lproof#1{\vspace{2ex}\noindent{\bf Proof of Lemma #1.} }
\def\cproof#1{\vspace{2ex}\noindent{\bf Proof of Corollary #1.} }
\def\clproof#1{\vspace{2ex}\noindent{\bf Proof of Claim #1.} }
% End of Proof Symbol at the end of an equation must precede $$.

\def\endproof{\relax\ifmmode\expandafter\endproofmath\else
  \unskip\nobreak\hfil\penalty50\hskip.75em\hbox{}\nobreak\hfil\bull
  {\parfillskip=0pt \finalhyphendemerits=0 \bigbreak}\fi}
\def\endproofmath$${\eqno\bull$$\bigbreak}
\def\bull{\vbox{\hrule\hbox{\vrule\kern3pt\vbox{\kern6pt}\kern3pt\vrule}\hrule}}
\addtolength{\textwidth}{1in}                  % Margin-setting commands
\addtolength{\oddsidemargin}{-.5in}
\addtolength{\evensidemargin}{.5in}
\addtolength{\textheight}{.5in}
\addtolength{\topmargin}{-.3in}
\addtolength{\marginparwidth}{-.32in}
\renewcommand{\baselinestretch}{1.6}
\def\hu#1#2#3{\hbox{$H^{#1}(#2;{\bf #3})$}}          % #1-Cohomology of #2
\def\hl#1#2#3{\hbox{$H_{#1}(#2;{\bf #3})$}}          % #1-Homology of #2
\def\md#1{\ifmmode{\cal M}_\delta(#1)\else  % moduli space, delta decay of #1
{${\cal M}_\delta(#1)$}\fi}
\def\mb#1{\ifmmode{\cal M}_\delta^0(#1)\else  %moduli space, based, delta
					      %decay of #1
{${\cal M}_\delta^0(#1)$}\fi}
\def\mdc#1#2{\ifmmode{\cal M}_{\delta,#1}(#2)\else    %moduli space, delta
						      %decay, chern class #1
						      %of #2
{${\cal M}_{\delta,#1}(#2)$}\fi}
\def\mbc#1#2{\ifmmode{\cal M}_{\delta,#1}^0(#2)\else   %as before, based
{${\cal M}_{\delta,#1}^0(#2)$}\fi}
\def\mm{\ifmmode{\cal M}\else {${\cal M}$}\fi}
\def\ad{{\rm ad}}
\def\msigma{\ifmmode{\cal M}^\sigma\else {${\cal M}^\sigma$}\fi}
\def\cancel#1#2{\ooalign{$\hfil#1\mkern1mu/\hfil$\crcr$#1#2$}}
\def\dirac{D\hskip-.67em\slash}
\newtheorem{thm}{Theorem}
\newtheorem{theorem}{Theorem}[subsection]
\newtheorem{proposition}[theorem]{Proposition}
\newtheorem{lemma}[theorem]{Lemma}
\newtheorem{claim}[theorem]{Claim}
\newtheorem{example}[theorem]{Example}
\newtheorem{corollary}[theorem]{Corollary}
\newtheorem{D}[theorem]{Definition}
\newenvironment{defn}{\begin{D} \rm }{\end{D}}
\newtheorem{addendum}[theorem]{Addendum}
\newtheorem{R}[theorem]{Remark}
\newenvironment{remark}{\begin{R}\rm }{\end{R}}
\newcommand{\note}[1]{\marginpar{\scriptsize #1 }} 
\newenvironment{comments}{\smallskip\noindent{\bf Comments:}\begin{enumerate}}{\end{enumerate}\smallskip}

\renewcommand{\thesection}{\Roman{section}}
\def\eqlabel#1{\addtocounter{theorem}{1}
\write1{\string\newlabel{#1}{{\thetheorem}{\thepage}}}
\leqno(\rm\thetheorem)}
\def\cS{{\cal S}}
\def\ov{\overline}












\title{Kaluza-Klein compactifications, supersymmetry, and Calabi-Yau
spaces: II}  
\author{Andrew Strominger\thanks{Notes by John Morgan}}
\date{}
\begin{document}
\maketitle

\addtocounter{section}{4}

\section{Review of material from the first lecture}

Let  us recall the fields and $N=1$
local supersymmetries that we 
developed last time for our ten-dimensional supergravity 
coupled to super Yang-Mills theory.
We have bosonic fields: a metric $g_{MN}$  on space-time $X^{10}$
which is a ten-dimensional manifold, a two-form potential
$B$, the dilaton $\phi$ which is a scalar field, and 
a gauge field in the adjoint bundle of a principal $E_8\times E_8$ or
$SO(32)$ bundle $P$ over space-time.
The fermion fields are the gravitino $\psi_M$ which is a Majorana-Weyl
spinor of plus chirality with values in the tangent bundle, the
dilation $\lambda$, which is a Majorana-Weyl spinor of minus
chirality, and the gaugino $\chi$ which is a Majorana-Weyl spinor of
plus chirality with values in the adjoint bundle of $P$. 
The fields $(g_{MN},B,\phi,\psi_M,\lambda)$ form an $N=1$ supermultiplet,
called the gravity supermultiplet, and $(A,\chi)$ form the $N=1$
Yang-Mills supermultiplet, which is a matter supermultiplet.
Because of anomalies we are not free to have arbitrary numbers of
these supermultiplets: there must be exactly one ($E_8\times E_8$ or
$SO(32)$ matter supermultiplet
to go with the supergravity supermultiplet.



Recall from last time that the supersymmetric (odd)
generators $\epsilon$ are Majorana-Weyl spinors on
ten-dimensional space-time, i.e., sections of $S^+_{\bf R}(X^{10})$.
Their action on the fields  is given by; 

\begin{eqnarray*}
\delta_\epsilon\chi & = & F_{MN}\gamma^{MN}\epsilon+({\rm fermions})^2 \\
\delta_\epsilon\psi_M  &  =  &  \nabla_M\epsilon -\frac{1}{4}
H_{MAB}\gamma^{AB}\epsilon +({\rm fermions})^2 \\
\delta_\epsilon\lambda  &  =  &  (\gamma^M\nabla_M\varphi)\epsilon
+\frac{1}{24}H_{MNP}\gamma^{MNP}\epsilon +({\rm fermions})^2 \\
\delta_\epsilon g_{MN}  &  =  &  \overline\epsilon
\gamma_N\psi_M+\overline\epsilon\gamma_M \psi_N+({\rm fermions})^2 \\
\delta_\epsilon A_M  &  =  &  \overline\epsilon \gamma_M\chi +({\rm
fermions})^2 \\
\delta_\epsilon B & = &  \cdots \\
\delta_\epsilon \phi & = & \cdots 
\end{eqnarray*}



\section{Partially breaking the Supersymmetry by compactifying down
to four-dimensions}
 
Now we are ready to study what happens to these $N=1$ local
supersymmetries as we compactify
from ten dimensions to four dimensions in the way we discussed last
time. 
Let us recall what we hope to achieve.  We wish to find a
four-dimensional low energy effective theory 
which has $N=1$ super Poincar\'e symmetry. The odd generators
of the $N=1$ super Poincar\'e group in dimension four are  constant
four-dimensional Majorana spinors over four-dimensional Minkowski
space-time. (Recall, that the spin group from Minkowski space time is
$SL_2({\bf C})$ which has an obvious real four-dimensional
representation which is the underlying real representation of the defining
$2$-dimenisonal complex representation.  A section of this spin bundle
is a Majorana spinor.) 
We begin with a supersymmetry group generated by the sections of a
$16$-dimensional real spin bundle over ten-dimensional space-time, and
we must somehow break this down so that after compactification the
remaining supersymmetry generators are the constant Majorana spinors
over Minkowski four-space. We need to break the symmetries in
three-quarters of the components as well as for the non-constant
sections. 

As a basic model of how this works, recall what happens in the
Kaluza-Klein model, where we compactify along a circle from five
dimensions to four dimensions. We began with a five-dimensional theory
with the full five-dimensional diffeomorphism group as symmetries. But
the only diffeomorphisms which remained after compactifying and taking
the low energy effective action for small circle radius were the
diffeomorphisms which did not mix up the massive and massless modes.
Recall that the five-dimensional vacuum we consider has a metric of
the form 
$$\eta_{\mu\nu}dx^\mu dx^\nu+\frac{1}{r^2}d\theta^2.$$ The group of global
symmetries which remain unbroken (i.e., leave the 
vacuum invariant) are exactly the
Poincar\'e group of Minkowski four-space times the group of constant
$U(1)$ rotations. 
The infinitessimal generators of the original group of
symmetries are the vector fields on the five-dimensional
space-times. The action of a vector field $v$ on a metric  $g$ on
space-time is given by
$\delta_vg={\cal L}_v(g)$, the Lie derivative of $g$ in the
$v$-direction. The condition on a vector field that it preserve a given
vacuum $g_0$ is that $\delta_vg_0={\cal L}_v(g_0)=0$,  in other words,
that $v$ be a Killing vector field for $g_0$. 

Now let us return to our ten-dimensional theory. The situation is
analogous to the Kaluza-Klein model.  The main difference is that this
time our generators are supersymmetric (odd) generators instead of
(even) vector fields.  Nevertheless, the condition on a generator
$\epsilon$ that it be unbroken, i.e., preserve a given vacuum is that
the first variation of all the fields with respect to $\epsilon$ must
be trivial.  As we develop the argument we will list the properties
our solutions have.
Some come by assumption: others are consequences of these assumptions.
 The first property comes from the basic set-up of the reduction as
discussed last time:

\smallskip
\noindent{\bf Assumption (i): 
The ten-dimensional space-time manifold of the solution is of the form  
${\bf R}^4\times K^6$, where $K^6$ is compact, and the metric is of
the form 
$$\pmatrix{ \eta_{\mu\nu} & 0 \cr
		0 & (g_0)_{mn}}$$
where $g_0$ is a Ricci-flat metric on $K^6$ and $\eta_{mu\nu}dx^\mu
dx^\nu$ is the standard metric on Minkowski four-space. Furthermore,
the obvious isometric action of the four-dimensional Poincar\'e group
on this riemannian manifold is covered by an $N=1$ four-dimensional
super Poincar\'e symmetry group of the solution.}
 
Since the metric is split as a four-dimensional metric
times a six-dimensional metric, the same is true of the spin
bundles. They decompose 
$$S^+({\bf R}^4\times K^6)=\left(S^+({\bf R}^4)\otimes S^+(K^6)\right)\oplus
\left(S^-({\bf R}^4)\otimes S^-(K^6)\right).$$
Since there are no nonzero spinors in $S^\pm({\bf R}^4)$ which are
invariant under the Poincar\'e group, there are no nonzero spinors on
the ten-dimensional manifold which are invariant under the Poincar\'e
group.

\smallskip
\noindent{\bf Consequence (ii): All the fermionic fields
$(\psi_M,\lambda,\chi)$ of our solution must be trivial.} 

Now, since all the fermionic fields vanish, it follows that all the
boson fields are 
automatically invariant under any odd supersymmetry, simply by parity
considerations. Thus, to understand which of the $N=1$ local
supersymmetry generators are unbroken in this vacuum (i.e., leave it
invariant), we have only three equations to
study: 
\begin{eqnarray*}
\delta_\epsilon\chi & = & F_{MN}\gamma^{MN}\epsilon+({\rm fermions})^2 =0\\
\delta_\epsilon\psi_M  &  =  &  \nabla_M\epsilon -\frac{1}{4}
H_{MAB}\gamma^{AB}\epsilon +({\rm fermions})^2=0 \\
\delta_\epsilon\lambda  &  =  & (\gamma^m \nabla_m\varphi)\epsilon
+\frac{1}{24}H_{MNP}\gamma^{MNP}\epsilon +({\rm ferminos})^2=0. \\
\end{eqnarray*}
Furthermore, in these equations we can now drop the terms ${\rm (fermion})^2$
as well, leaving
\begin{eqnarray}\label{system1}
\delta_\epsilon\chi & = & F_{MN}\gamma^{MN}\epsilon =0 \\
\delta_\epsilon\psi_M  &  =  &  \nabla_M\epsilon -\frac{1}{4}
H_{MAB}\gamma^{AB}\epsilon  =0\label{system2} \\
\delta_\epsilon\lambda  &  =  &  \gamma^M\nabla_M(\phi)\cdot\epsilon
+\frac{1}{24}H_{MNP}\gamma^{MNP}\epsilon =0.\label{system3}
\end{eqnarray}


Our goal is, given the metric $g_0$ on $K^6$, to find $A, H,\phi$ so
that there are exactly four spinor fields $\epsilon$ (Majorana spinors
on  Minkowski four-space) for which these equations are
satisfied. 
One can then ask, if these equations are satisfied then are the
equations of motion satisfied as well, or is that an extra
condition? In this case, and in fact almost always, the (first-order)
supersymmetry equations imply the equations of motion.
As a simple example of this principle, let us suppose we are dealing
with a case of four-dimensional supergravity.
Thus, we have $H=F=0$.  Then the only equation for a Majorana spinor
$\epsilon$ to be an unbroken supersymmetry  is
$\nabla_\mu\epsilon=0$, i.e., $\epsilon$ is a supersymmetry if it is a
covariantly constant spinor over four-dimensional space-time. 
But the existence of a non-zero convariantly constant spinor field
implies that the metric on space-time must be Ricci-flat. 
The can be seen as follows:
$\nabla_\mu\epsilon=0$ implies that
$\gamma^\nu[\nabla_\nu,\nabla_\mu]\epsilon=0$ and hence that
$R_{\mu\nu}\gamma^\mu\epsilon=0$. Since $\epsilon$ is non-zero, this  
implies  that the Ricci curvature $R_{\mu\nu}=0$, i.e., that
space-time is Ricci-flat.

\section{Geometric consequence of the unbroken supersymmetry}

Our assumption is that we have a split
metric on ${\bf R}^4\times K^6$
$$g=\pmatrix{\eta_{\mu\nu} & 0 \cr 0 & (g_0)_{mn}}$$
with $g_0$ being a Ricci-flat metric on $K^6$.  We are also
assuming that we have nontrivial supersymmetry generator by Majorana
spinor fields $\epsilon$ on space-time, and that there is an unbroken
$N=1$ supersymmetry of our vacuum.  We shall make further assumptions
in order to keep the analysis simple.  Here, we are following the
analysis in [Candelas, Horowitz, Strominger, and Witten, Nucl. Phys. B
{\bf 258} (1985), p.46]. (To see what happens in the more general case
without these assumptions
consult [Witten, Nucl. Phys. B {\bf 268} (1986) p. 95] and [Strominger,
Nucl. Phys. B {\bf 274} (1986), p. 253].)

\smallskip
\noindent{\bf Assumption (iii): the scalar field $\phi=\phi_0$ is a constant
function and the three-form $H$ is zero.}
Recall from last time that the axion field $H$ and the $B$ field are
related by
$$H=\alpha'\left(CS(\nabla_g)-\frac{1}{30}CS(A)\right)+dB,$$
where $CS$ is the Chern-Simons functional. 
Thus, this assumption is equivalent to assuming that the field $B$ is
a closed two-form.


Of course, as we have already argued, all the fermion fields must be
zero. 
With all these assumptions, Equation~\ref{system1} is automatically
satisfied and Equations~\ref{system2} and~\ref{system3} simplify to 
\begin{eqnarray}\label{eqn1}
\delta_\epsilon\psi_M & = & \nabla_M\epsilon=0 \\ \label{eqn2}
\delta_\epsilon\chi & = & F_{MN}\gamma^{MN}\epsilon=0
\end{eqnarray}
In more invariant language, $\epsilon$ must be covariantly constant
Majorana Weyl spinor of positive chirality  and
the effect of Clifford multiplying this spinor by the curvature
two-form of the connection on the principal bundle is zero.

\subsection{Consequences of the first equation}

The $N=1$ supersymmetry that we are searching for is
to be  global super Poincar\'e symmetry.  In particular, the four
(odd) spinor generators of the superalgebra must make the standard
four-dimensional representation of the four-dimensional Poincar\'e
group acting in its usual way on the first factor of space-time. 
Since the spinor bundle $S^+({\bf R}^4\times K^6)$
decomposes as $S^+({\bf R}^4)\otimes S^+(K^6)\oplus S^-({\bf 
R}^4)\oplus S^-(K^6)$, the only Weyl-Majorana spinors of plus
chirality on the ten-dimensional manifold which transform as required
under the  Poincar\'e group of Minkowski four-space are
of the form  
$$\epsilon=\sum_i\eta^i_+(y)\cdot (\epsilon^i_+)^{(4)} + {\rm c.c.}$$
where $\eta^i_+(y)$ is a $6$-dimensional Weyl spinor  of positive
chirality on $K$ and
$(\epsilon^i_+)^{(4)}$ is a constant Weyl spinor of positive chirality on
Minkowski four-space.
(Here, ${\rm c.c.}$ means the complex conjugate.  The complex
conjugate  is a product of
Weyl spinors of negative chirality.)

Equation~\ref{eqn1} becomes two equations: one in the ${\bf
R}^4$-direction and one in the $K^6$-direction:
$$\delta_\epsilon\psi_\mu=\sum_i\eta_+^i\partial_\mu(\epsilon_+^i)^{(4)}
+ {\rm c.c.} =0$$
$$\delta_\epsilon\psi_m=\sum_i\nabla_m(\eta^i_+)(\epsilon^i_+)^{(4)}+ {\rm
c.c.}=0.$$ 
The first of these equations is automatic since $(\epsilon^i_+)^{(4)}$ is
constant. 
(We could also use it to prove that $(\epsilon^i_+)^{(4)}$ is constant.)
As to the second, since the complex conjugate of $\eta^i_+$ is
a spinor of opposite chirality, and since covariant differentiation
preserves chirality, it implies that
$\nabla_m\eta^i_+(y)=0$. 
This means that $\eta^i_+$ must be a covariantly constant
non-zero Weyl spinor of plus chirality on $K^6$.
It is clear that then that this spinor has constant norm, which we can
renormalize to be $1$.  In particular, $\eta^i_+$ is
nowhere zero.

In particular, we have seen that there must be at least one nonzero,
covariantly constant spinor $\eta_+$ on $K$.
Next,  we study the consequences of this fact  for the geometry of
$(K^6,g_0)$.




\begin{claim} The fact that $\eta_+$ is a covariantly constant spinor
field of norm one implies that 
the endomorphism of the tangent bundle of $K$ given by
$$J_m{}^n=i(\eta_+)^\dagger\gamma_m{}^n\eta_+$$
is a real endomorphism whose square is minus the identity.
Consequently, it defines an almost complex structure on $K$. This
endomorphism an isometry with respect to $g_0$
and is parallel under the metric.  This means that it defines an
integrable almost complex structure (i.e., a complex structure) for
which the metric $g_0$ is a K\"ahler metric.
\end{claim}

Here we are using the notation
$\gamma_m{}^n=\frac{1}{2}\left(\gamma_m\gamma^n-\gamma^n\gamma_m\right)$ and
$\gamma_n=\gamma^pg_{np}$.

\begin{proof}
In more invariant language, the element $J$ is determined by the
spinor field $\eta_+$ is simply the image under the natural maps
$$S(K)\otimes S^*(K)\cong {\rm Cl}(K)\cong \wedge ^*TK\to \wedge^2TK
\subset T^*K\otimes TK$$
of the element $\eta_+\otimes \eta_+^\dagger$.
It is clear from the defining formula that $J$ is a real tensor since
$\gamma_n{}^m=-(\gamma_n{}^m)^\dagger$. 
It follows  from the Fierz identity among Clifford matrices that
$$J_m{}^nJ_n{}^p=-\delta_m{}^p,$$
i.e., that $J$ is an endomorphism whose
square is minus the identity. 
Since $\eta_+$ and $\gamma_m{}^n$ are covariantly constant, it is clear
that $J$ is covariantly constant.
It is now a well-known fact that this implies that $J$ is integrable
and that the metric is K\"ahler, but let us briefly go through the
derivation of these facts.
The obstruction to integrability is the Nejenhuis tensor
$$N_{mn}{}^p=J_m{}^q(\nabla_qJ_n{}^p-\nabla_nJ_q{}^p)
-J_n{}^q(\nabla_qJ_m{}^p -\nabla_mJ_q{}^p).$$
Since $J$ is covariantly constant, it follows that $N=0$ and that $J$
defines a complex structure on $K$. 
Next, let us see that $J$ is an orthogonal transformation with respect
to the metric $g$ or equivalently that $g$ defines a hermitian metric
on the complex manifold determined by $J$.
This is simply the equation $J_m{}^qJ_n{}^pg_{pq}=g_{mn}$, which follows
easily from the definitions, the fact that $|\eta_+|^2=1$, and the
relations among the $\gamma$-matrices.
Last, since $J$ is covariantly constant, so is the  two-form
associated to the hermitian metric determined by $g$.   This means in
particular, that the two-form is closed, and consequently that the
metric is K\"ahler. 
\end{proof}

Note that when it is not assumed that $H= 0$, the almost complex
structure $J$ is not 
covariantly constant.  Nevertheless, it can still be shown that the
Nejenhuis tensor vanishes (see [Strominger, ibid.]), so that $J$
defines the structure of a complex manifold. Of course, it is not
K\"ahler. 

The complex structure on $K^6$ which is parallel under the metric
reduces the holonomy group of parallel translation for the metric from
$SO(6)$ to $U(3)$. But in fact, the existence of the self-parallel
spinor $\eta_+$ implies that the holonomy reduces further.
Decomposing the eight-dimensional Weyl spin bundle $S^+(Y)$ under
$U(3)$ we  get
$$8\mapsto 3_1\oplus 1_1\oplus \overline{3}_{-1}\oplus \overline{1}_{-1}$$
where the first integer indexes the representation of $SU(3)$ and the
subscript indexes the action of $U(1)$. (Thus, $3_1$ is the standard
three-dimensional complex representation of $U(3)$ and
$\overline{3}_{-1}$ is its dual, whereas $1_1$ is the determinant
representation of $U(3)$.) The existence of a parallel non-zero spinor
means that there must be a non-zero vector in the representation which
is invariant under the image of the holonomy representation. 
Clearly, the only vector in this space invariant under the $U(1)$
diagonal subgroup is the trivial vector.
 This means then that the image of holonomy lies in $SU(3)\subset U(3)$.
There are several possible subgroups with this property, $SU(3), SU(2)$,
etc, but $SU(3)$ is the only one which has exactly a one-dimensional
space of invariant spinors of each chirality.  Thus, if we want to
have exactly $N=1$ supersymmetry and no more, then the holonomy group
for the metric must be $SU(3)$. In this case there is a
one-dimensional space of covariantly constant spinor fields, generated
by say $\eta_+$.

There is another way to think about all of this.  Consider the action
of $Spin(6)=SU(4)$ on the complex four-dimensional space of Weyl spinors
$S^+$ of positive chirality. 
The stabilizer of an point in the unit sphere in $S^+$ is 
conjugate to $SU(3)\subset SU(4)=Spin(6)$. The image in $SO(6)$ is the
usual embedding of  $SU(3)\subset SO(6)$.
If there is a parallel non-zero spinor, then the
holonomy of the metric must lie in this $SU(3)$. 
This reduces the structure group of $K^6$ to $SU(3)$ and the reduction
is parallel under the metric. That is to say we have produced a parallel almost
complex structure $J$.  The arguments above show that this structure
is integrable and that the metric is K\"ahler with respect to this
complex structure.  Since the holonomy is reduced to $SU(3)$, the
metric is Ricci-flat. 


We have shown:

\smallskip
\noindent{\bf Consequence (iv): The odd generators of the super
Poincar\'e algebra are covariantly constant spinors $\epsilon$ of the
form $(\eta_+\epsilon_+^{(4)} + {\rm c.c})$
where $\eta_+$ is the unique (up to constant multiples)  covariantly
constant Weyl spinor of plus chirality 
on $K^6$ and $\epsilon_+^{(4)}$ is a constant Weyl spinor of plus
chirality on Minkowski four-space.}

In all cases, since  the holonomy is $SU(3)$, the Ricci tensor of
the metric is trivial, and in particular the first Chern class of the
metric is trivial. 
According to Yau's theorem on the existence of Ricci-flat metrics on
K\"ahler manifolds, $c_1=0$ is sufficient for the existence of such a
metric  (if the manifold is
non-simply connected then the condition must be interpreted as saying
that the canonical bundle of the K\"ahler manifold is trivial as a
holomorphic line bundle.)
According to the uniqueness part of Yau's theorem, the moduli of Ricci
flat K\"ahler metrics is the moduli space of complex structures and
K\"ahler classes. As we remarked in the last lecture, and as we shall
explain in more detail below, these moduli are associated with
massless fields in the low energy effective four-dimensional action.


This completes the study of Equations~\ref{system2} and~\ref{system3}
for the existence of an unbroken supersymmetry of the solution 
equations under the 
assumption that $H=0$ and $\phi=\phi_0$ is constant. We have
found the necessary and sufficient conditions for the existence of
non-zero supersymmetries satisfying these equations:

\smallskip
\noindent{\bf Consequence (v): The riemannian manifold $(K,g_0)$ be a
Calabi-Yau three-fold with holonomy $SU(3)$.} 
 
We have two more things to deal with: we must examine the consequences
of assuming $dH=0$, and we must show that
Equation~\ref{system1} is automatically satisfied under the
assumptions that we are making.

\subsection{The condition $H=0$.}

So far we have not mentioned the gauge field $A$ of the solution.
This field is constrained by our assumption that $H=0$.
We have
$$0=dH = tr (R\wedge R)-\frac{1}{30} Tr(F_A\wedge F_A),$$
where $tr$ refers to the trace in the standard representation, that is
to say the trace of the composite endomorphism of the tangent bundle,
and $Tr$ denotes the trace in the adjoint representation of the gauge
group. 
This means that
$$tr R\wedge R=\frac{1}{30} Tr F_A\wedge F_A,$$
There are two possible gauge groups $SO(32)$ and
$E_8\times E_8$. 
First, let us consider $SO(32)$. 
The trace in the 
adjoint representation is equal to  $30$ times the trace in the fundamental
representation of $SO(32)$.
Thus, we are looking for an $SO(32)$ connection
$A$ with $tr F_A\wedge F_A=tr R\wedge R$, where $tr$ on the
left-hand-side refers to the trace in the defining $SO(32)$ representation. 
One way to arrange this is to `embed the spin connection in the gauge
connections'.
That means that there is a natural embedding $SU(3)\subset
SO(32)$ with commutant being $SO(26)\times U(1)$.
Using this embedding, we can view the
connection on the  
tangent bundle (which has had its structure group reduced to $SU(3)$)
as being a principal $SO(32)$ connection.
Clearly, it satisfies the trace condition above.
In this way, we can construct the last of the fields $A$ necessary to
specify a solution.


We can perform a similar construction with $E_8\times E_8$, using only
one of the factors.  There is an embedding $SU(3)\subset E_8$ with
commutant being $E_6$.
Once again we can use this to embed the spin connection as a gauge
connection for the group $E_8\times E_8$, with symmetry group
$E_6\times E_8$. 



\smallskip
\noindent{\bf Assumption (vii): We obtain the gauge field by embedding
the spin connection.}

With this assumption, $H=0$ is consistent with the defining equation
for $H$.  

\subsection{The remaining supersymmetry equation}

Now let us examine the remaining equation for the existence of an
unbroken supersymmetry $\epsilon$:
$$\delta_\epsilon \chi =F_{MN}\gamma^{MN}\epsilon=0.$$
The two-form $F$ with values in the adjoint bundle to $P$ is acting on
the spinor field $\epsilon$ by Clifford multiplication of the two-form
to produce a section of the spin bundle tensored with the adjoint
bundle. 
We claim that with our hypotheses this equation is always satisfied. 
Since the gauge connection is, by assumption, obtained by embedding
the spin connection, this equation will follow immediately from the
analogous equation for the spin connection.
Thus, the result we need to establish is
\begin{equation}\label{needed}
\frac{1}{2}R_{MNA}{}^B\gamma^{MN}\epsilon =0,
\end{equation}
where $R_{MNA}{}^Bdx^M\wedge dx^N$ is the riemann curvature tensor.
The condition that $\epsilon$ is covariantly constant implies that the
curvature two-form evaluates trivially on $\epsilon$, i.e., that
$$R_{MNA}{}^B\gamma^A{}_B\epsilon=0.$$
Using the metric to raise and lower indices, we can rewrite this as
$$R_{MNAB}\gamma^{AB}\epsilon=0.$$
Now using the fact that our connection is torsion-free, we can switch
the roles of $AB$ and $MN$ and rewrite
$$R_{MNAB}\gamma^{MN}(\epsilon)=0,$$
which is equivalent to Equation~\ref{needed}.


Let us say a few words about what happens in the more general
situation where do not require $H=0$, following [Strominger, ibid].
The equation $F_{MN}\gamma^{MN}\epsilon=0$ is related to
the hermitian Yang-Mills condition for the connection. That is to say
the connection must be a holomorphic connection and the curvature must
be a primitive two-form, i.e., $\omega\wedge\omega\wedge F=0$ where
$\omega$ is the fundamental (K\"ahler) two-form. 
Then using this, the relationship between $F,R$, and $H$ and
supersymmetry we can then determine the value of $H$.


Let us return to the case of present interest where $H=0$.
In this case we have  completed the descriptions of all the fields and
a study of these supersymmetry equations.  We have shown: 

\begin{theorem}
Let $(K^6,J,g_0)$ is a Ricci-flat compact K\"ahler manifold with
holonomy $SU(3)$.
 Then  there is a classical solution to 
ten-dimensional supergravity coupled with super Yang-Mills with gauge
group either $E_8\times E_8$ or $SO(32)$. Space-time for this solution
is ${\bf R}^4\times K^6$, the 
metric on space-time is ,
$$\pmatrix{\eta_{\mu\nu} & 0 \cr 0 & (g_0)_{mn}}.$$
The axion field  $H$ is zero and the dilaton field $\phi$ a constant.
 The fermion fields (the dilatino, the gravitino, and the gaugino) of
this solution vanish and 
the gauge field $A$ is obtained by embedding the spin connection on
$K^6$ as a gauge connection. If the gauge group is $E_8\times E_8$,
then the commutant of the connection, (i.e., the automorphism group of
$P$ which leaves the connection invariant) is $E_6\times E_8$.
If the gauge group is $SO(32)$, then the
commutant is $SO(26)\times U(1)$.
Furthermore, this solution  has $N=1$ four-dimensional super
Poincar\'e symmetry lifting the natural action of the four-dimensional
Poincar\'e group space-time. 
\end{theorem}


\section{Massless fields in the low energy effective Lagrangian}

As we remarked when discussing the Kaluza-Klein theory, parameters for
the internal space of the high-dimensional theory (ten-dimensional in
this case) yield massless fields in the effective, low energy
four-dimensional Lagrangian. 
By Yau's existence and uniqueness theorem for Ricci-flat K\"ahler
metrics, the moduli space of Calabi-Yau structures on $K$ is the
moduli space of isomorphism classes of pairs $(J,\omega)$ consisting of a
 holomorphic structure $J$ on $K$ with $c_1=0$ (or more precisely with
trivial canonical bundle) and a K\"ahler class
$\omega$. Let us consider the space of holomorphic structures; a
similar  discussion can be made for the variation of the  K\"ahler
class.  The massless fields in the low energy
effective Lagrangian associated to the
variations of the Calabi-Yau metric come from expanding the metric
$g_{MN}$ on ten-space in terms of the decomposition of space-time
and keeping only the harmonic modes of eigenvalue zero on the internal
space:
$$g_{MN}(z)=M^\alpha(x)h_{mn}^\alpha(y)+ \cdots .$$
The other terms in this expansion are not relevant for the massless
fields coming from the variations of the Calabi-Yau structure:
They lead to massless fileds of other types.
Since we are working at a critical point, there will be no terms
in the effective low energy action involving only first derivatives of
$g_{MN}$.  As we take second 
derivatives we get three types of terms from
$M^\alpha(x)h_{mn}^\alpha(y)$. We can differentiate $M^\alpha(x)$ twice,
we can differentiate 
$h_{mn}^\alpha(y)$ twice, or we can differentiate each of
$M^\alpha(x)$ and $h_{mn}^\alpha(y)$ once. Terms of the last type cannot
contribute to the action 
because we have to pair the two derivatives using the metric of the
solution.  Since this metric is split, the pairing between the cross
terms vanishes. Applying two derivatives to $h_{mn}^\alpha(y)$ would
lead to a term in the Lagrangian which involves $M(x)^2$. Such a term
would give $M(x)$ a mass.  We know that there are no such terms
because $h_{mn}^\alpha(y)$ is a harmonic variation of the linearized
equations of motion.  Thus, the only terms that can
contribute to the action are where both derivatives are applied to
$M(x)$. 
This leads to a term in the low energy effective action of the form
$$\int_{{\bf R}^4} d^4x\sqrt{-{\rm
det}(g)}g^{\mu\nu}\partial_\mu M^\alpha(x)\partial_\nu M^\beta(x)
G_{\alpha\beta}(M(x)) $$
for some symmetric two-tensor $G=G_{\alpha\beta}dM^\alpha dM^\beta$ on
moduli space.  

In fact, it is possible to explicitly identity $G$.
In general, the tangent space to the moduli space of holomorphic
structures on a complex manifold $K$ is $H^1(TK)$. In the case that
$K$ is a Calabi-Yau three-fold, we can use the non-vanishing
holomorphic three-form $\Omega$ on $K$ (normalized to norm $1$) to give
an identification between $H^1(TK)$ and $H^{2,1}(K)$.
Let us suppose that we are at a smooth point of the moduli space of
holomorphic structures and let $M^\alpha, \alpha=1,\ldots,h^{2,1}$
be local coordinates on the moduli space of holomorphic structures
centered at the given point.
A direct computation shows that
under the identification of the tangent space of moduli space at $M$ with
$H^{2,1}(M)$ the pairing
$G(M)$ is the usual pairing of $(2,1)$-forms on $M$, namely
if $a,b\in H^{2,1}(M)$ then
$$\langle a,b\rangle=\int_Ma\wedge
\overline{b} .$$ 
If our manifold is simply connected, or in fact if $h^{1,0}=0$, then
this pairing is positive definite. (It is always non-degenerate.)
But, because it arises from an $N=1$ supersymmetric Lagrangian, we can
deduce highly non-trivial facts about this pairing.
It follows from a general $N=1$
supersymmetric  computation that this metric is in fact a K\"ahler
metric. 
Indeed, it is a Hodge metric.  Much more can be obtained using the
fact that not only can we work with the heterotic string in ten
dimensions, we can also work with the type II string. Putting these
together, we are able to extend this Lagrangian to a four-dimensional
$N=2$ supersymmetric Lagrangian.  This imposes even more stringent
restrictions, leading to what is referred to as `special
geometry'. For instance, the K\"ahler potential is locally 
determined by a certain holomorphic `potential'. In this way
supergravity makes contact with the theory of filtrations of Hodge
structures. 


\section{Relation to Grand Unification Theory.}

As we remarked in the first lecture, $E_6$ is a possible group for
GUT's.  In fact, the representation that we need to use of $E_6$ in
order to do GUT is the $27$-dimensional defining representation.  In
our context we begin with the adjoint representation of $E_8$ and
break it down to $SU(3)\times E_6$. Under this subgroup the
$248$-dimensional adjoint representation of $E_8$ decomposes
$$(8,1)\oplus (3,27)\oplus (\ov{3},\ov{27})\oplus (1,78).$$
Here, the first term is the adjoint representation of $SU(3)$ and the
last is the adjoint representation of $E_6$.  The second and third
terms are
tensor products of  three-dimensional representations of $SU(3)$ with
$27$-dimensional representations of $E_6$.  Thus, among the
representations that we are dealing with is the $27$-dimensional
representation of $E_6$, which fits with what GUT's requires.

Now the number of families massless modes in $(3,27)$ is given by an index
theorem.  Actually, the difference of massless modes of distinct (left
and right) chiralities is computed by the equation:
$$n^L_{27}-n^R_{27}={\rm
index}(i\dirac_3^{(6)})=\frac{1}{48}(2\pi)^3\int tr_3F\wedge
F\wedge F =\frac{-\chi(K)}{2},$$
where the Dirac operator and the trace are taken in the vector bundle
associated to the fundamental
three-dimensional representation of $SU(3)$.
The fact that it is the difference rather than the actual number of
families is not too serious a problem.  It turns out to be relatively
easy for massless modes of opposite chirality to combine and become
massive. Thus, one can believe that the absolute value of this index
is the correct number for the massless families.
Now experiment shows us that in the universe there are exactly three
families (quarks and leptons) of these light modes. 
This means that we should be looking for
Calabi-Yau's with Euler characteristic $\pm 6$.  A few of these have
been found.

{}From the physics point of view, it is very exciting that the simplest
compactifications lead naturally to realistic GUT gauge groups,
realistic chiral representations of matter, and the structure of family
replication.  This remains a significant source of optimism for the
eventual physical relevance of string theory.  But important and
difficult problems -- for example, supersymmetry breaking -- remain to 
be solved.


\end{document}







