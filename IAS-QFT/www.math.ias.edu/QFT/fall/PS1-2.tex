%From: Pierre Deligne <deligne@IAS.EDU>
%Date: Tue, 7 Jan 1997 11:44:24 -0500
%Subject: revised solution

\input amstex
\documentstyle{amsppt}
\magnification=1200
\pagewidth{6.5 true in}
\pageheight{8.9 true in}
\loadeusm

\catcode`\@=11
\def\logo@{}
\catcode`\@=13

\NoRunningHeads

\font\boldtitlefont=cmb10 scaled\magstep1

\def\dspace{\lineskip=2pt\baselineskip=18pt\lineskiplimit=0pt}
\def\wedgeop{\operatornamewithlimits{\wedge}\limits}
\def\w{{\mathchoice{\,{\scriptstyle\wedge}\,}
{{\scriptstyle\wedge}}
  {{\scriptscriptstyle\wedge}}{{\scriptscriptstyle\wedge}}}}
\def\Le{{\mathchoice{\,{\scriptstyle\le}\,}
{\,{\scriptstyle\le}\,}
  {\,{\scriptscriptstyle\le}\,}{\,{\scriptscriptstyle\le}\,}}}
\def\Ge{{\mathchoice{\,{\scriptstyle\ge}\,}
  {\,{\scriptstyle\ge}\,}
  {\,{\scriptscriptstyle\ge}\,}{\,{\scriptscriptstyle\ge}\,}}}
\def\undertext#1{$\underline{\vphantom{y}\hbox{#1}}$}
\def\vrulesub#1{\hbox{\vrule height7pt depth5pt\,}_{#1}}
\def\mapright#1{\smash{\mathop{\,\longrightarrow\,}%
     \limits^{#1}}}
\def\mynabla{{\nabla\!}}
\def\plus{\sssize +}

\def\eps{{\varepsilon}}

\def\xtil{\tilde{x}}

\def\Adot{\Dot{A}}
\def\Bdot{\Dot{B}}
\def\Xdot{\Dot{X}}
\def\xdot{\Dot{x}}
\def\psidot{\Dot{\psi}}

\def\dbR{{\Bbb R}}

%\def\GL{\text{\rm GL}} 
\def\Hom{\text{\undertext{\rm Hom}}}
%\def\aff{\text{\rm aff}} 
\def\red{\text{\rm red}}
\def\Lie{\text{\rm Lie}}
%\def\Map{\text{\rm Map}}


\def\scr#1{{\fam\eusmfam\relax#1}}

\def\scrA{{\scr A}}   \def\scrB{{\scr B}}
\def\scrC{{\scr C}}   \def\scrD{{\scr D}}
\def\scrE{{\scr E}}   \def\scrF{{\scr F}}
\def\scrG{{\scr G}}   \def\scrH{{\scr H}}
\def\scrI{{\scr I}}   \def\scrJ{{\scr J}}
\def\scrK{{\scr K}}   \def\scrL{{\scr L}}
\def\scrM{{\scr M}}   \def\scrN{{\scr N}}
\def\scrO{{\scr O}}   \def\scrP{{\scr P}}
\def\scrQ{{\scr Q}}   \def\scrR{{\scr R}}
\def\scrS{{\scr S}}   \def\scrT{{\scr T}}
\def\scrU{{\scr U}}   \def\scrV{{\scr V}}
\def\scrW{{\scr W}}   \def\scrX{{\scr X}}
\def\scrY{{\scr Y}}   \def\scrZ{{\scr Z}}


\document
\line{{\boldtitlefont Witten's Problems}, Set One --- 
N$^{\text{o}}$. 2 
\hfill (solution written by P. Deligne \& D. Freed)}
\smallskip
\hbox to \hsize{\hrulefill}

\bigskip
\subhead
Preface 
\endsubhead

\dspace
The space of maps from $R^{1,1}$ to $M$ is some kind of
infinite dimensional supermanifold $\scrF$.
We will not try to define it as a topological space
provided with a sheaf of $\dbR$-algebras.
We will only define it as a functor: \ for $B$ a
supermanifold, a map from $B$ to $\scrF$ is by
definition a morphism $R^{1,1}\times B\to M$.
It is also called a $B$-point of $\scrF$, and it is
viewed as a family of points parametrized by $B$.
In the exercise, the phrase ``let $X$ be a map from
$R^{1,1}$ to $M$'' is an abuse of language for ``let $X$
be a map from $\dbR^{1,1}$ to $M$, depending on some
parameter $b\in B$, for some super space $B$'', i.e.
``let $B$ be a supermanifold and let $X$ be a map from
$R^{1,1}\times B$ to $M$''.

This kind of interpretation of what a space of maps is
does not apply only in the superworld.
If we were to consider (as in exercise 1) the space of
maps from $\dbR$ to $M$, one could try to put on it a
structure of infinite dimensional manifold but, for the
kind of problem we are considering, this is extra
baggage.
What is important is to know, for $B$ a space ``of
parameters'' (now an ordinary manifold), what is a
family of maps $\dbR\to M$ parametrized by $B$.

``Working in components'' means reinterpreting the
functor $B\mapsto B$-points of $\scrF$ as follows.

\noindent
A morphism $X\colon\,\dbR^{1,1}\times B\to M$ gives us

\medskip
\roster
\item"{(a)}"
a restriction $x\colon\,\dbR\times B\to M$.
We view here $\dbR$ as the subvariety $\theta=0$ of
$\dbR^{1,1}$,

\smallskip
\item"{(b)}"
an odd section $\psi:=\frac{\partial}{\partial\theta}\,X
\vrulesub{\dbR\times B}$ of the pullback by $x$ of the
tangent bundle $T$ of $M$.
\endroster

\medskip
The construction $X\mapsto (x,\psi)$ is a bijection.
Note that $x^*T$ is an ``even vector bundle'', in the
sense that it is of rank $(*,0)$.
This does not prevent it from having odd sections,for
instance the product of the pullback of a section
of $T$ with an odd function on $B$.

This will usually be expressed with the abuse of
language consisting in omitting $B$: \ a basis $B$ is
tacitly assumed to always be there, and one will say
that to give a morphism $X$ from $\dbR^{1,1}$ to $M$ is
the same thing as to give its restriction $x$ to $\dbR$
and its derivative
$\psi=\frac{\partial}{\partial\theta}X$ on $\dbR$ which
is an odd section of $x^*T$.
Taken literally, this statement would be useless: \
with no $B$ tacitly assumed, the even vector bundle
$x^*T$ would have no nonzero odd section.
When a $B$ is present, one should beware that to give
$x\colon\,\dbR\times B\to M$ is not just giving a
morphism from $\dbR\times B_{\red}$ to $M$.

The lagrangian density is a map from
$\scrF:=\underline{\Hom}(\dbR^{1,1},M)$ to the space of
densities on $\dbR^{1,1}$.
This is to be interpreted in its functorial meaning.
A family of densities on $\dbR^{1,1}$, parametrized by
$B$, is by definition a section on $\dbR^{1,1}\times B$
of the pullback of the line bundle of densities on
$\dbR^{1,1}$.
The lagrangian density attaches, functorially in $B$ to
any family of maps from $\dbR^{1,1}$ to $M$ parametrized
by $B$, a family of densities on $\dbR^{1,1}$
parametrized by $B$.

``Families of densities'' are also called ``relative
densities''.
An integration map is defined, which to a relative
density on $V\times B$ over $B$ with support proper over
$B$, attaches a function on $B$.
If $V$ is a supermanifold of dimension $(p,q)$, the line
bundle of relative densities is of dimension $(1,0)$ or
$(0,1)$, depending on the parity of $q$, and the
integration map is even.
The spaces of critical points of the lagrangian is a
subspace of the space of all maps from $\dbR^{1,1}$ to
$M$.
As such, it is first to be defined as a functor: \
one has to define what it means for a map
$X\colon\,B\to\underline{\Hom}(\dbR^{1,1},M)$, i.e.
$\dbR^{1,1}\times B\to M$, to be a map to the subspace
$S$ of $\underline{\Hom}(\dbR^{1,1},M)$.
The following two definitions are equivalent, with the
second being closer to what is done in computations.

\medskip\noindent
(i)\enspace
For any supermanifold $V$, pointed by $o\in V$, and any
$X_V\colon\,B\times V\to
\underline{\Hom}(\dbR^{1,1},M)$, such that (a) \ 
it extends $X$: \ $X$ is $X_V$ restricted to $B\times
0$, (b) \ 
outside a compact region of $\dbR^{1,1}$, $X_V$ coincide
with $X$ ($X_V$ is a family of deformations of $X$, with
compact support, parametrized by $V$), the following
holds. 
Consider $\scrL(X_V)-\scrL(X)$, where
$\scrL(X)$ is abuse of language for ``pullback of
$\scrL(X)$ by $\dbR^{1,1}\times B\times
V\to\dbR^{1,1}\times B$.
It is a relative density on $\dbR^{1,1}\times B\times V$
with support proper over $B\times V$.
Integrate it, to get a function $\Delta\int\scrL$ on
$B\times V$.
One requires that the derivative of this function in the
$V$-direction be zero on $B\times\{0\}\subset B\times V$.

\smallskip\noindent
(ii)\enspace
In the second definition, one requires that after any
change of basis $B'\to B$, (i) holds for $V=(\dbR,0)$.

\medskip
This defines the space of stationary points as a
functor.
In the exercise at hand, and assuming $M$ to be
complete, this functor is representable: \ the space of
stationary points ``is'' an ordinary supermanifold of
dimension $(2\dim\,M, \dim\,M)$, as we will see.

\bigskip
\subhead
Solution
\endsubhead
(a)\enspace
The coordinates $(t,\theta)$ on $\dbR^{1,1}$ define a
density $[t,\theta]$, also written $dt\,d\theta^{-1}$.
The lagrangian density considered is
$$
\scrL=-\tfrac12 \,dt\,d\theta^{-1}(\Xdot,DX).
\tag2.1
$$
The minus sign is to make (2.5) below reasonable.

\smallskip\noindent
(b)\enspace
If we choose local coordinates $\xtil^i$ on $M$, a map
$X\colon\,\dbR^{1,1}\to M$ can be written
$$
X=(x^i(t)+\theta\psi^i(t)),
$$
meaning $X^*\xtil^i=x^i(t)+\theta\psi^i(t)$
with $x^i(t)$ (resp. $\psi^i(t)$) an even (resp. odd)
function on $\dbR$, pulled back to $\dbR^{1,1}$ by
$(t,\theta)\mapsto t$.
Independently of the local coordinates, $x$ is a map
from $\dbR$ to $M$ and $\psi$ is an odd section of
$x^*T$, for $T$ the tangent bundle of $M$.
The meaning of this, and in particular of ``odd section
of $x^*T$'', was explained in the preface.
The philosophy is that we tacitly work over a basis $B$,
functorially in $B$.
The map $x$ is the restriction of $X$ to the subvariety
$\dbR$ of $\dbR^{1,1}$ defined by $\theta=0$, while
$\psi$ is the restriction of $\partial_\theta X$ to
$\dbR$.
On $\dbR$, the vector fields $\partial_\theta$ and $D$
are equal, and it will often be more convenient to
describe $\psi$ as the restriction of $DX$ to $\dbR$.
We will compute the density $\scrL$ on $\dbR^{1,1}$ in
term of $(x,\psi)$.

In local coordinates, $\scrL$ is $-dt\,d\theta^{-1}$
times
$$
\tfrac12 X^*(g_{ij})\partial_t(X^*(\xtil^i))
D(X^*(\xtil^j)).\tag2.2
$$
As $\theta^2=0$, $X^*(g_{ij})$ is given by a finite
Taylor series
$$
X^*(g_{ij})=g_{ij}(x(t)+\theta \psi(t))=
g_{ij}+\theta\psi^k\partial_kg_{ij}
$$
with $g_{ij}$, $\partial_kg_{ij}$ and $\psi^k$ evaluated
at $x(t)$ and $t$.
Expanding (2.2) gives
$$
\align
\tfrac12(g_{ij}+\theta\psi^k\partial_k g_{ij})
&(\xdot^i+\theta\psidot^i)(\psi^j-\theta \xdot^j)\tag2.3\\
&=\tfrac12 g_{ij}\xdot^i\psi^j+\tfrac12\theta
(g_{ij}\psidot^i\psi^j+\partial_k
g_{ij}\xdot^i\psi^k\psi^j-g_{ij}\xdot^i\xdot^j).
\endalign
$$

As $\psi^k\psi^j=-\psi^j\psi^k$, we have$\partial_k
g_{ij}\psi^k\psi^j=-\partial_j g_{ik}\psi^k\psi^j$.
As $g_{ij}=g_{ji}$, we similarly have $\partial_i
g_{jk}\psi^k\psi^j=0$, and
$$
\partial_k g_{ij}\psi^k\psi^j=\tfrac12(\partial_k
g_{ij}-\partial_jg_{ik}+\partial_i g_{jk})=
g_{\ell j}\Gamma_{ik}^\ell,
$$
where the $\Gamma$ are the Christoffel symbols defining
the Levi-Civita connection.
In (2.3), this allows us to rewrite 
$$
g_{ij}\psidot^i+\partial_k g_{ij}\xdot^i\psi^k=
g_{ij}\psidot^i+g_{\ell j}
\Gamma_{ik}^\ell \xdot^i\psi^k=g_{ij}(\mynabla_t\psi)^i,
$$
giving for (2.2)
$$
\split
\tfrac12
g_{ij}\xdot^i\psi^j+\tfrac12\theta &(g_{ij}(\mynabla_t\psi)^i
\psi^j-g_{ij}\xdot^i\xdot^j)\\
&=\tfrac12(\xdot,\psi)+\tfrac12\theta((\mynabla_t\psi,
\psi)-(\xdot,\xdot)).
\endsplit
\tag2.4
$$

Here is a shortcut from (2.3) to (2.4).
Write (2.2) in the form $f(t)+\theta g(t)$.
We have to compute $f(t_0)$ and $g(t_0)$ for each $t_0$.
We now remember that we always work over a space $B$ of
parameters.
The point $t_0$ is in fact a section of $\dbR\times B\to
B$.
The local coordinates $\xtil_i$ used can be taken to vary
with the parameters.
This makes it possible to choose them so that at
$x(t_0)$, i.e. along the section $x(t_0)$ of $M\times
B\to B$, the metric tensor $g_{ij}$ has vanishing first
derivatives.
The Christoffel symbols then vanish at $x(t_0)$, and
(2.4) at $t_0$ results immediately from (2.3).

Here is a coordinate free derivation of (2.4).
We use that for any function $F$ on $\dbR^{1,1}$, if we
write $F(t,\theta)=f_1(t)+\theta f_2(t)$, we have
$$
\align
f_1 &=F\text{ restricted to $\dbR$}\\
f_2 &= DF\text{ restricted to $\dbR$}\,\,.
\endalign
$$
We have $\psi=DX$ restricted to $\dbR$, so that for
$F=(\Xdot,DX)$, $f_1$ is $(\xdot,\psi)$, while $f_2$ is
the restriction to $\dbR$ of
$$
D(\Xdot,DX)=(\mynabla_D\Xdot,DX)+(\Xdot,\mynabla_D DX).
$$
The Levi-Civita connection is torsion free, and $D$ and
$\partial_t$ commute, while
$D^2=\frac12[D,D]=-\partial_t$.
It follows that
$$
\align
\mynabla_D\Xdot &=\mynabla_t DX\\
\mynabla_D DX &=\tfrac12(\mynabla_D DX+\mynabla_D DX)
  =-\partial_t X
\endalign
$$
and $D(\Xdot,DX)$, restricted to $\dbR$, is
$$
(\mynabla_t\psi,\psi)+(\Xdot,-\Xdot).
$$
This agrees with (2.4).

Let $\scrL'$ be the density on $\dbR$ deduced from
$\scrL$ by integrating along the fibres of the
projection from $\dbR^{1,1}$ to $\dbR$ given by
$(t,\theta)\mapsto t$.
{}From (2.4), we get
$$
\scrL'=\tfrac12(\xdot,\xdot)dt-\tfrac12
(\mynabla_t\psi,\psi)dt\tag2.5
$$

The space of critical points is the same for $\scrL$ and
$\scrL'$: it is the space of $X=(x,\psi)$ such that, for
any deformation with compact support $X(u)$ of $X$, one
has $\int\partial_u\scrL=0$ (resp.
$\int\partial_u\scrL'=0$) at $u=0$, and the density
$\partial_u\scrL'$ on $\dbR$ is deduced by
$\theta$-integration from the density $\partial_u\scrL$
on $\dbR^{1,1}$.

\smallskip\noindent
(c)\enspace
The computation of the Euler-Lagrange equations is
parallel to the computations in problem 1.
If $X$ depends on an additional parameter $u$, and if
$\delta$ stands for $\partial_u$ or $\mynabla_u$, one
has
$$
\align
\delta(\Xdot,DX) &=(\delta\Xdot,DX)+(\Xdot,
  \delta DX)=(\mynabla_t\delta
X,DX)+(\Xdot,\mynabla_D\delta X)\\
&=\partial_t(\delta X,DX)-(\delta X,\mynabla_t DX)+
  D(\Xdot,\delta X)-(\mynabla_D\Xdot,\delta X).
\endalign
$$
As $\mynabla_t DX=\mynabla_D\Xdot$, this gives
$$
\delta\scrL=-dt\,d\theta^{-1}\left[
-(\delta X,\mynabla_t DX)+\tfrac12(\partial_t(\delta
X,DX)+D(\Xdot,\delta X))\right].\tag2.6
$$

The vector fields $\partial_t$ and $D$ are
divergence free: the corresponding Lie derivative
kills $dt\,d\theta^{-1}$.
It follows that $dt\,d\theta^{-1}$ multiplied by the
second term in [\quad] of (2.6) is an exact
differential (see (d) below) and the Euler-Lagrange
equation is
$$
\mynabla_t DX=0.
\tag2.7
$$

In the computation leading to (2.6), we took the
additional parameter $u$, and hence $\delta X$, to
be even.
Why this suffices is explained in Deligne's appendix
``Even rules'' to Bernstein's lecture.
Basically, an odd $\delta X$ can be replaced by
$\eps\delta X$, for $\eps$ a new odd parameter.

To express (2.7) in term of the components $(x,\psi)$ of $X$,
one restricts $\mynabla_t DX$ and $\mynabla_D\mynabla_t DX$ to
$\dbR\subset\dbR^{1,1}$ ($\theta=0$).
As $\dbR\subset\dbR^{1,1}$ is stable by $\partial_t$, the first
gives
$$
\mynabla_t\psi=0.\tag2.8
$$
For the second, permuting $\mynabla_D$ and $\mynabla_t$
introduces a curvature term, while $\mynabla_D DX=-\partial_t
X$: \ one obtains
$$
R(\psi,\xdot)\psi-\mynabla_t\xdot=0\tag2.9
$$

Another method to obtain the Euler-Lagrange equations (2.8)
(2.9) is to start from the lagrangian $\scrL'$ (2.5).

Writing again $\delta$  for $\partial_u$ or
$\mynabla_u$, we have
$$
\delta\scrL'=(\delta\xdot,\xdot)dt-\tfrac12
((\delta\mynabla_t\psi,\psi)+(\mynabla_t\psi,\delta\psi))
dt
$$
Permuting $\delta$ and $\mynabla_t$ introduces a
curvature term:
$$
\delta\mynabla_t\psi=R(\delta
X,\xdot)\psi+\mynabla_t\delta\psi.
$$
Integrating by part:
$$
\align
&(\mynabla_t\delta\psi,\psi)=\partial_t(\delta\psi,\psi)
  -(\delta\psi,\mynabla_t\psi)\\
&(\delta\xdot,\xdot)=(\mynabla_t\delta x,\xdot)=
  \partial_t(\delta x,\xdot)-(\delta x,\mynabla_t\xdot)
\endalign
$$
and observing that
$(\delta\psi,\mynabla_t\psi)=-(\mynabla_t\psi,\delta\psi)$,
we obtain
$$
\delta\scrL'=-\left[(\mynabla_t\psi,\delta\psi)+
  \tfrac12 (R(\delta x,\xdot)\psi,\psi)+(\delta x,
\mynabla_t\xdot)\right]dt+d\left((\delta x,\xdot)
-\tfrac12(\delta\psi,\psi)\right)
$$

The Bianchi identity tells that if in $(R(\delta
x,\xdot)\psi,\psi)$ we cyclically permute $\delta x$,
$\psi$ and $\psi$, taking parities into account, the
sum is zero:
$$
\align
&(R(\delta x,\xdot)\psi,\psi)-(R(\psi,\xdot)\delta
x,\psi)+(R(\psi,\xdot)\psi,\delta x),\quad\text{hence}\\
&(R(\delta
x,\xdot)\psi,\psi)=-2(R(\psi,\xdot)\psi,\delta x).
\endalign
$$
We obtain
$$
\align
\delta\scrL'= &-[(\mynabla_t\psi,\delta\psi)+
(\mynabla_t\xdot-R(\psi,\xdot)\psi,\delta x)]dt\tag2.10\\
&+d\left( (\delta x,\xdot)-\tfrac12(\delta\psi,\psi)
\right),
\endalign
$$
giving again (2.8) and (2.9) as Euler-Lagrange
equations.

The space $Y$ of critical points of $\scrL$ or $\scrL'$
is hence the space of solutions $(x,\psi)$ of (2.8),
(2.9).
The reduced space $Y_{\red}$ is obtained by imposing
$\psi=0$: \ It is the space found in problem 1, the
space of solutions $x(t)$ of $\mynabla_t\xdot=0$.

The normal bundle of $Y_{\red}$ in $Y$ is found by
linearizing (2.8), (2.9) around $\psi=0$: \ it is the
odd vector bundle whose (even) sections are the odd
sections $\psi$ of $x^*T$ obeying $\mynabla_t=0$.

The Cauchy data for (2.8) (2.9) are the data at some
$t_0$ of $x(t_0)$, $\xdot(t_0)$ and $\psi(t_0)$.
In term of $X$: \ $X$, $\Xdot$ and $DX$ at $(t_0,0)$.
If $M$ is complete, $Y$ maps isomorphically to the space
of Cauchy data.
This gives a description of $Y$ as the vector bundle
$T\times \Pi T$ over $M$ (viewed as a space).
In particular, we get a description of $Y$ as an odd
vector bundle over $Y_{\red}$.
Warning: \ this structure, and even the corresponding
map $Y\to Y_{\red}$, depend on the choice of $t_0$.

\smallskip\noindent
(d)\enspace
The $2$-form $\omega$ on the space $Y$ of critical
points can be computed using $\scrL$ or $\scrL'$.
We first use $\scrL'$.
The general recipe is to fix $t_0$, to extract from
(2.10) the $1$-form
$$
\alpha_{t_0}=(\delta x,\xdot)-\tfrac12(\delta\psi,\psi)
\quad\text{(at $t_0$)}\tag2.11
$$
on $Y$ and to take $\omega=d\alpha_{t_0}$.

Let us identify $Y$ with the space of Cauchy data at
$t_0=0$: \ a point of $Y$ becomes a triple
$(x,\xdot,\psi)$ with $x$ a point of $M$, $\xdot$ an
(even) tangent vector at $x$ and $\psi$ an (odd) tangent
vector at $M$.
If we identify the tangent bundle with the cotangent
bundle using the riemannian metric, $Y$ becomes the pull
back to $T^*M$ of the odd tangent bundle of $M$.
The term $(\delta x,\xdot)$ in (2.11) becomes the pull
back to $Y$ of the canonical $1$-form $pdq$ on $T^*M$,
and contributes to $\omega$ the pull back of the
symplectic form $dp\,dq$ of $T^*M$.

The second term in (2.11) makes sense whenever on a
manifold $X$ (here $T^*M$) one considers an odd vector
bundle $\Pi T$, for $T$ an orthogonal vector bundle
with connection (here the pull back of the tangent
bundle of $M$).
It is the $1$-form whose pull back by any section $\psi$
(an odd section of $T$) is $-\frac12(\nabla\psi,\psi)$.
Its exterior derivative is the $2$-form $\omega_T$ whose
pull back by any $\psi$ is given in local coordinates by
$$
\align
\psi^*(\omega_T)_{i,j} &=-\tfrac12\partial_i(\mynabla_j
\psi,\psi)+\tfrac12\partial_j(\mynabla_i\psi,\psi)\\
&=-\tfrac12(\mynabla_i\mynabla_j\psi,\psi)-
\tfrac12(\mynabla_j\psi,\mynabla_i\psi)\\
&\quad\,\,+\tfrac12(\mynabla_j\mynabla_i\psi,\psi)
+\tfrac12(\mynabla_i
\psi,\mynabla_j\psi)\\
&=-\tfrac12(R_{ij}\psi,\psi)+(\mynabla_i\psi,\mynabla_j\psi)
\endalign
$$
for $R$ the curvature $2$-form (with values in
endomorphisms of $T$), i.e.
$$
\psi^*\omega_T=\tfrac12(\nabla\psi,\nabla\psi)-\tfrac12
(R\psi,\psi)
$$
where the first term combines inner product (in $T$)
and exterior product of differential forms.
Final result:
$$
\omega=\omega_{T^*M}+\tfrac12(\nabla\psi,\psi)-
\tfrac12(R\psi,\psi).\tag2.12
$$

To compute in terms of $\scrL$, one should start with
the second term in (2.6) being
$$
d\left[\tfrac12 i_t(-dt\,d\theta^{-1}(\delta X,DX))
  +\tfrac12 i_D(dt\,d\theta^{-1}(\Xdot,\delta
X))\right].
$$
In [\quad], we have a $1$-form $A$ on the space of $X$,
with values in integral codimension $1$-forms  on
$\dbR^{1,1}$.
The general recipe is to fix a space-like hypersurface
$\Gamma$, for instance $t=t_0$, to integrate $A$ on
$\Gamma$ to get a $1$-form $\alpha_\Gamma$ on $Y$, and
to take $\omega=d\alpha_\Gamma$.

The restriction of $i_X(\,\ldots\,)$ to $\Gamma$ depends
only on the component of $X$ normal to $\Gamma$.
If $\Gamma$ is $t=t_0$, we have
$$
i_t(dt\,d\theta^{-1})\vrulesub{\Gamma}=d\theta^{-1}\quad
\text{and}\quad i_D(dt\,d\theta^{-1})\vrulesub{\Gamma}=-\theta
i_t(dt\,d\theta^{-1})\vrulesub{\Gamma}, 
$$
giving
$$
A\vrulesub{\Gamma}=-\tfrac12 d\theta^{-1}(\delta X,DX)
-\tfrac12\theta d\theta^{-1}(\Xdot,\delta X).
$$

The integral of $d\theta^{-1}f$ on $\Gamma$ is
simply $\partial_\theta f$, or $Df$, evaluated at
$(t_0,0)$:
$$
\align
\int\nolimits_{\Gamma}A &=-\tfrac 12 D(\delta X,DX)
  +\tfrac12(\Xdot,\delta X)\quad\text{at $(t_0,0)$}\\
&=-\tfrac12(\mynabla_D\delta X,DX)-\tfrac12
(\delta X,\mynabla_D DX)+\tfrac12 (\Xdot,\delta X)\quad
  \text{at $(t_0,0)$}
\endalign
$$
We have $\mynabla_D\delta X=\delta\psi$ and $\mynabla_D
DX=-\xdot$, at $(t_0,0)$, so that (not surprisingly)
$$
\alpha_\Gamma\quad\text{is given by (2.11)}.\tag2.13
$$

\smallskip\noindent
(e)\enspace
The space $\dbR^{1,1}$ is a group for the group law
$$
(t,\theta)*(t',\theta')=(t+t'+\theta\theta',
\theta+\theta').
$$
Let us compute the bracket in the Lie algebra
$\Lie(\dbR^{1,1})$, identified with the tangent space to
$\dbR^{1,1}$ at $(0,0)$.
To $\partial_\theta$ at $(0,0)$ correspond over the odd
dual numbers $\dbR[\eps]$ the ``infinitesimal''
element $(0,\eps)$, whose left translate by $(t,\theta)$
is $(t+\theta\eps,\theta+\eps)=(t-\eps\theta,\theta+\eps)$,
corresponding to the left invariant vector field
$\partial_\theta-\theta\partial_t=D$.
Similarly, $\partial_t$ at $(0,0)$ is the value at
$(0,0)$ of the left invariant vector field $\partial_t$.
On $\dbR^{1,1}$, $D^2=-\partial_t$.
It follows that 
$$
\text{in $\Lie(\dbR^{1,1})$},\,\,
\partial_\theta^2=-\partial_t.
\tag2.14
$$

On $\dbR^{1,1}$, the vector fields $D$, $\partial_t$ and
the density $dt\,d\theta^{-1}$ are invariant by the
group of left translations.
Being built from them, so is $X\mapsto \scrL(X)$.

For a while, let us consider more generally maps
$X\colon\, V\to M$ and a lagrangian density $\scrL(X)$
on $V$, with $X\mapsto\scrL(X)$ invariant by a group $G$
acting on $V$.
We also suppose given a $G$-stable cohomology class of
``space like'' hypersurfaces $\Gamma$, and assume that
the corresponding $2$-form $\omega$ on the space $Y$ of
extremals is non degenerate.
To each $\Gamma$ corresponds a $1$-form $\alpha_\Gamma$
on $Y$, with $\omega=d\alpha_\Gamma$ and
$$
\alpha_\Delta-\alpha_\Gamma=d\int\nolimits_\Gamma^\Delta
\scrL.
\tag2.15
$$

By transport of structures, the group $G$ acts on $\Hom(V,M)$,
respects $Y$ and $\omega$, and
$g_*(\alpha_\Gamma)=\alpha_{g\Gamma}$.
For $\tau\in\Lie(G)$, we will write $[\tau]$ for the
vector field
$$
\partial_\tau X(gv)\qquad\text{at}\qquad X(v)
$$
on $\Hom(V,M)$.
The derivation is taken in $g$, at $g=e$.
The action of $G$ on $\Hom(V,M)$ is $g\colon\,X\mapsto
X(g^{-1}v)$; it induces $\tau\mapsto -[\tau]$ from
$\Lie(G)$ to vector fields on $\Hom(V,M)$.
For a function on $\Hom(V,M)$, one has
$$
\align
&\partial_\tau(gF)=\partial_\tau(F(g^{-1}X))=[\tau]F\,\,,
\qquad\text{and}\tag2.16\\
&\tau\longmapsto[\tau]\quad\text{is compatible with
brackets}.\tag2.17
\endalign
$$

For $\tau\in\Lie(G)$, we now compute a generating
function $T(\tau)$ for the symplectic vector field
$[\tau]$ on $E$: \  $-dT(\tau)=i_{[\tau]}\omega$.
One has $\omega=d\alpha_\Gamma$, hence
$$
i_{[\tau]}\omega=\scrL_{[\tau]}\alpha_\Gamma-
di_{[\tau]}\alpha_\Gamma.
$$
The Lie derivative is
$$
\scrL_{[\tau]}\alpha_\Gamma=\partial_\tau g\alpha_\Gamma
=\partial_\tau\alpha_{g\Gamma}
$$
with $\partial_\tau$ being a derivative in $g$ at $g=e$.
By (2.15), this equals
$\partial_\tau d\int\nolimits_\Gamma^{g\Gamma}\scrL$, and
we take
$$
T(\tau)=i_{[\tau]}\alpha_\Gamma-\partial_\tau
\int\nolimits_\Gamma^{g\Gamma}\scrL.
\tag2.18
$$
It is independent of the choice of $\Gamma$:
$$
\split
\biggl(i_{[\tau]}\alpha_\Delta-\partial_\tau
  &\int\nolimits_\Delta^{g\Delta}\scrL\biggr)-
\biggl(i_{[\tau]}\alpha_\Gamma-\partial_\tau
\int\nolimits_\Gamma^{g\Gamma}\scrL\biggr)\\
&=i_{[\tau]}d\int\nolimits_\Gamma^\Delta\scrL-\partial_\tau
  \int\nolimits_{g\Gamma}^{g\Delta}\scrL=[\tau]
\int\nolimits_\Gamma^\Delta\scrL-\partial_\tau
\int\nolimits_{g\Gamma}^{g\Delta}\scrL,
\endsplit
$$
and
$$
[\tau]\int\nolimits_\Gamma^\Delta\scrL(X)=\partial_\tau
\int\nolimits_\Gamma^\Delta\scrL(X(gv))=\partial_\tau
\int\nolimits_{g\Gamma}^{g\Delta}\scrL(X).
$$

It follows from (2.17) that, up to a constant,
$$
T([\tau_1,\tau_2])=\{T(\tau_1),T(\tau_2)\}=[\tau_1]
T(\tau_2).\tag2.19
$$
In fact, (2.19) holds exactly: \ as $T$ is independent
of $\Gamma$, we have by transport of structures
$$
gT(\tau_2)=T(ad\,g(\tau_2)).\tag2.20
$$
By (2.16) and the linearity of $T(\tau)$ in $\tau$,
(2.19) follows from (2.20) by applying the derivative
$\partial_\tau$ in $g$.

We now take $V=\dbR^{1,1}$, with the supergroup
$G=\dbR^{1,1}$ acting by left translations.
The infinitesimal left translation (right invariant
vector field) $\partial_\tau(gv)$ in $\dbR^{1,1}$ is
$$
\alignat3
&D^{\plus}=\partial_\theta+\theta\partial_t
  &\qquad &\text{for} &\quad &\tau=(\partial_\theta
  \text{ at }(0,0)),\\
&\partial _t &&\text{for} &&\tau=(\partial_t\text{ at }
  (0,0))
\endalignat
$$
and the vector field $[\tau]$ on $\Hom(\dbR^{1,1},V)$ is,
respectively, $D^{\plus}X$ at $X$ and $\partial_t X$ at
$X$.
The generating functions are given by (2.18): \ 
respectively
$$
\align
Q &=i_{[\partial_\theta]}\alpha_\Gamma-\partial_\eta
\int\nolimits_\Gamma^{(0,\eta)\Gamma}\scrL
  \qquad\text{and}\tag2.21\\
H &=i_{[\partial_t]}\alpha_\Gamma-\partial_t
\int\nolimits_\Gamma^{(t,0)\gamma}\scrL,
\endalign
$$
with the derivation evaluated at $\eta$ (resp. $t)=0$.
Take for $\Gamma$ the hypersurface $t=0$.
Its transform by $(0,\eta)$ is the hypersurface
$t-\eta\theta=0$.
Let $Y$ be the function on $\dbR$ equal to $1$ for $x<0$
and to $0$ for $x>0$.
We have
$$
Y(t-\eta\theta)=Y(t)+\eta\theta\delta(t)
$$
and it follows that at $\eta=0$,
$$
\align
\partial_\eta\int\nolimits_\Gamma^{(0,\eta)\Gamma}\scrL
  &=\partial_\eta\int (Y(t-\eta\theta)-Y(t))\scrL\\
&=\int \theta\delta(t)\scrL.
\endalign
$$
Similarly,
$$
\partial_t\int\nolimits_\Gamma^{(t,0)\Gamma}\scrL=
\int \delta(t)\scrL.
$$
If $\scrL=dt\,d\theta^{-1}L$, as $\theta
d\theta^{-1}=-d\theta^{-1}\theta$, we get
$$
\align
&\partial_\eta\int\nolimits_\Gamma^{(0,\eta)\Gamma}
  \scrL=-L(0,0)\tag2.22\\
&\partial_t\int\nolimits_\Gamma^{(t,0)\Gamma}\scrL=
  (\partial_\theta L)(0,0)=DL(0,0).
\endalign
$$

We now apply this to our $\scrL$ (2.1).
Let us identify the space of extremals $Y$ to the space
of Cauchy data $(x,\xdot,\psi)$ at $(0,0)$.
The $1$-form $\alpha=\alpha_\Gamma$ is given by (2.11):
\ for a variation of $(x,\xdot,\psi)$, it is $(\delta
x,\xdot)-\frac12(\delta\psi,\psi)$, with $\delta\psi$ a
covariant derivative.
The vector field $[\tau]$ is given by
$$
\align
&\partial_\tau(in\,g)\text{ [Cauchy data for
$X(g(t,\theta)]$}\\
&=\mynabla_t[X(t,\theta),\delta_t X(t,\theta),
DX(t,\theta)]\quad\text{at $(0,0)$}
\endalign
$$
for $X$ the extremal with the given Cauchy data.
This gives
$$
\spreadmatrixlines{2\jot}
\matrix\format\l &\qquad\c &\quad\c &\quad\c\\
&\delta x &\delta \xdot &\delta\psi\\
[\partial_\theta] &\psi
&\mynabla_D\Xdot\vrulesub{(0,0)}=0 &\mynabla_D
DX\vrulesub{(0,0)}=-\xdot\\
[\partial_t] &\xdot &\mynabla_t\Xdot\vrulesub{(0,0)}
  &\mynabla_t DX\vrulesub{(0,0)}=0
\endmatrix
$$
and $i_{[\tau]}\alpha$ is

\medskip\noindent
for $[\partial_\theta]$: \
$(\psi,\xdot)-\frac12(-\xdot,\psi)$

\noindent
for $[\partial_t]$: \ $(\xdot,\xdot)$.

{}From (2.21) and (2.22), we get
$$
\align
Q &=(\psi,\xdot)-\tfrac12(-\xdot,\psi)-\tfrac12
  (\xdot,\psi)=(\psi,\xdot)\\
H &=(\xdot,\xdot)-D\left(-\tfrac12(\Xdot,DX)\right)
\vrulesub{(0,0)}=\tfrac12(\xdot,\xdot)
\endalign
$$
as $\mynabla_D\Xdot=\mynabla_t DX=0$ and $\mynabla_D
DX=-\xdot$.

By (2.14), (2.17) and (2.19), one has
$$
\{Q,Q\}=[D^{\plus}]Q=-H.
$$

Verification: \
$[D^{\plus}](\psi,\xdot)=(-\xdot,\xdot)$.



\enddocument










