\input amstex
\documentstyle{amsppt}
\magnification=1200
\input epsf.tex
\loadeufm

\font\dotless=cmr10 %for the roman i or j to be
                    %used with accents on top.
                    %(\dotless\char'020=i)
                    %(\dotless\char'021=j)
\font\itdotless=cmti10
\def\itumi{{\"{\itdotless\char'020}}}
\def\itumj{{\"{\itdotless\char'021}}}
\def\umi{{\"{\dotless\char'020}}}
\def\umj{{\"{\dotless\char'021}}}
\font\thinlinefont=cmr5
\font\smaller=cmr5
\font\boldtitlefont=cmb10 scaled\magstep2

\NoRunningHeads
\pagewidth{6.5 true in}
\pageheight{8.9 true in}
\loadeusm

\catcode`\@=11
\def\logo@{}
\catcode`\@=13

\def\eps{{\varepsilon}}
\def\WC{{W_{\dbC}}}
\def\XC{{X_{\dbC}}}
\def\YC{{Y_{\dbC}}}

\def\undertext#1{$\underline{\vphantom{y}\hbox{#1}}$}
\def\nspace{\lineskip=1pt\baselineskip=12pt%
     \lineskiplimit=0pt}
\def\dspace{\lineskip=2pt\baselineskip=18pt%
     \lineskiplimit=0pt}

%\def\wedgeop{\operatornamewithlimits{\wedge}\limits}
%\def\oplusop{\operatornamewithlimits{\oplus}\limits}
%\def\simover#1{\overset\sim\to#1}
\def\w{{\mathchoice{\,{\scriptstyle\wedge}\,}
  {{\scriptstyle\wedge}}
  {{\scriptscriptstyle\wedge}}{{\scriptscriptstyle\wedge}}}}
\def\Le{{\mathchoice{\,{\scriptstyle\le}\,}
{\,{\scriptstyle\le}\,}
{\,{\scriptscriptstyle\le}\,}{\,{\scriptscriptstyle\le}\,}}}
\def\Ge{{\mathchoice{\,{\scriptstyle\ge}\,}
{\,{\scriptstyle\ge}\,}
{\,{\scriptscriptstyle\ge}\,}{\,{\scriptscriptstyle\ge}\,}}}
%\def\doubleheaddownarrow#1{\hbox{$\Big\downarrow%
%     \kern-5.96pt\lower2pt\hbox{$\downarrow$}$}%
%     \rlap{$\vcenter{$\kern-18pt\scriptstyle#1$}$}}
%\def\rmapdown#1{\Big\downarrow\kern-1.0pt\vcenter{
%     \hbox{$\scriptstyle#1$}}}

\def\red{\text{\rm red}}
\def\Ber{\text{\rm Ber}}
%\def\Gal{\text{\rm Gal}}  \def\sign{\text{\rm sign}}
%\def\Br{\text{\rm Br}}  \def\im{\text{\rm Im}}
%\def\sBr{\text{\rm sBr}} \def\Spin{\text{\rm Spin}}
%\def\sgn{\text{\rm sgn}} \def\pr{\text{\rm pr}}
%\def\Id{\text{\rm Id}} \def\Sp{\text{\rm Sp}}
%\def\Gr{\text{Gr}} \def\SO{\text{\rm SO}}
%\def\Sym{\text{\rm Sym}} \def\SL{\text{\rm SL}}
%\def\End{\text{\rm End}}
\def\Diff{\text{\rm Diff}}
\def\HOM{\underline{\text{\rm Hom}}}
\def\Hom{\text{\rm Hom}}
%\def\Ker{\text{\rm Ker}}
%\def\Tr{\text{\rm Tr}}
%\def\Lie{\text{\rm Lie}} 

\def\fbar{\bar{f}}
%\def\kbar{\bar{k}}
\def\zbar{\bar{z}}
%\def\Abar{\bar{A}}
%\def\Sbar{\bar{S}}
\def\Xbar{\bar{X}}
\def\Ybar{\bar{Y}}

\def\dbC{{\Bbb C}}
%\def\dbG{{\Bbb G}}
%\def\dbH{{\Bbb H}}
\def\dbR{{\Bbb R}}
%\def\dbZ{{\Bbb Z}}


\def\scr#1{{\fam\eusmfam\relax#1}}

%\def\scrD{{\scr D}}
%\def\scrL{{\scr L}}
\def\scrO{{\scr O}}   
\def\scrS{{\scr S}}   
%\def\scrU{{\scr U}}   

%\def\gr#1{{\fam\eufmfam\relax#1}}

%Euler Fraktur letters (German)
%\def\gro{{\gr o}}
%\def\grs{{\gr s}}

\def\grso{{\grs\gro}}


\topmatter
\title\nofrills
{\boldtitlefont Real versus complex}
\endtitle
\author
P. Deligne
\endauthor
\endtopmatter

%\bigskip
%\centerline{School of Mathematics}
%\centerline{Institute for Advanced Study}
%\centerline{Princeton, NJ \ 08540}

%\smallskip
%\centerline{e-mail: deligne\@math.ias.edu}

%\smallskip
%\centerline{(corrected October 11, 1996)}

\NoBlackBoxes
\parindent=20pt
\frenchspacing
\document
\dspace
\bigskip
We want to explain how to go back and forth between two
modes of writing:

\subhead\nofrills
{\bf real:}\enspace
\endsubhead
one considers real varieties, functions on them (possibly
complex valued: $F=f+ig$), maps between them, $\ldots$

\subhead\nofrills
{\bf complex:}\enspace
\endsubhead
one considers complex varieties, and {\it only
holomorphic functions} on them, complex analytic maps
between them, $\ldots$ 

A real analytic variety $X$ can be complexified to a
complex variety $\XC$.
In fact, only the germ of $\XC$ around $X$ is
defined.
One can think of $X$ as being given by $\XC$, plus
some auxiliary data.
For some purposes, this auxiliary data is not needed, and
computations are better done in ``complex mode'', on
$\XC$.

To recover $X$ from $\XC$, one could give

\medskip\noindent
(A)\enspace
The antiholomorphic involution ``complex conjugation'' on
$\XC$, with fixed point $X\subset \XC$ --- or

\medskip\noindent
(B)\enspace
the data of which (holomorphic) functions are real (on
$X$).

\medskip
There is not much difference between complex valued
functions on $X$, and (holomorphic) functions on
$\XC$, and similarly for forms.
A local coordinate system on $X$ gives a (holomorphic)
local coordinate system on $\XC$.

If $\alpha$ is a form of maximal degree on $X$, viewed as
an (holomorphic) form on $\XC$, integration of
$\alpha$ becomes, on $\XC$, integration on a cycle,
and we need to know this cycle only up to suitable
homologies.

\example{Example}
To compute $\int e^{-x^2/2+iax}dx$, it is better to go
to the holomorphic picture, where the integration cycle
$\dbR\subset\dbC$ is needed up to homologies given by
$2$-chains on which the integrand decays fast.
In this setting, the completing the square change of
variables $y=(x-ia)$ makes good sense, giving the answer
$\sqrt{2\pi\,\,}e^{-a^2/2}$.
\endexample

Similarly, a $(0\vert n)$ dimensional real variety$X$ can
be complexified.
Here, (holomorphic) functions on $\XC$ are the same
as complex valued functions on $X$.
The real $X$ is recovered from $\XC$ by giving
$(A')$ an antilinear involutuion $\sigma$ of $\XC$:\,\,
$\overline{\XC}\to \XC$ or $(B')$ which
functions are real.
Here, to integrate a density is a purely algebraic story:
only the density on $\XC$ matter, the real form $X$
is irrelevant.

That integration is a purely algebraic story, making the
real structure irrelevant, can happen also for ordinary
varieties.
Example: \ let $V$ be a real vector space, $Q$ a
quadratic form, $P$ a polynomial and $dv$ a volume form.
Then, up to an annoying square root, $\int e^{-Q}P.dv$
depends only on $e^{-Q}P dv$ on the complexification
$V_{\dbC}$.

In dimension $(n\vert m)$, Bernstein introduced the
notion of $cs$-spaces.
I like to think of a $cs$ space as being a complex super
space $\XC$, plus the data of an ``integration''
cycle $C$ on the reduced space.
In many local computations only $\XC$ matters.
If one wants to integrate densities (holomorphic sections
of $\Ber(\Omega^1)$), the cycle $C$ matters only up to
suitable homologies.
Sometimes one wants to germify along $C$ (which is the
real locus for a real structure on the reduced space).
This gives Bernstein definition, where the $cs$ space is
$C$, with the sheaf of rings $\scrO_{\XC}$ restricted to
$C$.

If $X$ has complex structure $J$, $\XC$ is
$X\times\Xbar$, with $X$ sitting in $X\times\Xbar$ as the
diagonal.
As before, this is not much difference between a (complex
valued) function on $X$ and an (holomorphic) function on
$\XC=X\times\Xbar$, defined near the diagonal.
Holomorphic functions on $X$ become functions $f(x,y)$ on
$\XC=X\times\Xbar$ depending only on $x$.
In local coordinates: \ an holomorphic coordinate system
on $X$, $(z_i)$, gives one on $\XC=X\times\Xbar$: \
$(z_i,\zbar_i)$.
A function on $X$ is then viewed as an (holomorphic)
function of $z_i$ and $\zbar_i$.
It is an holomorphic function on $X$ if it is an
(holomorphic) function not depending on the $\zbar_i$.

For functions spaces also it can be convenient to work in
complexified spaces, with reality conditions coming as an
afterthought, if needed.
The complexification of $\HOM(X,Y)$ is the space
$\HOM(\XC,\YC)$ of (holomorphic) maps $f$ from $\XC$ to
$\YC$.
One requires $f$ to be defined only in a suitable
neighborhood of $X\subset\XC$, and that $f(X)$ is close
to $Y\subset\YC$.

\example{Example} \ 
(i)\enspace If $v$ is a section of the complexified
tangent bundle of $X$, with $X$ compact and $X$ and $v$
real analytic, $\exp(tv)$ ($t$ small) belongs to
$\Diff(X)_{\dbC}$.

\smallskip\noindent
(ii)\enspace
(After G. Segal),\enspace
 Let $S^1$ be the unit circle in $\dbC$.
A point of $\Diff(S^1)_{\dbC}$ is an holomorphic map from
an annulus $1-\eta<r<1+\eta$ to an annulus
$1-\eps<r<1+\eps$.
It should have a restriction to $S^1$ close to a
diffeomorphism of $S^1$.

Composition is only partially defined.
If $f$ maps $S^1$ inside $S^1$, one can attach to $f$ the
``annulus'' $[f]$ between $f(S^1)$ and $S^1$ in $\dbC$.
Both boundary components of $[f]$ are parametrized by
$S^1$, using the identity, resp. $f$.
To composition corresponds gluing of annuli, and this is
how a good substitute to $\Diff(S^1)_{\dbC}$ is the monoid
of isomorphism classes of annuli with parametrized
boundaries.
In conformal field theory, one meets complex
(projective) representations of real semi-group underlying 
the complex semi-group $\Diff(S^1)_{\dbC}$.
At the Lie algebra level, if $W$ is the Lie algebra of
vector fields on $S^1$, with complexification $\WC$, a
complex (projective) representation of the real Lie
algebra $r(\WC)$ underlying $\WC$ is the same as a
complex (projective) representation of
$r(\WC)\otimes\dbC=\WC\otimes_{\dbR}\dbC=\WC\times\WC$.


\smallskip\noindent
(iii)\enspace
Take $X=\dbR^{0\vert 1}$, hence $\XC=\dbC^{0\vert 1}$.
Here $\Hom(X,Y)=Y$, with the sheaf of ring $\Omega^*$,
while $\Hom(\XC,\YC)$, taken in the holomorphic world is
$\YC$, with the holomorphic $\Omega^*$, and $(Y,\Omega^*)_{\dbC}=
(\YC,\Omega^*)$.
\endexample

In the complex setting, if $X$ is of dimension $(0\vert
1)$, and $Y$ of dimension $(n\vert 0)$, a real structure
on $Y$ induces a $cs$ structure on $\HOM(X,Y)$.
Indeed, $\HOM(X,Y)$ has a reduced space $Y$.

The same is not true for $X$ of dimension $(0,2)$.
Indeed, take $Y=\dbC$ and coordinates $\theta_1$, 
$\theta_2$ on $X$.
A map to $Y$, i.e. an (holomorphic) function, is written
$$
a+\alpha_1\theta_1+\alpha_2\theta_2+b\theta_1\theta_2\,\,:
$$
the space $\HOM(X,Y)$ is a $\dbC^{2\vert 2}$ with
coordinates $(a,b,\alpha_1,\alpha_2)$.
If we give a real structure on $Y$, we know what it means
for $a$ to be real.
To know what it means for $b$ to be real, we need in
addition to give a real structure on $X$.

In practice, this will not matter, for two reasons:

\smallskip\noindent
(A)\enspace
The spaces of externals $\scrS\subset\Hom(X,Y)$ we will
consider will be such that $\scrS_{\red}$ embeds into
$\Hom(X_{\red},Y)$ (and is a space of extremal there), so
that a real structure on $Y$ gives one on $\scrS_{\red}$:
\ we get a $cs$ structure on $\scrS$.

\smallskip\noindent
(B)\enspace
In path integrals, integrals on fields playing the role
of ``$c$'' above can be handled algebraically.

\smallskip
In the real setting, if we are to consider $\Hom(X,Y)$
with $Y$ having a complex structure, it is usually
convenient to work on $\Hom(X,Y)_{\dbC}=\Hom(\XC,\YC)$.

The fact that $Y$ was complex gives us two transversal
foliations on $\YC=Y\times\Ybar$.
In local oordinates: \ for $(z_i)$ local coordinates on
$Y$, we get local coordinates $z_i$, $\zbar_i$ on $\YC$.
It follows that $f\in\Hom(\XC,\YC)$ is given by the
$f_i=z_i(f)$ and $\fbar_i=\zbar_i(f)$, where the $f_i$
and $\fbar_i$ are (independent) holomorphic functions on
$\XC$ (or complex valued functions on $X$).
If on $\XC$ we have a real structure $X$, we get the real
structure $\Hom(X,Y)$ on $\Hom(\XC,\YC)$ by telling that
$f$ is real in $f_i$ and $\fbar_i$ are complex conjugate
(on $X$).








\enddocument



