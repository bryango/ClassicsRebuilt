%Date: Tue, 24 Dec 1996 12:55:34 -0500 (EST)
%From: Pavel Etingof <etingof@abel.math.harvard.edu>

\input amstex
\documentstyle{amsppt}
\magnification 1200
\NoRunningHeads
\NoBlackBoxes
\document

\def\tW{\tilde W}
\def\Aut{\text{Aut}}
\def\tr{{\text{tr}}}
\def\ell{{\text{ell}}}
\def\Ad{\text{Ad}}
\def\u{\bold u}
\def\m{\frak m}
\def\O{\Cal O}
\def\tA{\tilde A}
\def\qdet{\text{qdet}}
\def\k{\kappa}
\def\RR{\Bbb R}
\def\be{\bold e}
\def\bR{\overline{R}}
\def\tR{\tilde{\Cal R}}
\def\hY{\hat Y}
\def\tDY{\widetilde{DY}(\g)}
\def\R{\Bbb R}
\def\h1{\hat{\bold 1}}
\def\hV{\hat V}
\def\deg{\text{deg}}
\def\hz{\hat \z}
\def\hV{\hat V}
\def\Uz{U_h(\g_\z)}
\def\Uzi{U_h(\g_{\z,\infty})}
\def\Uhz{U_h(\g_{\hz_i})}
\def\Uhzi{U_h(\g_{\hz_i,\infty})}
\def\tUz{U_h(\tg_\z)}
\def\tUzi{U_h(\tg_{\z,\infty})}
\def\tUhz{U_h(\tg_{\hz_i})}
\def\tUhzi{U_h(\tg_{\hz_i,\infty})}
\def\hUz{U_h(\hg_\z)}
\def\hUzi{U_h(\hg_{\z,\infty})}
\def\Uoz{U_h(\g^0_\z)}
\def\Uozi{U_h(\g^0_{\z,\infty})}
\def\Uohz{U_h(\g^0_{\hz_i})}
\def\Uohzi{U_h(\g^0_{\hz_i,\infty})}
\def\tUoz{U_h(\tg^0_\z)}
\def\tUozi{U_h(\tg^0_{\z,\infty})}
\def\tUohz{U_h(\tg^0_{\hz_i})}
\def\tUohzi{U_h(\tg^0_{\hz_i,\infty})}
\def\hUoz{U_h(\hg^0_\z)}
\def\hUozi{U_h(\hg^0_{\z,\infty})}
\def\hg{\hat\g}
\def\tg{\tilde\g}
\def\Ind{\text{Ind}}
\def\pF{F^{\prime}}
\def\hR{\hat R}
\def\tF{\tilde F}
\def\tg{\tilde \g}
\def\tG{\tilde G}
\def\hF{\hat F}
\def\bg{\overline{\g}}
\def\bG{\overline{G}}
\def\Spec{\text{Spec}}
\def\tlo{\hat\otimes}
\def\hgr{\hat Gr}
\def\tio{\tilde\otimes}
\def\ho{\hat\otimes}
\def\ad{\text{ad}}
\def\Hom{\text{Hom}}
\def\hh{\hat\h}
\def\a{\frak a}
\def\t{\hat t}
\def\Ua{U_q(\tilde\g)}
\def\U2{{\Ua}_2}
\def\g{\frak g}
\def\n{\frak n}
\def\hh{\frak h}
\def\sltwo{\frak s\frak l _2 }
\def\Z{\Bbb Z}
\def\C{\Bbb C}
\def\d{\partial}
\def\i{\text{i}}
\def\ghat{\hat\frak g}
\def\gtwisted{\hat{\frak g}_{\gamma}}
\def\gtilde{\tilde{\frak g}_{\gamma}}
\def\Tr{\text{\rm Tr}}
\def\l{\lambda}
\def\I{I_{\l,\nu,-g}(V)}
\def\z{\bold z}
\def\Id{\text{Id}}
\def\<{\langle}
\def\>{\rangle}
\def\o{\otimes}
\def\e{\varepsilon}
\def\RE{\text{Re}}
\def\Ug{U_q({\frak g})}
\def\Id{\text{Id}}
\def\End{\text{End}}
\def\gg{\tilde\g}
\def\b{\frak b}
\def\S{\Cal S}
\def\L{\Lambda}

\topmatter
\title Lecture 1: Renormalization groups 
\endtitle
\author {\rm {\bf David Gross} }\endauthor
\endtopmatter

\centerline{notes by Pavel Etingof and David Kazhdan}

{\bf 1.1. What is renormalization group?}

In these lectures we will consider the method of renormalization 
group and its applications to quantum field theory. 

Very generally speaking, the method of renormalization group is 
a method designed to describe how the dynamics of some system changes
when we change the scale (distances, energies) at which we probe it. 

Before describing more precisely what exactly this method does, recall 
two basic facts about physics:

{\it 1. Scale dependence.} Physics is scale-dependent.
For example, consider a fluid. At each scale of
distances, we need a different theory to describe its behavior:

At $\sim 1$cm -- classical continuum mechanics (Navier-Stokes equations)

At $\sim 10^{-5}$cm -- theory of granular structure

At $\sim 10^{-8}$cm -- theory of atom (nucleus + electronic cloud)

At $\sim 10^{-13}$cm -- nuclear physics (nucleons)

At $\sim 10^{-13}-10^{-18}$cm quantum chromodynamics (quarks)

At $\sim 10^{-33}$ string theory

At each scale, we have different degrees of freedom and different 
dynamics.

{\it 2. Decoupling.} Physics at a larger scale (largely) decouples
from the physics at a smaller scale. 
For example, to describe the behavior of fluid at the scale $\sim 1$cm,
we don't need to know about the granular structure, nor about atoms 
or nucleons. The only things we need to know is the viscosity 
and the density of the fluid. Of course, these values can be computed 
from the physics at a smaller scale, but if we found them out in some 
way (for example, measurement), we can do without smaller scale theories at
all. Similarly, if we want to describe atoms, we don't need to know
anything about the nucleus except its mass and electric charge. 

Thus, a theory at a larger scale remembers only finitely many parameters
from the theories at smaller scales, and throws the rest 
of the details away. More precisely, when we pass from a smaller scale
to a larger scale, we average over irrelevant degrees of freedom.
Mathematically this means that they become integration variables and
thus disappear in the answer. 

This decoupling is the reason why we are able to do physics. If 
there was no decoupling, it would be necessary for Newton to 
know string theory to describe the motion of a viscous fluid. 

{\bf Remark.} This transition of information 
from smaller scale theories 
to larger scale theories through finitely many parameters 
is of course irreversible, since in such transition we forget 
(=integrate out) 
many irrelevant degrees of freedom. For example, it is possible 
to reconstruct the thermodynamics of gas from molecular theory
(by averaging over all configurations), but it is impossible to
reconstruct the behavior of molecules from the macroscopic behavior
of gas itself. 

The general aim of renormalization group method is to explain how 
this decoupling takes place and why exactly information 
is transmitted from scale to scale through finitely many parameters.

{\bf Remark 1.} In quantum theory, decoupling of scales is not at all
obvious. Indeed, because of the uncertainty principle, we have to work
at all scales at a time. Renormalization group explains why 
decoupling survives in quantum theory. 

{\bf Remark 2.} In classical mechanics, there are 3 basic units 
of measurement (distance $D$, time $T$, mass $M$), and all others can be 
expressed through them. Thus, in classical mechanics
we deal with three scales. In nonrelativistic quantum theory
and in classical relativity there remains only two of them, as in the first
case we can express $M$ through $T$ and $D$ using the Planck constant, and
in the second $T$ can be expressed via $D$ using the speed of light. 
Thus, in relativistic quantum theory we only have one 
scale -- the scale of distances. 
Equivalently we can use the inverse scale -- the scale 
of momenta. Thus we have:

SMALL distances, times = LARGE momenta, energies, masses.

{\bf 1.2. The general scheme of the method
of renormalization group.}

As we already mentioned,
in quantum physics in general, in particular in any quantum field theory,
we must consider arbitrarily short scales. Thus, in order to understand
the theory at a given scale, we should know something about its behavior
at smaller scales. The theory is called renormalizable if all the
information we need from smaller scales can be absorbed in finitely many
parameters (renormalized masses and couplings). In this case, we can
measure these parameters (or some functions of them), and thus have a 
complete picture of the theory. However, if the theory is not
renormalizable, the number of parameters coming from smaller scales 
is infinite, so the theory is unmanageable. This is why we can 
successfully work only with renormalizable theories. 

Now we make a general review of renormalization group theory. 
To think about renormalization group, one can use 
the following general picture, due to Wilson.

The central object in Wilson's picture is ``the space of theories'' $S$.
In general, this object is extremely ill-defined. However, in special cases
it may be possible to give a precise definition of $S$.
In quantum field theory 
we can model this space as the space of all possible Lagrangians  
of certain types of fields. 

Wilson's procedure is the following. Let $M$ be the set of 
scales of momenta. Mathematically, $M$ is a principal 
homogeneous space (torsor) of $\R_+$.
(This means, $M$ is isomorphic to $\R_+$, but noncanonically; 
in particular, there is no distinguished element in $M$).
Let $\Lambda,\mu\in M$ be scales of momenta,
and $\mu<\Lambda$. 
For each theory $\Cal L\in S$, one finds another theory, 
$R_{\Lambda\mu}\Cal L\in S$ which is the 
effective theory, 
at the scale $\mu$, for the original theory $\Cal L$ at 
the scale $\Lambda$. Thus, we have a map $R_{\Lambda\mu}:S\to S$,
and $R_{\Lambda_1\Lambda_2}R_{\Lambda_2\Lambda_3}=R_{\Lambda_1\Lambda_3}$. Thus, we have an action 
of the semigroup $\Bbb R_+^{\le 1}$ on the space $S\times M$,
by $\l\circ (\Cal L,\Lambda)=
(R_{\Lambda,\l \Lambda}\Cal L,\l \Lambda)$. Abusing mathematical
terminology, physicists call this semigroup 
the renormalization group. 

{\bf 1.3. Wilsonian scheme for the theory of a scalar field:
a mathematical description} (an interpretation of D.Gross' lectures
by the preparers of these lecture notes). 

Let us give a precise mathematical description of renormalization group
in the special case of theories of a scalar bosonic field in 4 dimensions. 

Let $V$ be a 4-dimensional Euclidean space, 
and $\Cal S(V)$ the space of Schwarz functions 
on $V$. By a homogeneous interaction of degree $k$ we mean 
a continuous, homogeneous,
Poincare invariant polynomial functional $J: \Cal S(V)\to\C$.
That is, 
$$
J(\phi)=\int K_J(x_1,...,x_k)\phi(x_1)...\phi(x_k)dx_1...dx_k,
$$
where $K_J$ is a Poincare invariant tempered distribution. 

\proclaim{Definition} (i) A homogeneous interaction $J$ 
is called
local if $K_J$ is supported on the diagonal. 
 
(ii) $J$ is called quasilocal if
$K_J$ has rapid decay at infinity, i.e. if  
the Fourier transform $\tilde K_J$ of $K_J$ has the form
$$
\tilde K_J(p_1,...,p_k)=\delta(p_1+...+p_k)\tilde F_J(p_1,...,p_{k-1}),
$$
where $\tilde F_J$ is a smooth function.

(iii) We say that a quasilocal interaction $J$ is 
of order $\ge s$ if the function $\tilde F_J$ has order 
$\ge s$ at the origin. We will write $O(\nabla^s)$
for a quasilocal interaction of order
$\ge s$.     
\endproclaim

It is clear a local interaction is the same thing as a differential 
polynomial in $\phi$, and the order of a differential 
monomial is the number of derivatives in it. 
This statement can be generalized to
the case of quasilocal interactions, as follows. 

\proclaim{Proposition 1} Let $J$ be a quasilocal interaction
of degree $k\ge 1$. Then for any $j\ge 0$ there exists a unique 
homogeneous differential operator $D_j$ of order $j$ on $V^{k-1}$
with constant coefficients, 
such that for any $\phi\in \Cal S(V)$, and any
$s\ge 0$ the functional $J(\phi)$ is given by
a Taylor formula
$$
J(\phi)=\sum_{j=0}^{s-1}\int_V
D_j[\phi(x_1)...\phi(x_{k-1})]|_{x_1=...=x_{k-1}=x}\phi(x)dx+
J^{s}(\phi),\ J^{s}(\phi)=
O(\nabla^s).
$$
\endproclaim

\demo{Proof} We have 
$$
J(\phi)=\int_{(V^*)^(k-1)}\tilde F_J(p_1,...,p_{k-1})
\tilde\phi_1(p_1)...\tilde \phi(p_{k-1})\tilde\phi_k(-p_1-...-p_{k-1})
dp_1...dp_{k-1},
$$
where $\tilde \phi$ is the Fourier transform of $\phi$. 
Since $\tilde F_J$ is smooth, it has a unique  Taylor expansion 
$$
\tilde F_J(p_1,...,p_{k-1})=\sum_{j=0}^{s-1}D_j(p_1,...,p_{k-1})
+O(|p|^s).
$$
This implies the proposition. 
\enddemo

Now let $\tau$ be a formal variable. 
We will work over the ring $\R[[\tau]]$. 

Let $X$ be the space of all Lagrangians 
of $\phi$ with coefficients in $\R[[\tau]]$, 
invariant under $\phi\to -\phi$, of the form
$$
\Cal L=
\int (\frac{1}{2}(a(\nabla\phi)^2+m^2\phi^2)+I(\phi))d^4x,\tag 1.2
$$
where 
$I(\phi)=\sum_{k\ge 2} I_k(\phi)$ 
are interaction terms ($I_k$ has degree $k$), such that

(i) $a,m^2\in \R_+$ and $I_k=O(\tau)$ for all $k$;

(ii) For any $N\ge 0$ $I_k=O(\tau^N)$ for almost all $k$;

(iii) $I_k$ are quasilocal. 

Consider the space $S=X\times M$.  
The space $S$ is the relevant space of theories in which Wilson's 
flow will be defined. 

Let $F_k$ be the space of smooth functions on 
the hyperplane $\sum_{i=1}^k v_i=0$ in $(V^*)^{\oplus k}$
(we identify this hyperplane with $(V^*)^{(k-1)}$
by throwing out the last component $v_k$).
Let $F=\prod_{i=1}^\infty F_{2i}$. 

In the next few paragraphs we will define a map ${\Cal G}: S\to F$,
${\Cal G}=({\Cal G}_2,{\Cal G}_4,...)$.
The value of ${\Cal G}_k$ at $(\Cal L,\Lambda)$ is called
the $k$-point correlation function of the Lagrangian $\Cal L$ 
at scale (cutoff) $\L$.  

Let $P(r)$ be a smooth function on $\R_+$ that equals 
$1$ for $r\le 1/2$, $0$ for $r>2$, and varies between $0$ and $1$ in
between. Define the cutoff propagator (in momentum space) to be
$\tilde G_\L(p)=\frac{P(p^2/\L^2)}{p^2+m^2}$. 

Let $(\Cal L,\L)\in S$. 
By definition, the algorithm of computing ${\Cal G}_k(\Cal L,\L)$ 
is as follows. ${\Cal G}_k$ is the sum of amplitudes of all Feynman 
diagrams with $k$ external edges. Computation of the amplitude
is as usual (by Feynman rules), except that on each internal edge
instead of the free propagator $1/(p^2+m^2)$ one should put the cutoff 
propagator $\tilde G_\L(p)$. On external edges one should put the free 
propagator.  

One can also define connected and 1-particle irreducible 
correlation functions, which are obtained by summing 
the amplitudes of connected and 1-particle irreducible 
diagrams, respectively (see Kazhdan's lectures). 

For example, if $\Lambda=0$, 
the cutoff propagator vanishes, so that 
connected correlation functions are 
the same as interactions. 

Now let $\mu,\L$ be arbitrary scales.
Then one can make the following definition.

\proclaim{Definition} A Lagrangian $\Cal L_e$ of the form (1.2) 
is called an effective Lagrangian at the scale $\mu$ of
a Lagrangian $\Cal L$ at the scale $\L$ if 
${\Cal G}(\Cal L_e,\mu)={\Cal G}(\Cal L,\Lambda)$.
\endproclaim

\proclaim{Proposition 2} For any $\Cal L\in X$ 
and any $\mu,\Lambda$ there exists a unique 
effective Lagrangian
$\Cal L_e$ 
at the scale $\mu$ of $\Cal L$ at the scale $\L$. 
\endproclaim

\demo{Proof}
It is easy to check that a Lagrangian is uniquely 
determined by its correlation functions. 
This implies the uniqueness of $\Cal L_e$, if it exists. 
The existence of $\Cal L_e$ is proved by giving an explicit 
construction. The construction: the function $\tilde F_{J_k}$
for $\Cal L_e$ is defined to be the sum of amplitudes of all 
connected Feynman diagrams with $k$ external edges, computed 
for $\Cal L$ with
the propagator $\tilde G_{\Lambda\mu}(p):=\tilde G_\L(p)-\tilde G_\mu(p)$.
It is not difficult to check 
that $\Cal L_e$ is an effective Lagrangian at $\mu$
for $\Cal L$ at $\L$. 
\enddemo

We will denote the Lagrangian $\Cal L_e$ by $R_{\L\mu}\Cal L$.
This defines a map $R_{\L\mu}: X\to X$ for any two scales 
$\L,\mu$. It is clear that $R_{\L_1\L_2}R_{\L_2\L_3}=R_{\L_1\L_3}$,
and $R_{\L\L}=1$.

Thus, we have a flow on the space
$S$ given by the formula
$t(\Cal L,\L)=(R_{\L,t\L}\Cal L,t\L)$, 
$t\in \R^+$. This flow is called the renormalization group flow. 

{\bf Remark 1.} It is easy to see that if a Lagrangian $\Cal L$ 
is free (i.e. $I_k=0$) then $R_{\L\mu}\Cal L=\Cal L$, since in this case
we have no internal vertices and no nontrivial diagrams. 

{\bf Remark 2.} In principle, instead of the described ``smeared cutoff'', 
we could use the
most obvious cutoff,
i.e. define a cutoff propagator $\tilde G^*_\L(p):=\frac{1_\L}{p^2+m^2}$,
where $1_\L$ is
the characteristic function of the ball $|p|<\L$.
In the language of functional integral, this means the following:
the $k$-point correlation function (in position space) 
is obtained by integration of $\phi(x_1)...\phi(x_n)$ 
not over all fields $\phi(x)$, but only 
over fields whose Fourier tranform is supported in the region $|p|<\L$. 
Similarly, the effective Lagrangian $\Cal L_e$ 
at $\mu$ for $\Cal L$ at $\L$ is given (for $\mu<\L$) by the formula
$$
e^{-\Cal L_e(\phi)}=\int e^{-\Cal L(\phi+\eta)}D\eta \tag 1.3
$$
where $\phi$ is a field whose Fourier transform is supported in the ball 
$|p|\le \mu$, and the functional integral is taken over all 
fields $\eta$ whose Fourier transform is supported in $\mu\le |p|\le \L$.

This picture has a nice physical interpretation: 
in order to compute $R_{\Lambda\mu}$, 
we integrate out degrees of freedom -- the amplitudes of harmonics with
frequencies between $\Lambda$ and $\mu$. 

{\bf Remark 3.} 
The map $R_{\L\mu}$, although 
it is formally defined (by Feynman rules) for any $\L,\mu$, makes physical
sense only for $\L>\mu$. This is well illustrated by
Remark 2, as there $R_{\L\mu}$ by the definition reduces to 
integrating out degrees of freedom.
In fact, beyond perturbation theory the map
$R_{\L\mu}$ is usually not defined for $\L<\mu$. 

{\bf Remark 4.} Unfortunately, the sharp cutoff $\tilde G^*_\L$
defined in Remark 2 is not quite satisfactory, since 
the corresponding renormalization group flow on the space
of Lagrangians does not preserve quasilocality
(quasilocality is necessary for formulation of Theorem 4). 
Indeed, let us consider 
a Lagrangian
$$
\Cal L=\int
(\frac{1}{2}(a(\nabla\phi)^2+m^2\phi^2)+\frac{g}{4!}\phi^4)d^4x,
\tag 1.4
$$
where $g\sim \tau$. 
It is easy to see that in the order $\tau^2$ we will get a correction
to the quadratic term of the form
$$
\int \phi(x)\phi(y)(G^*_{\L}(x-y)-G^*_{\mu}(x-y))dxdy,
$$
where $G^*_{\L}(x)$ is the cutoff propagator in position space. 
Since the cutoff propagator in momentum space is not smooth, 
this term is not quasilocal. 

This is why it is better to use the smooth cutoff $\tilde G_\L$
instead of the sharp cutoff $\tilde G^*_\L$.
 
Now we will describe the main property of the renormalization group.  
Very roughly, this property is that 
the operator $R_{\L\mu}$ 
for small $\mu/\L$ makes all non-renormalizable terms in the Lagrangian
irrelevant, so in the limit $\L\to \infty$ only renormalizable terms matter.
If there is only finitely many renormalizable terms (which is usually 
the case), this means that the renormalization group flow
has a finite dimensional attractor in the space of theories -- 
the space of renormalizable, or critical theories.  
This property is the reason why we actually never have to consider
non-renormalizable theories in the perturbative setting. 

Let us describe this property more carefully.
First recall that in the theory of a scalar field, the only
renormalizable Lagrangians are of the form (1.4).
Let us denote the set of all renormalizable 
Lagrangians by $X_r$.

Let $T_0(X)$ be the tangent space to $X$ at any point
(all of them are naturally isomorphic). 

For any $\Cal L\in T_0(X)$, denote by $P_s(\Cal L)$ the part of the Lagrangian
$\Cal L$ whose coeffients have scaling dimension $\le -s$. 
That is, if $\Cal L(\phi)=\sum_k J_{2k}(\phi)$,
then by definition $P_s(\Cal L)(\phi)=\sum J_{2k}^{s-2k+4}(\phi)$,
where $J^l$ for $l>0$ was defined in Proposition 1, and
by convention for nonpositive $l$ $J^{l}(\phi)=J(\phi)$.

We will call a Lagrangian $\Cal L\in T_0(X)$ 
purely non-renormalizable if $\Cal L=P_1(\Cal L)$. 
It is clear that $\Cal L$ is purely nonrenormalizable if and only if
$\tilde F_{J_2}(0)=0$,
$\frac{\d \tilde F_{J_2}}{\d p^2}|_{p=0}=0$,
and $\tilde F_{J_4}(0,0,0)=0$. 
Denote by $X_n$ the set of all purely non-remormalizable 
Lagrangians. 

It is obvious that 
any Lagrangian can be written uniquely as a sum
$\Cal L=\Cal L_r+\Cal L_n$, where 
$\Cal L_r\in X_r$, $\Cal L_n\in X_n$. Thus, $X=X_r\times X_n$. 


Now fix an element $\Cal L\in X_r$, and
consider the intersection $(\Cal L+X_n)\cap R_{\L\mu} X_r\subset X$.

\proclaim{Lemma 3} The intersection 
$(\Cal L+X_n)\cap R_{\L\mu} X_r$ consists of 1 point. 
\endproclaim

\demo{Proof} The Lemma is obvious modulo $\tau$, 
and in higher orders in $\tau$ it is established by
induction (the implicit function theorem).
\enddemo

Let $B_{\L\mu}(\Cal L)$ be the unique point in $(\Cal L+X_n)\cap 
R_{\L\mu} X_r$.

For any $\nu \in \R^+$,
define the linear rescaling map $Q_\nu: T_0(X)\to T_0(X)$ 
by $Q_\nu(\Cal L)(\phi(x))=\Cal L(\nu^{-1}\phi(\nu^{-1} x))$. 
For example, $Q_\nu(\int \phi^6(x)d^4x)=\nu^{-2}\int \phi^6(x)d^4x$. 

The following theorem represents the main properties of
the renormalization group. 

\proclaim{Theorem 4} 

(i) (Convergence) There exists a limit $B_\mu(\Cal L)=\lim_{\L\to\infty}
B_{\L\mu}(\Cal L)$. That is, all interactions converge 
(in the weak sense). 

(ii) (Rate of convergence) 
The rate of this convergence is determined by the formula
$$
P_s(B_{\L\mu}(\Cal L))-P_s(B_\mu(\Cal L))=O(\L^{-s+\e}), \L\to\infty.
$$
for any $s\ge 1$ and $\e>0$. 

(iii) (Stability)
Let $Y_r(\mu)=B_\mu(X_r)$. Let $\Delta\Cal L\in X_n$,  
Then there set $(\Cal L+X_n)\cap R_{\L\mu}(X_r+Q_{\L/\mu}(\Delta\Cal L))$ consists 
of only one point, $B_{\L\mu}^{\Delta\Cal L}(\Cal L)$. 
Moreover, there  
exists a limit $\lim_{\L\to\infty} B_{\Lambda\mu}^{\Delta
L}(\Cal L)=B_\mu^{\Delta\Cal L}(\Cal L)$, which belongs to $Y_r(\mu)$. 

(iv) (Set convergence) Let $X(s)\subset X$ be the set of 
Lagrangians which are polynomials of degree $\le s$ in $\phi$. 
Then for $s\ge 4$ $Y_r(\mu)=\lim_{\L\to \infty}R_{\L\mu}(X_s)$. 
\endproclaim

The proof of this theorem will be partially discussed later.  
It is essentially contained in the paper by 
J. Polchinski ``Renormalization and effective Lagrangians'',
Nucl. Phys. B231(1984) 269-295.

Theorem 4 shows
that in our case the renormalization group flow has an attractor --
a three-dimensional family of trajectories $Y_r$, which can be parametrized
by the values of critical and subcritical coefficients in the Lagrangian.
Thus, the only things we 
(effectively) inherit from smaller scales 
are the values of these 3 parameters.  

We see that the space $Y_r(\mu)$ is defined
completely canonically, and $R_{\mu_1\mu_2}Y_r(\mu_1)=
Y_r(\mu_2)$. The space $Y_r(\mu)$ has the meaning of the space of
exact (renormalized) quantum field theories, i.e. those which are obtained 
by perturbative renormalization as described in Witten's lectures.
These are the only theories that we will care about. 
 
This very simple example demonstrates the main principle of the method
of renormalization group. 


{\bf 1.4. Applications of renormalization theory to phase transitions.}

As we mentioned, one of the main applications of renormalization group
in quantum field theory is that it allows to absorb all short-scale
information in finitely many constants. However, there are other
applications as well. One of them is to the theory of phase transitions
in many-body systems (statistical mechanics). 
For example, consider a magnetic chain on a lattice $L$. 
Each vertex $v$ of the lattice has a spin $s(v)$, $1$ or $-1$. 
The Lagrangian of the theory, for example, can be 
$$
\Cal L=\frac{c}{T}\sum_{v--v'}(s(v)-s(v'))^2,
$$
where $T$ is the temperature, and $v--v'$ means thet $v$ is a neighbor of
$v'$. 
For low $T$ the correlation is very strong, so there are 
large (macroscopic) clusters of aligned spin (magnetism)
For high $T$ the spins are almost 
uncorrelated, so there are only very small clusters of aligned spins
(no magnetism). The transition from one situation to the other is happening
at some fixed temperature $T_c$ called the critical temperature. 
At this temperature we expect to see clusters of all sizes,
and the theory to be scale invariant (conformal). 
Hence this theory should correspond to a ``fixed point'' 
(=an attracting trajectory) of the renormalization group 
flow on the relevant space of theories. We may try to find this trajectory
if we have some description of the relevant space of theories and of the
renormalization group flow, and thus understand the behaviour of the magnet
at the critical temperature. 

{\bf Remark.} In general (for example, in the 
theory of a scalar field), the renormalization group does not have 
a attracting trajectory, but has an attracting
finite-dimensional family of trajectories
-- the space of critical (renormalizable) theories. In this
case we have to make finitely many measurements, in order to determine at 
which point of this space we are. For instance, in the theory of
a scalar field we would have to measure three parameters,
which determine a point on the 3-dimensional variety $Y_r$.  
All other physical quantities can then be 
computed, i.e. expressed via the results of these measurements.  

{\bf 1.5. Reminder of renormalization theory.}

Before we discuss how renormalization group works in quantum field theory, 
we recall the basics of classical renormalization theory.
Since there were other lectures on this subject, we will do it briefly.

In quantum field theory, we consider Lagrangians depending on certain
kinds of fields (scalar fields, fermionic fields, gauge fields).
For example, we can consider a Lagrangian of the form (1.1). 
We are interested in correlation functions of the form
$\<T(\phi_1(x_1)...\phi_N(x_N))\>$, where $T$ means time ordering.
It is usually easier to compute the Fourier transform of this function. 
Feynman rules allow us to do it perturbatively, i.e. to obtain the answer
in the form of a power series in coupling constants.  
This answer comes with two problems:

1. Every term of the series is given by a divergent integral.

2. Even if all terms are convergent, the series is divergent in 
any neighborhood of zero. 

Renormalization theory helps to overcome the first problem.
The second problem has to do with the (difficult) question 
whether the theory actually exists nonperturbatively. 
If it does, this problem can be solved by regarding the series 
expansion not as a Taylor expansion but as an asymptotic
expansion of a certain function. 

Renormalization theory works in two steps:
1) regularization; 2) renormalization. 

Regularization is introducing some kind of cutoff, which makes 
the integral convergent. After regularization, the integral 
is convergent, but depends on the regularization parameter $\L$. 
If $\L$ tends to $\infty$ (i.e. when the cutoff is removed), 
the integral tends to infinity. This problem is removed by the second step
-- renormalization. Namely, we find a way to change the parameters in the 
Lagrangian in such a $\L$-dependent way that when $\L$ and the parameters 
in the Lagrangian are varied simultaneously, with $\L\to\infty$, 
the integral converges to a finite limit. 
This limit is then proclaimed to be the value of the integral.

The most naive cutoff scheme is suppressing large momenta
by changing the propagator, for example
$\frac{1}{k^2+m^2}\to \frac{1_{|k|\le \L}}{k^2+m^2}$ (where
$1_A$ is the characteristic function of a set $A$), or
$\frac{1}{k^2+m^2}\to \frac{1}{k^2+m^2}\left(\frac{\L^2}{k^2+\L^2}\right)^l$.
This scheme ha s some nice features (e.g. it can be applied in a
non-perturbative setting), but
is not always good enough (for example, in gauge theory), 
as it does not preserve 
all symmetries. Namely, it preserves Lorenz symmetry but does not preserve 
gauge symmetries or conformal symmetries. So if we use 
this scheme, we have no gauge or conformal symmetry for each finite $\L$,
and may not recover this symmetry even in the limit. 

There is a more ingenious regularization scheme, which is called
dimensional regularization. This scheme preserves all symmetries, 
but it is only valid in the perturbation expansion. 
Now we will describe what dimensional regularization is.

{\bf 1.6. Dimensional regularization.}

The idea of dimensional regularization is 
to compute the integral corresponding to each Feynman graph
in the ``space'' of dimension $D$, where $D$ is a generic complex number. 
If the graph is divergent, the answer turns out to be a meromorphic function 
of $D$, usually having a pole at $D$ equal to the physical dimension $d$.
The renormalization techique usually used with this kind of regularization
is minimal subtraction: we subtract the pole part 
(this is called ``minimal subtraction'') and proclaim the rest 
(evaluated at $D=d$) to be the value of the integral.

Let us now describe the algorithm of dimensional regularization.
For simplicity 
we assume that our theory includes only scalar massive fields.

Suppose we have a Feynman graph $G$ with $N$ external edges, and momenta
$k_1,...,k_M$ at them ($k_1+...+k_M=0$).  
Let $q_1,...,q_{N-M}$ be the loop variables over which we are integrating.
To uniformize the notation, we set $q_{N-M+i}=k_i$.  
Then the amplitude of the graph is (up to a factor) given by the formula
$$
\Sigma^G(q_{N-M+1},...,q_N)=\int_{V^{N-M}}g(q)dq_1...dq_{N-M},\
g(q)=\frac{Q(q)}{\prod_j(l_j(q)^2+m_j^2)},\tag 1.5
$$
where $Q$ is a polynomial, $q=(q_1,...,q_N)$ and $l_j:V^N\to V$ are 
given by linear forms
on $\R^N$.

Unfortunately, integral (1.5) is divergent, so we will explain 
how to regularize it using dimensional regularization.
As any regularization scheme, dimensional regularization
works inductively in the number of loops (or internal vertices)
of the graph. So we assume that the coefficients in the Lagrangian 
have been adjusted at the previous step of the process 
in such a way that all proper subgraphs of $G$ are convergent. 
 
In order to regularize the integral corresponding to $G$, we 
pretend that it exists and, using the formula
$\int_0^\infty e^{-at}dt=a^{-1}$, transform it to the form
$$
\gather
\Sigma^G=\int Q(q)\biggl[\int_{t_j\in [0,\infty)}e^{-\sum t_jl_j(q)^2-\sum
t_jm_j^2}dt\biggr]dq=\\
\int_{t_j\in [0,\infty)}e^{-\sum
t_jm_j^2} \biggl[\int Q(q)e^{-\sum t_jl_j(q)^2}dq\biggr]
dt.\tag 1.7\endgather
$$
The q-integral in brackets in the last expression is a Gaussian integral, 
so it can be computed according to the standard rules.
Namely, 
from Poincare invariance it is clear that 
the answer is of the form $\psi(t,A,d)\det(B(t))^{-d/2}$, 
where $B(t)$ is a quadratic form on $\Bbb R^{N-M}$ such that
$B(t)_{ij}=(\sum t_sl_s^2)_{ij}$, 
$A=(a_{ij})=(k_i\cdot k_j)$, and
$\psi$ is a polynomial in $d$. 

Thus we get
$$
\Sigma^G=\Sigma^G(A)=
\int_{t_j\in [0,\infty)}e^{-\sum
t_jm_j^2}\psi(t,A,d)(\det B(t))^{-d/2}
dt.\tag 1.8
$$
Of course, this integral is also divergent.
However, since it does not contain integration with respect to $q$, so 
we can replace $d$ in it with a generic complex number $D$. 
It is easy to see that the obtained integral $\Sigma^G_D(A)$ is convergent 
for $\text{Re} D<<0$, and one can prove that it continues meromorphically 
to the whole complex plane (this is a nontrivial theorem).

{\bf Definition.} $\Sigma^G(A)$ is the value of the regular part of
$\Sigma^G_D(A)$ at $D=d$. That is, 
$$
\Sigma^G(A)=\frac{1}{2\pi i}\oint_{|D-d|=\e}\Sigma^G_D(A)d\ln(D-d).\tag 1.9
$$

Now, if the analytic continuation of integral (1.8) is singular at 
$D=d$, we should adjust the Lagrangian by adding 
to it a counterterm $C_G(D)$ (infinite at $D=d$) in such a way
that the singular part cancels and the regular part remains unchanged. 
After this we can go to the next step of the process.

{\bf Remark.} The only divergences that produce a pole at $D=d$ are
logarithmic divergences. Pure power divergences produce poles at 
smaller values of $D$, and do not produce a singularity at $D=d$. 
  
For a more detailed description of the general theory of integration
in a D-dimension space, where D is a complex number, see the notes 
``On integration in a space with complex dimension'' by Pavel Etingof. 

\end


