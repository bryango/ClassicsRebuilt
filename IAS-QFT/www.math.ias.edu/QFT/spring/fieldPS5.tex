%Date: Wed, 19 Feb 1997 17:46:42 -0500
%From: Edward Witten <witten@sns.ias.edu>

\input harvmac

Term 2

Problem Set 5

(1) Consider in two dimensions an $SO(n)$ gauge theory with
scalar fields $\phi$ consisting of $n-1$ copies of the $n$ dimensional
representation of $SO(n)$ and a potential energy
\eqn\guggo{V=\sum_{i=1}^{n-1}\lambda_i\left[(\phi_i,\phi_i)-a_i^2\right]^2
+\sum_{i<j}c_{ij}(\phi_i,\phi_j)^2.}
The $\lambda_i$ and the $c_{ij}$ are all positive.  Here $(~,~)$ is
the invariant quadratic form on the $n$ dimensional representation of
$SO(n)$.

Because $\pi_1(SO(n))={\bf Z}/2{\bf Z}$, this theory
has a two-valued ``theta angle.''  Explain how you would
calculate, for weak coupling, the energy difference between the vacua
with the two possible values of theta.

(If you are unsure of how to think about theta for the case
that $\pi_1(G)$ is a finite group, you can consult my paper
on ``Theta Vacua In Two Dimensional Gauge Theories'' which
will be available from Paula Bozzay in C building.)

(2) Consider in two dimensions the $(\psi_+\psi_-)^2$
model (introduced in Gross's last lecture in December)
with the fields being two dimensional fermions $\psi_i$
with Lagrangian
\eqn\hobo{L=\int d^2x\left(\sum_i(\psi_{+i}\partial_-\psi_{+i}
+\psi_{-i}\partial_+\psi_{-i})+g\sum_i(\psi_{+i}\psi_{-i})^2\right) .}

(a) Consider the classical conservation law $\partial_-S=0$
where $S=\psi_{+1}\psi_{+2}\dots\psi_{+n}$.  By considering
its deformation in quantum mechanics, show that the model is
integrable.

In this way you obtain a conserved charge $R$ with
$[K,R]=(n/2-1)R$ where $K$ is the generator of the Lorentz
group (the subgroup $SO(1,1)$ of the two dimensional Poincar\'e group.)
By exchanging $+$ and $-$, you get another conserved charge $R'$ with
$[K,R']=-(n/2-1)R'$.

(b) Now set $n=4$, so $(n/2) -1 = 1$. Hence $R$ and $R'$ have
the same Lorentz transformation law as the momentum components $P_+$ and
$P_-$.  

You should be able to sharpen the argument in (a) above to show
for $n=4$ that the theory has two stress tensors  $T_{mn}$ and
$T'_{mn}$ where $T$ is the standard stress tensor whose
appropriate integrals give the Poincar\'e generators, and
$T'_{mn}$ has the property that its suitable moments are $R$ and $R'$.
But now, by ``contracting'' $T'$ with the generator  of $SO(1,1)$,
you get a new ``Lorentz generator'' $K'$; show that the usual
Poincar\'e operators together with $R,R'$, and $K'$ generate
two commuting copies of the Poincar\'e group.
(To get the commuting copies, you have to take sums and differences
of ``old'' and ``new'' operators.)

Show that no linear combination of $P_+$ and $R$ (or $P_-$ and $R'$)
vanishes so that each copy of the Poincar\'e group acts nontrivially.
In fact, you should be able to see that the outer automorphism of
$SO(4)$ (which acts in this theory by $\psi_1\to -\psi_1$ with
$\psi_i\to\psi_i$ for $i>1$) exchanges the two copies of Poincar\'e.

Deduce that for $n=4$ this theory is ``reducible'' as the product
of two decoupled theories, in one of which one copy of Poincar\'e acts
and in the other the second.


(c) Use the  bose-fermi correspondence to verify this and identify
the factors.  You should be able to draw an inference about the
symmetries of the quantum sine-Gordon model at a special value
of the coupling.  (You drew the same inference last week in a different
way.)

(d) Now consider the case $n=3$. Show that $R$ and $R'$ are
supersymmetries, so that the model is supersymmetric for $n=3$.
Can you verify this using the bose-fermi correspondence?

A reference for this problem is my paper ``Some Properties Of
The $(\bar\psi\psi)^2$ Model in Two Dimensions'' which will be available
from Paula Bozzay in C building.

(3) The following problem is actually one that was assigned in the fall but
not discussed (see problem set 3 from term one, which
you can also consult for some additional background).  
It is the beginning of a series that will hopefully
prepare us to do some things with supersymmetric  gauge theories
in two and four dimensions.

We work locally for the present; it takes some more work to make
a global version.

\def\R{{\bf R}}\def\Z{{\bf Z}}



We consider a (4,4) supermanifold $W$ with
the structure described in Bernstein's lectures -- two complex
conjugate integrable distributions $A_+$ and $A_-$ of dimension $(0,2)$
such that $\{A_+,A_-\}$ generates $TM/(A_+\oplus A_-)$,  $A_+$ and $A_-$
are endowed with volume forms.
A chiral superfield is a field $\phi$ that is a function on
$W/A_-$.  If $M$ is a complex manifold, a chiral map $\Phi:W\to M$ is
one such that, for $f$ a local holomorphic function on $M$, 
$\Phi^*(f)$ is a chiral function on $W$.  In that case, $\Phi^*(\bar f)$ is
antichiral, that is, it is a function on $W/A_+$.

Let $K$ be a function on $W$ and consider the Lagrangian
\eqn\uggu{L=\int_Wd^4yd^4\theta\,  \Phi^*(K).}
$d^4y\,d^4\theta$ is the section of the Berezinian coming from the
volume forms on $A_\pm$.  

(a) Show that $L$ is invariant under $K\to K+f+\bar f$, where $f$ is
a local holomorphic function on $W$.

(b) Recall that any $K$ determines the two-form 
$\omega=-i\bar\partial\partial K$
which for a suitable class of $K$'s is the Kahler class of a Kahler
metric $g$ on $M$.  Of course, $\omega$ is invariant under the
transformation considered in (a).

Show that $L$ can be written just in terms of $\omega$ (or equivalently
$g$) and that the ``reduced'' part of $L$ is the harmonic map action,
specialized to maps to a Kahler manifold:
\eqn\huggu{L=\int_{W_{red}} d^4y g_{i\bar j}(d X^i,dX^{\bar j})}
where $  X^i$ are local holomorphic coordinates on $M$.

(c) Specialize now to the flat model with $W=\R^{4,4}$.
$A_+$ is generated by
\eqn\ppp{D_A={\partial\over\partial\theta^A}-
i\sigma^{m}_{A\dot A}\bar\theta^{\dot A}
{\partial\over\partial y^m}}
where $\theta^A$ are odd coordinates transforming as ``positive''
spinors of $Spin(1,3)=SL(2,{\bf C})$, $\bar \theta^{\dot A}$ are
their complex conjugates, and $y^m$ are even coordinates.
$\sigma$ is the isomorphism $S_+\otimes S_-= V$.
$A_-$ is generated by 
\eqn\ppp{D_{\dot A}={\partial\over\partial\theta^{\dot A}}-
i\sigma^{m}_{A\dot A}\bar\theta^{ A}
{\partial\over\partial y^m}.}

Impose invariance under a space-like translation of $W$, say
a translation of $y^3$ (the metric being $(dy^0)^2-\sum_{i=1}^3 (dy^i)^2$ 
to get a model on $\R^{3,4}$.  Show that this is equivalent to
the $\R^{3,2}$ 
$\sigma$ model described in problem 1(e) of term 1, problem
set 3 (but specialized to the
case that $M$ is Kahler).  Here is an attempt  to explain the strategy:
under $Spin(1,2)$, the two spinor representations of $Spin(1,3)$
become isomorphic.  So after the reduction, $A_+$ and $A_-$ are
naturally isomorphic.  One can in an $SO(1,2)$-invariant way
take the ``real'' combination $\chi^A=\theta^A+\bar\theta^A$ of
the  $\theta$'s and $\bar\theta$'s. The $\R^{3,2} $ we want
has even coordinates $y^i$,          $i=0,1,2$, and odd coordinates
$\chi^A$.  By ``integrating over the fibers'' of a map from $\R^{3,4}$
to $\R^{3,2}$, one reduces \huggu\ (specialized to $y^3$-independent maps)
to the $\R^{3,2}$ Lagrangian of problem 1(e).

\end


