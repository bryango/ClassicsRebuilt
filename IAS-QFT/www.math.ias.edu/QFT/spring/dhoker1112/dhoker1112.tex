%% This is a plain TeX file
%%
\magnification=1200
\hsize=6.5 true in
\vsize=8.7 true in
\input epsf.tex

\input amssym.def
\input amssym.tex

\font\boldtitlefont=cmb10 scaled\magstep2
\font\smallboldtitle=cmb10 scaled \magstep1
\font\dun=cmdunh10 %scaled\magstep1

\footline={\hfil {\tenrm VII.\folio}\hfil}

\def\eps{{\varepsilon}}
\def\Eps{{\epsilon}}
\def\kap{{\kappa}}
\def\lam{{\lambda}}
\def\Lam{{\Lambda}}
\def\mynabla{{\nabla\!}}

\def\underNS{\underline{\NS}}
\def\underR{\underline{\R}}

\def\Bmu{{B_{\mu\nu}}}
\def\Gmu{{G_{\mu\nu}}}

\def\xdot{{\dot x}}
\def\xddot{{\ddot x}}

\def\undertext#1{$\underline{\vphantom{y}\hbox{#1}}$}
\def\nspace{\lineskip=1pt\baselineskip=12pt%
     \lineskiplimit=0pt}
\def\dspace{\lineskip=2pt\baselineskip=18pt%
     \lineskiplimit=0pt}

\def\half{\raise4.5pt\hbox{{\vtop{\ialign{##\crcr
  \hfil\rm $1$\hfil\crcr
   \noalign{\nointerlineskip}--\crcr
   \noalign{\nointerlineskip\vskip-1pt}$2$\crcr}}}}}
\def\third{\raise4.5pt\hbox{{\vtop{\ialign{##\crcr
  \hfil\rm $1$\hfil\crcr
  \noalign{\nointerlineskip}--\crcr
  \noalign{\nointerlineskip\vskip-1pt}$3$\crcr}}}}}
\def\fourth{\raise4.5pt\hbox{{\vtop{\ialign{##\crcr
  \hfil\rm $1$\hfil\crcr
  \noalign{\nointerlineskip}--\crcr
  \noalign{\nointerlineskip\vskip-1pt}$4$\crcr}}}}}
\def\sixth{\raise4.5pt\hbox{{\vtop{\ialign{##\crcr
  \hfil\rm $1$\hfil\crcr
  \noalign{\nointerlineskip}--\crcr
  \noalign{\nointerlineskip\vskip-1pt}$6$\crcr}}}}}
\def\eighth{\raise4.5pt\hbox{{\vtop{\ialign{##\crcr
  \hfil\rm $1$\hfil\crcr
  \noalign{\nointerlineskip}--\crcr
  \noalign{\nointerlineskip\vskip-1pt}$8$\crcr}}}}}

\def\oplusop{\mathop{\oplus}\limits}
\def\w{{\mathchoice{\,{\scriptstyle\wedge}\,}
  {{\scriptstyle\wedge}}
  {{\scriptscriptstyle\wedge}}{{\scriptscriptstyle\wedge}}}}
\def\Le{{\mathchoice{\,{\scriptstyle\le}\,}
{\,{\scriptstyle\le}\,}
{\,{\scriptscriptstyle\le}\,}{\,{\scriptscriptstyle\le}\,}}}
\def\Ge{{\mathchoice{\,{\scriptstyle\ge}\,}
{\,{\scriptstyle\ge}\,}
{\,{\scriptscriptstyle\ge}\,}{\,{\scriptscriptstyle\ge}\,}}}
\def\plus{{\hbox{$\scriptscriptstyle +$}}}
\def\xdot{\dot{x}}
\def\Condition#1{\item{#1}}
\def\Firstcondition#1{\hangindent\parindent{#1}\enspace
     \ignorespaces}
\def\Proclaim#1{\medbreak
  \medskip\noindent{\bf#1\enspace}\it\ignorespaces}
\def\finishproclaim{\par\rm
     \ifdim\lastskip<\smallskipamount\removelastskip
     \penalty55\medskip\fi}
\def\Item#1{\par\smallskip\hang\indent%
  \llap{\hbox to\parindent {#1\hfill\enspace}}\ignorespaces}
\def\ItemItem#1{\par\indent\hangindent2\parindent
     \hbox to \parindent{#1\hfill\enspace}\ignorespaces}
\def\vrulesub#1{{\,\vrule height7pt depth5pt}_{\,#1}}
\def\underbrake#1#2{\mathop{#1}\limits_{\raise3pt
  \hbox{%
\vrule height 3pt depth 0pt
  %\kern.1pt
  \hbox to #2{\hrulefill}
  \kern-3.4pt
  \vrule height 3pt depth 0pt}}}

\def\ominus{{$-$\kern-9pt $\bigcirc$}}
\def\Oplus{{+\kern-9pt $\bigcirc$}}
\def\ssbullet{{\scriptstyle\bullet\,\,\,}}

\def\Open{{\rm open}}
\def\Sp{{\rm Sp}}
\def\R{{\rm R}}  \def\NS{{\rm NS}}
\def\RNS{{\rm RNS}}
\def\Diff{{\rm Diff}}  \def\expt{{\rm expt}}
\def\cm{{\rm cm}}  \def\annulus{{\rm annulus}}
\def\cylinder{{\rm cylinder}}
\def\Closed{{\rm closed}}
\def\Map{{\rm Map}}  
\def\Met{{\rm Met}} 
\def\Spin{{\rm Spin}}
\def\phys{{\rm phys}}
\def\diag{{\rm diag}}  
\def\Null{{\rm null}} 
\def\mass{{\rm mass}}
\def\SO{{\rm SO}} 
\def\GSO{{\rm GSO}}
\def\Tr{{\rm Tr\,}}
\def\tr{{\rm tr}}
\def\Weyl{{\rm Weyl}} 
\def\Lorentz{{\rm Lorentz}}
\def\Ker{{\rm Ker}}
\def\Range{{\rm Range}} 
\def\Det{{\rm Det}} 
\def\IIA{{\rm II~A}}
\def\IIB{{\rm II~B}}

\def\mbar{\bar{m}}   \def\Lambar{\bar{\Lambda}}
\def\nubar{\bar{\nu}}
\def\Gammabar{\bar{\Gamma}}
\def\chibar{\bar{\chi}}
\def\rbar{\bar{r}}
\def\zbar{\bar{z}}  
\def\Gbar{\bar{G}}
\def\Kbar{\bar{K}}
\def\Sbar{\bar{S}}
\def\Tbar{\bar{T}}

\def\Ftil{\widetilde{F}}
\def\btil{\tilde{b}}
\def\dtil{\tilde{d}}
\def\xtil{\tilde{x}}
\def\Ptil{\widetilde{P}}
\def\betatil{\tilde{\beta}}
\def\scrFtil{\widetilde{\scrF}}


%These two files (in this order!!) are necessary
%in order to use AMS Fonts 2.0 with Plain TeX

\input amssym.def
\input amssym.tex

%Capital roman double letters(Blackboard bold)
\def\db#1{{\fam\msbfam\relax#1}}

\def\dbA{{\db A}} \def\dbB{{\db B}}
\def\dbC{{\db C}} \def\dbD{{\db D}}
\def\dbE{{\db E}} \def\dbF{{\db F}}
\def\dbG{{\db G}} \def\dbH{{\db H}}
\def\dbI{{\db I}} \def\dbJ{{\db J}}
\def\dbK{{\db K}} \def\dbL{{\db L}}
\def\dbM{{\db M}} \def\dbN{{\db N}}
\def\dbO{{\db O}} \def\dbP{{\db P}}
\def\dbQ{{\db Q}} \def\dbR{{\db R}}
\def\dbS{{\db S}} \def\dbT{{\db T}}
\def\dbU{{\db U}} \def\dbV{{\db V}}
\def\dbW{{\db W}} \def\dbX{{\db X}}
\def\dbY{{\db Y}} \def\dbZ{{\db Z}}

\font\teneusm=eusm10  \font\seveneusm=eusm7 
\font\fiveeusm=eusm5 
\newfam\eusmfam 
\textfont\eusmfam=\teneusm 
\scriptfont\eusmfam=\seveneusm 
\scriptscriptfont\eusmfam=\fiveeufm 
\def\scr#1{{\fam\eusmfam\relax#1}}


%Upper-case Script Letters:

\def\scrA{{\scr A}}   \def\scrB{{\scr B}}
\def\scrC{{\scr C}}   \def\scrD{{\scr D}}
\def\scrE{{\scr E}}   \def\scrF{{\scr F}}
\def\scrG{{\scr G}}   \def\scrH{{\scr H}}
\def\scrI{{\scr I}}   \def\scrJ{{\scr J}}
\def\scrK{{\scr K}}   \def\scrL{{\scr L}}
\def\scrM{{\scr M}}   \def\scrN{{\scr N}}
\def\scrO{{\scr O}}   \def\scrP{{\scr P}}
\def\scrQ{{\scr Q}}   \def\scrR{{\scr R}}
\def\scrS{{\scr S}}   \def\scrT{{\scr T}}
\def\scrU{{\scr U}}   \def\scrV{{\scr V}}
\def\scrW{{\scr W}}   \def\scrX{{\scr X}}
\def\scrY{{\scr Y}}   \def\scrZ{{\scr Z}}

\def\gr#1{{\fam\eufmfam\relax#1}}

%Euler Fraktur letters (German)
\def\grA{{\gr A}}	\def\gra{{\gr a}}
\def\grB{{\gr B}}	\def\grb{{\gr b}}
\def\grC{{\gr C}}	\def\grc{{\gr c}}
\def\grD{{\gr D}}	\def\grd{{\gr d}}
\def\grE{{\gr E}}	\def\gre{{\gr e}}
\def\grF{{\gr F}}	\def\grf{{\gr f}}
\def\grG{{\gr G}}	\def\grg{{\gr g}}
\def\grH{{\gr H}}	\def\grh{{\gr h}}
\def\grI{{\gr I}}	\def\gri{{\gr i}}
\def\grJ{{\gr J}}	\def\grj{{\gr j}}
\def\grK{{\gr K}}	\def\grk{{\gr k}}
\def\grL{{\gr L}}	\def\grl{{\gr l}}
\def\grM{{\gr M}}	\def\grm{{\gr m}}
\def\grN{{\gr N}}	\def\grn{{\gr n}}
\def\grO{{\gr O}}	\def\gro{{\gr o}}
\def\grP{{\gr P}}	\def\grp{{\gr p}}
\def\grQ{{\gr Q}}	\def\grq{{\gr q}}
\def\grR{{\gr R}}	\def\grr{{\gr r}}
\def\grS{{\gr S}}	\def\grs{{\gr s}}
\def\grT{{\gr T}}	\def\grt{{\gr t}}
\def\grU{{\gr U}}	\def\gru{{\gr u}}
\def\grV{{\gr V}}	\def\grv{{\gr v}}
\def\grW{{\gr W}}	\def\grw{{\gr w}}
\def\grX{{\gr X}}	\def\grx{{\gr x}}
\def\grY{{\gr Y}}	\def\gry{{\gr y}}
\def\grZ{{\gr Z}}	\def\grz{{\gr z}}

%\overfullrule=0pt

\parindent=25pt
\line{\dun --- DRAFT ---\hfill{\rm IASSNS-HEP-97/72}}

\bigskip\bigskip
\centerline{\boldtitlefont Lectures 11 and 12}
\bigskip
\centerline{\smallboldtitle VII. Free Superstrings}

\medskip
\centerline{Eric D'Hoker}

\frenchspacing

\dspace
\bigskip
Purely bosonic string theory, as defined and studied in
the preceding lectures, has a number of fundamental
flaws, which we shall now briefly summarize.

\medskip
\Item{(i)}\enspace
Bosonic string theory in flat --- or nearly flat ---
space-time possesses a tachyonic state.
The presence of a tachyon leads to a violation of the
physical principle of causality.
It signals an instability of (nearly) flat 
space-time, which may be detected from the
fact that the $1$-loop order dilaton tadople amplitude
is non-zero (in fact it is infinite).
Ultimately, the presence of the tachyon leads to the
divergence of transition amplitudes beyond tree level.
Thus, bosonic string, in (nearly) flat 
space-time is in itself inconsistent.

\medskip
\Item{(ii)}\enspace
Bosonic string theory does not possess any fermionic
states (i.e. transforming under a spinor
representation of the space-time Lorentz group).
This fact by itself does not \`a priori lead to
an inconsistency
of the bosonic string.
Rather, since Nature clearly displays fermionic states
(electrons, quarks, etc.), any viable physical
theory aimed at encompassing all physical particles in
Nature, as well as their interactions, must contain
fermions.
{}From this point of view, the bosonic string cannot be
a viable theory of Nature.

\medskip
\Item{(iii)}\enspace
In addition, it is desirable 
to have a theory of closed strings only that
contains Yang-Mills gauge states.
(We have shown already how Yang-Mills particles arise
in open string theories.)

\medskip
\Item{(iv)}\enspace
Finally, we do not perceive $26$ dimensions of
space-time, but only $4$, at distance scales
accessible to present day experiments.
Thus, it must be understood why and how the remaining
dimensions have remained invisible.

\medskip
Problems (i), (ii) and in some cases also (iii) may
all be resolved by introducing certain additional
dynamical degrees of freedom on the string worldsheet.
In particular, extra degrees of freedom may be
introduced that will result in
fermionic string states in the physical Hilbert space.
This extension will in general modify the critical
dimension $D$ of space-time.

There are several ways in which one can proceed to include
fermionic string states in the Hilbert space of string
states.

\medskip
\Item{(1)} {\it The Green-Schwarz (GS) formulation}.

\Item{}The theory has space-time vector fields
$x^\mu$, transforming under the vector representation
of $\SO(1,D-1)$ as well as space-time spinor fields
$\theta^\alpha$ transforming under a spinor representation
of $\Spin(1,D-1)$.
The ground state (i.e. lowest mass state of a single
string) is in general degenerate, with both vector and
spinor states.
Bosonic and fermionic states are obtained by applying
$x^\mu$ or $\theta^\alpha$ to the ground states.

\medskip
\Item{(2)} {\it The Neveu-Schwarz-Ramond (RNS) formulation}.

\Item{}The fundamental fields in the theory are
space-time vectors $x^\mu,\psi^\mu$.
There
are two sectors in the Hilbert space of states, built
on two distinct ground states:
The Neveu-Schwarz ground state is a boson, while the
Ramond ground state is a fermion.
All bosonic states are obtained by applying $x^\mu$ 
and $\psi^\mu$ to the Neveu-Schwarz ground state; this
generates the Neveu-Schwarz sector of the theory.
All fermionic states are obtained by applying $x^\mu$
and $\psi^\mu$ to the fermionic ground state; this
generates the Ramond sector of the theory.

\medskip\noindent
By quantizing each formulation in the lightcone
gauge, one demonstrates that (under certain
coniditons) the GS and RNS formulations are equivalent.
The Green-Schwarz formulation apears difficult (if not
impossible) to quantize in a manifestly Lorentz
 covariant way.

One distinguishes the following five, consistent,
tachyon free, space-time supersymmetric string
theories in flat Minkowski space-time.

\vfill\eject

$$
\vbox{\tabskip=8pt
\offinterlineskip
\halign to \hsize{#\hfill &#\hfill &\qquad #\hfill\cr
$\bullet$ &Type I &Open and closed unoriented
strings, with Yang-Mills\cr
\noalign{\smallskip}
&&degrees of freedom via Chan-Paton rules:\cr
\noalign{\smallskip}
&&$\SO(32)$ gauge group only.\cr
\noalign{\bigskip}
$\bullet$ &Type II$\,\,$A,B &Closed oriented strings
only;\cr
\noalign{\smallskip}
&&A: \ opposite chirality Ramond ground states\cr
\noalign{\smallskip}
&&B: \ same chirality Ramond ground states.\cr
\noalign{\bigskip}
$\bullet$ &Heterotic &Type II left-movers, 
     bosonic right movers,\cr
\noalign{\smallskip}
&$\Spin(32)/Z_2,$ &including gauge degrees of freedom:\cr
\noalign{\smallskip}
&$E_8\times E_8$ &\qquad $\Spin(32)/Z_2$ or 
$E_8\times E_8$ gauge groups only.\cr}}
$$
The goal of this Chapter (\S{VII}) and the next one
(\S{VIII}) is to define these theories, show their
uniqueness and analyze the Hilbert space of states, in
the RNS formulation.
Space-time supersymmetry and the GS formulation will
be discussed extensively in \S{X}.

\bigskip\noindent
\Item{\bf A)} {\bf Degrees of Freedom of the {\rm RNS} string}

To obtain fermionic physical states, we add further
degrees of freedom which are {\it spinors on the
worldsheet $\Sigma$}.
We begin by focussing on closed oriented worldsheets,
without boundary.

Worldsheet spinors exist on $\Sigma$, provided a spin
bundle $S=K^{1/2}$ can be constructed as the square
root of the canonical bundle $K$.
The worldsheet spinor fields are then sections of $S$.
Spin bundles exist on $\Sigma$ provided the second
Stiefel-Whitney class $H^2(\Sigma,\dbZ_2)$, which is
the first Chern class $\bmod{2}$ of $K$, vanishes.
This is indeed the case, since
$c_1(K)=-\chi(\Sigma)=2h-2$ which is even.
The different spin bundles are labeled by
$H^1(\Sigma,\dbZ_2)$ and may be distinguished by $\pm$
sign assignments around the homology cycles of
$\Sigma$.
For compact $\Sigma$ without boundary, there are $2^{2h}$
different spin structures.

Following the introductory remarks, the worldsheet
spinors transform as Lorentz $\SO(1,D-1)$ vectors.
We denote them by $\psi_\pm^\mu(z,\zbar)$, where
the subscript $\pm$ is the worldsheet spinor index.
In addition, $\psi_+^\mu$ and $\psi_-^\mu$ are to be
constrained by a requirement inherited from the
Minkowski signature worldsheet: each is a worldsheet
Majorana-Weyl spinor, which means that each is a one
component real spinor in a suitable basis.

On worldsheets with Euclidean signature, there are no
Majorana-Weyl spinors.
However, one may use a complex Weyl spinor
$\psi_+^\mu$, and take $\psi_-^\mu$ to be its complex
conjugate.
This construction preserves the number of degrees of
freedom.
Chiral splitting techniques, relying on the holomorphic
dependence on moduli --- as introduced and applied in
\S{V} --- may then be invoked to separate correlation
functions into parts arising from $\psi_+$ and
$\psi_-$ separately.
Thus, henceforth, $\psi_+^\mu$ is a complex Weyl
spinor, which is a section of a spin bundle $K^{1/2}$
for some given spin structure; $\psi_-^\mu$ is the
complex conjugate of $\psi_+^\mu$, and a section of
$\Kbar^{1/2}$ for the same spin structure.

It is natural (but, as we shall find later, it is
incomplete) to let the dynamics of $\psi_{\pm}^\mu$
be governed by the action for massless spinors
obtained from the $bc$ system of \S{IV} for $n=1/2$.
For $n=1/2$, the $bc$ system decomposes into two
identical spinor systems in a conformally invariant
way:
$$
b={1\over \sqrt{2\,\,}}(\psi_++i\psi'_+)\qquad
c={1\over\sqrt{2\,\,}}(\psi_+-i\psi'_+)\,\,.
\eqno{(7.A.1)}
$$
The action for $\psi_{\pm}^\mu$ is then found to be
$$
S_\psi[\psi]={1\over 4\pi}\int\nolimits_{\Sigma}
d\mu_g\left(\psi_+^\mu\nabla_{(1/2)}^z\psi_{+\mu}
+\psi_{-}^\mu\nabla_{(-1/2)}^{\zbar}\psi_{-\mu}\right)\,\,.
\eqno{(7.A.2)}
$$
It describes a conformal field theory with central
charge $D\times{1\over 2}={1\over 2}\,D$, namely
$c=1/2$ for each Majorana fermion.
We record here the stress tensor for this system:
$$
T_{zz}^{(\psi)}=\half\,\partial_z\psi_+^\mu\psi_{+\mu}
\eqno{(7.A.3)}
$$
as well as the field equations and the OPE (both can be
deduced from the $bc$ system)
$$
\nabla^z\psi_+^\mu=0\,;\qquad
\psi_+^\mu(z)\psi_+^\nu(w)\sim{1\over z-w}\,\eta^{\mu
\nu}\eqno{(7.A.4)}
$$
and analogously for $\psi_-^\mu$.

We now specialize to the case of a flat cylindrical
worldsheet, or by conformal mapping the annular
worldsheet $\Sigma$.
$$
\vbox{\epsfxsize=3.5in\epsfbox{fig1.eps}}
$$
First of all, this topology supports a single homology
cycle, and there are two different spin bundles,
corresponding to $\pm$ assignments around this cycle.
To preserve the action of Lorentz transformations
under which $\psi^\mu$ transforms as a
vector, all components of $\psi^\mu$,
$\mu=0,\ldots,D-1$ must carry the same spin structure,
i.e. be sections of the same spin bundle.
It is conventional to denote these spin structures by
Ramond (R) and Neveu-Schwarz (NS), defined as follows.
(The correspondence between the assignments on the
cylinder and annulus is unintuitive, so we shall be
completely explicit here.)

\noindent
{\it On the Cylinder}
$$
\eqalign{
\psi_\pm^\mu(\tau,\sigma+2\pi)=+\psi_\pm^\mu
  (\tau,\sigma) &\null\qquad\qquad\hbox{\rm R: periodic
on cylinder}\cr
\psi_\pm^\mu(\tau,\sigma+2\pi)=-\psi_\pm^\mu
  (\tau,\sigma) &\null\qquad\qquad\hbox{\rm NS:
anti-periodic on cylinder}\cr}
\eqno{(7.A.5)}
$$

\noindent
To obtain the corresponding boundary conditions on the
annulus, we use the conformal mapping $z=e^w$.
On spinors $\psi_+$, the effect is
$$
\eqalign{
\psi_+^{\annulus}(z)(dz)^{1/2} &=\psi_+^{\cylinder}(w)
  (dw)^{1/2}\cr
\left({dz\over dw}\right)^{1/2} &=e^{w/2}\cr}
\eqno{(7.A.6)}
$$
Under a $2\pi$ rotation on the cylinder, which
corresponds to a $2\pi$ rotation on the annulus, the
transition factor $e^{{1\over 2}\,w}$ changes by a sign.
Hence the (anti-) periodicity assignments are reversed
between the cylinders and the annulus.

\noindent
{\it On the Annulus}
$$
\eqalign{
\psi_\pm^\mu(e^{2\pi i})=-\psi_\pm^\mu(z)
 &\null\qquad\qquad\hbox{\rm R: anti-periodic on annulus}\cr
\psi_\pm^\mu(e^{2\pi i})=+\psi_\pm^\mu(z)
  &\null\qquad\qquad \hbox{\rm NS: periodic on annulus}\cr}
\eqno{(7.A.7)}
$$

We are now ready to decompose $\psi_{\pm}^\mu$ in
modes and deduce the algebra of the modes, and the
Virasoro algebra.
We work on the annulus; translation to the cylinder
is straightforward.
In the Ramond sectors, we have
$$
\left\{
\eqalign{
\psi_+^\mu(z) &=\sum\limits_{n\in\dbZ}d_n^\mu
  z^{-n-1/2}\cr
\psi_-^\mu(\zbar) &=\sum\limits_{n\in\dbZ}\dtil_n^\mu
  \zbar^{-n-1/2}\cr}
\right.
\qquad\qquad \hbox{\rm R (annulus)}
\eqno{(7.A.8)}
$$
$$
\null\qquad\qquad
\{d_m^\mu,d_n^\nu\}
  =\eta^{\mu\nu}\delta_{m+n,0}\qquad
\{\dtil_m^\mu,\dtil_n^\nu\}
 =\eta^{\mu\nu}\delta_{m+n,0}\,\,.
\eqno{(7.A.9)}
$$
And in the Neveu-Schwarz sector, we have%
\footnote{*}{It is conventional to denote the NS mode
operators by $b_r^\mu$, but this notation does not
suggest a direct relation with the $b$ operators of
the $bc$ system.}
$$
\left\{
\eqalign{
\psi_+^\mu(z) &=\sum\limits_{r\in{1\over 2}+\dbZ}
b_r^\mu\,z^{-r-1/2}\cr
\psi_-^\mu(\zbar) &=\sum\limits_{r\in{1\over 2}+\dbZ}
\btil_r^\mu\,\zbar^{-r-1/2}\cr}
\right.
\qquad\qquad\hbox{\rm NS (annulus)}
\eqno{(7.A.10}
$$
$$
\{b_r^\mu,\,b_s^\nu\}=
  \eta^{\mu\nu}\delta_{r+s,0}\qquad
\{\btil_r^\mu,\btil_s^\nu\}=\eta^{\mu\nu}
\delta_{r+s,0}
\eqno{(7.A.11)}
$$

\bigskip\noindent
\Item{\bf B)} {\bf Ramond and Neveu-Schwarz Fock
spaces}

Let $\scrF_k$ and $\scrFtil_k$ denote the Fock spaces
of the bosonic degrees of freedom at momentum $k$ for
left and right movers, respectively.
Let $\scrF^{\R}$ and $\scrF^{\NS}$ be the fock spaces
for Ramond and Neveu-Schwarz degrees of freedom for
left-movers and $\scrFtil^{\R}$ and $\scrFtil^{\NS}$
their right-moving counterparts.
Then, the ful RNS Fock spaces for open and closed RNS
strings are
$$
\eqalign{
\scrF_{\Open} &=\oplusop_{k}\,\scrF_k^{\RNS}\cr
\scrF_{\Closed} &=\oplusop_{k}\,\scrF_k^{\RNS}
\otimes\scrFtil_k^{\RNS}\qquad\qquad
k\in\dbR^D\,\,.\cr}
\eqno{(7.B.1)}
$$
Here we have defined the left and right moving RNS
Fock spaces by
$$
\eqalign{
\scrF_k^{\RNS} &=\scrF_k\otimes (\scrF^{\R}\oplus
\scrF^{\NS})\cr
\scrFtil_k^{\RNS} &=\scrFtil_k\otimes(\scrFtil^{\R}
\oplus \scrFtil^{\NS})\,\,.\cr}
\eqno{(7.B.2)}
$$
It thus suffices to construct $\scrF^{\R}$ and
$\scrF^{\NS}$.

The {\it Neveu-Schwarz ground state}
$\left.\vert0;\NS\right>\in\scrF^{\NS}$ is defined by%
\footnote{$^{**}$}{We use the notation $\dbN$,
$\dbZ^+$, $\dbZ^-$ for the sets of positive, positive or
zero and negative or zero integers, respectively.}
$$
b_r^\mu\left.\vert 0;\,\NS\right>=0\quad
  r\in{1\over 2}+\dbZ^+
\eqno{(7.B.3)}
$$
The full ground state in $\scrF_k\otimes\scrF^{\NS}$
will be denoted by $\left.\vert0,k;\,\NS\right>\equiv
\left.\vert0,k\right>\otimes\left.\vert0;\,\NS\right>$.
We normalize the ground state by
$\left<0;\,\NS\vert0;\,\NS\right>=1$, and the same for 
$^{\textstyle\sim}$.
By examining the Lorentz spin operator in the $\NS$
sector
$$
S_{\NS}^{\mu\nu}=\sum\limits_{r\in{1\over 2}+\dbZ^+}
\left(b_{-r}^\mu b_r^\nu-b_{-r}^\nu b_r^\mu\right)\,\,,
\eqno{(7.B.4)}
$$
we see that $\left.\vert0;\,\NS\right>$ is a scalar
under $\SO(1,D-1)$, and thus a {\it boson}.
The NS Fock space
$\scrF^{\NS}$ is now obtained by applying the algebra
of raising operators $\{b_r^\mu,r<0\}$ to
$\left.\vert0;\,\NS\right>$.
All the states obtained this way transform under
tensor representations of $\SO(1,D-1)$ and hence are
{\it bosons}.

The {\it Ramond ground state} is degenerate, just as
the ground state of the $bc$ system was degenerate.
We label the degenerate states in the ground state
multiplet by an index $\alpha$, and we have as usual
$$
d_n^\mu\left.\vert 0,\alpha;\,R\right>=0\qquad\qquad
n\in\dbN
\eqno{(7.B.5)}
$$
The full ground state in $\scrF_k\otimes\scrF^R$
will be denoted by $\left.\vert0,k;\alpha;R\right>\equiv 
\left.\vert0,k\right>\otimes\left.\vert0,\alpha;R\right>$.
The mode $d_0^\mu$ satisfies a  Clifford algebra
$$
\left\{d_0^\mu,d_0^\nu\right\}=\eta^{\mu\nu}
\eqno{(7.B.6)}
$$
so that $\left.\vert0,\alpha;R\right>$ transforms under a
spinor representation of the Lorentz group
$\Spin(1,D-1)$. 
Here, $\alpha$ ranges over the weights of the
Dirac spinor $\alpha=1,\ldots,2^{D/2}$
(we assume $D$ even).
The fact that the ground state
$\left.\vert0,\alpha;R\right>$ is a spinor may also be
confirmed from examining the expression for the
Lorentz spin operator in the R sector:
$$
S_R^{\mu\nu}=\half\,[d_0^\mu,d_0^\nu]+\sum\limits_{n\in\dbN}
(d_{-n}^\mu d_n^\nu-d_{-n}^\nu d_n^\mu)
\eqno{(7.B.7)}
$$
On $\left.\vert 0,\alpha;R\right>$, the operator
$S_R^{\mu\nu}$ reduces to the familiar Lorentz
generator in the spinor representation associated with
the Clifford algebra of $d_0^\mu$.
Thus, the Ramond ground state is a $\Spin(1,D-1)$
spinor, and a space-time {\it fermion}.
$\scrF^R$ is obtained by applying the algebra of
raising oprators $\{d_n^\mu,\,\,n<0\}$ to $\left.\vert
0,\alpha;R\right>$ and all states obtained this way are
spinors of $\SO(1,D-1)$ and space-time fermions!
The Ramond construction achieves the goal of
introducing fermions into string theory.

Both Fock spaces $\scrF^R$ and $\scrF^{\NS}$ contain
{\it negative norm states}, just as $\scrF_k$
contained negative norm states in the bosonic string.
The reason is simply that both $b_r^\mu$ and $d_n^\mu$
transform under vector representations, just as the
bosonic modes $x_n^\mu$ did.
Thus, we have for some vector $\Eps\in\dbR^D$:
$$
\eqalign{
&\Vert\Eps\cdot b_{r}\left.\vert0;\NS\right>\Vert^2=
  \Eps^2\cr
&\Vert \Eps\cdot d_{-n} \left.\vert0,\alpha;R\right>\Vert^2=
  \Eps^2\,\,\Vert\left.\vert0;\alpha;R\right>\Vert^2\cr}
\eqno{(7.B.8)}
$$
which may become negative when $\Eps $ is time-like and
$\Eps^2<0$.

\bigskip\noindent
\Item{\bf C)} {\bf Local supersymmetry on the
worldsheet}

In bosonic string theory, the negative norm states of
$x^\mu$ are eliminated by the constraints of 
diffeomorphism invariance (expressed through
the Virasoro algebra) which
result from the quantization of the worldsheet metric.

In the RNS string, the bosonic fields
$x^\mu$ of course still produce negative norm states,
which are still eliminated by the Virasoro algebra
constraints.
But now, the RNS fields $\psi_\pm^\mu$ produce
further negative norm states (independent of those
of $x^\mu$) which are not eliminated by the Virasoro
algebra.
In fact, since the fields $\psi_\pm^\mu$ are
Grassmannian, the constraints needed to eliminate 
the states created by the time component
$\psi_\pm^0$ will have to be Grassmannian.
In addition, since we are to eliminate roughly an
entire field component $\psi_\pm^0$, we need this
invariance to be {\it local on $\Sigma$}.
Grassmann odd local symmetries are so-called {\it
local supersymmetries on $\Sigma$}, and actions 
invariant under
those symmetries are so-called {\it supergravity
actions}.
They involve, in addition to $x^\mu$,
$\psi_\pm^\mu$ and the worldsheet metric
$g_{mn}$, also a {\it gravitino field}
$\chi_m^\pm$, which is a spinor-vector, i.e.
$\chi\in K^{3/2}\oplus K^{1/2}\oplus\Kbar^{1/2}
\oplus\Kbar^{3/2}$.
This type of
field was discovered by Rarita and Schwinger in their
study of spin $3/2$ fields in $4$ space-time
dimensions, and is therefore sometimes called the
Rarita-Schwinger field.


We shall here give the appropriate supergravity action
on any worldsheet, with arbitrary metric $g_{mn}$ and
$\chi_m^\sigma$-field with the spinor index
$\sigma=\pm$.
To do so, we introduce a local frame $e_m{}^a$, where
the $m$-indices refer to coordinate vectors, while the
$a$-indices refer to the frame vectors $a=1,2$.
We denote the inverse frame by $e_a{}^m$.
We then have
$$
g_{mn}=e_m{}^a e_n{}^b\delta_{ab}\,;\qquad\qquad
e_a{}^me_m{}^b=\delta_a{}^b
\eqno{(7.C.1)}
$$
where $\delta_{ab}$ is the $O(2)$-invariant frame
metric.
Replacing the metric by the frame introduces an
additional field. 
However, there is now also an entire
group of local frame $O(2)$ rotations, which 
can always be used to reduce the number of components to
that of the metric.
We shall denote the spin connection by $\omega_m$.

We also have a $2$-dimension Clifford algebra, defined
by
$$
\{\gamma^a,\gamma^b\}=-\delta^{ab}\,\,.
\eqno{(7.C.2)}
$$
The $N=1$ supergravity action, with a single supersymmetry,
is given by (for flat space-time
$M$):
$$
\eqalign{
S[x,\psi;g,\chi]={1\over 4\pi}\int\nolimits_{\Sigma}
d\mu_g &\Bigl[\half\,g^{mn}\partial_m x^\mu\partial_n
x_\mu+\psi^\mu \gamma^a e_a{}^m\partial_m\psi_\mu\cr
&-\psi^\mu\gamma^a\gamma^b\chi_a e_b{}^m\partial_m x_\mu-
\fourth\,\psi^\mu\gamma^a
  \gamma^b\chi_a(\chi_b\psi_\mu)\Bigr]\cr}
\eqno{(7.C.3)}
$$
(The generalization to arbitrary space-time manifolds
$M$ is straightforward.)
For this action to make sense, and be the integral of
a single-valued function, it is necessary that the
spin structures of $\psi^\mu$ and $\chi_m^\alpha$ be
the same.
The action is invariant under the following
transformations.

\medskip\noindent
\Item{1)}
Poincar\'e symmetry in $M$.

\Item{2)}
$\Diff(\Sigma)$.

\Item{3)}
$\Weyl(\Sigma)$ \qquad
$(\delta\psi^\mu=-{1\over 2}\,\delta\sigma\psi^\mu
\qquad 
\delta\chi_m^\sigma={1\over 2}\,
\delta\sigma\chi_m^\sigma)$.\hfill (7.C.4)

\Item{4)}
$\Lorentz(\Sigma)$ \qquad local $O(2)$ frame rotations.

\Item{5)}
super-Weyl\qquad $(\delta e_{}^a=0\qquad
\delta\chi_m=\gamma_m\delta\lam)$\hfill\break
(The spin structure of $\lam$ is the same as that of
$\chi$.)

\Item{6)}
Local supersymmetry, generated by a spinor $\zeta$

\medskip
$$
\left\{
\eqalign{
&\delta x^\mu=\zeta \psi^\mu\cr
&\delta \psi^\mu=\gamma^m(\partial_m
x^\mu-\half\,\chi_m\psi^\mu)\zeta\cr
&\delta e_m{}^a=\zeta\gamma^a\chi_m\cr
&\delta \chi_m^\sigma=-2\nabla_m\zeta^\sigma\cr}
\right.
$$
The spin structure of $\zeta$ is the same as that of
$\psi$ and $\chi$.
In view of $\Diff(\Sigma)$ and $\Weyl(\Sigma)$
invariance, the above action defines a conformal field
theory; as we shall explain shortly it will in fact be
{\it super-conformal}.

\bigskip\noindent
\Item{\bf D)} {\bf Functional Integral Representation
of Transition Amplitudes}

It will turn out to be useful in discussing the
quantization of free strings to have available the
general functional integral representation of
transition amplitudes for interacting strings as well,
on general worldsheet topologies.
Thus, we briefly interrupt the discussion of free
strings to include the functional integral
representation.

We give first a definition of the transition
amplitudes in terms of the fields $x$, $\psi$, $g$,
$\chi$.
Later on, we shall make use of superfields to
simplify the expressions.
Let $V_i$ be a collection of vertex operators for the
RNS string 
(we shall postpone the construction of the vertex operators).
The scattering amplitude of $N$
strings represented by vertex oprators $V_1,\ldots,V_N$
is given by: 
$$
\eqalign{
A=\sum\limits_{{{\scriptstyle \rm topologies}\atop
\scriptstyle h=0,\ldots,\infty}}
\sum\limits_{{{\scriptstyle \rm
spin}\atop{{\scriptstyle \rm structures}\atop
\scriptstyle \nu,\nubar}}}
\omega(\nu,\nubar) &\int\limits_{\Met(\Sigma)}
Dg {1\over \scrN(g)}\int D\chi \cr
\int\limits_{\Map(\Sigma,M)} D\chi\int D\psi
&\cdot V_1\ldots V_N e^{-S[x,\psi;g,\chi]}\cr}
\eqno{(7.D.1)}
$$
The measures are constructed with respect to the $L^2$
norms on the corresponding spaces of sections of
$K^r\otimes\Kbar^{\rbar}$ for a suitable choice of
$r$, and $w(\nu,\nubar)$ is a weight
factor that depends upon the spin structure, to be
determined later.

\vfill\eject
\noindent
\Item{1)} {\it A quick derivation of the critical dimension}

Using our experience from the bosonic string, we may
collect the various determinant factors that arise
from integrating out $x$ and $\psi$
$$
\left(
{\Det'\Delta_{(0)}\over \int\nolimits_{\Sigma}d\mu_g}
\right)^{-D/2}\left(\Det\,\Delta_{(-1/2)}^+\right)^{D/2}
\eqno{(7.D.2)}
$$
as well as from the Faddeev-Popov determinants arising
from gauge fixing $\Diff(\Sigma)$ and local supersymmetry.
$$
\left(\Det'\Delta_{(-1)}^-\right)\left(\Det'\,
\Delta_{(-1/2)}^-\right)^{-1}
\eqno{(7.D.3)}
$$
The inverse power occurs for the local supersymmetry
determinant because local susy is Grassmann valued.
(We have ignored finite-dimensional determinant
contributions.)
Now, we have the following contribution to the Weyl
anomaly from each determinant
$$
\left\{
\matrix{
C_{(0)}\hfill &= &1\cr
C_{(-1/2)}^+\hfill &= &-1/2\cr
C_{(-1)}^-\hfill &= &13\cr
C_{(-1/2)}^-\hfill &= &\null\quad 11/2\,.\cr}
\right.
\qquad\qquad
\left\{
\matrix{
-D/2(-2C_{(0)})\hfill &= &D\cr
\null\quad D/2(-2C_{(-1/2)}^+)\hfill &= &D/2\cr
\null\quad 1(-2C_{(-1)}^-)\hfill &= &-26\cr
-\,1(-2\,C_{(-1/2)}^-)\hfill &= &11\cr}
\right.
\eqno{(7.D.4)}
$$
Hence the total Weyl scaling factor of the determinant
parts is ${3\over 2}\,(D-10)$, and the critical
dimension is $D=10$.
The super-Weyl symmetry (item 5) in (7.C.4) 
also has an anomaly, and one
can show that it cancels in $D=10$.

\bigskip\noindent
\Item{2)} {\it A quick look at super-moduli}

Later on, we shall carry out a careful study of the
space of {\it super-geometries}, which is parametrized by
$(g_{mn},\chi_m^\sigma)$.
For the time being, let us take a quick approach that
will suffice for the study of free strings.

Given a metric $g$, and local complex coordinates $z$,
$\zbar$, with $g=2g_{z\zbar}\vert dz\vert^2$, let us
analyze the action of super-Weyl and supersymmetry on
$\chi_m^\pm$:
$$
\left\{
\eqalign{
\delta\chi_z^+ &=-2\nabla_z \zeta^++\delta\lam^-\cr
\delta\chi_{\zbar}^+ &=-2\nabla_{\zbar}\zeta^+\cr}
\right.
\eqno{(7.D.5)}
$$
and similarly for their complex conjugates.
We see that by a super-Weyl transformation, we may
always set $\chi_z^+=0$.
Locally, we may also set $\chi_{\zbar}^+=0$, but
globally, there will be obstructions, which are most
easily identified when $\delta\chi$ is parametrized by
$$
\left\{
\eqalign{
\delta\chi_{\zbar}^+ &=\Range\,\nabla_{\zbar}^{(-1/2)}
 \oplus\Ker\,\nabla_z^{(-3/2)}\cr
\delta\chi_z^- &=\Range\,\nabla_z^{(1/2)}\oplus
  \Ker\,\nabla_{\zbar}^{(3/2)}\,\,.\cr}
\right.
\eqno{(7.D.6)}
$$
The space $\Ker\,\nabla_{\zbar}^{(3/2)}$ is spanned by
holomorphic $3/2$ differentials.

Odd moduli are defined as the Grassmann odd
coordinates that parametrize those parts of $\chi$
that are in the quotient of $\delta\chi$ by super-Weyl and
supersymmetry:
$$
\left\{
\eqalign{
\delta m_\kap &=(\delta\chi_z^-,\phi_\kap)
  \qquad \phi_\kap\in \Ker\,\nabla_{\zbar}^{(3/2)}\cr
\delta\mbar_\kap &=(\delta\chi_{\zbar}^+,\phi_\kap^*)
\qquad \phi_\kap^*\in\Ker\,\nabla_z^{(-3/2)}\,\,.\cr}
\right.
\eqno{(7.D.7)}
$$
The number of odd moduli is given by the
dimension of $\Ker\,\nabla_{\zbar}^{(3/2)}$:
$$
\dim\,\Ker\,\nabla_{\zbar}^{(3/2)}=
\left\{
\Biggl\{
\matrix{
0 &\qquad h=0\hfill\cr
\noalign{\medskip}
0 &\qquad h=1, \hbox{ even }\nu\hfill\cr
\noalign{\medskip}
1 &\qquad h=1, \hbox{ odd }\nu\hfill\cr
\noalign{\medskip}
2h-2 &\qquad h\Ge 2\,\,.\hfill\cr}
\right.
\eqno{(7.D.8)}
$$
Here, $\nu$ denotes the spin structure of $\chi$.

Later on, we shall show that, together with the
even moduli $m_k$, $\mbar_k$, 
the odd moduli $m_\kap$, $\mbar_\kap$ 
parametrize a so-called
{\it super-moduli space} $s\scrM_k$, with supermoduli 
coordinates $(m_k,m_\kap;\mbar_k,\mbar_\kap)$.
Here, $k=1,\ldots,\dim\,\Ker\,\nabla_{\zbar}{}^{(2)}$
and
$\kap=1,\ldots,\dim\,\Ker\,\nabla_{\zbar}{}^{(3/2)}$.
This space may be defined as the quotient of the space
of all $\{e_m^a,\chi_m^\sigma\}$, denoted by
$s\Met(\Sigma)$, by the group of superdiffeomorphisms
$s\Diff(\Sigma)=\{\Diff(\Sigma)$, local susy $\}$, of
super-Weyl transformations
$s\Weyl(\Sigma)=\{\Weyl(\Sigma)$, super $\delta\lam\}$
and local $O(2)$ frame rotations $\Lorentz(\Sigma)$:
$$
s\scrM_h=s\Met(\Sigma)/(s\Diff(\Sigma)\rtimes
s\Weyl(\Sigma)\times\Lorentz(\Sigma))
\eqno{(7.D.9)}
$$

\vfill\eject
\noindent
\Item{\bf E)} {\bf Super-Virasoro Algebra and Physical
Spectrum}

Having clarified the r\^{o}le of $\chi_m^\sigma$, we
now apply the results obtained in \S{D} to the free
string with $\Sigma$ of cylindrical topology.
Using $\Diff(\Sigma)$, local supersymmetry, Weyl and
super-Weyl invariance, in the critical dimension, we
may choose a flat metric on $\Sigma$, $g=2\vert dz\vert^2$
with vanishing $\chi_m^\sigma=0$.
However, just as was the case in quantizing the free
bosonic string, we must retain the constraints arising
from the integration over $g$ and $\chi$.
They are given by
$$
\left\{
\eqalign{
T(z) &=-\half\,\partial_z x\cdot\partial_z x+\half
\,\partial_z\psi_+\cdot\psi_+\cr
G(z) &=\psi_+\cdot\partial_z x\cr}
\right.
\eqno{(7.E.1)}
$$
along with their complex conjugates $\Tbar$, $\Gbar$.
Their OPE's are easily obtained.
$$
\eqalign{
T(z)T(w) &\sim {c/2\over (z-w)^4}+{2\over (z-w)^2}
  T(w)+{1\over z-w}\partial_w T(w)\cr
T(z) G(w) &\sim {3/2\over (z-w)^2}G(w)+{1\over
  z-w}\partial_w G(w)\cr
G(z)G(w) &\sim {2c/3\over (z-w)^2}+{2\over
z-w}T(w)\,\,.\cr}
\eqno{(7.E.2)}
$$
Here $c=3/2\,D$.
The mode expansion must be carried out with care
here, since we must distinguish between Ramond and
Neveu-Schwarz sectors.
$$
\eqalign{
&T(z)=\sum\limits_{m\in\dbZ}L_m z^{-m-2}\cr
&G(z)=\sum\limits_{r}G_r z^{-r-3/2}\cr}
\qquad\lower10pt\hbox{$\cases{
r\in \half+\dbZ &$\underNS$\cr
\noalign{\medskip}
r\in\dbZ &$\underR\,\,.$\cr}$}
\eqno{(7.E.3)}
$$
The (super) algebra formed by $L_m$ and $G_r$ is the
{\it $N=1$ super-conformal} or {\it super-Virasoro
algebra}, with the following structure relations
$$
\eqalign{
[L_m,L_n] &=(m-n)L_{m+n}+{c\over
12}\,m(m^2-1)\delta_{m+n,0}\cr
[L_m,G_r] &=\left(\half\,m-r\right)G_{r+m}\cr
\{G_r,G_s\} &=2 L_{r+s}+{c\over 3}(r^2-\eps)
  \delta_{r+s,0}\cr}
\eqno{(7.E.4)}
$$
where $\eps=0$ for $\underR$ and $\eps=1/2$ for $\underNS$.
(Notice that $G_r$ and $G_s$ must be both either
$\underR$ or $\underNS$!)
It will be convenient in the sequel to have the
expressions for $L_n$ and $G_r$ in terms of the modes
$x_n^\mu$, $d_n^\mu$ and $b_r^\mu$.
One readily finds by combining (7.E.1) and (7.E.3):
$$
\eqalignno{
m\not=0\qquad
&\cases{
L_m =\half\sum\limits_{n}x_{m-n}\cdot x_n+\half
  \sum\limits_{r}r b_{m-r}\cdot b_r &$\quad\underNS\,\,.$\cr
\noalign{\medskip}
L_m=\half\sum\limits_{n}x_{m-n}\cdot x_n+
  \half \sum\limits_{n} nd_{m-n}\cdot 
  d_n &$\quad\underR\,\,.$\cr} &(7.E.5)\cr
\noalign{\bigskip}
&\cases{
L_0=\half\,p^2+\sum\limits_{n=1}^\infty x_{-n}\cdot
x_n+
\sum\limits_{r\in1/2+\dbZ^+}r 
  b_{-r}\cdot b_r &$\underNS\,\,.$\cr
\noalign{\medskip}
L_0=\half\,p^2+\sum\limits_{n=1}^\infty(x_{-n}\cdot
x_n+nd_{-n}\cdot d_n) &$\underR\,\,.$\cr}&(7.E.6)\cr
\noalign{\medskip}
&\cases{G_r=\sum\limits_{m}x_{-m}\cdot b_{r+m}
  &\qquad $\underNS \quad r\in\half+\dbZ$\cr
\noalign{\medskip}
G_r=\sum\limits_{m}x_{-m}\cdot d_{r+m}
  &\qquad $\underR\qquad r\in\dbZ\,\,$.\cr}&(7.E.7)\cr}
$$

The physical state conditions are enforced separately
in the Ramond and Neveu-Schwarz sectors and may be
formulated separately for left and right movers at
fixed momentum.
We have the following definitions for physical states
in the left moving sector.
(An analogous definition holds for the right-moving
sector.)

\medskip\noindent
{\it Ramond} sector

A left-moving state $\left.\vert\psi;k;R\right>_L$ 
is physical of momentum $k$ in the Ramond sector 
$(\left.\vert\psi;k;R\right>_L\in\scrF_{R,k}^{\phys})$ if

\medskip\noindent
$$
\matrix{\displaystyle
{\scriptstyle\bullet} 
  &\left.\vert\psi;k;R\right>_L\in\scrF_k\otimes\scrF^R\hfill
  &\quad\hbox{for some $k\in\dbR^{10}$\kern4.5 true cm}\cr
\noalign{\bigskip}
{\scriptstyle\bullet}
&L_n\left.\vert\psi;k;R\right>_L=G_n
\left.\vert\psi;k;R\right>=0\hfill
  &\quad\hbox{for all $n\in\dbN$.}\kern4.5 true cm\hfill\cr}
\raise0pt\hbox{(7.E.8)}
$$

\medskip\noindent
and similarly $\scrFtil_{R,k}^{\phys}$ is defined from
$\scrFtil_k\otimes\scrFtil^R$.

\bigskip\noindent
{\it Neveu-Schwarz}

A left-moving state $\left.\vert\psi;k;\NS\right>_L$ 
is  physical of momentum $k$
in the Neveu-Schwarz sector
$(\left.\vert\psi;k;\NS\right>_L\in\scrF_{\NS,k}^{\phys})$
if
$$
\matrix{\displaystyle
{\scriptstyle\bullet}
&\left.\vert\psi;k;\NS\right>_L\in\scrF_k
  \otimes\scrF^{\NS}\hfill
&\quad \hbox{for some $k\in\dbR^{10}$\kern5.0 true cm}\cr
\noalign{\bigskip}
{\scriptstyle\bullet}
&\left(L_n-\half\,a\delta_{n,0}\right)\left.\vert\psi;
k;\NS\right>=0 &\quad \hbox{for all $n\in\dbN$\kern5.0
true cm}\hfill\cr
\noalign{\bigskip}
{\scriptstyle\bullet}
&G_r\left.\vert\psi;k;\NS\right>=0 \hfill
&\quad \hbox{for all $r\in\dbZ+\half$\kern5.0 true cm}
  \hfill\cr}
\raise0pt\hbox{(7.E.9)}
$$

\medskip\noindent
and similarly $\scrFtil_{\NS,k}^{\phys}$
is defined from $\scrFtil_k\otimes\scrFtil^{\NS}$.
{}From the Fock spaces of physical states,
$\scrF_{R,k}^{\phys}$ and $\scrF_{\NS,k}^{\phys}$,
we construct the Fock spaces of open and closed
physical strings, in analogy with (7.B.1) and (7.B.2):
$$
\eqalign{
\scrF_{\Open}^{\phys} &=\mathop{\oplus}\limits_{k\in
\dbR^{10}}
\left(\scrF_{\NS,k}^{\phys}\oplus\scrF_{R,k}^{\phys}\right)\cr
\scrF_{\Closed}^{\phys}
&=\mathop{\oplus}\limits_{k\in\dbR^{10}}\left(\scrF_{\NS,k}
^{\phys}\oplus\scrF_{R,k}^{\phys}\right)\otimes
\left(\scrFtil_{\NS,k}^{\phys}\oplus
\scrFtil_{r,k}^{\phys}\right)\,\,.\cr}
\eqno{(7.E.10)}
$$
In complete analogy with the bosonic theory, we have a

\Proclaim{No Ghost Theorem.}
$\scrF_{\Open}^{\phys}$ and $\scrF_{\Closed}^{\phys}$ have
non-negative norm for $D=10$, $a=1$.
The positive definite Hilbert space of the RNS string
is obtained by taking the quotient by the null space,
which we denote by $\scrF^{\Null}$:
$$
\eqalign{
\scrF_{\Open}^+
&=\scrF_{\Open}^{\phys}/\scrF_{\Open}^{\Null}\cr
\scrF_{\Closed}^+ &=\scrF_{\Closed}^{\phys}/
\scrF_{\Closed}^{\Null}\cr}
\eqno{(7.E.11)}
$$
\finishproclaim

\noindent
We shall not prove this theorem here.

\bigskip\noindent
\Item{\bf F)} {\bf The spectrum of 
physical states at low mass}

In the {\it Ramond sector}, the ground state obeys
$x_n^\mu,d_n^\mu \left.\vert0,k;\alpha;R\right>=0$
for all $n\in\dbN$ as well as the physical state
condition from (7.E.8)
$$
G_0 \left.\vert0,k;\alpha;R\right>=0\eqno{(7.F.1)}
$$
Since $L_0=G_0^2$, the constraint by $L_0$ is
automatically satisfied; similarly the constraints by
$G_n$, $L_n$, $n\Ge 1$ are automatically satisfied.
Denoting the generators $d_0^\mu$ of the
10-dimensional Clifford algebra by standard Dirac
notation $d_0^\mu={1\over\sqrt{2}}
\Gamma^\mu$, we find that the $G_0$
condition amounts to the fact that the Ramond ground
state satisfies the $10$-dimension massless Dirac
equation:
$$
\Gamma_\mu k^\mu\left.\vert 0,k;\alpha;R\right>=0\,\,.
\eqno{(7.F.2)}
$$
This equation on the Ramond ground state confirms once
more that it is a fermion.
Higher mass states transform under higher spinor
representations of $\Spin(1,9)$, and are all {\it
fermions} as well.

In the {\it Neveu-Schwarz sector}, the ground state
obeys
$x_n^\mu,\, b_r^\mu \left.\vert 0,k;\NS\right>=0$ for
all $n\in\dbN$, $r\in\half+\dbZ^+$ as well as the
physical state condition from (7.E.9)
$$
\left(L_0-\half\right)
\left.\vert0,k;\NS\right>=0\eqno{(7.F.3)}
$$
The constraints by $L_n$, $G_r$, $n,r\Ge\half$ are
again automatically satisfied.
These conditions imply that $k^2=1$, i.e. $M^2=-1$, so
that this state is a {\it tachyon}! 
It is also a Lorentz scalar and thus a boson.
The first excited state in the NS sector is obtained
by applying $b_{-1/2}^\mu$, 
$\left\vert\zeta,k;\NS\right>\equiv\zeta_\mu b_{-1/2}^\mu
\left.\vert0,k;\NS\right>$
and obeys the constraints
$$
\left(L_0-\half\right),L_1 \left.
\vert\zeta,k;\NS\right>=0 \eqno{(7.F.4)}
$$
We find $k^2=0$ and $\zeta\cdot k=0$, so 
$\left.\vert\zeta,k;\NS\right>$
is a {\it massless vector state}, such as we
need for Yang-Mills particles.
The transversality condition $\zeta\cdot k=0$ is also what
we shall need for gauge invariance.
It implies that the negative norm state is eliminated,
and because $k^2=0$ that one further state is null.
Thus, we are left with $8$ components with positive
norm, just as expected for a gauge particle.
Higher mass states transform under higher tensor
representations of $\SO(1,9)$, and are all bosons as well.

Let us note that by using lightcone gauge, we may
construct all positive norm states by applying only
the {\it transverse operators}
$$
x_n^i,\quad
d_n^i,\quad
b_r^i
\qquad\qquad i=1,\ldots,8\eqno{(7.F.5)}
$$
to the ground states in each respective sector for
$n,r<0$.
This method in particular allows for an easy counting
of the number of physical states modulo null states,
at each mass level (see Problem Set \#9).

Actually, we have looked only at the left-moving
sector of the Hilbert space.
This suffices to construct the Hilbert space of the
{\it open strings}.
To summarize, we have found that
this sector contains a $\NS$ tachyon, a
massless Dirac spinor and a massless vector.
Adding Chan-Paton rules to the open string, the single
vector may be promoted to a Yang-Mills gauge field
multiplet under the Chan-Paton gauge group.
Similarly, the Dirac spinor (and the tachyon) will
form multiplets under this group.
We shall work out this construction for Type I strings
shortly.

For {\it closed strings}, we must put left and right
moving sectors together, in order to obtain the full
spectrum, with the condition that left and right
momenta should match $p_L=p_R$, i.e.
$x_0^\mu=\xtil_0^\mu$.
We have four sectors:
  
\medskip\noindent
{\it $\NS$-$\NS$: bosons.}

\Item{$\scriptstyle\bullet$}
The ground state $\left.\vert0,k;\NS\right>_L\otimes
  \left.\vert0,k;\NS\right>_R$ is a tachyon with mass
squared equal to  $M^2=-1$.

\smallskip
\Item{$\scriptstyle\bullet$}
The massless states are
$\Eps_{\mu\nu}(k) 
b_{-1/2}^\mu\btil_{-1/2}^\nu \left.\vert0,k;\NS\right>_L
\otimes\left.\vert0,k;\NS\right>_R$ and contain

\ItemItem{}\qquad
the graviton (symmetric $\mu$, $\nu$),

\ItemItem{}\qquad the anti-symmetric tensor (anti-symmetric
  $\mu$, $\nu$),

\ItemItem{}\qquad
the dilaton (tracepart).

\smallskip
\Item{$\scriptstyle\bullet$}
massive excited states.

\medskip\noindent
{\it $\R$-$\NS$ (and $\NS$-$\R$): fermions.}

\Item{}The Ramond sector always has $\mass^2\Ge 0$, and thus
no tachyons occur in the $\R$-$\NS$ and $\NS$-$\R$
sectors.

\Item{$\scriptstyle\bullet$}
The ground state 
$\left.\vert0,k,\alpha;\R\right>_L\otimes
\zeta_\mu(k)\btil_{-1/2}^\mu\left.\vert0,k;\NS\right>_R$
is massless.
This state is reducible into a spinor $\lam$ called
the {\it dilatino} and a
spinor-vector, called the {\it gravitino}. 

\Item{$\scriptstyle\bullet$}
massive excited states

\medskip\noindent
{\it $\R$-$\R$: more bosons.}

\Item{$\scriptstyle\bullet$}
The ground state
$\left.\vert0,k,\beta;\R\right>_L\otimes
  \left.\vert0,k,\alpha;\R\right>_R$ is massless.

\Item{}
These states decompose into a sum of tensor
representations of $\SO(1,9)$, that may be obtained as
antisymmetrized products of the vector representation.
In other words, these states transform as differential
forms of degree $0,\ldots,10$.

\Item{$\scriptstyle\bullet$}
massive excited states.

\medskip
We shall establish in \S{X} that the $\RNS$ spectrum
(after the $\GSO$ projection has been carried out in
\S{G}) exhibits space-time supersymmetry, under which
bosons and fermions are interchanged.
We shall then find that the dilaton and the dilatino
on the one hand, and the graviton and the gravitino on
other other hand, form supersymmetry multiplets.

\bigskip\noindent
\Item{\bf G)} {\bf The GSO Projection, Space-time
Supersymmetry}

By combining the $\NS$ and $\R$ sectors of the $\RNS$
string, we have obtained a string theory with both
bosons and fermions, with massless graviton,
anti-symmetric tensor, dilaton and anti-symmetric tensor
$\R$-$\R$ states which are all bosons, plus 
gravitino and dilatino states which are fermions.
This theory is still inconsistent, for two reasons.
First, in the $\NS$ sector, there is still a {\it
tachyon} present, which gives rise to 
the same type of problems as the tachyon
did in the bosonic string.
Second, there is a {\it massless gravitino}, which can
interact consistently only by coupling to a conserved
supercurrent.
(The gravitino shares many properties 
with a gauge field, which must couple to a conserved current.)
But conservation of a supercurrent means that the
theory must exhibit invariance under supersymmetry
transformations.
Clearly, the $\RNS$ theory is not supersymmetric
as it stands. 
The easiest way to see this is by
noticing that the tachyon has no
fermionic counterpart.
On the other hand, if the theory could be made
supersymmetric, the tachyon problem would be
eliminated right away and then the gravitino could be
coupled to a conserved supercurrent, thus eliminating
both of the above problems.

Gliozzi-Scherk and Olive (1976) proposed a truncation
of the $\RNS$ string theory that produces a spectrum
with space-time $(\dbR^{10})$ supersymmetry.
We shall introduce the $\GSO$ projection first at the
level of the free string Fock space, and later on show 
how to enfore it in the functional integral
formulation.
It will turn out that the $\GSO$ projection
corresponds there to carrying out a summation over all
spin structures of $\Sigma$, with certain weight
factors.
The weight factors will be arranged in such a way that
the amplitudes will be invariant under the action of
the modular group of the worldsheet $\Sigma$.
Such spin structure weight factors were already
included in the functional integral formulation in
(7.D.1) on general grounds.

The $\GSO$ projection is defined by using the {\it
worldsheet fermion number operator} $F$ in the form
$(-1)^F$. 
This operator has the property
$$
\left\{(-1)^F,\psi_\sigma^\mu(z,\zbar)\right\}=0
\eqno{(7.G.1)}
$$
both in the $\R$ and $\NS$ sectors.
This property defines the operator $(-1)^F$ only up to
a factor of $\pm 1$, which one may think of as the
assignment of the eigenvalue of $(-1)^F$
 on the string ground state in each sector.
We shall now provide a careful definition of
$(-1)^F$ in each sector.

In the {\it Neveu-Schwarz sector}, the renormalized
worldsheet 
fermion number operators in the left- and right-moving
sectors are given by
$$
F_{\NS}\equiv\sum\limits_{r\in 1/2+\dbZ^+}b_{-r}\cdot
b_r\qquad\qquad \Ftil_{\NS}\equiv\sum\limits_{r\in
1/2+\dbZ^+}\btil_{-r}\cdot\btil_r\,\,.
\eqno{(7.G.2)}
$$
and the assignment of $(-1)^F$ on the $\NS$ ground
state is $-1$.
Thus, we may represent $(-1)^F$ by
$$
(-1)^F\equiv(-1)^{F_{\NS}-1}\qquad\qquad
(-1)^{\Ftil}\equiv(-1)^{\Ftil_{\NS}-1}\,\,.
\eqno{(7.G.3)}
$$
The {\it $\GSO$ projection} in the Fock space of the
$\NS$ sector is then given by
$$
\eqalignno{
\scrF_{\NS,\GSO} &\equiv\left\{\left.\vert
\psi\right>_L\in\scrF^{\NS},(-1)^F\left.\vert\psi\right>_L
=\left.\vert\psi\right>_L\right\}&(7.G.4)\cr
\noalign{\hbox{and similarly}}
\scrFtil_{\NS,\GSO}
&\equiv\left\{\left.\vert\psi\right>_R\in\scrFtil^{\NS},
(-1)^{\Ftil}\left.\vert\psi\right>_R=
\left.\vert\psi\right>_R\right\}\,\,.&(7.G.5)\cr}
$$
Alternatively, $\scrF_{\NS,\GSO}$ consists of states
obtained from the ground state in the $\NS$ sector by
applying only an {\it even} number of $b$ operators.
(Similarly for $\scrFtil_{\NS,\GSO}$).
It is straightforward to see that the $\GSO$
projection is compatible with the physical state
conditions, since
$$
\left[(-1)^F,L_m\right]=0\qquad\qquad
\left\{(-1)^F,G_r\right\}=0\,\,.
\eqno{(7.G.6)}
$$
Thus, we define $\scrF_{\NS,\GSO}^{\phys}$ and
$\scrFtil_{\NS,\GSO}^{\phys}$ by imposing physical
state conditions on $\scrF_{\NS,\GSO}$ and 
$\scrFtil_{\NS,\GSO}$.

The tachyon ground state (as well as any state
obtained by applying an even number of $b$-operators
to it) is projected out by $\GSO$, and the lowest mass
state in $\scrF_{\GSO,\NS}^{\phys}$ is massless.
It consists of a vector of $\SO(1,9)$:
$$
b_{-1/2}^\mu\left.\vert0,k;\NS\right>
$$ 
with $8$ physical, positive norm components.
The first excited states have the form
$$
b_{-1/2}^{\mu_1}b_{-1/2}^{\mu_2}b_{-1/2}^{\mu_3}\left.
\vert0,k;\NS\right>\qquad
b_{-3/2}^\mu\left.\vert0,k;\NS\right>
\eqno{(7.G.7)}
$$
and have mass $M^2=2$.
(There are $8.7.6/6=56,+8=64$ such physical states.)

We now proceed to define the $\GSO$ projection in the
{\it Ramond sector}.
Supersymmetry in $\dbR^{10}$ will require that we have
equal numbers of bosonic and fermionic physical states
at each mass level.
(We assume here that supersymmetry is manifest, and
not spontaneously broken.)
The original Ramond sector of the $\RNS$ string has a
massless Dirac spinor ground state.
In $D=10$, such a spinor has $2^{D/2}=2^5=32$
components, which does not match the $8$ bosonic
components of the $\NS$ sector.
To have equality, we ought to reduce the number of
components of the spinor from $32$ to $8$, of course,
in a way consistent with $\Spin(1,9)$ invariance!

For groups of the form $\Spin(1,D-1)$, with $D$ even, we
have the following reducibility criteria for various
dimensions $D$.
The Clifford algebra is defined by $\{\Gamma^\mu,
\Gamma^\nu\}=2\eta^{\mu\nu}$.
The chirality matrix $\Gammabar$, defined by
$$
\Gammabar\equiv\Gamma^0\Gamma^1\ldots\Gamma^{D-1}\qquad
\qquad\Gammabar^2=(-1)^{D/2-1}
\eqno{(7.G.8)}
$$
commutes with all the generators of $\Spin(1,D-1)$ in
the Dirac spinor representation --- which we denote
by $S$.
Thus, $S$ may be reduced to a sum of {\it Weyl
spinors} $S=S_+\otimes S_-$, where both $S_{\pm}$ are
of dimension $2^{D/2-1}$.

For $D=2,4\pmod{8}$, there is an independent
reducibility criterion on Dirac spinors, which
decomposes $S$ into a sum of two {\it Majorana
spinors} $S=S_M+iS'_M$.
A Majorana spinor is of dimension $2^{D/2}$,
but is defined to be {\it real in some
basis}:
$$
S_M^*=BS_M\qquad\hbox{with}\qquad
\Gamma^{\mu*}=B\Gamma^\mu B^{-1}
\eqno{(7.G.9)}
$$
A spinor may be reduced to {\it Majorana-Weyl} spinors
provided the Weyl and Majorana conditions are
compatible (i.e. $\Gammabar$ and $B$ commute).
This is possible only in $D=2\pmod{8}$.
A Majorana-Weyl spinor is a real spinor
of dimension $2^{D/2-1}$.
In $D=10$, we see that the Majorana-Weyl condition
reduces the dimensionality of the $R$-ground state
precisely to an $8$ component spinor, just as is
needed for supersymmetry!

In the {\it Ramond sector}, the $\GSO$ projection is
defined in terms of the worldsheet fermion number
operator as well, with the following normalization.
$$
\matrix{
(-1)^F &\equiv\hfill \Gammabar(-1)^{F_R}
  &\qquad\qquad (-1)^F &=\hfill \Gammabar(-1)^{\Ftil_R}\cr
\noalign{\medskip}
F_R &\equiv\hfill \sum\limits_{n=1}^\infty d_{-n}\cdot
d_n
&\qquad\qquad \Ftil_R &=\hfill \sum\limits_{n=1}^\infty
  \dtil_{-n}\cdot\dtil_n\cr}
\eqno{(7.G.10)}
$$
The $\GSO$ projection in the Fock space of the $R$
sector involves a choice of chirality of the $R$
ground state in each sector (left movers and right
movers).
Let us denote these chiralities by $\chi_L$ and
$\chi_R$, respectively, with values $\pm$ each.
We now define
$$
\vbox{
\eqalignno{
\scrF_{\GSO}^{\R}(\chi_L)
&\equiv\left\{\left.\vert\psi,R\right>_L\in\scrF^R;
\qquad (-1)^F\left.\vert\psi,R\right>_L=\chi_L\left.
  \vert\psi;R\right>_L\right\}\cr
\noalign{\hbox{and}}
\scrFtil_{\GSO}^{\R}(\chi_R) &\equiv\left\{
  \left.\psi;R\right>_R\in\scrFtil^R;\qquad
(-1)^{\Ftil}\left.\vert\psi;R\right>_R= 
\chi_R\left.\vert\psi;R\right>_R\right\}\,\,.\cr}
}
\null\kern-37pt\raise20pt\hbox{(7.G.11)}
$$
This means that, given the chirality of the ground
state, the chirality of the higher states are
determined by the $\GSO$ projection.

\bigskip\noindent
1) {\it Chirality and Massive States in the Ramond
Sector}

The above $\GSO$ projection in the Ramond sector
selects the chirality of the massless ground state.
But it also appears to select the chirality of massive
states, which would appear to be 
in contradiction with the fact that
they are massive.
The resolution of this apparent contradiction lies in
the fact that opposite chirality components of a
massive state arise in different ways.
Consider the first excited level for left movers; 
it has states (in the lightcone gauge)
$$
x_{-1}^i\left.\vert0,k;\alpha;R\right>_L\qquad\qquad
d_{-1}^i\left.\vert0,k;\beta;R\right>_L
\eqno{(7.G.12)}
$$
The $\GSO$ condition (for example $\chi=+$) requires that
$$
\eqalign{
\Gammabar x_{-1}^i\left.\vert0,k;\alpha;R\right>_L
&=+ x_{-1}^i\left.\vert0,k;\alpha;R\right>_L\cr
\Gammabar d_{-1}^i\left.\vert0,k;\beta;R\right>_L &=-
d_{-1}^i\left.
\vert0,k;\beta;R\right>_L\cr}
\eqno{(7.G.13)}
$$
Hence these two states have opposite chiralities and
together form the $16$
components of a massive Majorana spinor!
The same phenomenon persists at higher mass levels,
and is responsible for correctly filling out the
chirality components of all massive states.

\bigskip\noindent
2) {\it Space-time supersymmetry}

Now that we have defined the $\GSO$ projection in all
sectors of the $\RNS$ string, it remains to
investigate whether the spectrum thus obtained is
supersymmetric.
To carry out this analysis, we would need to construct
the supersymmetry operators, and this requires
technology that we shall develop only in \S{X}.
Instead, we can check here that the $\GSO$ projection
passes an elementary test necessary to have a
supersymmetric spectrum.
It is an elementary consequence of the structure
relations of supersymmetry that the supercharges commute
with the $\mass^2$ matrix.
Thus, the number of bosonic and fermionic states at
each mass level must be the same.
We now show that this is indeed the case for the
$\RNS$ spectrum, after $\GSO$
projection.

To do so, we count the number of fermionic states (in
the $\R$ sector) and compare with the number of bosonic
states (in the $\NS$ sector).
To count states, it suffices to use the lightcone
construction, in terms of transverse operators $x^i$,
$b^i$, $d^i$, with $i=1,\ldots 8$,
and we calculate the generating function
in each sector.
In the Ramond sector, we have
$$
\eqalign{
f_{\R}(q) &=8\Tr q^{2M^2}=\sum\limits_{n=1}^\infty
d_R(n)q^{2n}\cr
&=8\prod\limits_{m=1}^\infty(1-q^{2m})^{-8}(1+q^{2m})^8\cr}
\eqno{(7.G.14)}
$$
Here, the factor $(1-q^{2m})^{-8}$ arises from the $x^i$
oscillators 
and the factor $(1+q^{2m})^8$ from the $d^i$ modes. 
The prefactor of $8$ arises from the eight-fold
degeneracy of the $R$ ground state.
Similarly, in the Neveu-Schwarz sector, we have
$$
f_{\NS}(q)=\prod\limits_{m=1}^\infty
(1-q^{2m})^{-8}{1\over 2q}
\biggl\{\,\prod\limits_{n=1}^\infty(1+q^{2n-1})^8
-\prod\limits_{n=1}^\infty(1-q^{2n-1})^8\biggr\}
\eqno{(7.G.15)}
$$
Using the product formulas for Jacobi
 $\vartheta$-functions, and
the equality
$$
\vartheta_2(0\vert\tau)^4+\vartheta_4(0\vert\tau)^4=
\vartheta_3(0\vert\tau)^4\eqno{(7.G.16)}
$$
with $q=e^{2\pi i\tau}$, we see that
$f_{\R}(q)=f_{\NS}(q)$.
Thus, at each mass level, the number of bosonic and
fermionic states agrees.

Notice that we have established evidence for the presence
of a space-time supersymmetry in each sector, 
left-moving and right-moving independently.
It is conventional to denote the number of independent
supersymmetries by the letter $N$.
In this sense, the open $\GSO$ projected $\RNS$ string
has $N=1$, while the closed $\GSO$ projected $\RNS$
string has $N=2$.

\bigskip\noindent
3) {\it Summation over spin structures}

We are now ready to exhibit the relation between the
$\GSO$ projection of the states in string transition
amplitudes to any loop order, and summation over spin
structures of string worldsheets.
We concentrate on processes involving only $\NS$-$\NS$
{\it external states}, which are created by vertex operators
that are single-valued on the worldsheet.
(Vertex operators for $R$ states produce quadratic
branch cuts and will be introduced in \S{IX}.)
Of course, the internal states will include both the
$\R$ and the $\NS$ sectors.

We consider $1$-loop amplitudes for closed,  oriented
strings first, and choose a homology basis of
$1$-cycles $A$ and $B$.
We analyze the contribution from a given spin
structure. 
$$
\vbox{\epsfxsize=4.5in\epsfbox{fig2.eps}}
$$
The one loop amplitude may be {\it
factorized} and expressed as a sum over intermediate
 states, as shown in the diagram above.
In the functional integral, factorization is carried out by
integrating over all field values (for $x^\mu$
and $\psi_\sigma^\mu$) on the curve $A$.
Integration over all $\RNS$ fields $x^\mu$ and
$\psi_\sigma^\mu$, with given spin structure, will
give rise to the following structure.

The spin structure on the $A$-cycle determines the
boundary condition on $\psi_+^\mu$ (resp.
$\psi_-^\mu$) as either periodic (Ramond) or
anti-periodic (Neveu-Schwarz).
This information determines whether the state
$\left.\vert\eps\right>$ in the factorization is a
fermion $(\R)$ or a boson $(\NS)$.
In the full $1$-loop amplitude, both bosons and
fermions will contribute; thus, the spin structure on
the $A$-cycle must be summed over with unit
coefficients.

What is the physical significance of the spin structure
along a $B$-cycle?
Functional integration over worldsheet spinors
$\psi_+^\mu$ (or $\psi_-^\mu$) with {\it anti-periodic
boundary conditions} along the $B$-cycle produces a
sum over states $\left.\vert\eps\right>$ in the
factorization.
On the other hand, functional integration over spinors
$\psi_+^\mu$ (or $\psi_-^\mu$) with {\it periodic boundary
conditions} along the $B$-cycle produces a sum over
states $\left.\vert\eps\right>$ with an 
insertion of the operator $(-1)^F$ in the
factorization.
Here, $F$ stands for worldsheet fermion number, just
as we encountered in the definition of the $\GSO$
projection.
This result may be established via the usual
derivation of the Grassmann functional integral from
canonical quantization of spinors.
A summation over spin structures on the $B$-cycles
with coefficients $(\pm 1)$ produces the insertion of
the projection operator
$$
\half(1+(-1)^F)\eqno{(7.G.17)}
$$
in the summation over states $\left.\vert\eps\right>$
in the factorization, and this oprator precisely
enforces the $\GSO$ projection.
It is straightforward to extend the reasoning
presented above to the case of worldsheets of
arbitrary genus $h$.
We shall leave it as an exercise to the reader.

Just as the $\GSO$ projection had to be enforced
separately on left and right movers, we shall also
have to separate left and right movers in loop
amplitudes, using the chiral splitting theorem for the
$\RNS$ string (which generalizes the results of
\S{V}), and sum independently over spin structures
of left and right movers.

Finally, let us remark that we have shown that the
$\GSO$ projection is enforced by separate summation
over all spin structures of left and right movers with
weights $\pm 1$.
The actual sign choice for each spin structure must be
consistent with modular invariance of the full
amplitude.
There may be different consistent choices, yielding
different string theories, as we shall see in \S{H}.

\bigskip\noindent
\Item{\bf H)} {\bf Type II A, B Superstrings and Their
Spectra}

So far, we have analyzed the $\GSO$ projection in a
single sector of $\RNS$ string theory: either left- or
right-movers.
We now consider closed oriented string theories built
from $\GSO$ projected $\RNS$ strings.
We shall temporarily ignore the physical state
conditions, which may be easily imposed
independently.
There are 4 possible combinations, with the following
Hilbert spaces:
$$
\eqalign{
\scrF_{\GSO}^{\phys}(\chi_L;\chi_R) 
=\oplusop_{k\in\dbR^{10}} &\scrF_k\otimes
(\scrF_{\GSO}^{\NS}\oplus \scrF_{\GSO}^{\R}
(\chi_L))\cr
&\quad\otimes\scrFtil_k\otimes
  (\scrFtil_{\GSO}^{\NS}\oplus
  \scrF_{\GSO}^{\R}(\chi_R))\,\,.\cr}
\eqno{(7.H.1)}
$$
Now, a space-time parity operation reverses the
chirality of all spinors and in particular, of
the $R$ ground states: $\chi_L\to -\chi_L$,
$\chi_R\to-\chi_R$. 
Thus the string theory with assignment $(\chi_L,\chi_R)$
defines the same string theory as the one with
assignment $(-\chi_L,-\chi_R)$.
So, instead of 4 combiantions, we are left with only 2
physically inequivalent string theories:

\smallskip\noindent
{\bf Type \IIA :}\enspace $\chi_L=-\chi_R$
$$
\scrF_{\IIA}^{\phys}=\scrF_{\GSO}^{\phys}(+,-)\qquad
(\sim \scrF_{\GSO}^{\phys}(-,+))\,\,.
\eqno{(7.H.2)}
$$

\noindent
{\bf Type \IIB :}\enspace $\chi_L=\chi_R$
$$
\scrF_{\IIB}^{\phys}=\scrF_{\GSO}^{\phys}(+,+)\qquad
(\sim \scrF_{\SO}^{\phys}(-,-))\,\,.
\eqno{(7.H.3)}
$$
The Type $\IIA$ theory is parity invariant upon exchange
of left and right movers, while Type $\IIB$ is parity
violating.
One also says that {\it Type $\IIB$ is chiral}, since it
singles out one of the chiralities, while {\it Type $\IIA$
is non-chiral}.

To gain insight into these theories, we derive their
physical spectra, with the help of the lightcone gauge
formulation, in terms of the modes $x_{n}^i$, $b_{r}^i$,
$d_n^i$ with $i=1,\ldots,D-2=8$.
(See Problem Set \#9.)
We restrict attention first to massless states.
We begin by noticing that the $\GSO$ projection
eliminates the tachyon, but leaves the massless states
unchanged in the $\NS$ sector.
In the Ramond sector, the $\GSO$ projection selects
the chirality of the ground state, which is the only
state needed for massless string states.
We shall choose the chiralities to be $+$ in the left
moving sector and $-$ for Type $\IIA$ and $+$ for Type
$\IIB$ in the right moving sectors, respectively.
A list of massless states in each sector, together with
their standard identifications in terms of fields
familiar from quantum
field theory, is given in the table below.
In parentheses, we indicate the number of physical
degrees of freedom present in each state.

\def\spacedown{\noalign{\smallskip}}
\def\Spacedown{\noalign{\vskip1pt}}
$$
\vbox{\offinterlineskip
\hrule
\halign{\vrule# &\quad\hfil #\hfil
&\quad\vrule#\quad &\hfil #\hfil
&\quad\vrule#\quad &\hfil #\hfil &\quad\vrule#\cr
height6pt &\omit &&\omit &&\omit &\cr
&$L-R$ &&Type $\IIA$ &&Type $\IIB$ &\cr
height6pt &\omit &&\omit &&\omit &\cr
\noalign{\hrule height 1.5 pt depth0pt}
height6pt &\omit &&\omit &&\omit &\cr
&$\NS$-$\NS$ &&$b_{-1/2}^i\btil_{-1/2}^j\left.
  \vert0,k;\NS\right>_L\otimes
\left.\vert0,k;\NS\right>_R$\hfill && &\cr
height6pt &\omit &&\omit &&\omit &\cr
&\hbox{(bosons)} 
  &&$\matrix{\ssbullet (35) &G_{\mu\nu}
  &\hbox{(graviton)}\hfill\cr
\spacedown
  \ssbullet (28) &B_{\mu \nu} &\cr
\spacedown
  \ssbullet (\,\,\, 1)&\Phi &\hbox{(dilaton)}\hfill\cr}$\hfill
&&\hbox{idem}&\cr
height6pt &\omit &&\omit &&\omit &\cr
\noalign{\hrule}
height6pt &\omit &&\omit &&\omit &\cr
&$\R$-$\NS$ &&$\left.\vert0,k;\alpha;R\right>_L\otimes
     \btil_{-1/2}^i\left.\vert0,k;\NS\right>_R$\hfill && &\cr
height6pt &\omit &&\omit &&\omit &\cr
&\hbox{(fermions)} 
 &&$\matrix{\ssbullet (56) &\chi_\mu^\alpha
  &\hbox{(gravitinos) chirality $+$}\hfill\cr
\spacedown
  \ssbullet (\,\,\, 8) &\lambda_\alpha 
  &\hbox{(dilatinos) chirality $-$}\hfill\cr
\Spacedown \cr}$\hfill
&&\raise8pt\hbox{$\matrix{\hbox{idem} 
  &\hbox{chirality $+$}\cr
  &\hbox{chirality $-$}\cr}$}\hfill &\cr
% &&\raise13pt\hbox{idem\quad chirality $+$}&\cr
%& && &&\null\raise15pt\hbox{chirality $-$}&\cr
height6pt &\omit &&\omit &&\omit &\cr
\noalign{\hrule}
height6pt &\omit &&\omit &&\omit &\cr
&$\NS$-$\R$ &&
  $b_{-1/2}^i\left.\vert0,k;\NS\right>_L
  \otimes\left.\vert0,k;\alpha;R\right>_R$\hfill && &\cr
height6pt &\omit &&\omit &&\omit &\cr
&\hbox{(fermions)} 
  &&$\matrix{\ssbullet (56) &\chi_\mu^\alpha
  &\hbox{(gravitinos) chirality $-$}\hfill\cr
\spacedown
  \ssbullet (\,\,\, 8) &\lambda_\alpha 
  &\hbox{(dilatinos) chirality $+$}\hfill\cr
\Spacedown\cr}$\hfill
  &&\raise8pt\hbox{$\matrix{\hbox{idem} 
         & \hbox{chirality $+$}\cr
         &\hbox{chirality $-$}\cr}$}\hfill &\cr
height6pt &\omit &&\omit &&\omit &\cr
\noalign{\hrule}
height6pt &\omit &&\omit &&\omit &\cr
&$\R$-$\R$ &&$\left.\vert 0,k;\alpha;R\right>_L\otimes
  \left.\vert 0,k;\beta;R\right>_R$\hfill 
  &&$\left\vert0,k;\alpha;R\right>_L\otimes
    \left.\vert0,k;\beta;R\right>_R$\hfill &\cr
height6pt &\omit &&\omit &&\omit &\cr
&\hbox{(bosons)} 
  &&$\matrix{\ssbullet (\,\,\, 8) &A_\mu^{(1)}\cr
\spacedown
     \ssbullet (56) &A_{\mu \nu\rho}^{(3)}\cr}$\hfill
  &&$\matrix{\ssbullet (\,\,\,1) &A^{(0)} &\cr
\spacedown
    \ssbullet (28) &A_{\mu \nu}^{(2)} &\cr
\spacedown
    \ssbullet (35) &A_{\mu \nu\rho\sigma}^{(4)}
    &\kern-4pt\hbox{{\it self-dual $F$}}\cr}$\hfill &\cr
height6pt &\omit &&\omit &&\omit &\cr}
\hrule}
$$

The first entry that requires clarification is the
$\R$-$\NS$ (and $\NS$-$\R$) sector: the dilatino and
gravitino have opposite chiralities.
To see how this happens, consider the tensor roduct of
a spinor of definite chirality (say $+$ as in
$\R$-$\NS$) and a vector, which yields a field
$\chibar_\mu^\alpha$ of chirality $+$ with $64$
physical components.
It decomposes into irreducible spinors by
$$
\chibar_\mu^\alpha=\chi_\mu^\alpha+(\Gamma_\mu)^{\alpha\beta}
\lam_\beta
$$
where $\chi_\mu^\alpha$ is the gravitino field,
obeying $(\Gamma^\mu)_{\beta\alpha}\chi_\mu^\alpha=0$,
of chirality $+$, and $\lam$ is the dilatino of
chirality $-$, since $\{\Gammabar,\Gamma_\mu\}=0$.

The next entry that requires clarification is the
$\R$-$\R$ sector.
To see why the tensors that are indicated are the ones
that arise, we study the tensor product decomposition
properties of two spinors.
This may be done either on $\SO(1,9)$ spinors, or on
the reduced $\SO(8)$ spinors.
Notice that a Majorana-Weyl spinor of the Poincar\'e
group of $\SO(1,9)$ is
{\it irreducible} under its $\SO(8)$ stabilizer 
subgroup, where it is simply a Weyl spinor.

If $S_1$ and $S_2$ are two  Majorana-Weyl
spinors of {\it the same chirality}, then they reduce
to $\SO(8)$ Weyl spinors of the same chirality.
Their product decomposes into anti-symmetric tensors
of {\it even} rank
$$
S_1\otimes S_2=A^{(0)}\oplus A^{(2)}\oplus
A_+^{(4)}
\eqno{(7.H.5)}
$$
(For Weyl spinors, the components $A^{(6)}$, $A^{(8)}$
are dependent on $A^{(0)}$ and $A^{(2)}$,
respectively.)
$A_+^{(4)}$ stands for the part with self-dual field
strength, $F=dA^{(4)}$ with ${}^*F=F$.

If $S_1$ and $S_2$ are two Majorana-Weyl
spinors of {\it opposite chirality}, they reduce to
$\SO(8)$ Weyl spinors of opposite chirality.
Their product decomposes into anti-symmetric tensors
of {\it odd} rank
$$
S_1\otimes S_2=A^{(1)}\oplus A^{(3)}\,\,.
\eqno{(7.H.6)}
$$
(For Weyl spinors, the componens $A^{(5)}$ and $A^{(7)}$
are dependent on $A^{(3)}$ and $A^{(1)}$.)

The counting of the number of physical states for each
of the fields $A_\mu^{(1)}$, $A_{\mu \nu}^{(2)}$,
$A_{\mu \nu\kap}^{(3)}$ and $A_{\mu \nu\kap\rho}^{(4)}$
indeed confirms that they are all to be viewed as 
{\it gauge fields}, in fact, as differential forms
$A^{(p)}$, invariant under
$$
A^{(p)}\to A^{(p)}+dC^{(p-1)}
\eqno{(7.H.7)}
$$
where $C^{(p-1)}$ is a form of degree $p-1$.
Thus, they are all Abelian gauge fields.

\bigskip\noindent
{\it $D$-branes}

It can be shown that the fundamental strings are
neutral under these gauge fields, and do not carry
their quantum numbers.
A gauge field of the type $A^{(p)}$ couples naturally,
however, to an extended  object of space-time
dimension $p$ called a $D(p-1)$ brane.
By analogy with electromagnetism, one says that the
$D(p-1)$ brane carries the electric charge of
$A^{(p)}$.
By Poincar\'e duality of their field strength
$F^{(p)}=dA^{(p)}$, the field $*F^{(p)}\equiv
F^{(10-p)}$ may also naturally couple to
extended objects of space-time dimensions $8-p$
again called a $D(8-p)$-brane.
$$
\matrix{
A^{(1)} &\sim &D0 &\hbox{branes} &\cr
\noalign{\medskip}
A^{(2)} &\sim &D1 &\hbox{branes} &\hbox{(strings)}\cr
\noalign{\medskip}
A^{(3)} &\sim &D2 &\hbox{branes} &\hbox{(membranes)}\cr
\noalign{\medskip}
A^{(4)} &\sim &D3 &\hbox{branes} &\cr}
\eqno{(7.H.8)}
$$
Polchinski showed that $D$-branes indeed carry the
electric charges of these gauge fields.

\bigskip\noindent
{\it The massive spectra of Type \IIA and Type \IIB
coincide}.

We conclude this section on Type \IIA, B superstrings
with a discussion of their massive spectra.
The full spectra are obtained by tensor product of
left- and right-moving sectors as usual.
The spectrum of each sector (say left movers) is
identical to the spectrum of an $\scrN=1$ chiral open
superstring.
While its massless spinor states are chiral, its
massive states cannot be chiral.
Instead, we have already exhibited the chiral pairing
of states of opposite chirality previously.
For example, at the first excited level in the $R$
sector, we have
$$
x_{-1}^i\left.\vert 0,k;\alpha;R\right>\qquad\qquad\qquad
d_{-1}^i\left.\vert0,k;\beta;R\right>
\eqno{(7.H.9)}
$$
which are of opposite chirality and combine into a
representation of massive Majorana fermions.
Thus, the massive spectrum is insensitive to the choice
of chirality for the massless ground state.
As a result {\it the massive spectra of {\rm Type
$\IIA$} and {\rm Type $\IIB$} superstrings coincide}.
In other words, Type $\IIA$ and Type $\IIB$ superstrings
differ only in their massless spectrum.

In fact, one can show that upon compactifying Type
$\IIA$ on a circle of radius $R_A$ and Type $\IIB$ on
a circle of radius $R_B$, one obtains identical
theories, including the massless sector, when $R_A
R_B=\alpha'$.
This identity may be seen as a result from a
transformation between the two theories, called
$T$-duality, which we investigate in Problem Set \#10.

\bigskip\noindent
\Item{\bf I)} {\bf Type I Superstring}

The type I superstring is constructed from open and
closed strings, including orientable as well as
unorientable worldsheets, and with Chan-Paton
Yang-Mills charges on the open string end points.

We begin by reviewing the results from \S{II} on
Chan-Paton charges: the end points of open strings
carry a multiplet of charges, transforming under
representations $\Lam$ and $\Lambar$ of a compact
gauge group $G$.
These charges may be obtained by quantization of
Grassmann variables $\lam^i$ that move at the end
points of the open strings.
$$
\vbox{\epsfxsize=1.00in\epsfbox{fig3.eps}}
$$
For the open string excitation states to be precisely
those of Yang-Mills gauge states, it is necessary that
$\Lam\otimes\Lambar$ be equal to the adjoint
representation of $G$, possibly after imposing certain
reality and/or symmetrization rules.
In fact, there are only 3 series of solutions, given by
classical groups, and the followng rules:
$$
\matrix{
U(N)\hfill &\qquad \Lambar=\Lam^* 
  &\qquad \hbox{orientable strings}\hfill\cr
\noalign{\bigskip}
\SO(N)\hfill &\qquad \Lambar=\Lam 
  &\qquad \hbox{unorientable strings}\hfill\cr
\noalign{\bigskip}
\Sp(N)\hfill &\qquad \Lambar=\Lam 
  &\qquad \hbox{unorientable strings}\hfill\cr}
$$
(We shall examine the interplay between Chan-Paton
charges and orientability shortly.)

Next, we need to clarify the construction of string
theories with unorientable strings.
One defines the {\it worldsheet parity operator
$\Omega$} (sometimes also called the twist operator)
by its action on worldsheet coordinates:
$$
\left\{
\eqalign{
\Omega z &=-\zbar\cr
\Omega \zbar  &=-z\cr
\Omega^2 &=1\cr}
\right.
\eqno{(7.I.1)}
$$
The action of $\Omega$ is exhibited in the figure
below for an open string represented as a half
annulus, with boundary $B$.
$$
\vbox{\epsfxsize=2.50in\epsfbox{fig4.eps}}
$$
\medskip

We begin by considering the bosonic string.
Worldsheet parity is a symmetry of the bosonic string
action, when $x^\mu$ transforms as
$$
\Omega x^\mu(z,\zbar)=x^\mu(-\zbar,-z)\,\,.
\eqno{(7.I.2)}
$$
For {\it open strings}, we have the mode expansion
$$
x^\mu(z,\zbar)=q^\mu-i p^\mu\,\ln \vert z\vert^2+i
\sum\limits_{n\not=0}^\infty{1\over n}\,
x_n^\mu(z^{-n}+\zbar^{-n})\eqno{(7.I.3)}
$$
so that
$$
\Omega q^\mu=q^\mu;\quad
\Omega p^\mu=p^\mu;\quad
\Omega x_m^\mu=(-)^m x_m^\mu\,\,.
\eqno{(7.I.4)}
$$
Thus, to define the worldsheet parity of each state in
$\scrF_{\Open}$, it suffices to have the $\Omega$
assignment of the (tachyonic) ground state.
Now, $\Omega$ is a symmetry of the worldsheet action
and thus preserved in string interactions.
Since there is a non-zero $3$ tachyon vertex, we must have
$\Omega^3=1$, so that $\Omega=1$ on the tachyon:
$$
\Omega\left.\vert
0,k\right>=+\left.\vert0,k\right>\,\,.
\eqno{(7.I.5)}
$$
As a result, the $\Omega$-assignment of Yang-Mills
states is negative:
$$
\Omega\,\,x_{-1}^\mu\left\vert0,k\right>=
-x_{-1}^\mu\left.\vert0,k\right>\,\,
\eqno{(7.I.6)}
$$
For {\it closed strings}, we have the mode expansion
$$
x^\mu(z,\zbar)=q^\mu-i p^\mu\,\ln\vert z\vert^2+i
  \sum\limits_{m\not=0}{1\over m}
  (x_m^\mu z^{-m}+\xtil_m^\mu\,\zbar^{-m})
\eqno{(7.I.7)}
$$
and
$$
\Omega q^\mu=q^\mu;\,\,
\Omega p^\mu=p^\mu\qquad\qquad
\left\{
\eqalign{
&\Omega x_m^\mu=(-)^m \xtil_m^\mu\cr
&\Omega \xtil_m^\mu=(-)^m x_m^\mu\cr}\right.\,\,.
\eqno{(7.I.8)}
$$
The tachyon, dilaton and graviton now have $\Omega=1$,
but the antisymmetric tensor state $x_{-1}^{[\mu}$
$\xtil_{-1}^{\nu]}\left.\vert0;k\right>$ has $\Omega=-1$,
because $\Omega$ interchanges the indices $\mu$ and
$\nu$.

In addition to acting on the string coordinate
$x^\mu$, $\Omega$ will also act on the Chan-Paton
charges, since the end points of the open string are
exchanged.
We consider here $\SO(N)$ or $\Sp(N)$ Chan-Paton
charges, where the representations $\Lam$ are real or
pseudo-real.
We introduce a basis of open string states with
Chan-Paton charges
$$
\left.\vert\eps,k;ij\right>
\eqno{(7.I.9)}
$$
where $k$ denotes the momentum, the indices
$i,j=1,\ldots,\dim\,\Lam=\dim\,\Lambar=N$ label the
Chan-Paton charges, and $\eps$ label the remaining
quantum numbers, such as polarizations.
A general state may be described by a matrix of
Chan-Paton charges $\lam_{ij}$, and we shall
specialize here to the ground state:
$$
\lam_{ij}\left.\vert 0,k;ij\right>\,\,.
\eqno{(7.I.10)}
$$
Now apply $\Omega$ to this state, and use the fact
that charges $ij$ are interchanged,
$$
\Omega\,\lam_{ij}\left.\vert0,k; ij\right>=
\lam'_{ij}\left.\vert0,k;ij\right>\,\,.
\eqno{(7.I.11)}
$$
Clearly, $\lam'$ must be linear in $\lam^T$, so we
should have
$$
\lam'=M\lam^T M'\,\,.
\eqno{(7.I.12)}
$$
Now, parity is a symmetry of the worldsheet action,
and thus of the transition
amplitudes, so we must have an additional reqirement
on the Chan-Paton charges:
$$
\tr\lam_1\ldots\lam_p=\tr\,\lam_1^T\ldots
\lam_p^T\,\,.
\eqno{(7.I.13)}
$$
Hence $M'=M^{-1}$.
In addition, from $\Omega^2=1$, we have
$$
\lam=M(M\lam^T M^{-1})^T=M(M^{-1})^T\lam
(M(M^{-1})^T)^{-1}\,\,.
\eqno{(7.I.14)}
$$
Assuming that the set of matrices $\lam$ is complete,
we have by Schur's lemma that $M^T=\pm M$.
Up to changes of bases, we then have $2$ possibilities
$$
\matrix{
1) &M^T=M=I_N:\hfill &\Omega\lam_{ij}\left.\vert
  0,k;ij\right>=\lam_{ij}^T\left.\vert0,k;ij\right>\hfill\cr
\noalign{\medskip}
 & &\Omega\lam_{ij}x_{-1}^\mu\left.\vert0,k;ij\right>
  =-\lam_{ij}^Tx_{-1}^\mu\left.\vert0,k;ij\right>\hfill\cr}
\eqno{(7.I.15)}
$$
\medskip
$$
\null\kern20pt
\matrix{
\noalign{\hbox{$2)\quad M^T=-M=J=
  \diag(\sigma_2\ldots\sigma_2):$}\medskip}
 & &\qquad\qquad\qquad\Omega\lam_{ij}\left.\vert0,k;ij\right>=
  (J\lam^T J)_{ij}\left.\vert0,k;ij\right>\hfill\cr
\noalign{\medskip}
& &\qquad\qquad\qquad
  \Omega\lam_{ij}x_{-1}^\mu\left.\vert0,k;ij\right>=-(J\lam^T
J)_{ij}x_{-1}^\mu \left.\vert0,k;ij\right>\,\,.\hfill\cr}
\eqno{(7.I.16)}
$$

Now, we are in a position to identify the r\^{o}le of
{\it unorientable} strings.
$\Omega$ is a (global) symmetry of the worldsheet
action and of string interactions.
This symmetry may be {\it gauged}, which means that
the Hilbert space of states is projected to states
that are invariant under  $\Omega$.
The relevant projection operator is
$$
\half(1+\Omega)\,\,.
\eqno{(7.I.17)}
$$
In the closed (unoriented) string, this projection
eliminates the anti-symmetric tensor; in the open
string, with Chan-Paton charges, as described above,
it will project out the tachyon in 1) and 2) provided
we require $\lam^T=-\lam$ in 1), i.e. $\SO(N)$ gauge
group, and $\lam^T=-J\lam J$ in 2), i.e. $\Sp(N)$
gauge group.
We see that we then retain the Yang-Mills vector
states in the spectrum.

Non-orientable worldsheets now arise in the following
way.
An open string worldsheet, to tree level, with initial
state $I$ and final state $F$ and possibly some vertex
operators inserted, will give rise to a one loop
diagram by summing over all possible intermediate
states.
For oriented open strings, we have
$$
\vbox{\epsfxsize=4.00in\epsfbox{oriented-open.eps}}
$$
But when the open string spectrum is projected onto
$\Omega=1$ states only, then the projection operator
${1\over 2}(1+\Omega)$ must be inserted in the trace.
Then, the contributions arise from the annulus and the
M\"obius string:
$$
\vbox{\epsfxsize=5.00in\epsfbox{mobius.eps}}
$$

In the $\GSO$ projected
RNS superstring, the worldsheet parity operator
exchanges left and right moving $\psi_\sigma$'s and
thus interchanges left and right moving sectors of the
$\GSO$ projected RNS spectrum.
The unoriented $(\Omega=1)$ projection of the closed
string is achieved by imposing symmetry between left
and right moving sectors.
This can be achieved starting from the Type $\IIB$
superstring, and retaining only states invariant under
interchange of left and right movers.
There remains in the {\it Type I closed string sector}

$$
\matrix{
\NS-\NS\hfill  &{\rm bosons}\hfill &(35) &G_{\mu \nu};
  &(1) &\Phi\hfill\cr
\R-\R\hfill &{\rm bosons}\hfill &(28) &A_{\mu \nu}^{(2)} &&\cr
\noalign{\bigskip}
\NS-\R\hfill &{\rm fermions} && &\cr
\R-\NS\hfill &{\rm fermions}\hfill &(56)
&\chi_\mu^\alpha,\hfill
&(8) &\lam_\alpha\hfill\,\,.\cr}
\eqno{(7.I.18)}
$$
In the {\it open superstring sector}, we have an $N=1$
super-Yang-Mills multiplet in $D=10$, with gauge group
$\SO(32)$.
(The restriction to $\SO(32)$ arises only at $1$-loop
level, and will not be justified further here.)
$$
\matrix{
\NS:\hfill &\qquad 8\times 496 &\qquad A_\mu\cr
\noalign{\medskip}
\R:\hfill &\qquad 8\times 496 &\qquad \psi^\sigma\cr}
\eqno{(7.I.19)}
$$




\bye



