%From: Pierre Deligne <deligne@IAS.EDU>
%Date: Tue, 26 Nov 1996 14:39:20 -0500
%Subject: Problem Set 2, No. 5

\documentclass[12pt,leqno]{article}
%\usepackage{amsthm,amsbsy,amsfonts,amssymb,amsmath,amscd}
\usepackage{amsthm}
\usepackage{amstex}
\usepackage{latexsym}
\usepackage{fontenc}
\usepackage[mathscr]{euscript}
\setlength{\textwidth}{6.5in}
\setlength{\textheight}{8.5in}
\setlength{\topmargin}{0pt}
\setlength{\oddsidemargin}{0pt}
\setlength{\evensidemargin}{0pt}
\setlength{\headheight}{0pt}
\setlength{\headsep}{0pt}

%\hoffset=-1.75 true cm
%\font\fiverm=cmr5

%\input{pictex.tex}

\DeclareMathAlphabet{\scrb}{U}{eus}{b}{n}%for bold script letters


\newcommand{\dbA}{{\mathbb{A}}} %Blackboard Bold
\newcommand{\dbB}{{\mathbb{B}}}
\newcommand{\dbC}{{\mathbb{C}}}
\newcommand{\dbD}{{\mathbb{D}}}
\newcommand{\dbQ}{{\mathbb{Q}}}  
\newcommand{\dbR}{{\mathbb{R}}}
\newcommand{\dbZ}{{\mathbb{Z}}}

\newcommand{\grA}{{\mathfrak{A}}}  %Gothic or German
\newcommand{\grB}{{\mathfrak{B}}}
\newcommand{\grC}{{\mathfrak{C}}} 
\newcommand{\grD}{{\mathfrak{D}}}
\newcommand{\grG}{{\mathfrak{G}}} 

\newcommand{\bfA}{{\mathbf A}}   %Bold
\newcommand{\bfB}{{\mathbf B}}
\newcommand{\bfC}{{\mathbf C}} 
\newcommand{\bfD}{{\mathbf D}}

\newcommand{\ScrA}{{\mathscr{A}}} %Script 
\newcommand{\ScrB}{{\mathscr{B}}}
\newcommand{\ScrC}{{\mathscr{C}}} 
\newcommand{\ScrL}{{\mathscr{L}}} 
\newcommand{\ScrS}{{\mathscr{S}}}


%\newcommand{\bfscrA}{{\scrb{A}}} %for bold script letters
%\newcommand{\bfscrB}{{\scrb{B}}}
%\newcommand{\bfscrC}{{\scrb{C}}} 
%\newcommand{\bfscrD}{{\scrb{D}}}


%\renewcommand{\baselinestretch}{1.5}


\theoremstyle{plain}
{\theorembodyfont{}\newtheorem{thm}{Theorem}}
\newtheorem{prop}{Proposition}
\newtheorem{lem}{Lemma}
\newtheorem{cor}{Corollary}
\theoremstyle{remark}
{\theorembodyfont{\rmfamily} \newtheorem*{Rem}{\bf Remark}}
%\renewcommand{\theRem}{}


\newcommand{\Le}{{{\mathchoice{\,{\scriptstyle\le}\,}
  {\,{\scriptstyle\le}\,}
  {\,{\scriptscriptstyle\le}\,}{\,{\scriptscriptstyle\le}\,}}}}
\newcommand{\Ge}{{{\mathchoice{\,{\scriptstyle\ge}\,}
  {\,{\scriptstyle\ge}\,}
  {\,{\scriptscriptstyle\ge}\,}{\,{\scriptscriptstyle\ge}\,}}}}
\newcommand{\Ln}{\text{\rm ln}}



\font\boldtitlefont=cmb10 scaled\magstep1

\newcommand{\dspace}{\lineskip=2pt
     \baselineskip=18pt\lineskiplimit=0pt}
\newcommand{\wedgeop}{\mathop{\wedge}\limits}
\newcommand{\w}{{\mathchoice{\,{\scriptstyle\wedge}\,}
  {{\scriptstyle\wedge}}
  {{\scriptscriptstyle\wedge}}{{\scriptscriptstyle\wedge}}}}
\newcommand{\vrulesub}[1]{\hbox{\vrule height7pt depth5pt\,}_{#1}}
\newcommand{\mapright}[1]{\smash{\mathop{\,\longrightarrow\,}%
     \limits^{#1}}}
\newcommand{\plus}{{\sssize +}}
\newcommand{\upvee}{{\sssize\vee}}

\newcommand{\eps}{{\varepsilon}}
\newcommand{\lam}{{\lambda}}
\newcommand{\Lam}{{\Lambda}}
\newcommand{\mynabla}{{\nabla\!}}

\newcommand{\xtil}{\widetilde{x}}

\newcommand{\Adot}{\Dot{A}}
\newcommand{\Bdot}{\Dot{B}}
\newcommand{\Xdot}{\Dot{X}}
\newcommand{\xdot}{\Dot{x}}
\newcommand{\psidot}{\Dot{\psi}}


\newcommand{\GL}{\text{\rm GL}} 
\newcommand{\Spec}{\text{\rm Spec}}
\newcommand{\Hom}{\text{\rm Hom}}
\newcommand{\aff}{\text{\rm aff}} 
\newcommand{\red}{\text{\rm red}}
\newcommand{\Map}{\text{\rm Map}} 
\newcommand{\Lie}{\text{\rm Lie}}
\newcommand{\Diff}{\text{\rm Diff}} 
\newcommand{\Vol}{\text{\rm Vol}}
\newcommand{\Tr}{\text{\rm Tr}}

\title{Witten's Problems, Set Five --- N$^{\text{o}}$. 2}
\author{(solution written by D. Freed)}
\date{}

\overfullrule=5pt
\begin{document}

\maketitle

\hbox to \hsize{\hrulefill}

\bigskip
\dspace
\noindent
(a) \ 
We should compute $\Tr\,e^{-TH}$ for
$$
\frac{1}{2m}\left(-\frac{\partial^2}{\partial x^2}+
m^2\omega^2 x^2\right)\,\,.
$$
First, we can compute directly in the Hamiltonian picture.
Introduce
\begin{align*}
A &=\frac{1}{\sqrt{2m}}\left(\frac{\partial}{\partial x}
  +m\omega x\right)\\
A^* &=\frac{1}{\sqrt{2m}}\left(-\frac{\partial}{\partial x}
+ m\omega x\right)
\end{align*}
Then $H=A^*A+\frac\omega2$ and $[A,A^*]=\omega$.
So the spectrum of $H$ is
$$
\Spec\,H=\left\{\frac\omega2+n\omega
\right\}_{n\in\dbZ^{^{\Ge 0}}}
$$
and
\begin{equation}
\begin{aligned}
\Tr\,e^{-TH} &=e^{-\omega T/2}\sum\limits_{n=0}^\infty
  e^{-n\omega T}\\
&= e^{-\omega T/2}\frac{1}{1-e^{-\omega T}}\\
&= \frac{1}{2\sinh^{\omega T/2}}
\label{one}
\end{aligned}
\end{equation}
To compute the heat kernel trace as a path integral we use
the Feynman-Kac formula
\begin{equation}
\Tr\,e^{-TH}=\int e^{-L(X)}dX,
\label{two}
\end{equation}
where $X\colon\, S^1(T)\to\dbR$ is a function on the circle
$S^1(T)=\dbR/T\dbZ$, and
$$
L(X)=\frac m2\int_0^T\left\{\Xdot(t)^2+
  \omega^2 X(t)^2\right\}dt
$$
Since $X$ is periodic we can integrate by parts to obtain
\begin{align}
L(X) &=(QX,X)/2\label{three}\\
Q &=m\left(-\frac{\partial^2}{\partial t^2}+\omega^2\right)
\label{four}
\end{align}
where the inner product in (\ref{three}) is the $L^2$
inner product on $S^1(T)$.
So the path integral (\ref{two}) is formally
\begin{equation}
\Tr\,e^{-TH}=\frac{1}{\sqrt{\det\,Q}}\label{five}
\end{equation}
Now $Q$ has eigenfunctions $\{e^{2\pi int/T}\}_{n\in\dbZ}$
with corresponding eigenvalues
$\left\{m\left(\frac{4\pi^2n^2}{T^2}
+\omega^2\right)\right\}_{n\in\dbZ}$.
Therefore, the determinant is the divergent product
\begin{align*}
\det\,Q
&=\prod\limits_{n\in\dbZ}m
\left(\frac{4\pi^2n^2}{T^2}+\omega^2\right)\\
&=\prod\limits_{n\in\dbZ^+}\left(\frac{4\pi^2m}{T^2}\right)^2
\prod\limits_{n=1}^\infty\left(1+\left(\frac{\omega T}
{2\pi n}\right)^2\right)\cdot m\omega^2\\
&=\prod\limits_{n\in\dbZ^+}\left(\frac{4\pi^2m}{T^2}
  n^2\right)^2\cdot\frac{\sinh^2\left(\frac{\omega T}{2}\right)}
{\left(\frac{\omega T}{2}\right)^2}\cdot m\omega^2\,\,.
\end{align*}
We evaluate the infinite constants using the Riemann
$\zeta$-function:
\begin{align*}
\prod\limits_{n\in\dbZ^+}\left(\frac{4\pi^2m}{T^2}\,n^2\right)^2
&=\left(\frac{4\pi^2m}{T^2}\right)^{2\zeta(0)}e^{-4\zeta'(0)}\\
&=\frac{T^2}{4\pi^2m}\,4\pi^2\,\,,
\end{align*}
since $\zeta(0)=-1/2$ and $\zeta'(0)=-\frac12\,\ln\,2\pi$.
Altogether we have
$$
\det\,Q=\left(2\sinh\,\frac{\omega T}{2}\right)^2\,\,,
$$
and plugging in (\ref{five}) we obtain agreement with
(\ref{one}).

\bigskip\noindent
(b) \ 
We first evaluate the path integral.
We must diagonalize (\ref{four}) with antiperiodic boundary
conditions.
So the eigenfunctions are $\{e^{2\pi int/T}\}_{n\in\dbZ+1/2}$
with corresponding eigenvalues
$\left\{m\left(\frac{4\pi^2n^2}{T^2}
+\omega^2\right)\right\}_{n\in\dbZ+1/2}$.
So
\begin{equation}
\begin{aligned}
\det\,Q
&=\prod\limits_{n\in\dbZ+1/2}m\left(\frac{4\pi^2n^2}{T^2}
  +\omega^2\right)\\
&=\prod\limits_{n\in\dbZ+1/2}\left(\frac{4\pi^2m}{T^2}
  \,n^2\right)\prod\limits_{n\in\dbZ+1/2}\left(1+
  \left(\frac{\omega T}{2\pi n}\right)^2\right)^2\\
&=4\cosh^2\,\frac{\omega T}{2}\,\,,
\label{six}
\end{aligned}
\end{equation}
from which the path integral is
\begin{equation}
\frac{1}{\sqrt{\det\,Q}}=\frac{1}{2\cosh\,\frac{\omega T}{2}}.
\label{seven}
\end{equation}
In (\ref{six}) we use the fact that the $\zeta$-normalized
number of integers is zero to compute
$$
\prod\limits_{n\in\dbZ+1/2}\left(\frac{4\pi^2m}{T^2}\,n^2\right)
=\prod\limits_{n\in\dbZ+1/2}n^2=\prod\limits_{\text{$n$
odd}}n^2=\frac{\prod\limits_{n\not=0}n^2}
{\prod\limits
\Sb
\text{$n$ even}\\
n\not=0
\endSb
n^2}=\frac{\prod\limits_{n\not=0}n^2}
{\prod\limits_{n\not=0}4n^2}=4\,\,,
$$
since the $\zeta$-normalized number of nonzero integers is
$-1$.

On the Hilbert space the operator $J$ is the action on
functions induced by $x\to-x$.
Notice that $J$ anticommutes with $A^*$, so that its
action on the n$^{\text{th}}$ eigenfunction
$\frac{(A^*)^n}{n!}\,\Omega$ is multiplication by $(-1)^n$.
Hence
\begin{align*}
\Tr\,Je^{-TH} &=e^{-\omega T/2}\sum\limits_{n=0}^\infty
  (-1)^n e^{-n\omega T}\\
&=e^{-\omega T/2}\frac{1}{1+e^{-\omega T}}\\
&=\frac{1}{2\cosh\omega T/2}
\end{align*}
This agrees with (\ref{seven}).



\end{document}



