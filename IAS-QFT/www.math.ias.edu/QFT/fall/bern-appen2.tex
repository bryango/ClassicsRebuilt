%From: Pierre Deligne <deligne@math.ias.edu>
%Date: Fri, 25 Oct 1996 16:20:41 -0400

\input amstex
\documentstyle{amsppt}
\magnification=1200
\loadbold
\loadeusm

\font\boldtitlefont=cmb10 scaled\magstep2

\NoRunningHeads
\pagewidth{6.5 true in}
\pageheight{8.9 true in}

\catcode`\@=11
\def\logo@{}
\catcode`\@=13

\def\eps{{\varepsilon}}

\def\dbR{{\Bbb R}}

\def\undertext#1{$\underline{\vphantom{y}\hbox{#1}}$}
\def\nspace{\lineskip=1pt\baselineskip=12pt%
     \lineskiplimit=0pt}
\def\dspace{\lineskip=2pt\baselineskip=18pt%
     \lineskiplimit=0pt}

\def\upvee{{\sssize \vee}}
\def\plus{{\sssize +}}
\def\w{{\mathchoice{\,{\scriptstyle\wedge}\,}
  {{\scriptstyle\wedge}}
  {{\scriptscriptstyle\wedge}}{{\scriptscriptstyle\wedge}}}}
\def\Le{{\mathchoice{\,{\scriptstyle\le}\,}
{\,{\scriptstyle\le}\,}
{\,{\scriptscriptstyle\le}\,}{\,{\scriptscriptstyle\le}\,}}}
\def\Ge{{\mathchoice{\,{\scriptstyle\ge}\,}
{\,{\scriptstyle\ge}\,}
{\,{\scriptscriptstyle\ge}\,}{\,{\scriptscriptstyle\ge}\,}}}
\def\mapright#1{\smash{\mathop{\,\longrightarrow\,}%
     \limits^{#1}}}
\def\rmapdown#1{\Big\downarrow\kern-1.0pt\vcenter{
     \hbox{$\scriptstyle#1$}}}
\def\bfs{\bold{s}}
\def\arrowsim{\smash{\mathop{\,\longrightarrow\,}%
   \limits^{\lower1.5pt\hbox{$\scriptstyle\sim$}}}}
\def\leftarrowsim{\smash{\mathop{\,\longleftarrow\,}%
   \limits^{\lower1.5pt\hbox{$\scriptstyle\sim$}}}}

\def\Sym{\text{\rm Sym}} 
\def\Hom{\text{\rm Hom}} 
\def\Tor{\text{\rm Tor}} \def\Gr{\text{\rm gr}}
\def\Ext{\text{\rm Ext}} \def\ev{\text{\rm ev}}
\def\I{\text{\bf I}} \def\Ber{\text{\rm Ber}}
\def\II{\text{\bf II}} \def\ber{\text{\rm ber}}


%Script letters:
\def\scr#1{{\fam\eusmfam\relax#1}}
\def\scrT{{\scr T}} 


\topmatter
\title\nofrills
{\boldtitlefont Appendix: Even rules}
\endtitle
\author
P. Deligne
\endauthor
\endtopmatter

\NoBlackBoxes
\parindent=20pt
\frenchspacing
\document
\bigskip
\dspace
Let $A$ be a commutative super ring and $M$, $N$ be
$A$-modules.
A morphism $f\colon\,M\to N$ is a parity respecting map
$f$ such that $f(am)=af(m)$.

The ``even rules'' point of view is that

\medskip\noindent
(a)\enspace
one should think to $M$ in terms of the functor which to
any extension of scalars $A\to A'$ attach the even part
$(M')^{\plus}$ of $M':= A'\otimes_A M$;

\smallskip\noindent
(b)\enspace
it amounts to the same to give a morphism of modules
$f\colon\,M\to N$ or to give for each commutative super
$A$-algebra $A'$ a $(A'){^{\plus}}$-linear morphism
$F[A']$ from $(M')^{\plus}$ to $(N')^{\plus}$,
functorial in $A'$.

More precisely, the map $f\mapsto F$:
$$
F[A']=f'\vert M'{^{\plus}}\,\,,
$$
with $f'$ deduced from $f$ by extension of scalars, is
bijective.

\medskip\noindent
{\bf Proof.}\enspace
To a functorial $F[A']\colon\, (M')^{\plus}\to
(N')^{\plus}$ one attaches a morphism $f\colon\, M\to N$
as follows.
On $M^{\plus}$, $f:=F[A]$.
Take now $A'=A[\theta]$ with $\theta$ odd.
An element n of $N'{^{\plus}}$ can be written uniquely in
the form $n_0+\theta n_1$ with $n_0\in N^{\plus}$ and
$n_1\in N^-$.
Define $\varphi_0$ and $\varphi_1$ on $M^-$ by
$$
F[A'](\theta m)=\varphi_0(m)+\theta_1(m),
$$
and define $f$ on $M^-$ to be $\varphi_1$.

Apply functoriality to $A'\to A\colon\, \theta\mapsto 0$.
The corresponding map from $(M')^{\plus}$ to $M^{\plus}$
maps $\theta m$ ($m\in M^-$) to zero, and it follows
that $\varphi_0=0$: \ for $A'=A[\theta]$, we have
$$
F(A')(m_0+\theta m_1)=f(m_0)+\theta f(m_1)
$$
for $m_0$ in $M^{\plus}$ and $m_1$ in $M^-$.

We now check that $f$ is $A$-linear.
The $A^{\plus}$-linearity is clear.
If $a$ is in$A^-$ and $m$ in $M^{\plus}$,
$$
\theta f(am)=F(A')(\theta am)=\theta a F(A')(m)=
\theta af(m)\,\,.
$$
For $m$ in $M^-$, applying functoriality to $A'\to
A\colon\, \theta\to a$, we get $f(am)=af(m)$ as well.

If $F$ is obtained from a morphism $f$, one has for
$A'=A[\theta]$
$$
f'(m_0+\theta m_1)=f(m_0)+\theta f(m_1)
$$
and it follows that this construction gives back $f$.

Conversely, the $f$ deduced from $F$ gives back $F$ by
extensions of scalars: \ for $a'\in A'$ and $m\in M$
even,
$$
F(a'm)=a'F(m)=a'f(m)
$$
by $A\to A'$ functoriality.
For $a'$ and $m$ odd, functoriality for $A[\theta]\to
A'\colon\,\theta\mapsto a'$ gives
$$
F(a'm)=a'f(m)
$$
as well.

A similar, more complicated argument, gives the

\proclaim{Theorem}
Let $M_i$ be a finite family of $A$-modules.
The data of a morphism
$$
f\colon\,\otimes M_i\to N
$$
amounts to the data of a 
$A'{^{\plus}}$-multilinear map, functorial in
$A'$, of the $(M'_i)^{\plus}$ to $(N')^{\plus}$.
\endproclaim

The proof is left to the reader.

\proclaim{Corollary}
Let $B$ be an $A$-module.
To give on $B$ a structure of $A$ algebra amounts to
give on $(B')^{\plus}$ a structure of $(A')^{\plus}$
algebra, functorially in $A'$.
The algebra $B$ will be associative (resp. commutative,
resp. Lie, $\ldots\,$) if and only if the algebras
$(B')^{\plus}$ are.
\endproclaim

Note that the $(B')^{\plus}$ are purely even objects: at
the cost of having to think functorially, which one
wants to do anyway, this removes signs from the
definitions.

\example{Example}
Let $X$ be a supermanifold and $L$ be an (even) function
on $X$.
The locus where $L$ is stationary is the subspace $S$ of
$X$ (a sub-supermanifold in good cases) defined by the
vanishing of the section $dL$ of the cotangent bundle of
$X$.

When thinking functorially, one thinks to $X$ as
determined by the functor which to any supermanifold $B$
attaches the set $B$-points of $X$,
where a $B$-point is a morphism $x\colon\,B\to X$.
Intuitively: \ a family of points of $X$ parametrized by
$B$.
In this language, describing $S$ means describing which
$B$-points $x\colon\, B\to X$ map to $S$, i.e. factors
through $S\hookrightarrow X$.
Such $B$-points will be called stationary.

By definition, a $B$-point $x$ is stationary if for
any even or odd section $s$ on $B$ of $x^*T_X$ (the
``fiber of $T_X$ at $x$''), $\left<dL,s\right>=0$ on
$B$.
If $s$ is odd, let $B'=B\times\dbR^{0,1}$: \ 
we add an odd parameter $\theta$.
Then, $\theta s$ is even and $\left<dL,s\right>=0$ on
$B$ if and only if $\left<dL,\theta s\right>=0$ on $B'$.
We see that a 
$B$-point $x$ is stationary if and only if after any
change of basis $B'\to B$, the  $B'$-point $B'\to B\to
X$, still noted $x$, is such that for any
{\it even} section $s$ of $x^*T_M$,
$\left<dL,s\right>=0$ on $B'$.

This can be rephrased as follows: \ 
after any change of basis $B'\to B$, one wants that for
any family of points $x(t)$ ($-\eps<t<\eps$): \
$B'\times]-\eps,\eps[\to X$, one has
$$
\tfrac{d}{dt}\,L(x)=0\quad\text{at}\quad t=0
$$
or, more cumbersomely written: \ the pull back of $L$ to
$B'\times]-\eps,\eps[$ is such that
$\frac{\partial L}{\partial t}=0$ at $t=0$.

This formulation is convenient when $X$ is ``infinite
dimensional'', and when what makes good sense is not the
space $X$ and the function $L$, but the notion of map
$x\colon\,B\to X$ and of the pull back by $x$ of $L$.
\endexample

\bigskip\noindent
{\bf Minority report: \ Odd rules.}

The notion of super Lie algebra can be expressed nicely
as follows:

\noindent
Data: \ after any extension of scalars $A\to A'$, a
quadratic map $x^2$ from $L^-$ to $L^{\plus}$,
functorial in $A'$.

Applying the theorem to the associated bilinear form,
from $(\Pi L)^{\plus}\otimes(\Pi L)^{\plus}$ to
$L^{\plus}$, we see that $x\mapsto x^2$ defines a
bracket $[x,y]\colon\, L\otimes L\to L$ such that for
$x$ and $y$ in $L^-$, $[x,y]=(x+y)^2-x^2-y^2$.
In particular, for $x$ in $L^-$, $[x,x]=2x^2$.

\noindent
Axiom: \ 
after any extension of scalars $A\to A'$, for any $x$ in
$L^-$, one has
$$
[x,x^2]=0\,\,.
$$

Provided $6$ is invertible in $A$, this definition is
equivalent to the one in term of the Jacobi identity
(which is a polarized version of the axiom).

\enddocument





