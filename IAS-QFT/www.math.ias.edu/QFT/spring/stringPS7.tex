%%%%%%%%%%%%%%%%%%%%%%%%%%%%%%%%%%%%%%%%%%%%%%%%%%%%%%%%%%%%%%%%%%%%%%%%%%%
%%%%
%%%%    STRING THEORY : Problem Set 7, March 13, 1997
%%%%
%%%%%%%%%%%%%%%%%%%%%%%%%%%%%%%%%%%%%%%%%%%%%%%%%%%%%%%%%%%%%%%%%%%%%%%%%%%


\magnification=\magstep1
\overfullrule=0pt
\baselineskip=17pt
\def\det{{\rm det}}
\def\Det{{\rm Det}}
\def\tr{{\rm tr}}
\def\Tr{{\rm Tr}}
\def\12{{1 \over 2}}
\def\ker{{\rm Ker}}
\def\O{{\cal O}}

\centerline{{\bf STRING THEORY}}
\centerline{ Problem Set \# 7}
\centerline{March 13, 1997}

\bigskip
\bigskip


\noindent
{\it Problem 1}

\medskip

Let $M$ be a Riemannian manifold with local coordinates $x^\mu$,
metric $G=G_{\mu \nu}dx^\mu \otimes dx^\nu$, two form
$B=B_{\mu \nu} dx^\mu \wedge d x^\nu$, and scalar field $\Phi$,
and let $l$ be a real positive parameter.
The generalized non-linear sigma model action on a surface $\Sigma$,
with metric $g$, Gaussian curvature $R_g$, volume form $d \mu _g$,
and anti-symmetric tensor $\epsilon ^{mn}$, is given by
$$
S[x;g] = {1 \over 8 \pi l^2} \int _\Sigma d \mu _g
\bigl [ g^{mn} G_{\mu \nu} + \epsilon ^{mn} B_{\mu \nu} \bigr ]
\partial _m x^\mu \partial _n x ^\nu 
+ {1 \over 2 \pi } \int _\Sigma d \mu _g R_g \Phi
$$
a) Derive the form of the vector field $V^\mu(x)$ in $TM$, such that
the infinitesimal Weyl transformation $\delta g = 2 \delta \sigma g$,
combined with the transformation 
$\delta x^\mu = l^2 \delta \sigma V^\mu (x)$ yields a trace of the stress
tensor $T_{mn} g^{mn}$ which is of the same form as the Lagrangian
density of $S[x;g]$, i.e. such that
$$
\delta S[x;g] = -{1 \over 2 \pi} \int _\Sigma d \mu _g
\bigl [ g^{mn} \beta ^G _{\mu \nu} + \epsilon ^{mn} \beta ^B_{\mu \nu} \bigr ]
\partial _m x^\mu \partial _n x ^\nu 
+ {1 \over 2 \pi } \int _\Sigma d \mu _g R_g \beta ^\Phi
$$
b) Derive the resulting expressions for $\beta ^G, ~ \beta ^B$ and 
$\beta ^\Phi$. (These are the tree level contributions to the 
$\beta$-functions defined for Weyl transformations.)

\bigskip

\noindent
{\it Problem 2}

\medskip

Let $\phi$ be a real scalar field on a Riemann surface $\Sigma$, with action
$$
S[\phi;g] = { 1 \over 8 \pi} \int _\Sigma d \mu _g \bigl [
g^{mn} \partial _m \phi \partial _n \phi + m^2 \phi ^2 \bigr ]
$$
where $m$ is an IR regulator mass, which we let to zero whenever
the limit is well-defined.

\noindent
a) Use dimensional regularization to derive the Weyl transformation
properties of the Green function $G(\xi, \xi')=\langle \phi (\xi)
\phi (\xi') \rangle$ and its first and second derivatives,
evaluated on the diagonal (i.e. at coincident points).

\noindent
b) Repeat the calculation of a), but now with a heat-kernel regularization
of the Green function. Compare your result with that of a).
\end


