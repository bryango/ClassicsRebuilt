%% This is a plain TeX file
%%
\magnification=1200
\hsize=6.5 true in
\vsize=8.7 true in
\input epsf.tex

\input amssym.def
\input amssym.tex
%\input eight.tex
% beginning of file eight.tex

\font\eightrm=cmr8
\font\sevenrm=cmr7
\font\sixrm=cmr6
\font\fiverm=cmr5
\font\eighti=cmmi8
\font\sixi=cmmi6
\font\fivei=cmmi5
\font\eightsy=cmsy8
\font\sixsy=cmsy6
\font\fivesy=cmsy5
\font\tenex=cmex10
\font\eightit=cmti8
\font\eightsl=cmsl8
\font\eighttt=cmtt8
\font\eightbf=cmbx8
\font\sixbf=cmbx6
\font\fivebf=cmbx5
% Cindy's attempt:
\font\eightmsb=msbm8
\font\sixmsb=msbm6

\def\eightpoint{\def\rm{\fam0\eightrm}% switch to 8-point type
  \textfont0=\eightrm \scriptfont0=\sixrm \scriptscriptfont0=\fiverm
  \textfont1=\eighti \scriptfont1=\sixi \scriptscriptfont1=\fivei
  \textfont2=\eightsy \scriptfont2=\sixsy \scriptscriptfont2=\fivesy
  \textfont3=\tenex \scriptfont3=\tenex \scriptscriptfont3=\tenex
  \textfont\itfam=\eightit  \def\it{\fam\itfam\eightit}%
  \textfont\slfam=\eightsl  \def\sl{\fam\slfam\eightsl}%
  \textfont\ttfam=\eighttt  \def\tt{\fam\ttfam\eighttt}%
  \textfont\bffam=\eightbf  \scriptfont\bffam=\sixbf
   \scriptscriptfont\bffam=\fivebf  \def\bf{\fam\bffam\eightbf}%
% Cindy's attempt:%
%\textfont\msbfam=\eightmsb%
%\scriptfont\msbfam=\sixmsb%
%  \textfont4=\eightmsb%         %=\msbfam =\eightmsb%
%  \scriptfont4=\sixmsb%         %=\msbfam=\sixmsb%
%  \textfont\msbfam=\eightmsb%
%  \scriptfont\msbfam=\sixmsb%
%  \def\msb{\fam\msbfam\eightmsb}%
% end of Cindy's attempt%
  \normalbaselineskip=9pt
  \setbox\strutbox=\hbox{\vrule height7pt depth2pt width0pt}%
  \let\sc=\sixrm  \normalbaselines\rm}

% end of file eight.tex

\font\dotless=cmr10 %for the roman i or j to be
                    %used with accents on top.
                    %(\dotless\char'020=i)
                    %(\dotless\char'021=j)
\font\itdotless=cmti10
\def\itumi{{\"{\itdotless\char'020}}}
\def\itumj{{\"{\itdotless\char'021}}}
\def\umi{{\"{\dotless\char'020}}}
\def\umj{{\"{\dotless\char'021}}}
\font\smaller=cmr5
\font\boldtitlefont=cmb10 scaled\magstep2
\font\smallboldtitle=cmb10 scaled \magstep1
\font\ninerm=cmr9
\font\dun=cmdunh10 %scaled\magstep1
\font\Rfont=cmss10

\footline={\hfil {\tenrm IX.\folio}\hfil}

\def\eps{{\varepsilon}}
\def\Eps{{\epsilon}}
\def\kap{{\kappa}}
\def\lam{{\lambda}}
\def\Lam{{\Lambda}}
\def\mynabla{{\nabla\!}}

\def\underNS{\underline{\NS}}
\def\underR{\underline{\R}}

\def\Bmu{{B_{\mu\nu}}}
\def\Gmu{{G_{\mu\nu}}}

\def\xdot{{\dot x}}
\def\xddot{{\ddot x}}

\def\undertext#1{$\underline{\vphantom{y}\hbox{#1}}$}
\def\nspace{\lineskip=1pt\baselineskip=12pt%
     \lineskiplimit=0pt}
\def\dspace{\lineskip=2pt\baselineskip=18pt%
     \lineskiplimit=0pt}

\def\Half{\raise4.5pt\hbox{{\vtop{\ialign{##\crcr
\hfil\rm ${\scriptstyle 1}$\hfil\crcr
  \noalign{\nointerlineskip\vskip1.5pt}%
   \hbox to 4pt{\hrulefill}\crcr
\noalign{\nointerlineskip\vskip1.5pt}%
   ${\scriptstyle 2}$\crcr}}}}}
%\def\Half{{{\scriptstyle 1}\over{\scriptstyle 2}}}

\def\half{\raise4.5pt\hbox{{\vtop{\ialign{##\crcr
  \hfil\rm $1$\hfil\crcr
   \noalign{\nointerlineskip}--\crcr
   \noalign{\nointerlineskip\vskip-1pt}$2$\crcr}}}}}
\def\third{\raise4.5pt\hbox{{\vtop{\ialign{##\crcr
  \hfil\rm $1$\hfil\crcr
  \noalign{\nointerlineskip}--\crcr
  \noalign{\nointerlineskip\vskip-1pt}$3$\crcr}}}}}
\def\fourth{\raise4.5pt\hbox{{\vtop{\ialign{##\crcr
  \hfil\rm $1$\hfil\crcr
  \noalign{\nointerlineskip}--\crcr
  \noalign{\nointerlineskip\vskip-1pt}$4$\crcr}}}}}
\def\sixth{\raise4.5pt\hbox{{\vtop{\ialign{##\crcr
  \hfil\rm $1$\hfil\crcr
  \noalign{\nointerlineskip}--\crcr
  \noalign{\nointerlineskip\vskip-1pt}$6$\crcr}}}}}
\def\eighth{\raise4.5pt\hbox{{\vtop{\ialign{##\crcr
  \hfil\rm $1$\hfil\crcr
  \noalign{\nointerlineskip}--\crcr
  \noalign{\nointerlineskip\vskip-1pt}$8$\crcr}}}}}

\def\oplusop{\mathop{\oplus}\limits}
\def\w{{\mathchoice{\,{\scriptstyle\wedge}\,}
  {{\scriptstyle\wedge}}
  {{\scriptscriptstyle\wedge}}{{\scriptscriptstyle\wedge}}}}
\def\Le{{\mathchoice{\,{\scriptstyle\le}\,}
{\,{\scriptstyle\le}\,}
{\,{\scriptscriptstyle\le}\,}{\,{\scriptscriptstyle\le}\,}}}
\def\Ge{{\mathchoice{\,{\scriptstyle\ge}\,}
{\,{\scriptstyle\ge}\,}
{\,{\scriptscriptstyle\ge}\,}{\,{\scriptscriptstyle\ge}\,}}}
\def\plus{{\hbox{$\scriptscriptstyle +$}}}
\def\xdot{\dot{x}}
\def\Condition#1{\item{#1}}
\def\Firstcondition#1{\hangindent\parindent{#1}\enspace
     \ignorespaces}
\def\Proclaim#1{\medbreak
  \medskip\noindent{\bf#1\enspace}\it\ignorespaces}
  %the way to use this is:
  %"\Proclaim{Theorem 1.1.}" for instance.
\def\finishproclaim{\par\rm
     \ifdim\lastskip<\smallskipamount\removelastskip
     \penalty55\medskip\fi}
\def\Item#1{\par\smallskip\hang\indent%
  \llap{\hbox to\parindent {#1\hfill\enspace}}\ignorespaces}
\def\ItemItem#1{\par\indent\hangindent2\parindent
     \hbox to \parindent{#1\hfill\enspace}\ignorespaces}
\def\vrulesub#1{{\,\vrule height7pt depth5pt}_{\,#1}}
\def\underbrake#1#2{\mathop{#1}\limits_{\raise3pt
  \hbox{%
\vrule height 3pt depth 0pt
  %\kern.1pt
  \hbox to #2{\hrulefill}
  \kern-3.4pt
  \vrule height 3pt depth 0pt}}}
\def\notD{{\slash\kern-7pt D}}

\def\ominus{{$-$\kern-9pt $\bigcirc$}}
\def\Oplus{{+\kern-9pt $\bigcirc$}}
\def\ssbullet{{\scriptstyle\bullet\,\,\,}}

\def\im{{\rm Im}}  
\def\A{{\rm A}}
\def\P{{\rm P}}
\def\EL{{\rm L}}
\def\Open{{\rm open}}
\def\osc{{\rm osc}}
%\def\Pic{{\rm Pic}} 
\def\Sp{{\rm Sp}}
\def\R{{\rm R}}  \def\NS{{\rm NS}}
\def\RNS{{\rm RNS}}
\def\Diff{{\rm Diff}}  \def\expt{{\rm expt}}
\def\cylinder{{\rm cylinder}}
\def\Closed{{\rm closed}}
\def\Map{{\rm Map}}  
%\def\spurious{{\rm spurious}}
\def\Met{{\rm Met}} 
\def\Spin{{\rm Spin}}
\def\spin{{\rm spin}} 
\def\phys{{\rm phys}}
\def\diag{{\rm diag}}  
%\def\Vir{{\rm Vir}}
%\def\Res{{\rm Res}}
\def\Null{{\rm null}} 
\def\mass{{\rm mass}}
\def\SO{{\rm SO}} 
\def\GSO{{\rm GSO}}
\def\Tr{{\rm Tr\,}}
%\def\SU{{\rm SU}} 
\def\tr{{\rm tr}}
\def\Weyl{{\rm Weyl}} 
\def\Lorentz{{\rm Lorentz}}
\def\Ker{{\rm Ker}}
\def\Range{{\rm Range}} 
%\def\SL{{\rm SL}} \def\prim{{\rm primitive}}
\def\Det{{\rm Det}} 
\def\re{{\rm Re}}
%\def\dist{{\rm dist}} \def\PSL{{\rm PSL}}
\def\Vol{{\rm Vol}} 
%\def\ghosts{{\rm ghosts}}
%\def\Fock{{\rm Fock}} 
\def\BRST{{\rm BRST}}
%\def\Weil-Peterson{{\rm Weil-Peterson}}
\def\IIA{{\rm II~A}}
\def\IIB{{\rm II~B}}

\def\1bar{\bar{1}}
\def\2bar{\bar{2}}
\def\3bar{\bar{3}}
\def\4bar{\bar{4}}
%\def\fbar{\bar{f}}  
\def\mubar{\bar{\mu}}
\def\nubar{{\bar{\nu}}}
\def\taubar{{\bar{\tau}}}
%\def\barh{\bar{h}}  
\def\gammabar{\bar{\gamma}}
\def\betabar{\bar{\beta}}
%\def\kbar{\bar{k}}   
\def\psibar{\bar{\psi}}
\def\lambar{\bar{\lambda}} \def\xibar{{\bar{\xi}}}
\def\mbar{\bar{m}}   \def\Lambar{\bar{\Lambda}}
\def\thetabar{{\bar{\theta}}}
\def\zetabar{{\bar{\zeta}}}
%\def\phibar{\bar{\phi}}
%\def\wbar{\bar{w}}  \def\etabar{\bar{\eta}}
%\def\vbar{\bar{v}}  \def\partialbar{\bar{\partial}}
%\def\xbar{\bar{x}}  
\def\cbar{\bar{c}}
\def\bbar{\bar{b}}
%\def\epsbar{\bar{\epsilon}}
\def\Gambar{\bar{\Gamma}}
\def\rbar{\bar{r}}
\def\zbar{\bar{z}}  
%\def\Abar{\bar{A}}
\def\Bbar{{\bar{B}}}  \def\Cbar{{\bar{C}}}
\def\Fbar{{\bar{F}}}
\def\Gbar{\bar{G}}
\def\Kbar{\bar{K}}
\def\Mbar{\bar{M}}
%\def\Pbar{\bar{P}}
\def\Sbar{\bar{S}}
\def\Tbar{\bar{T}}
\def\scrFbar{\bar{\scrF}}
\def\scrCbar{\bar{\scrC}}

%\def\scrFbc{{\scrF^{(bc)}}}
%\def\scrFbcbar{{\scrF^{(\bbar\cbar)}}}

\def\chat{\hat{c}}
%\def\ghat{\hat{g}}
%\def\muhat{\hat{\mu}}
\def\Rhat{{\widehat{R}}}
\def\scrDhat{{\widehat{\scrD}}}

\def\Atil{\tilde{A}}
\def\Btil{\widetilde{B}}
\def\atil{\tilde{a}}
\def\btil{\tilde{b}}
\def\dtil{\tilde{d}}
\def\xtil{\tilde{x}}
%\def\htil{\tilde{h}}
%\def\Ctil{\widetilde{C}}
%\def\Dtil{\widetilde{D}}
\def\Ltil{\tilde{L}}
\def\Ntil{\widetilde{N}}
\def\Ptil{\widetilde{P}}
\def\Wtil{\widetilde{W}}
\def\betatil{\tilde{\beta}}
%\def\Ttil{\widetilde{T}}
\def\scrFtil{\widetilde{\scrF}}
%\def\epstil{\tilde{\eps}}
%\def\psitil{\tilde{\psi}}

\def\dbR{{\Bbb R}}
%\def\dbZ{{\Bbb Z}}

%These two files (in this order!!) are necessary
%in order to use AMS Fonts 2.0 with Plain TeX

\input amssym.def
\input amssym.tex

%Capital roman double letters(Blackboard bold)
\def\db#1{{\fam\msbfam\relax#1}}

\def\dbA{{\db A}} \def\dbB{{\db B}}
\def\dbC{{\db C}} \def\dbD{{\db D}}
\def\dbE{{\db E}} \def\dbF{{\db F}}
\def\dbG{{\db G}} \def\dbH{{\db H}}
\def\dbI{{\db I}} \def\dbJ{{\db J}}
\def\dbK{{\db K}} \def\dbL{{\db L}}
\def\dbM{{\db M}} \def\dbN{{\db N}}
\def\dbO{{\db O}} \def\dbP{{\db P}}
\def\dbQ{{\db Q}} \def\dbR{{\db R}}
\def\dbS{{\db S}} \def\dbT{{\db T}}
\def\dbU{{\db U}} \def\dbV{{\db V}}
\def\dbW{{\db W}} \def\dbX{{\db X}}
\def\dbY{{\db Y}} \def\dbZ{{\db Z}}

\font\teneusm=eusm10  \font\seveneusm=eusm7 
\font\fiveeusm=eusm5 
\newfam\eusmfam 
\textfont\eusmfam=\teneusm 
\scriptfont\eusmfam=\seveneusm 
\scriptscriptfont\eusmfam=\fiveeufm 
\def\scr#1{{\fam\eusmfam\relax#1}}


%Upper-case Script Letters:

\def\scrA{{\scr A}}   \def\scrB{{\scr B}}
\def\scrC{{\scr C}}   \def\scrD{{\scr D}}
\def\scrE{{\scr E}}   \def\scrF{{\scr F}}
\def\scrG{{\scr G}}   \def\scrH{{\scr H}}
\def\scrI{{\scr I}}   \def\scrJ{{\scr J}}
\def\scrK{{\scr K}}   \def\scrL{{\scr L}}
\def\scrM{{\scr M}}   \def\scrN{{\scr N}}
\def\scrO{{\scr O}}   \def\scrP{{\scr P}}
\def\scrQ{{\scr Q}}   \def\scrR{{\scr R}}
\def\scrS{{\scr S}}   \def\scrT{{\scr T}}
\def\scrU{{\scr U}}   \def\scrV{{\scr V}}
\def\scrW{{\scr W}}   \def\scrX{{\scr X}}
\def\scrY{{\scr Y}}   \def\scrZ{{\scr Z}}

\def\gr#1{{\fam\eufmfam\relax#1}}

%Euler Fraktur letters (German)
\def\grA{{\gr A}}	\def\gra{{\gr a}}
\def\grB{{\gr B}}	\def\grb{{\gr b}}
\def\grC{{\gr C}}	\def\grc{{\gr c}}
\def\grD{{\gr D}}	\def\grd{{\gr d}}
\def\grE{{\gr E}}	\def\gre{{\gr e}}
\def\grF{{\gr F}}	\def\grf{{\gr f}}
\def\grG{{\gr G}}	\def\grg{{\gr g}}
\def\grH{{\gr H}}	\def\grh{{\gr h}}
\def\grI{{\gr I}}	\def\gri{{\gr i}}
\def\grJ{{\gr J}}	\def\grj{{\gr j}}
\def\grK{{\gr K}}	\def\grk{{\gr k}}
\def\grL{{\gr L}}	\def\grl{{\gr l}}
\def\grM{{\gr M}}	\def\grm{{\gr m}}
\def\grN{{\gr N}}	\def\grn{{\gr n}}
\def\grO{{\gr O}}	\def\gro{{\gr o}}
\def\grP{{\gr P}}	\def\grp{{\gr p}}
\def\grQ{{\gr Q}}	\def\grq{{\gr q}}
\def\grR{{\gr R}}	\def\grr{{\gr r}}
\def\grS{{\gr S}}	\def\grs{{\gr s}}
\def\grT{{\gr T}}	\def\grt{{\gr t}}
\def\grU{{\gr U}}	\def\gru{{\gr u}}
\def\grV{{\gr V}}	\def\grv{{\gr v}}
\def\grW{{\gr W}}	\def\grw{{\gr w}}
\def\grX{{\gr X}}	\def\grx{{\gr x}}
\def\grY{{\gr Y}}	\def\gry{{\gr y}}
\def\grZ{{\gr Z}}	\def\grz{{\gr z}}

\def\epsbar{\bar{\epsilon}}
\def\thetadot{{\dot{\theta}}}
\def\Pidot{{\dot{\Pi}}}
\def\GS{\hbox{\rm GS}}
\def\Ebar{\bar{E}}
\def\twosquares{\hbox{$\square\kern-2.255pt\square$}}
\def\splitsquare{\hbox{$\square$}\kern-7.79pt%
  \raise6pt\hbox{$\square$}}
\def\threesquares{\hbox{$\square$}\kern-7.79pt%
  \raise6pt\hbox{$\square$}\kern-7.79pt%
  \raise12pt\hbox{$\square$}}
\def\rhobar{{\bar{\rho}}} 
\def\abar{\bar{a}}
\def\chibar{\bar{\chi}}
\def\sigmabar{\bar{\sigma}}
\def\omebar{\bar{\omega}} 
\def\Fhat{\widehat{F}}
\def\omehat{\widehat{\omega}}
\def\etabar{\bar{\eta}}
\def\SU{\hbox{\rm SU}}
\def\Htil{\widetilde{H}}

%\overfullrule=0pt

\parindent=25pt
\line{\dun --- DRAFT --- \hfill{\rm IASSNS-HEP-97/72}}

\bigskip\bigskip
\centerline{\boldtitlefont Lecture 15}
\bigskip
\centerline{\smallboldtitle X. Supersymmetry and Supergravity}

\medskip
\centerline{Eric D'Hoker}

\frenchspacing

\dspace
\bigskip
On many occasions during the discussion of superstring
theory we have obtained results that are consistent
with the presence of space-time supersymmetry.
These include the counting of the number of bosonic
versus fermionic states at all mass levels, as well as
the study of the spectrum at zero mass.
In this section, we shall exhibit supersymmetry on the
spectrum and on the scattering amplitudes in an
explicit fashion.
We begin by completing the study of the RNS formulation of
 string theory, and
use the Ramond (fermion) vertex operator to produce a
conserved supercharge that generates space-time
supersymmetry.
Next, we introduce and study the Green-Schwarz
formulation of superstring theory, in which space-time
supersymmetry is manifestly realized.
Finally, we exhibit the low energy field theory
approximations to the various superstring theories we have
introduced, and
identify them with known supergravity theories, in
which we have local or gauged supersymmetry.
We show how the Green-Schwarz formulation of
superstrings may be used to couple strings to
supergravity in a  covariant way.

\bigskip\noindent
{\bf \S 10.1.} {\bf Global space-time supersymmetry in 
the $\RNS$ Formulation}

In the previous chapter, we have provided a complete
prescription for the vertex operators and for the calculation
of transition amplitudes in the $\RNS$ formulation of
superstrings.
The $\RNS$ formulation does not exhibit space-time
supersymmetry in a manifest fashion however:
the construction of bosonic and fermionic vertex
operators is asymmetrical.

Nonetheless, it is possible to construct a conserved
current $j_\alpha(z)$ on the worldsheet, whose charge
$Q_\alpha$, defined by
$$
Q_\alpha=\oint\,{dz\over 2\pi}\,j_\alpha(z)\leqno{(10.1)}
$$
obeys the supersymmetry algebra of $10$-dimensional
space-time:
$$
\eqalign{
[P_\mu, P_\nu] &=[P_\mu,Q_\alpha]=0\cr
\{Q_\alpha,Q_\beta\} &=2\,\Gamma_{\alpha\gamma}^\mu
(C^{-1})^\gamma{}_\beta\,
P_\mu=2\,\gamma_{\alpha\beta}^\mu\,P_\mu\cr}
\leqno{(10.2)}
$$
Here $P_\mu$ is the translation operator, and we use
the Dirac algebra defined in Lecture~9.
For $N=2$ supersymmetric theories such as Type II A, B,
there will be an independent supercharge $Q'$ arising
from the $\zbar$ sector of the theory, whose discussion is
analogous to that of $Q$.
The current $j_\alpha(z)$ is related to the Ramond
vertex operator.
To see this, we notice that $Q_\alpha$ should map states
of given $M^2$ into states of the same $M^2$, and should
transform as a spinor.
The $\R$ vertex operator, at zero space-time momentum,
precisely has these properties.
Thus, we identify $j_\alpha(z)$ with
$$
\leqalignno{
j_\alpha(z) &=e^{-\phi/2}(z) S_\alpha(z) &(10.3)\cr
\noalign{\hbox{of ghost number $-1/2$ or}}
j_\alpha(z) &=e^{\phi/2}(\gamma_\mu)_{\alpha\beta}
S'{^{\beta}}(z)\partial_z x^\mu &(10.4)\cr}
$$
of ghost number $1/2$.
Both are holomorphic and, with the help of
$$
\leqalignno{
e^{-\phi/2}(z) e^{\phi/2}(w) &\sim
  {1\over (z-w)^{-1/4}} &(10.5)\cr
S_\alpha(z) S'{^{\beta}} &\sim
  {\delta_\alpha^\beta\over(z-w)^{5/4}}\cr
\noalign{\hbox{we find}}
j_\alpha(z) j_\beta(w) &\sim
(\gamma_\mu)_{\alpha\beta}\partial_z x^\mu{1\over
z-w} &(10.6)\cr}
$$
and recover the superalgebra of $Q_\alpha$ and $P_\mu$
upon integration.

Under the action of $Q_\alpha$, vertex operators in the
NS and R sectors are mapped into one another.
Considering left moving vertex operators for massless
states, for example, 
$$
\eqalign{
W_{\NS}(z,\eps,k) &=\eps_\mu\partial_z
  x^\mu e^{ik\cdot x}\cr
W_{\R}(z,u,k) &=e^{-\phi/2}S_\alpha u^\alpha
  e^{ik\cdot x}\cr}
\leqno{(10.7)}
$$
where we have considered $W_{\NS}$ in the picture with
zero ghost number and $W_{\R}$ in the picture with ghost
number $-1/2$.
We have 
$$
\eqalign{
[\zeta^\alpha Q_\alpha, W_{\NS}(z,\eps,k)]
&=W_{\R}(z,u,k)\cr
[\zeta^\alpha Q_\alpha,W_{\R}(z,v,k)]
&=W_{\NS}(z,\eta,k)\cr}
\leqno{(10.8)}
$$
with
$$
\eqalign{
u^\alpha &=k^\mu\Eps^\nu(\gamma_{\mu\nu})^\alpha
{}_\beta \zeta^\beta\cr
\eta^\mu
&=v^\alpha(\gamma^\mu)_{\alpha\beta}\zeta^\beta\cr}
\leqno{(10.9)}
$$
Pictorially, we may represent this action by
$$
\vbox{\epsfxsize=3.0in\epsfbox{fig10.1.eps}}
$$
and similarly for $W_{\R}$ insertions.

{}From the action of supersymmetry, we obtain several
important relations between transition amplitudes.
For the one-point function, we have
$$
\vbox{\epsfxsize=3.0in\epsfbox{fig10.2.eps}}
$$
by conservation of the current $j_\alpha(z)$.
The only vertex operator that enters is the dilaton at
zero momentum, and the above argument now shows
directly that this amplitude must vanish, as we have
indeed established to tree and one-loop level, by explicit
calculation.

For the two point function, we obtain
$$
\vbox{\epsfxsize=4.5in\epsfbox{fig10.3.eps}}
$$
By chiral symmetry, the $W_F$ two point function must
vanish for massless fermions, 
and hence the $W_B$ two point function must
also vanish, as we have indeed established to tree and
one-loop level, by explicit calculation.

It is hoped that the use of the supersymmetry currents
will ultimately allow one to show that superstring
theory in the $\RNS$ formulation is finite ot all
orders, but subtleties with the superghost system have
prevented us so far from carrying out this program in
full.

\bigskip\noindent
{\bf \S 10.2.} {\bf The Green-Schwarz formulation}

The string degrees of freedom in the Green-Schwarz (GS)
formulation of superstring theory (in $D$ space-time
dimensions) are

\medskip
\item{(1)}
the bosonic coordinate $x^\mu$,\quad
$\mu=0,1,\ldots,D-1$

\smallskip
\item{(2)}
the fermionic coordinate $\theta^{\alpha I}$

\medskip\noindent
The $x^\mu$ coordinates are as before.
The $\theta^{\alpha I}$ coordinates are spinors under
$\Spin(1,D-1)$, labeled by the spin index $\alpha$, with
$I=1,2,\ldots,N$ describing the degree of extension of
the supersymmetry algebra of the Poincar\'e group.
$\theta^{\alpha I}$ are Grassmann valued worldsheet
scalars.
For simplicity, we shall assume that $\Spin(1,D-1)$
admits Majorana spinors, and we shall use Majorana
notation.
(For $D=10$, $\theta^{\alpha I}$ are Majorana-Weyl
spinors, with $N=1$ for Type I and heterotic and $N=2$ for
Type II.)

Poincar\'e supersymmetry acts on the coordinates
$x^\mu$ and $\theta^{\alpha I}$ as follows
(we use the conventions of \S9):
$$
\left\{
\eqalign{
\delta\theta^I &=\eps^I\cr
\delta\thetabar^I &=\epsbar^I\cr
\delta x^\mu &=i\epsbar^I\Gamma^\mu\theta^I\cr}
\right.
\leqno{(10.10)}
$$
where $\eps^I$ is a constant set of Majorana spinors,
labeled by $I=1,\ldots,N$.

We seek a worldsheet action in terms of $x^\mu$ and
$\theta^I$ for the various superstring theories, that
possesses manifest space-time supersymmetry as defined
above.

As a warm-up, we obtain the action for a massless point
particle, propagating on a worldline $L$, parametrized
by $\tau$ (and $\,\,\,\dot{}={d\over d\tau}$).
To do so, we need a generalization of $\xdot^\mu$ that
has good supersymmetry transformation properties.
The following combination is easily seen to be {\it
invariant} under supersymmetry
$$
\Pi^\mu\equiv \xdot^\mu-i\thetabar^I\Gamma^\mu
\thetadot^I
\leqno{(10.11)}
$$
and the following worldline action is susy invariant as
well as $Diff(L)$-invariant:
$$
S=\int\nolimits_{L}d\tau\,e^{-1}(\tau)(\xdot^\mu-
i\thetabar^I\Gamma^\mu\thetadot^I)^2
\leqno{(10.12)}
$$
Here, $e$ is the worldline metric.
The variational equations are
$$
\Pi^2=0,\quad
\Pidot^\mu=0,\quad
\Gamma_\mu\Pi^\mu\thetadot^I=0
\leqno{(10.13)}
$$
Interpreting $\Pi^\mu$ as particle momentum, we find
that the particle propagates with constant momentum,
that it is massless, and that because
$(\Gamma\cdot\Pi)^2=0$, half of the $\theta^J$'s are
unconstrained.
This last property is due to the fact that $S$ exhibits
a further symmetry, called {\it $\kap$-symmetry}:
$$
\cases{
\delta\theta^I i\Gamma\cdot\Pi\kap^I &\cr
\delta x^\mu =i\thetabar^I\Gamma^\mu\delta\theta^I &\cr
\delta e =4e\dot{\thetabar}^I \kap^I &\cr}
\leqno{(10.14)}
$$
The presence of $\kap$-symmetry guarantees that only half of
the $\theta$'s are physical degrees of freedom of the particle.

We now generalize the above construction to obtain a
worldsheet action with space-time supersymmetry.
There is an obvious generalization of momentum, which
is space-time supersymmetric: 
$$
\Pi_m{}^\mu\equiv\partial_m x^\mu-i
\thetabar^I\Gamma^\mu\partial_m\theta^I
\leqno{(10.15)}
$$
Thus there is an obvious worldsheet action (with
worldsheet metric $g$, which does not transform under
supersymmetry)
$$
S_1[x,\theta]={1\over 8\pi}\int\nolimits_{\Sigma}
d\mu_g g^{mn}\Pi_m{}^\mu\Pi_n{}^\nu
\eta_{\mu\nu}
\leqno{(10.16)}
$$
invariant under $Diff(\Sigma)$, $Weyl(\Sigma)$ and
space-time supersymmetry.
However, the $\kap$-symmetry of the point particle does
not extend to $S_1$.
This is a problem, because in $D=10$, $N=1$
superstrings, e.g. we will now have $8$ physical
degrees of freedom of $x^\mu$, but $16$ for $\theta$.
The $\kap$-symmetry is required to have the correct
on-shell particle content of the string spectrum.

When $N=1,2$ or $N=1$, the action $S_1$ may be
modified to
$S=S_1+S_2$ in such a way that $S$ is $Diff(\Sigma)$,
$Weyl(\Sigma)$, space-time supersymmetric, but also
invariant under $\kap$-supersymmetry.
The additional part $S_2$ is found to be
$$
\eqalign{
S_2[x,\theta]={1\over 4\pi}\int\nolimits_{\Sigma}
\bigl\{ &-i\,dx^\mu\w(\thetabar^1\Gamma_\mu d\theta^1-
\thetabar^2\Gamma_\mu d\theta^2)\cr
&\thetabar^1\Gamma_\mu d\theta^1\w
\thetabar^2\Gamma^\mu d\thetabar^2\bigr\}\cr}
\leqno{(10.17)}
$$
Actually, the space-time supersymmetry of $S_2$ is not
automatic.
Restricting to $N=1$ only for simplicity, the
$\eps$-transform of $S_2$ yields
$$
\delta S_2={1\over 4\pi}\int\nolimits_{\Sigma}
(\epsbar\Gamma^\mu d\theta\w\thetabar\Gamma_\mu
d\theta-i\,dx^\mu\w\epsbar\Gamma_\mu d\theta)
\leqno{(10.18)}
$$
The second term in $\delta S_2$ is a total derivative and
leads to no further requirements.
The vanishing of the first term in $\delta S_2$
requires that a certain
$\Gamma$-matrix identity be satisfied:
$$
\Gamma^\mu \psi_1(\psibar_2\Gamma_\mu\psi_3)
+\Gamma^\mu\psi_2(\psibar_3\Gamma_\mu\psi_1)
+\Gamma^\mu\psi_3(\psibar_1\Gamma_\mu\psi_2)=0
\leqno{(10.19)}
$$
This identity can be realized in the following situations:
$$
\matrix{
\displaystyle
\hbox{(i)}\qquad \hfill &D=10,\hfill &\quad\theta\qquad 
&\hbox{Majorana-Weyl}\hfill 
  &\hbox{8 $\theta$'s, 8 $x$'s}\hfill\cr
\noalign{\medskip}
\hbox{(ii)}\qquad \hfill &D=6,\hfill &\quad\theta\qquad 
  &\hbox{Weyl}\hfill &\hbox{4 $\theta$'s, 4 $x$'s}\hfill\cr
\noalign{\medskip}
\hbox{(iii)}\qquad \hfill &D=4,\hfill &\quad\theta\qquad 
  &\hbox{Majorana}\hfill &\hbox{2 $\theta$'s,
  2 $x$'s}\hfill\cr
\noalign{\medskip}
\hbox{(iv)}\qquad \hfill &D=3,\hfill &\quad\theta\qquad  
&\hbox{Majorana}\hfill &\hbox{1 $\theta$, 1 $x$}\hfill\,\,.
}
\leqno{(10.20)}
$$
(Note, for $D=6$, certain formulas must be adapted,
since the corresponding spinors $\theta$ are not Majorana.)

The $\kap$-symmetry may be expressed in terms of a
worldsheet vector $\kap_m^J$, which is also a
space-time spinor:
$$
\left\{
\eqalign{
\delta\theta^I &=2i\,\Gamma\cdot\Pi_m\kap^{Im}\cr
\delta x^\mu &=\thetabar^I \Gamma^\mu\delta\theta^I\cr
\delta g_{mn} &=\qquad\ldots\cr}
\right.
\leqno{(10.21)}
$$
The worldsheet vector $\kap_m^I$ produces a $1$-form
$\kap^I\equiv \kap_m^Id\xi^m$ with the following
self-duality properties
$$
\left\{
\eqalign{
{}^*\kap^1 &=-i\,\kap^1\cr
{}^*\kap^2 &=i\,\kap^2\cr}
\right.
\leqno{(10.22)}
$$
The $\kap$-symmetry closes onto a combination involving a
$\kap$-symmetry, a
$ Diff(\Sigma)$, $Weyl(\Sigma)$ transformation, and an
additional bosonic symmetry that has no dynamical
consequences.

On a worldsheet with flat metric $g$, the field
equations for $g$, $x$ and $\theta$ are
$$
\matrix{
{\scriptscriptstyle \bullet} 
  &\quad \Pi_z{}^2=\Pi_{\zbar}{}^2=0\hfill\cr
\noalign{\medskip}
{\scriptscriptstyle \bullet} 
  &\quad \partial_z(\partial_{\zbar}x^\mu
  -2i\thetabar^1\Gamma^\mu\partial_{\zbar}\theta^1)
  +\partial_{\zbar}(\partial_z x^\mu-2i\thetabar^2
  \Gamma^\mu\partial_z\theta^2)=0\hfill\cr
\noalign{\medskip}
{\scriptscriptstyle \bullet} 
  &\quad\Gamma_\mu\Pi_z^\mu\partial_{\zbar}
  \theta^1=\Gamma_\mu\Pi_{\zbar}^\mu\partial_z
  \theta^2=0\,\,.\hfill
}
\leqno{(10.23)}
$$
These equations of motion are highly non-linear.
As a result, quantization of the $\GS$ string in
covariant form is fraught with difficulties, arising in
particular from the fact that $\theta$'s do not have a
standard kinetic term.
No satisfactory approach exists to date.


\bigskip\noindent
{\bf \S 10.3.} {\bf Lightcone gauge quantization of the $\GS$ Formulation}

In the lightcone gauge, the non-linearities of the
$\GS$ field equations disappear, and quantization may
be carried out in complete parallel with the lightcone
quantization of the $\RNS$ string (Problem Set \#9).
As usual, we impose
$$
x^+=q^+ +p^+(z+\zbar)
\leqno{(10.24)}
$$
and assume that $p^+\not=0$.
(If $p^+=0$, then by positivity of energy, we must have
$p=0$, and we only have the vacuum state.)
Hence
$$
\Pi_z^+=\Pi_{\zbar}^+=p^+
\leqno{(10.25)}
$$
On-shell, we have $\Pi_z^2=\Pi_{\zbar}^2=0$, so at a
given point on the worldsheet, we may rotate $\Pi$ so
that $\Pi^-=\Pi^i=0$, with only $p^+$ remaining as the
only non-vanishing component.
At that point, the $\kap$-transformation on $\theta^I$
becomes
$$
\delta\theta^I=2i\,\Gamma^-p^+(\kap^{Iz}+\kap^{I\zbar})\,\,.
\leqno{(10.26)}
$$
Thus, by suitable $\kap$-transformation, all
$\theta^I$'s in the image of $\Gamma^-$ may be rotated
away.
Thus, by $\kap$-symmetry, we may impose the additional
gauge choice
$$
\Gamma^+\theta^I=0
\leqno{(10.27)}
$$
This choice implies that the equations for
$\partial_{\zbar}\theta^1$ and $\partial_z\theta^2$
simplify; to see this, multiply each to the left by
$\Gamma^+$, and use $\{\Gamma^+,\Gamma^-\}=2$:
$$
\partial_{\zbar}\theta^1=\partial_z\theta^2=0
\leqno{(10.28)}
$$
As a result, the equation for $x^\mu$ linearizes as well
$$
\partial_z \partial_{\zbar}x^\mu=0\,\,.
\leqno{(10.29)}
$$
The only remaining non-linearity is the mass-shell
constraint, given by
$$
\eqalign{
\Pi_z^2 &=2p^+\Pi_z^-+\Pi_z^i\Pi_z^i=0\cr
\Pi_{\zbar}^2 &=2p^+\Pi_{\zbar}^-+\Pi_{\zbar}^i
  \Pi_{\zbar}^i=0\,\,.\cr}
\leqno{(10.30)}
$$

As always in the lightcone gauge, the Lorentz group is
reduced (in $D=10$) to $\SO(8)$, under which $x^i$
transforms as a vector $8_v$, while $\theta^I$ (or
more precisely, its eight dimensional restriction
$S^I$) transforms as a spinor of $\SO(8)$: \ $8_s$ or
$8_c$.
These representations are permuted into one another
under {\it triality} of $\SO(8)$.

The lightcone quantization of the $\GS$ string is
equivalent to the lightcone quantization of the $\RNS$
superstring.
Let us exhibit how this works on a cylinder, where the
$\RNS$ field $\psi_+^\mu$ has periodic boundary
conditions, suitable to a space-time spinor state.
In complete parallel with the covariant construction of
the Ramond vertex operator, we may construct $S^{\alpha
1}(z)$ from $\psi_+^i(z)$ (and $S^{\alpha 2}(\zbar)$
out of $\psi_-^i(\zbar)$).
One begins by bosonizing the Cartan generators of
$\SO(8)$ in the $\RNS$ representation:
$$
\psi_+^{2i-1}(z)\psi_+^{2i}(z)=-i\partial_z
\phi^i\qquad i=1,2,3,4
\leqno{(10.31)}
$$
Recall that the fields $\psi_+^i$ then have an
exponential form in terms of $\phi^i$, with $\pm$
weights
$$
{1\over\sqrt{2\,\,}}\,(\psi_+^{2i-1}\pm i\psi_+^{2i})
(z)=e^{\pm i\phi^i}
\leqno{(10.32)}
$$
and the spin fields are also exponentials with 
$\pm\,{1\over 2}$ weights:
$$
\eqalign{
S_\alpha^1(z) &=e^{i\alpha\cdot\phi(z)}\cr
\alpha &=\left(\pm\,\Half,\,\,\pm\,\Half,\,\,\pm\,
\Half,\,\,\pm\,\Half\right)\,\,
}
\leqno{(10.33)}
$$
The conformal weight of $S_\alpha^1(z)$ is now ${1\over 2}\,
\alpha^2={1\over 2}\,$!
This means that the lightcone gauge has made
$S_\alpha^1(z)$ into a worldsheet spinor.

\bigskip\noindent
{\bf \S 10.4.} {\bf Flat superspace $\GS$ formulation}

A more intrinsic formulation of the $\GS$ worldsheet
action is obtained by expressing it in superspace
We concentrate on the $D=10$, $N=2$ case, for which
superspace is of dimension $(10\vert 32)$, with $32$
Majorana spinor components.
Coordinates are $X^M=(x^m,\theta^{\mu I})$, where
$m=0,1,\ldots,9$ and $\mu=1,\ldots,16$ while $I=1,2$
labels the two irreducible Majorana-Weyl (MW) multiplets.

Flat superspace is characterized by a frame $E^A=dX^M
E_M{}^A$ where $A=(a,\alpha I)$ are frame indices,
with $a=0,1,\ldots,9$ labeling the defining
representation of $\SO(1,9)$, $\alpha$ labeling the
MW spinor of $\Spin(1,9)$ and $I=1,2$.
The $\Spin(1,9)$ connection vanishes in flat
superspace, and the superderivatives are
$$
\scrD_A=E_A{}^M\partial_M\qquad\qquad
\partial_M\equiv{\partial\over \partial X^M}
\leqno{(10.34)}
$$
and they satisfy
$$
\{\scrD_{\alpha I},\scrD_{\beta
J}\}=2(\Gamma^a)_{\alpha\gamma}(C^{-1})^\gamma{}_\beta
\scrD_a\delta_{IJ}
\leqno{(10.35)}
$$
with all others commuting.
The combination $\Pi_m^r$, encountered previously, has
a very natural interpretation as a pull-back of the
superspace frame $E^A$ to the worldsheet.
Indeed
$$
\eqalign{
X^*(E^a) &\equiv E_p{}^a d\xi^p=(\partial_p
  x^a-i\thetabar^I\Gamma^a\partial_p\theta^I)d\xi^p\cr
X^*(E^{\alpha I}) &\equiv E_p{}^{\alpha I}d\xi^p
  =\partial_p\theta^{\alpha I}d\xi^p\cr}
\leqno{(10.36)}
$$
so that $\Pi_p{}^a=E_p{}^a$.
This identification readily allows us to write the first part
of the GS superstring action in superspace:
$$
S_1[X]={1\over 8\pi}\int\nolimits_{\Sigma}
d\mu_g g^{pq}E_p{}^a E_q{}^b \eta_{ab}
\leqno{(10.37)}
$$

The second part of the $\GS$ action may be recast in
the form of a Wess-Zumino-Witten action.
To show that this is indeed the case, we consider the
following $3$-form
$$
H\equiv E^a\w(\Ebar^1\Gamma_a\w E^1-\Ebar^2\Gamma_a \w E^2)
\leqno{(10.38)}
$$
where we have used the notation $\Ebar^I\equiv (E^I)^T
\Gamma^0$ and suppressed spinor index contraction.
To construct a Wess-Zumino-Witten action, we require as
usual that the form $H$ be closed $dH=0$.
This condition implies
$$
\eqalign{
0=dH &=dE^a\w(\Ebar^1\Gamma_a\w E^1-\Ebar^2\Gamma_a\w E^2)\cr
&=-i \thetabar^I \Gamma^a
d\theta^I(\Ebar^1\Gamma_a\w E^1-\Ebar^2\Gamma_a\w E^2)\cr
&=-i(\Ebar^1\Gamma^a E^1+\Ebar^2\Gamma^a E^2)
  (\Ebar^1\Gamma_a\w E^1-\Ebar^2\Gamma_a E^2)\cr
&=-i(\Ebar^1\Gamma^a E^1\Ebar^1\Gamma_a E^1-(1\to 2))\cr
&=-(\Gamma^0\Gamma^a)_{\alpha\beta}
  (\Gamma^0\Gamma_a)_{\gamma\delta}E^{1\alpha}
E^{1\beta}E^{1\delta}-(1\to 2)\,\,\cr}
\leqno{(10.39)}
$$
The last line vanishes since the wedge product of the
$E^1$'s is symmetry in $(\alpha\beta\gamma\delta)$ in
view of the identity (10.19) we used to show that the
$S_2$ action is supersymmetric.

It is easy to show that $H=db$, where $b$ is the
$2$-dimensional integrand of $S_2$.
Indeed,
$$
\eqalign{
H &=(dx^a-i\thetabar^j \Gamma^a d\theta^j)\w
  (d\thetabar^1\Gamma_a d\theta^1-d\thetabar^2\Gamma_a
 d\theta^2)\cr
&=d(dx^a(\thetabar^1\Gamma_a d\theta^1-\thetabar^2
  \Gamma_a d\theta^2))\cr
&\quad -i(\thetabar^1\Gamma^a \theta^1
  d\thetabar^1\Gamma_a d\theta^1-\thetabar^2\Gamma^a
d\theta^2 d\thetabar^2\Gamma_a d\theta^2)\cr
&\quad -i\thetabar^2\Gamma^a d\theta^2 d\thetabar^1
\Gamma_a d\theta^1+\thetabar^1\Gamma^a d\theta^1
  d\thetabar^2\Gamma_a d\theta^2\cr}
\leqno{(10.40)}
$$
Now, the second line again vanishes by the
$\Gamma$-identity (10.19), and the last term is a
total differential:
$$
H=d\{dx^a(\theta^1\Gamma_a
d\theta^1-\thetabar^2\Gamma_a d\theta^2)+i\thetabar^1
\Gamma^a d\theta^1\thetabar^2\Gamma_a d\theta^2\}
\leqno{(10.41)}
$$
Identifying with the term in $S_2$, we find
$$
S_2[x]={-i\over 4\pi}\int\nolimits_{B}
X^*(H)\,\,.
\leqno{(10.42)}
$$

Finally, it is very instructive to exhibit the action
of the $\kap$-symmetry in the superspace formulation.
It turns out that the $\kap$-symmetry is most easily
expressed in terms of a worldsheet scalar (space-time
spinor)
$$
\kap\equiv i\,E_p^a\,\Gamma_a(\kap^{p1}+\kap^{p2})
\leqno{(10.43)}
$$
In terms of this parameter, one has
$$
\eqalign{
\delta X^M E_M{}^a &\equiv \delta X^a=0\cr
\noalign{\bigskip}
\delta X^M E_M{}^1 &=(1+\Gamma)\kap\cr
\delta X^M E_M{}^2 &=(1-\Gamma)\kap
}\leqno{(10.44)}
$$
where the operator $\Gamma$ is defined by
$$
\Gamma\equiv{1\over 2\sqrt{h\,\,}}\Eps^{pq}E_p{}^a
E_q{}^b \Eps_{ab}
\leqno{(10.45)}
$$
and $h$ is such that $\Gamma^2=I$.

\bigskip\noindent
{\bf \S 10.5.} {\bf Supergravity and low energy superstrings}

We have shown in the first part of this chapter that
the $\RNS$ and $\GS$ formulations of superstring
theory exhibit $10$-dimensional (global) Poincar\'e
supersymmetry, in flat Minkowski
space-time $M=\dbR^{10}$.

Actually, superstring theory also exhibits $ Diff(M)$
invariance when formulated on a general manifold $M$.
For the bosonic string, we had shown this in \S{VI} by coupling
bosonic string theory to background fields, including
the metric $G_{\mu\nu}$, the anti-symmetry tensor
field $B_{\mu\nu}$ and the dilaton $\Phi$, in a
$ Diff(M)$ covariant fashion.
For the superstring, we may again couple superstring
theory to $G_{\mu\nu}$, $B_{\mu\nu}$ and $\Phi$ in a
$ Diff(M)$ covariant way.
The relevant ($N=1$ on the worldsheet) supersymmetric
non-linear sigma model for the type II superstrings is
given by
$$
\eqalign{
S[X;G,B,\Phi] &={1\over 4\pi}\int_\Sigma
d\mu_E\scrD_-X^\mu\scrD_+X^\nu
(G_{\mu\nu}(X)+iB_{\mu\nu}(X))\cr
&+{1\over 4\pi}\int_\sigma d\mu_E R_{+-}\Phi(X)\cr}
\leqno{(10.46)}
$$
This worldsheet action defines the $\RNS$ superstring
in an arbitrary space-time manifold $M$, with the
fields $G_{\mu\nu}$, $B_{\mu\nu}$ and $\Phi$.
(For the $N=1$ theories, this sigma model must be
modified: for heterotic theories, one replaces
$\scrD_-$ by $\partial_{\zbar}$ and $R_{+-}$ by
$R_{+\zbar}$, while for Type I theories, the
$B_{\mu\nu}$ coupling will not occur here, but will
result from the $\R$-$\R$ sector.)

Of course, it must be realized that for any of the superstrings,
the fields
$\Gmu$, $\Bmu$ and $\Phi$ do {\it not} correspond to
the full set of massless states discussed in \S{VII} and
VIII.
First of all, there are additional space-time fermions:
the gravitino and the dilatino.
But, there are also additional boson fields: in Type II
theories, these are the $\R$-$\R$ gauge fields; in
heterotic and Type I theories these are the Yang--Mills
bosons.
Following the discussion provided for the study of
bosonic strings, on general manifolds, with general
background fields, we should include in the non-linear
sigma model all space-time background fields
corresponding to massless string states.
The inclusion of fermions or $\R$-$\R$ bose fields is
complicated by the fact that it requires the use of the
Ramond vertex operator.
Nonetheless, let us imagine that such a formulation
were used.
(We shall exhibit it explicitly later, with the use of
the $\GS$ formulation.)

Now, since the superstring theory is covariant under
$ Diff(M)$ as well as under global Poincar\'e
supersymmetry, it is fully expected that it will in
fact be invariant under {\it local supersymmetry
transformations} (also called supergravity
transformations).
This is expected because the composition of $ Diff(M)$
and Poincar\'e supersymmetry transformations does {\it
not} close onto these, but does close onto local
supersymmetry.
(We have already examined in \S{9} how this happens in $D=2$
within the context of super-Riemann surfaces.)
Field theories that are invariant under local
supersymmetry transformations are so-called {\it
supergravities}.

In the low energy approximation, superstring theory can
be approximated by the spectrum and dynamics of its
massless states only, summarized in terms of a {\it
local field theory}.
The massless sector of superstring theory will inherit
invariance under local supersymmetry, and will then be
a supergravity.

The dynamics of the $\NS$-$\NS$ sector of Type \IIA, B
superstrings, in the massless sector, will yield the
part of the corresponding supergravity involving only
the fields $\Gmu$, $\Bmu$ and $\Phi$ (with the fermion
fields and $\R$-$\R$ fields set to zero).
This dynamics may be deduced directly from the above
non-linear sigma model, and results from demanding that
its quantum dynamics be superconformal invariant.
It turns out that the conditions of superconformal
invariance of $S[X;G,B,\Phi]$ to leading order (i.e.
two derivatives or fewer on fermion fields) are
precisely the same as the condition found for the
bosonic string, except that $D=10$ now.
The field equations for $G$, $B$ and $\Phi$ could be
deduced from an action for those fields on the
space-time manifold $M$:
$$
I_1(G,B,\Phi)=-{1\over 2K^2}\int_M
d\mu_G\left\{R_G-4D_\mu\Phi D^\mu\Phi+{1\over 12}H^2
\right\}e^{-2\Phi}
\leqno{(10.47)}
$$
Here, we have expressed the metric $\Gmu$ in the {\it
string} convention, so that an extra overall factor of
the dilaton coupling appears.

For the heterotic strings, the fields $G$, $B$ and
$\Phi$ result from the $\NS$ sector of the left-movers
coupled to the space-time part of the right movers, and
the resulting action is still given by $I_1$.

\bigskip\noindent
{\bf \S 10.6.} {\bf Type II$\,$A, $D=10$, $N=2$ and
$D=11$, $N=1$ supergravities}

We begin by discussing the low energy field theory (at
most 2 derivatives on bose fields, at most $1$ on
fermion fields) associated with Type \IIA superstring
theory, i.e. Type \IIA, $N=2$, $D=10$ supergravity.
The existence of this theory was predicted by Nahm,
based on representation theory of supersymmetry algebras
and was constructed by Cremmer, Julia and Scherk, by
dimensional reduction from the $D=11$, $N=1$
supergravity.
The Type \IIA field content is readily identified from
\S{VII}, 4:
$$
\halign{#\hfill &\quad#\hfill &\quad#\hfill\cr
\null\kern20pt
$\left.\matrix{
\ssbullet\quad \Gmu\hfill\cr
\ssbullet\quad \Bmu\hfill\cr
\ssbullet\quad\Phi\hfill}\,\right\}$ &$\NS$-$\NS$
&\raise15pt\hbox{\vtop{\hbox{space-time metric 
  (equivalently frame $e_m{}^a$)}
 \hbox{anti-symmetric $\NS$-$\NS$ tensor}
\hbox{dilaton}}}\cr
\noalign{\bigskip}
\null\kern20pt
$\left.\matrix{
\ssbullet\quad A_\mu^{(1)}\hfill\cr
\ssbullet\quad A_{\mu\nu\rho}^{(3)}\hfill}\right\}$ 
  &$\R$-$\R$ 
  &\raise10pt\hbox{\vtop{\hbox{graviphoton}
  \hbox{rank $3$ anti-symmetric tensor}}}\cr
\noalign{\bigskip}
\null\kern20pt
$\left.\matrix{
\ssbullet\quad \chi_\mu{}^\alpha\hfill\cr
\ssbullet\quad \lam_\alpha\hfill}\,\,\,\right\}$ 
  &$\R$-$\NS$ and $\NS$-$\R$ 
&\raise15pt\hbox{\vtop{\hbox{$32$ component Majorana gravitino}
\hbox{$32$ component Majorana dilatino}
\hbox{(both contain both chiralities.)}}}\cr}
$$
The action $I_1$ corresponds to the dynamics in which
$A^{(1)}=A^{(3)}=\chi=\lam=0$, and we now seek the
generalization of $I_1$ that also describes the
remaining fields.

It turns out that the easiest way to do this is to
reorganize the field contents in such a way that it
manifestly results from dimensional reduction of an
$D=11$ supergravity theory with $N=1$ supersymmetry.
The precise statement is that the low energy limit of
the $D=11$ supergravity, compactified on a circle with
small radius, is precisely the Type \IIA supergravity.
This  re-organization of field variables is advantageous
because it will turn out that the $N=1$, $D=11$ supergravity
is described by a relatively simple Lagrangian in terms of
relatively few fields.

We begin by showing that the field contents may be
organized according to multiplets of $\Spin(1,10)$.
We use space-time indices $\mu,\nu,\ldots=0,1,\ldots,9$
and $\mubar,\nubar=0,1,\ldots,9,10$.
We denote the compactified coordinate by $x^{10}$.
Upon dimensional reduction on a circle, the only parts
that survive to the low energy approximation of the
$D=10$ theory are the fields that are independent of
$x^{10}$.
Indeed, $x^{10}$-dependence renders the fields massive,
with mass on the order of $1/R$.

A metric tensor in $D=11$ decomposes as follows:
$$
G_{\mubar\nubar}(x^\mu,x^{10})\longrightarrow
G_{\mubar\nubar}(x^\mu)
\leqno{(10.48)}
$$
Under the action of $\SO(1,10)$, this multiplet is
irreducible, but under $\SO(1,9)$, we find
$\twosquares\to\twosquares\oplus\square\oplus\cdot\,\,$, 
so that the
objects of Type \IIA theory that naturally fit into
the $D=11$ metric multiplet are
$$
(\Gmu,A_\mu^{(1)},\Phi)\to G_{\mubar\nubar}
\leqno{(10.49)}
$$
Proceeding analogously for an anti-symmetric rank $3$
tensor
$$
A_{\mubar\nubar\rhobar}(x^\mu,x^{10})\to
A_{\mubar\nubar\rhobar}(x^\mu)
\leqno{(10.50)}
$$
and under $\SO(1,10)$, this tensor is again
irreducible, but decomposes under $\SO(1,9)$ as
$$
\hbox{\threesquares}\,\,
\raise6pt\hbox{$\to$}\,\,
\hbox{\threesquares}\,\,\raise6pt\hbox{$\oplus$}\,\,
\hbox{\splitsquare}\,\,,
$$
so that 
$$
(A_{\mu\nu\rho}^{(3)},\Bmu)\to A_{\mubar\nubar\rhobar}
\leqno{(10.51)}
$$
Thus, all the $\NS$-$\NS$ and $\R$-$\R$ bosons fit into
two fields: $G_{\mubar\nubar}$ and
$A_{\mubar\nubar\rhobar}$.

The spinors also fit together.
The smallest dimensional representation of a spinor of
$\Spin(1,10)$ is a Majorana spinor of dimension $32$.
Under $\Spin(1,9)$, it will reduce into $2$ Majorana-Weyl
spinors of opposite  chirality, and this is precisely
what we need for Type \IIA, where the gravitino as well
as the dilatino are non-chiral Majorana spinors.
Furthermore, the dilatino may be appended to
$\chi_\mu{}^\alpha$ as its $11$-th component:
$$
(\chi_\mu{}^\alpha,\lam_\alpha)\to\chi_{\mubar}^\alpha\,\,.
\leqno{(10.52)}
$$

\medskip

\noindent{\bf \S 10.6.1} {\bf $N=1$, $D=11$ supergravity}

Based upon general arguments by Nahm, the largest
possible dimension $D$ in which supersymmetric
multiplets can exist with $spin\Le 2$ only is $D=11$,
with a single local supersymmetry.
Its field contents are precisely the ones that we have
described above:
$$
\matrix{ 
\displaystyle
\ssbullet\quad G_{\mubar\nubar}\hfill
  &\hfill \hbox{(equiv. } e_{\mubar}{}^{\abar})
  &\hbox{graviton:}\hfill &\hfill\hbox{44 states}\cr
\noalign{\bigskip}
\ssbullet\quad A_{\mubar\nubar\rhobar}\hfill
  &&U(1) \hbox{\rm gauge field:} &\hfill\hbox{84 states}\cr
\noalign{\bigskip}
\ssbullet\quad 
\chi_{\mubar}^\alpha\hfill &&\hbox{gravitino:}\hfill
  &\hbox{128 states}.
}
$$
The dimensional reduction is carried out by setting
$x^{10}$ dependence to zero, and by letting
$G_{\mubar\nubar}$ decompose as
$$
\Gmu\,;\qquad
G_{\mu10}=G_{10\mu}=A_\mu^{(1)}\qquad
G_{10\,10}=e^{-2\Phi}
\leqno{(10.53)}
$$
which yields the action $I_1$ when $A_\mu^{(1)}=0$, and
$A_{\mu\nu\rho}^{(3)}=\chi_\mu^\alpha=\lam_\alpha=0$.
The normalization of the field strength $H=dB$ of $B$
in $I_1$ provides a normalization of the field strength
$F=dA$ in the $D=11$ supergravity theory.
The remaining terms are fixed by local supersymmetry
invariance.

The full Lagrangian is obtained by seeking $N=1$ local
supersymmetry and was established by Cremmer, Julia and
Scherk:
$$
I(G,A,\chi)=-{1\over 2\kap^2}\int_{\Mbar}d\mu_G
\scrL-{\sqrt{2\,\,}\kap\over 3^456}\int_{\Mbar}A\w F\w F
\leqno{(10.54)}
$$
where $\scrL$ is given by
$$
\eqalign{
\scrL &=R_G+{1\over 12}\kap^2F^2+\kap^2\chibar_{\mubar}
\Gamma^{\mubar\nubar\rhobar}D_{\nubar}\chi_{\rhobar}\cr
&+{\sqrt{2\,\,}\kap^3\over 384}\left(
\chibar_{\mubar}\Gamma^{\mubar\nubar\rhobar\sigmabar
\taubar\omebar}\chi_{\omebar}+12\chibar^{\nubar}
\Gamma^{\rhobar\nubar\sigmabar}\chi^{\taubar}\right)
(F+\Fhat)_{\nubar\rhobar\sigmabar\taubar}\cr}
\leqno{(10.55)}
$$
In this action, we have used the standard notation
$$
\Gamma^{\mu_1\ldots\mu_\rho}\equiv
\Gamma^{[\mu_1}\Gamma^{\mu_2}\ldots
\Gamma^{\mu_\rho]}
$$
where the square brackets stand for complete
anti-symmetrization of the indices.
$F=dA$ was defined previously, and
$$
\Fhat=F+\chi\hbox{-terms}
$$
such that the supersymmetry variation of $\Fhat$ does
not contain derivatives of $\eta$ --- the susy
parameter.
$D_{\nubar}$ is the covariant derivative with respect
to ${1\over 2}(\omega_{\Mbar}+\omehat_{\Mbar})$ where
$$
\omehat_{\mubar\abar\bbar}=\omega_{\mubar\abar\bbar}+
{1\over 8}\chibar^{\nubar}
\Gamma_{\nubar\mubar\abar\bbar\rhobar}\chi^{\rhobar}
\leqno{(10.58)}
$$
and $\omega_{\mubar}$ is the spin connection, which is
determined by its equation of motion.
For completeness, we record the supergravity
transformations:
$$
\eqalign{
\delta e_{\mubar}{}^{\abar} &={\kap\over 2}\etabar
  \Gamma^{\abar}\chi_{\mubar}\cr
\delta A_{\mubar\nubar\rhobar} &=-{\sqrt{2\,\,}\over 8}
 \etabar\Gamma_{[\mubar\nubar}\chi_{\rhobar]}\cr
\delta\chi_{\mubar} &={1\over \kap}D'_{\mubar}\eta+
{\sqrt{2\,\,}\over 288}\left(\Gamma_{\mubar}{}^{\nubar
\rhobar\sigmabar\taubar}-8\delta_{\mubar}{}^{\nubar}
\Gamma^{\rhobar\sigmabar\taubar}\right)\Fhat_{\nubar
\rhobar\sigmabar\taubar}\cr}
\leqno{(10.59)}
$$
The covariant derivative $D'$ is here defined with the
connections $\omehat$.

\medskip

\noindent{\bf \S 10.6.2} {\bf Type \IIA supergravity}

The Type \IIA supergravity Lagrangian may now be deduced
from $D=11$ supergravity by the compactification and
dimensional reduction discussed previously.
The reduction of the metric $G_{\mubar\nubar}$ was
already given; the reduction of the
$A_{\mubar\nubar\bar{\kap}}$ field proceeds by
$$
\eqalign{
A_{\mu\nu \kap} &=\kap^{{1\over 4}}A_{\mu\nu \kap}^{(3)}\cr
A_{\mu\nu 10} &={1\over \kap}\,\Bmu\cr}
\leqno{(10.60)}
$$
The bosonic part of the Type \IIA supergravity is then
$$
\eqalign{
I_{\hbox{\rm II A}} &(G,B,\Phi; A^{(1)},A^{(3)})\cr
&=-{1\over 2\kap^2}\int_M d\mu_G\Bigl\{e^{-2\Phi}
  \Bigl(R_G-4D_\mu\Phi D^\mu\Phi+{1\over 12}\,H^2\Bigr)\cr
&\qquad\qquad\qquad\qquad
+\sqrt{\kap\,\,}\,G_A^2+\sqrt{\kap\,\,}\,
{1\over 12}\,F^2\Bigr\}\cr
&-{\sqrt{\kap\,\,}\over 288 \kap^2}\int_M B\w F\w F\cr}
\leqno{(10.61)}
$$
here $G_A=dA^{(1)}$, $H=dB$, $F=dA^{(3)}$.
Notice that the R-R part of the action does
not occur with the dilaton rescaling factor
$e^{-2\Phi}$; thus, these contributions are in some
sense of order $1$ in the string loop expansion.

\bigskip\noindent
{\bf \S 10.7.} {\bf Type \IIB, $D=10$, $N=2$ supergravity}

The Type \IIB field contents are as follows
(see Lecture~7).
$$
\matrix{
\ssbullet\hfill &\Gmu \hbox{(equiv. } e_\mu{}^a)\hfill
  &\hbox{graviton.}\hfill\cr
\ssbullet\hfill &\Bmu+i\,A_{\mu\nu}^{(2)} \hfill
  &\hbox{complex rank 2 antisymmetric tensor.}\hfill\cr
\ssbullet\hfill &\Phi+i\,A^{(0)}\hfill
  &\hbox{complex dilaton-axion.}\hfill\cr
\ssbullet\hfill &A_{\mu\nu K\lam}^{(4)}\hfill 
  &\hbox{real, self-dual } F=dA^{(4)}.\hfill\cr
\noalign{\bigskip}
\ssbullet\hfill &\chi_\mu^\alpha\qquad
  \hbox{(chirality} +) \hfill
  &\hbox{complex Majorana-Weyl spinor gravitino}.\hfill\cr
\ssbullet\hfill &\lam_\alpha \qquad
  \hbox{(chirality} -)\hfill
  &\hbox{complex Majorana-Weyl spinor dilatino}.\hfill
}
$$
Clearly, the field contents do not arise from a simple
dimensional reduction, since the spinor content is
chiral here.
In fact, since the theory contains a self-dual field,
$F$, no off-shell Lorentz-covariant formulation is
known for this theory.

The field equations may be written covariantly though,
either in components or in superspace.
The gauge group is $\Spin(1,9)\times U(1)_R$ and the
complex dilaton-axion field $\Phi+i\,A^{(0)}$ lives in
$\SU(1,1)/U(1)_R$.

\bigskip\noindent
{\bf \S 10.8.} {\bf Type I -- Heterotic, $D=10$, $N=1$
supergravities}

Both for the Type I and for the heterotic theories, 
we have only a single space-time supersymmetry: $N=1$.
However, both theories now also have an $N=1$
super-Yang--Mills multiplet, which we shall have to
include.

The field contents of the $N=1$ {\it
supergravity multiplet} are obtained (say for the
heterotic string) by considering the left-moving part
of Type II, tensored with the right-moving
$10$-dimensional part of the bosonic string.
The corresponding fields are
$$
\matrix{
\ssbullet &\Gmu\hfill
&\kern50pt \hbox{space-time metric}\hfill\cr
\ssbullet &\Bmu\hfill 
&\kern50pt \hbox{real rank $2$ anti-symmetric tensor}\hfill\cr
\ssbullet &\Phi\hfill
&\kern50pt \hbox{real dilaton}\hfill\cr
\ssbullet &\chi_\mu{}^\alpha\hfill
&\kern50pt \hbox{$1$ Majorana-Weyl (real) gravitino
($+$ chirality)}\hfill\cr
\ssbullet &\lam_\alpha\hfill
&\kern50pt \hbox{$1$ Majorana-Weyl (real) dilatino 
($-$ chirality)}\hfill
}
$$
the Lagrangian for this theory exists (since there are
no fields with self-dual field strength) and may be
obtained from the Type \IIA theory by setting
$$
A_\mu^{(1)}=A_{\mu\nu\rho}^{(3)}=0\qquad\qquad
\chi_\mu^\alpha\vrulesub{-}=\lam_\alpha\vrulesub{+}=0
\leqno{(10.62)}
$$
where the $\vrulesub{\pm}$ indicates the chirality.
This truncation is consistent with the field equations of
Type \IIA, since both $A^{(1)}$ and $A^{(3)}$ occur
quadratically in the bosonic Lagrangian, and since the
fermionic Lagrangian does not mix between chiralities.
We limit ourselves to examining only the bosonic
action of the supergravity multiplet, and notice
immediately that it must be just $I_1$.
The fermionic part may be deduced from the $D=11$
supergravity Lagrangian by the above dimensional
reductions and restrictions.

The field contents of the $N=1$ {\it super Yang--Mills
multiplet} are obtained (again for the heterotic
string) by considering the left-moving part of the Type
II string, tensored with the right-moving internal part
of the bosonic string.
The corresponding fields are
$$
\matrix{
\ssbullet\hfill &A &\kern50pt \hbox{gauge field of
$E_8\times E_8$ or $\Spin(32)/Z_2=G$}\hfill\cr
&&\qquad\qquad\qquad\qquad\hbox{(adjoint rep.)}\hfill\cr
\noalign{\bigskip}
\ssbullet\hfill &\psi &\kern50pt \hbox{gaugino of
$E_8\times E_8$ or $\Spin(32)/Z_2=G$}\hfill\cr
&&\qquad\qquad\qquad\qquad\hbox{(adjoint rep.)}\hfill
}
$$
The dynamics are governed by the action (in flat $M$,
with $G=\eta$)
$$
I_{YM}^0=\int_M d\mu_G\left(-{1\over 4}\,
F_{\mu\nu}^a\,F^{a\mu\nu}-{1\over
2}\,\psibar^a\Gamma^\mu(D_\mu\psi)^a\right)
\leqno{(10.63)}
$$
which is {\it global} $N=1$ Poincar\'e supersymmetry
invariant.
Here, we use the notation ($a=1,\ldots,\dim\,G$) and
$$
A=A_\mu^a\,t^a\,dx^\mu\qquad\qquad
F=dA+gA\w A={1\over 2}\,F_{\mu\nu}^a\,
t^a\,dx^\mu\,dx^\nu\,\,.
\leqno{(10.64)}
$$

It remains to construct an action with $N=1$ local
supersymmetry, that combines both the supergravity
multiplet and the super-Yang--Mills multiplet.
First, the action $I_{YM}^0$ is adapted as follows.
$$
I_{YM}={1\over \kap}\int_M d\mu_G e^{-2\Phi}
\left\{-{1\over 4}\,F_{\mu\nu}^a\,F^{a\mu\nu}-{1\over 2}
\,\psibar^a\Gamma^\mu D_\mu\psi^a\right\}
\leqno{(10.65)}
$$
For $G=U(1)$ gauge theory, $I_1+I_{YM}+$ the fermionic
terms of the supergravity multiplet yield an $N=1$ susy
action.
For non-Abelian $G$, this is not so, and the form of
the $B$-field strength $H$ in $I_1$ must be modified as
follows:
$$
\Htil=dB-{\kap\over\sqrt{2\,\,}}\,\omega_3(A)
\leqno{(10.66)}
$$
Here, $\omega_3(A)$ is the so-called Chern-Simons form
of degree $3$:
$$
\omega_3(A)\equiv\tr\left(A\w dA+{2\over 3}
\,gA\w A\w A\right)\,\,, \leqno{(10.67)}
$$
with the properties that
$$
\eqalign{&d\omega_3(A)=\tr\,F\w F\cr
&\delta_\Lam \omega_3(A)=\tr(d\Lam\w dA)\qquad
\hbox{under $\delta_\Lam A=d\Lam+[A,\Lam]$}\,\,.}
\leqno{(10.68)}
$$
To render $\Htil$ gauge invariant under non-Abelian
transformation in $G$, $B$ must transform as
$$
\delta_\Lam B={\kap\over\sqrt{2\,\,}}\,\tr\Lam dA
$$
The supergravity $+$ super Yang--Mills action is now (for
bosons)
$$
I=-{1\over 2K^2}\int_M d\mu_G
\left\{e^{-2\Phi}\left(R_G+4 D_\mu \Phi D^\mu\Phi-
{1\over 12}\,\Htil^2\right)+e^{-2\Phi}
\left(-{1\over 4}\,F_{\mu\nu}^a
F^{a\mu\nu}\right)\right\}
\leqno{(10.69)}
$$
where the first part coincides with $I_1$, except for
the fact that $H$ has been replaced with its
$G$-invariant version $\Htil$.

\bigskip\noindent
{\bf \S 10.9.} {\bf Superspace formulation of
supergravities in $D=11$ and $D=10$}

In \S{10.6}, \S{10.7} and \S{10.8},
we have discussed supergravities in
$D=11$ and $D=10$ in the component formulation.
While supersymmetry invariance transformations of these
actions are known, there is the problem that the
composition of two supergravity transformations closes
onto $Diff(M)$ only upon the use of the equations of
motion.
Also, the supergravity invariance is not manifest.

The superspace approach provides a formulation in which
$Diff(M)$ and local supersymmetry transformations
appear on the same footing and are united into the
group of diffeomorphisms of superspace.
However, the superspace formulation in $D=11$ and
$D=10$ supergravity --- and super Yang-Mills-theory ---
is known only for the equations of motion.
No superspace action is known that describes the
off-shell configurations as well.

To exhibit local supersymmetric covariant coupling of
superstrings to supergravity, it is most convenient to
make use of the superspace formulation of supergravity,
as we shall show in the next section.
For this issue, the fact that only the on-shell
formulation is available causes no problem.
In fact, already superstring theory itself, and as we
showed previously, Type \IIB supergravity, do not admit
an off-shell formulation; it is thus perhaps not so
surprising that superspace supergravity does not
either.

\medskip\noindent
{\it $N=1$, $D=11$ supergravity in superspace}

{}From many points of view, the $D=11$ theory is the
simplest theory, and we shall begin by giving its formulation.
The superspace data are as follows:
local supercoordinates $X^M=(x^m,\theta^\mu)$, with
$m=0,1,\ldots,10$ and $\mu=1,\ldots,32$ parametrize a
real supermanifold $M$ of dimension $(11\vert 32)$.
The fields are most conveniently expressed as
$$
\eqalign{
\ssbullet\hfill &\quad E^A=dX^ME_M{}^A\cr
\ssbullet\hfill &\quad \Omega_A{}^B=dX^M\Omega_{MA}
  {}^B \cr
\ssbullet\hfill &\quad X={1\over 3!}E^C\w E^B\w 
E^A X_{ABC} 
}
\leqno{(10.70)}
$$

\medskip\noindent
Here, $A$, $B$, $C$ sum over frame indices: 
$(a,\alpha)$, $(b,\beta)$, $(c,\gamma)$
$a,b,c=0,1,\ldots 10$;
$\alpha,\beta,\gamma=1,\ldots,32$, and transform under
the defining representation and the Majorana spinor
representations of $\Spin(1,10)$, respectively.
The field $E^A$ is a frame, belonging to $\Omega^1(M)$;
the field $\Omega_A{}^B$ is a $\Spin(1,10)$ connection,
which admits the following decomposition under
$\Spin(1,10)$:
$$
\matrix{ 
\displaystyle
\Omega_{ab}=-\Omega_{ba}\hfill &\Omega_{\alpha b}=
  \Omega_{\alpha\beta}=0\hfill\cr
\noalign{\medskip}
\Omega_{\alpha\beta}=\fourth(\Gamma^{ab})_{\alpha\beta}
\Omega_{ab} \hfill
}
\leqno{(10.71)}
$$
Finally $X$ is a $3$ form in superspace, and a
$\Spin(1,10)$ invariant.
It contains the rank $3$ field
$A_{\mubar\nubar\rhobar}^{(3)}$.

Curvature, torsion and $X$ field strength are defined
by
$$
\matrix{
T^A &\equiv dE^A+E^B\w\Omega_B{}^A
&\quad &\left(\equiv \half\,E^c\w E^BT_{BC}{}^A\right)\cr
R_A{}^B &\equiv d\Omega_A{}^B+\Omega_A{}^C\w\Omega_C{}^B
  &\quad&\left(\equiv\half\,E^D\w E^c R_{CD,A}{}^B\right)\cr
H &\equiv d X &\quad&\left(\equiv{1\over 4!}\,E^D\w E^C\w 
E^B\w E^A\w H_{ABCD}\right)\,\,.
}
\leqno{(10.72)}
$$
These fields satisfy Bianchi identities

$$
\matrix{
&\hbox{\rm (a)} &\quad &DT^A=E^B\w R_B{}^A 
  &\qquad\qquad & &\qquad\qquad\qquad &\cr
&\hbox{\rm (b)}   &\quad &DR_A{}^B=0 &\qquad\qquad & 
  &\qquad\qquad\qquad& \cr
&\hbox{\rm (c)}   &\quad &dH=0. &\qquad\qquad & &\qquad\qquad\qquad &
}
\leqno{(10.73)}
$$
written in differential form notation.
It is most convenient to write these Bianchi identities
out in terms of the $\Spin(1,10)$ tensors $T_{AB}{}^C$,
$R_{AB,C}{}^D$ and $H_{ABCD}$; (parentheses
$(ABC\ldots\,\,)$ stand for cyclic, graded summation of
indices)
$$
\matrix{
&\hbox{\rm (a)}
&\quad &\sum\limits_{(ABC)}(R_{ABC}{}^D-D_A T_{BC}{}^D
-T_{AB}{}^E T_{EC}{}^D)=0\cr
&\hbox{\rm (b)}
&\quad &\sum\limits_{(ABC)} (D_A R_{BCD}{}^E+
T_{AB}{}^F R_{FCD}{}^E)=0\cr
&\hbox{\rm (c)}
&\quad &\sum\limits_{(ABCDE)}(D_A H_{BCDE}
+T_{AB}{}^F H_{FCDE})\cr
& & &\qquad\qquad\qquad
  -\sum\limits_{(ADBEC)}(-)^{BC+BD+CE}T_{AD}{}^F H_{FBEC}=0
}
\leqno{(10.74)}
$$

\medskip\noindent
This geometry is too general, and contains many more
fields than are necessary for the $D=11$ supergravity theory
that we constructed in components.
We must impose suitable ``constraint equations''.
(We already encountered this problem in $D=2$, $N=1$
supergravity, where certain constraints were also to be
imposed, see \S{9.1}.)
The choice of constraints is dictated by the structure
of the component fields that one expects to remain
after the constraints have been imposed, by
dimensional analysis, and of course by Lorentz
invariance.
We shall limit ourselves here to quoting the results
from Brink and Howe (1980):
$$
\eqalign{
T_{\alpha\beta}{}^\gamma &=T_{\alpha b}{}^c
=T_{ab}{}^c=0;\,\,
T_{\alpha\beta}{}^c=-i(\Gamma^c)_{\alpha\beta}\cr
H_{\alpha\beta\gamma\delta} &=H_{\alpha\beta\gamma
c}=H_{\alpha bcd}=0;\,\,
H_{\alpha\beta cd}=i(\Gamma_{cd})_{\alpha\beta}\cr}
\leqno{(10.75)}
$$
It can be shown that all the torsion constraints can be
solved in terms of a single superfield $H_{abcd}$.
We shall not list out the various components here, but
limit ourselves to providing the two field equations
that result.
We introduce
$$
R_{ab}=\eta^{cd}R_{acbd}\qquad
R=\eta^{ab}R_{ab}
\leqno{(10.76)}
$$
and have the equations of motion
$$
\matrix{ 
\displaystyle
R_{ab}={1\over 2}\,\eta_{ab}R &=-{1\over 12}\,
H_{acde}H_b{}^{cde}+{1\over 96}\,\eta_{ab}
H_{cdef}H^{cdef}\cr
\noalign{\medskip}
D^a H_{abcd}\hfill &=-{1\over 1728}\Eps_{bcde_1\ldots
e_4f_1\ldots f_4}
HJ^{e_1\ldots e_4}H^{f_1\ldots f_4}
}
\leqno{(10.77)}
$$
Retaining only the $\theta=0$ part of these equations, we
readily observe that they are the bosonic field
equations that would also have been obtained from the
component formulation.

We conclude by remarking that the $D=10$, $N=2$ Type
\IIA supergravity may be obtained again by dimensional
reduction.
The resulting formulation is rather involved.
There also exist a superspace formulation for Type \IIB
supergravity.

\medskip\noindent
{\it $N=1$, $D=10$ supergravity in superspace}

The superspace data are as follows: local
supercoordinates $X^M=(x^m,\theta^\mu)$, with
$m=0,1,\ldots,9$ and $\mu=1,\ldots 16$, parametrize a
real supermanifold of dimension $(10\vert 16)$.
The fundamental fields are

$$
\matrix{
&\ssbullet &\quad &E^A=dX^M E_M{}^A &\qquad\qquad &
  &\qquad\qquad &\cr
&\ssbullet &\quad &\Omega_A{}^B=dX^M\Omega_{MA}{}^B
  &\qquad\qquad & &\qquad\qquad &
}
\leqno{(10.78)}
$$

\medskip\noindent
where $A$,$B$ run over frame indices: $(a,\alpha)$,
$(b,\beta)$, $a,b=0,1,\ldots,9$ and
$\alpha,\beta=1,\ldots,16$.
The spinors here transform under a single Majorana-Weyl
representation of $\Spin(1,9)$.
Notice that there is {\bf no} further anti-symmetric
tensor as there was in $D=11$.
The $\Bmu$ field that occurs in $D=10$, $N=1$
supergravity is contained in the fields $E^A$ and
$\Omega_A{}^B$.
The field $\Omega_A{}^B$ is again a spin connection,
but now for $\Spin(1,9)$, acting on the direct sum of
the defining representation and the Majorana-Weyl
representation:
$$
\matrix{ \displaystyle
\Omega_{ab}=-\Omega_{ba}\hfill &\Omega_{\alpha b}=
  \Omega_{a\beta}=0\cr
\noalign{\medskip}
\Omega_{\alpha\beta}={1\over 4}(\Gamma^{ab})_{\alpha\beta}
\Omega_{ab}\hfill &}
\leqno{(10.79)}
$$

Torsion and curvature are defined by
$$
\matrix{ 
\displaystyle
T^A\equiv dE^A+E^B\w\Omega_B{}^A\hfill
&\left(\equiv {1\over 2}\,E^C\w E^B T_{BC}{}^A\right)\hfill\cr
\noalign{\medskip}
R_A{}^B\equiv d\Omega_A{}^B+\Omega_A{}^C\w
  \Omega_C{}^B\hfill &\left(\equiv {1\over 2}\,
  E^D\w E^C R_{CDA}{}^B\right)\,\,.
}
\leqno{(10.80)}
$$
As in the case of $D=11$, $N=1$, these quantities
satisfy Bianchi identities
$$
DT^A=E^B\w R_B{}^A\qquad\qquad
DR_A{}^B=0\,\,.
\leqno{(10.81)}
$$
(A component decomposition for these equations is given
in (10.74).)

Again, this geometry is too general, and torsion
constraints should be imposed.
The choice of constraints is not unique, and depends
upon the choice of connection, among others.
The constraints here were first derived by 
Nilsson, and
reformulated by Witten:
$$
\matrix{ 
\displaystyle
T_{\alpha\beta}{}^\gamma=T_{\alpha b}{}^c=0\hfill
  &\hfill\qquad
T_{\alpha\beta}{}^c=2\Gamma_{\alpha\beta}^c\cr
\noalign{\medskip}
T_{\alpha\beta}{}^\gamma=(\Gamma_a)_{\beta\delta}
  \phi^{\delta\gamma}\hfill &\hbox{for some }
\phi^{\delta\gamma}\hfill
}
\leqno{(10.82)}
$$
One may show that the resulting fields, after
elimination of all auxiliary fields, are precisely
those of the $N=1$ supergravity multiplet in $D=10$,
with the same equations of motion as those that we
obtained from the component formulation.

\bigskip\noindent
{\bf \S 10.10.} {\bf Local supersymmetric coupling of
superstrings to supergravity} 

It is now very easy to produce a $\GS$ worldsheet
action, formulated in superspace, that couples in a
manifestly locally supersymmetric way to the superspace
fields of supergravity in $D=10$.
For simplicity, we consider the case with $N=1$
supersymmetry only.

Recall that the flat superspace action was
$$
S[X]={1\over 8\pi}\int_\Sigma d\mu_g g^{pq}E_p{}^a
E_a{}^b
\eta_{ab}-{i\over 4\pi}\int_B
X^*(H)
$$
where $E_p{}^a d\xi^p\equiv X^*(E^a)$ is the pullback
to $X$ of $E^a$, and where $\partial B=\Sigma$, and $H$
was a closed $3$-form on $M$.
The form of $H$ was dictated by the condition that we
had to have $\kap$-symmetry in order to obtain the correct
spectrum.

In general curved superspace backgrounds, we may again
propose $S[X]$ as a suitable action, provided the
superspace geometry admits a closed $3$-form $H$, and
upon inserting a factor of a scalar superfield $k$:
$$
S[X]={1\over 8\pi}\int_\Sigma d\mu_g
g^{pq}E_p{}^a E_q{}^b\eta_{ab}e^{4k}-{i\over 4\pi}
\int_B X^*(H)
$$
where $E_p{}^a d\xi^p=X^*(E^a)$, just as in the flat
superspace case.
The superfield $k$ cannot be arbitrary, of course.

It can be shown that when the torsion constraints for
$N=1$, $D=10$ supergravity are satisfied, then there
exists a natural closed $3$-form $H$, such that $S[X]$
is invariant under $\kap$-symmetry.
In other words, if we assume (locally) that $H=dB$,
then the requirement of $\kap$-invariance demands that
certain torsion constraints are satisfied.
Up to equivalences, these constraints are precisely the
torsion constraints.


\bye
