%From: Pierre Deligne <deligne@IAS.EDU>
%Date: Mon, 13 Jan 1997 10:10:14 -0500
%Subject: Quantization

\input amstex
\documentstyle{amsppt}
\magnification=1200
\input pictex
\loadeufm

\font\dotless=cmr10 %for the roman i or j to be
                    %used with accents on top.
                    %(\dotless\char'020=i)
                    %(\dotless\char'021=j)
\font\itdotless=cmti10
\def\itumi{{\"{\itdotless\char'020}}}
\def\itumj{{\"{\itdotless\char'021}}}
\def\umi{{\"{\dotless\char'020}}}
\def\umj{{\"{\dotless\char'021}}}
\font\thinlinefont=cmr5
\font\smaller=cmr5
\font\boldtitlefont=cmb10 scaled\magstep2

\NoRunningHeads
\pagewidth{6.5 true in}
\pageheight{8.9 true in}
\loadeusm

\catcode`\@=11
\def\logo@{}
\catcode`\@=13

\def\eps{{\varepsilon}}

\def\undertext#1{$\underline{\vphantom{y}\hbox{#1}}$}
\def\nspace{\lineskip=1pt\baselineskip=12pt%
     \lineskiplimit=0pt}
\def\dspace{\lineskip=2pt\baselineskip=18pt%
     \lineskiplimit=0pt}

%\def\wedgeop{\operatornamewithlimits{\wedge}\limits}
%\def\oplusop{\operatornamewithlimits{\oplus}\limits}
%\def\simover#1{\overset\sim\to#1}
\def\w{{\mathchoice{\,{\scriptstyle\wedge}\,}
  {{\scriptstyle\wedge}}
  {{\scriptscriptstyle\wedge}}{{\scriptscriptstyle\wedge}}}}
\def\Le{{\mathchoice{\,{\scriptstyle\le}\,}
{\,{\scriptstyle\le}\,}
{\,{\scriptscriptstyle\le}\,}{\,{\scriptscriptstyle\le}\,}}}
\def\Ge{{\mathchoice{\,{\scriptstyle\ge}\,}
{\,{\scriptstyle\ge}\,}
{\,{\scriptscriptstyle\ge}\,}{\,{\scriptscriptstyle\ge}\,}}}
%\def\uphat{{\sssize\wedge}}
\def\uphat{{\hat{\phantom{a}}}}
%\def\upvee{{\check{\phantom{a}}}} 
\def\upvee{{\sssize \vee}}
\def\plus{{\sssize +}}
%\def\doubleheaddownarrow#1{\hbox{$\Big\downarrow%
%     \kern-5.96pt\lower2pt\hbox{$\downarrow$}$}%
%     \rlap{$\vcenter{$\kern-18pt\scriptstyle#1$}$}}
%\def\rmapdown#1{\Big\downarrow\kern-1.0pt\vcenter{
%     \hbox{$\scriptstyle#1$}}}

%\def\Gal{\text{\rm Gal}}  \def\sign{\text{\rm sign}}
%\def\Br{\text{\rm Br}}  \def\im{\text{\rm Im}}
%\def\sBr{\text{\rm sBr}} 
\def\Spin{\text{\rm Spin}}
%\def\sgn{\text{\rm sgn}} \def\pr{\text{\rm pr}}
\def\Id{\text{\rm Id}} 
\def\Sp{\text{\rm Sp}}
%\def\Gr{\text{Gr}} 
\def\SO{\text{\rm SO}}
%\def\Sym{\text{\rm Sym}} \def\SL{\text{\rm SL}}
%\def\End{\text{\rm End}}
%\def\Hom{\text{\rm Hom}} 
%\def\Ker{\text{\rm Ker}}
\def\Tr{\text{\rm Tr}}
\def\Supp{\text{\rm Supp}}
\def\cl{\text{c}\ell}
%\def\Lie{\text{\rm Lie}} 

\def\fbar{\bar{f}}
\def\kbar{\bar{k}}
\def\zbar{\bar{z}}
\def\Abar{\bar{A}}
\def\Sbar{\bar{S}}

\def\dbC{{\Bbb C}}
%\def\dbG{{\Bbb G}}
%\def\dbH{{\Bbb H}}
\def\dbR{{\Bbb R}}
\def\dbZ{{\Bbb Z}}


\def\scr#1{{\fam\eusmfam\relax#1}}

%\def\scrD{{\scr D}}
\def\scrH{{\scr H}}
\def\scrL{{\scr L}}
%\def\scrS{{\scr S}}   
%\def\scrU{{\scr U}}   

%\def\gr#1{{\fam\eufmfam\relax#1}}

%Euler Fraktur letters (German)
%\def\gro{{\gr o}}
%\def\grs{{\gr s}}

%\def\grso{{\grs\gro}}


\topmatter
\title\nofrills
{\boldtitlefont Quantization}
\endtitle
\author
P. Deligne
\endauthor
\endtopmatter

%\bigskip
\centerline{School of Mathematics}
\centerline{Institute for Advanced Study}
\centerline{Princeton, NJ \ 08540}

\smallskip
\centerline{e-mail: deligne\@math.ias.edu}


\NoBlackBoxes
\parindent=20pt
\frenchspacing
\document
\dspace
\bigskip
The present notes do not pretend to any originality.
We have tried to present different aspects of what
quantization can mean, including the case of odd
variables.
Complex polarizations are not considered.

\bigskip\bigskip
\subhead
1
\endsubhead
Let $(M,\omega)$ be a symplectic manifold of dimension
$2d$.
We will use vague words ``big'' and ``small''.
For this, we need some notion of size $\sim 1$ on $M$.
For instance $M$ could carry a Riemannian metric, with
curvature of size at most $\sim 1$ and injectivity
radius of size at least $\sim 1$.
The symplectic form should be of size $\sim 1$, and $\hslash$
is a small number.
The $2$-form which really matters is $\omega/\hslash$.

Our sign convention for the Poisson bracket
corresponding to $\omega$ is the following:
the Hamiltonian vector field $X(f)$ defined by a
function $f$ is given by $X(f)g=\{f,g\}$, and
$df=-i_{X(f)}\omega$.
If $M=\dbR^2$ with coordinates $p$, $q$ and if $\omega=dp\w dq$,
then $\{p,q\}=1$.

A quantization of $M$ consists in a complex Hilbert
space $\scrH$ and in a rule to attach to functions $f$
on $M$ operators $f^{\w}$ acting on $\scrH$.
The rule $f\mapsto f^{\w}$ should be $\dbC$-linear, for
$f$ real, $f^{\w}$ should be hermitian and, in a sense I
will not try to make precise, $f\mapsto f^{\w}$ should
almost be an homomorphism: $1^{\w}$ should be the
identity and for slowly varying functions $f$ and $g$,
$$
(fg)^{\w}\,\sim f^{\w}g^{\w}\qquad
\bmod{O(\hslash)}.
\tag1.1
$$
If $(1.1)$ did hold exactly, the $f^{\w}$ would commute
the one with the other.
The symplectic structure controls the failure of
commutativity
$$
[\,f^{\w},g^{\w}]\sim
-i\hslash\{f,g\}^{\w}\qquad \bmod{O(\hslash^2)}.
\tag1.2
$$

Formula (1.2) can be rewritten
$$
(X(f)g)^{\w}\sim i/\hslash[\,f^{\w},g^{\w}]\qquad
\bmod{O(\hslash)}.
\leqno{\hbox{$(1.2)'$}}
$$
The symplectic vector field $X(f)$, if it is well
behaved at infinity, exponentiates to an automorphism
$\exp(X(f))$ of $(M,\omega)$, with Taylor's formula
reading
$$
\exp(X(f))^*g=\exp(\text{derivation $X(f)$})(g).
$$

Formula $(1.2)'$ should hold in its integrated form
$$
(\exp(X(f))^*g)^{\w}\sim\exp(i/\hslash f^{\w})g^{\w}
\exp(-i/\hslash f^{\w})\qquad
\bmod{O(\hslash)}.
\tag1.3
$$

We don't care about the exact rule $f\mapsto f^{\w}$.
It matters mainly $\bmod{\,O(\hslash)}$.

In the deformation quantization point of view, $(1.1)$
is the first term in an expansion ($*$-{\it product})
$$
f^{\w}g^{\w}=(fg)^{\w}+c_1(f,g)^{\w}\hslash+c_2(f,g)^{\w}
\hslash^2+\cdots\,\,,
\leqno{\hbox{$(1.1)'$}}
$$
with $c_i(f,g)$ given by a $\dbC$-bilinear differential
operator in $f$ and $g$.
The formula (1.2) becomes $c_1(f,g)-c_1(g,f)=-i\{f,g\}$.
If one disposes of quantizations $\scrH(\hslash)$,with
$\hslash$ tending to $0$, $(1,1)'$ can be an asymptotic
expansion in $\hslash$.
if one redefines $f\mapsto f^{\w}$ to be
$f^{\w}+d_1(f)^{\w}\hslash+\cdots$ with $d_i(f)$ a
linear differential operator in $f$, $(1.1)'$ changes to
an {\it equivalent} $*$-product.

Assumption $(1.1)$ means that $\scrH$ almost localizes
on $M$: an element $h$ of $\scrH$ can be decomposed as a
sum $\sum h_i$ of elements localized in small regions
$U_i$ of $M$: if $x\in U_i$, $f^{\w}h_i$ is about
$f(x).h_i$.
Because of the non commutativity $(1.2)$, localization
does not make sense for regions smaller than expressed
by the uncertainty principle $\Delta p \Delta q\sim
2\pi\hslash$.

If we were only to assume $(1.1)$ and $(1.2)$, a multiple
$\scrH^n$ of a quantization $\scrH$ of $M$ would again
be a quantization.
A ``finite multiplicity'' assumption is that if $f$ is a
$C^\infty$ function with compact support on $M$, the
operator $f^{\w}$ is of trace class.
If this holds, I would expect that for some integer $m$
(rather: one for each connected component of $M$), one
has
$$
\Tr(f^{\w})\sim m\int f(x)dx
$$
where, in Darboux local coordinates, $dx$ is the product
of the $dp_i\,dq_i/2\pi\hslash$ (Liouville measure).
We want $m=1$:
$$
\Tr(f^{\w})\sim\int f(x)dx\
\tag1.4
$$
\medskip
\subhead
2. Example: cotangent bundles
\endsubhead

The basic example of almost localization, in the sense
meant above, is when $M$ is a cotangent bundle $T^*V$
and $\scrH$ the space of half densities on $V$.
If $f$ is peaked at $x$, say in a gaussian fashion, and
if $g$ is a real function, with $dg$ slowly varying,
then $fe^{ig/\hslash}$ is localized around $(x,dg_x)$ in
$M$.
If $f$ is slowly varying, then $f.e^{ig/\hslash}$ is
localized around the section $x\mapsto dg_x$ of $M\to V$,
above the support of $f$.

The operator $f{\uphat}$ has a natural definition when
$f$ is affine linear on the fibers of $T^*V$ over $V$.
If $f$ is the pull back of a function on $V$, $f{\uphat}$
is multiplication by $f$.
If $f$ is linear on each cotangent space, hence
identified with a vector field $F$ on $V$,
$f{\uphat}=-i\hslash\scrL_F$.
For those $f$, (1.2) holds exactly.
To define $f{\uphat}$ for more complicated functions
requires auxiliary choices. 

On $T^*V$, we have a canonical $1$-form $\alpha$: \ in
local coordinates $q^i$ on $V$, giving local coordinates
$(p_i,q^i)$ on $T^*V$, $\alpha$ is $\sum p_idq^i$.
The trivial unitary line bundle $\scrL$, with the
connection $-i\alpha/\hslash\colon\,\nabla
f=df-i\alpha/\hslash f$, is a prequantization line
bundle: \ its curvature is $-i\omega/\hslash$, for
$\omega$ the symplectic form $d\alpha$.
A real function $g$ on $V$ defines the lagrangian section
$L(g)\colon x\mapsto (dg)_x$ of $T^*V$.
The pull back to $L(g)$ of $e^{ig/\hslash}$ is an
horizontal section of $\scrL$.

For any symplectic manifold $M$, let $\Lambda_M$ or
simply $\Lambda$ be the fiber space over $M$ whose fiber
$\Lambda_m$ at $m\in M$ is the space of lagrangian ($=$
maximal isotropic) subspace of the tangent space $T_m$ of
$M$ at $m$.
To take into account the Maslov index story, it is best
to take prequantization line bundles as living on
$\Lambda$.
They should verify (a) (b) below.

\medskip\noindent
(a)\enspace
The curvature $2$-form $R$ is the pull back from $M$ of
$-i\omega/\hslash$:
$$
\nabla^2=-i\omega/\hslash.
$$
In particular, on each fiber $\Lambda_m$ of $\Lambda\to
M$, $(\scrL,\nabla)$ is flat.
One has $\pi_1(\Lambda_m)=\dbZ$ and 

\smallskip\noindent
(b)\enspace
the monodromy of $(\scrL\nabla)$ on $\Lambda_m$ is
multiplication by $i$.

\medskip\noindent
{\it Remarks:} \ (i)\enspace
In the same way that an hamiltonian vector field $X(f)$
acts on a prequantization line bundle, with the action
depending on $f$, and not only on $X(f)$, it also acts on
$(\scrL,\nabla)$ as above: \ $X(f)$ is symplectic, hence
lifts to $\Lambda$, the question is local on $\Lambda$,
and locally on $\Lambda$ $(\scrL,\nabla)$ is the inverse
image of a prequantization line bundle on $M$.
If we exponentiate, we see that $\exp(X(f))$ acts by an
automorphism of $(M,\omega,\scrL,\nabla)$.
The infinitesimal action on sections of $\scrL$ is given
by
$$
\partial_{X(f)}=\nabla_{X(f)}-if/\hslash\tag2.1
$$

\medskip\noindent
(ii)\enspace
To make sense of (b), we should make precise our sign
convention identifying $\pi_1(\Lambda_m)$ with $\dbZ$.
For the symplectic vector space $\dbR^2$, form $dp\w dq$,
the generator is given by the path
$$
\theta\longmapsto \text{ line spanned by }
(q,p)=(\cos\theta,\sin\theta)\qquad (0\Le\theta\Le\pi).
$$

\medskip\noindent
(iii)\enspace
Let $V$ be a symplectic vector space, and $\Lambda_0$ be
the space of lagrangian linear subspaces of $V$.
If $S_1$ and $S_2$ in $\Lambda_0$ intersect
transversally, one defines as follows a preferred
homotopy class of path from $S_1$ to $S_2$.
Identify $S_2$ to $S_1^{\upvee}$ by $s_2\mapsto
\omega(s_2,s_1)$.
Choose a basis $e_i$ of $S_1$ and let $e'_1$ be the dual
basis of $S_2$.
The path is 
$$
\theta\longmapsto\text{ subspace spanned by the }
\cos\theta e_i+\sin\theta e'_i\qquad (0\Le\theta\Le\pi/2).
$$
It depends only on the quadratic form for which the basis
$(e_i)$ is orthonormal.
The choice of this quadratic form running over a
contractible set, the homotopy class of the path does not
depend on the choice.
If $V$ is of dimension $2d$, the preferred path from
$S_1$ to $S_2$, followed by the preferred path from $S_2$
to $S_1$, is $d$ times the preferred generator of
$\pi_1(\Lambda_0)$.

\medskip
We now come back to the case where $M$ is a cotangent
bundle $T^*V$.
Let $\Lambda^0\subset\Lambda$ be the subbundle, with
$\Lambda_m^0\subset\Lambda_m$ consisting of those
lagrangian subspaces $L$ of $T_m$ intersecting
transversally the vertical subspace of $T_m^*$ (tangent
to the fiber of $T^*V\to V$).
If we pull back the prequantization line bundle on $T^*V$
to $\Lambda^0$, it extends uniquely to $\scrL_\Lambda$ on
$\Lambda$ satisfying the conditions (a) (b).

Let $L$ be a lagrangian subvariety of $M$.
Let $s$ be the section of $\Lambda$ over $L$: \ 
$x\mapsto$ tangent space of $L$ at $x$.
The section $s$ is with values in $\Lambda^0$ if and only
if $L$ is locally of the form $L(g)$.
As $L$ is lagrangian, $s^*\scrL_\Lambda$ is flat and it
makes sense ot speak of a slowly varying section
of $s^*\scrL$ on $L$.
Generalizing the $f.e^{ig/\hslash}$ considered
previously, to a slowly varying half density $u$ on $L$,
with values in $s^*\scrL$, corresponds $[u]$
in $\scrH$, with
$$
\left\Vert [u]\right\Vert^2 \sim\int\nolimits_{S}
\left<u,u\right>.\tag2.2
$$
It is localized near $\Supp(u)\subset S\subset M=T^*V$.

A function $f$ defines an automorphism $\exp(X(f))$of
$(M,\scrL,\nabla)$ (Remark (ii) above), and an
automorphism $\exp(-i/\hslash f{\uphat})$ of $\scrH$.
Those automorphisms should (almost) preserve the
construction $u\mapsto [u]$: 
$$
[\exp(X(f))^*u]\sim\exp(i/\hslash f{\uphat})[u].\tag2.3
$$

Inner products are given as follows.
Fix $(S_i,u_i)$ ($i=1,2$) as above, and assume that $S_1$
and $S_2$ intersect transversally.
Let $x$ be an intersection point.
The lagrangian subspaces $s_1(x)$ and $s_2(x)$ of $T_xM$
intersect transversally, and the preferred class of path
$\gamma$ from $s_1(x)$ to $s_2(x)$ (Remark (iii) above)
gives an isomorphism
$\gamma^*\colon\,\scrL_{s_2(x)}\to\scrL_{s_1(x)}$.
The symplectic structure puts $s_1(x)$ and $s_2(x)$ in
duality, identifying the line of half densities of $S_1$
at $x$ with the dual of the similar line for $S_2$.
The inner product $\left<u_1(x),\gamma^* u_2(x)\right>$
is hence just a number.
The stationary phase gives
$$
\left<[u_1],[u_2]\right>\sim
\sum\limits_{x\in S_1\cap S_2}(2\pi i\hslash)^{d/2}
\left<u_1(x),\gamma^* u_2(x)\right>
\tag2.4
$$
for $M$ of dimension $2d$.

Our convention for inner products $\left<a,b\right>$ is: \ 
antilinear in $a$, linear in $b$.

\medskip
\subhead
3
\endsubhead
The flat case is when $M$ is an affine space, with a
translation invariant symplectic form.
In the flat case, one can define a quantization as the
data of $f\mapsto f{\uphat}$, just for $f$ a linear
function, with (1.2) holding exactly.
The purely imaginary linear functions form a Lie algebra,
for $-i\hslash\{f,g\}$, and we want a unitary
representation $f\mapsto f{\uphat}$ of this Lie aglebra,
with $i{\uphat}=i.\Id$.
It is more convenient to ask for a unitary representation
of the corresponding Lie group.
Formula (1.4) becomes a request for irreducibility.
Weyl's quantization $f\mapsto f{\uphat}$, for a general
$f$, is given by 
$$
\exp(\ell){\uphat}=\exp(\ell{\uphat})\tag3.1
$$
for $\ell$ a purely imaginary linear function, a
general$f{\uphat}$ being deduced from (3.1) by
considering $f$ as a superposition of $exp(\ell)$, by
Fourier transform.
For Weyl's quantization, formula (1.4) holds exactly.

Understood in those terms, quantization in the flat case
is unique, up to isomorphisms.
If $M$ is the cotangent bundle to an affine space $V$, 2.
gives its Schrodinger model as $L^2(V)$: \ if $V_0$ is
the vector space of translations of $V$, one has
$M=V_0^{\upvee}\times V$; for $f$ (the pull back of) a
function on $V$, $f{\uphat}$ is multiplication by $f$.
For $\ell$ a linear form on $V_0^{\upvee}$, identified
with a vector $v\in V_0$, $\ell{\uphat}$ is $-i\hslash
\partial_v$, and $\exp(i\ell){\uphat}$ is
$\psi(x)\mapsto\psi(x+\hslash v)$.

This unicity assertion depends on the assumption that the
flat symplectic variety $M$ is of finite dimension.

\remark{Remark}
In the flat case, the data on $\Lambda_M$ of a
prequantization line bundle defines a quantization
$\scrH$, as defined above, up to unique isomorphism.
This is made plausible by the fact that the group of
automorphisms of$$
\align
&\text{(affine symplectic space
$M,\scrL_\Lambda,\nabla)$}\\
\intertext{and of}
&\text{(affine symplectic space $M,\scrH,f\mapsto
f{\uphat})$}
\endalign
$$
are the same extension of the affine symplectic group
$(\Sp\ltimes\text{translations})$ by $U^1$.

The relation between $\scrL_\Lambda$ and the quantization
$\scrH$ can be fixed as follows.
Let $\scrH_\infty$ be the $C^\infty$-vectors in $\scrH$, 
for the action of the Heisenberg group.
In the Schroedinger model: \ the Schwartz functions on
$V$.
Let $\scrH_{-\infty}\supset\scrH$ be the dual of
$\scrH_\infty$.
Then, a linear lagrangian subspace $L$ of $M$, given with
$u$ on $L$ of the form (translation invariant half
density). (flat section $s^*\scrL_\Lambda$), define
$[u]$ in $\scrH_{-\infty}$, and the inner product 
formula (2.4) holds exactly, to give the inner product 
of continuous superpositions of $[u]$.
Formula (2.3) holds exactly, for $f$ quadratic and $u$ as
above.
In particular, if a linear form $\ell$ vanishes on $L$,
one has $\ell{\uphat}[u]=0$.
\endremark

\medskip
\subhead
4
\endsubhead
We now go over to the superworld.
A super Hilbert space is a $\mod 2$ graded complex vector
space $H$, with an even sesquilinear form $(u,v)$,
antilinear in $u$ and linear in $v$, for which
$$
(v,u)=(-1)^{p(u)p(v)}(u,v)^-,
$$
with a positivity and a completeness condition.
As positivity condition, we take
$$
\alignedat4
&(u,u) >0 &\quad &\text{for} &\quad &u\text{ even},
  &\quad &u\not=0\\
-i&(u,u)>0 &&\text{for} &&u\text{ odd}, && u\not=0.
\endalignedat
\tag4.1
$$
This positivity condition is stable by tensor product, if
$(\,\,,\,\,)$ for $H'\otimes H''$ is defined by
$$
(u'\otimes u'',v'\otimes v'')=(-1)^{p(u'')p(v')}
(u',v')(u'',v''),
$$
in accordance with the sign rule.
The adjoint $T^{\dagger}$ of an operator $T$ is defined
by
$$
(Tu,v)=(-1)^{p(T)p(u)}(u,T^{\dagger}v).
\tag4.2
$$

The physicist don't like (4.1), and prefer to work with
the ordinary $\mod 2$ graded Hilbert space with the inner
product $\left<\,\,,\,\,\right>$ defined by
$$
\alignedat3
(u,v)= &\left<u,v\right> &\quad &\text{for} 
  &\quad &u,v\text{ even}\\
(u,v)=i &\left<u,v\right> &&\text{for} &&u,v\text{ odd}.
\endaligned
\tag4.3
$$
The adjoint $T^{\dagger}$ is related to the corresponding
ordinary adjoint $T^*$ by
$$
\alignedat2
T^{\dagger} &=T^* &\quad &\text{for $T$ even}\\
T^{\dagger} &=iT^* &&\text{for $T$ odd}.
\endaligned
\tag4.4
$$

The definition (4.2) has the unsettling effect that the
eigenvalues of an odd self adjoint operator are in
$i^{1/2}.\dbR$.
However, the eigen spaces being neither even nor odd,
there is not much wish to consider them.

In a quantization of a supermanifold $M$, to functions
$f$ on $M$ should correspond operators $f{\uphat}$,
acting on a super Hilbert space $\scrH$, with
$f{\uphat}$ of the same parity as $f$.
If $f$ is real, one should have
$f{\uphat}=f{\uphat}\!^\dagger$.
Because of (3.4), if $M$ is a supermanifold, the
physicists will declare to be ``real'' the real even $f$,
and the $i^{-1/2}\eta$ for $\eta$ real.
With this rule, if $\alpha_1$ and $\alpha_2$ are odd and
``real'', $i\alpha_1\alpha_2$ is ``real'' (as well as
real, being even).
Example: \ on $\dbR^{1,1}$, with coordinates $(x,\eta)$,
if we put $\theta=i^{-1/2}\eta$, the vector field
$\partial_\theta+i\theta\partial_x$ is a multiple of a
real vector field:
$$
\partial_\theta+i\theta\partial_x=i^{1/2}(\partial_\eta+\eta
\partial_x).
$$

\medskip
\subhead
5
\endsubhead
Let now $M$ be a super affine space, with a translation
invariant symplectic form.
In imitation of 3., we define a quantization as the data
of a super Hilbert space $\scrH$, and of
$\ell\mapsto\ell{\uphat}$ attaching to a linear
function on $M$ an operator on $\scrH$, with $\ell$ and
$\ell{\uphat}$ of the same parity.
The quantization $\ell\mapsto\ell{\uphat}$ should be
linear, real: \ $\ell{\uphat}=(\ell{\uphat})^{\dagger}$
for $\ell$ real, map $1$ to $\Id$, and obey
$$
\left[\ell{\uphat},m{\uphat}\right]=-i\hslash\{f,g\}{\uphat}.
\tag5.1
$$
One should also require some irreducibility.

The condition (5.1) requires that for real linear odd
functions $\ell$, the Poisson bracket $\{\ell,\ell\}$ (a
constant) be negative:
$$
\left\{\ell,\ell\right\}\Le 0.\tag5.2
$$
Indeed, $[\ell{\uphat},\ell{\uphat}]=2\ell{\uphat 2}=
\ell{\uphat}\!^\dagger\ell{\uphat}$ must be
$-i\hslash\{\ell,\ell\}$.
The sign in (5.2) is due to the choices of sign in (4.1)
and (5.1), the latter repeating (1.2).

As in 3., one extends $\ell\mapsto\ell{\uphat}$ to
polynomial functions by requiring that for a product of
even or odd linear functions, one has
$$
(\ell_1\ldots\ell_n){\uphat}=1/n!\sum\limits_{\sigma}
\pm\,\ell{\uphat}_{\!\!\!\!\sigma(1)}
\ldots\ell{\uphat}_{\!\!\!\!\sigma(n)}
\qquad(\sigma\in S_n)
$$
with $\pm$ given by the sign rule.
In particular, $\pm$ is $+$ if the $\ell_i$ are all even,
and $\eps(\sigma)$ if they are all odd.

\medskip
\subhead
6
\endsubhead
Let $W$ be a real vector space with a positive defined
symmetric bilinear form $B$.
Let $M$ be the affine space defined by $\Pi W$: \ it is
of dimension $(0,\dim\,W)$, and linear forms on $W$ are
odd functions on $M$.
Define a Poisson bracket by 
$$
\{\ell,m\}=-B^{-1}(\ell,m)\tag6.1
$$
for $\ell,m\in W^{\upvee}$.

If $e_i$ is an orthonormal basis of $W$, with dual basis
$e^i$, and if $e^i$ is veiwed as an odd function on $M$,
the Poisson bracket (6.1) is given by the super
symplectic form
$$
\omega=\tfrac12\sum de^ide^i.
\tag6.2
$$

One $W^*$, let $Q$ be the quadratic form $\frac12
B^{-1}(\ell,\ell)$.
If $\scrH$ is a $\mod 2$ graded module over the ($\text{mod } 2$
graded) Clifford algebra $C(W^*,Q)$, with the module
structure written $\cl(\ell)$ for $\ell\in W^*$, (5.1)
holds for
$$
\ell{\uphat}=(i\hslash)^{1/2}\cl(\ell).\tag6.3
$$

For $\dim\,W=1$, one can take $\scrH=(\dbC\text{ even})\oplus
(\dbC\text{ odd})$, with $\cl(e^1)=\left(\smallmatrix
0 &1/\sqrt{2}\\ 1/\sqrt{2} &0\endsmallmatrix\right)$.
For the obvious Hilbert space structure of $\scrH$, and
the corresponding super Hilbert space structure, one has
then $\cl(e^1)^*=\cl(e^1)$ and for any $\ell$ in $V^*$,
$$
\ell{\uphat}=(\ell{\uphat})^\dagger.\tag6.4
$$

A general $W$ can be written as an orthogonal direct sum
of lines, and the corresponding tensor product of super
Hilbert spaces is a Clifford module for which (6.4)
holds, if $\ell{\uphat}$ is defined by (6.3).
If $\scrH_1$ is a graded submodule with the induced
Hilbert space structure, (6.4) continues to hold for
$\scrH_1$.
This proves the existence of a super Hilbert space
$\scrH$ with a Clifford module structure, irreducible as
a graded module, for which (6.4) holds when
$\ell{\uphat}$ is defined by (6.3).

We now assume $W$ of even dimension.
They are the two isomorphy classes of $\scrH$ as above,
exchanged by the parity change $\scrH\mapsto\scrH\otimes L$, 
for $L$ a $(0,1)$-dimensional super Hilbert space.
Let $(e^1,\dotsc,e^n)$ be an orthonormal basis (for
$B^{-1}$), and $[e^1,\dotsc,e^n]$ be the corresponding
density on $M$.
Another orthonormal basis would give the same, or the
opposite density.
The choice of the density $[e^1,\dotsc,e^n]$ picks out
one of the two isomorphy classes of $\scrH$: \ the one
for which the following analogue of (1.4) holds.
For any $f$ on $M$, the supertrace $\Tr(f{\uphat})$ is
given by 
$$
\Tr(f{\uphat})=\hslash^{n/2}\int [e^1,\dotsc,e_n]f
\tag6.5
$$
Justification: \ writing $\scrH$ as a tensor product, one
reduces to the case $n=2$.
For $n=2$, $\cl(e^1)\cl(e^2)$ has square $-1/4$ and
eigenvalues $\pm i/2$.
It follows that
$(e_1e_2){\uphat}=\frac12(e{\uphat}\!\!\!_1e{\uphat}\!\!\!_2-
e{\uphat}\!\!\!_2e{\uphat}\!\!\!_1)=e{\uphat}\!\!\!_1
e{\uphat}\!\!\!_2=i\hslash\cl(e_1)\cl(e_2)$ has
eigenvalues $\mp \hslash/2$, hence supertrace
$\pm\hslash$.

\medskip
\subhead
7
\endsubhead
A super symplectic flat space $M$ as in 5. can be
decomposed as $M^{\plus}\times M^-$, with $M^{\plus}$
even and $M^-$ odd, as in 6.
The tensor product of quantizations of $M^{\plus}$ and
$M^-$ gives one for $M$.
If $M^{\plus}$ is the cotangent bundle of an affine space
$V$, and if $M^-$ is obtained as in 6. from an even
dimensional quadratic vector space $W$, this tensor
product can be realized as the space of $L^2$-density on
$V$ with values in a Clifford module for $W^{\upvee}$, and
an analogue of 1.4, 6.5 holds.

\medskip
\subhead
8
\endsubhead
Let $V$ be a manifold and $W$ be an orthogonal vector
bundle on $V$, with a connection $\nabla$ respecting the
structural symmetric bilinear form $B$.
We take $M$ to be the fiber product over $V$ of $T^*V$
and $\Pi W$.
Over any basis $B$, a section of $M$ over $V$ is the data
of an even section $p$ of $T^*V$ and of an odd section
$\psi$ of $W$.
A section $e$ of $W^{\upvee}$ defines an odd function
$\psi\mapsto e(\psi)$ on $M$.
A local coordinate system $(q^i)$ on $V$ and a local basis
$(e_\alpha)$ of $W$ give a local coordinate system
$(q^i,p_i,\psi^\alpha)$ on $M$.

We assume that $W$ is even dimensional, oriented and
$\Spin$.
It gives rise to a bundle of super Hilbert spaces
$\scrH_V$ on $V$, reproducing point by point what we got
in 6., and the connection $\nabla$ on $W$ gives one on
$\scrH_V$.

Let $\scrH$ be the super Hilbert space of half densities
on $V$ with values in $\scrH_V$.
To any function $f$ on $M$, of degree $\Le 1$ on the
fibers of $M/V$, when $p_i$ is viewed as of degree $1$
and $\psi^\alpha$ as of degree $1/2$,
one attaches an operator $f{\uphat}$ as follows.

\medskip\noindent
(a)\enspace
For (the pull back of) a function $f$ on $V$, $f{\uphat}$
is multiplication by $f$.

\smallskip\noindent
(b)\enspace
For the odd function defined by a section $e$ of
$W^{\upvee}$,
$$
e{\uphat}=(i\hslash)^{1/2}\cl(e)
$$
as in 6.3.
This is extended as in 5. to define $f{\uphat}$ for
any function $f$ on $\Pi W$.

\smallskip\noindent
(c)\enspace
If $f$ is a function on $T^*V$, linear on each fiber of
$T^*V/V$ and identified with a vector field $F$, 
$$
f{\uphat}=-i\hslash\nabla_F.
$$

\medskip
In (c), if $x$ in $\scrH$ is the product of a section $h$
of $\scrH_V$ by a half density $v$,
$\nabla_F(x):=\nabla_F(h).v+h.\scrL_F(v)$.

The super vector space of operators $f{\uphat}$, for $f$
of degree $\Le 1$ on the fibers of $M/V$, 
is stable by bracket.
By (1.2), the bracket of operators corresponds to a
Poisson bracket, which we now compute.

For functions on $\Pi W$, it is as in 6.: \ functions on
$V$ commute with functions on $\Pi W$ and for $\ell$, $m$
odd linear functions, identified with sections of
$W^{\upvee}$, one has
$$
\{\ell,m\}=-B^{-1}(\ell,m).
$$
For $f$ a function on $T^*V$, linear on the fibers and
corresponding to a vector field $F$ on $V$, $\{f,\quad\}$
is $\nabla_F$ on functions on $\Pi W$.

It remains to compute $[f{\uphat},g{\uphat}]$ as being
of the form $-i\hslash\{f,g\}{\uphat}$ when $f$ and $g$
are linear functions on $T^*V$, corresponding to vector
fields $F$ and $G$.
We have
$$
\align
[f{\uphat},g{\uphat}] &=[-i\hslash \nabla_F,
-i\hslash\nabla_G]=(-i\hslash)^2\left(\nabla_{[F,G]}+
  R(F,G)\right)\\
&=-i\hslash\left(\{f,g\}{\uphat}\!\!\!_{_{\ssize T^*V}}-i\hslash
  R(F,G)\right)
\endalign
$$
for $\{\,\,,\,\,\}_{T^*V}$ the Poisson bracket on $T^*V$
and $F$ the curvature $2$-form of $W$, with values in
$\SO(W)$ (which acts on $\scrH_V$).

Let $S^\beta \alpha$ be the matrix of $S\in\SO(W)$,
relatively to a local basis $e_\alpha$ of $W$: \ $S$ is
$(x^\alpha)\mapsto (S^\alpha{}_\beta x^\beta)$.
Let $e^\alpha$ be the dual basis, and move the indices
up and down using $B$.
The action of $S$ on $\scrH_V$ is then the Clifford
multiplication by $S_{\alpha\beta}e^\alpha e^\beta$ (sum
on $\alpha$ and $\beta$) in the Clifford algebra.
This follows from
$$
\multline
[S_{\alpha\beta}e^\alpha
e^\beta,e^\gamma]=S_{\alpha\beta}
(e^\alpha[e^\beta,e^\gamma]-[e^\alpha,e^\gamma]e^\beta)
  \qquad\text{(superbrackets)}\\
=\tfrac12 S_{\alpha\beta}(e^\alpha b^{\beta,\gamma}
-b^{\alpha,\gamma}e^\beta)=-S_{\alpha\beta}b^{\alpha\gamma}
  e^\beta=-S^\gamma \beta e^\beta=
   S(e^\gamma)
\endmultline
$$
for the Lie algbra action.

If we apply this to the curvature, and if $\psi^\alpha$
are the odd functions on $M$ corresponding to the
$e^\alpha$, this gives $[f{\uphat},g{\uphat}]=-i\hslash
\{f,g\}{\uphat}$ for
$$
\{f,g\}=\{f,g\}_{T^*V}-R(F,G)_{\alpha,\beta}
\psi^\alpha\psi^\beta
$$

This Poisson bracket corresponds to the following
symplectic $2$-form $\Omega$.
Let $\alpha_{T^*V}$ be the canonical $1$-form $pdq$ on
$T^*V$, as well as its pull back to $M$.
Let $\alpha_{\Pi W}$ be the $1$-form on $\Pi W$ whose
inverse image by a section $\psi$ is
$\frac12(\psi,\nabla\psi)$.
If $(e_\alpha)$ is a local basis of $W$, and if the
connection is given by the endomorphism valued $1$-form
$\gamma^\alpha{}_\beta$:
$$
\nabla(x^\alpha e_\alpha)=(dx^\alpha+\gamma^\alpha{}_\beta 
x^\beta) e_\alpha,
$$
one has
$$
\alpha_{\Pi W}=\tfrac12\psi_\alpha(d\psi^\alpha
  +\gamma^\alpha{}_\beta \psi^\beta).
$$
With those notations,
$$
\Omega=d\alpha_{T^*V}+d\alpha_{\Pi V}.
$$
If $(e_\alpha)$ is an orthonormal basis, one has
$$
d\alpha_{\Pi V}=\tfrac12 \sum\limits_{\alpha}(d\psi^\alpha
  +\gamma^\alpha{}_\beta
\psi^\beta)^2+\sum\limits_{\alpha,\beta}
R^\alpha{}_\beta \psi^\alpha \psi^\beta.
$$
















\enddocument



