%Date: Fri, 3 Apr 1998 15:34:24 -0500 (EST)
%From: Pavel Etingof <etingof@math.harvard.edu>

\input amstex
\documentstyle{amsppt}
\magnification 1200
\NoRunningHeads
\NoBlackBoxes
\document

\def\opsi{\overrightarrow{\psi}}
\def\h{\frak h}
\def\tW{\tilde W}
\def\Aut{\text{Aut}}
\def\tr{{\text{tr}}}
\def\ell{{\text{ell}}}
\def\Ad{\text{Ad}}
\def\u{\bold u}
\def\m{\frak m}
\def\O{\Cal O}
\def\tA{\tilde A}
\def\qdet{\text{qdet}}
\def\k{\kappa}
\def\RR{\Bbb R}
\def\be{\bold e}
\def\bR{\overline{R}}
\def\tR{\tilde{\Cal R}}
\def\hY{\hat Y}
\def\tDY{\widetilde{DY}(\g)}
\def\R{\Bbb R}
\def\h1{\hat{\bold 1}}
\def\hV{\hat V}
\def\deg{\text{deg}}
\def\hz{\hat \z}
\def\hV{\hat V}
\def\Uz{U_h(\g_\z)}
\def\Uzi{U_h(\g_{\z,\infty})}
\def\Uhz{U_h(\g_{\hz_i})}
\def\Uhzi{U_h(\g_{\hz_i,\infty})}
\def\tUz{U_h(\tg_\z)}
\def\tUzi{U_h(\tg_{\z,\infty})}
\def\tUhz{U_h(\tg_{\hz_i})}
\def\tUhzi{U_h(\tg_{\hz_i,\infty})}
\def\hUz{U_h(\hg_\z)}
\def\hUzi{U_h(\hg_{\z,\infty})}
\def\Uoz{U_h(\g^0_\z)}
\def\Uozi{U_h(\g^0_{\z,\infty})}
\def\Uohz{U_h(\g^0_{\hz_i})}
\def\Uohzi{U_h(\g^0_{\hz_i,\infty})}
\def\tUoz{U_h(\tg^0_\z)}
\def\tUozi{U_h(\tg^0_{\z,\infty})}
\def\tUohz{U_h(\tg^0_{\hz_i})}
\def\tUohzi{U_h(\tg^0_{\hz_i,\infty})}
\def\hUoz{U_h(\hg^0_\z)}
\def\hUozi{U_h(\hg^0_{\z,\infty})}
\def\hg{\hat\g}
\def\tg{\tilde\g}
\def\Ind{\text{Ind}}
\def\pF{F^{\prime}}
\def\hR{\hat R}
\def\tF{\tilde F}
\def\tg{\tilde \g}
\def\tG{\tilde G}
\def\hF{\hat F}
\def\bg{\overline{\g}}
\def\bG{\overline{G}}
\def\Spec{\text{Spec}}
\def\tlo{\hat\otimes}
\def\hgr{\hat Gr}
\def\tio{\tilde\otimes}
\def\ho{\hat\otimes}
\def\ad{\text{ad}}
\def\Hom{\text{Hom}}
\def\hh{\hat\h}
\def\a{\frak a}
\def\t{\hat t}
\def\Ua{U_q(\tilde\g)}
\def\U2{{\Ua}_2}
\def\g{\frak g}
\def\n{\frak n}
\def\hh{\frak h}
\def\sltwo{\frak s\frak l _2 }
\def\Z{\Bbb Z}
\def\C{\Bbb C}
\def\d{\partial}
\def\i{\text{i}}
\def\ghat{\hat\frak g}
\def\gtwisted{\hat{\frak g}_{\gamma}}
\def\gtilde{\tilde{\frak g}_{\gamma}}
\def\Tr{\text{\rm Tr}}
\def\l{\lambda}
\def\I{I_{\l,\nu,-g}(V)}
\def\z{\bold z}
\def\Id{\text{Id}}
\def\<{\langle}
\def\>{\rangle}
\def\o{\otimes}
\def\e{\varepsilon}
\def\RE{\text{Re}}
\def\Ug{U_q({\frak g})}
\def\Id{\text{Id}}
\def\End{\text{End}}
\def\gg{\tilde\g}
\def\b{\frak b}
\def\S{\Cal S}
\def\L{\Lambda}
\def\tl{\tilde\lambda}

\topmatter
\title Lecture II-11: Supersymmetric field theories
\endtitle
\author {\rm {\bf Edward Witten} }\endauthor
\endtopmatter

\centerline{Notes by Pavel Etingof and David Kazhdan}

\vskip .1in

{\bf 11.1. General remarks on supersymmetry}

Starting from today we study field theories with supersymmetry, i.e.
theories whose symmetry group has a nontrivial extension to a supergroup.
As usual, by even and odd (infinitesimal) symmetries we mean even,
respectively odd, elements of the Lie superalgebra of this
supergroup.

It often happens that a solution to the classical
field equations has nontrivial odd symmetries.
Examples:

gradient flowlines in Morse theory

holomorphic curves

instantons

monopoles

Seiberg-Witten solutions

hyperKahler structures

Calabi-Yau metrics

Metrics of $G_2$ and $Spin_7$ holonomy

In the next sections we will consider some of these examples.

{\bf 11.2. Supersymmetric solitons (BPS states).}

First consider a case in which the
gradient flowlines of Morse theory will appear. Let
$h:\R^n\to \R$ be a Morse function
(i.e. its critical points are nondegenerate
and $|\nabla h|$ grows at infinity). A Morse function
always has finitely many critical points.

Consider (in Minkowski signature)
the theory of maps $\Phi: \R^{2|2}\to \R^n$ with the Lagrangian
$$
\Cal L=\int d^2xd^2\theta(\frac{1}{2}D_+\Phi D_-\Phi-h(\Phi)),\tag 11.1
$$
where $D_\pm=\frac{\d}{\d\theta_\pm}-\theta_\pm\d_\pm$,
$\d_\pm=\frac{\d}{\d x_\pm}$,
$x_\pm =\frac{1}{2}(t\pm x)$. This model has an obvious supersymmetry under
$Q_\pm=\frac{\d}{\d\theta_\pm}+\theta_\pm\d_\pm$.
These supersymmetry operators satisfy the obvious commutation relations
$Q_\pm^2=\d_\pm$, $\{Q_+,Q_-\}=0$.

We have $(Q_+\pm Q_-)^2=2\frac{\d}{\d t}$. So if we look for classical
solutions with time translational symmetry (i.e. for solitons),
we may in particular look for those of them which are invariant
under one of the supersymmetry, say $Q_+\pm Q_-$.

We have
$$
Q_+\pm Q_-=(\frac{\d}{\d\theta_+}\pm \frac{\d}{\d\theta_-})
+ (\theta_+\pm \theta_-)\frac{\d}{\d t}
+ (\theta_+\mp \theta_-)\frac{\d}{\d x}.\tag 11.2
$$
Thus, the supersymmetry condition for time-independent solutions is
$$
[(\frac{\d}{\d\theta_+}\pm \frac{\d}{\d\theta_-})
+ (\theta_+\mp \theta_-)\frac{d}{d x}]\Phi=0.\tag 11.3
$$
Let us look for even solutions. It is easy to show that
such solutions are of the form
$\Phi=\phi+\theta_+\theta_-\nabla h(\phi)$. For them, the supersymmetry
condition is
$$
\frac{d\phi}{d x}\mp \nabla h(\phi)=0,\tag 11.4
$$
which is the condition for the gradient flowline.
Thus, supersymmetric solitons are the flowlines of the gradient flow.

Notice that these 1-st order equations imply the 2-nd order equations of
motion. Indeed, it is easy to show that the equations of motion are
$$
\d_+\d_-\phi+\nabla(\nabla h)^2(\phi)=0,\tag 11.5
$$
or for time-independent solutions
$$
\frac{d^2\phi}{d x^2}=\nabla(\nabla h)^2(\phi),\tag 11.6
$$
which can be obtained by differentiation of (11.4) with respect to $x$.

Another way to see this: the Lagrangian for time-independent
fields (i.e the Hamiltonian) is
$$
H(\phi)=\int dx(\frac{1}{2}(\frac{d\phi}{d x})^2+\frac{1}{2}(\nabla
h(\phi))^2)
.\tag 11.7
$$
Rewriting $H$, we get
$$
H(\phi)=\frac{1}{2}\int dx(\frac{d \phi}{d x}\mp \nabla(\phi))^2\pm
\int_{-\infty}^{\infty} dh(s).\tag 11.8
$$
Since the last term is locally constant on the space of fields of finite
energy, a supersymmetric solution provides the global minimum for the
energy in each connected component of the space of fields.
The value of energy at this minimum is $S=|\int dh|$.

\proclaim{Definition} Supersymmetric solitons,
that is classical
solutions invariant under some supersymmetries, are called
classical BPS states.
\endproclaim

{\bf 11.3. The role of BPS states in quantum theory.}

We have mentioned above and used the fact that the vector fields
$Q_+,Q_-$ commute. Since the space of solutions of the classical
field equations is a symplectic supermanifold, these vector fields
must be (at least locally) generated by some Hamiltonian functions
 $\tilde Q_+,\tilde Q_-$ (defined up to adding a locally constant function).
But these functions need not Poisson commute: their Poisson bracket
has to be a locally constant function, not necessarily equal to zero.

In our case, the functions $\tilde Q_+,\tilde Q_-$ are easy to write down:
if $\Phi=\phi+\theta_+\psi_++\theta_-\psi_-+\theta_+\theta_-F$,
then
$$
\tilde Q_+=\int dx(\psi_+\d_+\phi+\psi_-\nabla h(\phi)),
\tilde Q_-=\int dx(\psi_-\d_-\phi-\psi_+\nabla h(\phi)).\tag 11.9
$$
The computation of the Poisson bracket gives $\{\tilde Q_+,\tilde Q_-\}=
-2S$, $S=\int dh=h(\phi(\infty))-h(\phi(-\infty))$.

 From this picture it is clear what will happen with operators
$Q_+,Q_-$ in quantum theory. We will have $Q_+^2=P_+,Q_-^2=P_-$
(where $P_+,P_-$ are the corresponding momentum operators),
and
$(Q_+\pm Q_-)^2=2(H\mp \hat S)$, where $H=(P_++P_-)/2$ is the Hamiltonian,
and $\hat S$ commutes with local operators.

In quantum theory, to every connected component $X_a$ of the space $X$
of fields of finite energy there corresponds a summand $\Cal H_a$ of
the Hilbert space. We expect that,
if there is no breaking of supersymmetry, to every supersymmetric soliton
$\Phi\in X_a$ there corresponds a state $\Psi$ in $\Cal H_a$ which is also
supersymmetric: $(Q_+\pm Q_-)\Psi=0$. Then
$(H\mp S_a)\Psi=0$, where $S_a=\hat S|_{\Cal H_a}$ is a scalar.
Therefore, since
$H\ge 0$, we have $H\Psi=|S_a|\Psi$. In particular, for every connected
component of $X$
there is only one supersymmetry (out of the two), for which there
can be supersymmetric states in this component.

In general, on $\Cal H_a$ we have $H\ge |S_a|$.

Now we want to determine whether there is a supersymmetric quantum
state, that is a quantum state annihilated by $Q_+\pm Q_-$, corresponding
to the supersymmetric classical state.
In the lowest order of perturbation theory one
finds no bosonic zero mode except the translations and no fermionic
zero mode.  Hence, in that approximation the lowest energy state of given
momentum is
unique and nondegenerate.  Also, the theory in the vacuum sector
has a mass gap (classically and therefore for weak enough coupling)
so the unique ground state that is found in the leading approximation
is isolated from any continuum.  Hence the quantum theory for weak
enough coupling has an isolated and unique ground state in this sector
of the Hilbert space, and it follows from the supersymmetry
algebra that this state must be annihilated by $Q_+\pm Q_-$.
This massive state is called a quantum BPS state.

However, if (in a family of theories)
two critical points of $h$ collide and become a degenerate critical point,
the mass of BPS paths between them goes to zero, so we can
expect that the corresponding sector of Hilbert space loses its mass gap,
and massless particles appear. At such a point,
supersymmetric states can appear or disappear in the quantum theory. We will
make this more explicit later in the context of $N=2$ supersymmetry.

{\bf 11.4. N=2 supersymmety in 2 dimensions.}

Consider the space $\R^{2|4}$ with coordinates
$x_+,x_-,\theta_+,\theta_-,\bar\theta_+,\bar\theta_-$.
This space admits an action of the N=2 supersymmetry algebra,
with supersymmetry generators
$$
\gather
Q_+=\frac{\d}{\d \theta_+}+\bar\theta_+\frac{\d}{\d x_+},
\bar Q_+=\frac{\d}{\d\bar\theta_+}+\theta_+\frac{\d}{\d x_+}, \\
Q_-=\frac{\d}{\d \theta_-}+\bar\theta_-\frac{\d}{\d x_-},
\bar Q_-=\frac{\d}{\d\bar\theta_-}+\theta_-\frac{\d}{\d x_-}, \tag 11.10
\endgather
$$
Let us also introduce vector fields
$$
\gather
D_+=\frac{\d}{\d \theta_+}-\bar\theta_+\frac{\d}{\d x_+},
\bar D_+=\frac{\d}{\d\bar\theta_+}-\theta_+\frac{\d}{\d x_+}, \\
D_-=\frac{\d}{\d \theta_-}-\bar\theta_-\frac{\d}{\d x_-},
\bar D_-=\frac{\d}{\d\bar\theta_-}-\theta_-\frac{\d}{\d x_-}, \tag 11.11
\endgather
$$
which commute with the supersymmetry generators. Therefore, any Lagrangian
which
is written in terms of the $D$'s is supersymmetric.

Recall that a chiral function (or superfield) on $\R^{2|4}$ is a function
satisfying the equations $\bar D_+\Phi=\bar D_-\Phi=0$.
A general solution to these equations has the form
$$
\gather
\Phi=\phi-\theta_+\bar\theta_+\d_+\phi-
-\theta_-\bar\theta_-\d_-\phi+\theta_+\bar\theta_+\theta_-\bar\theta_-
\d_+\d_-\phi+\theta_+\theta_-F+\\
\theta_+\psi_++\theta_-\psi_-
-\theta_+\bar\theta_+\theta_-\d_+\psi_--
\theta_-\bar\theta_-\theta_+\d_-\psi_+.\tag 11.12\endgather
$$
You can read more about chiral functions in the superhomework.

Consider the theory of
chiral maps $\Phi$ of $\R^{2|4}$ into $\C^n$, with the
Lagrangian
$$
\frac{1}{2}\int d^2x d^4\theta \Phi\bar\Phi+
\int d^2xd\theta_+d\theta_- W(\Phi)+
\int d^2xd\bar\theta_+d\bar\theta_-\bar W(\Phi),\tag 11.13
$$
where $W$ is a holomorphic function on $\C^n$, called the superpotential.
This Lagrangian is N=2 supersymmetric. In components
(for $x_+=z,x_-=\bar z$), it looks like
$$
\gather
\frac{1}{2}\int d^2x(|d\phi|^2-|F|^2+
W'(\phi)F+\bar W'(\phi)\bar F+\text{terms with fermions}).\tag 11.14
\endgather
$$
Setting all fermions to zero and the ``dummy'' field $ F$ to the
stationary point $ F=\bar W'(\phi)$, we get the bosonic energy functional
$$
\gather
H=\frac{1}{2}\int d^2x(|d\phi|^2
+|W'(\phi)|^2).\tag 11.15
\endgather
$$
It is easy to see that this functional coincides with (11.7),
for the function $h=\text{Re}(e^{-i\alpha}W)$, where $\alpha$ is any real
number.
Thus, at the classical level
we are doing a special case of the previous problem.
However, now we have more supersymmetry and therefore a more interesting
theory.

{\bf 11.5. N=2 BPS states}

In the theory we are considering, there is an important symmetry called
the R-symmetry. It acts according to $\theta_+\to e^{i\beta}\theta_+,
\theta_-\to e^{-i\beta}\theta_-$. If we require that $\Phi$ is unchanged under
this symmetry (i.e. Bose fields are unchanged and
$\psi_+\to e^{-i\beta}\psi_+,\psi_-\to e^{i\beta}\psi_-$), then Lagrangian
(11.13) is obviously invariant under the symmetry.

The commutation relations for the supersymmetry Hamiltonians are
$$
\gather
\{Q_\pm,\bar Q_\pm\}=2P_\pm, \\
\{Q_+,\bar Q_-\}=\{\bar Q_+,Q_-\}=0 \text{ (by R-symmetry) },\\
\{Q_+,Q_-\}=T,\ \{\bar Q_+,\bar Q_-\}=\bar T,\tag 11.16
\endgather
$$
where $T$ is analogous to $S$ in Section 11.2 -- it is a locally constant
function on the space of classical solutions (which is, unlike $S$, not
necessarily real), and for brevity we drop twiddles over $Q$'s.
Also, the squares of all the $Q$'s are zero.

In fact, the function $T$ is easy to compute, like
the function $S$ in the previous problem. Namely,
$$
T=W(\phi(\infty))-W(\phi(-\infty)).\tag 11.17
$$

Now we will look at supersymmetric states. Choose a real number
$\alpha$ and look for states which are invariant under
two supersymmetries $Q_1(\alpha)==Q_++e^{i\alpha}\bar Q_-$,
$Q_2(\alpha)=\bar Q_+-e^{-i\alpha}Q_-$.

We have
$$
\{Q_1,Q_2\}=2i(H-\text{Re}(e^{-i\alpha}T)),\tag 11.18
$$
which implies that supersymmetric classical states have to be
time-independent.

The equation $Q_1\Phi=0$ for an even function $\Phi$ gives
(in the time-independent case):
$$
\frac{\d \phi}{\d x}=e^{-i\alpha}\overline{W'(\phi)},\tag 11.19
$$
and the second equation gives the same result.
This implies, in particular, that
$$
\bar T=\bar W(\phi(\infty))-\bar W(\phi(-\infty))=
\int\bar W'(\phi)\frac{d\phi}{d x}dx=
e^{-i\alpha}\int_{\-\infty}^{\infty} |W'(\phi)|^2dx,\tag 11.20
$$
which implies that $\alpha=\text{arg}T$.
Thus, from equation (11.28) we get that for a supersymmetric solution
of the classical equations we have
$$
H=Re(e^{-i\alpha}T)=|T|.\tag 11.21
$$
For other states in the connected component of this solution
we have $H\ge |T|$.

Now, what are the supersymmetric solutions (of finite energy) geometrically?
It is clear from equation (11.19) that they are separatrices
between critical points of $W$ for the gradient flow of
$h=\text{Re}(e^{-i\alpha}W)$.

Now let us turn to quantum theory. From classical considerations
we saw that in our theory $H\ge |T|$, and in a nondegenerate
case $H=|T|$ only for BPS states. Therefore, we should hope
that in quantum theory the same situation takes place, apriori with
a corrected value of $T$.

Consider the generic situation when all
zero-modes of the Hamiltonian near a classical BPS state
arise from the superPoincare group.
This is the case if for any 3 critical points
$a,b,c$ of the potential we have
$|T_{ac}|<|T_{ab}|+|T_{bc}|$, where $T_{ab}$ is the value of $T$
on the component of the space of solutions which go from $a$ to $b$.
In other words, this is the case when the gradient flowline between $a,b$
never passes through $c$.

In this case, in quantum theory,
for small values of the coupling, we expect
that the point $|T|$ in the spectrum of $H$ occurs discretely.
 From our classical computations,
we expect that the eigenspace corresponding to this eigenvalue
is finite-dimensional, and is an irreducible representation
of the odd part of the superPoincare algebra, in which $Q_1,Q_2$ act by zero.
This representation is nothing but the space of sections
of the equivariant vector bundle on the upper part of the hyperboloid,
where the fiber is the standard 2-dimensional irreducible representation
of the 4-dimensional Clifford algebra generated by the two remaining
supersymmetries.

Notice that all other superPoincare representations occuring
in this theory have to be not 2-dimensional but rather 4-dimensional
over the ring of functions on the hyperboloid, since for levels of energy
above $T$ the Clifford algebra satisfied by the supersymmetry operators
corresponds to a nondegenerate quadratic form, and the only irreducible
represenation of this algebra is 4-dimensional.

{\bf 11.6. N=1 Supersymmetry in 4 dimensions.}

Now consider supersymmetry in 4 dimensions. We start with $N=1$ supersymmetry.
In this case the odd part of the supersymmetry algebra is
similar to the $N=2$ case in two dimensions. It is
generated by $Q_\pm$, $\bar Q_\pm$, with relations
$$
\gather
\{Q_\alpha,\bar Q_\beta\}=2P_{\alpha\beta},\\
\{Q_\alpha,Q_\beta\}=0,\\
\{\bar Q_\alpha,\bar Q_\beta\}=0.
\tag 11.22
\endgather
$$
where $\alpha,\beta\in \{+,-\}$, and $P_{\alpha\beta}$ is a basis
of the space of complex linear functions on the spacetime
(These relations exhibit the isomorphism of Poincare representations
$S_+\o S_-\to V_\C$, where $S_\pm$ are the spinor representations.)
A central extension like in (11.16) cannot arise here because
it is prohibited by the Poincare symmetry (this central extension
would have to be in the representation
$S^2\C^2$ of $SU(2)$ (where $\C^2$ is the standard representation),
which is the spin 1 representation and contains no invariants).


The determinant of the quadratic form corresponding to the Clifford algebra
(11.22) is $(P^2)^2$. The rank of this form if $P^2=0$ is 2.
Therefore, massless representations of the superPoincare correspond
to 2-dimensional representations of the Clifford algebra,
and massive representations correspond to
4-dimensional representations of the Clifford algebra.

More precisely,
consider a representation $W$ of the superPoincare and the subspace
$W_p$ on which $P=p$, where $p\in\R^{1,3}$. Let $G_p$ be the stabilizer
of $p$ in the group of rotations,
$H_p$ the maximal torus in $G_p$ (always isomorphic to $U(1)$).
With respect to $H_p$, $W$ has a decomposition in a direct sum
of representations of integer and half-integer spins.
These spins are called helicities of $W$, and each helicity has a
multiplicity.

It is easy to see that
the supersymmetry operators $Q_\alpha,\bar Q_\alpha$ raise
helicity by $1/2$ (since they live in the spinor representation
of the Poincare). Thus, the massless representations
have helicities $j,j+1/2$ with multiplicity 1, and
the massive representations have helicities $j,j+1/2,j+1$
with multiplicities $1,2,1$.

Now let us consider massless particle multiplets which are allowed by
$N=1$ supersymmetry. There are two such basic multiplets.

1. Vector multiplet: a gauge field $A$ and a chiral spinor $\l$ in the adjoint
representation. This is the N=1 analogue of the gauge field.
In particular, in the $U(1)$ case the theory is free. In the infrared,
it generates a massless vector and a massless spinor, so the helicities are
$-1,1$ (for vector) and $-1/2,1/2$ (for spinor). In particular,
the massless representation of the SuperPoincare arising
in this theory is reducible, and splits into two:
the one with helicities $-1,-1/2$ and the one with helicities
$1/2,1$. However, over $\R$ this splitting does not exist and our
representation is irreducible.
A more physical version of this statement is to say that the
helicity $-1,-1/2$ states are related to helicity $1/2,1$ states by CPT
conjugation, so that one pair must be present if the other is.
Thus, this field configuration is the minimal
supersymmetric one which contains a gauge boson.

2. Chiral multiplet: A massless complex
scalar $\phi$ and a massless chiral spinor $\chi$. In this case the story
is analogous. The obtained massless representation
has a spinor and a scalar, so it has helicities
$0,0$ (for scalar) and $-1/2,1/2$ (for spinor). Thus this
representation is again a sum of two, with
helicities $-1/2,0$ and $0,1/2$. This decomposition is only
valid over $\C$ and not over $\R$; over $\R$, the representation
is irreducible, so this combination is the smallest supersymmetric one
containing a scalar.

The massive versions of these multiplets are as follows.

1. Massive vector multiplet (or hypermultiplet).
The minimal real supersymmetric representation containing a massive vector
has helicities $-1,-1/2,0,1/2,1$ with multiplicities
$1,2,2,2,1$. This involves a massive vector, two massive spinors
(chiral and antichiral), and a real massive scalar.
The corresponding 8-dimensional representation of the Clifford algebra
is (over $\C$) a sum of two 4-dimensional representations.

2. Massive chiral multiplet. Same as massless chiral multiplet (we could
consider the same fields with masses, such that the mass of bosons
equals the mass of fermions).

{\bf 11.7. N=2 Supersymmetry in 4 dimensions.}

In the N=2 case, we have two copies of operators $Q$:
$Q^{(1)}$ and $Q^{(2)}$,
and they commute in the same way as before if the indices
are equal, and give zero commutator if they are not equal
(as vector fields).
However, now there is a possibility for a central extension: it is
no longer prohibited by the Poincare symmetry.

Consider first the case when the central term is zero.
In this case the quadratic form of the Clifford algebra
is nondegenerate (in 8 dimensions) in the massive case, and
has rank 4 in the massless case. Thus, an irreducible massive representation
should be 16-dimensional, and an irreducible massless representation should be
4-dimensional. The helicities for representations
are obained like in $N=1$ case. In particular,
in a massive representation the helicities are in groups of the form
(j,j+1/2,j+1,j+3/2,j+2) with multiplicities (1,4,6,4,1), and
in a massless representation they are in groups (j,j+1/2,j+1) with
multiplicities (1,2,1).

Now let us consider the simplest $N=2$ supersymmetric theory.
It is obtained by combining fields from a (classically)
massless vector multiplet
and a massless chiral multiplet in adjoint representation.
Thus the fields are: from vector multiplet -- $(A,\l)$,
from chiral multiplet -- $(\phi,\chi)$.
The Lagrangian is the minimal $N=2$ supersymmetric Lagrangian on these fields.
Such a Lagrangian exists and is uniquely determined by the minimality
condition. In the $U(1)$ case the theory is free, but
in the nonabelian case there are nontrivial interactions.

What does this theory do in the infrared? In the $U(1)$ case, the answer
is simple: we should add together the representations for the vector and
chiral multiplets. We get helicities (-1,-1/2,0,1/2,1)
with multiplicities (1,2,2,2,1). This is the sum of two complex conjugate
massless representations of the N=2 Clifford algebra: the one with
helicities (0,1/2,1) and multiplicities (1,2,1) and the one with
helicities (-1,-1/2,0) and multiplicities (1,2,1).

But now let us consider the nonabelian case (say the gauge group $G$
is simple). Then the bosonic fields are the gauge field $A$
and a scalar $\phi$ in the complexified adjoint representation
$\g_\C$. The bosonic part of the Lagrangian is
$$
L_{bosonic}=
\int (\frac{1}{4e^2}F^2+|d_A\phi|^2+|[\phi,\bar\phi]|^2).\tag 11.23
$$
Thus the bosonic part of the space of classical vacua is the set of
solutions of the equations $[\phi,\bar\phi]=0$ modulo the action of $G$.
This space can be identified with ${\frak t}_\C/W$, where
${\frak t}$ is the maximal commutative subalgebra in $\g$
and $W$ the Weyl group. In the case $G=SU(2)$, this quotient is identified
with $\C$ by introducing the global coordinate $u=Tr(\phi^2)$
(the u-plane).

Recall from Lecture 2 that in this situation we have gauge symmetry breaking
from $G$ to the centralizer $H$ of $\phi$. In particular, for $SU(2)$
near $u\ne 0$ the gauge symmetry is broken classically from $SU(2)$ to
$U(1)$ (Higgs mechanism). The components of the gauge field which are charged
nontrivially with respect to the surviving $U(1)$ will become massive.
By N=2 supersymmetry, the same will happen for the corresponding
components of $\l,\phi,\psi$. Thus, in the charge $2$ and $-2$ sectors
of the Hilbert space the lowest energy states will be in a massive
representation with helicities as before:
(-1,-1/2,0,1/2,1)
with multiplicities (1,2,2,2,1). However, we know that there is no
such representation without central extension. Thus,
without central extension we get a contradiction, and hence
the central extension
must appear.

The central extension will show up in the commutation relations
between $Q_\alpha^{(1)}$ and $Q_\beta^{(2)}$:
$$
\{Q_\alpha^{(1)},Q_\beta^{(2)}\}=\e_{\alpha\beta}Y,\tag 11.24
$$
where $Y$ is an operator which commutes with all local operators
(central charge). Before the central charge, the algebra
had a $U(2)$ R-symmetry (action on indices 1 and 2), but
the central charge breaks this symmetry down to $SU(2)$
(the chiral $U(1)$ symmetry is anomalous in our theory,
because of the index problems, and it is broken to $\Z/2\Z$).

It is easy to see that in this case the massive particles
considered above must have mass exactly $|Y|$ (this is where
the quadratic form of the Clifford algebra becomes degenerate).

It turns out that classically there are BPS states
which correspond to these massive particles. Namely
if we are at the vacuum $u\in\C$,
we have
$$
\phi\sim \frac{1}{\sqrt{2}}\left(\matrix a&0\\ 0&-a\endmatrix\right).\tag 11.25
$$
where $a=\pm u^{1/2}$. Let $a=\rho e^{i\alpha}$. We should look for
BPS states (i.e. states which are time-independent
and invariant under half of the supersymmetry)
which satisfy the condition (11.25) at infinity. If $Y=|Y|e^{i\beta}$
then the equation of being invariant under half of the supersymmetry
is
$$
F=e^{-i\beta}*d_A\phi,\tag 11.26
$$
where $*$ is in $\R^3$.
(the BPS monopole equations). Since $F$ is real, we must have
$\alpha=\beta$ modulo $\pi$.

It can be shown that such BPS states exist in sectors with charges
2 and $-2$ (charges are the eigenvalues of the infinitesimal operator
corresponding to the unbroken $U(1)$-gauge symmetry).
When these solutions are quantized,
we will get the same result as in 2 dimensions.
Namely, in quantum theory,
we will get a representation of the superPoincare
with helicities coming in groups (j,j+1/2,j+1) with multiplicities
(1,2,1). Adding two copies of such groups with $j=-1$ and $0$, we will get
the massive hypermultiplet (=massive vector multiplet); this multiplet has
the right helicities, which we found by considering the Higgs mechanism.
Thus, in presence of central charge we get no contradiction.

{\bf Remark.}
Computing the commutators of the $Q$-s using currents,
one can show that classically
$$
Y=\int_{\Sigma}(\phi *F+\frac{1}{e^2}\phi F),\tag 11.27
$$
where $\Sigma$ is a distant sphere in the space cycle.
So it is a combination of the electric and magnetic charge
for the effective $U(1)$ theory.


\end


