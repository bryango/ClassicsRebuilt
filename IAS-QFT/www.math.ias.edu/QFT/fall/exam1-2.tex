\input amstex
\magnification=1200
\documentstyle {amsppt}
\pagewidth{6.5 true in}
\pageheight{8.9 true in}
\baselineskip-18pt

\font\dotless=cmr10 

\font\itdotless=cmti10
                    
\def\umi{{\"{\dotless\char'020}}}
\def\umj{{\"{\dotless\char'021}}}
\nologo
\input pictex

\noindent
{\bf Fall Term Exam, N$^{\text{o}}$. 2}\qquad\qquad\qquad
\qquad\qquad\qquad\qquad\qquad(solution by D. Freed)
\smallskip
\hbox to \hsize{\hrulefill}
\bigskip
\noindent
{\bf Problem:}
\medskip
Compute in $\phi^4$ theory in four dimensions the
anomalous dimension of the operator $\phi^2$, in one--loop
approximation.
\bigskip
\noindent
{\bf Solution:}
\medskip
Consider the Lagrangian density in Euclidean space
$$
\Cal
L=\frac{1}{2}\,|d\phi|^2+\frac{m^2}{2}\,\phi^2+\frac{\lambda}
{4!}\,\phi^4-j\phi^2.
$$
The coefficient $j$ is renormalized to 1-loop 
by the diagram shown,\hfill\break
%\input aa.tex
%%% beginning of file aa.tex
%\input pictex
%\font\thinlinefont=cmr5
%
%\begingroup\makeatletter\ifx\SetFigFont\undefined
% extract first six characters in \fmtname
%\def\x#1#2#3#4#5#6#7\relax{\def\x{#1#2#3#4#5#6}}%
%\expandafter\x\fmtname xxxxxx\relax \def\y{splain}%
%\ifx\x\y   % LaTeX or SliTeX?
%\gdef\SetFigFont#1#2#3{%
%  \ifnum #1<17\tiny\else \ifnum #1<20\small\else
%  \ifnum #1<24\normalsize\else \ifnum #1<29\large\else
%  \ifnum #1<34\Large\else \ifnum #1<41\LARGE\else
%     \huge\fi\fi\fi\fi\fi\fi
%  \csname #3\endcsname}%
%\else
%\gdef\SetFigFont#1#2#3{\begingroup
%  \count@#1\relax \ifnum 25<\count@\count@25\fi
%  \def\x{\endgroup\@setsize\SetFigFont{#2pt}}%
%  \expandafter\x
%    \csname \romannumeral\the\count@ pt\expandafter\endcsname
%    \csname @\romannumeral\the\count@ pt\endcsname
%  \csname #3\endcsname}%
%\fi
%\fi\endgroup
$$
\hbox{\beginpicture
\setcoordinatesystem units < 1.000cm, 1.000cm>
%\unitlength= 1.000cm
\linethickness=1pt
%\setplotsymbol ({\makebox(0,0)[l]{\tencirc\symbol{'160}}})
%\setshadesymbol ({\thinlinefont .})
\setlinear
%
% Fig ELLIPSE
%
\linethickness= 0.500pt
%\setplotsymbol ({\thinlinefont .})
\ellipticalarc axes ratio  1.048:0.572  360 degrees 
	from  4.413 20.923 center at  3.365 20.923
%
% Fig POLYLINE object
%
\linethickness= 0.500pt
%\setplotsymbol ({\thinlinefont .})
\plot  3.778 19.082  3.778 19.082 /
%
% Fig POLYLINE object
%
\linethickness= 0.500pt
%\setplotsymbol ({\thinlinefont .})
\plot  3.778 19.082  3.778 19.082 /
%
% Fig POLYLINE object
%
\linethickness= 0.500pt
%\setplotsymbol ({\thinlinefont .})
\plot  4.255 20.669  5.524 21.780 /
%
% Fig POLYLINE object
%
\linethickness= 0.500pt
%\setplotsymbol ({\thinlinefont .})
\plot  4.413 20.987  5.048 19.558 /
%
% Fig TEXT object
%
\put{$j$} [lB] at  2.032 20.352
%
% Fig TEXT object
%
\put{$-\lambda$} [lB] at  4.255 20.034
%
% Fig TEXT object
%
\put{$x$} [lB] at  2.191 20.828
\linethickness=0pt
\putrectangle corners at  2.032 21.780 and  5.524 19.082
\endpicture}
$$
%%% end of file aa.tex
\noindent
which contributes to the correlation function
$\langle\phi^2\phi\phi\rangle$.  Evaluating in momentum
space, we obtain
$$
\frac{-j\lambda}{2}\,\,\int\,\,\frac{d^4q}{(2\pi)^4}\quad\frac
{1}{(q^2+m^2)\,((k-q)^2+m^2)},
$$
where the factor of 2 comes from the symmetry of the diagram.
This is logarithmically divergent, and cutting off at
$\Lambda$ we compute the divergence as
$\Lambda\longrightarrow\infty$ to be 
$$
\frac{-j\lambda}{2}\cdot\frac{1}{(2\pi)^4}\cdot\text{vol}\,
S^3\cdot
\ell n\,\Lambda\slash\mu =
\frac{-j\lambda}{16\pi^2}\,\,\ell
n\,\Lambda\slash\mu,
$$
where $\mu$ is a fixed constant.
\medskip
To cancel this divergence, we renormalize $j$ to 
$$
\hat{\jmath}_0=j\,\left(1+\frac{\lambda}{16\pi^2}\,\ell
n\,\Lambda\slash\mu\right).
$$
The anomalous dimension of $j$ is computed as
$\gamma_j=\frac{\mu}{j}\,\,\,\frac{\partial j}{\partial\mu}$
with $\hat{\jmath}_0$ fixed and $\Lambda\longrightarrow\infty$.
Thus
$$
0=\gamma_j\,\left(1+\frac{\lambda}{16\pi^2}\,\ell
n\,\Lambda\slash
\mu\right)-
\frac{\lambda}{16\pi^2}+\frac
{\beta}{16\pi^2}\,\ell n
\Lambda\slash\mu,
$$
where $\beta=\mu\,\frac{\partial\lambda}{\partial\mu}$ is
the $\beta$-function.  Since $\gamma_j=O(\lambda)$ and
$\beta=O(\lambda^2)$ we find
$\gamma_j=\frac{\lambda}{16\pi^2}$.  Now the anomalous
dimension of the Lagrangian vanishes, and so the anomalous
dimension $\gamma_{\phi^2}$ of $\phi^2$ equals 
$-\gamma_j$, or 
$$
\gamma_{\phi^2}=-\frac{\lambda}{16\pi^2}.
$$
\bye


