%From: Pierre Deligne <deligne@math.ias.edu>
%Date: Thu, 23 Jan 1997 17:51:56 -0500
%Subject: set6-2.tex

\input amstex
\documentstyle{amsppt}
\magnification=1200
\pagewidth{6.5 true in}
\pageheight{8.9 true in}
\loadeusm

\catcode`\@=11
\def\logo@{}
\catcode`\@=13

\NoRunningHeads

\input pictex.tex

\font\boldtitlefont=cmb10 scaled\magstep2

\def\lam{{\lambda}}
\def\eps{{\varepsilon}}

\def\psibar{\bar{\psi}}
\def\Lbar{\bar{L}}
\def\Sbar{\bar{S}}

\def\Ghat{\hat{G}}

\def\halfspace{\lineskip=1pt\baselineskip=15pt
     \lineskiplimit=0pt}
\def\vrulesub#1{\hbox{\vrule height7pt depth5pt\,}_{#1}}
\def\Ge{{\mathchoice{\,{\scriptstyle\ge}\,}
  {\,{\scriptstyle\ge}\,}
  {\,{\scriptscriptstyle\ge}\,}{\,{\scriptscriptstyle\ge}\,}}}

\def\eps{{\varepsilon}}
\def\Lam{{\Lambda}}


\def\dbC{{\Bbb C}}

\def\Tr{\text{\rm Tr}} \def\ln{\text{\rm ln}}
\def\End{\text{\rm End}} \def\Sym{\text{\rm Sym}}
\def\I{\text{\rm I}} \def\free{\text{\rm free}}
\def\II{\text{\rm II}}
\def\III{\text{\rm III}}


\def\scr#1{{\fam\eusmfam\relax#1}}

\def\scrL{{\scr L}}


\document
\line{{\boldtitlefont Witten's Problems}, Set Six --- 
N$^{\text{o}}$. 2 
\hfill (solution written by D. Freed)}
\smallskip
\hbox to \hsize{\hrulefill}

\bigskip\noindent
{\bf Problem:} 

\smallskip
(2) In this exercise we will compute the one-loop renormalization
of $U(1)$ gauge theory with fermions in four dimensions.  It is
of considerable theoretical importance, and also is part of the
computation of the ``Lamb shift'' which was one of the historic
tests of QED around 1950.

The Lagrangian is
$$
L={1\over 4e^2}\int d^4x F_{ij}F^{ij} 
+\int d^4x (\psi, (D+im)\psi).
\tag1
$$
Here the gauge field is $A=A_i dx^i$ -- a connection on 
a line bundle $\scrL$ --  the curvature is
$F=\frac12 F_{ij}dx^i\wedge dx^j$, and $\psi$ is a charged
spinor field (section of $(S_+\oplus S_-)\otimes \scrL$).

As explained in Fadde\'ev's lectures, to quantize the theory
one must ``pick a gauge.'' One can pick a gauge in which the
kinetic energy for $B=A/e$ is just
$$
L'={1\over 2}\int d^4x \sum_{i.j}(\partial_i B_j)^2.
\tag2
$$
This means that the free propagator of $B$ is
$$
\langle 
B_i(x)B_j(0)\rangle=\delta_{ij}\int {d^4k\over (2\pi)^4}
{1\over k^2}.\tag3
$$

The inverse of the free propagator is therefore, in momentum
space, $ D_{ij}^0=\delta_{ij}k^2$.
The inverse of the exact propagator of 
$B$ is $D_{ij}=D_{ij}^0+
\Sigma_{ij}$, where $\Sigma_{ij}$, the ``self-energy,'' is
to be computed from loops.

A very fundamental property of $\Sigma_{ij}$ 
is that in momentum space
$k^i\Sigma_{ij}(k)=0$.  (An explanation of why it is so will
be given later, maybe in the homework session.) 
This implies that
$$
\Sigma_{ij}(k)=(k^2\delta_{ij}-k_ik_j)F(k^2),
\tag4
$$
with a scalar function $F$.  

The problem here is to compute $F$ in the one-loop approximation.
(More exactly, you will compute $\Sigma_{ij}$, find it has the
form (4), and compute $F$, after some renormalization.)
The importance is twofold. (1) 
The sign of $F$ -- which can be predicted
without computation, but in this exercise we will just compute
it -- is the basic quantity that determines the different
behavior of abelian and nonabelian quantum gauge theories in
four dimensions (this is a slight 
simplification as the definition
of the analog of $F$ is a bit more 
elaborate in the nonabelian case).
(2) The renormalized $F$ is observable and, as I mentioned above,
its calculation is part of the  computation of the Lamb shift.

To compute $F$ in one-loop approximation, just write down
the one-fermion-loop diagram.
The fermion propagator is in momentum 
space $1/(\Gamma\cdot k+im)$
where $\Gamma\cdot k$ is Clifford multiplication by $k$. In the
one loop diagram for $\Sigma_{ij}$, there are two vertices on
the fermion loop; at these vertices one multiplies by $e\Gamma_i$
or $e\Gamma_j$ respectively (the $e$ comes from using $B=A/e$).
The loop gives a supertrace over the spin representation of
$SO(4)$; don't forget that it is four-dimensional as both
$S_+$ and $S_-$ are included here!  
(To avoid confusion, in this context supertrace 
just means $-1$ times
the ordinary trace, since the objects propagating in the loop
are all fermions.  
No distinction is made between $S_+$ and $S_-$.
We explained in a previous homework session 
where the $-1$ comes from.) 
You will have to rationalize
the fermion propagator, writing $1/(\Gamma\cdot k+im)=
(\Gamma\cdot k-im)/(k^2+m^2)$, introduce Feynman propagators
to combine the denominators, perform the $k$ integral, and
at that point you will know that you are on the right 
track when
you demonstrate that $\Sigma$ is of the form claimed in (4).
Before you demonstrate (4) it will appear that you are
dealing with a possibly quadratically divergent integral,
but once you've confirmed (4) it should be clear that there
is only a logarithmic divergence which you'll 
renormalize in the
by now familiar fashion (just subtracting the 
value of the integral
at any given value of $k$, such as $k=0$).  


\bigskip\bigskip\noindent
{\bf Solution:}

\smallskip
We work in Euclidean space.

\smallskip
\halfspace
Write $A=eB$ and add to the lagrangian density a gauge
fixing term
$$
\tfrac{\alpha}{2}\left\vert\sum\limits_{i}\partial_i
B_i\right\vert^2 d^4 x,
$$
where $\alpha$ is a real number to be determined.
Then the free part of the lagrangian for the field $B$ is
$$
\aligned
\tfrac14\sum\limits_{i\not=j} &\Vert\partial_iB_j-\partial_j
B_i \Vert^2 +\tfrac{\alpha}{2}\Vert\sum\limits_{i}
  \partial_i B_i\Vert^2\\
&=\tfrac12\sum\limits_{i\not=j}\Vert\partial_i B_j\Vert^2
 -\tfrac12\sum\limits_{i\not=j}\left<\partial_i B_j,
  \partial_j B_i\right>+\tfrac{\alpha}{2}
  \sum\limits_{i}\Vert\partial_i B_i\Vert^2\\
&\qquad\qquad\qquad
  +\tfrac{\alpha}{2}\sum\limits_{i\not=j}\left<\partial_i
  B_i,\partial_j B_j\right>.
\endaligned
\tag1
$$
Here we write $\left<\,\,\cdot\,\,,\,\,\cdot\,\,\right>$
and $\Vert\,\,\cdot\,\,\Vert$ for the integrated inner
products.
Take $\alpha=1$.
Then the second and fourth terms in (1) cancel using
integration by parts, and the free lagrangian for $B$ is
$$
\tfrac12\sum\limits_{i,j}\Vert\partial_i B_j\Vert^2.
$$
So the free propagator for $B$ in momentum space is
$$
(D_{ij}^0)^{-1}(k)=\tfrac{\delta_{ij}}{k^2}.\tag2
$$
The total lagrangian density (after cancelling the terms
in (1)) is now
$$
\scrL'=\Bigl\{\tfrac12\sum\limits_{i,j}\vert\partial_i
  B_j\vert^2+(i(\Gamma\cdot\partial+m)\psi,\psi)+
  e(\Gamma\cdot B\psi,\psi)\Bigr\}d^4 x.
\tag3
$$
Here $\Gamma^\mu$ is Clifford multiplication in the
$\mu^{\text{th}}$ coordinate direction, and
$i\Gamma\cdot\partial=i\Gamma^\mu \partial_\mu$ is the
flat space Dirac operator.
Our convention is
$$
\Gamma^\mu\Gamma^\nu+\Gamma^\mu\Gamma^\nu=
2\delta^{\mu\nu}.\tag4
$$
The $\Gamma^\mu$ are {\it hermitian} operators, so that
$i\Gamma\cdot\partial$ is hermitian, while $im$ is
skew-Hermitian.
Note that $\psi$ is an odd variable so that
$(i\Gamma\cdot\partial\psi,\psi)$ is purely imaginary,
while $(im\psi,\psi)$ is real.\footnote"*"{Continued to
Minkowski space the lagrangian is real.}
The fermionic propagator in momentum space is
$$
\tfrac{1}{\Gamma\cdot k+im}=\tfrac{\Gamma\cdot k-im}
{k^2+m^2}.\tag5
$$

We remark that our inner product is linear in the first
variable and conjugate linear in the second variable.
Think of $A$ as a connection in a complex line bundle
$L$, so that the spinor fields are sections of $S\otimes
L$, where $S$ is the (complexified) spin space.
We view
$$
Q(\psi,\psibar)=(i(\Gamma\cdot\partial+m)\psi,\psibar)\tag6
$$
as a bilinear form, where $\psi$ is a section of $S\otimes
L$ and $\psibar$ a section of $S\otimes\Lbar$.  
The hermitian inner product on the right hand side of (6)
is written as linear in both variables.
(The spin space $S$ is quaternionic in the Euclidean
signature, so there is an identification $S\cong\Sbar$.
Note that in Minkowski space the spinors are real.)
The ordering of the variable is important, since the
spinor fields are odd (anticommuting).
The same holds for the propagator $Q^{-1}$, and so
fermion lines in Feynman diagrams have a definite
orientation:
$$
\vbox{\beginpicture
\setcoordinatesystem units < .50cm, .50cm>
\setlinear
%
% Fig POLYLINE object
%
\linethickness= 0.500pt
\putrule from 10.128 19.082 to 11.430 19.082
%
% arrow head
%
\plot 11.176 19.018 11.430 19.082 11.176 19.145 /
%
%
% Fig POLYLINE object
%
\linethickness= 0.500pt
\putrule from 11.430 19.082 to 12.668 19.082
%
% Fig TEXT object
%
\put{$\ssize q$} [lB] at 11.208 19.590
%
% Fig TEXT object
%
\put{$\ssize\psi$} [lB] at 12.636 18.447
%
% Fig TEXT object
%
\put{$\ssize\psibar$} [lB] at 10.097 18.383
\linethickness=0pt
\putrectangle corners at 10.097 20.098 and 12.668 18.383
\endpicture}\qquad
\raise9pt\hbox{$\tfrac{\Gamma\cdot q-im}{q^2+m^2}$}
$$

Any loop in a Feynman diagram requires a trace, and by
general principles in a theory with fermions it should be
a {\it supertrace}.
The reasoning is as follows.
The edges around the loop are propagators, which are
bilinear forms.
The vertices represent polynomials and we are instructed
to contract the polynomials with the propagators.
After some contractions we can view the loop as a
composition of operators, where when we close the loop we
must take a trace.
In general, if $V$ is a graded vector space we agree to
contract $V^*$ and $V$ by letting $V^*$ operate on the
left.
An operator on $V$ is identified (without signs) with an
element of $V\otimes V^*$: \ the operator $T$ corresponds
to $Te_i\otimes e^i$, where $\{e_i\}$ is a basis of $V$
and $\{e^i\}$ the dual basis of $V^*$.
The natural contraction of an operator is the
composition
$$
\End\,V\overset\simeq\to\longrightarrow
V\otimes V^*\longrightarrow V^*\otimes V
\longrightarrow\dbC
$$
which is the supertrace.

Since we will use dimensional regularization, we compute
the scaling dimensions of the fields in $d$ dimensional
space:
$$
[B]=\tfrac{d-2}{2},\quad
[\psi]=\tfrac{d-1}{2},\quad
[m]=1,\quad
[e]=\tfrac{4-d}{2}
$$
so we replace $e$ by a dimensionless constant $\lam$ and
a mass parameter $\mu$ (i.e., $[\mu]=[m]=1$):
$$
e\to\lam\mu^{\frac{4-d}{2}}.
$$
It is convenient to define
$$
\epsilon=\tfrac{4-d}{2},
$$
so that the interaction term in the lagrangian density is
$$
\lam\mu^\epsilon(\Gamma\cdot B\psi,\psi).
$$

Now the $1$-loop correction to the $2$-point function of
$B$ comes by evaluating the graph shown.
Denote its value as $-\Sigma_{ij}(k)$.
Summing over all chains made out of this one particle
irreducible graph we see that the corrected $2$-point
function is $D_{ij}^{-1}$, where
$$
D_{ij}=D_{ij}^0+\Sigma_{ij}.\tag7
$$
(See Witten's lectures for an explanation.)
$$
\vbox{\beginpicture
\setcoordinatesystem units < .50cm, .50cm>
\setlinear
%
% Fig ELLIPSE
%
\linethickness= 0.500pt
\ellipticalarc axes ratio  1.492:1.492  360 degrees 
	from 12.732 17.431 center at 11.239 17.431
%
% Fig POLYLINE object
%
\linethickness= 0.500pt
\plot 11.303 18.923 11.208 19.082 /
%
% Fig POLYLINE object
%
\linethickness= 0.500pt
\plot 11.144 18.796 11.303 18.891 /
%
% Fig POLYLINE object
%
\linethickness= 0.500pt
\plot 11.303 16.002 11.462 16.097 /
%
% Fig POLYLINE object
%
\linethickness= 0.500pt
\plot 11.335 15.939 11.494 15.843 /
\linethickness= 0.500pt
%
% Fig CONTROL PT SPLINE
%
% open spline
%
\plot	 8.509 17.335  8.620 17.367
 	 8.747 17.375
	 8.823 17.361
	 8.906 17.335
	 8.992 17.312
	 9.077 17.304
	 9.159 17.312
	 9.239 17.335
	 9.320 17.358
	 9.402 17.363
	 9.486 17.350
	 9.573 17.320
	 /
\plot  9.573 17.320  9.747 17.240 /
\linethickness= 0.500pt
%
% Fig CONTROL PT SPLINE
%
% open spline
%
\plot	 7.017 17.272  7.176 17.335
 	 7.257 17.359
	 7.342 17.367
	 7.431 17.359
	 7.525 17.335
	 7.620 17.313
	 7.715 17.308
	 7.811 17.321
	 7.906 17.351
	 8.001 17.381
	 8.096 17.391
	 8.192 17.381
	 8.287 17.351
	 /
\plot  8.287 17.351  8.477 17.272 /
%
% arrow head
%
\plot  8.218 17.311  8.477 17.272  8.267 17.428 /
%
\linethickness= 0.500pt
%
% Fig CONTROL PT SPLINE
%
% open spline
%
\plot	12.700 17.209 12.859 17.272
 	12.940 17.296
	13.025 17.304
	13.115 17.296
	13.208 17.272
	13.303 17.249
	13.399 17.244
	13.494 17.257
	13.589 17.288
	13.684 17.318
	13.779 17.328
	13.875 17.318
	13.970 17.288
	 /
\plot 13.970 17.288 14.161 17.209 /
%
% arrow head
%
\plot 13.902 17.248 14.161 17.209 13.950 17.365 /
%
\linethickness= 0.500pt
%
% Fig CONTROL PT SPLINE
%
% open spline
%
\plot	14.129 17.240 14.240 17.272
 	14.299 17.282
	14.367 17.280
	14.442 17.266
	14.526 17.240
	14.612 17.216
	14.696 17.209
	14.779 17.216
	14.859 17.240
	14.939 17.263
	15.022 17.268
	15.106 17.255
	15.192 17.224
	 /
\plot 15.192 17.224 15.367 17.145 /
%
% Fig TEXT object
%
\put{$\ssize q-k$} [lB] at 11.144 15.208
%
% Fig TEXT object
%
\put{$\ssize q$} [lB] at 11.113 19.494
%
% Fig TEXT object
%
\put{$\ssize k$} [lB] at  8.255 17.716
%
% Fig TEXT object
%
\put{$\ssize i$} [lB] at  6.953 16.605
%
% Fig TEXT object
%
\put{$\ssize i$} [lB] at  9.557 16.732
%
% Fig TEXT object
%
\put{$\ssize j$} [lB] at 12.827 16.701
%
% Fig TEXT object
%
\put{$\ssize j$} [lB] at 15.335 16.605
%
% Fig TEXT object
%
\put{$\ssize k$} [lB] at 14.065 17.621
\linethickness=0pt
\putrectangle corners at  6.953 20.003 and 15.367 15.208
\endpicture}
$$

We remark that if $V$ is an inner product space and
$Q\colon\,V\to\Sym^2V^*$ a quadratic form which is
equivariant under the orthogonal group $O(V)$, then
$$
Q_v(\xi_1,\xi_2)=E(\vert v\vert^2)(\xi_1,\xi_2)+
  F(\vert v\vert^2)(v,\xi_1)(v,\xi_2),\quad
  v,\xi_1,\xi_2\in V,
$$
for some functions $E$, $F$.
Also, if $Q_v(v,\xi)=0$ for all $\xi$, then
$$
Q_v(\xi_1,\xi_2)=F(\vert
v\vert^2)\{(v,\xi_1)(v,\xi_2)-
  \vert v\vert^2(\xi_1,\xi_2)\}.
$$
This is the form taken by $\sum$.
(This is proved using the Ward identities, which we have
not yet discussed in lectures.)

At last we evaluate the diagram in $d$ dimensions as
$$
-\Sigma_{ij}(k)=\lam^2\mu^{2\eps}\int
\tfrac{d^dq}{(2\pi)^d}\left[-\Tr\left\{
\Gamma^i\tfrac{\Gamma\cdot(q-k)-im}
{(q-k)^2+m^2}\Gamma^j
\tfrac{\Gamma\cdot q-im}{q^2+m^2}\right\}\right].
\tag8
$$
We first expand the numerator of the expression in
braces:
$$
\split
\Tr\Bigl[\Gamma^i(\Gamma^\mu &(q_\mu-k_\mu) -im)\Gamma^j
  (\Gamma^\nu q_\nu-im)\Bigr]\\
&=\Tr\Bigl[(q_\mu-k_\mu)q_\nu\Gamma^i
  \Gamma^\mu\Gamma^j
\Gamma^\nu-m^2\Gamma^i\Gamma^j\Bigr]
\endsplit
\tag9
$$
Here we use the fact that the trace of an odd product of
$\Gamma$ matrices vanishes.\footnote"*"{We evaluate the
$\Gamma$ matrices in $4$ dimensions before continuing the
integral to $d$ dimensions.}
(Note that $\Tr(1)=4$ and
$\Tr\Gamma^i=\Tr\Gamma^i\Gamma^j=\Tr\Gamma^i\Gamma^j\Gamma^k
=\Tr\Gamma^1\Gamma^2\Gamma^3\Gamma^4=0$ if $i$, $j$,$k$
are distinct.)
For the moment we leave (9) unevaluated and introduce
Feynman parameters into (8):
$$
\align
\tfrac{1}{(q-k)^2+m^2}\tfrac{1}{q^2+m^2} &=\int_0^1
d\alpha\tfrac{1}{[\alpha(q-k)^2+(1-\alpha)q^2+m^2]^2}\\
&=\int_0^1d\alpha
  \tfrac{1}{[(q-\alpha k)^2+m^2+\alpha(1-\alpha)k^2]^2}
\endalign
$$
Let
$$
M^2=m^2+\alpha(1-\alpha)k^2\tag10
$$
and make the shift of integration $q\to q+\alpha k$ in
(8).
Then using (9) we obtain
$$
\split
-\Sigma_{ij}(k)= -\lam^2 &\mu^{2\eps}\int_0^1dx
\int\tfrac{d^dq}{(2\pi)^d}\\
&\tfrac{\Tr\left[(q_\mu+(\alpha-1)k_\mu)(q_\nu+\alpha
k_\nu)
\Gamma^i\Gamma^\mu
\Gamma^j\Gamma^\nu-m^2\Gamma^i\Gamma^j\right]}
{(q^2+M^2)^2}
\endsplit
\tag11
$$
Now it is easy to compute that for any (co)vectors $v$ and
$w$,
$$
v_\mu w_\nu\Tr(\Gamma^i\Gamma^\mu\Gamma^j\Gamma^\nu)=
4(v_iw_j+v_jw_i-\delta_{ij}v\cdot w),
$$
and
$$
\Tr(\Gamma^i\Gamma^j)=4\delta_{ij}.
$$
Observe that $\int d^dqF(q)=0$ for an odd function of $q$,
simply by changing variables $q\to -q$.
Thus the $\int d^dq$ in (11) reduces to
$$
\aligned
4\int\tfrac{d^dq}{(2\pi)^d} &\tfrac{(2q_iq_j-\delta_{ij}q^2)+
\alpha(\alpha-1)(2 k_i
k_j-\delta_{ij}k^2)-\delta_{ij}m^2}
{(q^2+M^2)^2}\\
&=-8\alpha(1-\alpha)(k_ik_j-\delta_{ij}k^2)\int
  \tfrac{d^dq}{(2\pi)^d}\tfrac{1}{(q^2+M^2)^2}\\
&\qquad\qquad +4\int\tfrac{d^dq}{(2\pi)^d}\left\{
\tfrac{2q_iq_j}{(q^2+M^2)^2}-\tfrac{\delta_{ij}}{q^2+M^2}
\right\}
\endaligned
\tag12
$$
The last integral vanishes for $i\not=j$ (take $q_j\to
-q_j$) and is independent of $i$ for $i=j$, so reduces to
$$
4\delta_{ij}\int\tfrac{d^dq}{(2\pi)^d}\left\{\tfrac{\frac2d
q^2}{(q^2+M^2)^2}-\tfrac{1}{q^2+M^2}\right\}.\tag13
$$
We claim that under dimensional regularization (13)
vanishes.
To see this consider
$$
I(\lam)=\int\tfrac{d^dq}{(2\pi)^d}\tfrac{1}{(\lam
q^2+M^2)}.
$$
Then by scaling $I(\lam)=\lam^{-d/2}I(1)$, and differentiating
in $\lam$ and $\lam=1$ we find $I'(1)=-d/2I(1)$.
This implies (13) vanishes.
Now we evaluate the remaining integral in (12) in $d$
dimensions as
$$
\int\tfrac{d^dq}{(2\pi)^d}\tfrac{1}{(q^2+M^2)^2}=
\tfrac{\Gamma(\epsilon)}{(4\pi)^{d/2}}
\tfrac{1}{(M^2)^\epsilon}.
$$
(See problem set 4, problem 2 for the derivation of a
similar formula.)

Finally, plugging back into (11) we have
$$
-\Sigma_{ij}(k)=(k_ik_j-\delta_{ij}k^2)F(k^2),\tag14
$$
where
$$
F(k^2)=\tfrac{\lam^2}{2\pi^2}
\Gamma(\epsilon)(4\pi\mu^2)^\epsilon
\int_0^1
d\alpha\,\alpha(1-\alpha)
(m^2+\alpha(1-\alpha)k^2)^{-\epsilon}.
$$
This is finite for $\epsilon>0$ (which is $d<4$) and has
a simple pole at $\epsilon=0$.
Using the expansion
$$
\Gamma(\epsilon)=\tfrac1\epsilon -\gamma+O(\epsilon),
\qquad \gamma=\text{ Euler's constant},
$$
we have
$$
F(k^2)=\tfrac{\lam^2}{2\pi^2}\left\{\tfrac{1}{6\epsilon}-
\tfrac\gamma6-\int_0^1d\alpha\,\alpha(1-\alpha)
\ln\left(\tfrac{m^2+\alpha(1-\alpha)k^2}{4\pi\mu^2}\right)
  +O(\epsilon)\right\}.\tag15
$$
The $\int_0^1d\alpha$ can be evaluated in terms of
elementary functions, but the result is complicated and
unilluminating.

To renormalize we must adjust the fields and parameters
in the lagrangian (3), which we write now including the
gauge fixing term:
$$
\scrL=\left\{\tfrac12 \Sigma_{i,j}\vert\partial_i
  B_j\vert^2+\tfrac{\alpha-1}{2}\left\vert\Sigma_{i}
 \partial_i B_i\right\vert^2+(i(\Gamma\cdot\partial+m)
  \psi,\psi)+e(\Gamma\cdot B\psi,\psi)\right\}dx^4.
\tag16
$$
The inverse propagator, written in terms of the Fourier
variable $k$, is
$$
D_{ij}=k^2\delta_{ij}+(\alpha-1)k_ik_j.
$$
The renormalization scheme defines the bare field $B_0$
and bare coupling $\alpha_0$ as
$$
\aligned
B_0 &=(1+P)^{1/2}B\\
\alpha_0 &=\tfrac{\alpha}{1+P}
\endaligned
\tag17
$$
so that the bare inverse propagator is
$$
\align
D_{ij}^0 &=(1+P)k^2\delta_{ij}+(\alpha-1-P)k_ik_j\\
&=D_{ij}-\Sigma_{ij}
\endalign
$$
Comparing to (14) we have for $\alpha=1$ to $1$-loop
accuracy, using (15),
$$
P=-\tfrac{\lam^2}{12\pi^2}\tfrac1\epsilon+\text{ finite}.
$$
Finally, we renormalize the charge $e=\lam\mu^\epsilon$ in
(16) by setting
$$
\aligned
e_0 &=\lam\mu^\epsilon\cdot(1+P)^{-1/2}\\
&\approx \lam\mu^\epsilon\left(1+\tfrac{\lam^2}{24\pi^2}
  \tfrac1\epsilon+\text{ finite}\right).
\endaligned
\tag18
$$
The $\beta$-function is
$$
\beta
=\mu\tfrac{\partial\lam}{\partial\mu}\vrulesub{\epsilon=0}
$$
with $e_0$ fixed.
Using minimal subtraction we set the finite part to zero
and\break
differentiate (18):
$$
0=\epsilon\left(\lam+\tfrac{\lam^3}{24\pi^2}
\tfrac1\epsilon\right)+\beta\left(1+\tfrac{\lam^2}{8\pi^2}
\tfrac1\epsilon\right)\tag19
$$
So to leading order in $\epsilon$ we have
$\beta=-\epsilon\lam$, and then taking the regular part
of (19) at $\epsilon=0$ we find
$$
\mu\tfrac{d\lam}{d\mu}=\beta=\tfrac{\lam^3}{12\pi^2}.
\tag20
$$
This ODE has solution
$$
\tfrac{1}{\lam^2(\mu)}-\tfrac{1}{\lam^2(\mu_0)}=-
\tfrac{1}{6\pi^2}\ln(\mu/\mu_0)
$$
for some constant $\mu_0$.
Equivalently,
$$
\lam^2(\mu)
=\tfrac{\lam^2(\mu_0)}{1-\frac{\lam^2(\mu_0)}{6\pi^2}
\ln(\mu/\mu_0)}
$$
which exhibits a (Landau) pole at
$$
\mu=\mu_0\,\exp\left(\tfrac{6\pi^2}{\lam^2(\mu_0)}\right).
$$

Finally, we given an {\it a priori} argument that the
coefficient of the divergence in $F(k^2)$ is positive
(see (15)).
The same argument in a simpler situation appears in
Problem 1 of Problem Set 7.
Here we identify the one particle irreducible diagram we
computed as minus the two point function of the {\it current}
$J_i$ in the free theory:
$$
-\mathop{\mathop{\Sigma}\limits^{\sssize\vee}}
\nolimits_{ij}(x) =-\lam^2
\left<J_i(x)\overline{J_j(0)}\,\right>_{\free}.
$$
Here $\mathop{\mathop{\Sigma}\limits^{\sssize\vee}}\nolimits_{ij}$
is the inverse Fourier transform of $\Sigma_{ij}$, and the
current is
$$
J_i(x)=\Gamma^i\psi_{(x)}\overline{\psi_{(x)}},
$$
where the product of the spinors is taken to be a vector.
The minus sign is due to the fact that
conjugation reverses the sign of $J_i$, since
we interchange $\psi$ and $\psibar$. 
Now reflection positivity implies that
$\left<J_i(x)\overline{J_j(0)}\,\right>_{\free}$
is a nonnegative quadratic form for $x\not=0$.
The behavior of the Fourier transform
$-\Sigma_{ij}(k)$ as $k\to\infty$ is determined by the
behavior of $-\mathop{\mathop{\Sigma}\limits^{\sssize\vee}}
\nolimits_{ij}(x)$ as $x\to 0$, and it follows easily
that $-\Sigma_{ij}(k)$ is nonpositive for large $k$.
Using (14) this translates into $F(k^2)>0$ for large
$k$.
This is consistent with (15).



\enddocument




