%%%%%%%%%%%%%%%%%%%%%%%%%%%%%%%%%%%%%%%%%%%%%%%%%%%%%%%%%%%%%%%%%%%%%%%%%%%
%%%%
%%%%    STRING THEORY : Problem Set 8, March 20, 1997
%%%%
%%%%%%%%%%%%%%%%%%%%%%%%%%%%%%%%%%%%%%%%%%%%%%%%%%%%%%%%%%%%%%%%%%%%%%%%%%%


\magnification=\magstep1
\overfullrule=0pt
\baselineskip=17pt
\def\det{{\rm det}}
\def\Det{{\rm Det}}
\def\tr{{\rm tr}}
\def\Tr{{\rm Tr}}
\def\12{{1 \over 2}}
\def\ker{{\rm Ker}}
\def\O{{\cal O}}
\def\G{{\cal G}}

\centerline{{\bf STRING THEORY}}
\centerline{ Problem Set \# 8}
\centerline{March 20, 1997}

\bigskip
\bigskip

Let $G$ be a semi-simple compact Lie group, with Lie algebra $\G$,
and structure constants $f_{abc}$, $a,b,c = 1,\cdots , \dim G$.
Let $\Sigma$ be a surface with the topology of the plane, with
a flat metric. The Wess-Zumino-Witten model is defined in terms 
of maps $g :~ \Sigma \to G$ and classical action 
$$
S[g] = { k \over 16 \pi } \int _\Sigma \tr \theta \wedge * \theta
+ { k \over 24 \pi} \int _B \tr \tilde \theta \wedge \tilde \theta
\wedge \tilde \theta .
$$
Here, $\theta$ denotes the $\G$-valued left-invariant one form 
$\theta = g^{-1} dg$, $B$ is a 3-dimensional manifold whose boundary
is $\Sigma$. Also,  we have
$\tilde \theta = \tilde g ^{-1}  d \tilde g$ where 
$\tilde g :~ B \to G$ is any differentiable extension of $g$ to 
$B$ such that $\tilde g |_\Sigma = g$.

\medskip

\noindent
{\it Classical Part}

\medskip

a) Show that the action $S[g]$ is of the same form as the generalized
non-linear sigma models studied in the lectures, with action (for 
definitions, see problem set 7)
$$
S_{G,B}[x] = {1 \over 8 \pi l^2} \int _\Sigma d \mu _\delta 
\bigl [ \delta ^{mn} G_{\mu \nu} + \epsilon ^{mn} B_{\mu \nu} \bigr ]
\partial _m x^\mu \partial _n x ^\nu 
$$

b) Show that the following multiplication formula holds
$$
S[gh] = S[g] + S[h] + {k \over 16 \pi} \int _\Sigma \tr
g^{-1} \bar \partial g h^{-1} \partial h
$$
and use it to show that $S$ is well-defined, 
up to shifts by integer multiples
of $2 \pi k$.

c) Using the formula of b), show that $S[g]$ is invariant under the 
infinite dimensional current (Kac-Moody) algebra $\hat \G \times \hat \G$ :
$$
 g(z,\bar z) \to \Omega (z) g(z,\bar z) \bar \Omega (\bar z) ^{-1}
$$
where $\Omega (z)$ and $\bar \Omega (\bar z)$ are holomorphic
and anti-holomorphic maps from $\Sigma$ into $G$ (actually into
the complexification of $G$) respectively.
Derive the $\G$-valued conserved current $J= J^a t^a$ resulting from
the transformations $\Omega$. ($t^a$ are generators of $\G$,
satsifying the structure relations $[t_a, t_b] =  f_{abc} t_c$.)
Derive the transformation law of $J$ under $\Omega$. 

d) Derive the stress tensor $T$, show that it is holomorphic
 and express $T$ as a quadratic form in terms of $J$.
 
\medskip

\noindent
{\it Quantum Part}

\medskip

We now quantize the theory defined by $S[g]$, and assume that the 
quantum current $J$ and stress tensor $T$ continue to be 
holomorphic.   
We also assume in part f) that the relation between the quantum current $J$
 and the quantum stress tensor $T$ is given by the relation derived 
 in d), except for an unknown multiplicative constant $\kappa$ :
 $$
 2 \kappa T(z) = : J^a(z) J^a(z):
 $$




e) Show that the Ward identities for Diff($\Sigma$) and $\hat \G$ yield the
following OPE's on $T$ and $J$
$$
\eqalign{
T(z) T(w) & \sim {c/2 \over (z-w)^4} + {2 \over (z-w)^2} T(w)
  + {1 \over z-w} \partial _w T(w) \cr
T(z) J^a(w) & \sim {1 \over (z-w)^2} J^a(w) 
  + {1 \over z-w} \partial _w J^a(w) \cr
J^a(z) J^b(w) & \sim {k \delta ^{ab} \over (z-w)^2} 
  + {f^{abc} \over z-w} J^c(w) \cr
  }
$$

f) By considering the current correlation functions 
$$
\langle J^{a_1} (z_1) \cdots J^{a_n} (z_n) \rangle
$$
and limits thereof when different $z_i$ approach one 
another, show that the OPE's for $J$ and $T$ are consistent
only if
$$
c= {k \dim G \over k + C_2(G)}
$$
where $C_2(G)$ is the quadratic Casimir value of the
adjoint representation of $\G$, defined by 
$f^{acd} f^{bcd} = C_2(G) \delta ^{ab}$.

g) Show that, in the large $k$ approximation, the metric
$G$ and the anti-symmetric tensor field $B$ of the
Wess-Zumino-Witten model, satisfy
the equations for conformal invariance and for the
central charge that were derived (in the
small $l^2$ limit) in the lectures.


\end
--4db4_4789-29be_2f3f-1684_556f--

