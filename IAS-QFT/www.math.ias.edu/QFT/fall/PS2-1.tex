%From: Pierre Deligne <deligne@math.ias.edu>
%Date: Wed, 23 Oct 1996 16:44:28 -0400
%Subject: Set Two, No. 1

\input amstex
\documentstyle{amsppt}
\magnification=1200
\pagewidth{6.5 true in}
\pageheight{8.9 true in}
\loadeusm

\catcode`\@=11
\def\logo@{}
\catcode`\@=13

\NoRunningHeads

\font\boldtitlefont=cmb10 scaled\magstep1

\def\dspace{\lineskip=2pt\baselineskip=18pt\lineskiplimit=0pt}
\def\wedgeop{\operatornamewithlimits{\wedge}\limits}
\def\w{{\mathchoice{\,{\scriptstyle\wedge}\,}
  {{\scriptstyle\wedge}}
  {{\scriptscriptstyle\wedge}}{{\scriptscriptstyle\wedge}}}}
\def\Le{{\mathchoice{\,{\scriptstyle\le}\,}
  {\,{\scriptstyle\le}\,}
  {\,{\scriptscriptstyle\le}\,}{\,{\scriptscriptstyle\le}\,}}}
\def\Ge{{\mathchoice{\,{\scriptstyle\ge}\,}
  {\,{\scriptstyle\ge}\,}
  {\,{\scriptscriptstyle\ge}\,}{\,{\scriptscriptstyle\ge}\,}}}
\def\vrulesub#1{\hbox{\,\vrule height7pt depth5pt\,}_{#1}}
\def\mapright#1{\smash{\mathop{\,\longrightarrow\,}%
     \limits^{#1}}}

\def\Adot{\Dot{A}}
\def\Bdot{\Dot{B}}
\def\phidot{\Dot{\phi}}

\def\dbR{{\Bbb R}}
\def\dbZ{{\Bbb Z}} 


\def\scr#1{{\fam\eusmfam\relax#1}}

\def\scrA{{\scr A}}   \def\scrB{{\scr B}}
\def\scrC{{\scr C}}   \def\scrD{{\scr D}}
\def\scrE{{\scr E}}   \def\scrF{{\scr F}}
\def\scrG{{\scr G}}   \def\scrH{{\scr H}}
\def\scrI{{\scr I}}   \def\scrJ{{\scr J}}
\def\scrK{{\scr K}}   \def\scrL{{\scr L}}
\def\scrM{{\scr M}}   \def\scrN{{\scr N}}
\def\scrO{{\scr O}}   \def\scrP{{\scr P}}
\def\scrQ{{\scr Q}}   \def\scrR{{\scr R}}
\def\scrS{{\scr S}}   \def\scrT{{\scr T}}
\def\scrU{{\scr U}}   \def\scrV{{\scr V}}
\def\scrW{{\scr W}}   \def\scrX{{\scr X}}
\def\scrY{{\scr Y}}   \def\scrZ{{\scr Z}}



\document
\line{{\boldtitlefont Witten's Problems}, Set Two --- 
N$^{\text{o}}$. 1
\hfill (solution written by P. Deligne)}
\smallskip
\hbox to \hsize{\hrulefill}

\bigskip
\dspace
We will write $\dbR^{d+1}$ for $\dbR^n$, coordinates
$t=x_0,x_1,\dotsc,x_d$.

\medskip\noindent
(a)\enspace
In the Minkowski signature, $*$ is defined as follows:

1.\enspace
Extend the given bilinear form on $V$ to $\wedgeop^{p}$
$V$, by
$$
(v_1\w\ldots\w v_p, w_1\w\ldots\w w_p)= \det(v_i,w_j).
$$
Same for $V^*$, identified with $V$.

\smallskip
2.\enspace
An orientation has to be chosen. 
Together with the metric, it defines a form $\omega$
of maximal degree: in standard coordinates, $dt\w dx_1\w
\ldots\w dx_d$.

\smallskip
3.\enspace
The operation $*$ is defined by $\alpha\w
*\beta=(\alpha,\beta).\omega$.

\bigskip
The lagrangian density $\scrL$ is
$$
-\tfrac{1}{4e^2}\,(F,F).\text{(standard density $dv$)}
$$
In this way of writing, no orientation is needed.
However, in the computations which follow, we choose an
orientation to identify densities and forms of maximal
degree.
The variation of the lagrangian density by $A\mapsto
A+\delta A$ is
$$
\align
&-\tfrac{1}{2e^2}\,(d\delta A, dA)dv =-\tfrac{1}{2e^2}\,
 d\delta A\w * dA\\
= &d\left(-\tfrac{1}{2e^2}\,\delta A\w *dA\right)-
  \tfrac{1}{2e^2}\,\delta A\w d* dA;
\endalign
$$
the variational derivative $\delta\scrL$ is $-1/2e^2$
$\delta A\w d * dA$, giving as Euler-Lagrange equation
for the extremals (a.k.a. equation of motion)
$$
d*dA=0,
$$
i.e., for $d^*$ the adjoint of $d$, $d^*dA=0$.

\bigskip\noindent
(b)\enspace
The general rule for defining $\omega_0$ is as follows:

1.\enspace
Pick a space like hypersurface $\Gamma$ (e.g. $t=0$).

\smallskip
2.\enspace
Let $a$ be the $d$-form whose exterior derivative is
used to justify the integration by part leading to
Euler-Lagrange equations.
Here,
$$
a=-\tfrac{1}{2e^2}\,\delta A\w *dA.
$$
it is a $1$-form on the space of fields, with values in
$d$-forms, and $\delta A\mapsto a(\delta A)$ is linear
over $C^\infty$-functions.

\smallskip
3.\enspace
Define a $1$-form $\alpha_\Gamma$ on the space of fields
by integrating $a$ on $\Gamma$.

\smallskip
4.\enspace
Restrict $\alpha_\Gamma$ to the space of extremals, and
define
$$
\omega_0=d\alpha_\Gamma\,\,.
$$

For this to make sense, one needs some control at
spatial infinity.
it is not unreasonable to consider only $A$ which vanish
at spatial infinity: we will see in (c) that if at $t=0$,
$A$ and $\Adot$ vanish for $\left(\sum x_i^2\right)^{1/2}>R$,
and if $A$ obeys the equations of motion $d*dA=0$, then
some gauge transform of $A$ vanishes for $\left(\sum
x_i^2\right)^{1/2}>R+\vert t\vert$.

On the space of extremals vanishing at spatial infinity,
the $1$-form $\alpha_\Gamma$ makes sense.
Its exterior derivative is
$$
\omega_0(\delta_1 A, \delta_2A)=\int_{\Gamma}
-\tfrac{1}{2e^2}\,(\delta_2 A\w *d\delta_1 A-
\delta_1 A\w *d\delta_2 A).
$$
The space of extremals is an affine space, and
$\omega_0$ is translation invariant.
The general lagrangian formalism ensures that this
$2$-form is independent of $\Gamma$.

\bigskip\noindent
(c)\enspace
Gauge transformations $A\mapsto A+d\phi$ preserve the
lagrangian density, hence the space of extremals.
Of course, if one imposes to $A$ to vanish at spatial
infinity, one has to impose to $\phi$ to be locally
constant at infinity (hence constant at spatial infinity
if $d+1\not=2$).
Suppose $d+1\not=2$: one can impose to $\phi$ to be zero
at spatial infinity, by subtracting a constant.
The formula for $\omega_0$ makes it clear that if
$A\mapsto A+\delta_1 A$ is an infinitesimal gauge
transformation with support disjoint from $\Gamma$, then
$\omega_0(\delta_1 A, \delta_2 A)=0$.
As $\Gamma$ can be chosen as it pleases, it follows that
$\omega_0$ comes from the quotient by gauge
transformations: decompose an infinitesimal gauge
transformation $\delta A_1$ in gauge transformations
with support in $t<1$, $-1<t<1$ and $-1<t$, and for each
evaluate $\omega_0(\delta_1 A, \delta_2 A)$ using a
$\Gamma$ for which it obviously vanishes.

\bigskip
Direct proof: if $B$ is a tangent vector to the space of
extremals: $d*d B=0$ and $B$ vanishing at spatial
infinity, then
$$
\int_{\Gamma} (d\phi)\w *dB+B\w *d(d\phi)=
\int_{\Gamma}d(\phi. *dB)=0\,\,.
$$

To investigate the quotient of the space of extremals by
the group of gauge transformation, we now fix a
decomposition of space-time as space $\times$ time.
Coordinates $x_0=t,\,x_i$.
Using $\phi=\int_0^t A_0(x,\tau)d\tau$ (vanishing at
spatial infinity if $A$ does), one gauge transforms $A$
into a new $A$ with $A_0=0$.
The equation $d*dA$ becomes
$$
\align
&\partial_0\left(\sum \partial_i A_i\right)=0\tag1\\
&\square\,A=-d\,\sum \partial_i A_i\tag2
\endalign
$$
if one gives $A$ at $t=0$, (1) gives
$f:=\sum\partial_i A_i$ at all times:
it is independent of $t$.
If $\Adot$ is given at $t=0$, one can then solve
$\square\,A=-df$ with the given Cauchy data $(A,\Adot)$
at $t=0$ (hyperbolicity of $\square$).
The support of the resulting $A$ is contained in what
can be reached by a time-like vector from the support of
$(A,\Adot)$ at $t=0$.

We have
$$
\align
&\sum \partial_i A_i=f\quad\text{at $t=0$, $f$
independent of $t$}\\
&\square\,A_i=-\partial_i f
\endalign
$$
It follows that
$$
\square\left(\sum \partial_i A_i-f\right)=\sum
\partial_i \square A_i-\square f=\sum
-\partial_i\partial_i f-\square f=Df-Df=0
$$
and if $\frac{d}{dt}\,\left(\sum \partial_i A_i-f\right)=0$ at
$t=0$, i.e. if (1) holds at $t=0$, then $\sum \partial_i
A_i-f=0$ always, i.e. (1) holds.

It follows that we have a bijection
$$
A\longmapsto (A,\Adot)\quad\text{at}\quad t=0
$$
from the space of extremals $A$ (vanishing at spatial
infinity), with $A_0=0$, to the space of Cauchy data
$(A,\Adot)$ at $t=0$, vanishing at infinity with
$A_0=0$, $\Adot_0=0$, $\sum\partial_i \Adot_i=0$.
The last condition is what keeps making sense, on the
Cauchy data, of the equation of motion $d*dA=0$.

Transported to the affine space of Cauchy data, the
$2$-form $\omega_0$ becomes the translation invariant
$2$-form
$$
(A,\Adot),(B,\Bdot)\longmapsto \int (A,\Bdot)-(\Adot,B)
\tag3
$$
(up to a constant).
It is degenerate, as we still have to divide by the time
independent gauge transformations $\phi$:
we have vanishing for $(A,\Adot)=(d\phi,0)$.
Conversely, if $(A,\Adot)$ is in the kernel, we have
$\int(\Adot,B)=0$ for any compactly supported
$B$, hence $\Adot=0$, and $\int(A,\Bdot)=0$ if $d^*\Bdot=0$.
One can take $\Bdot=d^*C$ and integration by part gives
$\int (dA,C)=0$, hence $dA=0$ and $A=d\phi$ for some
$\phi$.

This can be presented as follows:
one the space of all $(A,\Adot)$ at $t=0$ ($d^*\Adot=0$
not required), we have the symplectic structure (3).
We took the symplectic quotient $L^\bot/L$ by the
isotropic linear subspace $L$ of the $(d\phi,0)$.

Geometrically, the hyperbolicity modulo gauge of the
equation of motion $d*dA=0$ has the following meaning.

One thinks of the $1$-form $A$ as being a connection
$\nabla$ on the trivial $\dbR$-torsor ($=$ principal
$\dbR$-bundle) $T$.
For electromagnetism, it might be better to use rather
$\dbR/\dbZ=U^1$, but if one wants a change of
trivializations to be given by a function $\phi$ with
values in $\dbR$, not $\dbR/\dbZ$, one needs $\dbR$.
The gauge transform $A+d\phi$ of $A$ is the $1$-form
describing the same connection $\nabla$, in the
trivialization $\phi$ of the torsor $T$.

Call ``space'' the hypersurface $t=0$ and let $U$ be a
region of space. 
Suppose given a $1$-jet $P_1$ of $\dbR$-torsor along
$U$.
If $s_1$ is a trivialization of $P_1$, any other
trivialization can be written as $s_1+\phi_1$, for
$\phi_1$ a $1$-jet of function along
$U\colon\,\phi_1=(\phi,\phidot)$ on $U$.
Suppose a connection $\nabla_1$ is given on $P_1$.
Given a trivialization of $P_1$, it is the data of
$A_0$, $A_i$, 
$A_i^{\raise1pt\hbox{$\bold{\cdot}$}}$ ($i\not=0$) in $U$
(hence at $t=0$).
Suppose lastly that what makes sense of the equation of
motion holds.
In coordinates:
$$
\sum\limits_{i\not=0}\partial_i^2 A_0-\partial_i
\Adot_i=0
$$
This is the ``Cauchy data''.

Let $V$ be the region of space time which, in terms of
causality

\item{} for $t>0$: is influence only by $U$;

\item{} for $t<0$: influences only $U$.

\noindent
Otherwise said: $v$ is in $V$ if all time like
trajectories through $v$ hit $t=0$ in $U$.
For $U$ the ball $\left(\sum x_i^2\right)^{1/2}<R$, of space,
i.e. at $t=0$, $V$ is the region $\left(\sum
x_i^2\right)^{1/2}<R-\vert t\vert$.

Hyperbolicity gives that the Cauchy data
$(P_1,\nabla_1)$ extends uniquely to a torsor with
connection obeying the equation of motion on $V$:
an extension exists, any two extensions are isomorphic
by an isomorphism whose first jet along $U$ is the
identity, and such an isomorphism is unique.

\example{Example}
If $U$ is the region $\sum x_i^2>R$, then $V$ is the
region $\sum x_i^2>R+\vert t\vert$.
If $(P,\nabla)$ obeys the equation of motion and has a
$1$-jet along $U$ isomorphic ot the trivial flat torsor,
it follows that $(P,\nabla)$ is isomorphic to the
trivial flat torsor on $V$.

In the preceding stories, one can replace ``space''
$t=0$ by a space like hypersurface.
Proofs are similar.
One begins by mapping $V$ to $U$ by sending $v=(t,x)$ to
the point of $U$ with the same $x$-coordinates, and by
using the connection along the time-lines to write the
torsors with connection on $V$ as pull back, as torsors,
of a torsor on $U$ (``gauge transformation making
$A_0=0$'').

In this geometric language, the restriction to $A$ which
vanish at spatial infinity, means considering only
$(P,\nabla)$ such that $P$ admits a trivialization which
is horizontal at spatial infinity: flatness at spatial
infinity, and, for $d=2$, trivial monodromy at spatial
infinity and vanishing of the integral of the curvature
on a space like hypersurface, hence on all of them.
\endexample


\enddocument




