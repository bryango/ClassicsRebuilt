%Date: Tue, 17 Dec 1996 13:19:56 -0500
%From: Edward Witten <witten@IAS.EDU>

\input harvmac
\overfullrule=0pt
{\it Exam}

(1) Prove via path integrals the character-valued index theorem for the Dirac 
operator.
That is, let $M$ be a spin manifold and $g$ an isometry of $M$ with a lift to 
the
spin bundles.  Calculate
\eqn\hgg{\Tr\,g(-1)^F}
via path integrals.

That is, write down the path integral for sections of a suitable $M$ bundle
over a circle.  Taking the circumference of the circle to be small, compute
the integral explicitly and obtain the Atiyah-Bott formula for \hgg.

(2) Compute in $\phi^4$ theory in four dimensions the anomalous dimension of 
the
operator $\phi^2$, in one-loop approximation.

(3) Determine the asymptotic behavior of perturbation theory in soft or cutoff
scalar field theories, as follows.  (Fermions and gauge fields introduce 
different
behavior, not to mention other types of theories such as sigma models.)

(a) Consider the integral
\eqn\gg{Z(g)=\int_\infty^\infty d\phi\,\exp(-\half \phi^2-g\phi^4/4!).}
Expand in an asymptotic series in $g$
\eqn\ipp{Z(g)=\sum_ka_kg^k}
and determine the large $k$ behavior of $a_k$.   Hint: write an integral 
representation
for $a_k$ and find a method of computing the integral for large $k$, for 
instance,
Where is the maximum of the integrand?

(b) Interpret the answer to (a) as a statement about the number of graphs of
a particular kind, weighted according to their automorphism groups.

(c) Now consider the path integral
\eqn\ingo{Z(g)=\int D\phi\,e^{-L}}
in a scalar field theory in $n$ dimensions, with 
\eqn\pp{L=\int d^nx\left(\half |d\phi|^2+\half m^2\phi^2+{g\over 
4!}\phi^4.\right).}
Write down, again, an integral representation for the coefficients $a_k$
in the asymptotic expansion of $Z(g)$ and determine the large $k$ behavior,
at least at a physical level of rigor, by finding the maximum of the integrand.

You should get a non-linear differential equation, and you should be able to 
prove
that it has a spherically symmetric solution; it is believed that this solution
does determine the large $k$ behavior of the $a_k$ in a cutoff $\phi^4$ theory
or in dimensions below four.  In four dimensions, the  much more complicated
renormalization, which you've presumably ignored in the computation above,
changes things ($L$ must be redefined by incorporation of counterterms and the
expansion in powers of $g$ is not made as simply as you've presumably done 
above).

(4) Consider in two dimensions a theory of $n$ fermi fields $\psi^i$, 
$i=1,\dots, n$.  Each
of the     $\psi^i$ is the sum of a section of $S_+$ and a section of $S_-$, 
those
being the two spin bundles, which we recall are real if the two-dimensional 
world
has Lorentz signature. (But you will actually compute with Euclidean 
signature.)

The Lagrangian is
\eqn\nomo{L=n\int d^2x\left(\half\sum_i(\psi^i,D\psi^i)+\half 
m_0(\psi^i,\psi^i)
-{g_0\over 8}\left(\sum_i(\psi^i,\psi^i)\right)^2\right).}
Notice the crucial factor of $n$ in front of the Lagrangian.
Here $D$ is the Dirac operator.  In what follows, you can think of the
fermion propagator as being
\eqn\orpo{{1\over \Gamma\cdot p+im_0}.}  
Expressions such as $(\psi,\psi)$ (which physicists write as $\bar \psi \psi$)
are the only possible invariants of the given kind.

Consider the operator $\Sigma={1\over n}\sum_i(\psi^i,\psi^i).$

(a) Consider the problem of computing the one point function 
$\langle\Sigma\rangle$
(which is an abbreviation for the vacuum expectation value 
$\langle\Omega|\Sigma|
\Omega\rangle$
in perturbation theory.  In $k$-loop order, which Feynman diagrams survive
as $n\to\infty$?

(b) Write down an integral equation that formally sums all the Feynman diagram 
contributions
to $\langle \Sigma\rangle$ that survive for $n\to\infty$.

(c) The integral equation in (b) needs some regularization and renormalization.
Do this and obtain the renormalized equation obtained as the cutoff goes
to infinity.  

\def\Z{{\bf Z}}
(d) If one had set $m_0=0$ in \nomo, one would have an extra $\Z_2$ symmetry
that is broken by the mass.  What is this?          Is the operator $\Sigma$
even or odd under this symmetry?

The renormalization in (c) can be done so that one of the parameters, a 
renormalized
mass $m$, is odd under the $\Z_2$, while the other parameters are even.
Hopefully you've done so!

If $m$ is zero, how many solutions are there to the integral equation found in 
(c)?
Are they invariant under the $\Z_2$?

(5) QCD is an $SU(3)$ gauge theory with fermions (quarks) which describes the 
nuclear
forces.  For this problem you only need to know a few facts.
For instance, the theory contains a ``current''
operator, which is simply a Lorentz vector $J^i$  (that is a local operator
of spin one) that is conserved, $d_iJ^i=0$ (or $d*J=0$), has zero anomalous
dimension (we'll learn why later), and has the property that its two point 
function
needs an additive $c$-number renormalization.

The renormalized two point function of $J$ 
is therefore the limit as a cutoff $\Lambda$ goes to infinity
of
\eqn\jury{\int d^4x e^{ik\cdot x}\langle J^i(x)J^j(0)\rangle 
~~~ -~A(\Lambda/\mu,g) (k^ik^j-\delta^{ij}k\cdot k).}
Here $g$ is the (dimensionless)
coupling constant, $\Lambda$ the cutoff, $\mu$ a mass that
must be introduced in renormalization, and $A$ is a function of the
indicated arguments which must be ``subtracted'' to make the two point function
finite.  Also, the theory is asymptotically
free and for present purposes you can ignore quark bare masses and assume that
there are no parameters in the theory except $g, $ $\mu$, and $\Lambda$
(and of course $\Lambda$ is eventually eliminated by taking it to infinity).  
There is no multiplicative
renormalization of $J$ in \jury\ because the anomalous dimension vanishes.
Because $J$ is conserved, the expression in \jury\ is of the form
\eqn\ury{(k^ik^j-\delta^{ij}k^2)B(k^2,g,\mu)}
where $\Lambda$ is omitted as we want to consider the limit $\Lambda\to\infty$.

I suppose it also helps to know that $J$ is built in terms of quarks as
$J=\bar q\Gamma^i q$, so that the one loop approximation to $\langle J 
J\rangle$
is given by a one loop diagram with a fermion loop that we computed in 
homework.
(For those who are getting a paper and not internet copy of this, the loop
is drawn below at the end of problem 7.)  
But all you  really need to know is that in the free field
theory, that is if $g=0$, 
$A=a \ln(\Lambda/\mu)$, where $a$ is a constant, independent of $g$.

(a) Write down the  renormalization group equation obeyed by $B$.

(b) Use this equation, together with the fact that QCD is asymptotically
free, to determine the large $k^2$ behvavior of $B$.

This large $k^2$ behavior, after including some additional theoretical
wrinkles to put it in a form that can be compared with experiment,  
is one of the classic tests of QCD.

(6) Let ${\cal O}(x)$ be a local operator which annihilates the vacuum:
\eqn\hbb{{\cal O}(x)|\Omega\rangle=0}
Show at least at a physical level of rigor that ${\cal O}(x)=0$.

Now extend this slightly: suppose that for some distinct points $x$ and $y$
$\langle \Omega|{\cal O}^*(y) {\cal O}(x)|\Omega\rangle=0$.
(${\cal O}^*$ is the hermitian adjoint of ${\cal O}$.) Prove that ${\cal O}=0$.

(7) Prove that a $U(1)$ gauge theory in four dimensions (without magnetic 
monopoles) cannot
be conformally invariant unless it is free.

For present purposes, a $U(1)$ gauge theory is any theory in
four dimensions that has  a two-form valued local field $F$
which obeys the Bianchi identity
\eqn\polp{dF=0.}
Now in four-dimensions with Euclidean signature we can decompose $F=F_++F_-$
where $F_+$ and $F_-$ are selfdual and antiselfdual.

\def\R{{\bf R}}
\def\S{{\bf S}}
(a) Work in    four-dimensional (flat) Euclidean space $\R^4$
and show that conformal invariance implies vanishing of the two point function
$\langle F_+(x)R_-(y)\rangle$ for $x\not= y$.

(Conformal invariance means that the correlation functions can be extended
to be defined for points in the one-point compactification $\S^4$ of $\R^4$
and that when so extended the correlation functions are invariant under the
group of conformal motions of $\S^4$, which is $SO(5,1)$.  Use $SO(5,1)$ to
prove vanishing of $\langle  F_+(x)F_-(y)\rangle$.)


(b) Given this, use the Bianchi identity to prove that
$\langle d^*F(x) d^*F(y)\rangle=0$.

(c) To conclude the argument, use the result of  (6) to deduce that
$d^*F=0$, which (together with the Bianchi identity   $dF=0$) implies that
$F$ is a free field.
\end


