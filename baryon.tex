\section{D-Branes in AdS, Baryons and Instantons}
\label{baryons}

A conservative form of the AdS/CFT correspondence would be to say that
classical supergravity captures the large $N$ asymptotics of some
quantities in field theory which are algebraically protected against
dependence on the 't~Hooft coupling.  The stronger form which is usually
advocated, and which we believe is true, is that the field theory is
literally equivalent to the string theory, and the only issue is
understanding the mapping from one to the other.  To put this belief to the
test, it is natural to ask what in field theory corresponds to
non-perturbative objects, such as D-branes, in string theory.  The answer
was found in \cite{Witten:1998xy} for several types of wrapped branes (see
also \cite{Gross:1998gk} for an independent analysis of some cases), and
subsequent papers \cite{Brandhuber:1998xy,Imamura:1998hf,Imamura:1998gk,
Aharony:1998qu,Callan:1998iq,
Witten:1998cd,Morrison:1998cs,Alishahiha:1998ib,Gukov:1998kn,
Craps:1999nc,Callan:1999zf} have extended and elaborated on the story.  See
also \cite{Metsaev:1998hf,Claus:1999fh} for actions for D-branes in anti-de
Sitter space, and \cite{Bilal:1998ck,Pasti:1998tc} for other related
topics.  The connection between D-instantons and gauge theory instantons
has also been extensively studied, and we summarize the results at the end
of this section.

Let us start with wrapped branes which have no spatial extent in
$AdS_5$: they are particles propagating in this space.  The
field theory interpretation must be in terms of some vertex or
operator, as for any other particle in AdS (as described above in the
case of supergravity particles).  If
the compact manifold is $S^5$, then the only topologically stable
possibility is a wrapped 5-brane.  The key observation here is that
charge conservation requires that $N$ strings must run into or out of the
5-brane.  In the case of a D5-brane, these $N$ strings are fundamental
strings (one could also consider $SL(2,\IZ)$ images of this
configuration).  The argument is a slight variant of the ones used in the
discussion of anomalous brane creation
\cite{Hanany:1997ie,Bachas:1997ui,Danielsson:1997wq}.  There are $N$ units
of five-form ($F_5$) flux on the $S^5$, and the coupling ${1 \over
2\pi} a \wedge F_5$ in the D5-brane world-volume translates this flux
into $N$ units of charge under the $U(1)$ gauge field $a$ on the
D5-brane.  Since the D5-brane spatial world-volume is closed, the
total charge must be zero.  A string running out of the D5-brane
counts as $(-1)$ unit of $U(1)$ charge, hence the conclusion.
Reversing the orientation of the D5-brane changes the sign of the
charge induced by $F_5$, and correspondingly the $N$ strings should
run into the brane rather than out.

In the absence of other D-branes, the strings cannot end anywhere in
$AdS_5$, so they must run out to the boundary.  A string ending on the
boundary is interpreted (see section \ref{wilsonloops})
as an electric charge in the fundamental
representation of the $SU(N)$ gauge group: an external (non-dynamical)
quark. 
% From a
%D-brane perspective (that is, in the small 't~Hooft coupling regime),
%these strings are interpreted 
This interpretation comes from viewing the strings
as running from the D5-brane to a
D3-brane at infinity.  It was shown in
\cite{Bachas:1997ui} that such stretched strings have a unique ground
state which is fermionic, 
and the conclusion is that the D5-brane ``baryon'' is
precisely an antisymmetric combination of $N$ fermionic fundamental string
``quarks.''  The gauge theory interpretation is clear: because the gauge
group is $SU(N)$ rather than $U(N)$, there is a gauge-invariant
baryonic vertex for $N$
external fundamental quarks.  We will return to a discussion of baryonic
objects in section \ref{other_dynamical}.

To obtain other types of wrapped brane objects with no spatial extent in
$AdS_5$, we must turn to compact manifolds with more nontrivial homology
cycles.  Apart from the intrinsic interest of studying such objects and the
gauge theories in which they occur, the idea is to verify the claim that
every object we can exhibit in gauge theory has a stringy counterpart, and
vice versa.


%{\bf The next two paragraphs overlap a discussion in section 
%\ref{orientifolds}, one of the two discussions should be removed.}

Following \cite{Witten:1998xy} and the discussion in section
\ref{orientifolds}, we now examine wrapped branes in the $AdS_5
\times {\bf RP}^5$ geometry, which is the near-horizon geometry of
D3-branes placed on top of a $\IZ_2$ orientifold three-plane (the
$\IZ_2$ acts as $x_i \to -x_i$ for the six coordinates perpendicular to the
D3-branes).  
%The orientifold makes it possible to have unoriented strings:
%worldsheets can have topology ${\bf RP}^2$.  It is well-known
%\cite{Cicuta:1982fu} that $SO(N)$ and $Sp(N/2)$ gauge theories\footnote{Our
%convention here is that $Sp(N/2)$ has a Cartan subalgebra of 
%dimension $N/2$, so for instance $Sp(1) = SU(2)$.} have a genus 
%expansion at large $N$ that involves unorientable as well as 
%orientable Riemann surfaces, and so should correspond to theories 
%built out of closed, unoriented strings.  Indeed it was argued in 
%\cite{Kakushadze:1998tz,Fayyazuddin:1998fb} that $SO(N)$ or 
%$Sp(N/2)$ gauge symmetry arises in string theory backgrounds
%with orientifold planes.  In the $AdS_5 \times S^5$ geometry, the
%distinction between $SO(N)$ and $Sp(N/2)$ theories arises from the discrete
%torsion of the NS and RR $B$-fields.
%
$H_3({\bf RP}^5,\IZ) = \IZ_2$, and the generator of the homology
group is a projective space ${\bf RP}^3 \subset {\bf RP}^5$.  This seems to
offer the possibility of wrapping a D3-brane on a 3-cycle to get a particle
in $AdS_5$.  However, there is a caveat: as argued in \cite{Witten:1998xy}
the wrapping is permitted only if there is no discrete torsion for the NS
and RR $B$-fields.  In gauge theory terms, that amounts to saying that the
corresponding operator is permitted if and only if 
the gauge group is $SO(N)$ with $N$
even.  Direct calculation leads to a mass $m \simeq N/R$ for the wrapped brane,
so the corresponding gauge theory operator has dimension $N$ (at least to
leading order in large $N$).  A beautiful fact is that a candidate gauge
theory operator exists precisely when the gauge group is $SO(N)$ with $N$
even: it is the ``Pfaffian'' operator, 
  \eqn{PfaffianOp}{
   {1 \over (N/2)!} \epsilon^{a_1 a_2 \ldots a_N} \phi_{a_1 a_2} \ldots
    \phi_{a_{N-1} a_N} \ .
  }
 Here the fields $\phi_{ab}$ are the adjoint scalar bosons which are the
${\cal N}=4$ superpartners of the gauge bosons.  We have suppressed their
global flavor index. A similar wrapped 3-brane was discussed in section
\ref{conifolds}, where the 3-brane was wrapped around the 3-cycle of
$T^{11}$ (which is topologically $S^2\times S^3$).

It is also interesting to consider branes with spatial extent in $AdS_5$.
Strings in $AdS_5$ were discussed in section \ref{wilsonloops}.  A
three-brane in $AdS_5$ (by which we mean any wrapped brane with three
dimensions of spatial extent in $AdS_5$) aligned with one direction
perpendicular to the boundary must correspond to some sort of domain wall
in the field theory.  Some examples are obvious: in $AdS_5 \times S^5$, if
the three-brane is a D3-brane, then crossing the domain wall shifts the
5-form flux and changes the gauge group from $SU(N)$ to $SU(N+1)$ or
$SU(N-1)$.  A less obvious example was considered in \cite{Witten:1998xy}:
crossing a D5-brane or NS5-brane wrapped on some ${\bf RP}^2 \subset {\bf
RP}^5$ changes the discrete torsion of the RR or NS $B$-field, and so one
can switch between $SO(N)$ and $Sp(N/2)$ gauge groups.  D5-branes on
homology 2-cycles of the base of conifolds and orbifolds have also been
studied \cite{Gubser:1998fp,Gubser:1999ia,Gukov:1998kn,Dasgupta:1999wx},
and the conclusion is that they correspond to domain walls across which the
rank of some factor in the product gauge group is incremented.

Another brane wrapping possibility is branes with two dimensions of spatial
extent in $AdS_5$.  These become strings in the gauge theory when they are
oriented with one dimension along the radial direction.  In a particular
model (an $SU(N)^3$ gauge theory whose string theory image is $AdS_5 \times
S^5/\IZ_3$) the authors of \cite{Gukov:1998kn} elucidated their
meaning: they are strings which give rise to a monodromy for the
wave-functions of particles transported around them.  The monodromy belongs
to a discrete symmetry group of the gauge theory.  The familiar example of
such a phenomenon is the Aharonov-Bohm effect, where the electron's
wave-function picks up a $U(1)$ phase when it is transported around a tube
of magnetic flux.  The analysis of \cite{Gukov:1998kn} extends beyond their
specific model, and applies in particular to strings in $SO(N)$ gauge
theories, with $N$ even, obtained from wrapping a D3-brane on a generator
of $H_1({\bf RP}^5,\IZ)$, where the ${\bf RP}^5$ has no discrete
torsion.

Finally, we turn to one of the most familiar examples of a non-perturbative
object in gauge theory: the instanton.  The obvious candidate in string
theory to describe an instanton is the D-instanton, also known as the
D(-1)-brane.  The correspondence in this case has been treated extensively
in the literature
\cite{Banks:1998nr,Chu:1998in,Kogan:1998re,Bianchi:1998nk,Brodie:1998ke,
Balasubramanian:1998de,Dorey:1998xe,Dorey:1999pd}.  The
presentation in \cite{Dorey:1999pd} is particularly comprehensive, and the
reader who is interested in a more thorough review of the subject can find
it there.  Note that the analysis of instantons in large $N$ gauge theories
is problematic since their contribution is (at least naively) highly
suppressed; the $k$ instanton contribution comes with a factor of
$e^{-8\pi^2k/g_{YM}^2} = e^{-8\pi^2 kN/\lambda}$ which goes like $e^{-N}$
in the 't Hooft limit. Therefore, we can only discuss instanton
contributions to quantities that get no other contributions to any order in
the $1/N$ expansion.  Luckily, such quantities exist in the $\cn=4$ SYM
theory, like the one discussed below.

The Einstein metric on $AdS_5 \times S^5$ is unaffected by
the presence of a D-instanton.  The massless fields in five dimensions
which acquire VEV's in the presence of a D-instanton are the axion and the
dilaton: in a coordinate system for the Poincar\'e patch of $AdS_5$ where 
  \eqn{PoincarePatchMetric}{
   ds^2 = {R^2 \over z^2} \left( dx_\mu^2 + dz^2 \right) \ ,
  }
 we have
\cite{Chu:1998in,Kogan:1998re,Bianchi:1998nk,Balasubramanian:1998de}, 
asymptotically as $z \to 0$,
  \eqn{PhiChiVEV}{\eqalign{
   e^\phi &= g_s + {24 \pi \over N^2} {z^4 \tilde{z}^4 \over 
    \left[ \tilde{z}^2 + (x_\mu - \tilde{x}_\mu)^2 \right]^4}
    + \ldots \ ,  \cr  
   \chi &= \chi_\infty \pm (e^{-\phi} - 1/g_s) \ ,
  }}
for a D-instanton whose location in anti-de Sitter space is
$(\tilde{x}_\mu,\tilde{z})$. It can be shown using the general
prescription for computing correlation functions that this
corresponds in the gauge theory to a VEV
  \eqn{VEVF}{
   \langle \tr F^2(x) \rangle = 192 {\tilde{z}^4 \over 
    \left[ \tilde{z}^2 + (x_\mu - \tilde{x}_\mu)^2 \right]^4} \ ,
  }
 which is exactly right for the self-dual background which describes the
instanton in gauge theory.  The action of a D-instanton, $2\pi/g_s$, also
matches the action of the instanton, $8\pi^2/g_{YM}^2$, because of the
relation $g_{YM}^2 = 4\pi g_s$.\fixit{Conventions?  Also in \VEVF?}  The
result \VEVF\ is insensitive to whether the D-instanton is localized on the
$S^5$, since the field under consideration is an $SO(6)$ singlet.  It is a
satisfying verification of the interpretation of the variable $z$ as
inverse energy scale that the position $\tilde{z}$ of the D-instanton
translates into the size of the gauge theory instanton.  In other words, we
understand the $AdS_5$ factor (which appears in the moduli space of an
$SU(2)$ instanton) as merely specifying the position of the D-instanton in
the five-dimensional bulk theory.

In fact, at large $N$, a Yang-Mills instanton is parametrized not only by a
point in $AdS_5$, but also by a point in $S^5$.  The $S^5$ emerges from
keeping track of the fermionic instanton zero modes properly
\cite{Dorey:1999pd}.  The approach is to form a bilinear $\Lambda^{AB}$ in
the zero modes.  $\Lambda^{AB}$ is antisymmetric in the four-valued $SU(4)$
indices $A$ and $B$, and satisfies a hermiticity condition that makes it
transform in the real ${\bf 6}$ of $SO(6)$.  Dual variables $\chi_{AB}$ can
be introduced into the path integral which have the same antisymmetry and
hermiticity properties: the possible values of $\chi_{AB}$ correspond to
points in $\IR^6$.  When the fermions are integrated out, the resulting
determinant acts as a potential for the $\chi_{AB}$ fields, with a minimum
corresponding to an $S^5$ whose radius goes into the determination of
the overall
normalization of correlation functions.

Building on the work of \cite{Banks:1998nr} on $\alpha'$ corrections to the
four-point function of stress-tensors, the authors of \cite{Bianchi:1998nk}
have computed contributions to correlators coming from instanton sectors of
the gauge theory and successfully matched them with D-instanton
calculations in string theory.  It is not entirely clear why the agreement
is so good, since the gauge theory computations rely on small 't~Hooft
coupling (while the string theory computations are for fixed $g_{YM}^2$ in
the large $N$ limit)
and non-renormalization theorems are not known for the relevant
correlators.  The simplest example turns out to be the sixteen-point
function of superconformal currents $\hat{\Lambda}^A_\alpha =
\tr(\sigma^{\mu\nu}{}_\alpha{}^\beta F^-_{\mu\nu} \lambda_\beta{}^A)$,
where $F^-_{\mu\nu}$ is the self-dual part of the field-strength, $A$ is an
index in the fundamental of $SU(4)$, $\alpha$ and $\beta$ are Lorentz
spinor indices, and $\mu$ and $\nu$ are the usual Lorentz vector indices.
One needs sixteen insertions of $\hat\Lambda$ to obtain a non-zero result
from the sixteen Grassmannian integrations over the fermionic zero modes of
an instanton.  The gauge theory result for gauge group $SU(2)$ turns out to
be
  \eqn{SixteenAnswer}{\eqalign{
   \left\langle \prod_{p=1}^{16} g_{YM}^2 \hat{\Lambda}_{\alpha_p}^{A_p}(x_p)
     \right\rangle
   = & {2^{11} 3^{16} \over \pi^{10}} g_{YM}^8 
   e^{-{8\pi^2 \over g_{YM}^2} + i \theta_{YM}} 
   \int {d^4 \tilde{x} \, d\tilde{z} \over \tilde{z}^5} 
   \int d^8 \eta \, d^8 \bar\xi  \cr
   & \prod_{p=1}^{16} \left[ {\tilde{z}^4 \over 
    \left[ \tilde{z}^2 + (x_p - \tilde{x})^2 \right]^4}
    {1 \over \sqrt{\tilde{z}}} \left( \tilde{z} \eta_{\alpha_p}^{A_p} + 
    (x_p - \tilde{x})_\mu \sigma^\mu_{\alpha_p \dot\alpha_p} 
    \bar\xi^{\dot\alpha_p A_p} \right) \right] \ .
  }}
 The superconformal currents $\hat{\Lambda}^A_\alpha$ are dual to
spin~$1/2$ particles in the bulk: dilatinos in ten dimensions which we
denote $\Lambda$.  One of the superpartners of the well-known 
${\cal R}^4$ term in
the superstring action (see for example \cite{Green:1997tv}) is the
sixteen-fermion vertex \cite{Green:1998me}: in string frame,
  \eqn{SixteenLambda}{
   {\cal L} = {e^{-2\phi} \over \alpha'^4} {\cal R} + \ldots + 
    \left( {e^{-\phi/2} \over \alpha'} f_{16}(\tau,\bar\tau) \Lambda^{16} + 
    \hbox{c.c} \right) + \ldots \ ,
  }
 where $f_{16}(\tau,\bar\tau)$ is a modular form with weight $(12,-12)$,
and $\tau$ is the complex coupling of type~IIB theory: 
  \eqn{TauEquation}{
   \tau = \chi + i e^{-\phi} = 
    {\theta_{YM} \over 2\pi} + {4\pi i \over g_{YM}^2} \ .
  }
 There is a well-defined expansion of this modular form in powers of
$e^{2\pi i \tau}$, $e^{-2\pi i \bar\tau}$, and $g_{YM}^2$.  Picking out the
one-instanton contribution and applying the prescription for calculating
Green's functions laid out in section \ref{correlators},
one recovers the form~\SixteenAnswer\ up to an overall factor.  The overall
factor can only be tracked down by redoing the gauge theory calculation
with gauge group $SU(N)$, with proper attention paid to the saddle point
integration over fermionic zero modes, as alluded to in the previous
paragraph.

The computation of Green's functions such as \SixteenAnswer\ has been
extended in \cite{Dorey:1999pd} to the case of multiple instantons.  Here
one starts with a puzzle.  The D-instantons effectively form a bound state
because integrations over their relative positions converge.  Thus the
string theory result has the same form as \SixteenAnswer, with only a
single integration over a point $(\tilde{x},\tilde{z})$ in $AdS_5$.  In view 
of the emergence of an $S^5$ from the fermionic zero modes at large $N$, the 
expectation on the
gauge theory side is that the moduli space for $k$
instantons should be $k$ copies of $AdS_5 \times S^5$.  But through an
analysis of small fluctuations around saddle points of the path integral it
was shown that most of the moduli are lifted quantum mechanically, and what
is left is indeed a single copy of $AdS_5 \times S^5$ as the moduli space,
with a prefactor on the saddle point integration corresponding to the
partition function of the zero-dimensional $SU(k)$ gauge theory which lives
on $k$ coincident D-instantons.  It is assumed that $k \ll N$.  Although
the $k$ instantons ``clump'' in moduli space, their field configurations
involve $k$ commuting $SU(2)$ subgroups of the $SU(N)$ gauge group.  The
correlation functions computed in gauge theory have essentially the same
form as \SixteenAnswer.  In comparing with the string theory analysis, one
picks out the $k$-instanton contribution in the Taylor expansion of the
modular form in \SixteenLambda.  There is perfect agreement at large $N$
for every finite $k$, which presumably means that there is some unknown
non-renormalization theorem protecting these terms.

