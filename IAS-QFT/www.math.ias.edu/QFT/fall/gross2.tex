%Date: Sat, 18 Jan 1997 12:39:35 -0500 (EST)
%From: Pavel Etingof <etingof@abel.math.harvard.edu>

\input amstex
\documentstyle{amsppt}
\magnification 1200
\NoRunningHeads
\NoBlackBoxes
\document

\def\eps{\epsilon}
\def\tW{\tilde W}
\def\Aut{\text{Aut}}
\def\tr{{\text{tr}}}
\def\ell{{\text{ell}}}
\def\Ad{\text{Ad}}
\def\u{\bold u}
\def\m{\frak m}
\def\O{\Cal O}
\def\tA{\tilde A}
\def\qdet{\text{qdet}}
\def\k{\kappa}
\def\RR{\Bbb R}
\def\be{\bold e}
\def\bR{\overline{R}}
\def\tR{\tilde{\Cal R}}
\def\hY{\hat Y}
\def\tDY{\widetilde{DY}(\g)}
\def\R{\Bbb R}
\def\h1{\hat{\bold 1}}
\def\hV{\hat V}
\def\deg{\text{deg}}
\def\hz{\hat \z}
\def\hV{\hat V}
\def\Uz{U_h(\g_\z)}
\def\Uzi{U_h(\g_{\z,\infty})}
\def\Uhz{U_h(\g_{\hz_i})}
\def\Uhzi{U_h(\g_{\hz_i,\infty})}
\def\tUz{U_h(\tg_\z)}
\def\tUzi{U_h(\tg_{\z,\infty})}
\def\tUhz{U_h(\tg_{\hz_i})}
\def\tUhzi{U_h(\tg_{\hz_i,\infty})}
\def\hUz{U_h(\hg_\z)}
\def\hUzi{U_h(\hg_{\z,\infty})}
\def\Uoz{U_h(\g^0_\z)}
\def\Uozi{U_h(\g^0_{\z,\infty})}
\def\Uohz{U_h(\g^0_{\hz_i})}
\def\Uohzi{U_h(\g^0_{\hz_i,\infty})}
\def\tUoz{U_h(\tg^0_\z)}
\def\tUozi{U_h(\tg^0_{\z,\infty})}
\def\tUohz{U_h(\tg^0_{\hz_i})}
\def\tUohzi{U_h(\tg^0_{\hz_i,\infty})}
\def\hUoz{U_h(\hg^0_\z)}
\def\hUozi{U_h(\hg^0_{\z,\infty})}
\def\hg{\hat\g}
\def\tg{\tilde\g}
\def\Ind{\text{Ind}}
\def\pF{F^{\prime}}
\def\hR{\hat R}
\def\tF{\tilde F}
\def\tg{\tilde \g}
\def\tG{\tilde G}
\def\hF{\hat F}
\def\bg{\overline{\g}}
\def\bG{\overline{G}}
\def\Spec{\text{Spec}}
\def\tlo{\hat\otimes}
\def\hgr{\hat Gr}
\def\tio{\tilde\otimes}
\def\ho{\hat\otimes}
\def\ad{\text{ad}}
\def\Hom{\text{Hom}}
\def\hh{\hat\h}
\def\a{\frak a}
\def\t{\hat t}
\def\Ua{U_q(\tilde\g)}
\def\U2{{\Ua}_2}
\def\g{\frak g}
\def\n{\frak n}
\def\hh{\frak h}
\def\sltwo{\frak s\frak l _2 }
\def\Z{\Bbb Z}
\def\C{\Bbb C}
\def\d{\partial}
\def\i{\text{i}}
\def\ghat{\hat\frak g}
\def\gtwisted{\hat{\frak g}_{\gamma}}
\def\gtilde{\tilde{\frak g}_{\gamma}}
\def\Tr{\text{\rm Tr}}
\def\l{\lambda}
\def\I{I_{\l,\nu,-g}(V)}
\def\z{\bold z}
\def\Id{\text{Id}}
\def\<{\langle}
\def\>{\rangle}
\def\o{\otimes}
\def\e{\varepsilon}
\def\RE{\text{Re}}
\def\Ug{U_q({\frak g})}
\def\Id{\text{Id}}
\def\End{\text{End}}
\def\gg{\tilde\g}
\def\b{\frak b}
\def\S{\Cal S}
\def\L{\Lambda}

\topmatter
\title Lecture 2: Renormalization groups (continued)
\endtitle
\author {\rm {\bf David Gross} }\endauthor
\endtopmatter

\centerline{Notes by P.Etingof and D.Kazhdan}
\vskip .1in

{\bf 2.1. Finite renormalization.}

As we saw in the 
previous lecture and in lectures of Witten and Gawedzki,
renormalizable quantum field theories 
 always come in families (which may even be 
infinitely-parametric in 2 dimensions), and we need to specify 
certain quantities in order to pinpoint a specific theory among such a family.
The only ways to do this are
to measure these quantities or to set their values artificially. 
But when we know them, we can (in principle) compute all 
other quantities in the theory as functions of them. 

The procedure of picking a theory by setting its parameters to the
results of measurements (or artificially) 
is called finite renormalization, as opposed
to ``infinite'' renormalization which we discussed above. While infinite
renormalization deals with removing divergences (infinities), finite
renormalization adjusts the finite part of the 
coefficients in the Lagrangian (as functions of the cutoff $\Lambda$),
which infinite renormalization leaves ambiguous. 

Thus, a prescription of finite renormalization
is a way to define a system of coordinates (or a parametrization)
of the space of renormalizable theories of a given type. 

It is clear that the number of quantities to be specified (i.e.
the dimension of the space of renormalizable theories) equals to the number
of coefficients of the Lagrangian of nonnegative dimension. 
However, these coefficients themselves are, in general, 
not suitable as coordinates on the space of theories, 
as they depend on the cutoff $\Lambda$ and are infinite at $\Lambda\to\infty$.
Therefore, people use other systems of
coordinates. 

For example, for the space $Y$ of renormalizable theories 
of a scalar field, we could use the following 
parametrizations:

1. $T\to ({\Cal G}_2(0),\frac{\d {\Cal G}_2}{\d p^2}(0), 
{\Cal G}_4(0,0,0))$, where ${\Cal G}_i$ are connected
correlation functions. 

2. $T\to ({\Cal G}_2(\mu^2),\frac{\d {\Cal G}_2}{\d p^2}(\mu^2), 
{\Cal G}_4(p_1,p_2,p_3))$, where 
$\mu\in M$ is some scale, and 
$p_1,p_2,p_3,p_4=-p_1-p_2-p_3$ are vertices 
of a regular tetrahedron such that $p_i^2=\mu^2$. 

Notice that the the first recipe defines one fixed parametrization 
of the space of theories, while the second one is in fact a family 
of such parametrizations, depending on a scale $\mu\in M$. 
In the future, such families of parametrizations will be more important to
us than fixed parametrizations. 

{\bf Remark.} The coordinates in both of the above parametrizations 
are not measurable quantities, since 
momenta of all particles which can be generated in a lab 
satisfy the condition $p^2=-m^2$, where $m$ is the mass 
of the particle (people say that ``$p$ is on the mass shell''). 
In the case of $\phi^4$ theory, 
it is possible to make up a system of coordinates 
purely out of measurable quantities.   
(Of course, here we make the (invalid) assumption 
that the $\phi^4$ theory describes a real physical process.)
Let us explain how to do this. 

Suppose that we have an elementary particle $\phi$
which obeys a renormalizable $\phi^4$-theory. 
We have seen that renormalizable $\phi^4$-theories live in a 
3-parametric family $Y$. For $T\in Y$, let $\sigma_T$ 
be the residue of the 2-point function 
of $T$ (in momentum space) at its pole.
The dimensionless quantity $\sigma_T$ is called the scale of $\phi$
in $T$. 

Measurable quantities are masses of particles and scattering 
amplitudes, which are scale-of-$\phi$-independent objects. 
So let us be interested only in scale-of-$\phi$-independent 
characteristics of a given theory. 
In this case we can restrict our attention to the 2-parameter subfamily 
of theories $Y_1\subset Y$ in which 
the scale of $\phi$ equals 1. 
This causes no loss of generality, since any theory $T\in Y$ 
can be normalized to a theory $T_1\in Y_1$ by a transformation
$\phi\to \alpha \phi$, in such a way that all scale-of-$\phi$-independent 
quantities are unchanged. 

Now our space of theories is 2-dimensional,
 so we need to measure 2 things in order 
to specify a concrete theory. 
One of the possible prescriptions is the following.
The two measured things are

(i) The mass $m$ of the field $\phi$. Mathematically, 
it is determined from the condition that $-m^2$ is the pole 
of the 2-point function of the theory, regarded as a function
of the square of momentum $k^2$ (in the Euclidean picture). 

(ii) The effective coupling $g$ -- the scattering amplitude 
for two particles at the 
some point $(p_1,p_2,p_3,p_4)$ ``on shell'', i.e. 
with $p_i^2=-m^2$ and $\sum p_i=0$. 
One of the possible choices is to take 
$p_i$ to be vertices of a regular tetrahedron
lying on some 3-dimensional subspace of 
the dual spacetime $V^*$ (it is clear that all 
such configurations can be obtained from each other by rotation).
The effective coupling measures how strong the interaction is. 

However, as we will see in the future,
one often needs to consider parametrizations of the space of theories 
in which parameters are not necessarily measurable
or have any physical meaning. An example of such 
physically meaningless, but very useful parametrization  
is the dimensional regularization prescription, which is described in
the next section. 

{\bf 2.2.
The dimensional regularization prescription of finite renormalization.}

Now let us describe a prescription of finite 
renormalization which uses dimensional regularization. 

Recall from Lecture 1 that given a renormalizable $\phi^4$-Lagrangian,
we can compute its correlation functions using dimensional regularization
and minimal subtraction. At first sight it seems that the answer is completely
canonical and does not depend on anything. 
However, after careful consideration 
it becomes obvious that the answer depends on the choice 
of unit of length. This happens because the dimensions
of couplings in the Lagrangian, in general, 
depend on the dimension $D$ of the spacetime, so that they should be regarded 
not as functions but rather as sections of some line bundle
over the space of values of $D$. In order to work with functions,
we should trivialize this bundle. 

The correct way of computing correlation functions with dimensional 
regularization is, in general, the following. We should 
choose a scale of momentum $\mu$, and for any  
coupling (or mass) $\kappa$ in the Lagrangian find the dimensionless 
quantity $\kappa_d$ such that 
$\kappa=\kappa_d\mu^{\text{dim}(\kappa,d)}$, where
$\dim(\kappa,D)$ is the dimension of $\kappa$ 
when dimension of spacetime is $D$ (this is some 
function of the form $aD+b$), and $d$ is the physical dimension 
of the spacetime. Now we should
replace all masses and couplings 
$\kappa^i$ in the Lagrangian with the expressions 
$\kappa^i(D)=\kappa^i_d\mu^{\text{dim}(\kappa,D)}$, and  
perform dimensional regularization -- minimal subtraction 
with this new Lagrangian, as 
described in Section 1.6 of Lecture 1. 

{\bf Remark.} One should remember that the 
Feynman integrals used to define correlation functions 
will have d-dependent dimensions, so they have to be multiplied 
by appropriate powers of $\mu$ to make their dimensions constant, before 
the subtraction procedure is applied. 

It is clear that the answer 
(correlation functions) heavily depend on the choice of $\mu$. 
The reason is that some couplings $\k$ will nontrivially depend 
on $D$ (for fixed $\k_d$), and their first derivatives by $D$ 
(which contain $\ln \mu$) will influence the regular part of the 
correlation functions at $D=d$. 

For example, in the case of $\phi^4$-theory, 
we have a Lagrangian of the form 
$$
\Cal L=\int d^4x(\frac{a}{2}(\nabla\phi)^2+\frac{m^2}{2}\phi^2+
\frac{g}{4!}\phi^4),\tag 2.1
$$
and $m(D)=m_d\mu, a(D)=a_d,g(D)=g_d\mu^{4-D}$. 

Now, the dimensional regularization prescription
of finite renormalization is:  
for any $\mu\in M$, $T\to (m,a,g)$, where $(m,a,g)$  
are the parameters 
of the $\phi^4$-Lagrangian of the form 2.1
which gives the correlation functions of $T$ if 
renormalized with the help of dimensional regularization, using 
the scale $\mu$. 

The parameters $m,a,g$ have the meaning of 
``the effective mass at scale $\mu$'', ``the inverse of the effective  
scale of $\phi$ at $\mu$'', and ``the effective coupling 
at $\mu$'' (we will later explain why). 

{\bf 2.3. Scale-dependence of 
finite renormalization prescriptions.}

In physics we are interested in the behavior of a system
at different scales, and how it changes when going from scale to scale. 
Therefore, we are interested in scale-dependent parametrizations
of the space of theories, such that at each scale $\mu$ 
the corresponding parametrization represents in some way the behaviour
of the system at $\mu$. 
Among the prescriptions 1,2 
given in Section 2.1, only 2 satisfies this property,
while 1 is scale-independent. 

The dimensional regularization
prescription also satisfies this property.
That is, given a theory $T\in Y$, 
the parameters of the Lagrangian $\Cal L_\mu(T)$ which 
reproduces the correlation functions of $T$ if renormalized 
by dimensional regularization using $\mu$
represent the behavior of the theory 
at scale $\mu$. This is not immediately obvious, since the dimesional 
regularization procedure is formal and has no physical meaning.

The reason is that for large $\mu$ the mass
can be neglected, and thus the function $\Gamma_4$
is a function of $p_i/\mu$,
which does not explicitly depend on $\mu$. 
Therefore, the coupling $g(\mu)$, which is formally obtained from 
the dimensional regularization procedure, characterizes the magnitude
of $\Gamma_4$ at $p_i\sim\mu$. 

Let us illustrate this
by considering an example. 

Given a $\phi^4$-Lagrangian $\Cal L$ of the form (2.1),  
its 1-particle irreducible 4-point function, computed by dimensional 
regularization using the scale $\mu$, in the 1-loop approximation
equals  
$$
\Gamma_4(p_i,\mu,m,a,g)=-g+\frac{g^2}{a^2}
[f((p_1+p_2)^2)+f((p_1+p_3)^2)+f((p_1+p_4)^2)],\tag 2.2
$$
where $f(k^2)$ is the amplitude of the corresponding 
1-loop diagram. This amplitude was computed in Gawedzki lecture 3
(page 12; the formula for $\hat I_{4,ren}(k)$), and the answer is
$$
f(k^2)=-\frac{A}{6}\int_0^1\ln(\alpha(1-\alpha)\frac{k^2}{\mu^2}+\frac{m^2}
{a\mu^2})d\alpha+const,\tag 2.3
$$
where $A$ is some positive numerical constant.

For large $\mu$ the term $m^2/a\mu^2$ can be neglected, and we get
$$
f(k^2)\sim -\frac{A}{6}\int_0^1\ln(\alpha(1-\alpha)\frac{k^2}{\mu^2})d\alpha+const=
-\frac{A}{6}\ln(k^2/\mu^2)+const,\tag 2.4
$$
In particular, if $p_1,p_2,p_3,p_4$ are vertices 
of a regular terrahedron such that $p_i^2=\mu^2$
(a point representing the scale $\mu$), then for large $\mu$ in
the 1-loop approximation we get
$$
\Gamma_4=-g+\frac{g^2}{a^2}C,
$$
where $C$ is some numerical constant independent of $\mu$. 
In particular, the scale-of-$\phi$-independent quantity
$\frac{\Gamma_4}{a^2}$ has the form
$$
\frac{\Gamma_4}{a^2}=-g_*+g_*^2C, g_*=g/a^2
$$
If $g_*$ is small, $\Gamma_4/a^2\sim -g_*$. So the ``effective coupling 
constant'' $g_*$ does indeed characterize the intensity 
of interaction at the scale $\mu$. 

{\bf 2.4. The renormalization group flow 
corresponding to a scale-dependent renormalization prescription.}

Given a scale-dependent finite
renormalization prescription, we can define the
renormalization group flow in the space of parameters,
which is given by transition functions from the coordinates
at one scale to coordinates at another scale. This is done as follows. 

Let $Y$ denote the space of renormalizable theories of a single 
scalar field $\phi$.
For any scale $\mu\in M$ we have an injective map
$\pi_\mu: Y\to \tilde P$, where $\tilde P$ is a fixed ($\mu$-independent) 
space of parameters. 
We will assume that the image $P_\mu=\text{Im}\pi_\mu$ is independent
of $\mu$. Denote this image by $P$. 

Now we can fix a theory $T$ and look at the dynamics
of $\pi_\mu(T)$ as $\mu$ changes.
To do this, for any $\mu_1,\mu_2\in M$ define 
a transformation $R_{\mu_1\mu_2}: P\to P$ by 
$R_{\mu_1\mu_2}=\pi_{\mu_2}\circ \pi_{\mu_1}^{-1}$. 
This allows us to define a flow on $P\times M$:
$t(z,\mu)=(R_{\mu,t\mu}z,t\mu)$. The flowlines of this flow 
 are of the form
$(\pi_\mu(T),\mu)$, where $T\in Y$ is a fixed theory.
This flow is called the renormalization group flow
associated to the corresponding family 
$\pi_\mu$ of renormalization prescriptions.

It is convenient to represent the renormalization group flow as a 
vector field. Let $\pi_\mu(T)=(m_\mu(T),a_\mu(T),g_\mu(T))$, where 
$m=m_\mu,a=a_\mu,g=g_\mu$ are the effective mass, inverse scale, and coupling 
at $\mu$ for some renormalization prescription. Then the vector 
field of renormalization group looks like
$$
W=\mu\frac{\d}{\d\mu}+\tilde\gamma m\frac{\d}{\d m}+\tilde\beta
\frac{\d }{\d g}
+\tilde\delta a\frac{\d}{\d a},\tag 2.5
$$
where $\tilde\beta,\tilde\gamma,\tilde\delta$ are functions on $P\times M$. 

It is clear that the renormalization group commutes with the group $G_1$ of
rescalings of $\phi$: $a\to ta, m\to t^{1/2}m,g\to t^2 g,\mu\to\mu$, 
as well as with 
the group $G_2$ of changing the units of measurement:
$a\to a,m\to tm,g\to g,\mu\to t\mu$. 
The invariants of these actions are $g_*=g/a^2$ and $m_*=m/\mu\sqrt{a}$. 
This shows that $\tilde\gamma=\hat\gamma(g_*,m_*)$, 
$\tilde\beta=\hat\beta(g_*,\mu_*)a^2$, and
$\tilde\delta=\delta(g_*,m_*)a$. 

It is convenient to rewrite $W$ in the coordinates $\mu,g_*,m_*,a$.
The result is 
$$
W=\mu\frac{\d}{\d\mu}+(\gamma-1) 
m_*\frac{\d}{\d m_*}+\beta
\frac{\d }{\d g_*}
+\delta a\frac{\d}{\d a},\tag 2.6
$$
where $\gamma:=\hat\gamma-\delta/2$, and
$\beta=\hat\beta-2g_*\delta$.

Using this formula, we can write down a differential equation 
for correlation functions.  
Let $\Gamma_j(p_i,\mu,m_*,a,g_*)$ be the $j$-point function
of the theory with parameters $m_*,a,g_*$ computed at scale $\mu$. 
Since the ``engineering'' (i.e. classical) dimension of $\Gamma_j$
is $4-j$, we have
$$
\biggl(\mu\frac{\d}{\d\mu}+s\frac{\d}{\d s}+j-4\biggr)
\Gamma_j(sp_i,\mu,m_*,a,g_*)=0.\tag 2.7
$$
Subtracting from (2.7) the equation $W\Gamma_j=0$, we get
$$
\biggl(s\frac{\d}{\d s}-(\gamma-1) 
m_*\frac{\d}{\d m_*}-\beta
\frac{\d }{\d g_*}
-\delta a\frac{\d}{\d a}+j-4\biggr)
\Gamma_j(sp_i,\mu,m_*,a,g_*)=0.\tag 2.8
$$
This equation allows one to study the dynamics of $\Gamma_j$ 
under the scaling of external momenta. 

Now consider the renormalization group flow in terms 
of the coordinates $g_*,m_*,a$. According to formula (2.6),
integral curves of this flow are given by 
the formulas $g_*=g_*(\mu),m_*=m_*(\mu), a=a(\mu)$, 
where
$$
g_*'=\beta(g_*,m_*), m_*'=(\gamma(g_*,m_*)-1)m_*, a'=\delta(g_*,m_*)a.\tag 2.9
$$
and prime denotes the derivative with respect to $\ln \mu$. 
{}From these formulas it is clear that $a$ expresses trivially 
through $g_*,\mu_*$:
$$
a(\mu)=a_0e^{\int_{\mu_0}^\mu\delta(g_*(s),m_*(s))d\ln s}.\tag 2.10
$$
However, equations for $g_*$ and $m_*$ cannot be easily solved,
even if $\beta$ and $\gamma$ are known, 
unless $\beta$ and $\gamma$ have some special form.

However, suppose that
the vector field $W$ has a third one-parameter symmetry
group: $\mu\to\mu,a\to a,m_*\to tm_*,g_*\to g_*$
(mass rescaling). In this case we will call our renormalization 
prescription mass independent.
Then $\beta,\gamma$, $\delta$ are independent on $m_*$, 
and equations
(2.9) are easily solved. Namely, $g_*(\mu)$ is 
represented implicitly by
$$
\frac{\mu}{\mu_0}=e^{\int_{g_*(\mu_0)}^{g_*(\mu)}\frac{dg}{\beta(g)}},\tag 2.11
$$  
and $m_*$ is expressed via $g_*$ by
$$
m_*(\mu)=\frac{\mu_0}{\mu}
m_*(\mu_0)e^{\int_{\mu_0}^\mu \gamma(g_*(s))d\ln s}.\tag 2.12
$$

Unfortunately, most renormalization prescriptions are mass-dependent. 
This is the case, for example, for prescription 2 from Section 2.1.
However, this prescription is ``asymptotically mass-independent'', 
i.e. almost mass independent at high momenta. 
Indeed, the correlation functions, which
are used as parameters in this prescription, 
do not significantly depend on $m$ at high momenta. Therefore, although
$\beta$ and $\gamma$ depend on $m_*$, there exist limits 
$\beta(g_*):=\lim_{m_*\to 0}\beta(g_*,m_*)$, 
$\gamma(g_*):=\lim_{m_*\to 0}\gamma(g_*,m_*)$, 
so that formulas (2.11), (2.12) become valid 
for very large $\mu_0,\mu$, and, in particular, are applicable
for studying the asymptotics of solutions at $\mu\to\infty$. 

This ``asymptotic mass independence'' is characteristic of all 
prescriptions $\pi_\mu$ which are based on correlation functions at
$\mu$. But all of them are mass dependent at finite $\mu$. 

Nevertheless, mass independent renormalization prescriptions exist. 
The most convenient of them is the dimensional regularization prescription. 

Let us explain why the dimensional regularization prescription 
is mass-independent. We will work in the spacetime dimension $D$
(a generic complex number), with $\eps=4-D$. 
For a renormalizable Lagrangian $\Cal L$ of the form (2.1) 
with parameters $m,a,g$,  let 
$\Gamma_i^\eps(p_j,m,a,g)$
be the correlation functions of $\Cal L$ in dimension $D$,
computed by plain integration in $D$ dimensions as described in Lecture 1, and
$(\Gamma_i^{\eps,sub}(p_j,\mu,m,a,g)$ be the correlation functions 
in dimension $D$, computed using scale $\mu$, with pole parts subtracted.
 The functions $\Gamma_i^\eps$ are singular at $\eps=0$, 
while $\Gamma_i^{\eps,sub}$ are regular at $\eps=0$, and the
functions $\Gamma_i:=\lim_{\eps\to 0}\Gamma_i^{\eps,sub}$  
are the true correlation functions of the theory, computed at scale $\mu$
as in Section 2.2. 
 
Let $m^b,a^b,g^b$ be the bare parameters, 
defined by the condition
$$
\Gamma_i^\eps(p_j,m^b,a^b,g^b)=\Gamma_i^{\eps,sub}(p_j,\mu,m,a,g)
$$
(b stands for ``bare''). These bare parameters are functions of
renormalized parameters, $\eps$, and $\mu$, of the form
$$
m^b=m^b(m,a,g,\eps,\mu),
a^b=a^b(m,a,g,\eps,\mu),
g^b=g^b(m,a,g,\eps,\mu).
$$
{}From dimensional analysis (using the groups $G_1$ and $G_2$)
it follows that the functions
$m^b$, $a^b$, $g^b$ are defined by the equations
$$
a^b=aZ_a(m_*,g_*,\eps), m^b(a^b)^{-1/2}=
\mu m_*Z_m(m_*,g_*,\eps), g^b(a^b)^{-2}=\mu^\eps
g_*Z_g(m_*,g_*,\eps), 
$$
where 
$$
Z_i(m_*,g_*,\eps)=1+\sum_{k\ge 1}Z_i^k(g_*,m_*)\eps^{-k}.
$$
(no positive powers of $\eps$, as we used minimal subtraction).
 
The key point now is that $Z_i$ do not actually depend on $m_*$.
Indeed, the functions $Z_i$ are sums of counterterms introduced
in the process of renormalization. These counterterms at each step
of the process
are just sums of pole parts (in $\eps$)
of amplitudes of all diagrams with a given number
of loops. These pole parts are determined purely by the 
asymptotic expansion of the amplitudes at $|p_i|\to \infty$.
{}From this one can deduce that these pole parts are independent on $m$. 
Thus, $Z_i=Z_i(g_*,\eps)$. 

Let $W_\eps$ be the vector field of the form
$$
W_\eps=\mu\frac{\d}{\d\mu}+(\gamma_\eps-1) 
m_*\frac{\d}{\d m_*}+\beta_\eps
\frac{\d }{\d g_*}
+\delta_\eps a\frac{\d}{\d a},
$$
which preserves functions $\Gamma_i^{\eps,sub}(p_j,\mu,m,a,g)$.
It is clear that such $W_\eps$ exists and is unique, and 
$\lim_{\e\to 0}W_\e=W$. On the other hand, using the definition of 
$\Gamma_i^{\eps,sub}$, we get that $W_\e$ is defined by the condition 
$W_\eps m_*^b=0,W_\eps a^b=0$, $W_\eps g_*^b=0$. Writing these equations in 
components, we get
$$
\beta_\eps Z_m'+\gamma_\eps Z_m=0,\
\beta_\eps Z_a'+\delta_\eps Z_a=0,\ \eps g_*+\beta_\eps\frac{(g_*Z_g)'}
{g_*Z_g}=0
$$
(prime denotes the derivative by $g_*$). From these equations we find:
$$
\beta_\eps=-\frac{\e g_*^2Z_g}{(g_*Z_g)'},\ \gamma_\eps=-\beta_\eps 
\frac{Z_m'}{Z_m}, \delta_\eps=-\beta_\eps\frac{Z_a'}{Z_a}.
$$
Using the facts that $Z_i$ has a Laurent expansion in inverse powers of 
$\eps$, and the fact that $\beta,\gamma,\delta$ are by definition analytic at 
$\e=0$, we get the following formulas:
$$
\beta_\eps(g_*)=-g_*\eps+g_*^2(Z_g^1)',  \
\gamma_\eps(g_*)=g_*(Z_m^1)', \delta_\eps(g_*)=g_*(Z_a^1)'
$$
In the limit $\e\to 0$, these formulas give
$$
\beta(g_*)=g_*^2(Z_g^1)',  \
\gamma(g_*)=g_*(Z_m^1)', \delta(g_*)=g_*(Z_a^1)'
$$
The last formula shows that $\beta,\gamma,\delta$ depend only on $g_*$, and
gives a simple and useful rule of evaluating them. 

In particular we see that although counterterms in dimensional regularization 
can have poles of arbitrary order, the contribution to the renormalization 
group vector field comes only from residues of these poles. Thus, in principle 
the renormalization group vector field should be more elementary 
than the complete system of counterterms. This is one of the reasons 
why it is useful to write down the renormalization group equation.

We have seen 
that the asymptotic properties of our field theory at $\mu\to\infty$ 
depend  on the properties of the function $\beta(g_*)$, since this function
determines the dynamics of the coupling $g_*(\mu)$. 
This function is called the beta-function. Later we will see 
that the sign of this function near $g_*=0$ is especially important. 

{\bf 2.5. Computation of the
renormalization group flow in the 1-loop approximation.}

In this section we will compute the
vector field $W$  in the 1-loop approximation. 
Let $m_*=m_*(\mu)$, $g_*=g_*(\mu)$, $a=a(\mu)$ be an integral curve 
of the flow. We will express the derivatives  
$m_*',g_*',a'$ from the equations $\mu\frac{\d}{\d\mu}
\Gamma_2(p^2,m_*,a,g_*,\mu)=0$,
$\mu\frac{\d}{\d\mu}
\Gamma_4(p_i,m_*,a,g_*,\mu)=0$.

First of all, the 1-loop correction to the 2-point function
can be absorbed in mass renormalization:
$m(\mu)=m_0(1+b\frac{g}{a^2}
\ln(\mu/\mu_0))+O(g^2)$, where $b$ is a numerical 
constant. Thus, $a'=0, m_*'=-m_*(1+bg_*)$.

In order to compute $g_*'$, we differentiate equation (2.2) with respect 
to $\ln\mu$. This yields, modulo $g_*^3$, the following expression:
$$
g_*'=Ag_*^2.\tag 2.13
$$

Thus, we get in the one-loop approximation: $\beta(g_*)=Ag_*^2, A>0;
\gamma(g_*)=bg_*; \delta(g_*)=0$. 

{\bf Remark.} A more careful calculation gives: $A=\frac{3}{16\pi^2},
b=\frac{1}{32\pi^2}$.

Now consider the renormalization group equation (2.13).
The general solution of this equation is
$$
g_*=\frac{g_*^0}{1-Ag_*^0\ln(\mu/\mu_0)}.\tag 2.14
$$  
Thus, we have found an approximation to the function $g_*(\mu)$ based
on the 1-loop approximation to the $\beta$-function. 

Now we face the following fundamental questions.
In what respect is expression (2.14), obtained from the 1-loop 
approximation to the beta-function, any better than the expression 
$$
g_*=g_*^0+A(g_*^0)^2\ln(\mu/\mu_0),\tag 2.15
$$
obtained directly from (2.2)? (Nothe that (2.15) coincides with (2.14) modulo 
$(g_*^0)^3$!) In what sense does (2.14) better represent 
``reality''than (2.15)? In other words, why did we care to introduce
renormalization group, beta-function, and so on, instead 
of just getting (2.15) directly?

The answer is the following. Assume that $\phi^4$ theory exists in 
the non-perturbative setting (it is not believed to be the case, but
never mind; you can think instead of another field theory).
Then formula (2.15) has a chance to be valid only when $g_*^0\ln(\mu/\mu_0)$
is small, since the series for $g_*$ will contain terms of
the form $(g_*^0)^{k+1}\ln(\mu/\mu_0)^k$. On the other hand, 
formula (2.13) is valid for small $g_*$, since it is 
the beginning of the Taylor expansion of $\beta$. So in the region where
$g_*$ is small but $g_*^0\ln(\mu/\mu_0)$ is not 
(for example, $Ag_*^0\ln(\mu/\mu_0)=1/2,g_*^0<<1$), formula
(2.14) is reliable but (2.15) is not. 

{\bf 2.6. Asymptotic freedom.}

So far we have always worked with quantum field theories in 
the perturbative setting, primarily because this was the only setting 
in which we could define and study them. However, physics lives over 
$\R$ rather than $\R[[\tau]]$, so we should investigate when we can use
perturbative results to judge about the actual process. One 
necessary condition is clear: all couplings should be small,
so that expansions in formal series with respect to them have a chance 
to make sense. 

In particular, we can ask whether we can trust the rules of perturbative
renormalization, which were discussed in the previous lecture and 
in Witten's lectures. As you remember, these rules tell how to handle 
divergences arising at high momenta (ultraviolet divergences).  
Thus, in order to trust the theory of perturbative renormalization 
we should be sure that interactions are small at high momenta. 

A (Euclidean) quantum field theory (in the non-perturbative sense; 
for example, in the sense of Wightman axioms) is called 
asymptotically free if its interactions 
are small at high momenta. That is, 
a theory is asymptotically free if its correlation functions 
in momentum space are asymptotically 
(at high momenta) close to those of a free theory. 
Thus, in order to use perturbative renormalization legitimately,
we should be sure that our theory is asymptotically free. 

Of course, in the perturbative  setting we cannot 
even show that a quantum field theory exists, much less that 
it is asymptotically free. It is therefore a pleasant surprise that
if we assume that a theory with a given perturbation expansion 
exists, its asymptotic freedom can be checked perturbatively, 
usually already in the 1-loop approximation.
Later we will discuss it in more detail, but now we will just 
demonstrate it for $\phi^4$-theory. 

We will only consider renormalizable theories, i.e. points of the space $Y$. 
It is clear that such a theory $T\in Y$ is free
(i.e. quadratic) if the effective coupling $g_*$ 
vanishes.
Thus, the condition for asymptotic freedom is that 
the coupling $g_*$ vanishes asymptotically, at high momenta. 
That means, $\lim_{\mu\to\infty}g_*(\mu)=0$.

It is clear that we cannot see whether $g_*(\mu)$ approaches zero 
or not by looking 
at $g_*(\mu)$ in the perturbation expansion. 
This is clear from the example of the function
$f(\mu)=\frac{1}{1+\tau\ln(\mu/\mu_0)}$, which vanishes at infinity for any 
finite $\tau$, but diverges at infinity if reduced modulo $\tau^N$ for 
each $N$. What we can see, however, is whether, as $\mu$ increases, 
the value of $g_*(\mu)$ increases or decreases.

Let us do this for $\phi^4$ theory. In this case, the physically meaningful 
region of values of $g_*$ is $g_*>0$. So we should find out whether 
$g_*'$ is positive or negative when $g_*$ is in this region. 

Now it is very easy for us to answer this question. Namely, 
if $g_*$ is small, we can trust formula (2.13): $g_*'=Ag_*^2$. 
This means that $g_*$ increases, i.e. gets away from zero. 
Of course, we cannot claim on the basis of (2.13) that $g_*$ increases
all the time (remember that (2.13) is only valid for small $g_*$), but what 
we can definitely guarantee on the basis of (2.13)
is that $\lim_{\mu\to\infty}g_*(\mu)\ne 0$. Thus 
{\bf the $\phi^4$ theory is not asymptotically free},
and its perturbation expansion is not valid at high momenta. 
This is a non-perturbative result which we obtained 
using only perturbation theory. 

Thus, renormalization group gives us a subtle way of partial 
summation of the perturbation series, 
allowing us to extend the region of validity 
of perturbation theory, and thus pushing us closer toward
the world of non-perturbative field theories. 
 
\end 
 


