\documentstyle{article}


%These are the macros which are in common with all of the
% sections in the paper mmr
% Each section, for now, should begin with \documentstyle[11pt,cd]{article}
% and then have \input{mmrmacros} followed by \begin{document}
% The only exception is that the \Label macro is slightly different
% in each file and should be put in separately.
%New CD macros
\newcommand{\cdrl}{\cd\rightleftarrows}
\newcommand{\cdlr}{\cd\leftrightarrows}
\newcommand{\cdr}{\cd\rightarrow}
\newcommand{\cdl}{\cd\leftarrow}
\newcommand{\cdu}{\cd\uparrow}
\newcommand{\cdd}{\cd\downarrow}
\newcommand{\cdud}{\cd\updownarrows}
\newcommand{\cddu}{\cd\downuparrows}
% (S) Proofs.
% (S-1) Head is automatically supplied by \proof.

\def\proof{\vspace{2ex}\noindent{\bf Proof.} }
\def\tproof#1{\vspace{2ex}\noindent{\bf Proof of Theorem #1.} }
\def\pproof#1{\vspace{2ex}\noindent{\bf Proof of Proposition #1.} }
\def\lproof#1{\vspace{2ex}\noindent{\bf Proof of Lemma #1.} }
\def\cproof#1{\vspace{2ex}\noindent{\bf Proof of Corollary #1.} }
\def\clproof#1{\vspace{2ex}\noindent{\bf Proof of Claim #1.} }
% End of Proof Symbol at the end of an equation must precede $$.

\def\endproof{\relax\ifmmode\expandafter\endproofmath\else
  \unskip\nobreak\hfil\penalty50\hskip.75em\hbox{}\nobreak\hfil\bull
  {\parfillskip=0pt \finalhyphendemerits=0 \bigbreak}\fi}
\def\endproofmath$${\eqno\bull$$\bigbreak}
\def\bull{\vbox{\hrule\hbox{\vrule\kern3pt\vbox{\kern6pt}\kern3pt\vrule}\hrule}
}
\addtolength{\textwidth}{1in}                  % Margin-setting commands
\addtolength{\oddsidemargin}{-.5in}
\addtolength{\evensidemargin}{.5in}
\addtolength{\textheight}{.5in}
\addtolength{\topmargin}{-.3in}
\addtolength{\marginparwidth}{-.32in}
\renewcommand{\baselinestretch}{1.6}
\def\hu#1#2#3{\hbox{$H^{#1}(#2;{\bf #3})$}}          % #1-Cohomology of #2
\def\hl#1#2#3{\hbox{$H_{#1}(#2;{\bf #3})$}}          % #1-Homology of #2
\def\md#1{\ifmmode{\cal M}_\delta(#1)\else  % moduli space, delta decay of #1
{${\cal M}_\delta(#1)$}\fi}
\def\mb#1{\ifmmode{\cal M}_\delta^0(#1)\else  %moduli space, based, delta
                                              %decay of #1
{${\cal M}_\delta^0(#1)$}\fi}
\def\mdc#1#2{\ifmmode{\cal M}_{\delta,#1}(#2)\else    %moduli space, delta
                                                      %decay, chern class #1
                                                      %of #2
{${\cal M}_{\delta,#1}(#2)$}\fi}
\def\mbc#1#2{\ifmmode{\cal M}_{\delta,#1}^0(#2)\else   %as before, based
{${\cal M}_{\delta,#1}^0(#2)$}\fi}
\def\mm{\ifmmode{\cal M}\else {${\cal M}$}\fi}
\def\ad{{\rm ad}}
\def\msigma{\ifmmode{\cal M}^\sigma\else {${\cal M}^\sigma$}\fi}
\def\cancel#1#2{\ooalign{$\hfil#1\mkern1mu/\hfil$\crcr$#1#2$}}
\def\dirac{\mathpalette\cancel\partial}
\newtheorem{thm}{Theorem}
\newtheorem{theorem}{Theorem}[subsection]
\newtheorem{proposition}[theorem]{Proposition}
\newtheorem{lemma}[theorem]{Lemma}
\newtheorem{claim}[theorem]{Claim}
\newtheorem{example}[theorem]{Example}
\newtheorem{corollary}[theorem]{Corollary}
\newtheorem{D}[theorem]{Definition}
\newenvironment{defn}{\begin{D} \rm }{\end{D}}
\newtheorem{addendum}[theorem]{Addendum}
\newtheorem{R}[theorem]{Remark}
\newenvironment{remark}{\begin{R}\rm }{\end{R}}
\newcommand{\note}[1]{\marginpar{\scriptsize #1 }}
\newenvironment{comments}{\smallskip\noindent{\bf Comments:}\begin{enumerate}}{
\end{enumerate}\smallskip}

\renewcommand{\thesection}{\Roman{section}}
\def\eqlabel#1{\addtocounter{theorem}{1}
\write1{\string\newlabel{#1}{{\thetheorem}{\thepage}}}
\leqno(\rm\thetheorem)}
\def\cS{{\cal S}}
\def\ov{\overline}








\newcommand{\operatorname}[1]{\mathop{\rm #1}\nolimits}
\newcommand{\Tr}{\operatorname{Tr}}



\title{Lecture II-14, part II: Quantum Cohomology of K\"ahler Manifolds}
\author{Edward Witten\thanks{Notes by John Morgan}}
\date{May 1, 1997}


\begin{document}
\maketitle

\section{Introduction}

We return now to twisted $N=2$ supersymmetric $\sigma$-models in
dimension two with  target a K\"ahler manifold $X$.
Recall that there were four possibilities for twisting depending on
which  supersymmetry $Q\in\{\overline Q_++\overline Q_-,Q_++\overline
Q_-,\overline Q_++Q_-, Q_++\overline Q_-\}$ is preserved.
Whichever possibility
one selects, the cohomology of $Q$ as a vector space can be identified
with the cohomology of $X$.  But there is more structure; if
we restrict to the case that the Riemann surface is a fixed $\S^2$,
then the additional structure is a ring structure on the cohomology
of $Q$. The resulting rings are called the $(a,a)$, $(c,a)$, $(a,c)$,
and $(c,c)$ chiral rings.

With one of the choices,
the classical version of the chiral ring turns out to be the
usual ring structure on cohomology, but quantum mechanically there are
corrections, instanton corrections, to this classical answer. For this
reason the chiral ring of the $\sigma$-model
is called the quantum cohomology of
the K\"ahler manifold $X$.
In this half of the lecture, we will discuss quantum cohomology
generally and then make some explicit computations in the case $X={\bf
C}P^{n-1}$ and the case $X$ is a Fano hypersurface.
These computations will be done both mathematically, counting
holomorphic rational curves with certain properties, and will be done
from the physics point of view by using the renormalization group flow
(discussed in Lecture
II-12 and the first half of Lecture II-14)
from these $\sigma$-models to Landau-Ginsburg theories
with extra vacua.



\section{The space of $0$-energy states}

We wish to compute some of these chiral rings, but before we deal
directly with the operators, we compute the $0$-energy states in
the Hamiltonian framework. The reason for doing this computation is
that as we showed in the last lecture there is a map
$$\psi\colon {\rm chiral\ ring}\to \{0{\rm -energy\ states}\}$$
given by inserting a local operator at a point on the hemisphere.
In favorable circumstances this  map is an isomorphism.


In the Hamiltonian framework the bosonic space underlying the space
of fields is
$${\cal W}={\rm Maps}(S^1,X).$$
The minimum energy configurations are the constant maps and hence the
space of these is a copy of $X$.
Since there are fermions in two copies of the
spin bundle, when we quantize  the resulting Hilbert space is the
tensor product of two copies of the spin bundle over $X$, i.e., the space of
differential forms on $X$. The Hamiltonian is the Laplacian and its
space of ground states is then the space
$L^2$-harmonic  forms on $X$ which is identified with $H^*_{L^2}(X)$.
Of course, there are normal directions to the copy of $X$ in ${\cal
W}$ to consider, but supersymmetry implies that the
quadratic form in the normal directions is non-degenerate, and hence
these directions produce a tensor products of harmonic oscillators and
have a unique ground state. Thus, in the end we find that the ground
states for the entire theory are identified with the harmonic forms on
$X$.
We shall see that the computation of the chiral rings gives the same
answer (additively), so that in the cases we consider today the map
$\psi$ is indeed an isomorphism.  This is true generally when the
classical theory has the property that the energy grows as the fields
go to infinity.


\section{Generalities on the chiral ring}

In order to define a chiral ring we need a global supersymmetry. As
we saw in Lecture II-13 the way to obtain these is to twist the usual
form of the $N=2$ supersymmetric theory.
Today we shall only consider  twists making $\overline Q_++Q_-$
global. This is what we referred to in  Lecture II-13 as Case A.  It
is a twist that can be performed for any K\"ahler manifold -- no
assumption of vanishing first Chern class
is necessary. We shall consider local operators
which are functions of the basic bosonic field $\phi\colon S^1\to X$
and fermions. Recall that  one of the effects of twisting is to make the
fermions sections of bundles of integral rather than half-integral
spin. That is to say, they are differential forms instead of spinors.
It will suffice in describing the
$Q$-cohomology of operators,  to consider only fermions of zero spin.
The reason for this is that all of the $Q$-cohomology classes of operators
have representatives that are constructed using only those fermions.
To be more precise, the $Q$-cohomology classes have representatives
that depend only on the bosonic fields and spin zero fermions and
not their derivatives.
\footnote{This is the operator analog of the fact that in the Hamiltonian
description, the zero energy states are constructed from differential
forms on the target space with all oscillators in their ground state.
Operators that contain derivatives of fields would correspond to states
in which some oscillators are excited.}
Thus, our local operators will be of the form
$${\cal O}={\cal O}(\phi,\psi_L,\psi_R)$$
where in the untwisted version of the theory, $\psi_L$ is a section of
$K^{-1/2}\otimes \phi^*T^{0,1}X$ and $\psi_R$ is a section of
$K^{1/2}\otimes T^{1,0}X$.
After we twist, these fermion fields lie in
$\phi^*T^{0,1}X$ and $\phi^*T^{1,0}X$, since the twisting cancels out
the line bundle over $\Sigma$.
In particular, it is possible to form the sum $\psi=\psi_L+\psi_R$ and
consider operators ${\cal O}={\cal O}(\phi,\psi)$.

\subsection{Local Operators}

For each differential form $\lambda$ on $X$ there is an operator ${\cal
O}_\lambda(x)={\cal O}_\lambda(\phi(x),\psi(x))$. As we remarked in
the Lecture II-13, under the identifications of operators and forms, $Q$
is identified with the usual exterior derivative:
$$\{Q,{\cal O}_\lambda\}={\cal O}_{d\lambda}.$$



When, as in the case of these $N=2$ supersymmetric $\sigma$-models,
the operator-to-state
correspondence is an isomorphism we can compute the ring structure of
the chiral ring by computing correlation coefficients.
Our goal here is to compute in more classical topological and
geometric terms the correlation coefficient
$$\langle{\cal O}_{\lambda_1}(x_1)\cdots{\cal
O}_{\lambda_s}(x_s)\rangle$$
for closed forms $\lambda_i$.
As we have already remarked, the correlation coefficient is unchanged
if we replace the $\lambda_i$ by cohomologous forms. Also, to compute
this correlation function,
which means to compute the path integral
$$\int{\cal D}\phi{\cal D}\psi e^{2\pi i{\cal L}}\prod_i{\cal
O}_{\lambda_i}(x_i)$$
over the space of all fields, it suffices to compute the integral over
the subspace of $Q$-invariant fields.
We have already identified
 the bosonic part of the subspace
of $Q$-fixed points: it is the space of
holomorphic maps $\Sigma\to X$.
For each $i$ choose a cycle $H_i$ Poincar\'e dual to $[\lambda_i]$.
Then we can choose a representative for $\lambda_i$ with support in an
arbitrarily small neighborhood of $H_i$.
To compute the correlation function one will consider the
moduli space of holomorphic maps from the Riemann surface passing
through all of the $H_i$ at the points $x_i$. In the `best' case  when
there are only finitely many holomorphic maps passing through the
cycles, one simply counts each of these maps with a sign.  The sign
comes from evaluating the bosonic and fermionic determinants in an
expansion around the classical solution.
Then the result is weighted by the exponential of $2\pi i$ times the
value of the action at this component.

Mathematically, there is a dimension count which must be satisfied for
the differential form being integrated to be top dimensional on the
moduli space of holomorphic curves, and hence for the answer to have a
chance of being non-zero. From the physics point of view one
sees the same dimension restriction  coming from the anomaly of the
$R$-symmetry. The measure of integration in the path integral has an
anomaly under this symmetry, and one
must use a set of operators whose product has precisely the cancelling
anomaly for the path integral to have a chance to be non-zero. Not
surprisingly, these two conditions are the same: namely, that the sum of
the codimensions of the $H_i$ in $X$ must be equal to the dimension of
the instanton moduli space of non-constant holomorphic maps from
$\Sigma$ to $X$.

If one wants to see just the classical cohomology ring, rather
than the quantum cohomology, then one should consider just the constant
maps to $X$.  Then the dimension condition is
$$\sum_{i=1}^s{\rm degree}\lambda_i={\rm dim}X$$
and the classical answer is simply
$$\int_X\lambda_1\wedge\cdots\wedge \lambda_s.$$
That is to say, the classical chiral ring is the cohomology ring of
$X$.
This answer gets corrected quantum mechanically by instanton
corrections  given by the integrals we were describing over the moduli
space of non-constant holomorphic maps. Thus, the chiral ring is a quantum
correction to the usual cohomology ring of $X$. For this reason the
chiral ring is usually  called the quantum cohomology ring of $X$.


\section{More Details on the Ring Structure}

Having given the general definition of quantum cohomology as the
chiral ring and indicated the sorts of mathematics that go into
computing it, it is now time to be more concrete and compute an
example. We take the case of $\Sigma=S^2$ mapping into a compact
K\"ahler manifold $X$.
Let us begin with a correlation function with only two operators:
$$\langle {\cal O}_{\lambda_1}{\cal O}_{\lambda_2}\rangle,$$
with the degree of $\lambda_1$ plus the degree of $\lambda_2$ equal to
the dimension of $X$.
We of course know that the space of constant maps  gives a
contribution to this correlation function equal to
$$\int_X\lambda_1\wedge\lambda_2.$$

In this case there is no quantum correction and this is also the answer
in the chiral ring. The reason that there is no quantum correction is
that the space of non-constant holomorphic maps of $S^2\to X$
passing through a point of $H_1$ and $H_2$ has a free ${\bf
C}^*$-action on it. For a component of this space to possibly give a
non-zero correction to the correlation coefficient, it must be the
case that its formal dimension is zero. But if it has zero dimension
and has a free ${\bf C}^*$-action on it, then it must be empty.
(Even for a possible nonzero component of this space, the free ${\bf C}^*$
action ensures that the contribution is zero.)
Thus, we have shown that under the identification of the chiral ring
with the cohomology of $X$,  the two point correlation functions, which
compute the  inner product on the vector space underlying the chiral
ring, give the usual inner product on cohomology or equivalently give
the intersection form on homology.

Now let us consider the ring structure. Fix a basis $\{\lambda_i\}$
for $H^*(X)$.
Let the inner product in this basis be given by $\eta_{ij}$:
$$\eta_{ij}=\langle{\cal O}_{\lambda_i}{\cal O}_{\lambda_j}\rangle.$$
Then the ring structure is determined by structure constants
$c_{ij}^k$ defined by
$${\cal O}_{\lambda_i}{\cal O}_{\lambda_j}=c_{ij}^k{\cal
O}_{\lambda_k}.$$
Let
$\omega_{ijk}$ be the three point function
$$\langle{\cal O}_{\lambda_i}{\cal O}_{\lambda_j}{\cal
O}_{\lambda_k}\rangle.$$
Since this three point function is also the inner product of the
product of the first two local operators with the third, we see
$$\omega_{ijk}=\sum_rc_{ij}^r\eta_{rk}.$$
Since $\eta_{ij}$ is non-degenerate, this implies that the two and
three point correlation functions determine the ring structure of the
chiral ring.


\section{Calculations for ${\bf C}P^{n-1}$}

We will now consider an example.
We do the computation for ${\bf C}P^{n-1}$ by using a gauged linear
$\sigma$-model with linear bosonic fields $\phi_1,\ldots,\phi_n,p$ as
in Lecture II-12.
To get all of ${\bf C}P^{n-1}$ we set the superpotential $W$ equal to
zero.
We get a family of theories depending on a complex parameter
$t=-ir+\frac\theta{2\pi}$, and
as we have seen, for ${\rm Im}\,t\ll 0$ these are non-linear
$\sigma$-models on $X$ with some K\"ahler metric depending on $t$ and
K\"ahler class roughly proportional to ${\rm Im}\,t$.
The only observable is ${\cal O}_H$ where $H$ is the hyperplane
section. We call this operator $\sigma$.
Classically, in ${\bf C}P^{n-1}$ we have one relation, namely
$\sigma^{n}=0$. Let us see what happens quantum mechanically.
We need to compute the three point function
$$\langle \sigma^a\sigma^b\sigma^c\rangle$$
where $a+b+c=n-1+dn={\rm dim}{\cal M}_d$,
the dimension of the moduli space of rational curves of degree $d$.
Clearly, it suffices to consider the case when $a,b,c<n$. This implies
that the only possibilities for $d$ are $d=0,1$. Of course, $d=0$
is the moduli space of constant maps and its contribution is the
classical ring structure. For $d=1$, we are considering
straight lines in ${\bf C}P^{n-1}$.
To calculate
$$\langle \sigma^{n-1}(x)\sigma^{n-1}(y)\sigma(z)\rangle$$
we must consider the space of straight lines through two points
meeting a fixed hyperplane. There is exactly one such line.
Since the instanton action is $e^{-2\pi it}$,
this lead to the relation
$$\langle \sigma^{n-1}(x)\sigma^{n-1}(y)\sigma(z)\rangle= e^{-2\pi
it}.$$
Unraveling the ring structure from this three point function yields
$$\sigma^n=e^{-2\pi it}$$
in quantum cohomology.

We can make the same calculation physically. We are considering the
$\sigma$-model ${\bf C}P^1\to {\bf C}P^{n-1}$. At the end of Lecture II-12 we
found that  in the infra-red this flows to $n$ vacua with a mass gap
(since this is a case in which $c_1(X)>0$, the flow is from the
$\sigma$-model to the Landau-Ginsburg model). The field $\sigma$ has
different expectation values in each vacuum.  Since the fields are
massive, we simply set them equal to their expectation values. These
vacuum expectation values
are $\sigma=\mu e^{2\pi i(-t+k)/n}, \ k=0,\ldots,n-1$.
These give the idempotents of the quantum cohomology ring, from which
we deduce the same answer as before,\footnote{Except for a factor of $\mu$
which has to do with a mismatch between our physical and mathematical
conventions.  In lecture 12, $\sigma$ was defined as a field of dimension
one and no effort was made to compare to any topological normalization.
In our mathematical discussion today, $\sigma$ was normalized topologically
as a field related to a hyperplane section of projective space, and
in particular has dimension zero.  To compare the two definitions,
one should, in the setting of lecture 12, integrate out the $\sigma$
field in the region $r\gg 0$ in favor of the massless fields of the sigma
model.  In this process, $\sigma$ will turn into an operator of the low
energy theory that can be derived from a two-form on ${\bf CP}^{n-1}$.
The cohomology class of this operator will be a multiple $C$ of
the hyperplane section, and $C$ is the factor by which the physical
normalization of lecture 12 disagrees with the topological normalization
that gives $\sigma^n=e^{-2\pi it}$.}
 namely that the ring is generated
by $\sigma$ modulo the relation $\sigma^n=e^{-2\pi it}$.

\section{Calculations for Fano Hypersurfaces}

Now let us generalize these computations to the case of a hypersurface
of degree $d<n$ in ${\bf C}P^{n-1}$, so-called Fano hypersurfaces.

According to [Collino-Jinzenji] the answer, computed mathematically, is
that the quantum cohomology ring is generated by $\sigma$ with one
relation:
\begin{equation}\label{quantumFano}
\sigma^{d-1}\left(\sigma^{n-d}-e^{-2\pi it}\right)=0.
\end{equation}

Let us think about how this computation fits with the fact that again
this $\sigma$-model flows in the infra-red to a Landau-Ginsburg model
plus extra massive vacua.
The roots of the equation $\sigma^{d-1}(\sigma^{n-d}-e^{-2\pi i t})=0$
are of two types: a root at $\sigma=0$ of multiplicity $d-1$,
and nondegenerate roots (that is, roots of multiplicity one) at
nonzero sigma.
The roots of the equation correspond to the vacua of the quantum field
theory.  Nondegenerate roots correspond to massive vacua, which give
idempotents of the quantum cohomology ring.
In analyzing the vacuum structure of this theory in lecture 12, we found
$n-d$ massive vacua at nonzero sigma.  These give the factor
$(\sigma^{n-d}-e^{-2\pi i t})$ in the equation for the quantum cohomology.
The other factor $\sigma^{d-1}$ is the contribution of the Landau-Ginzburg
vacua at the origin.  This factor is present if $d>1$ (for $d=1$
we are discussing a projective space of codimension one, and there is
no Landau-Ginzburg vacuum), and corresponds to a root at the origin
of multiplicity greater than one if $d>2$ (for $d=2$, the central
charge of the Landau-Ginzburg theory, which in general is $\widehat
c=n(1-2/d)$,
vanishes; in this case the Landau-Ginzburg theory is massive and infrared
trivial).  For general $d$, the multiplicity of the $\sigma=0$
root of the quantum cohomology can be computed by evaluating $\Tr(-1)^F$
in the Landau-Ginzburg theory at $\sigma=0$.





\end{document}
