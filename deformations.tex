\section{Deformations of the Conformal Field Theory}
\label{deformations}

In this section we discuss deformations of the conformal field theory, and
what they correspond to in its dual description involving string theory on
AdS space. We will focus on the case of the $\cn=4$ field theory, though
the general ideas hold also for all other examples of the AdS/CFT
correspondence. We start in section \ref{defintro} with a general
discussion of deformations in field theory and in the dual description.
Then in section~\ref{cTheorem} we use the AdS/CFT correspondence to prove a
restricted c-theorem.  In section~\ref{deffield} we discuss the interesting
relevant and marginal deformations of the $\cn=4$ SYM field theory; and in
section \ref{defstring} we review what is known about these deformations
from the point of view of type IIB string theory on $AdS_5\times S^5$. The
results we present will be based on \cite{aks_unpublished,
Girardello:1998pd,Distler:1998gb,Khavaev:1998fb,Karch:1999pv,
Freedman:1999gp}.





 \subsection{Deformations in the AdS/CFT Correspondence}
 \label{defintro}

 Conformal field theories have many applications in their own right,
 but since our main interest (at least in the context of four
 dimensional field theories) is in studying non-conformal field
 theories like QCD, it is interesting to ask how we can learn about
 non-conformal field theories from conformal field theories. One way to
 break conformal invariance, described in section \ref{FiniteT}, is to examine
 the theory at finite temperature. However, it is also possible to
 break conformal invariance while preserving Lorentz invariance, by
 deforming the action by local operators,
 \eqn{deform}{S \to S + h\int d^4x {\cal O}(x),}
 for some Lorentz scalar operator $\cal O$ and some coefficient
 $h$.

 The analysis of such a deformation depends on the scaling dimension
 $\Delta$ of the operator $\cal O$ \footnote{If the operator does not
 have a fixed scaling dimension we can write it as a sum of operators
 which are eigenfunctions of the scaling operator, and treat the
 deformation as a sum of the appropriate deformations.}. If $\Delta <
 4$, the effect of the deformation is strong in the IR and weak in the
 UV, and the deformation is called {\it relevant}. If $\Delta > 4$,
 the deformation is called {\it irrelevant}, and its effect becomes
 stronger as the energy increases. Since we generally describe field
 theories by starting with some UV fixed point and flowing to the IR,
 it does not really make sense to start with a CFT and perform an
 irrelevant deformation, since this would really require a new UV
 description of the theory. Thus, we will not discuss irrelevant
 deformations here. The last case is $\Delta=4$, which is called a
 {\it marginal deformation}, and which does not break conformal
 invariance to leading order in the deformation. Generally, even if
 the dimension of an operator equals 4 in some CFT, this will no
 longer be true after deforming by the operator, and conformal
 invariance will be broken. Such deformations can be either {\it
 marginally relevant} or {\it marginally irrelevant}, depending on the
 dimension of the operator $\cal O$ for finite small values of $h$. In
 special cases the dimension of the operator will remain $\Delta=4$
 for any value of $h$, and conformal invariance will be present for
 any value of $h$. In such a case the deformation is called {\it
 exactly marginal}, and the conformal field theories for all values of
 $h$ are called a {\it fixed line} (generalizing the concept of a
 conformal field theory as a fixed point of the renormalization group
 flow). When a deformation is relevant conformal invariance will be
 broken, and there are various possibilities for the IR behavior of
 the field theory. It can either flow to some new conformal field
 theory, which can be free or interacting, or it can flow to a trivial
 field theory (this happens when the theory confines and there are no
 degrees of freedom below some energy scale $\Lambda$). We will
 encounter examples of all of these possibilities in section
 \ref{deffield}.

 The analysis of deformations in the dual string theory on AdS space
 follows from our description of the matching of the partition
 functions in sections \ref{correspondence} and \ref{correlators}. 
 The field theory with the deformation
 \eno{deform} is described by examining string theory backgrounds in
 which the field $\phi$ on AdS space, which corresponds to the
 operator $\cal O$, behaves near the boundary of AdS space like
 $\phi(x,U) \stackrel{U\to \infty}{\longrightarrow} hU^{\Delta-4}$,
 where $[{\cal O}] = \Delta$ and we use the coordinate system 
 (\ref{metric5}) (with $U$ instead of $u$). In
 principle, we should sum over all backgrounds with this boundary
 condition. Note that, as mentioned in section \ref{correlators}, 
 in Minkowski space
 this involves turning on the non-normalizable solution to the field
 equations for $\phi(x,U)$; turning on the normalizable mode (as done
 for instance in
 \cite{Nojiri:1998yx,
 Kehagias:1999tr,Gubser:1999pk,Girardello:1999hj,Liu:1999fc,
 Kehagias:1999iy,Constable:1999ch}) 
 cannot be understood as a deformation of the field
 theory, but instead corresponds to a different state in the same
 field theory \cite{Balasubramanian:1999sn}\footnote{Some of the
 solutions considered in \cite{Girardello:1999hj} may correspond 
 to actual deformations of the field theory.}. As in the field theory,
 we see a big difference between the cases of $\Delta > 4$ and $\Delta
 < 4$. When $\Delta > 4$, the deformation grows as we approach the
 boundary, so the solution near the boundary will no longer look like
 AdS space; this is analogous to the fact that we need a new UV
 description of the field theory in this case. On the other hand, when
 $\Delta < 4$, the solution goes to zero at the boundary, so
 asymptotically the solution just goes over to the AdS solution, and
 the only changes will be in the interior. For $\Delta=4$ the solution
 naively goes to a constant at the boundary, but one needs to analyze
 the behavior of the string theory solutions beyond the leading order
 in the deformation to see if the exact solution actually grows as we
 approach the boundary (a marginally irrelevant deformation),
 decreases there (a marginally relevant deformation) or goes to a
 constant (an exactly marginal deformation).

 An exactly marginal deformation will correspond to a space of
 solutions of string theory, whose metric will always include an
 $AdS_5$ factor\footnote{The full space does not necessarily have to
 be a direct product $AdS_5\times X$, but could also be a fibration of
 $AdS_5$ over $X$, which also has the $SO(4,2)$ isometry group.}, but
 the other fields can vary as a function of the deformation
 parameters. A relevant (or marginally relevant) deformation will
 change the behavior in the interior, and the metric will no longer be
 that of AdS space. If we start in the regime of large $g_s N$ where
 there is a supergravity approximation to the space, the deformation
 may be describable in supergravity terms, or it may lead to large
 fields and curvatures in the interior which will cause the
 supergravity approximation to break down. The IR behavior of the
 corresponding field theory will be reflected in the behavior of the
 string theory solution for small values of $U$ (away from the
 boundary). If the solution asymptotes to an AdS solution also at
 small $U$, the field theory will flow in the IR to a non-trivial
 fixed point\footnote{Four dimensional field theories are believed
 \cite{Cardy:1988cw} to have a c-theorem analogous to the
 2-dimensional c-theorem \cite{Zamolodchikov:1986gt} which states that
 the central charge of the IR fixed point will be smaller than that of
 the UV fixed point. We will discuss some evidence for this in 
 the AdS context, based
 on the analysis of the low-energy gravity theory, in the next
 subsection.}.
 Note that the
 variables describing this AdS space may be different from the
 variables describing the original (UV) AdS space, for instance the
 form of the $SO(4,2)$ isometries may be different
 \cite{Distler:1998gb}. If the solution is described in terms of a
 space which has a non-zero minimal value of $U$ (similar to the space
 which appears in the AdS-Schwarzschild black hole solution described
 in section \ref{FiniteT}, but in this case with the full $ISO(3,1)$ isometry
 group unbroken) the field theory will confine and be trivial in the
 IR. In other cases the geometrical description of the space could
 break down for small values of $U$; presumably this is what happens
 when the field theory flows to a free theory in the IR.


\subsection{A c-theorem}
\label{cTheorem}

Without a detailed analysis of matter fields involved in non-anti-de Sitter
geometries, there are few generalities one can make about the description
of renormalization group flows in the AdS/CFT correspondence\footnote{See
\cite{Akhmedov:1998vf,Alvarez:1998wr,Gorsky:1998wn,Porrati:1999ew,
Balasubramanian:1999jd}
for general discussions of the renormalization group flow in the context of
the AdS/CFT correspondence.}.  However,
there is one general result in gravity \cite{Freedman:1999gp} (see also
\cite{Girardello:1998pd}) which translates into a c-theorem via the
correspondence.  Let us consider $D$-dimensional metrics of the form
  \eqn{DefMet}{
   ds^2 = e^{2 A(r)} (-dt^2 + d\vec{x}^2) + dr^2 \ .
  }
 Any metric with Poincar\'e invariance in the $t,\vec{x}$ directions can be
brought into this form by an appropriate choice of the radial variable $r$.
Straightforward calculations yield
  \eqn{AInequality}{
      -(D-2) A'' = R^t_t - R^r_r = G^t_t - G^r_r =
     \kappa_D^2 (T^t_t - T^r_r) \geq 0 \ .
  }
 In the second to last step we have used Einstein's equation, and in the
last step we have assumed that the weak energy condition holds in the form
  \eqn{WeakEnergy}{
   T_{\mu\nu} \zeta^\mu \zeta^\nu \geq 0
  }
 for any null vector $\zeta^\mu$.  This form of the weak energy condition
is also known as the null energy condition, and it is obeyed by all fields
which arise in Kaluza-Klein compactifications of supergravity theories to
$D$ dimensions.  Thus, we can take it as a fairly general fact that $A''
\leq 0$ for $D > 2$.  Furthermore, the inequality is saturated precisely
for anti-de Sitter space, where the only contribution to $T_{\mu\nu}$ is
from the cosmological constant.  Thus in particular, any deformation of
$AdS_D$ arising from turning on scalar fields will cause $A$ to be concave
as a function of $r$.  If we are interested in relevant deformations
of the conformal field theory, then we should recover linear behavior in
$A$ near the boundary, which corresponds to the (conformal)
ultraviolet limit in the
field theory.  Without loss of generality, then, we assume $A(r) \sim
r/\ell$ as $r \to \infty$.

The inequality $A'' \leq 0$ implies that the function
  \eqn{CFunction}{
   {\cal C}(r) \equiv {1 \over A'^{D-2}} 
  }
 decreases monotonically as $r$ decreases.  Now, suppose there is a region
where $A$ is nearly linear over a range of $r$ corresponding to many orders
of magnitude of $e^{A(r)}$.  This is the bulk analog of a scaling region in
the boundary field theory.  The asymptotically linear behavior of $A(r)$ as
$r\to\infty$ indicates an ultraviolet scaling region which extends
arbitrarily high in energy.  If $A(r)$ recovers linear behavior as $r \to
-\infty$, there is an infrared scaling region; and there could also be
large though finite scaling regions in between.  Assuming odd bulk
dimension $D$, the perfect $AdS_D$ spacetime which any such scaling region
approximates leads to an anomalous VEV
  \eqn{TVEV}{
   \langle T^\mu_\mu \rangle = {\hbox{universal} \over A'^{D-2}} \ ,
  }
 where the numerator is a combination of curvature invariants which can be
read off from the analysis of \cite{Henningson:1998gx} (see section
\ref{anomalies}).  The point is that
in limits where conformal invariance is recovered, the expression
\CFunction\ coincides with the anomaly coefficients of the boundary field
theory, up to factors of order unity which are universal for all CFT's in a
given dimension.  Thus, ${\cal C}(r)$ is a c-function, and the innocuous
inequality $A'' \leq 0$ amounts to a c-theorem provided that Einstein
gravity is a reliable approximation to the bulk physics.

In geometries such as the interpolating kinks of
\cite{Girardello:1998pd,Distler:1998gb,Freedman:1999gp} (discussed in more
detail in section~\ref{defstring}), the outer anti-de Sitter region is
distinguishable from the inner one in that it has a boundary.  There can
only be one boundary (in Einstein frame) because $A$ gets large and
positive only once.  In fact, the inner anti-de Sitter region has finite
proper volume if the coordinates $t$ and $\vec{x}$ in \DefMet\ are made
periodic.  Supergravity is capable of describing irreversible
renormalization group flows despite the reversibility of the equations,
simply because the basic prescription for associating the partition
functions of string theory and field theory makes use of the unique
boundary.





 \subsection{Deformations of the $\cn=4$ $SU(N)$ SYM Theory}
 \label{deffield}

 The most natural deformations to examine from the field theory point
 of view are mass deformations, that would give a mass to the scalar
 and/or fermion fields in the $\cn=4$ vector multiplet. One is tempted
 to give a mass to all the scalars and fermions in the theory, in order
 to get a theory that will flow to the pure Yang-Mills (YM) theory in
 the IR. Such a deformation would involve operators of the form
 $\tr(\phi^I \phi^I)$ for the scalar masses, and $[\epsilon^{\alpha
 \beta} \tr(\lambda_{\alpha A} \lambda_{\beta B}) + c.c.]$ for the
 fermion masses. In the weak coupling regime of small $\lambda=g_{YM}^2
 N$, such deformations indeed make sense and would lead to a pure
 Yang-Mills theory in the IR. However, the analysis of this region
 requires an understanding of the string theory in the high-curvature
 region which corresponds to small $\lambda$, which is not yet
 available. With our present knowledge of string theory we are limited
 to analyzing the strong coupling regime of large $\lambda$, where
 supergravity is a good approximation to the full string theory. In
 this regime there are two problems with the mass deformation described
 above :
 \begin{itemize}
 \item{} The operator $\tr(\phi^I \phi^I)$ is a non-chiral operator, so
 the analysis of section \ref{chiralops} suggests that for large
 $\lambda$ it acquires a dimension which is at least as large as
 $\lambda^{1/4}$, and in particular for large enough values of
 $\lambda$ it is an irrelevant operator. Thus, we cannot deform the
 theory by this operator for large $\lambda$. In any case this operator
 is not dual to a supergravity field, so analyzing the corresponding
 deformation requires going beyond the supergravity approximation.
 \item{} The pure YM theory is a confining theory which dynamically
 generates a mass scale $\Lambda_{YM}$, which is the characteristic
 mass scale for the particles (glueballs) of the theory. When we deform
 the $\cn=4$ theory by a mass deformation with a mass scale $m$, a
 one-loop analysis suggests that the mass scale $\Lambda_{YM}$ will be
 given by $\Lambda_{YM} \sim m e^{-c/g_{YM}^2(m) N}$, where $c$ is a
 constant which does not depend on $N$ (arising from the one-loop
 analysis) and $g_{YM}^2(m)$ is the coupling constant at the scale
 $m$. Thus, we find that while for small $\lambda$ we have 
 $\Lambda_{YM} \ll
 m$ and there is a separation of scales between the dynamics of the
 massive modes and the dynamics of the YM theory we want to study, for
 large $\lambda$ we have $\Lambda_{YM} \sim m$ and there is no such
 separation of scales (for non-supersymmetric mass deformations the
 one-loop analysis we made is not exact, but an exact analysis is not
 expected to change the qualitative behavior we describe). Thus, we
 cannot really study the pure YM theory, or any other confining theory
 (which does not involve all the fields of the original $\cn=4$ theory)
 as long as we are in the strong coupling regime where supergravity is
 a good approximation.
 \end{itemize}
 We will see below that, while we can find ways to get around the first
 problem and give masses to the scalar fields, there are no known ways
 to solve the second problem and study interesting confining field
 theories using the supergravity approximation. Of course, in the full
 string theory there is no such problem, and the mass deformation
 described above, for small $\lambda$, gives an implicit string theory
 construction of the non-supersymmetric pure YM theory.

 In the rest of this section we will focus on the deformations that
 can arise in the strong coupling regime, and which may be analyzed in
 the supergravity approximation. As described in section
 \ref{chiralops}, the only operators whose dimension remains small for
 large $N$ and large $\lambda$ are the chiral primary operators, so we
 are limited to deformations by these operators. Let us start by
 analyzing the symmetries that are preserved by such
 deformations. Most of the chiral operators are in non-trivial
 $SU(4)_R$ representations, so they break the $SU(4)_R$ group to some
 subgroup which depends on the representation of the operator we are
 deforming by. Generic deformations will also completely break the
 supersymmetry. One analyzes how much supersymmetry a particular
 deformation breaks by checking how many supercharges annihilate
 it. For example, deformations which preserve $\cn=1$ supersymmetry
 are annihilated by the supercharges $Q_\alpha$ and ${\bar
 Q}_{\dot{\alpha}}$ of some $\cn=1$ subalgebra of the $\cn=4$
 algebra. Given the structure of the chiral representations described
 in section \ref{chiralops} it is easy to see if a deformation by such
 an operator preserves any supersymmetry or not. Examples of
 deformations which preserve some supersymmetry are superpotentials of
 the form $W=h\tr(\Phi^{i_1} \Phi^{i_2} \cdots \Phi^{i_n})$, which to
 leading order in $h$ add to the Lagrangian a term of the form
 $[h\epsilon^{\alpha \beta} \tr(\lambda_{\alpha A_1} \lambda_{\beta
 A_2} \phi^{I_1} \cdots \phi^{I_{n-2}}) + c.c.]$. These operators are
 part of the scalar operators described in section \ref{chiralops}
 arising at dimension $n+1$ in the chiral multiplet. In order to
 preserve supersymmetry one must also add to the Lagrangian various
 terms of order $h^2$, so we see that the question of whether a
 deformation breaks supersymmetry or not depends not only on the
 leading order operator we deform by but also on additional operators
 which we may or may not add at higher orders in the deformation
 parameter (note that the form of the chiral operators also changes
 when we deform, so an exact analysis of the deformations beyond the
 leading order in the deformation is highly non-trivial). Another
 example of a supersymmetry-preserving deformation is a superpotential
 of the form $W=h\tr(W_{\alpha}^2 \Phi^{i_1} \cdots \Phi^{i_{n-2}})$,
 which deforms the theory by some of the scalar operators arising at
 dimension $n+2$ in the chiral multiplet (e.g. the dilaton deformation
 for $n=2$, which actually preserves the full $\cn=4$ supersymmetry).

 The list of chiral operators which correspond to marginal or relevant
 deformations was given in section \ref{chiralops}. There is a total
 of 6 such operators, three of which are the lowest components of the
 chiral multiplets with $n=2,3,4$\footnote{In a $U(N)$ theory there is
 an additional scalar operator which is the lowest component of the
 $n=1$ multiplet.}. These operators are traceless
 symmetric products of scalars ${\cal O}_n=\tr(\phi^{\{I_1} \phi^{I_2}
 \cdots \phi^{I_n\}})$, which viewed as deformations of the theory
 correspond to non-positive-definite potentials for the scalar
 fields. Thus, at least if we are thinking of the theory on $\IR^4$
 where the scalars have flat directions before adding the potential,
 these deformations do not make sense since they would cause the
 theory to run away along the flat directions. In particular, the
 deformation in the $\bf 20'$ which naively gives a mass to the
 scalars really creates a negative mass squared for at least some of
 the scalars, so it cannot be treated as a small deformation of the UV
 conformal theory at the origin of moduli space. We will focus here
 only on deformations by the other 3 operators, which all seem to make
 sense in the field theory.

 One marginal operator of dimension 4 is the operator which couples to
 the dilaton, which is a $\bf 1$ of $SU(4)_R$, of the form $[\tr(F_{\mu
 \nu}^2) + i\tr(F \wedge F) + \cdots]$. Deforming by this operator
 corresponds to changing the coupling constant $\tau_{YM}$ of the field
 theory, and is known to be an exactly marginal deformation which does
 not break any of the symmetries of the theory.

 The other two relevant or marginal deformations are the scalars of
 dimension $n+1$ in the $n=2$ and $n=3$ multiplets. Let us start by
 describing the relevant deformation, which is a dimension 3 operator
 in the $\bf 10$ of $SU(4)_R$, of the form 
 \eqn{defthree}{\left[
 \epsilon^{\alpha \beta} \tr(\lambda_{\alpha A} \lambda_{\beta B}) +
 \tr([\phi^I, \phi^J] \phi^K) \right],} 
 where the indices are contracted to be in the $\bf 10$ of $SU(4)_R$
 (which is in the symmetric product of two $\bf {\bar 4}$'s and in the
 self-dual antisymmetric product of three $\bf 6$'s). This operator is
 complex; obviously when we add it to the Lagrangian we need to add it
 together with its complex conjugate. The coefficient
 parametrizing the deformation is a complex number $m^a$ in the $\bf
 10$ of $SU(4)_R$. Deforming by this operator obviously gives a mass
 to some or all of the fermion fields $\lambda$, depending on the
 exact values of $m^a$. For generic values of $m^a$, all the fermions
 will acquire a mass and supersymmetry will be completely broken. The
 scalars will then obtain a mass from loop diagrams in the field
 theory, so that the low-energy theory below a scale of order $m^a$
 will be the pure non-supersymmetric Yang-Mills theory. Unfortunately,
 as described above, for large $\lambda=g_{YM}^2 N$ this is not really
 a good description since this theory will confine at a scale
 $\Lambda_{YM}$ of order $m$. However, for small $\lambda$ this
 deformation does enable us to obtain the pure YM theory as a
 deformation of the $\cn=4$ theory.

 It is interesting to ask what happens if we give a mass only to some
 of the fermions. In this case we may or may not preserve some amount
 of supersymmetry (obviously, preserving $\cn=1$ supersymmetry
 requires leaving at least one adjoint fermion massless). The
 deformations which preserve at least $\cn=1$ supersymmetry correspond
 to superpotentials of the form $W = m_{ij} \tr(\Phi^i
 \Phi^j)$. Choosing an $\cn=1$ subgroup breaks $SU(4)_R$ to
 $SU(3)\times U(1)_R$, and (if we choose the $U(1)$ normalization so
 that the supercharges decompose as ${\bf 4} = {\bf 3}_1 + {\bf
 1}_{-3}$) the ${\bf 10}$ decomposes as ${\bf 10} = {\bf 6}_2 + {\bf
 3}_{-2} + {\bf 1}_{-6}$. The SUSY preserving deformation $m_{ij}$ is
 then in the ${\bf 6}_2$ representation, and it further breaks both
 the $SU(3)$ and the $U(1)$. In a supersymmetric deformation we
 obviously need to also add masses of order $m^2$ to some of the
 scalars; naively this leads to a contradiction because, as described
 above, there are no reasonable scalar masses to add which are in
 chiral operators. However, at order $m^2$ we have to take into
 account also the mixings between operators which occur at order $m$
 in the deformation\footnote{Similar mixings were recently discussed in
 \cite{Intriligator:1999ff}.}; the form of the chiral operators changes after we
 deform, and they mix with other operators (in particular, the form of
 the operator which is an eigenvalue of the scaling operator changes
 when we turn on $m$). In the case of the supersymmetric mass
 deformation, at order $m$ the chiral operator \eno{defthree}
 described above mixes with the non-chiral $\tr(\phi^I \phi^I)$
 operator giving the scalars a mass, so there is no contradiction. The
 simplest way to see this operator mixing in the SUSY-preserving case
 is to note that the $\cn=1$ SUSY transformations in the presence of a
 general superpotential include terms of the form $\{Q_{\alpha},
 \lambda_{\beta i}\} \sim \epsilon_{\alpha \beta} {d{\bar W}\over
 d{{\bar \Phi}^i}}$, which lead to corrections of order $m$ to $[Q^2,
 {\cal O}_2]$ which is the operator that we are deforming by.

 There are two interesting ways to give a mass to only one of the
 fermions. One of them is a particular case of the SUSY-preserving
 deformation described above, of the form $W = m \tr(\Phi^1 \Phi^1)$,
 which is an element of the ${\bf 6}_{2}$ of $SU(3)\times U(1)$, and
 breaks $SU(4)_R \to SU(2)\times U(1)$ while preserving $\cn=1$ SUSY
 (but breaking the conformal invariance). The other possibility is to
 use the deformation in the ${\bf 1}_{-6}$, which breaks SUSY
 completely but preserves an $SU(3)$ subgroup of $SU(4)_R$. To leading
 order in the deformation both possibilities give a mass to one
 fermion, but at order $m^2$ they differ in a way which causes one of
 them to break SUSY while the other further
 breaks $SU(3)\to SU(2)\times
 U(1)$. At weak coupling we can analyze the order $m^2$ terms in
 detail. In the SUSY-preserving deformation at order $m^2$ we turn on a
 scalar mass term of the form $|m|^2 \tr[(\phi^1)^2+(\phi^2)^2]$, which
 may be written in the form 
 \eqn{susymassterm}{{{|m|^2}\over 3}
 \tr[2(\phi^1)^2+2(\phi^2)^2-(\phi^3)^2-(\phi^4)^2-(\phi^5)^2-(\phi^6)^2]
 + {{|m|^2}\over 3}\tr[\phi^I \phi^I],}
 where the first term is one of
 the $\Delta=2$ chiral operators in the $\bf 20'$, and the second term
 is a non-chiral operator which arises from the operator mixing as
 described above (the appearance of the second term allows us to add
 the chiral operator in the first term without destroying the
 positivity of the scalar potential). In the non-SUSY deformation the
 chiral term is not turned on at any order in the deformation (the $\bf
 20'$ representation contains no singlets of $SU(3)$), and all the
 scalars get equal masses from the non-chiral term.

 Which theory do we flow to in the IR after turning on such a
 single-fermion mass term~? In the SUSY-preserving case one can show
 that we actually flow to an $\cn=1$ SCFT (and, in fact, to a fixed
 line of $\cn=1$ SCFTs). Naively, one chiral multiplet gets a mass,
 and we remain with the $\cn=1$ $SU(N)$ SQCD theory with two adjoint
 chiral multiplets, which is expected (based on the amount of matter
 in the theory) to flow to an interacting IR fixed point. In fact, one
 can prove \cite{Karch:1999pv} that there is an exactly marginal
 operator at that fixed point, which (generally) has a non-zero value
 in the IR theory we get after the flow described above. The full
 superpotential with the deformation is of the form $W = h\tr(\Phi^1
 [\Phi^2, \Phi^3]) + m\tr(\Phi^1 \Phi^1)$ (where $h$ is proportional
 to $g_{YM}$), and to describe the low-energy theory we can integrate
 out the massive field $\Phi^1$ to remain with a superpotential $W =
 -{h^2\over 4m} \tr([\Phi^2, \Phi^3]^2)$ for the remaining massless
 fields. Naively this superpotential is irrelevant (its dimension at
 the UV fixed point at weak coupling is 5), but in fact one can show
 (for instance, using the methods of \cite{Leigh:1995ep}) that it is
 exactly marginal in the IR theory, so there is a fixed line of SCFTs
 parametrized by the coefficient $\tilde h$ of the superpotential $W =
 {\tilde h} \tr([\Phi^2,\Phi^3]^2)$. Upon starting from a particular
 value of $g_{YM}$ in the UV and performing the supersymmetric mass
 deformation, we will land in the IR at some particular point on the
 IR fixed line (i.e. some value of $\tilde h$). The unbroken global
 $U(1)$ symmetry of the theory becomes the $U(1)_R$ in the $\cn=1$
 superconformal algebra in the IR.

 It is more difficult to analyze the mass deformation which does not
 preserve SUSY (but preserves $SU(3)$), since we cannot use the
 powerful constraints of supersymmetry. Naively one would expect this
 deformation to lead to masses (from loop diagrams) for all of the
 scalars, but not for the fermions, since the $SU(3)$ symmetry
 prevents them from acquiring a mass. Then, the IR theory seems to be
 $SU(N)$ Yang-Mills coupled to three adjoint fermions, which
 presumably flows to an IR fixed point (this is what happens for
 supersymmetric theories with one-loop beta functions of the same
 order, but it is conceivable also that the theory may confine and
 generate a mass scale). There is no reason for such a fixed point to
 have any exactly marginal deformations (in fact, there are no known
 examples in four dimensions of non-supersymmetric theories with
 exactly marginal deformations), so presumably the flow starting from
 any value of $g_{YM}$ always ends up at the same IR fixed point. We
 assumed that the deformation leads to positive masses squared for the
 scalars; it is also possible that it would give rise to negative
 masses squared for the scalars, in which case the theory on $\IR^4$
 would have no vacuum, as described above.

 If we give a mass to two of the fermions, it is possible to do this
 with a superpotential of the form $W = m\tr(\Phi^1 \Phi^2)$ which in
 fact preserves $\cn=2$ supersymmetry (it gives the $\cn=2$ SQCD
 theory with one massive adjoint hypermultiplet, which was discussed
 in \cite{Donagi:1996cf}). This theory is known to dynamically
 generate a mass scale, at which the $SU(N)$ symmetry is broken (at a
 generic point in the moduli space) to $U(1)^{N-1}$, and the
 low-energy theory is the theory of $(N-1)$ free $U(1)$ vector
 multiplets. The behavior of this theory for large $N$ was discussed
 in \cite{Douglas:1995nw}. At special points in the moduli space there
 are massless charged particles, and at even more special points in
 the moduli space \cite{Argyres:1995jj,Argyres:1996xn,Eguchi:1996vu}
 there are massless electrically and magnetically charged particles
 and the theory is a non-trivial $\cn=2$ SCFT. It is not completely
 clear which point in the moduli space one would flow to upon adding
 the mass deformation to the $\cn=4$ theory. Presumably, without any
 additional fine-tuning one would end up at a generic point in the
 moduli space which corresponds to a free IR theory.

 If we give a mass to two fermions while breaking supersymmetry (as
 above, this depends on the order $m^2$ terms that we add), we
 presumably end up in the IR with Yang-Mills theory coupled to two
 massless adjoint fermions. This theory is expected to confine at some
 scale $\Lambda_{YM}$ (which for large $g_{YM}^2 N$ would be of the
 order of the scale $m$), and lead to a trivial theory in the IR. A
 similar confining behavior presumably occurs if we give a mass to
 three or four of the fermions (for three fermions we can give a mass
 while preserving SUSY, and we presumably flow in the IR to the
 confining $\cn=1$ pure SYM theory).

 The only remaining deformation is the deformation by the $\Delta=4$
 operator in the $\bf 45$ representation, which is in the $n=3$
 multiplet. A general analysis of this deformation is rather difficult,
 so we will focus here on the SUSY preserving case where the
 deformation is a superpotential of the form $W = h_{ijk} \tr(\Phi^i
 \Phi^j \Phi^k)$, with the coefficients $h_{ijk}$ in the ${\bf 10}_0$
 representation in the decomposition ${\bf 45} = {\bf 15}_4+{\bf
 10}_0+{\bf 8}_0+{\bf 6}_{-4}+{\bf {\bar 3}}_{-4}+{\bf 3}_{-8}$. It
 turns out that one can prove (see \cite{Leigh:1995ep} 
 and references therein) that two of these ten deformations
 correspond to exactly marginal operators, that preserve $\cn=1$
 superconformal invariance. This can be done by looking at a general
 $\cn=1$ theory with three adjoint chiral multiplets, a gauge coupling
 $g$, and a superpotential of the form
 \eqn{exmarsup}{W = h_1 \tr(\Phi^1 \Phi^2 \Phi^3 + \Phi^1
 \Phi^3 \Phi^2) + h_2 \tr((\Phi^1)^3+(\Phi^2)^3+(\Phi^3)^3)
 + h_3\epsilon_{ijk}\tr(\Phi^i \Phi^j \Phi^k).}
 This particular superpotential is chosen to preserve a $\IZ_3\times
 \IZ_3$ global symmetry, where one of the $\IZ_3$ factors acts by
 $\Phi^1 \to \Phi^2, \Phi^2 \to \Phi^3, \Phi^3 \to \Phi^1$ and the
 other acts by $\Phi^1 \to \Phi^1, \Phi^2 \to \omega \Phi^2, \Phi^3
 \to \omega^2 \Phi^3$ where $\omega$ is a third root of unity. The
 second $\IZ_3$ symmetry prevents any mixing between the chiral
 operators $\Phi^i$, and the first $\IZ_3$ can then be used to show
 that they all have the same anomalous dimension
 $\gamma(g,h_1,h_2,h_3)$. The beta function may be shown (using
 supersymmetry) to be exactly proportional to this gamma function
 (with a coefficient which is a function of $g$), so that the
 requirement of conformal invariance degenerates into one equation
 ($\gamma=0$) in the four variables $g,h_1,h_2$ and $h_3$, which
 generically has a 3-dimensional space of solutions. This space of
 solutions corresponds to a 3-dimensional space of $\cn=1$ SCFTs. The
 general arguments we used so far do not tell us the form of the
 3-dimensional space, but we can now use our analysis of the $\cn=4$
 theory to learn more about it. First, we know that the $\cn=4$ line
 $g=h_3, h_1=h_2=0$ is a subspace of this 3-dimensional space. We also
 know that at leading order in the deformation away from this
 subspace, $(h_3+g)$, $h_1$ and $h_2$ correspond to marginal operators
 (as described above they couple to chiral operators of dimension 4),
 while $(h_3-g)$ couples to a non-chiral operator (in the $\bf 15$ of
 $SU(4)_R$) whose dimension is corrected away from $g=0$ (and seems to
 be large for large $g_{YM}^2 N$). Thus, we see that to leading order
 in the deformation around the $\cn=4$ fixed line, the exactly
 marginal deformations are given by $h_1$ and $h_2$ (which are two
 particular elements of the ${\bf 10}_0$ representation). It is not
 known if the other deformations in the $\bf 45$ are marginally
 relevant, marginally irrelevant or exactly marginal.

 \subsection{Deformations of String Theory on $AdS_5\times S^5$}
 \label{defstring}

 As described in section \ref{defintro}, to analyze the deformations
 of section \ref{deffield} in the AdS context requires finding
 solutions of string theory with appropriate boundary conditions. For
 the exactly marginal deformation in the $\bf 1$, which corresponds to
 the dilaton, we already know the solutions, which are just the
 $AdS_5\times S^5$ solution with any value of the string coupling
 $\tau_{IIB}$. The other operators discussed above are identified in
 string theory with particular modes of the 2-form field $B_{ab}$ with
 indices in the $S^5$ directions (we view $B$ as a complex 2-form
 field which contains both the NS-NS and R-R 2-form fields). Thus, the
 dimension 3 mass deformation would be related to string theory
 backgrounds in which $B_{ab}(x,U,y) \stackrel{U\to
 \infty}{\longrightarrow} m Y_{ab}^{(1)}(y) / U$ for some spherical
 harmonics $Y_{ab}^{(1)}(y)$ on $S^5$, and the dimension 4
 deformations would be related to backgrounds with $B_{ab}(x,U,y)
 \stackrel{U\to \infty}{\longrightarrow} h Y_{ab}^{(2)}(y)$. It is
 clear from the identification of the superconformal algebra in the
 field theory and in the string theory that these deformations break
 the same supersymmetries in both cases; this can also be checked
 explicitly (say, to leading order in the deformation
 \cite{aks_unpublished,Girardello:1998pd}) by analyzing the SUSY
 variations of the type IIB supergravity fields. The existence of an
 exactly marginal deformation breaking the $\cn=4$ superconformal
 symmetry to $\cn=1$ superconformal symmetry suggests that the theorem
 of \cite{Banks:1988yz}, that forbids flat space compactifications
 with different amounts of supersymmetry from being at a finite
 distance from each other in the string theory moduli space, is not
 valid in AdS compactifications
 \cite{aks_unpublished,Girardello:1998pd}.

 Since we know little about string theory in backgrounds
 with RR fields, our analysis of such solutions is effectively limited
 to the supergravity approximation. This already limits our discussion
 to large $\lambda=g_s N$, and it limits it further to cases where the
 solution does not develop large curvatures in the interior. In the
 supergravity limit one would want to find solutions of type IIB
 supergravity with the boundary conditions described above (with the
 rest of the fields having the same boundary conditions as in the
 $AdS_5\times S^5$ case). Unfortunately, no such solutions are
 known, and they seem to be rather difficult to construct. There are 3
 possible approaches to circumventing this problem of finding exact
 solutions to type IIB supergravity :
\begin{itemize} 
\item{} One can try to construct solutions perturbatively in the
deformation parameter, which should be easier than constructing the
full exact solution. Unfortunately, this approach does not make sense
for the relevant deformations, since already at leading order in the
deformation (corresponding to the linearized equations of motion
around the $AdS_5\times S^5$ solution) we find that the solution
($B_{ab} \sim 1/U$) grows to be very large in the interior, so the
perturbative expansion does not make sense. At best one may hope to
have a perturbative expansion in a parameter like $m/U$ (if $m$ is the
coefficient of a relevant operator of dimension $\Delta=3$),
but this only makes sense near the boundary. On the other hand, for
marginal deformations, and especially for deformations that are
supposed to be exactly marginal, this approach makes sense. Exactly
marginal deformations correspond to solutions which do not depend on
the AdS coordinates at all, so a perturbation expansion in the
parameters of the deformation seems to be well-defined. In practice
such a perturbation expansion is quite complicated, and can only be
done in the first few orders in the deformation. In the case of the
deformation by $h_1,h_2$ which was described in field theory above,
one can verify that it is an exactly marginal deformation to second
order in the deformation, even though additional SUGRA fields need to
be turned on at this order (including components of the metric with
$S^5$ indices). This is in fact true for any deformation in the $\bf
45$. At third order one probably gets non-trivial constraints on which
elements of the $\bf 45$ can be turned on in an exactly marginal
deformation, but the equations of motion of type IIB SUGRA have not
yet been expanded to this order. Verifying that the deformations that
are exactly marginal in the field theory correspond to exactly
marginal deformations also in string theory on $AdS_5\times S^5$ would
be a non-trivial test of the AdS/CFT correspondence.

\item{} There are no known non-trivial solutions of type IIB
supergravity which are asymptotically of the form described above for
the relevant or marginal deformations. However, there are several
known solutions \cite{Romans:1985an,vanNieuwenhuizen:1985ri} of type
IIB supergravity (in addition to the $AdS_5\times S^5$ solution) which
involve $AdS_5$ spaces and have $SO(4,2)$ isometries (these solutions
need not necessarily be direct products $AdS_5\times X$), and one can
try to guess that they would be the end-points of flows arising from
relevant deformations. As long as we are in the supergravity
approximation, only solutions which are topologically equivalent to
$AdS_5\times S^5$ can be related by flows to the $AdS_5\times S^5$
solution, so we will not discuss here other types of $AdS_5$
solutions.

One such solution was found in \cite{Romans:1985an}, which is of the
 form $AdS_5\times X$, where $X$ is an $S^1$ fiber over $CP^2$ (a
 ``stretched five-sphere''), and there is also a 3-form field turned
 on in the compact directions (this is called a Pope-Warner type
 solution \cite{Pope:1985jj}). This solution has an $SU(3)$ isometry
 symmetry (corresponding to an $SU(3)$ global symmetry in the
 corresponding field theory), and it breaks all the
 supersymmetries. Thus, it is natural to try to identify it with the
 deformation by the non-supersymmetric single-fermion mass operator
 described in section \ref{deffield}, which has the same
 symmetries. Unfortunately, as discussed below, this solution seems to
 be unstable.
% Additional evidence for this identification will be
% presented below. This classical supergravity solution, like all type
% IIB supergravity solutions, has the dilaton as an arbitrary
% parameter, corresponding to a fixed line in the corresponding field
% theory. However, for finite $N$ one would expect quantum corrections
% in this non-supersymmetric background to generate a potential for the
% dilaton (as well as for any other massless scalar fields if they
% exist in this compactification), which should have a unique vacuum to
% correspond with the field theory expectations described above. For
% infinite $N$ this correspondence would predict a fixed line in the
% corresponding field theory, but it is not clear how to analyze this
% directly in the field theory (perturbation theory is not valid since
% the field theory, which is the IR limit of QCD with three adjoint
% fermions, is always strongly coupled, so if there is a fixed line it
% does not pass through weak coupling). We will discuss this solution
% further below.

 An additional solution, found in \cite{vanNieuwenhuizen:1985ri},
 exhibits an $SO(5)$ global symmetry. As discussed below,
 this solution also appears to be unstable.

 \item{} The most successful way (to date) of analyzing the
 appropriate solutions of type IIB supergravity has been to restrict
 attention to the five dimensional $\cn=8$ supergravity
 \cite{Gunaydin:1986cu} sector of the theory, which includes only the
 $n=2$ ``supergraviton'' multiplet from the spectrum described in
 section \ref{chiralops}. Unlike the situation in flat-space
 compactifications, the five dimensional supergravity cannot be viewed
 as a low-energy limit of the ten dimensional supergravity
 compactification in any sense. For instance, the supergraviton
 multiplet contains fields of $m^2=-4/R^2$, while other multiplets (in
 the $n=3,4$ multiplets) which are not included in the truncation to
 the five dimensional supergravity theory involve massless fields on
 $AdS_5$. However, it is conjectured that there does exist a
 consistent truncation of the type IIB supergravity theory on
 $AdS_5\times S^5$ to the five dimensional $\cn=8$ supergravity, in
 the sense that every solution of the latter can be mapped into a
 solution of the full type IIB theory (with the other fields in type
 IIB supergravity being some functions of the five dimensional SUGRA
 fields). A similar truncation is believed to exist (\cite{deWit:1987iy,
 vN:99ct} and references therein)
 for the relation between 11 dimensional supergravity compactified on
 $AdS_4\times S^7$ and the four dimensional $\cn=8$ gauged
 supergravity, and
 for the relation between 11 dimensional supergravity compactified
 on $AdS_7\times S^4$ and the seven dimensional gauged supergravity, 
 and the similarities between the two cases suggest that
 it may exist also in the $AdS_5\times S^5$ case (though this has not
 yet been proven\footnote{Partial evidence for this was given in
 \cite{Cvetic:1999xp}.  See section~\ref{ConsistentTruncation} for
        further discussion.}). In the rest of this section we will assume
 that such a truncation exists and see what we can learn from
 it. Obviously, we can only learn from such a truncation about
 deformations of the theory by fields in the $n=2$ multiplet, so we
 cannot analyze the marginal deformations in the $\bf 45$ in this way.

 The first thing one can try to do with the five dimensional $\cn=8$
 supergravity is to find solutions to the equations of motion with an
 $SO(4,2)$ isometry. These correspond to critical points of the scalar
 potential of $d=5,\cn=8$ supergravity, which is a complicated
 function of the 42 (=${\bf 20'} + {\bf 10}_c + {\bf 1}_c$) scalar
 fields in the $n=2$ multiplet. A full analysis of the critical points
 of this potential has not yet been performed, but there are 4 known
 vacua in addition to the vacuum corresponding to $AdS_5\times S^5$ :

 (i) There is a non-supersymmetric vacuum with an unbroken $SU(3)$
 gauge group. This vacuum is conjectured to correspond to the
 $SU(3)$-invariant vacuum of the full type IIB supergravity theory
 described above, which, as mentioned above, could correspond to a
 mass deformation of the $\cn=4$ field theory. Additional evidence for
 this correspondence was presented in
 \cite{Girardello:1998pd,Distler:1998gb}, which constructed a solution
 of the five dimensional $\cn=8$ supergravity which interpolates
 between the $AdS_5\times S^5$ solution and the $SU(3)$-invariant
 solution, with the leading deformation from the $AdS_5\times S^5$
 solution corresponding exactly to the mass operator in the ${\bf
 1}_{-6}$ in the decomposition ${\bf 10} = {\bf 6}_2 + {\bf 3}_{-2} +
 {\bf 1}_{-6}$, which breaks $SU(4)_R \to SU(3)$. 
% If indeed there is a
% consistent truncation of the type IIB supergravity to 5d $\cn=8$
% SUGRA then this is convincing evidence that the $SU(3)$ invariant
% solution indeed corresponds (in the large $N$ limit) to the fixed
% point arising from the single-fermion-mass deformation of the $\cn=4$
% field theory. 
 Since this solution is non-supersymmetric, one must
 verify that the classical solution is stable, namely that it does not
 contain tachyons whose mass is below the Breitenlohner-Freedman
 stability bound (in supersymmetric vacua this is guaranteed; using
 equation (\ref{dimenmass}), such tachyons would correspond to operators
 of complex dimension in the field theory which would contradict its
 unitarity). 
% In \cite{Distler:1998gb} it was verified that this is
% true for the fields of the five dimensional supergravity multiplet;
% it would be interesting to verify this for the full spectrum of the
% type IIB supergravity theory, to confirm that the solution indeed
% describes a consistent supergravity 
% compactification.\footnote{Note added in proofs: It has recently come
 It has recently been shown \cite{KPPrivate} that there are scalars in
 the gauged supergravity multiplet which do violate the
 Breitenlohner-Freedman stability bound in the expansion around the
 $SU(3)$-invariant solution\footnote{
 Except for orbifold constructions, there is no example at the time of
 writing of a non-supersymmetric $AdS_5$ vacuum which is definitely
 known to satisfy the stability bound.  There are however non-orbifold,
 non-supersymmetric $AdS_3$ vacua which are perturbatively stable.}
% Excluding orbifold constructions,
% there is no example at the time of writing of a non-supersymmetric
% $AdS_5$ vacuum of supergravity which is definitely known to satisfy
% the stability bound. There do exist, however, orbifold examples and
% $AdS_3$ examples of such vacua.}. 
Thus, this is not a consistent
 vacuum of the supergravity theory. The AdS/CFT correspondence then
 implies that performing this mass deformation at strong coupling
 leads to some instability in the field theory (for instance, it could
 lead to negative masses squared for the scalar fields).
% The remarks in the main text following this
% footnote can still be read as a summary of the further criteria one
% might want to examine if such an example should arise in order to
% judge whether it describes a genuine non-trivial fixed point.}  The
% central charge of the corresponding field theory was computed in
% \cite{Khavaev:1998fb} and found to be smaller than that of the UV
% $\cn=4$ fixed point, consistent with the conjectured c-theorem. As
% discussed above, the supergravity solution contains a massless scalar
% (the dilaton) which is expected to obtain a non-trivial potential
% when the quantum corrections are included. It is not clear how to
% analyze these corrections and to check whether after their inclusion
% there is still a consistent vacuum of string theory on the
% corresponding manifold or not. Presumably, if such a consistent
% vacuum exists, it would not be a weakly coupled string theory (since
% it is unlikely that the dilaton can be stabilized at weak
% coupling). However, it could still have small curvatures (in the
% large $N$ limit), so that a supergravity analysis of this theory may
% be useful. If the dilaton does not stabilize in the small curvature
% region one would need to go beyond the supergravity approximation to
% learn anything about the theory, and in particular to learn whether
% there is any stable $AdS$-type background (corresponding to an IR
% fixed point in the field theory) or not (corresponding to confinement
% in the field theory).

 (ii) There is a non-supersymmetric vacuum with unbroken $SO(5)$ gauge
 symmetry, which is conjectured to be related to the $SO(5)$-invariant
 compactification of type IIB supergravity which we mentioned
 above. The mass spectrum in this vacuum was computed in
 \cite{Distler:1998gb}, where it was found that it has a tachyonic
 particle whose mass is below the stability bound. Thus, even
 classically this is not really a vacuum of the supergravity theory
 (presumably the tachyon would condense and the theory would flow to
 some different vacuum). It was found in
 \cite{Girardello:1998pd,Distler:1998gb} that this ``vacuum'' is
 related to the $AdS_5\times S^5$ vacuum by a deformation involving
 turning on one of the operators in the $\bf{20'}$ representation;
 presumably the instability of the supergravity solution is related to
 the instability of the field theory after performing this
 deformation.

 (iii) There is \cite{Khavaev:1998fb,Karch:1999pv,Freedman:1999gp} a
 vacuum with $SU(2)\times U(1)$ unbroken symmetry and 8 unbroken
 supercharges, corresponding to an $\cn=1$ SCFT in the field
 theory. There is no known corresponding solution of the full type IIB
 theory, but assuming that 5d SUGRA is a consistent truncation, such a
 solution must exist (though it is not guaranteed that all its
 curvature invariants will be small, as required for the consistency
 of the supergravity approximation). It is natural to identify this
 vacuum with the IR fixed point arising from the supersymmetric
 single-chiral-superfield mass deformation described in section
 \ref{deffield}. This is consistent with the form of the 5d SUGRA
 fields that are turned on in this solution, with the global
 symmetries of the solution, and with the fact that on both sides of
 the correspondence we have a fixed line of $\cn=1$ SCFTs (the
 parameter $\tilde{h}$ of the fixed line corresponds to the dilaton on
 the string theory side; supersymmetry prohibits the generation of a
 potential for this field). Recently this identification was supported
 by the construction of the full solution interpolating between the
 $\cn=4$ fixed point and the $\cn=1$ fixed point in the 5d SUGRA theory
 \cite{Freedman:1999gp}. Since we have some supersymmetry left in
 this case, one can also quantitatively test this correspondence by
 matching the global anomalies of the field theory described in
 section \ref{deffield} (the $SU(N)$ $\cn=1$ SQCD theory with two
 adjoint chiral multiplets and a superpotential $W\propto
 \tr([\Phi^2,\Phi^3]^2)$) with those of the corresponding SUGRA
 background, as described in section \ref{anomalies}. The conformal
 anomalies were successfully compared in
 \cite{Karch:1999pv,Freedman:1999gp} in the
 large $N$ limit, giving some evidence for this correspondence (in
 particular, the conformal anomalies of this theory satisfy $a=c$, as
 required for a consistent supergravity approximation). The fact that
 the central charge corresponding to this solution is smaller than
 that of the $AdS_5\times S^5$ solution with the same RR 5-form flux
 (note that the RR flux is quantized and does not change when we
 deform) means that this interpretation is consistent with the
 conjectured four dimensional c-theorem.

 (iv) There is an additional background found in \cite{Khavaev:1998fb}
 with $SU(2)\times U(1)\times U(1)$ unbroken gauge symmetry and no
 supersymmetry. The mass spectrum of this background has not yet been
 computed, so it is not clear if it is stable or not. The SUGRA
 solution involves giving VEVs to fields both in the $\bf 20'$ and in
 the $\bf 10$, but it is not clear exactly what deformation of the
 original $AdS_5\times S^5$ theory (if any) this background
 corresponds to.

In principle, one could also use the truncated five dimensional theory
to analyze other relevant deformations in the $\bf 10$, which are not
expected to give rise to conformal field theories in the
IR. Presumably most of them would lead to high curvatures in the
interior, but perhaps some of them do not and can then be analyzed
purely in supergravity.

\end{itemize}

To summarize, the analysis of deformations in string theory on
$AdS_5\times S^5$ is rather difficult, but the results that are known
so far seem to be consistent with the AdS/CFT correspondence. The only
known results correspond to deformations which lead to conformal
theories in the IR; as discussed in section \ref{deffield}, these are
also the only deformations which we would expect to be able to
usefully study in general in the supergravity approximation. The most
concretely analyzed deformation is the single-chiral-fermion mass
deformation, which seems to lead to another AdS-type background of
type IIB supergravity (though only the truncation of this background
to the five dimensional supergravity is known so far). In
non-supersymmetric cases the analysis of deformations is complicated
(see, for instance, \cite{Berkooz:1999qp}) by the fact that quantum
corrections are presumably important in lifting flat directions, so a
classical supergravity analysis is not really enough and the full
string theory seems to be needed.


