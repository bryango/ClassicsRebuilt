%From: Lisa C Jeffrey <jeffrey@math.ias.edu>
%Date: Thu, 17 Oct 1996 08:25:08 -0400






\documentstyle[12pt]{article}

\input amssym.def
\input amssym.tex

\newcommand{\Tr}{\,{\rm Tr}\,}
\newcommand{\tr}{{\rm tr}\,}

%L. Jeffrey preamble, 5 April 1996

\newcommand{\nc}{\newcommand}

\nc{\isq}{{i}}
\newcommand{\colvec}[2]{\left  ( \begin{array}{cc} #1  \\
     #2  \end{array} \right ) }

%%%%%\newcommand{\Tr}{\,{\rm Tr}\,}
\newcommand{\End}{\,{\rm End}\,}
\newcommand{\Hom}{\,{\rm Hom}\,}

\newcommand{\Ker}{ \,{\rm Ker} \,}

\newcommand{\bla}{\phantom{bbbbb}}
\newcommand{\onebl}{\phantom{a} }
\newcommand{\eqdef}{\;\: {\stackrel{ {\rm def} }{=} } \;\:}
\newcommand{\sign}{\: {\rm sign}\: }
\newcommand{\sgn}{ \:{\rm sgn}\:}
\newcommand{\half}{ {\frac{1}{2} } }
\newcommand{\vol}{ \,{\rm vol}\, }


% define abbreviations for most common commands
%

\newcommand{\beq}{\begin{equation}}
\newcommand{\eeq}{\end{equation}}
\newcommand{\beqst}{\begin{equation*}}
\newcommand{\eeqst}{\end{equation*}}
\newcommand{\barr}{\begin{array}}
\newcommand{\earr}{\end{array}}
\newcommand{\beqar}{\begin{eqnarray}}
\newcommand{\eeqar}{\end{eqnarray}}
\newtheorem{theorem}{Theorem}[section]
%\newtheorem{conjecture}{Conjecture}
\newtheorem{corollary}[theorem]{Corollary}
%\newtheorem{problem}{Problem}
\newtheorem{lemma}[theorem]{Lemma}
\newtheorem{prop}[theorem]{Proposition}
\newtheorem{definition}[theorem]{Definition}
\newtheorem{remit}[theorem]{Remark}
\newtheorem{conjecture}[theorem]{Conjecture}

\newtheorem{example}[theorem]{Example}

\newcommand{\matr}[4]{\left \lbrack \begin{array}{cc} #1 & #2 \\
     #3 & #4 \end{array} \right \rbrack}



\newenvironment{rem}{\begin{remit}\rm}{\end{remit}}




% black board bold face
%note \AA is already defined!
\newcommand{\aff}{{ \Bbb A }}
\newcommand{\RR}{{{\bf  R }}}
\newcommand{\CC}{{{\bf  C }}}
\nc{\FF}{ {\Bbb F} } 
\newcommand{\ZZ}{{{\bf   Z }}}
\newcommand{\PP}{ {\Bbb P } }
\newcommand{\QQ}{{\Bbb Q }}
\newcommand{\UU}{{\Bbb U }}








%***************************




%


%


%Replace greek letters by their roman equivalents with \
%Slightly nonstandard:  theta is \t, tau is \ta, no omicron
\def\a{\alpha}
\def\b{\beta}
\def\g{\gamma}
\def\d{\delta}
\def\e{\epsilon}
\def\z{\zeta}
\def\h{\eta}
\def\t{\theta}
%\def\i{\iota}
\def\k{\kappa}
\def\l{\lambda}
\def\m{\mu}
\def\n{\nu}
\def\x{\xi}
\def\p{\pi}
\def\r{\rho}
\def\s{\sigma}
\def\ta{\tau}
\def\u{\upsilon}
\def\ph{\phi}
\def\c{\chi}
\def\ps{\psi}
\def\o{\omega}

\def\G{\Gamma}
\def\D{\Delta}
\def\T{\Theta}
\def\L{\Lambda}
\def\X{\Xi}
\def\P{\Pi}
\def\S{\Sigma}
\def\U{\Upsilon}
\def\Ph{\Phi}
\def\Ps{\Psi}
\def\O{\Omega}




% calligraphic letters
\newcommand{\calA}{{\mbox{$\cal A$}}}
\newcommand{\calB}{{\mbox{$\cal B$}}}
\newcommand{\calC}{{\mbox{$\cal C$}}}
\newcommand{\calD}{{\mbox{$\cal D$}}}
\newcommand{\calE}{{\mbox{$\cal E$}}}
\newcommand{\calF}{{\mbox{$\cal F$}}}
\newcommand{\calG}{{\mbox{$\cal G$}}}
\newcommand{\calH}{{\mbox{$\cal H$}}}
\newcommand{\calI}{{\mbox{$\cal I$}}}
\newcommand{\calJ}{{\mbox{$\cal J$}}}
\newcommand{\calK}{{\mbox{$\cal K$}}}
\newcommand{\calL}{{\mbox{$\cal L$}}}
\newcommand{\calM}{{\mbox{$\cal M$}}}
\newcommand{\calN}{{\mbox{$\cal N$}}}
\newcommand{\calO}{{\mbox{$\cal O$}}}
\newcommand{\calP}{{\mbox{$\cal P$}}}
\newcommand{\calQ}{{\mbox{$\cal Q$}}}
\newcommand{\calR}{{\mbox{$\cal R$}}}
\newcommand{\calS}{{\mbox{$\cal S$}}}
\newcommand{\calT}{{\mbox{$\cal T$}}}
\newcommand{\calU}{{\mbox{$\cal U$}}}
\newcommand{\calV}{{\mbox{$\cal V$}}}
\newcommand{\calW}{{\mbox{$\cal W$}}}
\newcommand{\calX}{{\mbox{$\cal X$}}}
\newcommand{\calY}{{\mbox{$\cal Y$}}}
\newcommand{\calZ}{{\mbox{$\cal Z$}}}

%% To load script letters:

\font\teneusm=eusm10  \font\seveneusm=eusm7 
\font\fiveeusm=eusm5 
\newfam\eusmfam 
\textfont\eusmfam=\teneusm 
\scriptfont\eusmfam=\seveneusm 
\scriptscriptfont\eusmfam=\fiveeusm 
\def\Scr#1{{\fam\eusmfam\relax#1}}

% Script letters
\newcommand{\ScrA}{{\Scr A}} \newcommand{\ScrB}{{\Scr B}}
\newcommand{\ScrC}{{\Scr C}} \newcommand{\ScrD}{{\Scr D}}
\newcommand{\ScrE}{{\Scr E}} \newcommand{\ScrF}{{\Scr F}}
\newcommand{\ScrG}{{\Scr G}} \newcommand{\ScrH}{{\Scr H}}
\newcommand{\ScrI}{{\Scr I}} \newcommand{\ScrJ}{{\Scr J}}
\newcommand{\ScrK}{{\Scr K}} \newcommand{\ScrL}{{\Scr L}}
\newcommand{\ScrM}{{\Scr M}} \newcommand{\ScrN}{{\Scr N}}
\newcommand{\ScrO}{{\Scr O}} \newcommand{\ScrP}{{\Scr P}}
\newcommand{\ScrQ}{{\Scr Q}} \newcommand{\ScrR}{{\Scr R}}
\newcommand{\ScrS}{{\Scr S}} \newcommand{\ScrT}{{\Scr T}}
\newcommand{\ScrU}{{\Scr U}} \newcommand{\ScrV}{{\Scr V}}
\newcommand{\ScrW}{{\Scr W}} \newcommand{\ScrX}{{\Scr X}}
\newcommand{\ScrY}{{\Scr Y}} \newcommand{\ScrZ}{{\Scr Z}}

%German (Faktur) letters

\newcommand{\grA}{{\frak A}}



\def\eps{\varepsilon}

\setlength{\textwidth}{6.5in}
\setlength{\textheight}{9.1in}
\setlength{\evensidemargin}{0in}
\setlength{\oddsidemargin}{0in}
\setlength{\topmargin}{-.75in}
\setlength{\parskip}{0.3\baselineskip}

%\renewcommand{\theequation}{\thesection.\arabic{equation}}
%\newcommand{\renorm}{{ \setcounter{equation}{0} }}

\newcommand{\Le}{{{\mathchoice{\,{\scriptstyle\le}\,}
  {\,{\scriptstyle\le}\,}
  {\,{\scriptscriptstyle\le}\,}{\,{\scriptscriptstyle\le}\,}}}}
\newcommand{\Ge}{{{\mathchoice{\,{\scriptstyle\ge}\,}
  {\,{\scriptstyle\ge}\,}
  {\,{\scriptscriptstyle\ge}\,}{\,{\scriptscriptstyle\ge}\,}}}}


%\renewcommand{\baselinestretch}{1.5}



\begin{document}
\title{Lecture 1:\\
 Reminder of basics of quantum mechanics\\
and canonical quantization in Hilbert space}
\author{Ludwig Faddeev}
\date{8  October 1996}

\maketitle
%\renorm

After hearing some statements on the nature of quantum
theory during last week I decided to start my lecture by
presenting a suitable and reasonably natural framework into
which $QM$ satisfactorily fits.
The framework itself is an abstraction of thoughts and
essays of many people.
I mention P. Dirac, H. Weyl, J. v. Neumann, I. Segal and G.
Mackey. 
I shall be very cryptic.

A description of physical reality is made in terms of two
sets of objects: observables and states.
A set of observables $A$, $B,\ldots$ will be denoted by $\grA$
and states $\omega$, $\mu,\ldots$ by $\Omega$.
Each state $\omega$ assigns to each observable $A$ its
probability distribution $\omega_A(\lambda)$ on a real line.
The pairing (mean value)
$$
\left<\omega\vert A\right>=\int\lambda d\omega_A(\lambda)
$$
defines a duality between $A$ and $\Omega$.
It is natural to assume the separability (completeness) of
states: two observables with equal pairings in all states
are equal.
This immediately introduces a real linear structure on $\grA$,
whereas $\Omega$ is a convex set. The
next structure is a notion of functional dependence of
observables.
The observable $B$ is a function $f(A)$ of the observable $A$ if in
any state
$$
\left<\omega\vert B\right>=\int f(\lambda)
d\omega_A(\lambda).
$$
This definition of a function is rather unwieldy, so a
technical assymption is added that $A$ lies in an algebra
with product $AB$ such that the notion of function is
compatible with this product.

Now comes dynamics: each observable generates a motion of
$\grA$, that is a one-parameter group of automorphisms of $\grA$
via the differential equation
$$
dA/db=\{B,A\},
$$
where $\{\,\,,\,\,\}$ is a bracket, which is necessarily of
Lie type and is a derivation of the product
$$
\{A,BC\}=\{A,B\}C+B\{A,C\}.
$$
This essentially finishes the description of the framework.

Both classical and quantum mechanics fit perfectly in it and
give concrete and nonequivalent realizations.

All features of $CM$ are reconstructed as soon as one
assumes that $\grA$ is commutative.
It follows that $\grA$ can be realized as an algebra of
functions on a symplectic manifold (phase space) $M$:  states
are normalized measures on the phase space and
$\{\,\,,\,\,\}$ is a Poisson bracket, defined by the
symplectic structure on it.
In particular
$$
\omega_A(\lambda)=\int\theta\left(\lambda-f_A(p,q)\right)
d\mu_\omega,
$$
where $\theta(t)$ is a step function, $f_A(p,q)$ is a
function corresponding to observable $A$ and $\mu_\omega$ is
a measure corresponding to the state $\omega$.
Pure (extremal) states are defined by atomistic measures
concentrated in one point of $M$.
All observables have exact values (zero dispersions) in all
pure states.

In the case of $QM$, the set $\grA$ consists of real elements of
a complex associative algebra with involution, which is
usually realized as an algebra of linear operators in some
complex Hilbert space $H$.
The states are the positive trace class operators with unit
trace.
The distribution $\omega_A(\lambda)$ is given by
$$
\omega_A(\lambda)=\tr\left(MP_A(\lambda)\right),
$$
where $M$ (density matrix) defines the state and
$P_A(\lambda)$ is a special projector function of operator
$A$.
The dynamical bracket is defined in terms of product
$$
\{A,B\}=(i/\hslash)(AB-BA),
$$
where $i=(-1)^{1/2}$ and the normalizing factor $1/\hslash$
here employs the famous Planck constant $\hslash$.
Its introduction in this place is necessitated just from
dimensional considerations.

The pure states are given by one-dimensional projectors and
a given observable is exact (dispersionless) only in pure
states, defined by its eigenvectors.
This feature, which is reflected in the ``uncertainty''
inequality
$$
\Delta_\omega A\Delta_\omega B\Ge
(\hslash/2)\vert\left<\omega\vert\{A,B\}\right>\vert,
$$
where
$$
\Delta_\omega A=(\left<\omega\vert(A-\left<\omega\vert
A\right>)^2\right>)^{1/2}
$$
is the main physical distinction between $QM$ and $CM$.
By no means does it preclude the measurability of observables
in $QM$.
Indeed, the collection of eigenstates is enough to
characterize any observable.
On the contrary, the exactness of observables in all pure
states in $CM$ is a kind of unnecessary luxury.

It is instructive to compare the CM and QM on some common
playing field. We shall consider the simplest mechanical
 system with {\em one degree of freedom}: in
other words the  phase space is $M
 = \RR^2$, and points in 
phase space are denoted $(p,q)$. The Hilbert space is 
$$\calH = L^2 (\RR)  = \{ \psi(q) \in \CC: \int_{q = - \infty}^\infty
|\psi(q)|^2 dq < \infty \}.  $$
Quantum mechanically we replace the basic observables
({\em canonical} coordinates) 
$p$ and $q$ by operators $P$ and $Q$ acting on $\calH$: these are 
defined by 
\beq (P  \psi)(q) = \frac{\hbar}{\isq} \frac{d}{dq} \psi(q), \eeq
\beq (Q  \psi)(q) = q \psi(q); \eeq
this is the so called {\em coordinate realization}. 
An observable $A$ is characterized as an integral operator 
defined by its {\em kernel}
\beq 
(A \psi)(q'') = \int_{- \infty}^\infty 
A (q'', q') \psi (q') dq'. \eeq
Alternatively we may define the observable $A$ by means of its {\em symbol},
which is a function  on the phase space $M$: the symbol $f$ corresponds
to the integral kernel  $A_f$ by means of the {\em Weyl quantization}
\beq A_f (q'', q') = 
\frac{1}{2 \pi \hbar} \int_{- \infty}^\infty 
f(p, \frac{q'' +  q'}{2} ) e^{i p (q'' - q')/\hbar} dp. \eeq
There is no unique way to substitute operators for functions 
because the function $pq$ is equal to the function $qp$ while the 
operator $PQ$ does not equal the operator $QP$: one must choose a 
prescription for the 
 ordering of  the operators. 
The Weyl quantization corresponds to one choice of ordering, which 
is to completely symmetrize over the operators $P$ and $Q$. 
The Weyl quantization is not a homomorphism from the algebra of  functions 
to the algebra of operators: if one associates the operators $A_f$ and 
$A_g$ to the functions
$f$ and $g$ respectively, one may define the kernel of the product 
$A_f A_g$ by 
\beq \label{1.8} A_f A_g (q'', q') =  \int_{- \infty}^\infty
A_f (q'', q) A_g (q, q') dq. \eeq
This corresponds to a product $f \star_\hbar g $ on the algebra of 
functions which equals 
\beq \label{1.9} 
f \star_\hbar g = f g + \frac{\isq \hbar}{2} \{ f, g \} + O (\hbar^2), \eeq
%
%
%\newpage\mbox{}\clearpage
%
so that
$$
(i/\hslash)(f*g-g*f)=\{f,g\}+O(\hslash).
$$
We see that both $CM$ and $QM$ can be realized in terms of
the same objects (functions) but the structure constants of
the main operations (product, Lie bracket) of $QM$ are
defined as a series in positive powers of $\hslash$, the
zero order term being the structure constants of $CM$.
So we can say that $QM$ is a deformation of $CM$ in the
framework described above.
The Planck constant is a corresponding deformation
parameter.
This is for me the most concise formulation of the
correspondence principle and explains what is meant by
quantization.

Now comes the notion of equivalent deformation and that of
stability.
Beautiful results, which I learned from A. Lichnerowicz, M.
Flato and D. Sternheimer, allow one to say that $CM$ is
unstable and that $QM$ is essentially a unique deformation
of it into a nonequivalent stable structure.
The degeneracy of $CM$ (exactness of pure states) is
intimately connected with its instability.
So it is only natural that the passage from $CM$ to $QM$ was
prompted by the experimental activity of physicists.
The stability of $QM$, on the contrary, shows that an
analogous modification of it is less feasible.
In this case one has first to modify also all the general
framework, as emphasized above.

Now I turn to discussion of dynamics.
The goal is to make control of the evolution operator
$$
U(t)=e^{-iHt/\hslash},
$$
which defines the solution of the 
dynamical equation 
\beq \frac{dA(t)}{dt} = \{H, A(t) \}_{\hbar}, ~~~A(t)|_{t = 0 } = A \eeq
in the form 
\beq A(t) = U^{-1} (t) A U(t). \eeq
{}From now on we take the normalization $\hbar = 1$. 



%*************************












%more newcommands
\nc{\bra}{  < }
\nc{\ket}{ > }
%\nc{\isq}{ { \sqrt{-1} } }
%\nc{\isq}{{ i }}
%\nc{\hbar}{{ h}}
\nc{\triang}{ { \bigtriangleup} }

%\renewcommand{\theequation}{\thesection.\arabic{equation}}
%\newcommand{\renorm}{{ \setcounter{equation}{0} }}

It is convenient to consider arbitrary initial and final times, so we
want to 
construct the kernel of the evolution operator 
\beq \label{1.09} U = e^{- i H (t'' - t')}  \eeq
in terms of the symbol corresponding to the Hamiltonian 
$h (p,q)$. Because Weyl quantization is not a homomorphism taking 
the product of functions to the product of operators, one cannot 
substitute the kernel for the Hamiltonian into the Taylor series for the 
exponential. Nonetheless if $\triang$ is a small interval, one may make the 
approximation
\beq \label{1.10} U_\triang = e^{- i H \triang} \cong 1 - i H \triang \eeq
and then if $ t'' - t' = N \triang $ one may approximate 
\beq \label{1.11} e^{ - i H (t'' - t')} = (U_\triang)^N. \eeq
We write the integral kernel for $U_\triang$ as 
\beq \label{1.12} U_\triang (q'', q') = \frac{1}{2 \pi} \int_{- \infty}^\infty
dp e^{i p (q'' - q') } \Bigl (1 - i h (p, \frac{q'' + q'}{2}
 ) \triang \Bigr  )  \eeq
$$ = \frac{1}{2 \pi} \int_{- \infty}^\infty dp  e^{i p (q'' - q')  - i 
 h (p, \frac{q'' + q'}{2}) \triang }  + O (\triang^2). $$

To obtain the integral kernel for $e^{ - i H (t'' - t')}$ we 
compose $N$ copies of $U_\triang$  and integrate over the 
$N-1$ intermediate variables $q_j$ using (\ref{1.8}):
\beq \label{1.13} U (q'', q'; t'' - t') = 
\int \dots \int e^{  i p_N (q_N - q_{N-1} ) + i p_{N-1} 
(q_{N-1} - q_{N-2} ) + \dots + i p_1 (q_1 - q_0) }   \times \eeq
$$ e^{ - i h (p_N, \frac{q_N + q_{N-1} }{2} ) 
- \dots - i h (p_1, \frac{q_1 + q_0}{2}) } \frac{dp_1 dq_1} {2 \pi} 
\dots \frac{dp_{N-1} dq_{N-1} } {2 \pi} \frac{dp_N}{2 \pi}. $$
In the limit as $N$ tends to infinity while $q_N - q_{N-1} $ is 
proportional to $1/N$, we formally obtain an integral 
over trajectories $(p(t), q(t))$ in phase space:
\beq \label{1.14} U(q'',q'; t'' - t')  = 
\int  \exp i \left \{ \int_{t'}^{t''} (p(t) \dot{q} (t) - h (p(t), q(t) )) dt
\right \}
 \prod_{t} \frac{ dp(t) dq(t)}{2 \pi} 
. \eeq

One recognizes
\beq \label{action} 
  \int_{t'}^{t''} \left ( p(t) \dot{q} (t) - h (p(t), q(t) \right ) dt \eeq
as the {\em action functional}, while
$\prod_{t} \frac{ dp(t) dq(t)}{2 \pi} $ is the Liouville measure on the 
space of paths. 
 
A {\em real polarization} of a symplectic manifold or phase space of 
dimension $2n$
is a surjective map $\pi$ to a manifold of dimension $n$ such that
the fibres are Lagrangian submanifolds (in other words the fibres 
are of dimension $n$ and the symplectic form restricts to zero on them). 
The boundary 
condition on the trajectories in (\ref{1.14}) 
is  that $q (t'') = q', q(t') = q'$ 
while $p(t'') $ and $p(t')$ are unconstrained.  This condition 
says that when $t $ takes the extremal value $t'$ the endpoint
of the  path $(p(t), q(t)) $ 
lies in a prescribed fibre $L' \eqdef \pi^{-1} (q') $
 of the  real polarization
\beq \pi: (p,q) \mapsto q \eeq
of the phase space, while at the other extremal value $t = t''$ 
the path lies in a different fibre  
$$L'' = \pi^{-1} (q'') $$  
of the polarization. 
%The kernel should be thought of as a half density on the product of the 
%two Lagrangian submanifolds $L' \times L''$, so that products of two 
%such kernels can be integrated over $L' \times L''$. 

\noindent{\em Remark:} In our  example, a phase space arising
from a  physical system with one degree of freedom is $\RR^2$. Other
interesting examples arise where the phase space is $S^2$. 
No real polarization exists in those cases.
Complex polarization leads to finite dimensional
represetnation of the group $O(3)$, so that this
phase space corresponds to a spin degree of freedom.

If the Hamiltonian is given by 
\beq h (p,q) = \frac{1}{2} p^2 + v(q), \eeq
the change of variables
\beq p(t)  \mapsto p'(t)  = p(t)  + \dot{q}(t), ~~~ q(t) \mapsto  q(t) \eeq
transforms the integrand in  (\ref{action}) to a form in which 
no terms appear which involve both $p(t) $ and $q(t)$ (in other 
words these variables separate):
\beq \label{newaction}
p' \dot{q'} 
- h (p', q') = -\frac{1}{2} (p')^2 + \frac{1}{2} (\dot{q})^2 - v(q) \eeq
We may then integrate over $p(t) $ (to obtain 
a constant which is independent of $q(t)$) and recover the more usual form
of
 the action which depends only on $q(t)$:
\beq U(q'',q'; t'' - t') = {\rm const} \int \exp \{ \isq \int_{t'}^{t''}
( \half \dot{q}^2 - v(q) ) dt \} \prod_{t} dq(t) \eeq


 Here what appears is   the usual Lagrangian,
which depends on $q$ and $\dot{q}$ but not on $p$. 
This change of variables thus transforms the formula (\ref{1.14})
for the kernel of the time evolution operator  $U$ into a path 
integral over trajectories in {\em configuration} space (the space 
of possible values of the position) rather than over trajectories
in {\em phase} space (the space of possible values of position and 
momentum, whose dimension is twice that of the configuration 
space). We remark that this method of reducing from paths in phase space 
to paths in configuration space works only when the action is 
quadratic in the momentum $p$. However, more general actions 
quadratic in $p$ can also be treated by this method: for instance,
one may use a general Riemannian metric on the configuration space
rather than the flat metric $p^2$, which simply gives rise to a factor 
of the Riemannian volume element.

\nc{\astar}{{ a^*}}
\nc{\hata}{ { \hat{a} }}
\nc{\hatastar}{{ \hat{\astar} }}
\nc{\normfac}{{\frac{1}{\sqrt{2 \omega} } }}

Let us now consider an example where the form of the potential 
$v(q)$ is given explicitly: the {\em harmonic oscillator} 
\beq \label{harmosc} 
h(p,q) = \half p^2 + \half \omega^2 q^2. \eeq
We intentionally keep here the frequency parameter $\omega$. This
will be instructive when we turn to the free fields, which 
are nothing but a collection of oscillators with different 
frequencies.
The classical equation of motion corresponding to this Hamiltonian 
is the equation for simple harmonic motion
\beq \label{shm}
\frac{d^2}{dt^2} q(t)  + \omega^2 q(t) = 0. \eeq
In the approach to quantization which 
we have described, one finds that
the eigenvectors of the Hamiltonian are 
defined in terms of certain
orthogonal polynomials in $q$, the Hermite polynomials. 


The use of special functions can be avoided if one
alternatively considers
the {\em holomorphic} quantization of the harmonic oscillator:
this will complement the 
treatment given in K. Gawedski's lectures. 
We introduce complex coordinates
$a$ and $\astar$ on the phase space, defined by 
\beq  a= \normfac (\omega q + \isq p) \eeq
\beq \astar = \normfac (\omega q - \isq p) \eeq
so that
\beq q = \normfac (\astar + a). \eeq
After quantization, $a$ and $\astar $ become operators $\hata$ and 
$\hatastar$, which satisfy the commutation   relation
\beq [\hata, \hatastar ] = 1. \eeq

The Hilbert space $\calH$ now consists of entire functions in 
$\astar$: it is the closure of the space of polynomials
in $\astar$ in the norm 
\beq \label{normastar}
(f,g) = \int_{\RR^2} 
 \bar{f}(a) g(\astar) e^{ - \astar a} \frac{d\astar da}
{2 \pi \isq}. \eeq %****
One finds for instance that the elements 
\beq \psi_n = \frac{(\astar)^n}{\sqrt{n!} } \eeq
in $\calH$ are orthonormal with respect to the norm $( \cdot, \cdot )$. 
The operators $\hata$ and $\hatastar$ are represented on 
$\calH$ by 
\beq \hata f(\astar)  = \frac{d f(\astar) }{d \astar}, \eeq
\beq \hatastar f (\astar)  = \astar \cdot f (\astar). \eeq
The Hamiltonian may be written as 
\beq H = \omega \hatastar \hata, \eeq
and it has the $\psi_n$ as eigenvectors: we have
\beq H \psi_n = n \omega \psi_n.  \eeq
In particular, we have $\psi_0 = 1$, for which $H \psi_0 = 0 $; 
one thus sees that 
\beq \psi_n = \frac{(\hatastar)^n} {\sqrt{n!} } \psi_0. \eeq
The operators on $\calH$ may be represented by integral 
kernels $A((\astar)'', a')$, which are analytic functions
of two independent variables $(\astar)''$ and $a'$. 
The composition  of two operators corresponds to 
the following operation on their kernels $A$ and $B$:
\beq (AB)((\astar)'', a') = \int A ((\astar)'', \alpha) B (\alpha, a') 
e^{- \alpha^* \alpha} \frac{d \alpha^* d \alpha} {2 \pi \isq}. \eeq
Occasionally we shall drop the primes on $\astar$ and $a$ in cases when 
this does not lead to confusion.

Another possibility is to use {\em normal symbols}: 
if we write the operator $A$ as 
\beq A = \sum_{n,m} K_{mn} (\hatastar)^n \hata^m, \eeq
then the corresponding normal symbol is 
\beq K(\astar, a) = \sum_{n,m} K_{mn}(\astar)^n a^m. \eeq
It is an exercise to see that 
\beq K(\astar, a) = e^{ - \astar a} A (\astar, a). \eeq
If we now turn to dynamics,
the kernel of the evolution operator
 for a small time interval $\triang $ may be 
written in terms of the normal symbol of the 
Hamiltonian $h(\astar, a)$ as
\beq U_{\triang } = e^{- \astar a - i h(\astar, a) \triang}. \eeq
To write the time evolution operator for the time interval $t'' - t'$, we 
compose $N$ copies of $U_{\triang }$ as before: 
\beq \label{intker} U(\astar, a) = 
\int \dots \int e^{ a_N^* a_{N-1}
- a_{N-1}^* a_{N-1} + a_{N-1}^* a_{N-2} - \dots 
- a_1^* a_1 + a_1^* a_0}  \times \eeq
$$ e^{- \isq 
\{  h(a_N^*, a_{N-1} ) + \dots
+ h (a_1^*, a_0) \}  \triang } \prod_{k =1}^{N-1} \frac{da_k^* da_k}
{2 \pi \isq}. $$
In this context the analogue 
of the condition that $q_N$ and $q_0$ are fixed is that
$a_N$ and $a_0$ are fixed:
\beq \label{bound}  a_N = \astar, ~~~ a_0 = a. \eeq
In the limit as $N \to \infty$ we obtain 
\beq \label{upathint:ho} U = \int \exp \left \{ \astar a|_{t = t''}   + 
\int_{t'}^{t''} \{ - \astar \dot{a} - \isq h(\astar, a) \} dt \right \} 
\prod_{t' \le t \le t''}  \frac{ d\astar(t) da(t)}{2 \pi \isq}. \eeq
For values of $t$ that are not equal to the extremal values
$t' $ and $t''$, we assume that $\astar(t) =  \overline{a(t)}$, but we 
do {\em not} assume this on the boundary: thus we impose the condition 
that $\astar(t'') = \astar \in \CC$ (this is the value 
that was denoted by  $\astar$ in (\ref{bound})) and similarly  $\astar(t)
= a \in \CC$ (this is the value that was denoted by $a$ in (\ref{bound})).
The variables $\astar, a$ are the arguments of the integral 
kernel in (\ref{intker}):
they are arbitrary complex numbers, in particular we do {\em not} 
assume that $\astar = \overline{a}$.



If one now substitutes $h = \omega \astar a$, the integral is Gaussian;
using the equations of motion 
\beq \dot{a} + \isq \omega a = 0, ~~~ \dot{a}^* - \isq \omega \astar = 0\eeq
with the boundary conditions $a_N = \astar$, $a_0 = a$ 
(where $\astar$ and $a$ are arbitrary complex numbers which are not 
required to be complex conjugate) we then see that 
\beq a(t) = a e^{- i \omega (t - t')}, \eeq
\beq \astar (t) = \astar e^{i \omega (t - t') }. \eeq
Note that $a(t)$ and $\astar (t)$ are not complex conjugates of each other.
Substituting in (\ref{upathint:ho}) we obtain
\beq U_0(\astar, a, t'' - t') = 
\exp \left \{ \isq \astar a e^{-\isq \omega (t'' - t') } \right \}. \eeq
Applying this to an element $\phi$ of $\calH$, we find
\beq (U_0 \phi)(\astar) = 
\int \exp \left ( \astar \alpha e^{-\isq \omega (t'' - t') }\right )
\phi (\alpha^*) e^{ - \alpha^* \alpha} \frac{d \alpha d \alpha^*} 
{2 \pi \isq} \eeq
$$ = \phi (e^{-i \omega t} a^* ). $$
This shows that the evolution operator is obtained by simply 
substituting the same argument rotated {\em backward} in time. 
%The computation can be performed by expanding $\phi$ in polynomials, and 
%by using the identity that 
%\beq \int e^{(\astar - \alpha^*) a}da  = \delta (\astar^* - \alpha^*). \eeq













%\begin{thebibliography}{99}
%\end{thebibliography}


\end{document}


