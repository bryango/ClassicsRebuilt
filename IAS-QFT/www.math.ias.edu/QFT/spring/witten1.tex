%Date: Fri, 14 Feb 1997 16:48:24 -0500 (EST)
%From: Pavel Etingof <etingof@abel.MATH.HARVARD.EDU>

\input amstex
\documentstyle{amsppt}
\magnification 1200
\NoRunningHeads
\NoBlackBoxes
\document

\def\h{\frak h}
\def\tW{\tilde W}
\def\Aut{\text{Aut}}
\def\tr{{\text{tr}}}
\def\ell{{\text{ell}}}
\def\Ad{\text{Ad}}
\def\u{\bold u}
\def\m{\frak m}
\def\O{\Cal O}
\def\tA{\tilde A}
\def\qdet{\text{qdet}}
\def\k{\kappa}
\def\RR{\Bbb R}
\def\be{\bold e}
\def\bR{\overline{R}}
\def\tR{\tilde{\Cal R}}
\def\hY{\hat Y}
\def\tDY{\widetilde{DY}(\g)}
\def\R{\Bbb R}
\def\h1{\hat{\bold 1}}
\def\hV{\hat V}
\def\deg{\text{deg}}
\def\hz{\hat \z}
\def\hV{\hat V}
\def\Uz{U_h(\g_\z)}
\def\Uzi{U_h(\g_{\z,\infty})}
\def\Uhz{U_h(\g_{\hz_i})}
\def\Uhzi{U_h(\g_{\hz_i,\infty})}
\def\tUz{U_h(\tg_\z)}
\def\tUzi{U_h(\tg_{\z,\infty})}
\def\tUhz{U_h(\tg_{\hz_i})}
\def\tUhzi{U_h(\tg_{\hz_i,\infty})}
\def\hUz{U_h(\hg_\z)}
\def\hUzi{U_h(\hg_{\z,\infty})}
\def\Uoz{U_h(\g^0_\z)}
\def\Uozi{U_h(\g^0_{\z,\infty})}
\def\Uohz{U_h(\g^0_{\hz_i})}
\def\Uohzi{U_h(\g^0_{\hz_i,\infty})}
\def\tUoz{U_h(\tg^0_\z)}
\def\tUozi{U_h(\tg^0_{\z,\infty})}
\def\tUohz{U_h(\tg^0_{\hz_i})}
\def\tUohzi{U_h(\tg^0_{\hz_i,\infty})}
\def\hUoz{U_h(\hg^0_\z)}
\def\hUozi{U_h(\hg^0_{\z,\infty})}
\def\hg{\hat\g}
\def\tg{\tilde\g}
\def\Ind{\text{Ind}}
\def\pF{F^{\prime}}
\def\hR{\hat R}
\def\tF{\tilde F}
\def\tg{\tilde \g}
\def\tG{\tilde G}
\def\hF{\hat F}
\def\bg{\overline{\g}}
\def\bG{\overline{G}}
\def\Spec{\text{Spec}}
\def\tlo{\hat\otimes}
\def\hgr{\hat Gr}
\def\tio{\tilde\otimes}
\def\ho{\hat\otimes}
\def\ad{\text{ad}}
\def\Hom{\text{Hom}}
\def\hh{\hat\h}
\def\a{\frak a}
\def\t{\hat t}
\def\Ua{U_q(\tilde\g)}
\def\U2{{\Ua}_2}
\def\g{\frak g}
\def\n{\frak n}
\def\hh{\frak h}
\def\sltwo{\frak s\frak l _2 }
\def\Z{\Bbb Z}
\def\C{\Bbb C}
\def\d{\partial}
\def\i{\text{i}}
\def\ghat{\hat\frak g}
\def\gtwisted{\hat{\frak g}_{\gamma}}
\def\gtilde{\tilde{\frak g}_{\gamma}}
\def\Tr{\text{\rm Tr}}
\def\l{\lambda}
\def\I{I_{\l,\nu,-g}(V)}
\def\z{\bold z}
\def\Id{\text{Id}}
\def\<{\langle}
\def\>{\rangle}
\def\o{\otimes}
\def\e{\varepsilon}
\def\RE{\text{Re}}
\def\Ug{U_q({\frak g})}
\def\Id{\text{Id}}
\def\End{\text{End}}
\def\gg{\tilde\g}
\def\b{\frak b}
\def\S{\Cal S}
\def\L{\Lambda}

\topmatter
\title Lecture II-1: Symmetry breaking
\endtitle
\author {\rm {\bf Edward Witten} }\endauthor
\endtopmatter

\centerline{Notes by Pavel Etingof and David Kazhdan}

\vskip .1in

In this semester we will continue the discussion of quantum field theory, 
but now mostly dynamics rather than perturbation theory and purely formal 
things. The topics we will discuss are much more difficult to deal with
rigorously than the formal theory we studied in the fall semester. We will
try to do it when possible, but it will not always be possible. 
In general, we will try to form an intuitive picture of what is going on, 
and illustrate it by considering concrete examples. 


{\bf 1.0. Theories and realizations.} [This section is explanatory 
and was written by the 
preparers of these notes in order to create a false sense of security
i.e. an unsubstantiated feeling that we understand what we are talking
about in the rest of the lecture. Unfortunately, this is not the case, 
at least if ``understand'' means what it usually means among mathematicians.]

For the purposes of the present and forthcoming lectures, 
it will be important for us to distinguish theories and their realizations. 
So let us explain what we mean by a theory and what we mean by its 
realization. This explaination is not a mathematical 
definition (in fact, it is hard
to give a definition which is both rigorous and useful), but we hope that 
it will make clear what we are talking about. 

Recall that a classical physical system $\Sigma$ is usually described 
by defining the space of states $X$ of $\Sigma$ (a symplectic manifold, maybe
infinite-dimensional) and a 1-parameter group $g^t$ of time translations 
which preserves symplectic structure. Then $g^t$ 
produces a Hamiltonian flow, which is defined   
(ignoring topological problems)
by a Hamiltonian function $H$. This function is called the Hamiltonian,
or energy function of the system. One should remember that $H$ is defined 
only up to adding a (locally) constant function. 

In most examples, $H$ is bounded from below. In such a case, $H$ is always
normalized in such a way that the infinum of $H$ on (each connected 
component of) $X$ is zero. 

In this situation, by {\bf a theory} we mean a pair $(X,H)$, 
where $X$ is a symplectic manifold, and $H:X\to\R$ the energy function,
defined up to a (locally) constant function. 
By {\bf a vacuum state of this theory} we mean a lowest energy 
equilibrium state
$x\in X$ of the system $\Sigma$, 
i.e. a state where the energy functional $ H$
attains a global minimum. 

The same theory can have different vacuum states. For example, 
if we have a particle on the line with potential energy
$U(x)=\frac{g}{4!}(x^2-a^2)^2$, 
then the space $X$ is the phase plane $\R^2$ with 
coordinates $(x,p)$, the Hamiltonian is $\frac{p^2}{2}+U(x)$, 
and there are two vacuum states $(a,0)$ and $(-a,0)$. 

Often a classical system $\Sigma$ can be described 
by a Lagrangian $\Cal L$, defined on some
space of fields $S$ on the spacetime $V$. 
In this case its space of states is 
the space $X\subset S$ of extremals of $\Cal L$. 
As we have seen before, the space $X$ carries a 
natural closed 2-form $\omega$, which is nondegenerate, and 
thus defines a symplectic structure on $X$. Also, the group of time 
translations acts on $X$ and preserves $\omega$. Therefore, 
the flow on $X$ generated by this 1-parameter group 
is Hamiltonian, and is defined by a Hamiltonian function $H$. 

However, one should remember that 
the description of a theory by a Lagrangian is not intrinsic, 
since different Lagrangians defined on different spaces of fields $S$
may define the same theory. Indeed, consider, for example, 
the Lagrangian $\Cal L_1=\int x'(t)^2dt/2$ defined on 
$S_1=C^\infty(\R,M)$, where $M$ is a Riemannian manifold. 
This Lagrangian defines the geodesic flow. The space of states
in this case is $T^*M$, and the Hamiltonian function is $p^2/2$. 
On the other hand, we can write the Lagrangian 
$\Cal L_2=\int (-x'(t)p(t)+p^2(t)/2)dt$ defined on 
the space $S_2=C^\infty(\R,T^*M)$. It is easy to see 
that these two Lagrangians define the same theory. 

We will consider relativistically invariant theories $(X,H)$, i.e.
theories with an action of the Poincare group $\bold P$ on 
$X$, which preserves the symplectic structure
(in the case when the system is defined by a Lagrangian, 
the group $\bold P$ acts on $S$ and preserves the Lagrangian, and therefore, 
it acts on $X$ preserving the Poisson structure). The group of time 
translations is a subgroup of $\bold P$, so $\bold P$ also preserves 
the energy function $H$. 
In this case, when we talk about vacuum states of the theory $(X,H)$, 
we mean a vacuum state invariant under $\bold P$.  

Recall that to define a symplectic manifold $X$ is the same thing as to 
define the Poisson algebra $A=C^\infty(X)$
($X$ can be reconstructed as the spectrum of $A$). Therefore, we may say that
a theory is a pair $(A,H)$, where $A$ is a Poisson algebra and $H\in A$. 
The Poisson algebra $A$ is called the algebra of observables of the system. 

Now let us consider quantum systems. The definition of
a theory in this case is similar to the classical case. 
Namely, we will define {\bf a quantum theory} to be a pair $(A,H)$, 
where $A$ is a *-algebra (not necessarily commutative), and 
$H$ is a selfadjoint element of $A$, defined up to adding 
a real number. The algebra $A$ is called the 
algebra of quantum observables (operators). 
The element $H$, as before, is called the 
Hamiltonian. Everything here is dependent on a real positive parameter 
$\hbar$ (the Planck constant). 

By {\bf a realization (or solution)} of
a quantum theory $(A,H)$ we will mean an irreducible *-representation  
of the algebra $A$ in some Hilbert space $\Cal H$,
such that the spectrum of the operator $H$ is bounded from below 
(representations are considered up to an isomorphism which preserves $H$). 
We will always normalize $H$ so that the lowest point of its spectrum is zero. 
The space $\Cal H$ is called the quantum space 
of states of the system (in this realization). Of course, as in the 
classical case, the same theory can have different realizations, 
as the same algebra can have different representations. 

As we have mentioned, we will be interested in relativiatically 
invariant theories, i.e. theories with an action of the Poincare group 
on $A$, so that the subgroup of time translations acts by 
$a\to e^{itH/\hbar}ae^{-itH/\hbar}$. 
When we talk about realizations of such a theory, 
we will assume that $\bold P$ acts in $\Cal H$ by unitary operators, with the group 
of time translations acting by $e^{itH/\hbar}$. 

By a vacuum state we mean a vector $\Omega\in \Cal H$ such that 
$Hv=0$. In a relativistically invariant situation, a vacuum state
is the same thing as a $\bold P$-invariant vector. 

{\bf Remark.} In general, as we will see, an irreducible realization of
a theory can have many vacuum states. Therefore, the notions
of a realization and of a vacuum state are not equivalent. However, 
if the algebra of observables is commutative
( i.e. in the classical theory), each irreducible representation 
of this algebra is 1-dimensional, and there is no real difference 
between the notions of a realization and a vacuum state. 
Therefore, the word ``realization'' is not usually used 
when one refers to the classical theory. 

Suppose that we have a classical theory $(A_0,H_0)$ which has 
been quantized, and the corresponding 
$\hbar$-dependent family of quantum theories is $(A,H)$
(here by an $\hbar$-dependent family we mean a family depending
on the dimensionless parameter $\hbar/S_0$, where $S_0$ is a   
characteristic scale of
action).
This means that we have a quantization map --
some linear map $A_0\to A$, given by 
$a\to \hat a$, such that $\hat H_0=H$, and 
$[\hat a,\hat b]=i\hbar\widehat{\{a,b\}}+o(\hbar)$, $\hbar\to 0$.   
In this case, we will say that a state $v\in \Cal H$ of norm $1$ 
is localized near a classical solution $x\in X=\text{Spec} A_0$ 
if for any $a\in A_0$ $\<v,\hat av\>\to a(x)$, $\hbar \to 0$.

Let us now explain the connection between the classical and the quantum 
notions of a vacuum state. Suppose we have a quantum vacuum state
$\Omega$ of norm $1$ which is localized near a classical state $x$. 
In this case $x$ is a classical vacuum state. 
Indeed, $H_0(x)=\lim_{\hbar\to 0}\<\Omega,H\Omega\>=0$, and
for any $F\in A_0$ $\{F,H_0\}(x)=\lim_{\hbar\to 0}\frac{1}{i\hbar}
\<\Omega,[\hat F,H]\Omega\>=0$, so $x$ is a stationary point of $H_0$. 

{\bf Remark.} Note any quantum vacuum state is localized near a classical 
vacuum state. Sometimes a quantum vacuum state is ``spread'' with some density over 
the set of classical vacuum states. We will see examples of this in today's lecture. 

Given a quantum theory $(A,H)$, it is convenient to represent its
realizations by correlation functions. Namely, given a realization
$\Cal H$ of this system, and a vacuum state $\Omega\in \Cal H$, 
we can define correlation functions $\<\Omega, L_1...L_n\Omega\>$, 
where $L_i\in A$. Since the action of $A$ in $\Cal H$ is irreducible, 
the realization $\Cal H$ can be completely reconstructed from these 
correlation functions. 

{\bf Remark.} The irreducibility condition is not always satisfied in 
physically interesting examples. But here for simplicity we will assume
that it is satisfied. 

Sometimes a quantum system can be defined by a Lagrangian. 
Of course, as in the classical case, this is not always possible, and if 
possible, not in a unique way. However, 
such a presentation is very convenient for understanding the behaviour of the 
system. So let us explain (on examples) how to pass from a 
Lagrangian to the Hamiltonian  and the operator algebra. 

We will start with the case of quantum mechanics, when the spacetime is 
just the time line. Consider a Lagrangian for one boson:
$$
\Cal L=\int[\frac{(\phi')^2}{2}-U(\phi)]dt.
$$
Then, by definition, the operator algebra is generated by 
operators $\phi_0,\phi_0',H$, with the canonical commutation relations 
$$
[\phi_0,\phi_0']=i\hbar,\ [H,\phi_0]=-i\hbar \phi_0',\ [H,\phi_0']=
i\hbar U'(\phi_0).
$$
Define the local operators $\phi(t)=e^{iHt/\hbar}\phi_0
e^{-iHt/\hbar}$, $\phi'(t)=e^{iHt/\hbar}\phi_0'
e^{-iHt/\hbar}$.
It follows from the above definition 
that $d\phi/dt=\phi'$, and $\phi$ satisfies the
Newton's differential equation
$$
\phi''(t)=-U'(\phi(t)).
$$
The operators of the form $F(\phi(t),\phi'(t))$, where $F$ is a 
polynomial, are called local operators at $t$
(we order products in such a way that 
$\phi'$ stands on the right from $\phi$).

The operator algebra is spanned (topologically) 
by operators $\phi(t_1)...\phi(t_n)$. Thus, a realization of the theory 
is defined by prescribing expectation values of these operators --
the correlation functions. 

In any realization of the theory, 
the Hamiltonian is given by the following explicit formula:
$$
H=\frac{1}{2}(\phi')^2+U(\phi)+C.
$$
Indeed, the difference of the left and the right hand sides of this equation
commutes with $\phi$ and $\phi'$, so by irreducibility it acts by 
a scalar.

Now consider quantum field theory. We first consider the theory 
in a spacetime $V=L\times \R$, where $L$ is a lattice (finite or infinite). 
Let $\nabla_L$ be the discrete gradient operator on the lattice. 
Consider a Lagrangian
$$
\Cal L=\sum_{x\in L}\int dt[\frac{1}{2}(\phi_t^2(x,t)-(\nabla_L\phi(x,t))^2)-
U(\phi(x,t))]
$$
In this case the operator algebra 
is generated by operators $\phi(x,0), \phi_t(x,0)$, $x\in L$, and $H$, 
satisfying the commutation relations 
$$
\gather
[\phi(x,0),\phi_t(x,0)]=i\hbar,\ [H,\phi(x,0)]=-i\hbar \phi_t(x,0),\\ 
[H,\phi_t(x,0)]=i\hbar[-\Delta_L\phi(x,0)+U'(\phi(x,0))],
\endgather
$$
(where $\Delta_L$ is the lattice Laplacian)
and such that $\phi(x,0),\phi'(x,0)$ commute with $\phi(y,0),\phi'(y,0)$
if $x\ne y$ (causality). 
The local operators $\phi(x,t),\phi'(x,t)$ are defined as above. 
(Observe that since $H$ does not commute with $\phi$, for $t_1\ne t_2$
the operators $\phi(x_1,t_1)$, $\phi(x_2,t_2)$, in general, 
do not commute). 

As before, the operator algebra is spanned by the operators 
$\phi(x_1,t_1)...\phi(x_n,t_n)$. 
Thus, a realization of the theory is determined by 
expectation values of these operators -- the correlation functions. 

As in the case of quantum mechanics, in any realization
we can compute the Hamiltonian explicitly. Namely, 
the Hamiltonian is of the form 
$H=\sum_{x\in L}H_x$, where $H_x=\frac{1}{2}(\phi_t^2(x,0)+(\nabla_L
\phi(x,0))^2)   
+U(\phi(x,0))+C_x$. 

{\bf Remark.} Of course, if the lattice $L$ is infinite, the definition of $H$ 
can be problematic, since the sum over $L$ may be divergent. 
However, since commutators of $H$ 
with other operators are well defined, one may hope that the constants 
$C_x$ can in fact be adjusted in such a way that the sum converges. 
This is indeed true in many situations. 

Now let us consider field theory in continuous spacetime. 
In this case the operator algebra and the Hamiltonian are 
defined similarly to the case of discrete space, which was considered above. 
Namely, the operator algebra will be generated by the 
operators $\phi(x,0)$, $\phi_t(x,0)$, and also $H_b$, $b\in{\frak p}$, 
where ${\frak p}$ is the Lie algebra of the Poincare group $\bold P$, with the
commutation relations between $\phi$, $\phi_t$ and $H_b$ similar to the above. 

However, we will face an additional problem -- now expressions like
$\phi^2(x,t)$ may not be well defined, because of ultraviolet divergences. 
This problem can be cured by the ultraviolet renormalization theory,
which we discussed last semester, if the Lagrangian we started with
was renormalizable. In this case, the algebra of local operators
is not quite an algebra, but an OPE algebra (an algebra with operator 
product expansion). It is almost never possible to compute 
the structure constants of this algebra exactly
(rational conformal field theory in 2 dimensions is the main exception), 
but it is possible 
to compute them in perturbation expansion to any finite order. 
In general, for continuous space we have additional analytic
difficulties (compared to the case of discrete space), but they will not be very
important in the present lecture, so we will not discuss them here.  
We will just need the rough general picture, which has been outlined in this 
introduction. 

Finally, let us say what we will mean by a {\bf vacuum} for a quantum
theory $(A,H)$. We will mean by a vacuum for $(A,H)$ one of two,
roughly equivalent, things:

1. A linear functional $\<,\>:A\to \C$ on the operator algebra
(the expectation value), which 
satisfies some field theory axioms (e.g. axioms for Wightman functions);
 
2. A realization $\Cal H$ of $(A,H)$ together with a vacuum state $\Omega$, 
normalized to unity. 

The passage from 2 to 1 is trivial, and the passage from 1 to 2 is a part of 
the general formalism of field theory (see Kazhdan's lecture 1).

In general, a vacuum is not the same thing as a realization, since 
the same realization can have different vacua. For example, 
in the theory of Dirac operator on a manifold the space of vacua 
is the space of harmonic spinors. However, in a Wightman field theory 
in infinite volume, 
one can show that any realization has exactly one vacuum state.  

{\bf 1.1. What is symmetry breaking, and why it does not happen in 
quantum mechanics.}

Suppose we have some classical
physical theory $(A,H)$, which has a symmetry group $G$. 
Let us ask the following question: does this theory have 
a $G$-invariant vacuum state?

If we have a quantum theory $(A,H)$, which has a symmetry group $G$, then 
the correct analogue of this equestion is: does this theory have 
a $G$-invariant realization?

If the answer is no, one says that the symmetry is broken. 
If the answer is yes, one says that the symmetry is preserved. 

In classical mechanics and classical field theory symmetry breaking 
can easily happen. For example, consider a classical particle of mass 1
on the line whose potential energy is $U(x)=g(x^2-a^2)^2/4!$, where
$a>0,g>0$. The space of states of this particle is the plane 
with coordinates $x,p$, and its Hamiltonian is $\frac{p^2}{2}+U(x)$. 
There is an action of the group $G=\Z/2\Z$ on the space of states, by
$(x,p)\to (-x,-p)$, which preserves the equations of motion, and there are 
two lowest energy states: $s_+=(a,0),s_-=(-a,0)$, which 
are permuted by $G$. But 
there is no lowest energy state which is $G$-invariant.

In quantum mechanics,
symmetry breaking does not occur. This is a simple, 
but nontrivial and very important result. Let us show why this is true in the 
case of quantum mechanics of bosons, with a real Lagrangian. 
In this case, the operator algebra is generated by operators
$x_i,p_i$, $i=1,...,n$, satisfying the Heisenberg algebra relations, and 
has a Hamiltonian $H=\frac{p^2}{2}+U(x)$.   
There is a realization of the theory in $\Cal H=L^2(\R^n)$, with $x_i$ acting 
by multiplication by the coordinate functions, and 
$p_i=-i\hbar\frac{\d}{d x_i}$. 
It is well known this representation is irreducible.
Any symmetry group $G$ of the potential $U$ also acts in $\Cal H$. Thus, 
symmetry breaking does not occur. 

{\bf Remark.} In fact, according to the Stone-von-Neumann theorem, 
$\Cal H$ is the unique realization of the theory. 

In the case of bosons with real Lagrangian we can in fact make a stronger 
statement. Namely, not only is the realization unique, but the vacuum 
is also unique (and therefore invariant under any symmetry group of $U$).  
For simplicity we will show it in the case of only one boson
on the line, but the argument 
we will give generalizes to any number of bosons in a space of any dimension. 

We will consider a single boson on the line, 
in a field with potential $U(x)$ as above. 

\proclaim{Theorem 1.1} Let $H=-\frac{1}{2}\frac{d^2}{d x^2}+U(x)$ be any
Schr\"odinger operator, such that the potential $U(x)$ tends to $+\infty$ 
at infinity (so that $H$ has discrete spectrum). Let $E_0$ be the smallest
eigenvalue of $H$. Then there exists a unique, up to a factor,
function $\psi\in L^2(\R)$
(called the vacuum state wave function) such that $H\psi=E_0\psi$. 
\endproclaim

\demo{Proof} For any $f\in L^2(\R)$ we have
$$
(f,Hf)=\int_{-\infty}^{\infty}\biggl(\frac{1}{2}|f'(x)|^2+U(x)|f(x)|^2\biggr)dx
\tag 1.1
$$
Thus, $\psi$ is defined by the condition that it is a global minimum point
for the energy functional $E(f):=(f,Hf)$ on the sphere $||f||=1$. 
The proof of uniqueness of $\psi$ rests 
on the following Lemma. 

{\bf Lemma.} If $\psi$ is a real global minimum point of (1.1) 
than $\psi$ has constant sign. 

{\it Proof of the Lemma.} Let $E(\psi)=E_0$. 
Suppose that $\psi$ changes sign at the point $x_0$. 
Since $\psi$ satisfies the Euler-Lagrange (=Schr\"odinger) 
equation $H\psi=E_0\psi$, we have
$\psi'(x_0)\ne 0$. Consider the function $|\psi|$. It is clear that
$E(|\psi|)=E(\psi)=E_0$, but $|\psi|$ is not smooth, so it does not satisfy 
the Euler-Lagrange equation $Hf=E_0f$, and thus cannot be the global minimum 
of $E$. So, the smallest value of $E$ on the sphere is less than $E_0$
-- a contadiction. 

Now it is easy to prove the theorem. If the space of solutions of 
$H\psi=E_0\psi$ is more than 1-dimensional, then there exist 
two linearly independent, orthogonal 
real solutions $\psi_1,\psi_2$. On the other hand, both of them
have to be of constant sign, so $(\psi_1,\psi_2)\ne 0$ -- a contradiction. 
$\square$\enddemo

It is useful to consider how symmetry breaking, which is absent in the quantum 
theory, arises in the quasiclassical limit. For this purpose, we should
introduce the Planck's constant $\hbar$, and consider the $\hbar$-dependent 
Hamiltonian
$$
H=-\frac{\hbar^2}{2}\frac{d^2}{d x^2}+U(x),\tag 1.2
$$
where $U(x)=g(x^2-a^2)^2/4!$. 
Let $E_0,E_1$ be the lowest eigenvalues of $H$ in the space of even and 
odd functions, respectively, and $\psi_0,\psi_1$ the corresponding 
eigenvectors. We assume that 
$\psi_0$, $\psi_1$ have unit norm, and are normalized in such a way
that $\psi_0(0)>0$, $\psi_1'(0)>0$. It is easy to show in the same way as above
that $\psi_1$ does not change sign in the regions 
$x>0$, $x<0$. 
Define
$\psi_+=\frac{1}{\sqrt{2}}(\psi_0+\psi_1)$,
$\psi_-=\frac{1}{\sqrt{2}}(\psi_0-\psi_1)$.

Then it is possible to prove the following.

\proclaim{Theorem 1.2}
(i) As $\hbar\to 0$, $E_0,E_1\sim a\hbar\sqrt{g/4!}$, and 
$E_1-E_0\sim C e^{-S_0/\hbar}$, where $S_0$ is a positive constant. 

(ii) In the sense of distributions, 
$$
\lim_{\hbar\to 0}|\psi_\pm|^2=\delta(x\mp a).\tag 1.3
$$

\endproclaim

This theorem shows that for very small $\hbar$, although there is
only one lowest energy state $\psi_0$ with energy $E_0$, 
there is another stationary state $\psi_1$
with energy $E_1$ almost indistinguishable from the lowest one, 
and in the 2-dimensional space spanned by $\psi_0,\psi_1$, 
there are two orthogonal states $\psi_+$, $\psi_-$ localized near 
the classical equilibrium states $a,-a$. 
The states $\psi_+$, $\psi_-$ are not stationary
(i.e. are not eigenvectors of $H$), but their 
failure to be stationary is indistinguishable to any finite order in $\hbar$
(in fact, the angle between $H\psi_\pm$ and $\psi_\pm$ is dominated by
$\text{const}e^{-S_0/\hbar}$. In particular, symmetry restoration 
in quantum theory is not seen at the perturbation theory level. 

Thus, we have seen how symmetry is lost in the quasiclassical limit. 
One can consider this effect from a slightly different prospective, 
by looking how symmetry appears in the process of quantization.
 
Recall that we have two classical lowest-energy states $a,-a$, near which
the operator $H$ looks approximately like a harmonic oscillator.
Therefore, we can look for eigenfunctions of $H$ perturbatively, in the form
$$
\tilde\psi_\pm(x)=f_\pm(\frac{x\mp a}{\sqrt{\hbar}}),
\ f_\pm(z) =(\pi\hbar)^{-1/4}g^{1/8}
e^{-\sqrt{g}z^2/2}(1+u_1(z)\hbar^{1/2}+u_2(z)\hbar+...)   
\tag 1.4
$$
(the 0-th term of this expansion is the lowest eigenfunction of the harmonic
oscillator). It is easy to see that the 
real ``eigenfunction'' of the form (1.4),
normalized to have unit norm, is 
unique for each sign. 

Now, one can see the following. 

(i) The formal series $f_\pm(z)$ do not converge. However, 
they represent asymptotic expansions of actual functions 
$\psi_\pm(z\sqrt{\hbar}\pm a)$ (where $\psi_\pm$ are as above), which 
are smooth from the right in $\hbar$ on $[0,\infty)$, but not analytic. 

(ii) The formal series $\tilde\psi_\pm(x)$ are eigenfunctions of
$H$ with the same eigenvalue. On the other hand, the actual functions 
$\psi_\pm(x)$ are not eigenvectors of $H$, although their failure to 
be ones is exponentally small. The actual lowest eigenvector of $H$ is 
unique up to a factor, and 
equals $\psi_0=\frac{1}{\sqrt{2}}(\psi_++\psi_-)$. 
In particular, it is $G$-invariant, unlike $\psi_+$, $\psi_-$. 

(iii) Let $W_+^n(t_1,...,t_n)$, 
$W_-^n(t_1,...,t_n)$ be the correlation functions 
computed perturbatively (by Feynman calculus)
using the formal vacua $\tilde\psi_+$,
$\tilde\psi_-$. Then $W_+^n,W_-^n$ do not 
serve as small $\hbar$ asymptotic expansions
of the correlation functions of any realization of our quantum theory. 
However, the averages $\frac{1}{2}(W_+^n+W_-^n)$ do 
serve as such asymptotic expansions. 

This shows how symmetry appears in quantization, when one goes from the 
perturbative to the nonperturbative setting. 

{\bf 1.2. Still no symmetry breaking in quantum field theory
in finite volume.}

Before going over to quantum field theory, we will give one more argument,
at the physical level of rigor, which shows why there is no symmetry 
breaking in quantum mechanics (this argument is more of an explanation 
than a proof). It is based on the path integral 
approach.
It does not use representation theory
of the Heisenberg algebra, nor the positivity of the vacuum wave function,
 and has the advantage that it also works for quantum field
theory on a spacetime with a ``space'' part of finite volume. 

As before, we will consider the $\Z/2\Z$-symmetric quartic potential $U(x)$. 
Suppose that symmetry breaking were the case. 
Then the two perturbative vacua $\tilde\psi_+,\tilde\psi_-$ 
do indeed exist nonperturbatively, i..e serve as asymptotic expansions 
of actual lowest eigenstates $\psi_+,\psi_-$ of the Hamiltonian,
in two different realizations of the theory, $\Cal H_+,\Cal H_-$. 
Consider the space $\Cal H=\Cal H_+\oplus\Cal H_-$.  
The vectors $\psi_+,\psi_-\in\Cal H$ are ``localized'' near $a,-a$, and are 
orthogonal to each other. Thus, the inner product 
$(\psi_+, e^{-Ht/\hbar}\psi_-)$ must vanish. 
Let us now compute the same inner product using path integrals. 

Recall Feynman-Kac formula:
$$
(\delta_{x_1},e^{-Ht/\hbar}\delta_{x_2})=
\int_{\phi: [0,t]\to \R, \phi(0)=x_1, \phi(t)=x_2}e^{-S(\phi)/\hbar}D\phi,
\tag 1.5
$$
where 
$$
S(\phi)=\int_{0}^t[\frac{1}{2}(\phi')^2
+U(\phi)]ds.\tag 1.6
$$
The states
 $\psi_+$, $\psi_-$ are ``localized'' near $a,-a$ (i.e. are close 
to delta-functions at $a,-a$ after a suitable normalization).
So, if we believe the Feynman-Kac formula in this situation, we can 
substitute $\delta(x-a)$, $\delta(x+a)$ instead of them, and 
apply the Feynman-Kac formula. Using the small $\hbar$ 
stationary phase estimate 
on the right hand side of (1.5), we will get
$$
(\psi_+, e^{-tH/\hbar}\psi_-)\sim Ce^{-S_*(t)/\hbar},\tag 1.7
$$
where $S_*(t)$ is the least possible action of a path 
$\phi: [0,t]\to \R$ such that $\phi(0)=-a,\phi(t)=a$. 

The least action $S_*(t)$ is attained at a classical trajectory
$\phi=\phi_*(\tau)$, which is a solution of the 
Euler-Lagrange differential equation,
i.e. the Newton's equation $\phi''=\frac{g}{6}\phi(\phi^2-a^2)$
with boundary conditions $\phi(0)=-a$, $\phi(t)=a$. 
(Such a solution exists and is unique).

{\bf Remark.} The function $x=\phi_*(\tau)$ describes the motion 
of a ball in the potential field with potential $-U(x)$ (a camel's back),
from one hump to the other. The initial velocity of the ball is such that
the time needed to go from the top of one hump to the top of the other is $t$. 
The reason that the potential $U(x)$ is replaced here by $-U(x)$ is that 
we are doing a Euclidean path integral, which means that we performed 
a Wick rotation $t\to it$. This rotation transforms the Newton's equation 
for the potential $U$ into the Newton's equation for the potential $-U$. 

Formula (1.7) contradicts the fact that $(\psi_+,e^{-Ht/\hbar}\psi_-)$
vanishes. So our assumption that there are two vacua was false. 

{\bf Remark.} Formula (1.7) actually gives the correct estimate 
of the inner product $(\psi_+,e^{-tH}\psi_-)$. In particular,
$S_0=S_*(\infty)$. 
This estimate can be confirmed
by rigorous methods. 

In this argument, we have never used the fact that $\psi_+$, $\psi_-$ 
are functions on the real line. All we used is that they are ``localized'' 
near classical equilibrium states $a,-a$, i.e. that for any local 
observable $A$ the expectation value of $A$ on $\psi_\pm$ is close
to the value of the corresponding classical observable at 
the point $(\pm a,0)$ in the phase space. Thus, our argument is independent
of the realization of the space of states as $L^2(\R)$. This makes it easy 
to generalize this argument to the case of field theory.

Consider a spacetime $M\times \R$, where 
$M$ is the ``space''. We will assume that we have already performed
a Wick rotation, so that the metric on the spacetime is a Riemannian product metric. 
We will assume that the volume of $M$ is finite and equals $V$. 

Consider the field theory on $M\times R$
with one scalar Bose field $\phi$, described by the Euclidean Lagrangian 
$$
\Cal L=\int (\frac{1}{2}(\nabla\phi)^2+U(\phi))dx, \tag 1.8
$$
where $U$ is the quartic potential as above. As before, we have two 
equilibrium states $a,-a$.

Now consider this theory quantum mechanically.
Then we can see that there is still no symmetry breaking. 
The simplest way to see it is to consider first a discrete space 
$M$. In this case, the operator algebra is a finite tensor product 
of Heisenberg algebras corresponding to points of $M$, so it is a 
Heisenberg algebra itself, and the representation-theoretic argument 
the we used in the case of quantum mechanics shows that symmetry breaking 
does not occur. 

However, it is more instructive to use the path integral argument. 
As before, assume that symmetry breaking occurs. 
Then we have two realizations $\Cal H_+$, $\Cal H_-$, and 
two orthogonal quantum vacua $\Omega_\pm\in \Cal H=\Cal H_+\oplus\Cal H_-$
whic are ``localized'' near the 
equilibrium points $a,-a$ for small $\hbar$, in the sense 
that $\<\Omega_\pm,\phi(x_1)...\phi(x_n)\Omega_\pm\>\to (\pm a)^n$, 
$\hbar\to 0$. 

As in the quantum mechanics case, the inner product $(\Omega_+,e^{-tH/\hbar}
\Omega_-)$ vanishes. On the other hand, computing it using the Feynman-Kac 
formula, we will get 
$$
(\Omega_+,e^{-tH/\hbar}\Omega_-)\sim e^{-S_*(t)V/\hbar},\tag 1.9
$$
where $S_*(t)$ is as above. The reason is that the least action 
is attained on the space-independent classical solution 
$\phi_*^M(\mu,\tau)=\phi_*(\tau)$, $\mu\in M$, $\tau\in \R$. 

Since (1.9) is nonzero (here it is essential that the volume $V$ is finite),
we get a contradiction. 

As in quantum mechanics, we can define two sets of correlation functions
$W_+^n$, 
$W_-^n$, evaluated by using perturbation theory 
near the lowest energy points $a,-a$ (of course, we can only define 
them in the renormalizable case, i.e. in 4 dimensions and below;
also, one should remember that renormalization is not uniquely determined).
Our reasonings show that $W_+^n$, $W_-^n$ are not small $\hbar$ expansions 
of the correlation functions of a realization of our quantum
 theory. On the other hand, the functions $W_0^n=\frac{1}{2}(W_+^n+W_-^n)$
have a chance to be asymptotic expansions of the actual 
(nonperturbative) correlation functions. 

For $\text{dim}(M)=1,2$ the existence of the quantum theory $\Cal H$ has been 
established in constructive field theory. In this case, it is possible to show
that there exists a $G$-invariant vacuum $\Omega_0$, such that 
the correlation functions of the theory with respect to $\Omega_0$ 
indeed have the asymptotic expansion given by $W_0^n$. 


{\bf 1.3. Symmetry breaking in quantum field theory in infinite volume.} 

When the volume $V$ of the space $M$ becomes infinite, the arguments 
of the previous section fail. The representation-theoretic argument fails,
because the canonical representation of the operator algebra, whioch we 
used in the case of finite volume, 
is now an infinite tensor product of spaces
corresponding to points; so we have to make sense of it, and there 
may be no G-invariant way of doing so. The path integral argument also fails.  
Indeed, the right hand side of (1.9)
vanishes, so both computations of $(\Omega_+,e^{-tH/\hbar}\Omega_-)$
give the same answer, and we can derive no contradiction. 
Moreover, if we assume that we have a representation $\Cal H$ of
the operator algebra with two vacua $\Omega_+,\Omega_-$, then 
using the formula
$$
\<\Omega_+,\phi(x_1^*)....\phi(x_n^*)\Omega_-\>=
\int_{\phi: X\to\R, \phi\to \pm a, t\to \pm\infty}
\phi(x_1)...\phi(x_n)e^{-S(\phi)/\hbar}D\phi,\tag 1.10 
$$
where $x^*=(\mu,it)$ for $x=(\mu,t)$,
we infer that the inner product (1.10)
vanishes for any $x_1,...,x_n$. This shows that the space
$\Cal H$ splits in an orthogonal direct sum: $\Cal H=\Cal H_+\oplus\Cal H_-$,
where the spaces $\Cal H_\pm$ are the Hilbert spaces
of separate realizations generated by the vacua
$\Omega_+,\Omega_-$. This means, symmetry breaking could occur, 
like in the classical theory: quantum effects are not strong enough 
to restore symmetry. 

This is what in fact happens in quantum field theory. 
More precisely, the situation is the following. 

(i) If the symmetry group $G$ is a finite group, symmetry
can be broken in infinite volume starting with spacetime dimension 2.
For example, in the example we considered symmetry breaking does occur. 

(ii) If the symmetry group $G$ is a connected Lie group, symmetry breaking 
can occur starting with spacetime dimension 3. The fact that it does not
occur in dimension 2 is a remarkable and a very important fact, which 
we will discuss in the next section. This fact was proved by S.Coleman 
in 1973 (see S.Coleman's paper, CMP, vol. 31, page 259).

Thus, symmetry breaking is an ``infrared'' effect, associated 
with the behavior of the theory at large distances. 

{\bf 1.4. Infinite volume asymptotics of correlation functions.}

Let us see how symmetry is broken in the infinite volume limit 
of finite volume quantum field theories. We will assume that 
$M$ is a torus $T_r=(\R/r\Z)^{d-1}$, and $r\to\infty$. 
In the limit, we hope
to recover a Poincare invariant field theory corresponding to our Lagrangian.
However, as we know, there are two such theories: $\Cal H_+$ and $\Cal H_-$. 
So which of them do we recover?

Let us formulate the question more precisely. 
Consider the correlation functions $W^n_r$,
corresponding to the field theory with space being the torus $T_r$.
Let also $W_+^n,W_-^n$ be the correlation functions of the theories 
$\Cal H_+$, $\Cal H_-$. The question is, what is the asymptotics 
of $W^n_r$ as $r\to \infty$, in terms of $W_+^n$, $W_-^n$?

The answer is: there exists an $r$-dependent normalization constant 
$C(r)$ such that 
$$
\lim_{r\to \infty}C(r)W^n_r=\frac{1}{2}(W^n_++W^n_-).\tag 1.11
$$ 

Now consider a field theory defined by a 
Lagrangian $\Cal L=\int(\frac{1}{2}(\nabla\phi)^2)+U(\phi))d^nx$,
where $U(\phi)$ is a general potential, with a finite symmetry group $G$.  
Suppose that $a_1,...,a_m$ are the 
global minimum points of $U$, 
transitively acted on by $G$, and $U(a_i)=0$. Consider first the 
quantum field theory with space being the torus of volume $V=r^{d-1}$. 
We have seen that there is one realization $\Cal H$ of this theory, and it 
has certain correlation functions $W^n_r$. On the other hand, in infinite 
volume we will have $m$ different realizations $\Cal H_1,...,\Cal H_m$, 
with correlation functions $W^n_1,...,W^n_m$. 

Let us consider the asymptotics of $W^n_r$ as $r\to \infty$. 
The answer is the following: for a suitable 
$V$-dependent normalization constant $C(r)$, 
$$
\lim_{r\to \infty}C(r)W^n_r=\frac{1}{m}\sum W^n_i,\tag 1.12
$$

This 
shows that the limit of normalized 
finite volume correlation functions may, in general,
fail to satisfy the cluster decomposition axiom, 
(see Kazhdan's lectures). 

This story can be slightly generalized. Namely, we can make $M$ a ball 
of radius $r$, and impose some boundary conditions $B$ on 
fields at the boundary of the spacetime. This means, we will define 
correlation functions by the formula
$$
W_r^n(B)(x_1,...,x_n)=\int_{\phi \text{satisfying boundary
conditions}}\phi(x_1)...\phi(x_n)e^{-S(\phi)}D\phi.
$$ 
Then the infinite volume asymptotics of $W^n_r(B)$ looks like
$$
\lim_{r\to \infty}C(r)W^n_r(B)=\sum p_iW^n_i,\tag 1.12
$$
The collection of numbers $p_i$ represents 
the ``density'' with which the quantum vacuum in finite volume is
spread over the set of classical minima (=quantum vacua in 
infinite volume). This density depends on the boundary conditions. 
The word ``density'' should not be taken literally,
however, because the numbers $p_i$ are in general complex numbers. 

If we take the simplest boundary condition $\phi=a_i$, then 
in the limit we will get the correlation functions $W_i^n$
(i.e. $p_i=1,p_j=0$, $i\ne j$), so we get purely the i-th vacuum. 
However, if we impose some other boundary condition, we will, in general, 
get a mixture of vacua. 

For example, in the case of $\Z/2\Z$-invariant quartic potential, 
we can impose boundary conditions $\phi=a$, $\phi=-a$, or $\phi=0$
(do not worry that the action of all fields in the third case is infinite;
since we are considering a normalized path integral, i.e. divided 
by the partition function, this infinity will cancel).
Let $W_r^n(s)$ are the corresponding correlation functions 
$s=a,-a,0$. Then in the first case $p_+=1,p_-=0$
(as it it hard to get from the boundary anywhere except $a$ with a small action), 
in the second case $p_+=0,p_-=1$ (for a similar reason), and in the third case 
$p_+=p_-=1/2$.

{\bf 1.5. Continuous symmetry breaking.}

Now we will consider symmetry breaking in quantum field theory, when 
the symmetry group is a connected Lie group. 
The typical example is
 a complex valued Bose field $\phi$, and the Lagrangian
$$
\Cal L(\phi)=\int d^dx\left(\frac{1}{2}|\nabla \phi|^2+
(|\phi|^2-a^2)^2\right).\tag 1.13
$$
This Lagrangian has a $U(1)$-symmetry, acting by $\phi\to e^{i\theta}\phi$. 
In the classical theory, we have lowest energy states $\phi=ae^{i\theta}$, 
$\theta\in [0,2\pi)$, so we have symmetry breaking. 
We will try to find out whether symmetry breaking 
exists also in the quantum theory. 

Above we showed that symmetry breaking does not occur in the 1-dimensional 
case (quantum mechanics). We did it for the case of a real boson $\phi$, 
but for the complex boson the argument (with path integral) works even better. 
For instance, for Lagrangian (1.13) the set of classical minima of energy is 
the circle $|\phi|=a$, which is connected. Therefore, 
to go from one classical minimum, $ae^{i\theta_1}$, 
to another, $ae^{i\theta_2}$, one does not need to go over the ``hump'' 
of the potential, and can go along the circle of minima, so one can do
it with even less action than before. Therefore, the path integral 
computation described above would show that in the complex case
there is even more linking between the states $\psi_+,\psi_-$
than in the real case. This effect might make us think that 
perhaps in field theory, 
continuous symmetry breaking does not happen as easily 
as discrete symmetry breaking. And indeed, it turns out 
that continuous symmetry breaking cannot happen in two dimensions, 
and can happen only in dimensions $\ge 3$. 

Unfortunately, the path integral method is too crude to show that
continuous symmetry breaking does not occur in 2 dimensions. 
Indeed, it is easy to see 
that all paths in the integral which computes $(\Omega_+,e^{-tH}\Omega_-)$ 
have infinite action in infinite volume, so we can derive 
no contradiction. So let us demonstrate 
why symmetry is preserved in 2 dimensions and broken 
above 2 dimensions by considering the simplest example.  

The simplest example is the theory
of a free massless real scalar Bose field
in $d$ dimensions, defined by the Lagrangian
$$
\Cal L_0(\phi)=\int d^dx(\frac{1}{2}(\nabla\phi)^2).\tag 1.14
$$
This theory has a translation symmetry $\phi\to \phi+c$. 
Let us show that this symmetry is broken for $d>2$, and preserved for $d=2$.

For $d>2$, symmetry breaking is obvious. Indeed, 
in this case we have a Wightman field theory 
generated by an elementary field $\phi$, satisfying Wightman axioms
(see Kazhdan's lectures). In this theory, the 
1-point function of the operator $\phi$ is zero, while 
the 1-point function of the operator $\phi+c$ is $c$. 
Therefore, the transformation $\phi\to\phi+c$ does not 
preserve the 1-point function. This shows, that in the case
$d>2$ we in fact have not a single realization, but 
a family of realizations $\Cal H_c$ parametrized by 
points $c$ of the line. The theory $\Cal H_c$ is defined by
the condition that $\<\Omega_c,\phi \Omega_c\>=c$, where 
$\Omega_c$ is the vacuum of $\Cal H_c$. The vacuum $\Omega_c$ is
``localized'' near the classical equilibrium state $c$. 
The map $\phi\to \phi+c$ transforms $\Cal H_b$ to $\Cal H_{b+c}$. 

For $d=2$, the situation is not the same. 
The problem is that the operator $\phi$ is not defined in 2 dimensions, 
although its derivatives are. Indeed, in 2 dimensions, we have
$$
\<\d\phi(x)\d\phi(y)\>=-\d_x\d_y\ln|x-y|,\tag 1.15
$$
so if the operator $\phi$ was defined in some way, we would have 
$$
\<\phi(x)\phi(y)\>=-\ln|x-y|+C,\tag 1.16
$$
which contradicts positivity (the function on the RHS of (1.16) is not positive).

Let us say this more precisely. 
What we have in 2 dimensions is a quantum field theory generated by a  
 Wightman map 
$\phi$ from Schwarz functions to operators, which 
is defined not on the whole space of Schwarz functions 
$S(V)$ but only on the space $S_0(V)$ 
of Schwarz functions on $V$ with integral zero. 
Then derivatives of $\phi$ can then be defined on all Schwarz functions
by
$$
\d_i\phi(f)=-\phi(\d_if)
$$
(the right hand side makes sense since $\int\d_if=0$).  

However, in the theory defined by such $\phi$ the question of symmetry breaking
does not arise, since there is no symmetry to begin with: on the space
$S_0(V)$, the maps $\phi$ and $\phi+c$ are the same. So, in order
to raise the question about symmetry breaking, 
we should consider an extension of our operator algebra,
which will have a nontrivial action of symmetry. 
The most reasonable way of doing so is the following.

Instead of considering the theory of an $\R$-valued massless scalar
$\phi$, we will consider the theory of a circle-valued field $\phi$
with the same Lagrangian. We take the circle to be $S_\l=\R/2\pi\l \Z$. 
The Lagrangian is the same as before. In this case, the local functional 
$\phi$ is not defined, but instead we have local functionals
$e^{ik\phi/\l}$, $k\in\Z$. The local functionals in this theory 
are Laurent polynomials in $e^{i\phi/\l}$ whose coefficients 
are differential polynomials 
in the derivatives of the field $\phi$, but not in $\phi$ itself.
The translational symmetry in this theory is the $U(1)$-symmetry:
for a complex number $z$ with $|z|=1$,
$z\circ e^{ik\phi/\l}=z^ke^{ik\phi/\l}$, and 
$z$ acts trivially on the derivatives of $\phi$. 
Classically, this theory has a family of vacua (equilibrium states),
given by $\phi=c$, $c\in S_\l$. 

Let us show that in this system symmetry is not broken
quantum-mechanically. It is enough to show that the 1-point function 
$\<\Omega,\O(0)\Omega\>$ vanishes for any local operator $\O$ of
the form $\O=P(\phi)e^{ik\phi/\l}$ for $k\ne 0$. Let us show this 
in the case $P=1$ (in general, the proof is analogous). Proof: 
{}From the OPE in the free theory 
(cf. Witten's lecture 3 from the fall term) we get 
$$
\<\Omega,\O(x)\O^*(0)\Omega\>=|x|^{-k^2/\l^2},
$$
so this 2-point function vanishes at infinity. 
But by clustering, the limit of this function at infinity
is $|\<\Omega,\O(0)\Omega\>|^2$. So, $\<\Omega,\O(0)\Omega\>=0$. 
 
Another way to see that there is no symmetry breaking is as follows. 
The Hilbert space of the quantized theory is of the form 
$\Cal H=F\o F^*\o l_2(\l\Z)=\oplus_{k\in\Z}(F\o F^*)_k$, 
where $F$ is the Fock space. 
The operators corresponding to derivatives of
$\phi$ respect this decomposition, while the operator $e^{ik\phi/\l}$
maps the space $(F\o F_*)_{k'}$ to $(F\o F^*)_{k+k'}$. 
The vacuum vector $\Omega$ belongs to the zero
component $(F\o F^*)_0$. This implies that all correlation 
functions of operators in this theory are invariant under the action
of $U(1)$. In particular, $\<e^{ik\phi/\l}\>=\delta_{0k}$,
which shows that the vacuum $\Omega$ is not ``localized'' near any
classical vacuum but is ``spread'' uniformly over the space $S_\l$ of
classical vacua. Thus, symmetry under $U(1)$ is not broken. 

{\bf Remark 1.} It is instructive to see why our argument 
that there is symmetry breaking for $d>2$ fails in finite 
volume, where, as we know, there should be no symmetry breaking. 
The problem is that in finite volume the field $\phi$ is not defined, 
although its derivatives are. Indeed, 
since the spacetime is the product of a time line with a compact space, 
at large distances in looks simply like a line,
the 2-point function of $\phi$ (which is
the Green's function
of the spacetime) at large $|t|$ looks like the 1-dimensional 
Green's function, i.e. $-|t|+C+o(1)$. This function is not 
bounded from below, so it violates positivity.   
Moreover, as follows from considering 
the case $d=2$, 
the situation is the same if all spatial directions but one are 
compactified.  

{\bf Remark 2.} Above we have considered the theory of a free massless 
scalar $\phi$ in dimension $d>2$ and found that its space
of quantum vacua is the space of values of $\phi$
(i.e. the target space $\R$). 
More generally, if one considers the sigma-model
with spacetime $\R^d$, $d\ge 2$, and target space $M$ (a 
Riemannian manifold),
the space of quantum vacua, as well as the space of classical vacua, will be 
$M$. (Here you should forget for a moment that this sigma-model 
for nonlinear $M$ is not renormalizable, and so it is not clear 
what this statement means. We will clarify this point later.)
In particular, $M$ as a Riemannian manifold 
can be recovered from the quantum theory as moduli space 
of quantum vacua (see Remark 3 below). 

On the other hand, we saw that for $d=2$ the theory of a circle-valued 
field $\phi$ (which is the same as the 2-dimensional 
sigma-model with target space $S_\l$) has only one vacuum $\Omega$.   
This is the case for 2-dimensional sigma-model in general: its moduli space
of quantum vacua is generally very small, and does not coincide 
with the space of classical vacua (=the target space). In particular,
the target space cannot be recovered intrinsically from the quantum theory. 
For example, the circle $S_\l$ cannot be recovered intrinsically 
from the theory of maps into $S_\l$ considered above. This, in fact, happens
for a good reason -- one can show that the theories attached to the circles
$S_\l$ and $S_{1/\l}$ are equivalent (for example, their partition functions
coincide -- see Gawedzki's lecture 1, formula (9)). This is the starting point
for the theory of mirror symmetry.

{\bf Remark 3.} Consider a field theory with the Lagrangian
$\Cal L=
\int d^dx(\frac{1}{2}(\nabla\phi)^2+U(\phi))$, where $\phi$ takes values in 
some Riemannian manifold $M$, 
and $U$ is a potential function on $M$ ($U\ge 0$). 
Let $M(0)$ is the set of zeros of $U$. Assume that $M(0)$ is nonempty
and smooth, and that $d^2U$ is nondegenerate on $T_xM/T_xM(0)$ for 
$x\in M(0)$. Suppose that there is 
a Lie group $G$ which acts by isometries on $M$, 
fixes $U$, and acts transitively on $M_0$.
In this case, one can show 
(at the physical level of rigor)
that the ``infrared behavior''
of the theory described by $\Cal L$ is the same as the ``infrared behavior''
of the sigma-model with target space being the space $M(0)$ of classical 
vacua. The precise meaning of this statement is explained in Section 1.7. 
(We can ignore nonrenormalizability problems by defining the theories 
by a cutoff path integral, where integration is taken over fields 
defined on a lattice with step $\L^{-1}$, or over fields 
having only Fourier modes with $|k|<\L$, with respect to some 
coordinate system; in this setting, the cutoff $\L$ is not sent to infinity).
 This fact can be explained heuristically: if 
a function $\phi$ has only low Fourier modes, it cannot oscillate 
rapidly, so in order to have a small action and thus give a 
noticeable contribution to the path integral, it has to stay closely 
to the minimum locus $M(0)$, i.e. has to be close to a map into $M(0)$. 

Thus, continuous symmetry breaking, being an infrared effect, will occur in 
the theory described by $\Cal L$ iff it occurs in 
the corresponding sigma-model. So, as follows from remark 2, 
symmetry breaking tends not to occur in dimension 2, but tends to occur
in dimension $>2$. 

These are, however, mostly heuristic arguments. 
In the next section we will treat the issue of continuous 
symmetry breaking in a more systematic way, using Goldstone's theorem. 

{\bf 1.6. Goldstone's theorem.}

Recall the standard formalism of Noether's theorem and currents 
in classical field theory. Suppose we have a Lagrangian 
$\Cal L=\Cal L(\phi)$ in a $d$-dimensional spacetime $V$.
Denote the space of solutions of the corresponding Euler-Lagrange 
equations by $X$. 

Let  $G^s$ be a 1-parameter symmetry group 
of this Lagrangian. Let $D_\phi:=\frac{d}{ds}|_{s=0}G^s\phi$. 
We assume that $D_\phi$ is a local functional of $\phi$. 
 
Let $\eta\in \Omega^1(X,\Omega^{n-1}(V))$ be the canonical
1-form on the space of solutions that was discussed in Bernstein's 
lectures and in  
Witten's problem sets
(the canonical 2-form on $X$ was defined as
$\int_Cd_X\eta$, where $C$ is an $n-1$-dimensional cycle).
For instance, the formula for $\eta$ for the free theory of a massless scalar
is
$$
\eta(\delta\phi)(x)=\delta\phi(x)*d\phi(x).\tag 1.17
$$
Let $J\in \Omega^0(X,\Omega^{d-1}(V))$ be defined by the formula
$J=\eta(D_\phi)$. Since $D_\phi$ is local, so is $J$. 
Thus, $J$ is a local functional on $X$ with values 
in $\Omega^{d-1}(V)$. For instance, if $\Cal L$ is the Lagrangian of the 
theory of a free massless scalar, and $G^s\phi=\phi+s$, then 
$D_\phi=1$, so $J(x)=*d\phi(x)$. The local functional $J$ is called the 
{\it current} corresponding to the symmetry $G^s$. The main property of
the current is that it is {\it conserved}, i.e. $d_VJ=0$. 
This follows form the fact that $d_V\eta=0$. 

In finite volume 
it is useful to define the charge functional
$Q=\int_CJ(x)$, where $C$ is some spacelike cycle (e.g. $t=\text{const}$).
Since the current is conserved, this quantity is independent of the choice
of the cycle, as long as it represents the fundamental homology class of 
``space''. If $\Omega=\int_Cd_X\eta$ is a
nondegenerate 2-form on $X$, then $Q$ 
is a Hamiltonian which defines the 
symmetry group $G^s$, in the sense that
$\frac{d}{ds}|_{s=0}(G^s)^*F=\{F,Q\}$, where $\{,\}$ is the Poisson bracket
on $S$, and $F$ any local functional on $X$.

In infinite volume, the functional $Q$ is not necessarily 
defined, since the integral
does not converge.  
In this case, it is convenient to set $C=C_0=\{t=0\}$ 
(here we have chosen a time coordinate on the spacetime), 
and define the ``cutoff charge functional''
$$
Q_f=\int_{C_0}f(x)J(x),\tag 1.18
$$
where $f: C_0\to \R$ is a Schwarz function.
The limit $\lim_{f\to 1}Q_f$ in this case does not exist, but for any 
local functional $F$ 
$$
\lim_{f\to 1}\{F,Q_f\}=\frac{d}{ds}|_{s=0}(G^s)^*F.\tag 1.19
$$
  
{\bf Remark.} By $f\to 1$ we mean that $f$ converges to $1$ uniformly on any 
compact set, and all derivatives of $f$ go to zero uniformly on the 
whole space. 

In quantum theory, the story is the same, except that (local) functionals
are replaced with local operators, and Poisson bracket with commutator
times $i$. 
That is, to any one-parameter symmetry $G^s$ there corresponds 
a $d-1$-form-valued local operator $J(x)$, which is conserved, 
and the charge operator $Q=\int_CJ(x)$ (in finite volume) has the property
$$
[F,Q]=-i\frac{d}{ds}|_{s=0}(G^s)^*F.\tag 1.20
$$
The case of infinite volume is dealt with in the same way as 
in the classical theory, by considering cutoff operators $Q_f$. 

Now we will discuss Goldstone's theorem. Suppose that the symmetry $G^s$
in the theory defined by $\Cal L$ is broken not only classically
but also quantum mechanically. In this case, if we have a solution
of the theory $\Cal H$ (a Hilbert space with an action of
the operator algebra), then there exists a scalar local operator $\phi$ 
whose 1-point function is not invariant under symmetry. 

{\bf Remark.} Strictly speaking, we only know that some n-point function
is not invariant; but in all known situations 
with symmetry breaking there is also a non-invariant 1-point function.  

Let $Q_f$ be the cutoff charge operator for the symmetry $G^s$. 
We have 
$$
\<\Omega|[Q_f,\phi(0)]|\Omega\>\ne 0,\tag 1.21
$$
for $f$ sufficiently close to 1. Thus, 
$\<\Omega|[J(x)\phi(0)]|\Omega\>\ne 0$ for some $x$. 

Consider the 2-point functions
$$
M_+(x):=\<\Omega|J(x)\phi(0)|\Omega\>, 
M_-(x):=\<\Omega|\phi(0)J(x)|\Omega\>\ \tag 1.22
$$
(these are $d-1$-forms on $V$). Since the symmetry is broken, 
$M_+\ne M_-$, although by space-like separation, they coincide if $x$ is 
spacelike. 

Let $\Cal H=\int_{p\in V_+}\Cal H_p$ be the spectral decomposition 
of $\Cal H$ with respect to the action of the translation group
(here $V_+$ is the positive part of the full light cone).
Using the decomposition of the inner product in a sum over intermediate 
states, ($\<J(x)\Omega,\phi(0)\Omega\>=\sum_n\<J(x)\Omega,n\>
\<n,\phi(0)\Omega\>$), we get
$$
M_\pm(x)=\int_{V_+} K_\pm(x,p)dp,\tag 1.23
$$
where 
$K_+(x,p), K_-(x,p)$ are the contributions to $M_+,M_-$ 
from intermediate states of 4-momentum $p$ 
($K_\pm$ are vector-valued distributions in $p$). 

Because of Lorentz invariance, 
the distributions $K_+,K_-$ look like 
$$
K_\pm=pe^{\pm ipx}\rho_\pm(-p^2),\tag 1.24
$$
where $\rho_\pm(s)$ are distributions on the half-line 
$s\ge 0$. 

This yields 
$$
M_\pm(x)=i*d_V\int_0^\infty\rho_\pm(m^2)W_m(\pm x)dm^2,\tag 1.25
$$
where $W_m(x)=\int_{\O_m^+}e^{ipx}dp$ 
is the Klein-Gordon propagator with mass $m$ defined in Lecture 
1 last term ($\O_m^+$ is the upper sheet of the hyperboloid $p^2=-m^2$).

Let $M(x)=M_+(x)-M_-(x)$. 
Since $W_m(x)=W_m(-x)$ when $x$ is spacelike, for spacelike $x$
(1.25) yields
$$
M(x)=i*d_V\int_0^\infty(\rho_+(m^2)-\rho_-(m^2))W_m(x)dm^2.\tag 1.26
$$
However, as we have mentioned, by spacelike separation
$M(x)$ vanishes for spacelike $x$. This implies that $\rho_+=\rho_-=\rho$, and
so (1.25) yields 
$$
M(x)=i*d_V\int_0^\infty\rho(m^2)(W_m(x)-W_m(-x))dm^2.\tag 1.27
$$
Differentiating both sides of (1.27), at a point $x$ such that $x^2>0$, 
and using the conservation of the current 
and the Klein-Gordon equation $\nabla^2W_m=-m^2W_m$, we get
$$
\int_0^\infty m^2\rho(m^2)(W_m(x)-W_m(-x))dm^2=0.\tag 1.28
$$
Taking the Fourier transform, we get
$$
p^2\rho(-p^2)=0.\tag 1.29
$$
Thus, $\rho(m^2)=c\delta(m^2)$, where $c$ is a constant. 
The constant $c$ cannot vanish, otherwise we will prove that 
$[J(x),\phi(0)]$ has a zero expectation value at the vacuum, 
which contradicts the assumption 
of symmetry breaking. 

This argument shows 
that all contributions to the 2-point function $\<\Omega,J(x)\phi(0)\Omega\>$
comes from intermediate states of zero mass. This implies 
that $L^2(\O_0^+)$ is contained in the discrete spectrum of $\Cal H$ 
(i.e. as an honest subrepresentation of the Poincare group).
Such a subrepresentation is interpreted in quantum field theory 
as a massless particle of zero spin.  
Thus, we have proved the following statement,
which goes under the name of Goldstone's theorem: 

\proclaim{Theorem} In the Hilbert space 
of a realization of a field theory with continuous symmetry breaking, 
there is a massless scalar (i.e. a subrepresentation 
of the Poincare group isomorphic to $L^2(\O_0^+)$) 
which is created by the current of the broken symmetry. 
\endproclaim

{\bf Remark.} ``Created'' means that $J(x)\Omega$ is not 
orthogonal to the subrepresentation. 

{\bf Definition.}
The massless scalar which we found is called the Goldstone boson
corresponding to the broken symmetry $G^s$. 

{\bf Remark 1.} If the symmetry with respect to $G^s$ was not broken, then
$\lim_{f\to 1}Q_f\Omega$ would be zero. 
On the other hand, when symmetry breaking occurs, 
$Q_f$ creates Goldstone bosons from the vacuum. 
Thus, Goldstone bosons ``measure''
the failure of symmetry.

{\bf Remark 2.} The Goldstone boson can be created not only 
by the current operator of the symmetry, but also by other local 
operators. In fact, as we saw in the proof of Goldstone's theorem,
it will be created by any scalar local operator whose 1-point 
function is not invariant under the symmetry. 

{\bf Remark 3.} There is no claim in Goldstone's theorem that 
the Goldstone boson is free, i.e. that 
it can be created by a free field $\phi(x)$. 
In fact, as we will see, this is often not the 
case. 

{\bf Remark 4.} If continuous symmetry breaking occurs classically, 
the Goldstone boson can already be seen in
perturbation theory. As an example consider Lagrangian (1.13). Consider the 
classical vacuum state $\phi=a$. This vacuum state is degenerate. 
Therefore, if we introduce real variables $\phi_1=\text{Re}\phi-a$,  
$\phi_2=\text{Im}\phi$, and rewrite the Lagrangian in terms of
these variable, then because of the degeneracy of the minimum
the field $\phi_2$ will be classically massless. 
Therefore, if we compute the 2-point function of 
$\phi_2$ it will have a pole at $k^2=0$
(modulo the perturbation parameters). Of course, in principle loop terms 
might shift this pole. In other words, the classically massless $\phi_2$
may get nonzero mass quantum-mechanically. What Goldstone theorem tells us
is that this will not happen if symmetry is broken. 

\proclaim{Corollary
from Goldstone theorem} 
Symmetry breaking does not happen in 2 dimensions. 
\endproclaim

{\it Proof.} Otherwise, by Goldstone's theorem 
Goldstone bosons would have to exist. But in a 2-dimensional 
quantum field theory, there can be no massless particles
created by a local operator. Indeed, 
the 2-point function of this operator in momentum space 
equals $w(k^2)=\int_0^\infty\frac{d\mu(m^2)}{k^2+m^2}$, where 
$\mu$ is the spectral measure. If there is a massless particle, 
this measure will have an atom at $m=0$. But in this case
the 2-point function $W(x)$ in position space will behave 
like $-C\ln x^2$ at infinity, i.e. would
violate the positivity axiom. $\square$

Now assume that we have a Lagrangian which has a Lie group $G$ 
of symmetries. Assume that $\Cal H$ is a 
realization of the quantum field theory defined 
by this Lagrangian, whose stabilizer is $H\subset G$. In this case 
one says that in the realization $\Cal H$, the $G$-symmetry is spontaneously 
broken to $H$. 

Let $\g,\frak h$ be the Lie algebras of $G,H$. 
Goldstone's theorem implies

\proclaim
{Corollary} $\Cal H$ contains in its discrete spectrum a subrepresentation 
isomorphic to $L^2(\O_0^+)\o (\g/\frak h)$. 
\endproclaim

\demo{Proof} The proof is clear: if 
this is not so than there exists an element $Y\in\g$, $Y\notin \frak h$, such that
$Q^Y_f\Omega\to 0$ (weakly) when $f\to 1$. This means that the symmetry with respect to $Y$ is not broken -- a contradiction.
\enddemo 

The corollary means that Goldstone bosons corresponding to linearly
independent broken infinitesimal symmetries are also linearly independent.   

{\bf Example.} Let $\phi: \R^d\to \R^N$ be a scalar field, and consider
the Lagrangian
$$
\Cal L(\phi)=\int d^dx(\frac{1}{2}(\nabla\phi)^2+\frac{g}{4!}(\phi^2-a^2)^2).
\tag 1.30
$$

This Lagrangian has an $SO(N)$-symmetry, and  
the space of its classical vacua is $S^{N-1}$.
Therefore, classically the $SO(N)$-symmetry is broken to
$SO(N-1)$. As we know, if $d>2$, the same will happen 
quantum mechanically (for a weakly coupled theory), 
and so any realization (solution) of the theory has 
$N-1$ independent Goldstone bosons. 

Let $P$ be the classical minimum with coordinates $(0,0,...,0,a)$.  
Consider the realization $\Cal H_P$ of the theory, where 
the vacuum $\Omega$ is localized near $P$.
Let $Y_i\in \frak{so}_N$, $i=1,...,N-1$, be 
the infinitesimal rotations in the planes 
generated by basis vectors $e_i,e_N$ of $\R^N$.  
The 1-point function of the operator $\phi_i$ is not invariant under 
$Y_i$, so $\phi_i$ creates the Goldstone boson corresponding to $Y_i$. 

We can construct low energy (non-vacuum) states 
localized near other classical vacua than $P$. Indeed, let $P'$ be
another classical vacuum, and $Y\in \frak{so}(N)$ is 
an element such that $e^YP=P'$. Let $J_Y$ be the current 
corresponding to $Y$, and $Q^Y_f$ the corresponding cutoff charge.  
Then the state $e^{iQ^Y_f}\Omega$ is a low energy state 
localized near $P'$. 

At long distances the theory will behave 
as a sigma-model into the space of classical vacua. 
This is an interesting statement for $N\ge 3$, because in this case 
the target ($S^{N-1}$) is not flat, so the sigma-model is not free. 
More precisely, at low energies (or long distances), the theory 
of bosons $\phi_i$, $i=1,...,N-1$, will be free in the zero approximation, 
but in the first approximation it will not be free but will be described
by the Lagrangian of the sigma-model into the sphere. 

{\bf 1.7. Infrared behavior of purely non-renormalizable field theories.}

In this section we will clarify the meaning of
the statement that for $d>2$ a quantum field theory behaves in the infrared 
limit as a sigma-model into the space of classical vacua.

Suppose we have a purely nonrenormalizable field theory 
described by a Lagrangian $\Cal L$. We will call a Lagrangian 
purely nonrenormalizable if all its couplings have negative dimension.
An example of such a Lagrangian is the Lagrangian of a sigma-model for 
$d>2$. Such Lagrangians are not good for perturbative renormalization 
in the UV limit, but create no problem in the IR limit, since
all their interactions are IR irrelevant from the point of view of 
the Wilsonian renormalization group flow. Namely, if 
we introduce an UV momentum cutoff $\L$ (which is now not being sent 
to infinity), we can define correlation functions of $\Cal L$ 
perturbatively: the correlation function
is the sum of amplitudes of all Feynman diagrams, 
which are evaluated as usual, with integration carried out
with cutoff $|q|<\L$. Because $\Cal L$ has no mass terms, 
there will be some IR divergences, but they can be dealt 
with in the same way as we dealt with UV divergences 
in Lectures 1-3 last semester. Moreover, since the theory is purely 
non-renormalizable, only finitely many graphs will be divergent
for each number of external legs (like in a superrenormalizable 
theory in UV renormalization). 

Now suppose that we want to compute the asymptotic expansion  
of the n-point function in momentum representation, around 
the point $k_i=0$. 
We will have (for $\sum k_i=0$):
$$
G_n(k_1,...,k_n)=k_1^{-2}...k_n^{-2}G_n^0(k_1,...,k_n),\tag 1.31
$$
where $G_n^0$ is a certain series, having a limit at $k_i=0$ 
(In general, this will not be a power series; it may contain terms 
of the form $k^4\ln k^2$). 

The key property of this series, which follows from pure
nonrenormalziability, is that modulo 
terms of any finite power, it is determined by finitely many Feynman graphs. 
Thus, we can obtain the IR asymptotics of the correlation functions 
to any order in $k_i$ without 
having to sum the perturbation series.

Now suppose we have an actual quantum field theory, given by some 
renormalizable Lagrangian 
$\Cal L'$. When we say that the theory defined by $\Cal L'$ is
described in the IR limit by a purely nonrenormalizable 
Lagrangian $\Cal L$ (on the same fields), we mean that
to a certain order in $k_i$ (near $k_i=0$), the functions 
$G_n^0$ given by (1.31) are the same for $\Cal L$ as for $\Cal L'$. 

For instance, when at the end of the previous section we said that
at low energies (or long distances), the theory 
of bosons $\phi_i$, $i=1,...,N-1$, is free in the zero approximation, 
and described
by the Lagrangian of the sigma-model into the sphere,
$$
\Cal L_\sigma=\int d^dx(\frac{1}{2}\sum (\nabla\phi_i)^2
+R(\sum \phi_i^2)(\sum (\nabla\phi_i)^2))
$$
in the first approximation (where $R$ is proportional to the curvature), 
we meant that the 
functions $G_n^0(k_1,...,k_n)$ for $\phi_i$ are the same as in the free theory
modulo $o(1)$, and the same as in the sigma-model modulo $o(k^2)$.   

Computing higher terms of the $k$-expansion, one can construct a purely nonrenormalizable
low energy effective theory, which will describe our theory in the infrared to
any required accuracy.

{\bf Remark.} The restriction of the function $G_0^n(k_1,...,k_n)$ 
to the locus $k_i^2=0$ (but $k_i$ is not necessarily zero) has
a physical meaning: it is the scattering amplitude (S-matrix) of $n$ 
Goldstone bosons. Thus, the statement is that scattering matrix of
$n$ Goldstone bosons in model (1.30) is like in the free theory for $N=2$ 
(as the circle is flat), but has a quadratic correction due to curvature
for $N>2$. 

\end



Therefore, asymptotic expansion (1.31) can be computed to any order in $k_i$
by computing amplitudes of finitely many diagrams.  


\end


