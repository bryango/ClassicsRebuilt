%% This is a plain TeX file
%%
\magnification=1200
\hsize=6.5 true in
\vsize=8.7 true in
\input epsf.tex

%These two files (in this order!!) are necessary
%in order to use AMS Fonts 2.0 with Plain TeX

\input amssym.def
\input amssym.tex

\font\boldtitlefont=cmb10 scaled\magstep2
\font\smallboldtitle=cmb10 scaled \magstep1
\font\sans=cmss10 scaled\magstep1

\footline={\hfil {\tenrm III.\folio}\hfil}

\def\eps{{\varepsilon}}
\def\Eps{{\epsilon}}
\def\kap{{\kappa}}
\def\lam{{\lambda}}

\def\undertext#1{$\underline{\vphantom{y}\hbox{#1}}$}
\def\nspace{\lineskip=1pt\baselineskip=12pt%
     \lineskiplimit=0pt}
\def\dspace{\lineskip=2pt\baselineskip=18pt%
     \lineskiplimit=0pt}

\def\half{{\textstyle 1\over\textstyle 2}}
\def\w{{\mathchoice{\,{\scriptstyle\wedge}\,}
  {{\scriptstyle\wedge}}
  {{\scriptscriptstyle\wedge}}{{\scriptscriptstyle\wedge}}}}
\def\Le{{\mathchoice{\,{\scriptstyle\le}\,}
{\,{\scriptstyle\le}\,}
{\,{\scriptscriptstyle\le}\,}{\,{\scriptscriptstyle\le}\,}}}
\def\Ge{{\mathchoice{\,{\scriptstyle\ge}\,}
{\,{\scriptstyle\ge}\,}
{\,{\scriptscriptstyle\ge}\,}{\,{\scriptscriptstyle\ge}\,}}}
\def\plus{{\hbox{$\scriptscriptstyle +$}}}
\def\Proclaim#1{\medbreak
  \medskip\noindent{\bf#1\enspace}\it\ignorespaces}
  %the way to use this is:
  %"\Proclaim{Theorem 1.1.}" for instance.
\def\finishproclaim{\par\rm
     \ifdim\lastskip<\medskipamount\removelastskip
     \penalty55\medskip\fi}
\def\Item#1{\par\smallskip\hang\indent%
  \llap{\hbox to\parindent {#1\hfill\enspace}}\ignorespaces}

\def\strutdepth{\dp\strutbox}
\def\marginmark{\strut\vadjust{\kern-\strutdepth
     \hskip-25pt\specialmark}}
\def\specialmark{\vtop to \strutdepth{%
     \baselineskip8pt\vss
     \llap{\raise20pt\hbox{\vtop{\hbox{\sevenrm See Prob.}
\hbox{\sevenrm Set \#5.}}}}}\null}

\def\im{{\rm Im}}  \def\Open{{\rm open}}
\def\Diff{{\rm Diff}}  \def\Closed{{\rm closed}}
\def\Map{{\rm Diff}}  \def\spurious{{\rm spurious}}
\def\Met{{\rm Met}} \def\phys{{\rm phys}}
\def\diag{{\rm diag}}  \def\Vir{{\rm Vir}}
\def\spin{{\rm spin}} \def\Res{{\rm Res}}
\def\Null{{\rm null}} \def\mass{{\rm mass}}
\def\SO{{\rm SO}} \def\Tr{{\rm Tr\,}}
\def\SU{{\rm SU}} \def\tr{{\rm tr}}
\def\Weyl{{\rm Weyl}} \def\Ker{{\rm Ker}}
\def\Range{{\rm Range}} \def\SL{{\rm SL}}
\def\Det{{\rm Det}} \def\re{{\rm Re}}
\def\dist{{\rm dist}} \def\psl{{\rm psl}}
\def\Vol{{\rm Vol}} \def\PSL{{\rm PSL}}

\def\fbar{\bar{f}}  \def\mubar{\bar{\mu}}
\def\barh{\bar{h}}  \def\gammabar{\bar{\gamma}}
\def\kbar{\bar{k}}  \def\lambar{\bar{\lambda}}
\def\mbar{\bar{m}}  \def\phibar{\bar{\phi}}
\def\wbar{\bar{w}}  \def\etabar{\bar{\eta}}
\def\xbar{\bar{x}}
\def\zbar{\bar{z}}
\def\Abar{\bar{A}}
\def\Kbar{\bar{K}}
\def\Pbar{\bar{P}}
\def\Sbar{\bar{S}}
\def\Tbar{\bar{T}}

\def\ghat{\hat{g}}
\def\muhat{\hat{\mu}}

\def\htil{\tilde{h}}
\def\xtil{\tilde{x}}
\def\Dtil{\widetilde{D}}
\def\Ltil{\tilde{L}}
\def\Ntil{\widetilde{N}}
\def\Ttil{\widetilde{T}}
\def\scrFtil{\widetilde{\scrF}}
\def\epstil{\tilde{\eps}}

\def\dbR{{\Bbb R}}
\def\dbZ{{\Bbb Z}}

%Capital roman double letters(Blackboard bold)
\def\db#1{{\fam\msbfam\relax#1}}

\def\dbA{{\db A}} \def\dbB{{\db B}}
\def\dbC{{\db C}} \def\dbD{{\db D}}
\def\dbE{{\db E}} \def\dbF{{\db F}}
\def\dbG{{\db G}} \def\dbH{{\db H}}
\def\dbI{{\db I}} \def\dbJ{{\db J}}
\def\dbK{{\db K}} \def\dbL{{\db L}}
\def\dbM{{\db M}} \def\dbN{{\db N}}
\def\dbO{{\db O}} \def\dbP{{\db P}}
\def\dbQ{{\db Q}} \def\dbR{{\db R}}
\def\dbS{{\db S}} \def\dbT{{\db T}}
\def\dbU{{\db U}} \def\dbV{{\db V}}
\def\dbW{{\db W}} \def\dbX{{\db X}}
\def\dbY{{\db Y}} \def\dbZ{{\db Z}}

\font\teneusm=eusm10  \font\seveneusm=eusm7 
\font\fiveeusm=eusm5 
\newfam\eusmfam 
\textfont\eusmfam=\teneusm 
\scriptfont\eusmfam=\seveneusm 
\scriptscriptfont\eusmfam=\fiveeufm 
\def\scr#1{{\fam\eusmfam\relax#1}}


%Upper-case Script Letters:

\def\scrA{{\scr A}}   \def\scrB{{\scr B}}
\def\scrC{{\scr C}}   \def\scrD{{\scr D}}
\def\scrE{{\scr E}}   \def\scrF{{\scr F}}
\def\scrG{{\scr G}}   \def\scrH{{\scr H}}
\def\scrI{{\scr I}}   \def\scrJ{{\scr J}}
\def\scrK{{\scr K}}   \def\scrL{{\scr L}}
\def\scrM{{\scr M}}   \def\scrN{{\scr N}}
\def\scrO{{\scr O}}   \def\scrP{{\scr P}}
\def\scrQ{{\scr Q}}   \def\scrR{{\scr R}}
\def\scrS{{\scr S}}   \def\scrT{{\scr T}}
\def\scrU{{\scr U}}   \def\scrV{{\scr V}}
\def\scrW{{\scr W}}   \def\scrX{{\scr X}}
\def\scrY{{\scr Y}}   \def\scrZ{{\scr Z}}

\def\gr#1{{\fam\eufmfam\relax#1}}

%Euler Fraktur letters (German)
\def\grA{{\gr A}}	\def\gra{{\gr a}}
\def\grB{{\gr B}}	\def\grb{{\gr b}}
\def\grC{{\gr C}}	\def\grc{{\gr c}}
\def\grD{{\gr D}}	\def\grd{{\gr d}}
\def\grE{{\gr E}}	\def\gre{{\gr e}}
\def\grF{{\gr F}}	\def\grf{{\gr f}}
\def\grG{{\gr G}}	\def\grg{{\gr g}}
\def\grH{{\gr H}}	\def\grh{{\gr h}}
\def\grI{{\gr I}}	\def\gri{{\gr i}}
\def\grJ{{\gr J}}	\def\grj{{\gr j}}
\def\grK{{\gr K}}	\def\grk{{\gr k}}
\def\grL{{\gr L}}	\def\grl{{\gr l}}
\def\grM{{\gr M}}	\def\grm{{\gr m}}
\def\grN{{\gr N}}	\def\grn{{\gr n}}
\def\grO{{\gr O}}	\def\gro{{\gr o}}
\def\grP{{\gr P}}	\def\grp{{\gr p}}
\def\grQ{{\gr Q}}	\def\grq{{\gr q}}
\def\grR{{\gr R}}	\def\grr{{\gr r}}
\def\grS{{\gr S}}	\def\grs{{\gr s}}
\def\grT{{\gr T}}	\def\grt{{\gr t}}
\def\grU{{\gr U}}	\def\gru{{\gr u}}
\def\grV{{\gr V}}	\def\grv{{\gr v}}
\def\grW{{\gr W}}	\def\grw{{\gr w}}
\def\grX{{\gr X}}	\def\grx{{\gr x}}
\def\grY{{\gr Y}}	\def\gry{{\gr y}}
\def\grZ{{\gr Z}}	\def\grz{{\gr z}}

%\overfullrule=0pt

\parindent=18pt
\line{\sans --- DRAFT ---\hfill{\rm IASSNS-HEP-97/72}}

\bigskip\bigskip
\centerline{\boldtitlefont Lectures 5, 6}
\medskip
\centerline{\smallboldtitle III. String Amplitudes
and Moduli Space of Curves}

\medskip
\centerline{Eric D'Hoker}

\frenchspacing

\dspace
\bigskip
In the previous Chapter, we have quantized free
bosonic strings, determined their spectrum and Hilbert
space of states under certain conditions:
$D=26$ and the presence of massless gravitons and
Yang-Mills particles.
(The case $D\Le 25$, allowed by the no-ghost theorem,
was shown to be a subcase.)

Now, we shall quantize interacting strings, and
investigate quantum mechanical transition amplitudes
for the scattering of string states.
While operator methods were convenient for
determining the spectrum of free strings, the
functional integral formulation in terms of a
summation over surfaces is better suited for dealing
with interactions.
Thus, our starting point is the transition
amplitude
$$
A=\sum\limits_{{{\scriptstyle {\rm topologies}\,
\Sigma}\atop{\scriptstyle (h=0,\ldots,\infty)}}}
\int\nolimits_{\Met(\Sigma)}Dg{1\over
\scrN(g)}\int\nolimits_{\Map(\Sigma,M)}
V_1\ldots V_N e^{-S[x,g]}\eqno{(3.1)}
$$
with Riemannian space-time $M$, and external string
states represented by vertex operators $V_i$,
$i=1,\ldots,N$.
This was precisely our original starting point with
Euclidean signature metric on $M$.
Minkowski amplitudes (for example, for flat $M=\dbR_M^D$) are
obtained by analytic continuation in external momenta
and polarization tensors.
The vertex operators were determined in \S{II} in
terms of conformal primary fields of weight $(1,1)$.
The summation in the transition amplitude $A$
 is over topologies of compact oriented
surfaces, since we shall limit ourselves here to the
study of the closed oriented string.

The strategy is as follows:

\medskip\noindent
1)\enspace
At fixed topology of $\Sigma$, we use the product
structure of $\Met(\Sigma)=\Weyl(\Sigma)\times$
conformal classes $(\Sigma)$ to decompose the measure
$Dg$ on $\Met(\Sigma)$ into a product of measures on
the orbits of Weyl, on the orbits of $\Diff(\Sigma)$
and on the moduli space
$\scrM_h=\Met(\Sigma)/\Diff(\Sigma)\ltimes
\Weyl(\Sigma)$.
A Jacobian determinant arises, which is essentially
the Faddeev-Popov determinant.
This decomposition holds for any matter conformal
field theory, in particular for any space-time $M$,
defining a conformal field theory.

\medskip\noindent
2)\enspace
For flat Euclidean space-time $M$, the combined
measure --- including the Faddeev-Popov determinant
--- and the correlation functions, are found to be
Weyl-invariant (i.e. $\Weyl(\Sigma)$ is a
symmetry group) if and only if $D=26$ and gravitons
are massless on surfaces of any topology
(equivalently, the $a=1$ condition of \S{II} holds).

\medskip\noindent
3)\enspace
Under the assumptions of 2) the string theory is usually
called critical. 
In this case, the formal measure $Dg/\scrN(g)$ on
$\Met(\Sigma)$, divided by the volume of the orbits of
the symmetry group
$\Diff(\Sigma)\ltimes\Weyl(\Sigma)$, {\it naturally
projects down} to an {\it intrinsically defined
measure on $\scrM_h$}.
Thus, the transition amplitudes for critical strings
reduce to well-defined integrals over $\scrM_h$
of the matter conformal field theory correlation functions.

\medskip\noindent
4)\enspace
More generally, the assumptions of 2) may be relaxed,
and the matter part may be any conformal field theory
of central charge $c=26$ and with correlation
functions satisfying certain Weyl invariance conditions.

\medskip\noindent
5)\enspace
In particular, in Lecture 9, we shall generalize the
Weyl invariance conditions to the case of curved
space-time manifolds $M$ --- in certain limits.

\medskip\noindent
6)\enspace
Under the assumptions of 2), we shall obtain in this
chapter $0$- and $1$-loop transition amplitudes for
critical bosonic string theory.

\bigskip\noindent
\Item{\bf A.} {\bf Finite-Dimensional Case}

\smallskip
Let $P$ be a finite-dimensional manifold, with
Riemannian metric $G$, and associated measure
$d\mu_G$.
Let $H$ be a Lie group acting on $P$, and assume that
$d\mu_G$ is $H$-invariant.
We wish to reduce the integrals of $H$-invariant
functions on $P$ to integrals on the quotient
$\Pbar=P/H$.
Let $f(x)\colon\,P\to\dbC$ be $H$-invariant,
$$
f(xh)=f(x)\qquad\hbox{for all}\qquad
x\in P,\,\, h\in H.
$$
Then for any section $s\colon\,\Pbar\to P$, we have
$$
\int\nolimits_{P}d\mu_G f=\int\nolimits_{\Pbar}
d\mubar\,\,s^*(f)J\int\nolimits_{H}d\nu_H\,\,,
$$
where $d\mubar$ is the measure on $\Pbar$ induced from
$d\mu_G$, $d\nu_H$ is the left-invariant measure on $H$, and $J$
is a Jacobian (or Faddeev-Popov)
determinant which we define below, as follows.
First, let us present a figure:
$$
\vbox{\epsfxsize=2.5in\epsfbox{fig1.eps}}
$$
The tangent space at $x$ to $P$ may be decomposed
into the part tangent to the orbit of $H$,
generated by the Lie algebra $\scrH$,
$$
i\colon\,\scrH\to T_x P
$$
and a part from the push-forward of the tangent space to
$T_{\xbar}\Pbar$:
$$
s_*\colon\,T_{\xbar}\Pbar\to T_x(P)\,\,,
$$
as follows:
$$
s_*\oplus i\colon\,T_{\xbar}\Pbar \oplus\scrH\to
T_x(P).
$$
The Jacobian $J$ is now obtained as
$$
J=\det(T_{\xbar}\Pbar\oplus\scrH\to T_x P).
$$
\vfill\eject
%\bigskip\noindent
\noindent
\Item{\bf B.} {\bf Basic Notation: Tensors,
Derivatives}

\smallskip
Fix a positive definite metric $g$ on $\Sigma$ and
choose local complex coordinates $z$, $\zbar$ in which
$$
g=g_{z\zbar}(dz\otimes d\zbar+d\zbar\otimes dz)=
2g_{z\zbar}\vert dz\vert^2
$$
Tensors decompose into one-dimensional components of weight
$(m,n)$, which are sections of a line bundle $K^{(m,n)}$:
$$
\phi=\phi_{{\underbrace{z\ldots z}_m}\,\,
  {\underbrace{\zbar\ldots\zbar}_n}}
(dz)^m (d\zbar)^n\in K^{(m,n)}\,\,.
$$
Using the metric $g$, we identify tensors  of weight
$(m,n)$ with tensors of weight $(m-n,0)$.
The space of tensors of weight $(m,0)$ is denoted
$K^m$, with $K^1$ the space of sections of the
canonical line bundle $T_{(1,0)}^*(\Sigma)$.
(Generally, we do not distinguish between a bundle and
the space of its sections.)

On sections of $K^m$ we have the $L^2$ inner product
$$
(\phi_1,\phi_2)_g=\int\nolimits_{\Sigma} d\mu_g
(g_{z\zbar})^{-m}\phi_1^*\phi_2\eqno{(3.2)}
$$
as well as the covariant derivatives
$$
\eqalignno{
\nabla\phi
   &=\nabla_{\zbar}^{(m)}\phi+\nabla_z^{(m)}\phi\cr
\noalign{\hbox{where
$\nabla_{\zbar}^{(m)}=\bar{\partial}_{(m)}$ is the
Cauchy Riemannn operator}}
\nabla_{\zbar}^{(m)}\colon\,K^m &\to K^{(m,1)}\,\,
\nabla_{\zbar}^{(m)}\phi 
  ={\partial\over\partial\zbar}\,\phi
\otimes d\zbar\,\,. &(3.3)\cr}
$$
Equivalently, identifying $(m,1)\sim(m-1,0)$, we have
$$
\cases{
\nabla_{(m)}^z\colon\,K^m\to K^{m-1}
  &\qquad $\nabla_{(m)}^z\phi=(g_{z\zbar})^{-1}
   {\partial\over\partial\zbar}\phi\otimes(dz)^{-1}$\cr
&\cr
\nabla_z^{(m)}\colon\,K^m\to K^{m+1}
  &\qquad $\nabla_z^{(m)}\phi=(g_{z\zbar})^m
   {\partial\over\partial z}(g_{z\zbar})^{-m}\phi\otimes
  dz\,\,.$\cr}
\eqno{(3.4)}
$$
(In general, we shall not exhibit the differentials $dz$
and $d\zbar$ in $\nabla$.)
The adjoints of these operators with respect to
$(\quad,\quad)_g$ are
$$
\left(\nabla_{(m)}^z\right)^\dagger=
  -\nabla_{z}^{(m-1)};\quad
  \left(\nabla_z^{(m)}\right)^\dagger=-
  \nabla_{(m+1)}^z\eqno{(3.5)}
$$
and the Laplace operators are
$$
\eqalign{
\Delta_{(m)}^{\plus}
&=-2\nabla_{(m+1)}^z\nabla_z^{(m)}\cr
\Delta_{(m)}^-
&=-2\nabla_z^{(m-1)}\nabla_{(m)}^z\cr}
\eqno{(3.6)}
$$
We also make use of the Riemann-Roch-Atiyah-Singer
theorem
$$
\dim\,\Ker\nabla_z^{(m)}-\dim\,\Ker\nabla_{(m+1)}^z=
\left(m+{1\over 2}\right)\chi(\Sigma)=(2m+1)(1-h)\,\,,
\eqno{(3.7)}
$$
as well as the vanishing theorems:

{\narrower\medskip
\noindent
$$
\matrix{
\bullet &h=0, m\Ge 1,  
     &\Ker\,\nabla_{\zbar}^{(m)}=0\cr
&&\hbox{(no holomorphic forms on the sphere)}\cr
\noalign{\bigskip}
\bullet &h\Ge 2, m\Le -1,  
&\Ker\,\nabla_{\zbar}^{(m)}=0\,\,.\cr
&&\hbox{(no holomorphic vector fields for $h\Ge 2$)}\cr}
\eqno{(3.8)}
$$
\medskip}

\bigskip\noindent
\Item{\bf C.} {\bf Space of Metrics -- Moduli Space
of Riemann Surfaces}

\smallskip
Let $\Met(\Sigma)$ denote the space of positive
definite metrics on $\Sigma$, and $T_g\Met(\Sigma)$
its tangent space at $g$, consisting of infinitesimal
deformations $\delta g\in K^{(1,1)}\oplus
K^{(2,0)}\oplus K^{(0,2)}$.
Following the results of \S{B}, we have an $L^2$ norm
$\Vert\delta g\Vert_g^2=(\delta g,\delta g)_g$ on
$T_g\Met(\Sigma)$, which makes $\Met(\Sigma)$ into an
infinite dimensional Riemannian manifold.
The measure $Dg$ is defined with respect to
$\Vert\delta g\Vert$.

$\Diff(\Sigma)$ acts isometrically on $\Met(\Sigma)$,
but $\Weyl(\Sigma)$ does not.
The space of orbits of $\Diff(\Sigma)\ltimes
\Weyl(\Sigma)$ in $\Met(\Sigma)$ is the {\it moduli
space} of Riemann surfaces (or complex curves) of genus $h$:
$$
\scrM_h\equiv\Met(\Sigma)/\Diff(\Sigma)
\ltimes\Weyl(\Sigma)\,\,.\eqno{(3.9)}
$$
Actually, the group $\Diff(\Sigma)$ is not connected.
The group of connected components is the {\it mapping
class group} $G_\Sigma$:
$$
G_\Sigma=\pi_0(\Diff(\Sigma))\,\,,
$$
which is an infinite discrete group for $h\Ge 1$.
The component connected to the identity in
$\Diff(\Sigma)$ is denoted by $\Diff_0(\Sigma)$.
Thus, there arises another natural space: $J_h$,
which is Teichm\"uller space
$$
\scrT_h\equiv\Met(\Sigma)/\Diff_0(\Sigma)\ltimes
  \Weyl(\Sigma)\,\,,
$$
which is the simply connected covering space of
$\scrM_h$:
$$
\scrM_h=\scrT_n/G_\Sigma\,\,.
$$
Teichm\"uller space is a complex finite dimensional
manifold, while moduli space is a complex finite
dimensional orbifold (the action of $G_\Sigma$ on
$\scrT_h$ produces singularities), both of dimension
$$
\dim_{\dbC}\scrT_h=\dim_{\dbC}\scrM_h=
\cases{0 &$h=0$\cr
1 &$h=1$\cr
3h-3 &$h\Ge 2\,\,.$\cr}
\eqno{(3.10)}
$$
We have explained that
string transition amplitudes can be reduced to an
integral over moduli space.
We wish to factorize the measure $Dg$ into its
components along $\Diff(\Sigma)$, $\Weyl(\Sigma)$ and
$\scrM_h$, i.e. we want to fix the gauge.
To examine this factorization concretely, we
decompose $T_g\Met(\Sigma)$ into the tangent spaces
to the orbits of $\Diff_0(\Sigma)$ and
$\Weyl(\Sigma)$, as explained in the finite
dimensional example.
We perform this factorization at an arbitrary metric
$g\in\Met(\Sigma)$, and shall choose a section later on.

We use the fact that $\Diff_0(\Sigma)$ is generated by
vector fields $v=v^z\oplus v^{\zbar}\in
K^{(-1,0)}\oplus K^{(0,-1)}$.
A Weyl transformation will in general be denoted by
$e^{2\sigma}$, where $\sigma$ takes real values.

Let the reference metric be $g=2g_{z\zbar}\vert
dz\vert^2$; an arbitrary metric $g+\delta g$ in
the neighborhood of $g$ may then be conveniently
parametrized by a change in $g_{z\zbar}$ as well as by
Beltrami differentials $\delta\eta_{\zbar}^z\in
K^{(-1,1)}$ and $\delta\etabar_z^{\zbar}\in
K^{(1,-1)}$.
$$
g+\delta g=2\left(g_{z\zbar}+\delta g_{z\zbar}\right)
\vert dz+\delta\eta_{\zbar}^z d\zbar\vert^2\,\,.\eqno{(3.11)}
$$
The space $T_g\Met(\Sigma)$ admits the orthogonal
decomposition
$$
\eqalignno{
T_g\Met(\Sigma) &=K^{(1,1)}\oplus K^{(2,0)}\oplus
  K^{(0,2)}\cr
\delta g &=2\delta g_{z\zbar}\vert dz\vert^2+
  2g_{z\zbar}\delta\eta_{\zbar}^z(d\zbar)^2
 +2g_{z\zbar}\delta\etabar_z^{\zbar}(dz)^2\,\,. &(3.12)\cr}
$$
The infinitesimal action of $(\delta v^z$, $\delta
v^{\zbar}$, $2\delta\sigma)$ in the Lie aglebra of
$\Diff_0(\Sigma)\ltimes\Weyl(\Sigma)$ is
then simply
$$
\left\{
\eqalign{
\delta g_{z\zbar} &=(2\delta\sigma+\nabla_z\delta
v^z+\nabla_{\zbar}\delta v^{\zbar})g_{z\zbar}\cr
\delta\eta_{\zbar}^z  &=\nabla_{\zbar}\delta v^z\cr
\delta\etabar_z^{\zbar} &=\nabla_z \delta v^{\zbar}\,\,.\cr}
\right.
\eqno{(3.13)}
$$
While the range of $\Weyl(\Sigma)$ is the full
$K^{(1,1)}$, in general the range of
$\Diff_0(\Sigma)$ need not be all of $K^{(2,0)}\oplus
K^{(0,2)}$.
Instead, we have the orthogonal decomposition
$$
K^{(2,0)}=\Range\,\nabla_z^{(1)}\oplus
\Ker(\nabla_z^{(1)})^\dagger\,\,.\eqno{(3.14)}
$$

The space $\Ker\,(\nabla_z^{(1)})^\dagger=
\Ker\,\nabla_{\zbar}^{(2)}$ consists of holomorphic
quadratic differentials on $\Sigma$, and can be
identified with the holomorphic cotangent space of
$\scrM_h$:
$$
T_{(1,0)}^*(\scrM_h)=\Ker\,\nabla_{\zbar}^{(2)}\eqno{(3.15)}
$$
(Analogously, we have
$T_{(0,1)}^*(\scrM_h)=\Ker\,\nabla_z^{(-2)}$.)
To see this, we notice that holomorphic
differentials
$\phi_j\in\Ker\,\nabla_{\zbar}^{(2)}$ provide
linear forms on the space of differentials
$\delta\eta$:
$$
(\delta\eta,\phi_j)=\int\nolimits_{\Sigma}dz\,d\zbar\,\,
\delta\eta_{\zbar}^z \phi_{jzz}\eqno{(3.16)}
$$
The above pairing is $\Weyl(\Sigma)$-invariant and
the kernel of $(\,\,\cdot\,\,,\phi_j)$ (and
$(\,\,\cdot\,\,,\phi_j^*)$) is precisely the tangent
space to the orbits of $\Diff_0(\Sigma)$ and
$\Weyl(\Sigma)$; hence the identification.
By the Riemann-Roch theorem, and the vanishing
theorems listed above, it is easy to see that the
dimensions of $\scrM_h$ and of
$\Ker\,\nabla_{\zbar}^{(2)}
\oplus\Ker\,\nabla_z^{(-2)}$ coincide.

More precisely, the holomorphic quadratic
differentials $\phi_j\in\Ker\,\nabla_{\zbar}^{(2)}$
form a complex vector space, thus providing a natural
almost complex structure on the tangent space of $\scrM_h$.
In fact, this almost complex structure is integrable and
results in a complex structure on $\scrM_h$;
for a given choice of $\phi$, the above
linear forms define local complex coordinates $m_j$,
$\mbar_j$ on $\scrM_h$ given by
$$
\delta m_j=(\delta\eta,\phi_j)\qquad\qquad
\delta\mbar_j=(\delta\etabar,\phibar_j)\,\,.\eqno{(3.17)}
$$
We may represent this projection onto $\scrM_h$ by a
picture analogous to the one used ot discuss the
finite dimensional example at the beginning of this
lecture.
$$
\vbox{\epsfxsize=3.5in\epsfbox{fig2.eps}}
$$

Vice-versa, for a given set of complex coordinates
$(m_j,\mbar_j)$ on $\scrM_h$, we may choose a section
$s$ of $\scrM_h$ in $\Met(\Sigma)$ --- also called a
slice --- of dimension $\dim\,\scrM_h$, and
parametrized by $(m_j,\mbar_j)$:
$$
\vbox{\epsfxsize=2.5in\epsfbox{fig3.eps}}
$$
The tangent space at $g(m_j,\mbar_j)$ then decomposes
into the tangent directions to
the action of $\Weyl(\Sigma)$, the action of
$\Diff_0(\Sigma)$ on
the differentials $\delta\eta$, $\delta\etabar$,
and  moduli deformation, given by a basis of
{\it Beltrami differentials} $\mu_j$ and $\mubar_j$:
$$
\eqalignno{
&\cases{
\delta\eta_{\zbar}^z=\nabla_{\zbar}^{(-1)}
  \delta v^z+\sum\limits_{j=1}^{\dim\,\scrM_h}
  \delta m_j\mu_{j\zbar}^z &\cr
\noalign{\medskip}
\delta\etabar_z^{\zbar}=\nabla_z^{(1)}\delta
v^{\zbar}+\sum\limits_{j=1}^{\dim\,\scrM_h}
  \delta\mbar_j\mubar_{jz}^{\zbar}\,\,. &\cr} &(3.18) \cr
\noalign{\hbox{We may choose the Beltrami 
     differentials as follows}}
&\cases{
\mu_{j\zbar}^z=g^{z\zbar}{\partial
g_{\zbar\zbar}\over\partial m_j}(m,\mbar) &\cr
\noalign{\medskip}
\mubar_{jz}^{\zbar}=g^{z\zbar}{\partial g_{zz}\over
  \partial\mbar_j}(m,\mbar)\,\,,\cr} &(3.19)\cr}
$$
so that $\mu_j$ and $\mubar_j$ are unchanged under
Weyl transformations $g\to
e^{2\sigma}g\colon\,\mu_j\to\mu_j$;
$\mubar_j\to\mubar_j$.
Thus, under a change of section $s$, $\mu_j$ and
$\mubar_j$ will be shifted only by the action of a
vector field in the Lie algebra of $\Diff_0(\Sigma)$.

Finally, for $\Sigma$ with the topology of either
the sphere or the torus, we also have holomorphic
vector fields in $\Ker\,\nabla_{\zbar}^{(-1)}$.
These generate conformal automorphisms, and do not
act on $\eta$ at all.
We have
$$
\left\{
\matrix{
h=0 &\dim\,\Ker\nabla_{\zbar}^{(-1)}=3
  &\SL(2,\dbC)\cr
h=1 &\dim\,\Ker\nabla_{\zbar}^{(-1)}=1
  &T^2\cr
h\Ge 2 &\dim\,\Ker\nabla_{\zbar}^{(-1)}=0
  &\hbox{possibly finite automorphism group}\,\,.\cr}\right.
\eqno{(3.20)}
$$

\bigskip\noindent
\Item{\bf D.} {\bf Factorizing the Integration Measure}

\smallskip
We decompose the integration measure $Dg$ on
$\Met(\Sigma)$ into measures on the orbits of
$\Diff(\Sigma)$ and $\Weyl(\Sigma)$ and on $\scrM_h$
by using the parametrization developed above for
$\delta g$ in terms of Weyl changes $\delta\sigma$
and differentials $\delta\eta$, $\delta\etabar$:
$$
\delta g=2\delta\sigma
g+2g_{z\zbar}\delta\eta_{\zbar}^z(d\zbar)^2+
2g_{z\zbar}\delta\etabar_z^{\zbar}(dz)^2\,\,.\eqno{(3.21)}
$$
The relation
$$
\Vert\delta g\Vert_g^2=4\Vert\delta\sigma\Vert_g^2
+2\Vert\delta\eta\Vert_g^2+2\Vert\delta\etabar\Vert_g^2
\eqno{(3.22)}
$$
implies that $Dg=D\sigma\,D\eta\,D\etabar$, in
obvious notation.

Next, we make use of the decomposition of
$\delta\eta$ and $\delta\etabar$ into orbits of
$\Diff_0(\Sigma)$ and moduli deformations, which we
parametrize by coordinates 
$(\delta m_j,\delta\mbar_j)$ on $\scrM_h$ and a
choice of Beltrami differentials $\mu_{j\zbar}^z$ and
$\mubar_{jz}^{\zbar}$, as in (3.18).
Using the orthogonal decomposition of the
differentials $\delta\eta_z^{\zbar}$ $g_{z\zbar}$ according to
$$
K^{(2,0)}=\Range\nabla_z^{(1)}\oplus
\Ker\,\nabla_{\zbar}^{(2)}\,\,,\eqno{(3.23)}
$$
we evaluate $\Vert\delta\eta\Vert_g^2$ and find
$$
2\Vert\delta\eta\Vert_g^2=2\Vert\nabla_{\zbar}^{(-1)}
\delta v^z\Vert_g^2+\Vert\delta\eta\bot
\hbox{ \rm projection of $\delta\eta_z^{\zbar}$ 
$g_{z\zbar}$ onto }
\Ker\,\nabla_{\zbar}^{(2)} \Vert_g^2\,\,.\eqno{(3.24)}
$$
The second term on the right hand side is easily
worked out by choosing an orthogonal basis for 
$\Ker\,\nabla_{\zbar}^{(2)}$, with basis vectors
$\phi_j$, $j=1,\ldots,\dim\,\scrM_h$:
$$
\eqalign{
\Vert\delta\eta &\bot \hbox{ projection of
$\delta\eta_z^{\zbar}$ $g_{z\zbar}$ onto } 
  \Ker\,\nabla_{\zbar}^{(2)}\Vert_g^2\cr
&=\sum\limits_{j}(\delta\eta,\phi_j)_g(\phi_j,\phi_j)_g^{-1}
  (\phi_j,\delta\eta)_g\cr
&=\sum\limits_{j,k,\ell}\delta m_k\delta\mbar_\ell
  (\mu_k,\phi_j)_g(\phi_j,\phi_j)_g^{-1}
  (\phi_j,\mu_\ell)_g\cr}
\eqno{(3.25)}
$$

The first term on the right hand side of (3.24) is conveniently
rewritten in terms of the Laplace operator
$\Delta_{(-1)}^-$:
$$
2\Vert\nabla_{\zbar}^{(-1)}\delta v^z\Vert_g^2=
  (\delta v^z,\Delta_{(-1)}^-\delta v^z)_g
$$

A last point must be clarified before evaluating the
measure $D\eta\,D\etabar$.
For the special cases of the sphere $(h=0)$ and the
torus $(h=1)$,
$\Ker\,\nabla_{\zbar}^{(-1)}\not=0$, and there
are (continuous families of) conformal automorphisms of $\Sigma$:
$\SL(2,\dbC)$ for $h=0$ and $T^2$ for $h=1$.
The corresponding vector fields leave $\eta$ and $\etabar$
invariant.
Thus, in the change of variables,
$\delta\eta\to(\delta v,\delta m_j)$, we must
restrict to vector fields $\delta v$ that are
orthogonal to $\Ker\,\nabla_{\zbar}^{(-1)}$.
The corresponding restricted volume element on vector
fields $v$ is denoted by $D'v$.
The associated determinant of $\Delta_{(-1)}^-$ must
be similarly restricted to non-zero modes.

Putting all together, we find
$$
Dg=\det{'}\Delta_{(-1)}^-
{\vert\det(\mu_j,\phi_k)_g\vert^2\over
\det(\phi_j,\phi_k)_g}\prod\limits_{j}
dm_j\,d\mbar_j\,D'v\cdot D\sigma\,\,.\eqno{(3.26)}
$$
While originally we chose
$\phi_j\in\Ker\,\nabla_{\zbar}^{(2)}$ to form an
orthogonal basis, it is now clear that the above
formula holds for any basis 
$\Ker\,\nabla_{\zbar}^{(2)}$.
(The finite dimensional determinants are of
$\dim\,\scrM_h\times\dim\,\scrM_h$ dimensional
matrices.)

It is standard practice --- and extremely useful ---
to introduce the following combination of
determinants
$$
Z_{(n)}^{\pm}(g)\equiv
{\Det'\Delta_{(n)}^{\pm}\over\det(\phi_j,\phi_k)_g\det
(\psi_a,\psi_b)_g}\eqno{(3.27)}
$$
where
$$
\matrix{
{\scriptstyle \bullet} 
  &\phi_j\in\Ker\,\nabla_{(n+1)}^{z}\hfill
  &\hbox{ for $\plus$ }; &\quad
\phi_j\in\nabla_{z}^{(n-1)} 
  &\hbox{ for $-$}\hfill &\qquad \cr
{\scriptstyle \bullet} 
  &\psi_a\in\Ker\,\nabla_{z}^{(n)}\hfill
  &\hbox{ for $\plus$ }; &\quad
  \psi_a\in\nabla_{(n)}^{z}\hfill &\hbox{ for
$-$}\hfill\,\,. &\qquad \cr}
$$
While $Z_{(n)}^{\,\pm}$ have non-trivial Weyl
transformation properties, we shall show in the next
section that this particular combination has simple
transformation properties.

Putting all together, we have the following
expression for the measure $Dg$: 

\Proclaim{Theorem:}
$$
Dg=D\sigma\,\Dtil v\,Z_{(-1)}^-(g)\,\,
\vert\det(\mu_j,\phi_k)\vert_g^2
\prod\limits_{j=1}^{\dim\,\scrM_h}dm_j\,d
  \mbar_j\,\,,\eqno{(3.28)}
$$
\finishproclaim

\noindent
where we have

\medskip
\item{\rm 1)}
$\Dtil v\equiv D'v$ $\det(\psi_a,\psi_b)_g$.
For $h\Ge 2$, we have $\Dtil v=Dv$, i.e. the measure
over all vector fields of $\Diff_0(\Sigma)$.
For $h=0,1$, we shall postpone the explicit
construction of $\Dtil v$ until \S{G}.

\smallskip
\item{\rm 2)}
The measure
$$
\vert\det(\mu_j,\phi_k)_g\vert^2\prod\limits_{j}
dm_j\,d\mbar_j\eqno{(3.29)}
$$
is intrinsically defined on $\scrM_h$.
To see this, notice that we chose $\mu_j$ to be Weyl
independent, that $\phi_j$ can be chosen Weyl
independent, and that the pairing $(\mu_j,\phi_k)_g$
is Weyl invariant.
Furthermore, we defined Beltrami differentials by
$$
\mu_{j\zbar}^z=g^{z\zbar}
{\partial g_{\zbar\zbar}\over\partial m_j}\eqno{(3.30)}
$$
so that the above combination is manifestly unchanged
under reparametrizations of $m_j$ and $\mbar_j$.

\smallskip
\item{\rm 3)}
$Z_{(-1)}^-(g)$ will be Weyl dependent, as we shall
establish in the next section.
This object is called the Faddeev-Popov determinant
(up to the finite dimensional determinants),
expressing the fact that we gauge fixed the
$\Diff_0(\Sigma)$-invariance.

\smallskip
\item{\rm 4)}
To complete the construction outlined in the
finite-dimensional case, for the projection down onto
$\scrM_h$ of the measure $Dg$ on $\Met(\Sigma)$, we
will have to understand the Weyl dependence of the
combined $Z_{(-1)}^-(g)$ and matter contributions,
which we do in the next section.
\finishproclaim

\vfill\eject

\noindent
\Item{\bf E.} {\bf Weyl Rescalings of Functional
Determinants}

\smallskip
The determinant combinations $Z_{(n)}^{\pm}$
(including those with $n\in{1\over 2}+\dbZ$, which
will enter when dealing with the
superstring, as we shall see later on) will be
considered repeatedly, and we shall need their
behavior under $\Weyl(\Sigma)$ rescalings of the
metric.
Throughout, we shall adopt a regularization and
renormalization scheme that preserves $\Diff(\Sigma)$
invariance.

\Proclaim{Theorem.}
$$
Z_{(n)}^{\pm}(ge^{2\sigma})=Z_{(n)}^{\pm}(g)
e^{-2c_{(n)}^{\pm}S_L(\sigma,g)}\eqno{(3.31)}
$$
where $c_{(n)}^{\pm}$ is given by
$$
c_{(n)}^{\pm}\equiv 6n^2\pm 6n+1\eqno{(3.32)}
$$
and the {\rm Liouville action} is independent of $n$ and
given by
$$
S_L(\sigma,g)={1\over 12\pi}\int\nolimits_{\Sigma}
d\mu_g\left[\half g^{mn}\partial_m\sigma
\partial_n\sigma+R_g\sigma+\beta^2(e^{2\sigma}-1)\right]\,\,.
\eqno{(3.33)}
$$
The constant $\beta$ depends upon the renormalization
scheme used, and will vanish when determinants are
defined with $\zeta$-function regularization, or with
dimensional regularization.
\finishproclaim

Since special care must be paid to the effects of the
zero modes, we outline the proof.
(Various intermediate steps will be proven in Problem
Set IV.)
First, it suffices to prove the formula for the Weyl
transformation of $Z_{(n)}^{\pm}$ by infinitesimal
$\sigma$; the result for general $\sigma$ follows by
straightforward integration (Gaw\c{e}dzki, prob. 8).
Thus, we must prove
$$
\delta_\sigma \ln\,Z_{(n)}^{\pm}(g)=-
{c_{(n)}^{\pm}\over 6\pi}\int\nolimits_{\Sigma}
d\mu_g[R_g\delta\sigma+2\beta^2\delta\sigma]
$$
(Notice that from the form of this infinitesimal variation, it is
clear that the determinant $Z_{(n)}^{\pm}$ defines a
conformal field theory with central charge related to $\pm
c_{(n)}^{\pm}$.)

We define determinants by introducing the
$\zeta$-function associated with the eigenvalues
$\lam_\alpha$ of $\Delta_{(n)}^{\pm}$:
$$
\zeta(s)\equiv
\mathop{{\sum}'}\limits_{\alpha}
\lam_\alpha^{-s}
$$
Here, $\sum'$ denotes the sum over all non-zero
eigenvalues $\lam_\alpha$ of $\Delta_{(n)}^{\pm}$.
It is standard to represent this sum in terms of the
heat-kernel of $\Delta_{(n)}^{\pm}$:
$$
\zeta(s)={1\over \Gamma(s)}\int\nolimits_0^\infty
dt\,t^{s-1}(\Tr\,e^{-t\Delta_{(n)}^{\pm}}-N_{(n)}^{\pm})
\eqno{(3.35)}
$$
where $N_{(n)}^{\pm}=\dim\,\Ker\,\Delta_{(n)}^{\pm}$.

The $\zeta(s)$ function series and the heat-kernel
integral representation are absolutely convergent for
$\re(s)>1$.
$\zeta(s)$ is holomorphic for $\re(s)>1$, but may be
analytically continued throughout $\dbC$, with a
simple pole at $s=1$.
The determinants are then defined by
$$
\ln\,\Det'\Delta_{(n)}^{\pm}=-\zeta'(0)
\eqno{(3.36)}
$$
The non-zero spectra of $\Delta_{(n)}^{\plus}$ and
$\Delta_{(n+1)}^-$ coincide and thus their
determinants are equal, so we shall henceforth focus
on $\delta_{(n)}^-$.

To evaluate the Weyl dependence of determinants, we
compute that of $\zeta(s)$ and analytically continue to
$s=0$.
We use the following intermediate results (proven in
Problem Set IV).

\medskip
\item{1)}
$N_{(n)}^{\pm}$ are Weyl independent

\smallskip
\item{2)}
$\delta_\sigma\Tr
e^{-t\Delta_{(n)}^-}=-t{\partial\over\partial t}\Tr
[2n\delta\sigma e^{-t\Delta_{(n-1)}^{\plus}}-
  (2n-2)\delta\sigma e^{-t\Delta_{(n)}^-}]$

\smallskip
\item{3)}
$\Tr\delta\sigma e^{-t\Delta_{(n)}^{\pm}}=
{1\over 4\pi t}\int\nolimits_{\Sigma}d\mu_g
\delta\sigma +{1\pm 3n\over
12\pi}\int\nolimits_{\Sigma}d\mu_g \,
R_g\,\delta\sigma +O(t)$ \hfill (3.37)

\smallskip
\item{4)}
$\Tr\delta\sigma e^{-t\Delta_{(n)}^-}=-{1\over
2n-2}\delta_\sigma
\ln\,\det(\phi_j,\phi_k)_g+O(e^{-\lam_1 t})$

\smallskip
\item{5)}
$\Tr\delta\sigma e^{-t\Delta_{(n-1)}^{\plus}}=
{1\over 2n}\delta_\sigma \ln\,\det(\psi_a,\psi_b)_g+O
 (e^{-\lam_1 t})$

\medskip\noindent
where $\lam_1$ is the smallest non-zero eigenvalue of
$\Delta_{(n)}^-$.

We make use of the above results and carry out the
analytic continuation of $\delta_\sigma$ $\zeta(s)$ to $s=0$,
and then consider $\delta\sigma\zeta'(0)$.
For $\re(s)>1$, using 1) and 2), we have
$$
\delta_\sigma\zeta(s)=-{1\over \Gamma(s)}
\int\nolimits_0^\infty dt\,t^s{\partial\over\partial t}
\Tr[\,\,\cdot\,\,](t)\eqno{(3.38)}
$$
where we introduced the abbreviation
$$
\Tr[\,\,\cdot\,\,](t)\equiv\tr[2n\delta\sigma
e^{-t\Delta_{(n-1)}^{\plus}}-
(2n-2)\delta\sigma e^{-t\Delta_{(n)}^-}]\eqno{(3.39)}
$$
We shall make use of the asymptotic behaviors of
$\Tr[\,\,\cdot\,\,](t)$ as $t\to 0$ and $t\to\infty$,
which are obtained from 3), 4), 5):
$$
\cases{
\Tr[\,\,\cdot\,\,](t)={1\over 2\pi
t}\int\nolimits_{\Sigma}d\mu_g \delta\sigma +
{c_{(n)}^-\over 6\pi}\int\nolimits_{\Sigma}d\mu_g
  \delta\sigma R_g+O(t)\cr
\noalign{\bigskip}
\Tr[\,\,\cdot\,\,](t)=\delta_\sigma \ln
\,\det(\phi_j,\phi_k)_g
  \det(\psi_a,\psi_b)_g+O(e^{-\lam_1 t})\,\,.\cr}
\eqno{(3.40)}
$$
Here, $c_{(n)}^-$ was defined in (3.32).

To separate the asymptotic behaviors as $t\to 0$,
$t\to\infty$, we cut $[0,\infty]$ at $t=t_0>0$, and
define
$\delta_\sigma\zeta(s)
=\delta_\sigma\zeta_0(s)+\delta_\sigma\zeta_\infty(s)$
$$
\cases{
\delta_\sigma \zeta_0(s)=-{1\over\Gamma(s)}\int\nolimits_0
  ^{t_0}dt\,t^s{\partial\over\partial
t}\Tr[\,\,\cdot\,\,](t) &\cr
\noalign{\bigskip}
\delta_\sigma\zeta_\infty(s)=-{1\over \Gamma(s)}
  \int\nolimits_{t_0}^\infty dt\,t^s
  {\partial\over\partial t}
\Tr[\,\,\cdot\,\,](t)\,.\cr} \eqno{(3.41)}
$$
The first term fails to converge at $s=0$, so we
subtract the asymptotic behavior of $\Tr[\,\,\cdot\,\,](t)$ in
$1/t$ as $t\to 0$:
$$
\eqalign{
\delta_\sigma\zeta_0(s)= &-{1\over \Gamma(s)}
  \int\nolimits_0^{t_0}dt\,t^s{\partial\over\partial t}
\left(\Tr[\,\,\cdot\,\,](t)-{1\over 2\pi
t}\int\nolimits_{\Sigma}d\mu_g \delta\sigma\right)\cr
&+{1\over \Gamma(s)}{1\over
2\pi}\int\nolimits_{\Sigma}d\mu_g\delta\sigma
  \int\nolimits_0^{t_0}dt\,t^{s-2}\,\,.\cr}
$$
The first integral on the right hand side is
convergent at $s=0$, while the second may be
explicitly analytically continued to $s=0$.
Taking the derivative at $s=0$:
$$
\eqalign{
\delta_\sigma\zeta'_0(0) &=-\int\nolimits_0^{t_0}dt
  {\partial\over\partial t}\left(\Tr[\,\,\cdot\,\,]
(t)-{1\over 2\pi t}\int\nolimits_{\Sigma}
  d\mu_g \delta\sigma\right)
  -{1\over 2\pi t_0}\int\nolimits_{\Sigma}
  d\mu_g \delta\sigma\cr
&=-\Tr[\,\,\cdot\,\,](t_0)+\left.\left(\Tr[\,\,\cdot\,\,]
  (t)-{1\over 2\pi t}\int\nolimits_{\Sigma}d\mu_g
  \delta\sigma\right)\right\vert_{t=0}\cr
&=-\Tr[\,\,\cdot\,\,](t_0)+{c_{(n)}^-\over 6\pi}
  \int\nolimits_{\Sigma}d\mu_g
R_g\delta\sigma\,\,.\cr}
$$
The term $\delta_\sigma\zeta_\infty(s)$ is given by
$$
\delta_\sigma\zeta_\infty(s)=-{1\over \Gamma(s)}
\int\nolimits_{t_0}^\infty
dt\,t^s{\partial\over\partial t}\Tr[\,\,\cdot\,\,](t)
$$
As $t\to\infty$, $\Tr[\quad]\to \hbox{\rm constant }
+O(e^{-\lam_1 t})$, $\lam_1>0$, so that the integral
is exponentially convergent, and analytic around
$s=0$.
Hence, using the asymptotics of $\Tr[\,\,\cdot\,\,](t)$
as $t\to\infty$:
$$
\eqalign{
\delta_\sigma\zeta'_\infty(0)
&=-\int\nolimits_{t_0}^\infty
dt{\partial\over\partial t}
Tr[\quad](t)\cr
&=-\delta_\sigma
\ln\,\det(\phi_j,\phi_k)_g\det(\psi_a,\psi_b)_g
+\Tr[\,\,\cdot\,\,](t_0)\,\,.\cr}
$$
Putting all together, we have
$$
\eqalign{
\delta_\sigma \ln\,\Det'\Delta_{(n)}^- &=-\delta\sigma\zeta'(0)\cr
&=\delta_\sigma
\ln\,\det(\phi_j,\phi_k)_g\det(\psi_a,\psi_b)_g
+{c_{(n)}^-\over 6\pi}\int\nolimits_{\Sigma}
  d\mu_g R_g\delta\sigma\,\,,\cr}
$$
as announced previously.
Notice that in $\zeta$-function definitions of the
determinants, no ``area term'' involving the
additional constant $\beta$ arises.

\bigskip\noindent
\Item{\bf F.} {\bf Critical Central Charge -- 
 Critical Dimension}

\smallskip
We complete the reduction of the measure $Dg$ to a
measure on $\scrM_h$, making use of the Weyl
transformation properties of functional determinants,
derived in \S{E}.

Just as we did in the case of the finite-dimensional
example, we choose a section $s$ of $\Met(\Sigma)$
and denote the corresponding metric
$\ghat(m_j,\mbar_j)$.
A general metric $g\in\Met(\Sigma)$ may then be
parametrized in terms of the action
of $\Diff(\Sigma)$ and $\Weyl(\Sigma)$ on
$\ghat(m_j,\mbar_j)$:
let $f\in\Diff(\Sigma)$ and
$e^{2\sigma}\in\Weyl(\Sigma)$, then
$$
\vbox{\epsfxsize=2.5in\epsfbox{fig4.eps}}
$$
The action of $\Diff(\Sigma)$ is by isometries, and our
definition of functional determinants was invariant
with respect to $\Diff(\Sigma)$.
Thus, the effect of $f^*$ is immaterial and may be
dropped.
The effect of Weyl transformations is obtained by
combining the results of \S{D, E}, and we have
$$
Dg=\Dtil vD\sigma
Z_{(-1)}^-(\ghat)e^{-26S_L(\ghat,\sigma)}\vert\det
(\muhat_j,\phi_k)_{\ghat}\vert^2\,dm_j\, d\mbar_j\,\,.
\eqno{(3.42)}
$$

We now consider a full transition amplitude $A_h$,
evaluated on surfaces $\Sigma$ of genus $h$.
(To avoid certain complications due to conformal
automorphisms, we postpone the cases $h=0,1$ and
restrict to $h\Ge 2$.
The cases $h=0,1$ are completely analogous and will
be dealt with separately in \S{I}.)
Recall that
$$
A_h=\int\nolimits_{\Met(\Sigma)}Dg\,
{1\over \scrN(g)}\left<V_1\ldots V_N\right>_g\,\,,
\eqno{(3.43)}
$$
where
$$
\left<V_1\ldots
V_N\right>_g=\int\nolimits_{\Map(\Sigma,M)}Dx\,V_1\ldots
V_N e^{-S[x,g]}\,\,. \eqno{(3.44)}
$$

Given the Weyl transformation properties of $Dg$
established above, it is clear that a {\it matter
theory} (i.e. the quantum field theory of $x$ fields for
given metric $g$) is singled out that makes the Weyl
dependence of the amplitude trivial.
We define a {\it critical $($bosonic$)$ string
theory} as one governed by a matter conformal field
theory with central charge $c=26$, and with physical
vertex operators $V_i$, $i=1,\ldots,N$ such that
$$
\left<V_1\ldots V_N\right>_{e^{2\sigma}\ghat}
=e^{26S_L(\ghat,\sigma)}
\left<V_1\ldots V_N\right>_{\ghat}\,\,.\eqno{(3.45)}
$$
(This last condition puts restrictions on the allowed
vertex operators in the theory.)
For critical string theories, the amplitudes at fixed
topology $(h\Ge 2)$ reduce to
$$
\eqalign{
A_h=\int\nolimits_{\scrM_h}\prod\limits_{j}dm_j\,
d\mbar_j
  &\vert\det(\muhat_j,\phi_k)_{\ghat}\vert^2
  Z_{(-1)}^-(\ghat)\left<V_1\ldots
V_N\right>_{\ghat}\cr
&\cdot\int\nolimits_{\Weyl(\Sigma)}D\sigma
\int\nolimits_{\Diff_0(\Sigma)}Dv/\scrN_0(g)\cr}
\eqno{(3.46)}
$$
(Recall that for $h\ge 2$, $\Dtil v=Dv$.)
Now, recall that the formal measure $Dg/\scrN(g)$ was
the measure on $\Met(\Sigma)$, divided by the volume
of the orbits of the group
$\Weyl(\Sigma)\times\Diff(\Sigma)$.
Since we restricted our integration to one over
$\scrM_h$ (and not Teichm\"uller space $T_h$), we
effectively quotiented by the mapping class group.
The remaining normalization factor is
$\scrN_0(g)$, formally the volume of
$\Weyl\times\Diff_0(\Sigma)$.
Thus, the formal integrals over $\Weyl(\Sigma)$ and
$\Diff_0(\Sigma)$ cancel $\scrN_0(g)$:
$$
\int\nolimits_{\Weyl(\Sigma)}D\sigma
\int\nolimits_{\Diff_0(\Sigma)} Dv/\scrN_0(g)=1\,\,.
\eqno{(3.47)}
$$

Thus, we finally obtain an ultimate expression for
amplitudes in {\it critical string theory} 
at fixed topology ($h\Ge 2$); the expression reduces to
integrals over moduli of Riemann surfaces: 
$$
A_h
=\int\nolimits_{\scrM_h}\prod\limits_{\scrM_h}dm_j\,d\mbar_j
\vert\det(\muhat_j,\phi_k)_{\ghat}\vert^2
Z_{(-1)}^-(\ghat)\left<V_1\ldots V_N\right>_{\ghat}
\eqno{(3.48)}
$$
The cases for $h=0,1$ will be discussed in \S{I, J}.

\bigskip\noindent
\Item{\bf G.} {\bf Flat Space-Time Manifold $M$}

\smallskip
For flat Euclidean space-time $M=\dbR^D$, we have
already established that the central charge of the
conformal field theory defined by maps
$x\in\Map(\Sigma,M)$ is $c=D$.
Thus, the flat space-time string theory is a critical
string theory precisely when $D=26$.
(We now notice that this condition is the same as the
one found for the free string: this is rather
fortunate!)
Let us work out in detail under what conditions the
vertex operators also satisfy the criteria of a
critical string.

To do so, it is convenient to start with a study of
the Weyl transformation properties of correlation
functions of unintegrated exponents:
$$
\bigl<\prod\limits_{i=1}^N e^{ik_i\cdot
x(z_i)}\bigr>_g=\int\nolimits_{\Map(\Sigma,M)}
Dx \prod\limits_{i=1}^N e^{ik_i\cdot x(z_i)}
e^{-S_0[x,g]}
\eqno{(3.49)}
$$
where $S_0$ is the Gaussian action, including
the constant dilaton $\Phi_0$:
(We ignore the constant tachyon.)
$$
S_0[x,g]={1\over 8\pi}
\int\nolimits_{\Sigma}d\mu_g\,x\cdot \Delta_{(0)}x+
{\Phi_0\over 4\pi}\int\nolimits_{\Sigma}d\mu_g R_g
\eqno{(3.50)}
$$
The Gaussian integral is carried out by separating
the constant mode in $x$:
$$
x=x_0+x'\,\,,\qquad\qquad
\int\nolimits_{\Sigma}d\mu_g\,x'=0
$$
and making use of the Green functions for $x'$:
$$
\left\{
\eqalign{
&\Delta_{(0)}G_g(z,w)=4\pi\delta_g(z,w)-
{4\pi\over \int_\Sigma d\mu_g}\,\,.\cr
&\int\nolimits_{\Sigma}d\mu_g G_g(z,w)=0\cr}
\right.
\eqno{(3.51)}
$$

The Green function at coincident points must be
regularized and renormalized in a $\Diff(\Sigma)$-invariant way.
This may be carried out with the help of heat-kernel
methods, or simply by renormalizing by the
$\Diff(\Sigma)$-invariant distance:
$$
G_g^R(z,z)\equiv\lim\limits_{w\to z}
(G_g(z,w)+2\ln\dist_g(z,w))\,\,.
\eqno{(3.52)}
$$
The Weyl scaling properties of the Green function are
then
$$
\left\{
\eqalign{
G_{e^{2\sigma}\ghat}(z,w) &=G_{\ghat}(z,w)+
  f(z)+f(w)\cr
G_{e^{2\sigma}\ghat}^R(z,w) &=G_{\ghat}^R(z,z)+
  2f(z)+2\sigma(z)\,\,,\cr}
\right.
\eqno{(3.53)}
$$
where $f(z)$ is a function required to maintain
$\int\nolimits_{\Sigma}d\mu_g G_g=0$ under Weyl
rescalings.
We do not need its explicit form here.
One finds
$$
\Bigl<\prod\limits_{i}e^{k_i\cdot x(z_i)}\Bigr>_g=
\delta(k) e^{-{1\over 2}\chi\Phi_0}
\left({\Det'\Delta_{(0)}\over\int\nolimits_{\Sigma}d\mu_g}
\right)^{-D/2}\prod\limits_{i,j}e^{{1\over 2}k_i\cdot
k_j G_g(z_i,z_j)}\,\,,\eqno{(3.54)}
$$
where $k\equiv\sum\limits_{i}k_i$, $\delta(k)$ is the
$D$-dimensional $\delta$-function, and the factor
$\int\nolimits_{\Sigma}d\mu_g$ under the determinant
arises from the normalization of the zero mode $x_0$.
As a result, under $g=e^{2\sigma}\ghat$, we have
$$
\prod\limits_{i,j}e^{-{1\over 2}k_i\cdot
k_jG_g(z_i,z_j)}=\prod\limits_{i,j}
e^{-{1\over 2}k_i\cdot k_j G_{\ghat}(z_i,z_j)}
e^{-k_i^2\sigma(z_i)}\,\,.\eqno{(3.55)}
$$
All dependence on $f(z)$ cancels in view of momentum
conservation $k=\sum\limits_{i}k_i=0$, as guaranteed
by the overall $\delta(k)$ factor in $(3.54)$.
The Weyl scaling property of the determinant factor
is precisely that of $Z_{(0)}(g)^{-D/2}$.
($Z_{(0)}(g)$ contains an additional factor involving
holomorphic Abelian differentials, but this factor is
manifestly Weyl invariant.)

Combining all of the above, we find for $D=26$
$$
\Bigl<\prod\limits_{i}^N e^{ik_i\cdot
x(z_i)}\Bigr>_{\ghat
e^{2\sigma}}=\prod\limits_{i=1}^N
e^{-k_i^2\sigma(z_i)}
e^{26 S_L(\ghat,\sigma)}\Bigl<\prod\limits_{i=1}^N
e^{ik_i\cdot x(z_i)}\Bigr>_{\ghat}\,\,.
\eqno{(3.56)}
$$
It is now straightforward to examine the Weyl
covariance conditions of vertex operators for low
mass cases.

\medskip\noindent
1)\enspace
Tachyon vertex operator
$$
V_T(k)=\eps(k)\int\nolimits_{\Sigma}d\mu_g
e^{ik\cdot x}\,\,.
$$
In order to cancel the explicit Weyl dependence of
the measure $d\mu_g$ on $\Sigma$:
$$
d\mu_{e^{2\sigma}\ghat}=d\mu_{\ghat}e^{2\sigma}
$$
against the Weyl dependence of the correlation
function, we recover the familiar condition $k^2=2$
for the tachyon.
(This condition also corresponded to $a=1$ in
\S{III}.)

\medskip\noindent
2)\enspace
The massless vertex operators
$$
V_\eps(k)=\eps_{\mu\nu}(k)\int\nolimits_{\Sigma}
d\mu_g g^{mn}\partial_m x^\mu
\partial_n x^\nu e^{ik\cdot x}\,\,;
$$
the Weyl dependence of the part $d\mu_g g^{mn}$ is
trivial, so we need the familiar condition $k^2=0$.
In addition, Weyl dependence appears through the
contraction of $\partial_m x^\mu$ with the
exponential, and is cancelled by the transversality
conditions
$$
k^\mu\eps_{\mu\nu}(k)=k^\nu\eps_{\mu\nu}(k)=0\,\,.
$$
Finally, from the contraction  of $\partial_m x^\mu$
and $\partial_n x^\nu$, (actually only the trace-part
contributes), a curvature $R_g$ term appears, and Weyl
invariance for the trace-part is achieved through the
addition of a (classically non-Weyl invariant) part
involving the curvature:
\marginmark 
$$
V_\eps(k)=\eps_{\mu\nu}(k)\int\nolimits_{\Sigma}d\mu_g
g^{mn}\partial_m x^\mu \partial_n x^\nu
e^{ikx}+\alpha\eps_\mu^\mu\int\nolimits_{\Sigma}
d\mu_g R_g e^{ik\cdot x}\,\,.
$$
The physical particle corresponding to this vertex
operator is the dilaton.
Recall that the dilaton field $\Phi$
entered the generalized sigma model
action with a Gaussian curvature term $R_g$ as well.

\medskip\noindent
3)\enspace
Higher mass states: the general structure of vertex
operators is given by
$$
V_\eps(k)=\eps_{\mu_1\ldots\mu_{2p}}\int\nolimits_{\Sigma}
d\mu_g g^{m_1n_1}\ldots g^{m_pn_p}\partial_{m_1}
x^{\mu_1}\ldots\partial_{n_p}x^{\mu_{2p}}
e^{ik\cdot x}
$$
plus terms involving higher derivatives on $x$.
To cancel the overall exponential Weyl dependence, we
must require $k^2=-2p+2$, yielding the same mass
spectrum as in the free string case.
In complete analogy with the case 2), there will be
further conditions of transversality on $\eps$ and
and further terms involving $R_g$ will have to be
added to render the trace parts Weyl invariant.

\bigskip\noindent
\Item{\bf H.} {\bf Non-Critical Strings}

\smallskip
Conformal matter field theories with $c\not=26$ are
called {\it non-critical string theories}.
The amplitudes for these theories retain a
non-trivial Weyl dependence through the Liouville
action, and through the vertex operators.
Its direct quantization requires understanding the
dynamics of Liouville theory.
Despite much effort, there is no complete picture to
date of how this can be achieved consistently, and
without the presence of a tachyon.

A tremendous number of results was obtained
indirectly (by matrix model techniques) when
$c$ is less than $1$ and rational.

\bigskip
\Item{\bf I.} {\bf Tree Level Amplitudes}

\smallskip
Tree level amplitudes correspond to the worldsheet
topology of a sphere $S^2$, which we
stereographically project onto the complex plane
$\dbC$.
The moduli space $\scrM_0$ of the sphere consists of
just a single point, which we may choose to represent
by the {\it round metric}, of curvature $R_{\ghat}=1$,
and area $4\pi$:
$$
\ghat={4\vert dz\vert^2\over(1+\vert z\vert^2)^2}
\qquad\qquad z\in\dbC\,\,.\eqno{(3.57)}
$$
The group of conformal automorphisms is $\PSL(2,\dbC)$
and acts by M\"obius transformations
$$
\zeta\colon\, z\to \zeta(z)={az+b\over cz+d}\qquad
\qquad \pmatrix{a &b\cr c &d\cr}\in \PSL(2,\dbC)\,\,.
$$
A convenient basis for the associated vector fields
(called conformal Killing vectors) is
$$
\psi_1^z=1\qquad\qquad
\psi_2^z=z\qquad\qquad
\psi_3^z=z^2\,\,,
$$
and similarly for $\psi_a^{\zbar}$, $a=1,2,3$.

To obtain the general expression for tree level
transition amplitudes, we must first complete the
treatments of conformal automorphisms, which we had
postponed in \S{F}, and of Weyl transformation of the
measure.
The starting point is our expression (3.28) for the
measure $Dg$, considered now for $h=0$
$$
Dg=D_\sigma \Dtil v Z_{(-1)}^-(\ghat)
e^{-26 S_L(\ghat,\sigma)}\eqno{(3.58)}
$$
where $\Dtil v=D'v\det(\psi_a,\psi_b)_g$,
$\psi_a\in\Ker\nabla_{\zbar}^{(-1)}$, and the measure
$D'v$ is over $\Diff_0(\Sigma)$ vector fields orthogonal
to $\Ker\nabla_{\zbar}^{(-1)}\oplus\Ker\nabla_z^{(1)}$.
Tree level amplitudes are then given by
$$
A_0=\int\nolimits_{\Weyl(\Sigma)}D\sigma
\int\nolimits_{\Diff_0(\Sigma)/\PSL(2,\dbC)}
\Dtil v/\scrN(g) Z_{(-1)}^-(\ghat)
\left<V_1\ldots V_N\right>_{\ghat}\eqno{(3.59)}
$$
using the same notations as in III.19.

While the group $\PSL(2,\dbC)$ leaves the metric $g$
invariant (up to Weyl rescaling), 
it does transform the positions of vertex
operator insertions non-trivially.
It is natural and --- as we shall establish later on ---
required for unitarity of the transition amplitudes, 
to divide by the (formal) volume of the {\it full}
$\Diff_0(\Sigma)$ group, and not just of
$\Diff_0(\Sigma)/\PSL(2,\dbC)$.
To achieve this, we proceed as follows.

\bigskip\noindent
{\it 1)\enspace For $N\Ge 3$ cases}

Let the last three vertex operators be given by (of
course, we may choose {\it any} three vertex operators)
$$
V_i=\int\limits_{\dbC}d^2z_i\,W_i(z_i)\,\,,\qquad\qquad
i=N-2,\,N-1,\,N\,\,.
$$
We propose to make a change of variables from the vertex
insertion points $z_i$, $\zbar_i$
$i=N-2,\,N-1,\,N$ to the group elements of
$\PSL(2,\dbC)$, thus completing the measure $D'v$ into
$Dv$.
This is possible since, given any three points
$z_i\in\dbC$, there is a unique element
$\zeta\in\PSL(2,\dbC)$, such that $\zeta$ maps $z_i^0$ onto
any triplet $z_i\in\dbC$, with
$$
z_i=\zeta(z_i^0)\,\,.
$$
to carry this out in detail, let $\psi_a^z$ be a basis of
the $\Ker\nabla_{\zbar}^{(-1)}$, $a=1,2,3$, chosen to be
Weyl invariant.
We parametrize general vector fields $v$ in
$\psl(2,\dbC)$ by complex parameters $\lam_a$,
$\lambar_a$ with
$$
v^z=\sum\limits_{a=1}^3 \lam_a\psi_a^z\qquad\qquad
v^{\zbar}=\sum\limits_{a=1}^3 \lambar_a\psi_a^{\zbar}\,\,.
$$
The measure on $\PSL(2,\dbC)$ vector fields is then
$$
d\mu_v^{(6)}=Dv/D'v=\prod\limits_{a=1}^3
d\lam_a\,d\lambar_a\,\,\det(\psi_a,\psi_b)_g
$$
where the determinant $\det(\psi_a,\psi_b)_g$ arises as 
the Jacobian of the coordinate change from $v$ to $\lam$.
On the other hand, from the action of $\PSL(2,\dbC)$ on
the points $z_i^0\colon\, z_i=\zeta(z_i^0)=e^v z_i^0$, we
have $dz_i=\sum\limits_{a}d\lam_a\,\psi_a(z_i)$, so that
the measure on $d^2 z_i$ becomes:
$$
\prod\limits_{i=N-2}^N d^2 z_i=\prod\limits_{a=1}^3
d\lam_a\,d\lambar_a\,\vert\det\,\psi_a(z_i)\vert^2\,\,.
$$
Clearly, all factors in this identity are Weyl-invariant.
Combining all the results obtained, we have
$$
\eqalign{
\Dtil v\prod\limits_{i=N-2}^N d^2z_i &=D'v\,
  \det(\psi_a,\psi_b)_g\prod\limits_{a=1}^3
d\lam_a\,d\lambar_a\,\vert\det\,\psi_a(z_i)\vert^2\cr
&=Dv\,\vert\det\,\psi_a(Z_i)\vert^2\cr}
\eqno{(3.60)}
$$
We may now divide by the formal volume of
$\Weyl(\Sigma)\ltimes \Diff(\Sigma)$ just as in the cases
$h\Ge 2$, and we obtain the final expression for tree
level amplitudes with $N\Ge 3$:
$$
A_0=z_{(-1)}^-(\ghat)\vert\det\,\psi_a(z_i)\vert^2
\bigl<V_1\ldots V_{N-3}\prod\limits_{i=N-2}^N
W(Z_i)\bigr>_{\ghat}\,\,,
\eqno{(3.61)}
$$
valid for any $3$-points $z_i$.
We shall evaluate some of these amplitudes momentarily
for flat space-time.

\bigskip\noindent
{\it 2)\enspace $N=0$, $1$, $2$ cases}

Clearly, here, no three vertex operators are available to
complete the integration $D'v$ into $Dv$.
In fact, these amplitudes vanish! 
To justify this, notice that $D'v$ can at most be
completed by the vector fields in $\PSL(2,\dbC)/S_N$
where $S_N$ is the stabilizer of $N$ points 
on the sphere.
Thus, we are left to divide out by the volume of $S_N$
$$
\int\nolimits_{\Weyl(\Sigma)}D\sigma
\int\nolimits_{\Diff_0/(\PSL(2,\dbC)/S_N)}
D'v/\scrN(g)\sim{1\over \Vol(S_N)}=0\,\,.
\eqno{(3.62)}
$$
For $N=2$, fix the two points at $0$ and $\infty$, and
$S_2$ is the group of dilations, whose volume is
$\infty$.

An alternative argument that the amplitudes should vanish
is that no conformal invariant amplitudes can be
constructed, except with value $0$.
The physical interpretations of these vanishing
amplitudes are as follows:

{\narrower{\narrower\smallskip
$$
\vbox{
\halign{#\hfill &\qquad #\hfill &\qquad #\hfill\cr
1) &$N=0$ &tree-level vacuum energy is zero\cr
&&``vanishing of the cosmological constant''.\cr
2) &$N=1$ &the ``tadpole'' amplitudes vanish,\cr
&&i.e. conformal field theory provides solutions\cr
&&(no linear terms) to string theory, at tree level.\cr
3) &$N=2$ &no mass corrections to tree level.\cr}
}
$$
\smallskip}}

\bigskip\noindent
{\it 3)\enspace Green function and determinants}

The scalar Green function on the sphere, with metric
$\ghat$, is given by
$$
G(z,z')=-\ln\,
{\vert z-z'\vert^2\over (1+\vert z\vert^2)(1+\vert
z'\vert^2)}\,\,.\eqno{(3.63)}
$$
The determinants involving $\Delta_{(0)}$ and
$\Delta_{(-1)}^-$ are just constants.
They may be computed explicitly, but we shall not do so
here.
Instead, we abbreviate
$$
\grz\equiv Z_{(-1)}^-(\ghat)
\left({\Det'\Delta_{(0)}\over
\int\limits_{\Sigma}d\mu_{\ghat}}\right)^{-13}\,\,.\eqno{(3.64)}
$$
Furthermore, the only remaining finite-dimensional
determinant is
$$
\eqalign{
\vert\det\,\psi_a(z_i)\vert^2 &=\vert\det
\pmatrix{
1\hfill &1\hfill &1\hfill\cr
z_{N-2}\hfill &z_{N-1}\hfill &z_N\hfill\cr
z_{N-2}^2\hfill &z_{N-1}^2\hfill &z_N^2\cr}\vert^2\cr
\noalign{\medskip}
&=\vert(z_{N-2}-z_{N-1})(z_{N-1}-z_N)(z_N-z_{N-2})\vert^2\,\,.
\cr}
\eqno{(3.65)}
$$

\bigskip\noindent
\noindent
{\it 4)\enspace Tachyon Amplitudes}

The tachyon vertex operator is particularly simple:
$$
V_T(k)=\int\nolimits_{\Sigma}d^2 z\,W_T(z,k)\qquad\qquad
W_T(z,k)=\eps \sqrt{\ghat\,\,}e^{ik\cdot x}\,\,.\eqno{(3.66)}
$$
Here, $\eps$ is a constant, independent of $k$, and plays
the same r\^{o}le as the field normalization
 $Z$-factors in the Lehmann Symanzik Zcmmerman (LSZ)
formalism in quantum field theory:
$\left<k\vert\varphi(x)\vert0\right>=Z\,e^{ik\cdot x}$
for a scalar field $\varphi(x)$ and its associated
$1$-particle state $\left.\vert k\right>$ momentum $k$,
and $\left.\vert 0\right>$ the ground state.

We now make use of the results in p. II.23 on the
correlation function of exponential operators.
We introduce the renormalized $\eps_R$ by absorbing a
multiplicative renormalization factor of the exponential
$$
\eps_R=\eps\,e^{-G_R(z_i,z_i)_{\ghat}}\eqno{(3.67)}
$$
The amplitude now takes the form
$$
A_0=\delta(k)e^{-2\Phi_0}\eps_R^N\,\,\grz
\prod\limits_{i=1}^{N-3}\int\nolimits_{\dbC}
d^2z_i\prod\limits_{i=1}^N\sqrt{\ghat(z_i)\,\,\,}e^{-{1\over 2}
\sum\limits_{i\not=j}k_i\cdot k_j G(z_i,z_j)}
\cdot\vert\det(\psi_a(z_i))\vert^2\,\,.\eqno{(3.68)}
$$
The contributions to $G$ of the form $\ln(1+\vert
z_i\vert^2)$ sum up and, using momentum conservation and
$k_i^2$, precisely cancel the $\sqrt{\ghat\,\,}$
prefactors.
We are left with
$$
A_0(k_1\ldots k_N)=\delta(k)e^{-2\Phi_0}\eps_R^N\,\,\grz
\vert\det\,\psi_a(z_i)\vert^2\prod\limits_{i=1}^{N-3}d^2z_i
\prod\limits_{i<j}^N\vert z_i-z_j\vert^{2k_i\cdot k_j}
\eqno{(3.69)}
$$

Now, $z_{N-2}$, $z_{N-1}$, $z_N$ were {\it any} 3 points
on the space, so we may let them equal $0$, $1$, $\infty$,
respectively, and we find
$$
A_0(k_1,\ldots,k_N)=\delta(k)e^{-2\Phi_0}\eps_R^N\,\,
\grz\prod\limits_{i=1}^{N-3}\int\nolimits_{\dbC}
d^2z_i \prod\limits_{i<j}^{N-1}\vert
z_i-z_j\vert^{2k_i\cdot k_j}\,\,.
\eqno{(3.70)}
$$
It is very instructive to examine amplitudes for small
$N$:

\medskip\noindent
(1)\enspace\undertext{three point function}
$$
A_0(k_1,k_2,k_3)=\grz\,\, e^{-2\Phi_0}\eps_R^3\delta(k)\,\,.
\eqno{(3.71)}
$$
This $3$-point function provides the $3$-tachyon on shell
coupling.

\medskip\noindent
(2)\enspace\undertext{four point function}
$$
A_0(k_1,\ldots,k_4)=\grz\,\, e^{-2\Phi_0}\eps_R^4\delta(k)
\int\nolimits_{\dbC}d^2z\,\, \vert z\vert^{2k_1\cdot k_2}
\vert 1-z\vert^{2k_1\cdot k_3}\,\,.
\eqno{(3.72)}
$$
It is standard to introduce the $3$ Lorentz invariants
$s$, $t$, $u$
that characterize the kinematics of the $4$-point
function:
$$
\lower25pt\hbox{\vbox{\epsfxsize=1.75in\epsfbox{fig5.eps}}}\qquad
\cases{
s\equiv -(k_1+k_2)^2 &\cr
t\equiv -(k_2+k_3)^2 &\cr
u\equiv -(k_1+k_3)^2\,\,. &\cr}
\eqno{(3.73)}
$$
(Note that $s+t+u=-8$.)
In terms of $s$, $t$, $u$ the four point function is
$$
A_0(k_1,\ldots,k_u)=\grz\,\, e^{-2\Phi_0}\eps_R^4\delta(k)
\int\nolimits_{\dbC}d^2z \vert z\vert^{-s-4}
\vert z-1\vert^{-u-4}\,\,.\eqno{(3.74)}
$$
This integral expression is absolutely convergent in the
following region $\scrD$, defined by
$$
\cases{
\re(s)<-2 & \cr
\re(u)<-2 &\cr
\re(t)<-2 &\cr}\qquad\qquad\hbox{or}
\qquad\qquad\cases{
\re(s)<-2 &\cr
\re(t)<-2 &\cr
\re(s+t)>-6\,\,. &\cr}
\eqno{(3.75)}
$$
Taking just the real parts:
$$
\vbox{\epsfxsize=3.0in\epsfbox{fig6.eps}}
$$
Let $s,t\in\scrD$, then we may evaluate the amplitude by
elementary methods, and we find the {\it
Virasoro-Shapiro} amplitude:
$$
A_0(s,t)=\grz e^{-2\Phi_0}\eps_R^4\delta(k)\pi\,
{\Gamma(-1-s/2)\Gamma(-1-t/2)\Gamma(-1-u/2)\over
\Gamma(2+s/2)(\Gamma(2+t/2)\Gamma(2+u/2)}\,\,.
\eqno{(3.76)}
$$
Since we know the analytic continuation of $\Gamma(z)$
throughout $\dbC$, with poles only at $z\in -\dbN$, we
automatically have an analytic continuation of the full
amplitude $A_0(s,t)$, throughout
$(s,t)\in\dbC\times\dbC$.
As a byproduct, we also immediately have the amplitude in
Minkowski space-time momenta, by letting
$$
\eqalignno{
&\null\quad
\raise30pt\hbox{$k_i^0\to -ik_i^0$}\qquad\qquad\qquad
\vbox{\epsfxsize=1.5in\epsfbox{fig7.eps}}\cr
&\cases{
s\to s+i\epsilon &\cr
t\to t+i\epsilon &\null\qquad\qquad $\delta(k)\to i\delta(k)$\cr
u \to u+i\epsilon &\cr}\cr} 
$$
where it is understood that $\epsilon>0$ and that we only
consider the limit in which $\epsilon\to 0$.
Thus, the $4$-point amplitude in Minkowski space-time is
given by
$$
A_0(s,t)=\grz e^{-2\Phi_0}\epsilon_R^4 \delta(k)\pi i
{\Gamma(-1-2/2-i\epsilon)\Gamma(-1-t/2-i\epsilon)
  \Gamma(-1-u/2-i\epsilon)\over
\Gamma(2+s/2+i\epsilon)\Gamma(2+t/2+i\epsilon)
  \Gamma(2+u/2+i\epsilon)}\,\,.\eqno{(3.77)}
$$
The amplitude has only pole singularities --- as expected
at tree level --- and their location is easily
identified.
The $\Gamma$-function contributions from the denominator
are entire.
{}From the numerator, we have poles at
$$
\cases{
s=-2+2n_s &$n_s\in\dbN$\cr 
&\cr
t=-2_2n_t &$n_t\in\dbN$\cr 
&\cr
u=-2+2n_u &$n_u\in\dbN\,\,.$\cr}
\eqno{(3.78)}
$$
These poles precisely correspond to the creation of
intermediate string states whose mass is the above value
for $s$, $t$ or $u$.
The states may occur either in the {\it $s$, $t$ or $u$
channel\/}:
$$
\kern.75true cm\vbox{\epsfxsize=5.00in\epsfbox{fig8.eps}}
$$
It is instructive to identify where these singularities
arise from in the integral representation for the
amplitude:
$$
\eqalign{
&s\hbox{ poles:} \qquad z\to 0\cr
&t\hbox{ poles:} \qquad z\to\infty\cr
&u\hbox{ poles:} \qquad z\to 1\,.\cr}
\qquad\qquad
\lower15pt\hbox{\vbox{\epsfxsize=2.00in\epsfbox{fig9.eps}}}
$$
The fact that three QFT diagrams arise from a single string
diagram is the simplest example of a general
phenomenon mentioned in the introduction.

Let us concentrate on the $s=-2$ pole, which corresponds
to a tachyon intermediate state.
The diagram there {\it factorizes} into a propagator, and
2 three point functions for external tachyon:
$$
\vbox{\epsfxsize=4.50in\epsfbox{fig10.eps}}
$$
Insisting upon this factorization, we obtain a non-linear
equation between the $3$-point function and $4$-point
function, expressing factorization --- or unitarity ---
of the transition amplitudes (or so-called $S$-matrix
elements).
This relation fixes $\eps_R$:
$$
\eps_R^2={2\pi\over \grz}\,e^{2\Phi_0}\,\,.
\eqno{(3.79)}
$$

\vfill\eject

\noindent
\Item{\bf J.} {\bf One loop amplitudes}

\smallskip
At one loop, the worksheet has the topology of a torus,
and there is a single complex modulus $\tau\in\dbC$.
We represent $\Sigma$ by
$$
\Sigma=\dbC/(\dbZ+\tau Z)\qquad\qquad\qquad
\im\,\tau>0
$$
with the flat metric $\ghat=2\vert dz\vert^2$.
Teichm\"uller space is $\scrT_1=\{\tau\in\dbC,\,\,
\im\,\tau>0\}$, the mapping class 
group is $G_\Sigma=\SL(2,\dbZ)$,
acting on $\tau$ as $\PSL(2,\dbZ)$ by
$$
\tau\to\tau'={a\tau+b\over c\tau+d}\qquad\qquad
\left(\matrix{a &b\cr c &d\cr}\right)\in\SL(2,\dbZ).
\eqno{(3.80)}
$$
Moduli space is conveniently taken as
$\scrT_1/\PSL(2,\dbZ)=\scrM_1$, but we shall have to recall
that the full mapping class group also divides out by
$\{\pm I\}$ in $\SL(2,\dbZ)$.
It is standard to take the fundamental domain
$$
\scrM_1=\left\{\tau=\tau_1+i\tau_2, \tau_{1,2}\in\dbR,
\,\,\tau_2>0,\,\,
\vert\tau_1\vert\Le \half,\,\,
\vert\tau\vert\Ge 1\right\}\,\,.
\eqno{(3.81)}
$$

Determinants of Laplace operators on tensors of any
integer weight are given by the case of Laplacians on
functions, which was computed by Gaw\c{e}dzki:
$$
\eqalign{
\Det'\Delta_{(0)} &=\tau_2^2\,\vert\eta(\tau)\vert^4\cr
\eta(\tau)\equiv e^{i\pi\tau/12}
&\prod\limits_{n=1}^\infty(1-e^{2\pi i n\tau})\,\,.\cr}
\eqno{(3.82)}
$$
The Green function on the scalar field $x$ is given by
$$
G(z,w\vert \tau)=-\ln
\left\vert{\vartheta_1(z-w\vert\tau)\over
\vartheta'_1(0\vert\tau)}\right\vert^2-
{\pi\over 2\tau_2}(z-w-\zbar+\wbar)^2\,\,,\eqno{(3.83)}
$$
where $\vartheta_1(z\vert\tau)$ is the Jacobi theta
function that has a zero at $z=0$.

The $N$-string transition amplitude is given by the
following expression
$$
A_1(k_1,\ldots,k_N)=\int_{\scrM_1}
{d^2\tau\over 2\tau_2^2}{1\over (4\pi^2\tau_2)^{12}}
{1\over \vert\eta(\tau)\vert^{48}}
\ll V_1,\ldots,V_N\gg\eqno{(3.84)}
$$
where $\ll V_1\ldots V_N\gg$ stands for the {\it
normalized} correlation function of the vertex operators
$V_1\ldots V_N$: $\ll V_1\ldots V_N\gg=\left<V_1\ldots
V_N\right>/\left<1\right>$.

For example, considering only tachyon vertex operators,
we have
$$
\ll V_1\ldots V_N\gg=\delta(k)\eps _R^N
\prod\limits_{i=1}^N \int d^2z_i\exp\left(
-\half \sum\limits_{i\not=j}k_i\cdot k_j
G(z_i,z_j\vert\tau)\right)\eqno{(3.85)}
$$
There is no dependence on the dilaton constant $\Phi_0$
here, because the Euler number on the torus vanishes.
Other correlation functions may be similarly derived.

To one loop order, we may now also have non-zero
$N=0,1,2$ point functions.
Let us consider $N=0$ first:
$\ll 1\gg=1$.
$$
A_1(\,\,\cdot\,\,)=\Vol(M)\int_{\scrM_1}
{d^2\tau\over 2\tau_2^2}\,\,{1\over (4\pi^2\tau_2)^{12}}\,\,
{1\over \vert\eta(\tau)\vert^{48}}\,\,.\eqno{(3.86)}
$$
Using the asymptotics for $\im\,\tau\to\infty$,
$$
\eta(\tau)\sim e^{i\pi \tau/12}=e^{i\pi \tau_1/12}
e^{-\pi \tau_2/12}\eqno{(3.87)}
$$
we see immediately that this amplitude diverges for large
$\tau_2$.
In fact, this $\eta(\tau)$ factor is common to all
amplitudes, for any type of vertex operators, and
produces a divergence in any amplitude, for any
arrangement of external momenta.
We shall not show this here in general, but leave this
point to be checked by the reader.
Thus, bosonic closed oriented string theory is divergent
in flat space-time.

\bigskip\noindent
{\it Nature of Divergence}

Let us try to gain additional insight into the physical
origin of this divergence.
To do so, it is helpful to exhibit the dependence on
internal momenta by inserting a factor:
$$
1=\int d^{26}k\,\,\delta\left(k-{1\over 2\pi i}
\oint_A dz\,\partial_z x\right)\,\,,\eqno{(3.88)}
$$
where $A$ is the cycle in $\Sigma$ whose pre-image in
$\dbC$ runs from $z=0$
to $z=1$.
We find
$$
A_1(\,\,\cdot\,\,)=\Vol(M)\int\limits_{\dbR^{26}} 
d^{26} k\int_{\scrM_1}
{d^2\tau\over 2\tau_2}\,\,{1\over (4\pi^2)^{12}}\,\,
{1\over \vert\eta(\tau)\vert^{48}}e^{-2\pi\tau_2 k^2}\,\,.
\eqno{(3.89)}
$$
$$
\vbox{\epsfxsize=4.00in\epsfbox{fig11.eps}}
$$
Substituting the asymptotics of the $\eta(\tau)$-function
$$
A_1(\,\,\cdot\,\,)=\Vol(M)\int\limits_{\dbR^{26}} d^{26}k
\int_{\scrM_1}{d^2\tau\over 2\tau_2}\,\,
{1\over (4\pi^2)^{12}}e^{-2\pi\tau_2(k^2-2)}
(1+\#e^{-2\pi\tau_2}+\cdots\,\,\,)\,\,,\eqno{(3.90)}
$$
it is now immediately apparent why a divergence occurs,
and where it occurs.
For $\tau_2\to\infty$, the momentum range $0\Le k^2<2$
contributes an exponentially growing factor.
In fact, integrating out $\tau_2$ for large $\tau_2$:
$$
\int\,\,{d\tau_2\over \tau_2} e^{-2\pi\tau_2(k^2-2)}\sim
{1\over k^2-2}
$$
This contribution arises from the tachyon!
It is an I.R. problem related to the stability of flat
space-time.

It is equally instructive to see where the integral is
convergent: at large $k^2$, since the range of $\tau_2$
is cut off from below by the choice of $\scrM_2$, the
integral is exponentially suppressed for large momenta
$k^2$.
Thus, this loop amplitude is superconvergent at high
energy, in contrast with field theory.

In fact, we can use this simple example to exhibit the
distinction between what this partition function would
look like in quantum field theory and what it looks like
in string theory.
In field theory, we may imagine the partition function
for the same states, given by the same generating
function.
The only difference is the region $\scrM_1$:
$$
\eqalignno{
\vbox{\epsfxsize=2.00in\epsfbox{fig12.eps}}
&\qquad
\raise50pt\hbox{\vtop{\hbox{\rm Integration region for QFT}
\hbox{\rm with the same states as string theory.}}}\cr}
$$
Now, since $\tau_2$ reaches down to $\tau_2=0$, the
superconvergence disappears and the integrand is power
behaved in momenta.
Thus, we see here that a crucial difference between
string theory and quantum field theory arises from
modular invariance, producing ultraviolet finite answers
in the case of strings



\bye



