%Date: Fri, 24 Jan 1997 12:07:32 -0500
%From: Edward Witten <witten@sns.ias.edu>

\input harvmac
Spring Term
Problem Set 1

(1) The strong interactions are described to good approximation
by a theory which has a symmetry group $G=SU(2)\times SU(2)$
spontaneously broken down to a diagonal subgroup $H\cong SU(2)$.
There are no massless particles except the Goldstone bosons, which,
to the extent $G$ is a valid symmetry, are exactly massless.
The theory, to good approximation, has a $G$-invariant Lagrangian $L$.

\def\O{{{\cal O}}}
\def\tilde{\widetilde}
However, $G$ is not an exact symmetry of nature.  It is broken
by perturbations that transform in the $(2,2)$ representation of $G$
(that is, the tensor product of the two dimensional  representations
 of the two $SU(2)$'s.)
Among the operators of the $G$-invariant theory are two ``multiplets''
of operators $\O_i$ and $\tilde \O_j$, $i,j=1\dots 4$, each
transforming as $(2,2)$.  The Lagrangian is, more accurately,
\eqn\duffo{\tilde L= L+\sum_{i=1}^4\left(\epsilon_i\O_i+\tilde \epsilon_i
\tilde O_i\right).}
The $\epsilon_i$ and $\tilde \epsilon_j$ are, of course, all real.
(That there are precisely two possible multiplets $\O$ and $\tilde \O$ of
perturbations is explained by the underlying gauge theory, but for the
present exercise we simply assume it.)

For a generic choice of the $\epsilon_i$ and $\tilde \epsilon_i$
the exact symmetry group of the model is only $H'=U(1)$, and this is,
in fact, the situation in nature.  

Treat the effective theory to first order in $\epsilon$ and $\tilde\epsilon$
(which are in fact in nature small and comparable to each other).

(a) Show that, in first order, you would expect the degeneracy of the
vacuum to be lifted, and there to appear a unique vacuum with mass gap.
In particular, you expect the $H'$ symmetry to be unbroken.
(Abstractly, $H'$ can be embedded in $H$; the point that is slightly
non-trivial is that the perturbation picks       a vacuum, of the
unperturbed theory, that is invariant under an $H\cong SU(2)$ that actually
does contain $H'$.  $H$ could of course have been conjugated to a group
that does not contain $H'$.) In particular, there is no exact symmetry
that is spontaneously broken, and no exact Goldstone boson; the former
Goldstone bosons all get mass.

(b) More surprisingly, show that to first order in $\epsilon$ and $\tilde
\epsilon$, the three Goldstone bosons -- the one that is invariant
under $H'$ and the two that transform in a non-trivial representation
of $H'$ -- all get {\it equal} masses.  (This is also observed in nature,
to high precision, the particles being respectively the neutral and
charged pions.)

(c) The three Goldstone bosons will not have {\it exactly} equal masses.
In what order in $\epsilon$ and $\tilde \epsilon$ will the degeneracy
be lifted and how would you describe this lifting in the effective
Lagrangian?

(2) Here we consider (as in the standard model of weak interactions)
an $SU(2)\times U(1)$ gauge theory in four dimensions.
The $SU(2)$ and $U(1)$ gauge fields will be called $A_2$ and $A_1$,
the     curvatures $F_2$ and $F_1$, and the gauge couplings $e_2$ and $e_1$.
The Lagrangian of the pure gauge theory would be
\eqn\norgo{L=\int d^4x\left({1\over 4e_1^2}|F_1|^2+{1\over 4e_2^2}\Tr F_2^2
\right) }
where $\Tr $ is the trace in the two-dimensional representation of $SU(2)$.

In addition there are complex
scalar fields $\phi$ transforming in the  two-dimensional
representation of $SU(2)$ and with ``charge $1/2$'' for the $U(1)$.
Their Lagrangian is
\eqn\juurbo{L_\phi=\int d^4x\left(|D\phi|^2+{\lambda\over 8}\left(
|\phi|^2-v^2\right)^2\right).}
$v^2$ is positive.

(a) Show that $SU(2)\times U(1)$ is spontaneously broken to a diagonal
$U(1)$.

(b)  Determine the spectrum of the model: what are the massless and
massive scalars and vectors, and how do they transform under the
unbroken $U(1)$?

(c)  The quantities of the model that are observed to the highest
precision are the two gauge couplings and the masses of the charged
and neutral massive vectors.  This makes four observables.
As the model has four parameters ($e_1,$ $e_2$, $\lambda$, and $v$),
one might not expect any relation among these four observables.
However, the four observables are independent of $\lambda$ in the classical
approximation which (as the model is weakly coupled) is a good approximation.
So in practice the model does predict a relation between the two vector masses
(given a measurement of $e_1$ and $e_2$).  This relation was one
of the early successes of the electroweak standard model.

Of course, this relation will not be exact.
It will be modified by loops, and the loops will depend on additional
fields that are present in nature but were not mentioned so far
(and do not affect the classical discussion carried out here).
Present experimental accuracy is just about good enough to see the
loop effects, and thereby to get an estimate of the parameter $\lambda$
that does not enter the classical relation but does enter the loops.

\end



