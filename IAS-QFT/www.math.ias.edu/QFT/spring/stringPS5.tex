%From: "Eric D'Hoker" <dhoker@IAS.EDU>
%Date: Tue, 18 Feb 1997 15:34:54 -0500 (EST)

%%%%%%%%%%%%%%%%%%%%%%%%%%%%%%%%%%%%%%%%%%%%%%%%%%%%%%%%%%%%%%%%%%%%%%%%%%%
%%%%
%%%%    STRING THEORY : Problem Set 5, February 20, 1997
%%%%
%%%%%%%%%%%%%%%%%%%%%%%%%%%%%%%%%%%%%%%%%%%%%%%%%%%%%%%%%%%%%%%%%%%%%%%%%%%


\magnification=\magstep1
\overfullrule=0pt
\baselineskip=17pt
\def\det{{\rm det}}
\def\Det{{\rm Det}}
\def\tr{{\rm tr}}
\def\Tr{{\rm Tr}}
\def\12{{1 \over 2}}
\def\ker{{\rm Ker}}
\def\O{{\cal O}}

\centerline{{\bf STRING THEORY}}
\centerline{ Problem Set \# 5}
\centerline{February 20, 1997}

\bigskip
\bigskip

\noindent
{\it Problem 1}

\medskip

In this problem, we determine the dilaton vertex operator on 
a worldsheet $\Sigma$ with general metric $g$. Show that the operator
$$
V^{\Phi} (k) = \int _\Sigma d \mu _g
\{ g^{mn} \epsilon _{\mu \nu} (k) \partial _m x ^\mu \partial _n x ^\nu 
+ \alpha R_g
\} e^{ik\cdot x}
$$
is Weyl invariant (when inserted into correlation functions) 
provided certain conditions on the polarization
tensor $\epsilon _{\mu \nu}$ and the parameter $\alpha$ are satisfied.
(Here, $R_g$ is the Gaussian curvature of the metric $g$.)
Derive these conditions and solve for them 
(i.e. find $\epsilon _{\mu \nu}$ and $\alpha$, up to
an overall constant rescaling) for the vertex operator describing the
dilaton.

\bigskip

\noindent
{\it Problem 2}

\medskip

In this problem, we perform a number of calculations in closed
oriented string theory, in flat space-time, 
in the critical dimension $D=26$, and to tree level.

\noindent
a) Using the transition amplitude for the scattering of four
tachyons, deduce the strength of the interaction between two
tachyons as a function of distance, in the limit where they are far apart.

\noindent
b) Assuming that the effects of the couplings of the dilaton
and anti-symmetric tensor states can be ignored,
use the results of 
a) to derive a relation between Newton's constant of gravity,
the string coupling $e^{-2 \Phi _0}$, 
and the string tension $\kappa = 1/4\pi \alpha '$. 
(We have mostly set $\alpha '=2$ during the lectures,
so you will have to restore the $\alpha '$ dependence in your answer.)

\noindent
c) Compute the transition amplitude for the scattering of 3 tachyons,
and one graviton with polarization tensor $\epsilon _{\mu \nu}$.
Verify that the amplitude is invariant under infinitesimal
diffeomorphism transformations on the polarization tensor 
$\epsilon _{\mu \nu}$. Verify that the amplitude exhibits the
singularities (as a function of external momenta) expected on
physical grounds.
 
\bigskip

\noindent
{\it Problem 3}

\medskip

In this problem, we perform a number of calculations in open
oriented string theory, in flat space-time, 
in the critical dimension $D=26$, and to tree level.

\noindent
a) Construct a general expression for the transition 
amplitude for the scattering of N open string tachyons (to tree level).
(Hint : make use of the fact that open string worldsheets
may be conveniently obtained from closed string worldsheets
that have an involution.)

\noindent
b) Compute the four (open string) tachyon amplitude and verify that it
exhibits the singularities (as a function of external momenta)
expected on physical grounds.

\noindent
c) Compute the transition amplitude for the scattering  of 
one graviton and two photons. Give the simplest interaction
in quantum field theory that would produce this scattering
amplitude.


\end


