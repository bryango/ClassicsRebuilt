\input amstex
\documentstyle{amsppt}
\magnification 1200
\NoRunningHeads
\NoBlackBoxes
\document

\def\opsi{\overrightarrow{\psi}}
\def\h{\frak h}
\def\tW{\tilde W}
\def\Aut{\text{Aut}}
\def\tr{{\text{tr}}}
\def\ell{{\text{ell}}}
\def\Ad{\text{Ad}}
\def\u{\bold u}
\def\m{\frak m}
\def\O{\Cal O}
\def\tA{\tilde A}
\def\qdet{\text{qdet}}
\def\k{\kappa}
\def\RR{\Bbb R}
\def\be{\bold e}
\def\bR{\overline{R}}
\def\tR{\tilde{\Cal R}}
\def\hY{\hat Y}
\def\tDY{\widetilde{DY}(\g)}
\def\R{\Bbb R}
\def\h1{\hat{\bold 1}}
\def\hV{\hat V}
\def\deg{\text{deg}}
\def\hz{\hat \z}
\def\hV{\hat V}
\def\Uz{U_h(\g_\z)}
\def\Uzi{U_h(\g_{\z,\infty})}
\def\Uhz{U_h(\g_{\hz_i})}
\def\Uhzi{U_h(\g_{\hz_i,\infty})}
\def\tUz{U_h(\tg_\z)}
\def\tUzi{U_h(\tg_{\z,\infty})}
\def\tUhz{U_h(\tg_{\hz_i})}
\def\tUhzi{U_h(\tg_{\hz_i,\infty})}
\def\hUz{U_h(\hg_\z)}
\def\hUzi{U_h(\hg_{\z,\infty})}
\def\Uoz{U_h(\g^0_\z)}
\def\Uozi{U_h(\g^0_{\z,\infty})}
\def\Uohz{U_h(\g^0_{\hz_i})}
\def\Uohzi{U_h(\g^0_{\hz_i,\infty})}
\def\tUoz{U_h(\tg^0_\z)}
\def\tUozi{U_h(\tg^0_{\z,\infty})}
\def\tUohz{U_h(\tg^0_{\hz_i})}
\def\tUohzi{U_h(\tg^0_{\hz_i,\infty})}
\def\hUoz{U_h(\hg^0_\z)}
\def\hUozi{U_h(\hg^0_{\z,\infty})}
\def\hg{\hat\g}
\def\tg{\tilde\g}
\def\Ind{\text{Ind}}
\def\pF{F^{\prime}}
\def\hR{\hat R}
\def\tF{\tilde F}
\def\tg{\tilde \g}
\def\tG{\tilde G}
\def\hF{\hat F}
\def\bg{\overline{\g}}
\def\bG{\overline{G}}
\def\Spec{\text{Spec}}
\def\tlo{\hat\otimes}
\def\hgr{\hat Gr}
\def\tio{\tilde\otimes}
\def\ho{\hat\otimes}
\def\ad{\text{ad}}
\def\Hom{\text{Hom}}
\def\hh{\hat\h}
\def\a{\frak a}
\def\t{\hat t}
\def\Ua{U_q(\tilde\g)}
\def\U2{{\Ua}_2}
\def\g{\frak g}
\def\n{\frak n}
\def\hh{\frak h}
\def\sltwo{\frak s\frak l _2 }
\def\Z{\Bbb Z}
\def\C{\Bbb C}
\def\d{\partial}
\def\i{\text{i}}
\def\ghat{\hat\frak g}
\def\gtwisted{\hat{\frak g}_{\gamma}}
\def\gtilde{\tilde{\frak g}_{\gamma}}
\def\Tr{\text{\rm Tr}}
\def\l{\lambda}
\def\I{I_{\l,\nu,-g}(V)}
\def\z{\bold z}
\def\Id{\text{Id}}
\def\<{\langle}
\def\>{\rangle}
\def\o{\otimes}
\def\e{\varepsilon}
\def\RE{\text{Re}}
\def\Ug{U_q({\frak g})}
\def\Id{\text{Id}}
\def\End{\text{End}}
\def\gg{\tilde\g}
\def\b{\frak b}
\def\S{\Cal S}
\def\L{\Lambda}
\def\tl{\tilde\lambda}

\topmatter
\title Lecture II-5: The Bose-Fermi correspondence and its applications
\endtitle
\author {\rm {\bf Edward Witten} }\endauthor
\endtopmatter

\centerline{Notes by Pavel Etingof and David Kazhdan}

{\bf 5.1. 2-dimensional gauge theories with fermions.}

Today we will consider 2-dimensional gauge theories with fermions. 
We will work with Euclidean Lagrangians. 

In 2 dimensions, the two spinor representations $S_+,S_-$ are 1-dimensional
and complex. They are 
defined by the formula $\theta\to e^{\pm i\theta/2}$, $\theta\in 
[0,4\pi)$, where $\theta$ is the parameter on $Spin(2)$.  

Let $\Sigma$ be a Riemann surface with a specified Spin structure.
Then $S_+,S_-$ define holomorphic line bundles on $\Sigma$, which
are called the Spin bundles, and denoted by the same letter. 
We have $S_+S_-=\C$, and $S_+^2=T\Sigma,S_-^2=T^*\Sigma$. 
Sections of $\Pi S_-$ (respectively, $\Pi S_+$) will be called left 
(respectively, right) moving fermions, by analogy with the 
Minkowski picture.
  
We denote by $D_+:S_-\to S_+$, $D_-: S_+\to S_-$ the corresponding 
Dirac operators in Spin bundles. 

We will do gauge theory with gauge group $G$ (a compact Lie group).
Let $E_L,E_R$ be orthogonal, unimodular representations of $G$. 
Let $P$  be a principal G-bundle on $\Sigma$, and let $E_R,E_L$ 
denote the orthogonal vector bundles associated to 
the representations $E_R,E_L$.  
Let $\overrightarrow{\psi_+},\overrightarrow{\psi_-}$ 
be sections of the bundles $E_R\o S_+$, $E_L\o S_-$. 

The Lagrangian of a 2-dimensional gauge theory with fermions is
$$
\Cal L=\int d^2x\left(\frac{1}{4e^2}|*F_A|^2+\frac{1}{4\pi}
\overrightarrow{\psi_+}(D_-^A)\overrightarrow{\psi_+}
+\frac{1}{4\pi }\overrightarrow{\psi_-}(D_+^A)
\overrightarrow{\psi_-}\right),\tag 5.1
$$
where $A$ is a connection in $P$, and $D_\pm^A$ are the corresponding 
Dirac operators.

In order for this theory to make sense quantum mechanically, the 
representations $E_L,E_R$ have to satisfy an additional condition.
To derive this condition, recall that the partition function of
(5.1) is given by 
$$
Z=\int DA\int D\overrightarrow{\psi_+}D\overrightarrow{\psi_-}e^{-\Cal L}.\tag 5.2
$$ 
(in this integral, we sum over all topological types of principal bundles).
The fermion integral is easy to compute: it equals
$I_A=Pf(D_-^A|_{E_R})Pf(D_+^A|_{E_L})$, where $Pf$ denotes the Pfaffian. 
The expression $I_A$ is a section of the Pfaffian line bundle
$B=PF(D_-^A|_{E_R})PF(D_+^A|_{E_L})$ on the space of 
gauge classes of connections. 
In order for the A-integral to make sense, this expression 
should be a function, i.e. the bundle $B$ has to be trivial.
It is easy to show that this boils down to the condition
$$
\Tr(\rho_L(t)\rho_L(t'))=\Tr(\rho_R(t)\rho_R(t')),\tag 5.3
$$
where $t,t'\in \g$, where
$\g$ is the Lie algebra of the Gauge group $G$, 
and $\rho_{L,R}: \g\to SO(E_{L,R})$
are the representation maps. If $G$ is simple, and $E_L,E_R$ are irreducible, 
this condition means that the Casimirs of $E_L,E_R$ are the same.
Equation (5.3) is called the condition of cancelations of anomalies. 

Today we will consider a simple case: $G=U(1)$, and $E_L,E_R$ are irreducible 
2-dimensional real representations. In this case (5.3) says that 
$E_L,E_R$ are the same: $E_L=E_R=E$. We will take $E=\C$
with metric $|z|^2$, and $U(1)$ acting
by multiplication (but remember that tensor products 
$S_\pm\o E$ are over $\R$). We 
decompose $S_\pm\o E$ in  a direct sum of two 
2-dimensional representations. Using this decomposition, we will write 
 $\opsi_\pm=(\psi_\pm,\bar\psi_\pm)$, where
for $z\in U(1)$ one has $z(\psi,\bar\psi)=(z\psi,\bar z\bar\psi)$.

In the new notation, Lagrangian (5.1) has the form
$$
\Cal L=\int d^2x\left(\frac{1}{4e^2}|*F_A|^2+\frac{1}{2\pi}
\bar\psi_+D_-^A\psi_+
+\frac{1}{2\pi}\bar\psi_-D_+^A\psi_-\right),\tag 5.4
$$
We can also add to this Lagrangian a topological term:
$$
\Cal L_\theta=\Cal L-\frac{i\theta}{2\pi}\int F,\tag 5.5
$$
and a mass term:
$$
\Cal L_{\theta,m}=\Cal L_\theta+\frac{m}{2\pi}\bar\psi_-\psi_++
\frac{\bar m}{2\pi}\bar\psi_+\psi_-,
\tag 5.6
$$
(here $m$ is complex). 

{\bf 5.2. Chiral symmetry.} 

Chiral symmetry is a $U(1)$-symmetry of the classical Lagrangian (5.4) or (5.5)
given by
$$
\psi_+\to \psi_+,\psi_-\to e^{i\gamma}\psi_-.\tag 5.7
$$
This symmetry is violated by the mass term. However, even if there is
no mass term, this symmetry may be violated quantum-mechanically, for the 
following topological reason.

Let the spacetime be a closed Riemann surface. Then 
 under the chiral symmetry, 
the ``measure'' $\mu=D\psi_-D\bar \psi_-$ transforms as  
$\mu\to e^{i\gamma I}\mu$, where $I$ is the index of the operator $D_+^A:
S_-\o_{\C} E\to S_+\o_\C E$. ``Proof'': $I=dim(\{\psi_-\})-dim
(\{\bar\psi_-\})$; since $(\{\bar\psi_-\})=(\{\psi_-\})^*$, 
we have $I=dim(\{\psi_-\})-dim
(\{\psi_+\})=ind(D_+^A|_{S_+\o_{\C}E})$.

It is known that the index $I$ equals $c_1(P)=\int \frac{F_A}{2\pi}$. 
Thus, since the bundle $P$ may be nontrivial topologically, chiral symmetry 
is violated in the quantum theory. 

This effect is called an anomaly: a classical symmetry 
does not hold quantum mechanically, because the measure is not invariant. 
The difference with spontaneous symmetry breaking, discussed in 
Lecture II-1, is that in the case of spontaneous symemtry breaking, 
the classical symmetry does exist in the quantum theory, but cannot be
realized. Unlike spontaneous symmetry breaking, 
anomalies can occur even in 1 dimension. 

Let us discuss the mechanism of chiral symmetry breaking in the language 
of currents. Classically, chiral symmetry is generated by the current
$$
J_A=\bar\psi_-\psi_-\tag 5.8
$$
(this is a 1-form of type (0,1)),
which is of course conserved: $dJ_A=0$. 
Quantum mechanically, this differential equation 
may be deformed: $dJ_A=\O$, where $\O$ is some operator
with values in 2-forms on the surface. 
It is not hard to show by listing all possible operators
that there is only one operator 
(up to a factor) that can arise: it is the curvature operator $F$. 
Namely, it is enough to show that $dJ_A$ is a functional of $F$, which can 
be seen by considering Feynman diagrams.  
Thus, $dJ_A=\alpha F$, where $\alpha$ is a constant, and the previous 
topological computation shows that $\alpha=1/2\pi$. 

Now consider Lagrangian (5.5), which depends on the theta-angle. 
The thing we learn from the above index computation is that the value of
the correlation functions 
defined by the path integral with Lagrangian (5.5) depend on $\theta$
in a very trivial way. 
Namely, if $\O_i$ are any operators such that $\prod_i \O_i\to e^{in\gamma}
\prod_i \O_i$ under chiral symmetry, then all nontrivial contributions 
to the correlator $\<\prod \O_i\>$ are from bundles $P$ with $c_1=n$, so 
this correlator has the form $e^{in\theta}\<\prod O_i\>_0$, where 
$\<\prod O_i\>_0$ is $\theta$-independent. For example, 
$\<\psi_-\psi_+^*\>=Ce^{i\theta}$, where $C$ is theta-independent. 

If instead of (5.5) we consider Lagrangian (5.6) with the mass term, 
then this argument shows that 
$$
\<\prod \O_i\>(m,\theta)=
e^{in\theta}\<\prod \O_i\>_0(\tilde m),\tag 5.9
$$
where $\tilde m=me^{-i\theta}$. Thus, the really important parameter 
of the theory is $\tilde m$, which we will write as $m_*e^{-i\theta_*}$,
$m_*\ge 0$. Our goal in this lecture to study this theory 
as a function of $\tilde m$. 

{\bf 5.3. Behavior of 2-dimensional gauge theory with massive fermions}

Now we will describe the behavior of the theory defined by 
Lagrangian (5.6), and later will justify this conclusion. 

First of all,  we will see that for large $m_*$ (i.e. $m_*>>e$) 
the theory is similar to the 2-dimensional gauge theory with 
massive bosons, considered in the previous lecture. In particular, it has
a mass gap. It also has a unique vacuum for $\theta\ne \pi$, and 
two of them for $\theta=\pi$. The 
discrete symmetry of reversal of space
orientation ($t\to t,x\to -x$), which acts by $m_*\to m_*$, 
$\theta_*\to -\theta_*$, is broken on the negative real axis
(far away from $0$), but not on positive real axis.   

Next, we will show that for $\tilde m=0$ the theory is in fact free,
i.e. becomes a free massive theory after a change of variables.
Thus the theory has a unique vacuum and a mass gap for small $m_*$
(i.e. for $m_*<<e$).  

Unfortunately, we do not know for sure what happens in the region $m_*\sim e$. 
The most natural thing would be
that the cut in the plane of $m_*$, representing points with symmetry 
breaking which starts at $-\infty$, ends at some point $-m_c$,
$m_c=e/\l$, and $\l$ is  dimensionless. 
At $\tilde m=-m_c$, the theory should have no mass gap (since it is a point 
of transition from two vacua to one vacuum).

This is what is in fact believed. Furthermore, there is a conjecture 
that the theory at the critical value $\tilde m=-m_c$ is in fact conformal, 
and isomorphic to the theory of a free neutral fermion. 

{\bf 5.4. Heavy fermions.}

In this section we will study the case $m_*>>e$. In this case we can regard 
our theory as a perturbation of a theory with $e=0$ with perturbation 
parameter $\l=e/m_*$. At $\l=0$, we have a direct product of a 2-dimenisonal
pure gauge theory (which is free), and a free theory of massive 
fermions (to be safe here, we should introduce $B=A/e$; this makes sense, as 
for $e=0$, only the trivial $U(1)$-bundle contributes to the path integral). 

It turns out that the situation here is similar to the bosonic case.
Namely, the small $e$ perturbation of the free theory for $e=0$ is singular. 
This means, the space of states 
of the deformed theory is actually smaller
than that of the undeformed theory. More precisely, confinement of 
fermions takes place: the only allowed states 
(for $\theta\ne \pi$) are 
states of total charge $0$ (here the charge of $\psi_\pm$ is 
$1$ and of $\bar\psi_\pm$ is $-1$, and $\theta\in [0,2\pi)$). 
In particular, the fermions can 
only exist in pairs, quadruples, and so on, and a single fermion cannot 
exist.   

Like in the bosonic case, the theory has a mass gap
by deformation argument (the fact that the deformation is singular
does not invalidate this argument, since the Hilbert space does not 
increase but only becomes smaller). More precisely, we have 
one realization for $\theta\ne \pi$, and two of them for $\theta=\pi$
(as in the pure gauge theory), and any realization has a mass gap.


{\bf 5.5. Bose-Fermi correspondence.}

Before studying the case of light fermions, we will consider Bose-Fermi 
correpondence, which will be useful in studying the case of small $m_*$. 

Consider two 2-dimensional free field theories

1. The fermionic theory defined by the Lagrangian
$$
\Cal L_f=\frac{1}{2\pi}
\int d^2x(\bar\psi_-D_+\bar \psi_-+\bar\psi_+D_-\bar\psi_+).
\tag 5.10
$$

2. The bosonic theory defined by the Lagrangian 
$$
\Cal L_b=\frac{1}{4\pi R^2}\int d^2x |d\phi|^2,\tag 5.11
$$

These two theories are conformal (both classically and quantum-mechanically),
and have Virasoro central charge 1. So we can suspect there may be some 
connection between them. And indeed, such a connection exists, and 
it is called the Bose-Fermi correspondence. 

{\bf Remark.} Conformal field theories with small central charge are very 
scarce. For instance, for $c<1$ they are completely classified
by Friedan, Qiu, Shenker (Ref ???). The answer is that there is no continuous 
parameters, and the theory is almost completely determined by $c$,
which can take only a discrete sequence of values. For $c=1$, such a 
classification is unavailable, but very few examples are known, and 
all of them have a construction in terms of a free Bose field. 

Let us now take a look at the bosonic theory (5.11). 
As we remember from Lecture II-1, in order for this theory to make sense, 
$\phi$ has to be angle-valued, i.e. take values in the circle
$\R/2\pi \Z$. The constant $R$ in the Lagrangian has the meaning 
of the radius of this circle.
 
{\bf Remark.} We always model the target circle as $\R/2\pi \Z$, 
but consider various Riemannian metrics on it, which are parametrized 
by values of the radius $R$.
 
Recall from Lecture II-1 that the Hilbert space 
of this theory (in its unique realization)
is of the form $\Cal H_b=\oplus_{k\in \Bbb Z}
(F_b\o F_b^*)_k$, where $F_b$ is the bosonic Fock space. The operator algebra 
$A_b$ of the theory is generated by the 
derivatives of $\phi$ (but not $\phi$ itself), 
and $:e^{in\phi}:$ (for brevity in the future we will drop 
the colons). Operator product expansion is given by 
formula (3.9) of Lecture 3 in the fall, where $D(x-y)=-R^2\ln|x-y|$.
The action of $A_b$ in $\Cal H_b$:
Derivatives of $\phi$ do not change $k$, and 
$:e^{in\phi}:$ maps $(F_b\o F_b^*)_k$ to $(F_b\o F_b^*)_{k+n}$. 
  
Now consider the theory (5.12). The Hilbert space of this theory is
$\Cal H_f=
F_f\o F_f^*$, where $F_f$ is generated from the vacuum by holomorphic 
operators $\psi_+,\bar\psi_+$, and $F_f^*$ is generated from the vacuum 
by $\psi_-,\bar\psi_-$. The operator algebra $A_f$ is generated 
by $\psi_+,\bar\psi_+, \psi_-,\bar\psi_-$, 
with the standard OPE of the free theory. 
  
Consider more closely the operator $:e^{in\phi}:$ 
for $n\in \Z$. As we know
(Lecture 3 in the fall, Lecture II-1), 
this operator has holomorphic dimension
$n^2R^2/4$ and antiholomorphic dimension $n^2R^2/4$
(the total of $n^2R^2/2$, as we saw in Lecture 3 in the fall). 

Classically, the operator $e^{in\phi}$ 
locally factors as a product of 
a holomorphic one and an antiholomorphic one: 
$e^{in\phi}=e^{in\phi_+}e^{in\phi_-}$, where $\phi=\phi_++\phi_-$, and
$\d_-\phi_+=\d_+\phi_-=0$. (Of course, here $\phi_+,\phi_-$ are defined only 
up to adding a constant). There is no analogs of $e^{in\phi_\pm}$ in our 
operator algebra $A_b$. However, imagine for a second that the
operators $e^{in\phi_\pm}$ make sense. Then we will find 
from the OPE for $e^{i\phi}$ 
that $e^{i\phi_+}$ has holomorphic dimension $R^2/4$ and antiholomorphic 
dimension $0$, and 
$$
\gather
\<e^{i\phi_+(x_1)}...e^{i\phi_+(x_n)}
e^{-i\phi_+(y_1)}...e^{-i\phi_+(y_n)}\>=\\
\frac{\prod_{1\le i<j\le n}
(x_i-x_j)^{R^2/2}\prod_{1\le i<j\le n}
(y_i-y_j)^{R^2/2}}{\prod_{1\le i,j\le n} (x_i-y_j)^{R^2/2}}.\tag 5.12
\endgather
$$
where $x_i,y_j$ are viewed as complex numbers. 
{}From this formula it is clear that in order for 
$e^{i\phi_+}$ to make any sense, we need 
$R^2/2\in \Bbb Z$, so that the function on the R.H.S. of (5.12) 
is single-valued. Similarly, in order for $e^{in\phi_+}$ to make sense, 
$n^2R^2/2$ has to be an integer. 

Let us consider the simplest case where $e^{in\phi_+}$ can make sense, 
i.e. $R=\sqrt{2}$. In this case, we have
$$
\<e^{i\phi_+(x_1)}...e^{i\phi_+(x_n)}
e^{-i\phi_+(y_1)}...e^{-i\phi_+(y_n)}\>=\frac{\prod_{1\le i<j\le n}
(x_i-x_j)\prod_{1\le i<j\le n}
(y_i-y_j)}{\prod_{1\le i,j\le n} (x_i-y_j)}.\tag 5.13
$$
It is clear from (5.13) that 
$e^{in\phi_+}$ behaves like a fermion when $n$ is odd, and like a boson 
when $n$ is even. This makes us hope that it is at $R=\sqrt{2}$ that 
our theory is related to the theory of fermions.  

Let us see why this is indeed the case. 
Let us compute the fermionic correlation function
$\<\psi_+(x_1)...\psi_+(x_n)\bar\psi_+(y_1)...\bar\psi_+(y_n)\>$.
Using Wick's formula, it is easy to find that 
$$
\<\psi_+(x_1)...\psi_+(x_n)\bar\psi_+(y_1)...\bar\psi_+(y_n)\>=
\text{det}(\frac{1}{x_i-y_j}).\tag 5.14
$$
The coincidence of the right hand sides of (5.13),(5.14) is a famous 
combinatorial identity, which follows from comparison of zeros and poles, 
and asymptotics at infinity. 

{\bf Remark.} In fact, the explicit form of (5.13),(5.14) is not relevant to 
the proof of the fact that they are equal. What is relevant is only the 
structure of zeros and poles, and the asymptotics at infinity, 
which in both cases are obvious from the OPE. 

Let $\hat A_b$ be the operator algebra generated 
by $A_b$ and $e^{in\phi_\pm}$. Equalities (5.13),(5.14) show that 
we have  a homomorphism $\xi: A_f\to \hat A_b$ defined by 
$\xi(\psi_\pm)=e^{\pm i\phi_\pm}$, $\xi(\bar\psi_\pm)=e^{\mp i\phi_\pm}$, 
which preserves expectation values. 

Using the OPE, it is easy to find
$$
\xi(:\bar\psi_+\psi_+:)=\lim_{\e\to 0}\xi
(\bar\psi_+(x+\e)\psi_+(x)-\frac{1}{\e})=
\lim_{\e\to 0}(e^{-i\phi_+}(x+\e)e^{i\phi_+}(x)-\frac{1}{\e})=
-i\d_+\phi.\tag 5.15
$$
Likewise, $\xi(:\bar\psi_-\psi_-:)=i\d_-\phi$. 
Also, $\xi(\psi_+\bar\psi_-)=e^{i\phi}$, $\xi(\psi_-\bar\psi_+)=
e^{-i\phi}$. 
In fact, it is not difficult to see 
that this homomorphism is an isomorphism. 

At the level of Hilbert spaces, $\xi$ induces a 
bigraded isomorphism $F_f\to F_b\o l_2(\Z)$
(bidegree=(quantum scaling dimension, charge)), where
the charge of $e^{i\phi_+}$ and the charge of $\psi_+$ 
equal $1$. Writing the corresponding character formula, we obtain
$$
\frac{\sum_{n\in\Z}q^{n^2/2}z^n}{\prod_{n\ge 1}(1-q^n)}=
\prod_{n\ge 1}(1+q^{n-1/2}z)(1+q^{n-1/2}z^{-1}),\tag 5.16
$$
which is the famous Jacobi triple product identity. 

The correpondence $\xi$ is called the Bose-Fermi correspondence.

{\bf Remark 1.} If the radius of the circle is not $\sqrt{2}$, but $1$, then 
the operator $e^{i\phi_+}$ is meaningless, as its 2-point function 
would be $(x-y)^{-1/2}$, which is not single-valued. However, the operator
$e^{2i\phi}$ is defined, and its 2-point function is $\frac{1}{(x-y)^2}$. 
This indicates that $e^{2i\phi}$ behaves like a current of some symmetry. 
And indeed, it turns out that the corresponding model is equivalent to 
the $\widehat{SU(2)}$-WZW model with Kac-Moody central charge 1, 
so it has an $SU(2)$-symmetry. In fact, the Fourier components 
of the operators 
$e^{\pm 2i\phi_+}$ and $\d_+\phi$ generate the left-moving $\widehat{SU(2)}$, 
and the  Fourier components 
of the operators 
$e^{\pm 2i\phi_-}$ and $\d_-\phi$ generate the right-moving $\widehat{SU(2)}$.
  
{\bf Remark 2.}
In fact, the Bose-Fermi correspondence is true not only locally (at the level 
of operators), but also globally (at the level of path integral). Namely, 
for any Riemann surface $\Sigma$ one has the identity of partition functions
$$
\int D\phi\ e^{-\frac{1}{8\pi}\int |d\phi|^2}=
\sum_{\e}
\int D\psi_+D\bar\psi_+D\psi_-D\bar\psi_- e^{-\Cal L_f(\psi)},\tag 5.17
$$
where $\Cal L_f$ is the Lagrangian given by (5.10), and $\e$ runs over spin 
structures on $\Sigma$ (the same spin structure is taken for $\psi_+$ and 
$\psi_-$). There is a similar identity for correlation 
functions of operators, if the correspondence between operators is made
as explained above.   

{\bf 5.6. Bose-Fermi correspondence for nonlinear theories.}

We have established a correspondence between two free theories -- the 
theory of a boson and the theory of fermions. A remarkable fact is
that this correspondence generalizes to the case when the Lagrangian 
of one or both of the theories is not free. Consider examples of such 
situations.  

1. Recall that under our correspondence $\Cal L_f\to \Cal L_b(\sqrt{2})$, 
where $\Cal L_b(R)$ is given by (5.11). Since
$\xi(\bar\psi_\pm\psi_\pm)=\mp i \d_\pm \phi$, we 
find
$$
\Cal L_f+\frac{1}{2\pi}\int d^2x(g\bar\psi_+\psi_+\bar\psi_-\psi_-)\to 
\Cal L_b(\sqrt{2}(1+ g)^{-1/2}).\tag 5.18
$$ 
This shows that, to our surprise, the theory with the Lagrangian
$\Cal L_f+\frac{1}{2\pi}\int d^2x(g\bar\psi_+\psi_+\bar\psi_-\psi_-)$ is free. 
In particular, its $\beta$-function is zero. This is obvious when 
the theory is described in Bose variables, but not obvious 
in Fermi variables.  

2. On the other hand, consider the theory of free massive 
fermions, with the Lagrangian 
$\Cal L_f+\frac{1}{2\pi}\int d^2x(m\bar\psi_-\psi_++\bar m\bar\psi_+\psi_-)$. 
Using the fact that $\xi(\bar\psi_\pm\psi_\mp)=-e^{\mp i\phi}$, we
get that under $\xi$, this free Lagrangian goes to
$$
\Cal L_b(\sqrt{2})-\frac{1}{2\pi}
\int d^2x(me^{i\phi}+\bar me^{-i\phi}).\tag 5.19
$$   
So we get another surprising fact that the nonlinear theory defined by 
(5.19) is in fact free. 

{\bf Remark.} It may appear that the second term in (5.19) has a wrong 
scaling dimension. This is not the case, because the 
operator $e^{\pm i\phi}$ has anomalous dimension 1. More precisely, 
(5.19) does not fix a theory but fixes a family of theories 
depending on a scale $\mu$ of momenta, which is introduced when the operators
$e^{i\phi}$ are renormalized. This scale enters in front of the 
corresponding term in the Lagrangian and cancels the discrepancy in 
dimensions. 

Using the symmetry $\phi\to \phi+\theta$
 we can reduce (5.19) to the case of real $m$. In this case,
(5.19) looks like
$$
\Cal L_b(\sqrt{2})-\frac{m}{\pi}\int d^2x \text{cos}\phi.\tag 5.20
$$   
Thus,  
the classical equation of motion is $\Delta\phi=-4 m \text{sin} \phi$. 
This equation is called the sine-Gordon equation, and it 
is a well-known completely integrable soliton equation. 

Now consider The Lagrangian
$$
\Cal L_b(R)-\frac{m}{\pi}\int d^2x \text{cos}\phi.\tag 5.21
$$
This Lagrangian is proportional to (5.20), 
with $m'=mR^2/2$, so classically the two Lagrangians 
are equivalent. However, quantum mechanically, this is not the case, 
as the scale of Lagrangian is now relevant. In fact, the theory now 
essentially depends on $R$. If $R=\sqrt{2}$, the theory is free, 
but for a general $R$ it is not. The map $\xi$ shows that for a general 
$R$ the theory is equivalent to the fermionic theory with the Lagrangian
$$
\Cal L_f+\int d^2x(m\bar\psi_-\psi_++m\bar\psi_+\psi_-+ 
g\bar\psi_+\psi_+\bar\psi_-\psi_-),\tag 5.22
$$
where $R=\sqrt{2}(1+g)^{-1/2}$. 

As we mentioned, 
the theory described by the Lagrangian (5.21)
is not free for $R\ne \sqrt{2}$. However, it is solvable, in the sense
that its S-matrix can be computed explicitly. This computation and the 
result are similar to the computation of Lecture II-3, for the sigma-model 
into the sphere. Solvability for this theory for large $R$ 
(i.e. in the classical limit) is related to the complete 
integrability of the sin-gordon equation at the classical level. 

Now we want to apply the Bose-Fermi correspondence to gauge theory. 
Consider the Lagrangian $\Cal L_{\theta,m}$ given by (5.6).
Define
$$
\Cal L^f_{\theta,m}=\Cal L_{\theta,m}+\frac{1}{2\pi}\int d^2x g\bar\psi_+ 
\psi_+\bar\psi_-\psi_-.\tag 5.23 
$$
Let us rewrite it in Bose variables. Then we will get 
the Lagrangian 
$$
\gather
\Cal L^b_{\theta,m}=\\
\int d^2x\left(
\frac{1+ g}{8\pi}|d\phi|^2-\frac{m}{2\pi}e^{i\phi}-\frac{\bar m}{2\pi}
 e^{-i\phi}+
A_+(\frac{i\d_-\phi}{2\pi})+A_-(\frac{-i\d_+\phi}{2\pi})
+\frac{|*F|^2}{4e^2}-\frac{i\theta}{2\pi} F
\right).\tag 5.24\endgather
$$

For simplicity we assume that $\phi$ is a homotopically trivial map. 
(It is easy to generalize everything to the homotopically nontrivial case). 
Then the integral $\int (A_-\d_+\phi-A_+\d_-\phi)$ can be taken by parts, 
and it equals $\int \phi(\d_-A_+-\d_+A_-)$, where $\phi$ is now understood 
as a lifting of the original $\phi$ to a map $\Sigma\to \R$. 
The expression $\d_-A_+-\d_+A_-$ equals to the curvature $F$, so 
(5.24) is simplified:
$$
\gather
\Cal L^b_{\theta,m}=\\
\int d^2x\left(
\frac{1+ g}{8\pi}|d\phi|^2+\frac{m}{2\pi}e^{i\phi}+\frac{\bar m}{2\pi}
 e^{-i\phi}
+\frac{|*F|^2}{4e^2}-\frac{i(\phi+\theta)}{2\pi} F
\right).\tag 5.25\endgather
$$

{}From this equation it is clear that when $m=0$, the theory is 
free, and there is no essential 
$\theta$-dependence. This is the first thing we 
promised to show. Now, for $m\ne 0$, by changing $\phi$ to $\phi+\theta$
we find that theta can be absorbed in $m$.  

Let us now see what happens for $m\ne 0$. It follows from 
Lecture II-4 that 
for an external field $\phi(x)$  
$$
\int DA\ e^{\frac{i}{2\pi}\int \phi F_A-\frac{1}{4e^2}\int |*F_A|^2}=
Ce^{-\frac{e^2}{2\pi}\int d^2x\min_n(n-\frac{\phi(x)}{2\pi})^2},\tag 5.26 
$$

Therefore, the effective Lagrangian for $\phi$ for the Lagrangian (5.25) 
is
$$
\gather
\Cal L_{eff}^{\theta_*,m_*}=\\
\int d^2x\left(
\frac{1+ g}{8\pi}|d\phi|^2-\frac{m_*}{\pi}\text{cos}(\phi-\theta_*)+
\frac{e^2}{2\pi}\min_n(n-\frac{\phi}{2\pi})^2
\right).\tag 5.27\endgather
$$

{}From this formula, it is seen, that the theory has a mass gap for 
small $m_*$: the potential 
$$
U(\phi)=-\frac{m_*}{\pi}\text{cos}(\phi-\theta_*)
+\frac{e^2}{2\pi}\min_n(n-\frac{\phi}{2\pi})^2 \tag 5.28
$$
has a unique global minimum (modulo $2\pi$) with positive second derivative. 
This is the case for all $m_*$ if $\theta\ne \pi$. However, if $\theta=\pi$ 
and $m_*$ grows from $0$ to $\infty$, the global minimum at 
$\phi=0$ keeps flattening, and at some point splits in two symmetric 
minima, when the second derivative becomes zero. This should indicate that 
starting at some finite value of $e/m_*$ there should be symmetry breaking.
Of course, this proves nothing, because (5.28) is not the quantum
effective potential 
for our system (it is the potential only classically). However, we hope 
that the true effective potential behaves similarly, and the picture is 
qualitatively the same. 

For more on this model, see S.Coleman's article in Annals of Physics, 
vol. 101, p. 239-267 (1976). 

\end


Let $J_0$ be the operator $\bar\psi_-\psi_-+\bar\psi_+\psi_+$ 
with valuus in 1-forms on the surface. This operator is called
``the charge density'' operator, and it is the current for the 
diagonal $U(1)$ symmetry acting on fermions. This current is conserved, 
and moreover by Gauss's law it is proportional to the derivative of the 
field strength $*F$:
$$
\frac{1}{e^2}d*F=J_0.\tag  5.8
$$
(One can easily see that Gauss' law survives quantum mechanically, 
under suitable renormalization of the operators). 

Integrating this differential equation, we find
$$
\frac{1}{e^2}(*F(\infty,t)-*F(-\infty,t))=
\int_{\infty}^{\infty}J_0(x,t)dx\tag 5.9
$$
(here, as you might guess, we are working in Minkowski signature). 
Denote the right hand side of (5.9) by $Q$. This is the operator of total 
charge, and it is time-independent. We have commutation relations
$[Q,\psi_\pm]=\psi_\pm,[Q,\bar\psi_\pm]=-\psi_\pm$, so 
$\psi_\pm$ have charge 1, and $\bar\psi_\pm$ have charge $-1$. 

Now we can see confinement: for $\theta\ne \pm \pi$, 
any state has total charge zero. 
This follows from the fact that even classically, total energy 
of any nonzero-charge configuration minus total energy 
of the vacuum is infinite. 

