%% This is a plain TeX file
%%
\magnification=1200
\hsize=6.5 true in
\vsize=8.7 true in
\input epsf.tex

\input amssym.def
\input amssym.tex

\font\dotless=cmr10 %for the roman i or j to be
                    %used with accents on top.
                    %(\dotless\char'020=i)
                    %(\dotless\char'021=j)
\font\itdotless=cmti10
\def\itumi{{\"{\itdotless\char'020}}}
\def\itumj{{\"{\itdotless\char'021}}}
\def\umi{{\"{\dotless\char'020}}}
\def\umj{{\"{\dotless\char'021}}}
\font\smaller=cmr5
\font\boldtitlefont=cmb10 scaled\magstep2
\font\smallboldtitle=cmb10 scaled \magstep1
\font\ninerm=cmr9
\font\dun=cmdunh10 scaled\magstep1

\footline={\hfil {\tenrm V.\folio}\hfil}

\def\eps{{\varepsilon}}
\def\Eps{{\epsilon}}
\def\kap{{\kappa}}
\def\lam{{\lambda}}
\def\Lam{{\Lambda}}
\def\mynabla{{\nabla\!}}

\def\undertext#1{$\underline{\vphantom{y}\hbox{#1}}$}
\def\nspace{\lineskip=1pt\baselineskip=12pt%
     \lineskiplimit=0pt}
\def\dspace{\lineskip=2pt\baselineskip=18pt%
     \lineskiplimit=0pt}

\def\half{{\textstyle 1\over\textstyle 2}}
\def\w{{\mathchoice{\,{\scriptstyle\wedge}\,}
  {{\scriptstyle\wedge}}
  {{\scriptscriptstyle\wedge}}{{\scriptscriptstyle\wedge}}}}
\def\Le{{\mathchoice{\,{\scriptstyle\le}\,}
{\,{\scriptstyle\le}\,}
{\,{\scriptscriptstyle\le}\,}{\,{\scriptscriptstyle\le}\,}}}
\def\Ge{{\mathchoice{\,{\scriptstyle\ge}\,}
{\,{\scriptstyle\ge}\,}
{\,{\scriptscriptstyle\ge}\,}{\,{\scriptscriptstyle\ge}\,}}}
\def\plus{{\hbox{$\scriptscriptstyle +$}}}
\def\xdot{\dot{x}}
\def\Condition#1{\item{#1}}
\def\Firstcondition#1{\hangindent\parindent{#1}\enspace
     \ignorespaces}
\def\Proclaim#1{\medbreak
  \medskip\noindent{\bf#1\enspace}\it\ignorespaces}
  %the way to use this is:
  %"\Proclaim{Theorem 1.1.}" for instance.
\def\finishproclaim{\par\rm
     \ifdim\lastskip<\medskipamount\removelastskip
     \penalty55\medskip\fi}
\def\Item#1{\par\smallskip\hang\indent%
  \llap{\hbox to\parindent {#1\hfill\enspace}}\ignorespaces}

\def\im{{\rm Im}}  
\def\Pic{{\rm Pic}} \def\Sp{{\rm Sp}}
\def\Diff{{\rm Diff}}  
\def\Map{{\rm Map}}  
\def\spurious{{\rm spurious}}
\def\Tr{{\rm Tr\,}}
\def\SL{{\rm SL}} \def\prim{{\rm primitive}}
\def\Det{{\rm Det}} 
\def\Weil-Peterson{{\rm Weil-Peterson}}

\def\fbar{\bar{f}}  \def\mubar{\bar{\mu}}
\def\barh{\bar{h}}  \def\gammabar{\bar{\gamma}}
\def\kbar{\bar{k}}  \def\lambar{\bar{\lambda}}
\def\mbar{\bar{m}}  \def\phibar{\bar{\phi}}
\def\wbar{\bar{w}}  \def\etabar{\bar{\eta}}
\def\vbar{\bar{v}}  \def\partialbar{\bar{\partial}}
\def\xbar{\bar{x}}  \def\cbar{\bar{c}}
\def\zbar{\bar{z}}  \def\bbar{\bar{b}}
\def\Abar{\bar{A}}
\def\Kbar{\bar{K}}
\def\Pbar{\bar{P}}
\def\Sbar{\bar{S}}
\def\Tbar{\bar{T}}

\def\scrFbc{{\scrF^{(bc)}}}
\def\scrFbcbar{{\scrF^{(\bbar\cbar)}}}

\def\ghat{\hat{g}}
\def\muhat{\hat{\mu}}

\def\htil{\tilde{h}}
\def\xtil{\tilde{x}}
\def\Ctil{\widetilde{C}}
\def\Dtil{\widetilde{D}}
\def\Ltil{\tilde{L}}
\def\Ntil{\widetilde{N}}
\def\Ttil{\widetilde{T}}
\def\scrFtil{\widetilde{\scrF}}
\def\epstil{\tilde{\eps}}
\def\psitil{\tilde{\psi}}

\def\dbR{{\Bbb R}}
\def\dbZ{{\Bbb Z}}

%These two files (in this order!!) are necessary
%in order to use AMS Fonts 2.0 with Plain TeX

\input amssym.def
\input amssym.tex

%Capital roman double letters(Blackboard bold)
\def\db#1{{\fam\msbfam\relax#1}}

\def\dbA{{\db A}} \def\dbB{{\db B}}
\def\dbC{{\db C}} \def\dbD{{\db D}}
\def\dbE{{\db E}} \def\dbF{{\db F}}
\def\dbG{{\db G}} \def\dbH{{\db H}}
\def\dbI{{\db I}} \def\dbJ{{\db J}}
\def\dbK{{\db K}} \def\dbL{{\db L}}
\def\dbM{{\db M}} \def\dbN{{\db N}}
\def\dbO{{\db O}} \def\dbP{{\db P}}
\def\dbQ{{\db Q}} \def\dbR{{\db R}}
\def\dbS{{\db S}} \def\dbT{{\db T}}
\def\dbU{{\db U}} \def\dbV{{\db V}}
\def\dbW{{\db W}} \def\dbX{{\db X}}
\def\dbY{{\db Y}} \def\dbZ{{\db Z}}

\font\teneusm=eusm10  \font\seveneusm=eusm7 
\font\fiveeusm=eusm5 
\newfam\eusmfam 
\textfont\eusmfam=\teneusm 
\scriptfont\eusmfam=\seveneusm 
\scriptscriptfont\eusmfam=\fiveeufm 
\def\scr#1{{\fam\eusmfam\relax#1}}


%Upper-case Script Letters:

\def\scrA{{\scr A}}   \def\scrB{{\scr B}}
\def\scrC{{\scr C}}   \def\scrD{{\scr D}}
\def\scrE{{\scr E}}   \def\scrF{{\scr F}}
\def\scrG{{\scr G}}   \def\scrH{{\scr H}}
\def\scrI{{\scr I}}   \def\scrJ{{\scr J}}
\def\scrK{{\scr K}}   \def\scrL{{\scr L}}
\def\scrM{{\scr M}}   \def\scrN{{\scr N}}
\def\scrO{{\scr O}}   \def\scrP{{\scr P}}
\def\scrQ{{\scr Q}}   \def\scrR{{\scr R}}
\def\scrS{{\scr S}}   \def\scrT{{\scr T}}
\def\scrU{{\scr U}}   \def\scrV{{\scr V}}
\def\scrW{{\scr W}}   \def\scrX{{\scr X}}
\def\scrY{{\scr Y}}   \def\scrZ{{\scr Z}}

\def\gr#1{{\fam\eufmfam\relax#1}}

%Euler Fraktur letters (German)
\def\grA{{\gr A}}	\def\gra{{\gr a}}
\def\grB{{\gr B}}	\def\grb{{\gr b}}
\def\grC{{\gr C}}	\def\grc{{\gr c}}
\def\grD{{\gr D}}	\def\grd{{\gr d}}
\def\grE{{\gr E}}	\def\gre{{\gr e}}
\def\grF{{\gr F}}	\def\grf{{\gr f}}
\def\grG{{\gr G}}	\def\grg{{\gr g}}
\def\grH{{\gr H}}	\def\grh{{\gr h}}
\def\grI{{\gr I}}	\def\gri{{\gr i}}
\def\grJ{{\gr J}}	\def\grj{{\gr j}}
\def\grK{{\gr K}}	\def\grk{{\gr k}}
\def\grL{{\gr L}}	\def\grl{{\gr l}}
\def\grM{{\gr M}}	\def\grm{{\gr m}}
\def\grN{{\gr N}}	\def\grn{{\gr n}}
\def\grO{{\gr O}}	\def\gro{{\gr o}}
\def\grP{{\gr P}}	\def\grp{{\gr p}}
\def\grQ{{\gr Q}}	\def\grq{{\gr q}}
\def\grR{{\gr R}}	\def\grr{{\gr r}}
\def\grS{{\gr S}}	\def\grs{{\gr s}}
\def\grT{{\gr T}}	\def\grt{{\gr t}}
\def\grU{{\gr U}}	\def\gru{{\gr u}}
\def\grV{{\gr V}}	\def\grv{{\gr v}}
\def\grW{{\gr W}}	\def\grw{{\gr w}}
\def\grX{{\gr X}}	\def\grx{{\gr x}}
\def\grY{{\gr Y}}	\def\gry{{\gr y}}
\def\grZ{{\gr Z}}	\def\grz{{\gr z}}

%\overfullrule=0pt

\parindent=18pt
\line{\dun --- DRAFT ---\hfill{\rm IASSNS-HEP-97/72}}

\bigskip\bigskip
\centerline{\boldtitlefont Lecture 8}
\medskip
\centerline{\smallboldtitle V. Moduli Dependence of
Determinants and Green Functions}

\medskip
\centerline{Eric D'Hoker}

\frenchspacing

\dspace
\bigskip
In \S{III} and \S{IV}, we have reduced the calculation of
string transition amplitudes to integrals of
combinations involving certain determinants and Green
functions over the
finite-dimensional moduli spaces $\scrM_h$.
In this section, we complete the analysis by producing
explicit formulas for these determinants and Green
functions.
We shall begin by collecting results on surfaces with
constant curvature metrics.
Then, we proceed and exploit the complex structure of
$\scrM_h$ to obtain explicit expressions in terms of
Jacobi theta functions.
(Derivations of the results stated in this section will
not be given here; they may be found in E. D'Hoker, D. H.
Phong, Rev. Mod. Phys. Vol. 60 (1988), p. 917--1065, and
references therein.)

\bigskip\noindent
\Item{A)} {\bf Worldsheets with constant curvature metric
($h\Ge 2$)}

Let $\Sigma$ be a compact surface of genus $h\Ge 2$.
By the uniformization theorem, the simply connected cover
of $\Sigma$ is the upper half plane $H$, and
$\Sigma=H/\Gamma$, where $\Gamma=\pi_1(\Sigma)$ is the
fundamental group of $\Sigma$.
For any metric $g$ on $\Sigma$, there is a unique Weyl
transformation $\sigma$ such that $g=e^{2\sigma}\ghat$
with $\ghat$ of constant curvature $R_{\ghat}=-1$.
The metric $\ghat$ lifts to the Poincar\'e metric on $H$
and $\Gamma$ is a discrete subgroup of $\SL(2,\dbR)$
acting on $H$, with compact quotient $\Sigma=H/\Gamma$.
Moduli space may now be identified as follows:
$$
\scrM_h=\left\{\ghat,R_{\ghat}=-1\right\}/\Diff(\Sigma)\,\,.
\eqno{(5.1)}
$$
We now have the following results for determinants and
for the measure on $\scrM_h$ arising in string
amplitudes:
$$
\eqalignno{
{\scriptstyle\bullet} &\qquad 
  \Det\,\Delta_{(n)}^{\plus}=S(n+1)\,;\quad
  \Det'\Delta_{(0)}=S'(1)&{(5.2)}\cr
{\scriptstyle\bullet} &\qquad 
  \prod\limits_{j}dm_jd\mbar_j \vert\det(\mu_j,
  \phi_k)\vert^2\det(\phi_j,\phi_k)^{-1}=d
  (\Weil-Peterson)\,\,.&{(5.3)}\cr}
$$
Here $S(s)$ is the {\it Selberg zeta function}, defined
by
$$
S(s)\equiv \prod\limits_{\gamma\,\, \prim}
\prod\limits_{k=1}^\infty (1-e^{-(s+k)\ell_\gamma})
\eqno{(5.4)}
$$
for $\ell_\gamma$ the lengths of the {\it simple closed
$($i.e. primitive$)$ geodesics} $\gamma$ on $\Sigma=H/\Gamma$.

\bigskip
\Item{B)} {\bf Holomorphicity in Moduli}

The most powerful results on Green functions and
determinants are obtained by exploiting the fact that
$\scrM_h$ is a complex orbifold, and that Green functions
and determinants primarily involve complex analytic
dependence on $\scrM_h$.

To examine this complex dependence, we first work locally
on $\scrM_h$, in a neighborhood of a point
$m^0\in\scrM_h$, represented by a metric $g^0$.
We now parametrize the neighborhood $U_\alpha$ of $m^0$
and thus of $g^0$ by Beltrami differentials
$\mu=\mu_{\zbar}^z$, which provide complex coordinates
on $\scrM_h$, as established in \S{III}.
We have a general metric $g$ in the neighborhood of
$g^0$:
$$
g =2g_{z\zbar}\vert dz+\mu d\zbar\vert^2
\qquad \hbox{where}\qquad
g^0 =2g_{z\zbar}\vert dz\vert^2\,\,.\eqno{(5.6)}
$$
We retain the Weyl factor $g_{z\zbar}$ for now, since we
shall also study the interplay of Weyl dependence and
moduli dependence.
$$
\vbox{\epsfxsize=2.5in\epsfbox{fig1.eps}}
$$

The dependence of the functional determinant combinations
$Z_{(n)}^{\pm}(g)$ on the complex coordinates of $\scrM_h$
is simple, and may be evaluated explicitly.
This result is the content of the so-called {\it
Belavin-Knizhnik theorem}; or holomorphic anomaly
theorem:
$$
\delta_\mu\delta_{\mubar}\ln Z_{(n)}^{\pm}(g)=-
{c_{(n)}^{\pm}\over 12\pi}\int\nolimits_{\Sigma}
d\mu_g\left(\nabla_z\delta\mu
\nabla_{\zbar}\delta\mubar-R_g \delta\mu
\delta\mubar\right)
\eqno{(5.7)}
$$
where the coefficients $c_{(n)}^{\pm}=6n^2\pm 6n+1$ are
the same as those entering the Weyl anomaly.

It follows immediately from the Belavin-Knizhnik
theorem that Weyl
invariant combinations of $Z_{(n)}^{\pm}(g)$'s are also
especially simple in their dependence on moduli.
Indeed, the $\delta_\mu\delta_{\mubar}$ variation of any
Weyl invariant combination vanishes, so that such a
combination is a sum of a holomorphic and an
anti-holomorphic part.
We have, e.g.
$$
Z_{(n)}^{\pm}(g)\,Z_{(0)}(g)^{-c_{(n)}^{\pm}}=
\vert\scrZ_{(n)}^{\pm}(m_j)\vert^2\,\,.
\eqno{(5.8)}
$$
where $\scrZ_{(n)}^{\pm}(m_j)$ is holomorphic inside
$\scrM_h$.
In particular, the combination entering the integrand of
the bosonic string amplitudes is precisely of this form,
for superscript $-$ and $n=-1$:
$$
Z_{(-1)}^-(g)\,Z_{(0)}(g)^{-13}=\vert
\scrZ_{(-1)}^-(m_j)\vert^2\,\,.
\eqno{(5.9)}
$$
We shall make use of this powerful result in string
theory in \S{C}.

The BK Theorem may be understood geometrically in the
language of determinant line bundles over $\scrM_h$,
where the right hand side of BK --- the holomorphic
anomaly --- arises as a non-zero curvature of the
determinant line bundles $\Det(\nabla_{(n)}^z)$ and
$\Det(\nabla_z^{(n)})$ with certain hermitian metrics,
the so-called Quillen metrics.
The vanishing of the BK anomaly 
$\scrZ_{(n)}^{\pm}$ then indicates that the corresponding
line bundles
$$
\Det\nabla_{(n)}^z\otimes(\Det\nabla_{(1)}^z)^{-\otimes
c_{(n)}^-}\,\,,
\eqno{5.10}
$$
are flat.
In string theory, the case is $n=2$, and this case is
related to the flatness of $\scrK\otimes\lam^{-13}$,
where $\scrK$ is the canonical bundle of $\scrM_h$ and
$\lam$ is the Hodge bundle.

We shall not prove the BK Theorem here (it can be proven
using heat-kernel methods, just as in the derivation of
the Weyl anomaly), but only indicate the origin of the
holomorphic anomaly.
The key observation is that the covariant derivatives
$\nabla_{(n)}^z$ depend only on $\delta\mu$, but not
on $\delta\mubar$ to first order in $\delta\mu$ and 
$\delta\mubar$; thus
$\nabla_{(n)}^z$ is holomorphic in $\delta\mu$ to first order.
$$
\cases{
\delta_\mu\nabla_{(n)}^z=\delta\mu\nabla_{(n)}^{\zbar}+
  n(\nabla^{\zbar}\delta\mu) &\cr
\noalign{\medskip}
\delta_{\mubar}\nabla_{(n)}^z=0\,\,. &\cr}
\eqno{(5.11)}
$$
Functional determinants of Laplace operators would then
naively be expected to be products of a holomorphic times
anti-holomorphic factor:
$$
\Det\Delta_{(n)}^{-}\,
{\buildrel ?\over\sim}\,
\Det\nabla_z^{(n-1)}\Det\nabla_{(n)}^z\,\,,
\eqno{(5.12)}
$$
with $\Det\nabla_{(n)}^z$ holomorphic on $\scrM_h$ and
$\Det\nabla_z^{(n-1)}$ anti-holomorphic.
Actually, determinants must be properly defined (e.g.
using $\zeta$-function techniques, as in \S{III}.), in a
fashion that preserves their $\Diff(\Sigma)$ invariance.
When this analysis is carried out, anomalies, such as the
Weyl anomaly, may appear.
Here, the above holomorphic factorization of determinants
is also beset by an anomaly.
In fact, these anomalies are intimately related, as can
be seen from the argument below.
As was explained in \S{III}, the variations of $\mu$,
$\mubar$ and Weyl rescalings are independent.
Hence, using the Weyl anomaly of $Z_{(n)}^{\pm}(g)$:
$$
\eqalign{
\delta_\sigma\delta_\mu\delta_{\mubar}\ln\,
Z_{(n)}^{\pm}(g) &=\delta_\mu\delta_{\mubar}
\delta_\sigma\ln\,Z_{(n)}^{\pm}(g)\cr
&=-{c_{(n)}^{\pm}\over 6\pi}\,\delta_\mu\delta_{\mubar}
\int\nolimits_{\Sigma}d\mu_g\,R_g\,\,.\cr}
\eqno{(5.13)}
$$
But, the variation of the Gaussian curvature is easily
computed
$$
\delta_\mu\delta_{\mubar}(d\mu_g\,R_g)=(\nabla_z
\delta\mu\nabla_{\zbar}\delta\mubar)d\mu_g\not=0\,\,.
\eqno{(5.14)}
$$
Thus, clearly, the holomorphic anomaly
$\delta_\mu\delta_{\mubar}\ln\,Z_{(n)}^{\pm}(g)$ cannot
cancel unless also the Weyl anomaly cancels.

\bigskip
\Item{C)} {\bf The Chiral Splitting Theorem}

In critical string theory, we also have strong results on
the holomorphicity properties of the correlation
functions for vertex operators of physical on-shell
states, given by the Chiral Splitting Theorem.
We begin by assembling the data we need to formulate
this theorem.

Let $V_i=\int\nolimits_{\Sigma}d^2z_i\,W_i(z_i,\zbar_i)$
be vertex operators of physical on-shell states whose
polarization tensors are factorized in left and right
movers.
Let $A_I$, $B_I$, $I=1,\ldots,h$ be a canonical basis
of homology $1$-cycles on $\Sigma$, with
$$
\#(A_I,A_J)=\#(B_I,B_J)=0\qquad\qquad\qquad
\#(A_I,B_J)=\delta_{IJ}\,\,.
\eqno{(5.15)}
$$
We define internal loop momenta $p_I\in\dbR^{26}$
associated with each $A_I$ cycle.
Because the $A_I$ cycles have trivial interaction
numbers, these internal loop momenta are independent.
$$
\vbox{\epsfxsize=3.5in\epsfbox{fig2.eps}}
$$
The correlation function of unintegrated vertex operators
$W_i(z_i,\zbar_i)$ at fixed internal loop momenta $p_I$
are defined by
$$
\eqalign{
\left<W_1\ldots W_N\right>_g(p_I) \equiv
\int\nolimits_{\Map(\Sigma,M)}
&Dx\,W_1(z_1,\zbar_1)\ldots W_N(z_N,\zbar_N)\cr
&\prod\limits_{I=1}^h \delta\left(p_I^\mu-
\int\nolimits_{A_I}{dz\over 2\pi i}\,\partial_z
x^\mu\right)e^{-S[x,g]}\cr}
\eqno{(5.16)}
$$
The {\it Chiral Splitting Theorem} states that
$$
\left<W_1\ldots W_N\right>_g(p_I)=
\delta(k)Z_0(g)^{-13}\scrF\scrFtil
\eqno{(5.17)}
$$
where $\scrF(z_i,m_j;\,p_I,k_i)$ is a complex analytic
function of $z_i$ and $m_j$, and $\scrFtil$ is its
complex conjugate
$$
\scrFtil(\zbar_i,\mbar_j;\,p_I,k_i)=
\scrF(z_i,m_j;\,p_I,k_i)^*\,\,.
\eqno{(5.18)}
$$
Here, we have suppressed the dependence of $W$ and
$\scrF$ on polarization tensors occurring in the
construction of $W$.
Also, the function $\scrF$ depends upon the choice of
curves representing the $A_I$ boundary cycles, and
this dependence has been subsumed in the definition of
the internal momenta $p_I$.

An outline of the proof of this Theorem will be given in
\S{E}.
Let us simply remark here that the proper conformal field
theory interpretation of the integration over loop
momenta is the following.
At fixed values of the internal loop momenta, $p_I$ through
all $A_I$ cycles, only a single conformal family
traverses each $A_I$-cycle.
The contributions from left and right movers are
factorized, so that the entire amplitude --- at fixed
momenta $p_I$ --- is factorized.

Mathematically, we have gained considerably through the
Belavin-Knizhnik and Chiral Splitting Theorems, since we
may now apply techniques of complex function theory,
theta-functions and algebraic geometry to study
amplitudes.

Physically, we have uncovered a method for separating the
contributions of left and right movers in the integrand of
transition amplitudes, via the holomorphic versus
anti-holomorphic dependence on $\scrM_h$ and $\Sigma$.
(This will be crucial in the study of superstrings.)

An immediate result from the combination of the
Belavin-Knizhnik and Chiral Splitting Theorems is that
the full transition amplitude at genus $h$ is an integral
over a product of a holomorphic times anti-holomorphic
function.
$$
A_h=\delta(k)\int\nolimits_{\dbR^{26h}}\prod\limits_{I}
d\,p_I^{26}
\int\nolimits_{\scrM_h}\prod\limits_{j}d\,m_j\mbar_j
\int\nolimits_{\Sigma}d^2z_1\ldots\,\,
\int\nolimits_{\Sigma}d^2z_N\,C\,\Ctil\,\,,
\eqno{(5.19)}
$$
where
$$
\eqalignno{
&C(z_i,m_j;\,p_I,k_i)=\det(\mu_j,\phi_k)\scrZ_{(-1)}^-
  (m_j)\scrF(z_i,m_j;\,p,k) &{(5.20)}\cr
\noalign{\hbox{and}}
&\Ctil(\zbar_i,\mbar_j;\,p_I,k_i)
=C(z_i,m_j;\,p_I,k_i)^*\,\,. &{(5.21)}\cr}
$$
(Of course, for $h\Ge 1$, these expressions are only
formal, since the integration ultimately diverges.
The generalization to the case of the superstring on the
other hand will be well-defined.)

\vfill\eject

\noindent
\Item{D)} {\bf Holomorphic and Meromorphic Differentials}
(a brief review of basics)

We begin by recalling some basic tools of complex
fucntion theory on Riemann surfaces.
The space of holomorphic line bundles of degree $d$ is
the {\it Picard variety}
$$
\Pic_d(\Sigma)=\{\hbox{holomorphic line bundles $L$
on $\Sigma\,,$ $c_1(L)=d$}\}\,\,.
$$
The subspace of line bundles $L$ of degree $0$ is the
{\it Jacobian variety}: $J(\Sigma)=\Pic_0(\Sigma)$, which
may also be viewed as
$$
J(\Sigma)=H^1(\Sigma,\dbR)/H^1(\Sigma,\dbZ)\,\,.
\eqno{(5.22)}
$$
Choosing a reference $L_0\in\Pic_d(\Sigma)$, any other
$L\in\Pic_d(\Sigma)$ may be obtained as a tensor product
$L=L_0\otimes\ell$, $\ell\in J(\Sigma)$.

A completely explicit realization is obtained by making a
choice of homology basis $(A_I,B_I)$, $I=1,\ldots,h$, with
canonial intersection matrix:
$$
\#(A_I,A_J)=\#(B_I,B_J)=0\qquad\qquad
\#(A_I,B_J)=-\#(B_J,A_I)=\delta_{IJ}\,\,.
\eqno{(5.23)}
$$
$$
\vbox{\epsfxsize=4.5in\epsfbox{fig3.eps}}
$$
To this choice corresponds a unique basis of holomorphic
Abelian differentials (holomorphic sections of $K$), 
$(\omega_1,\ldots,\omega_h)=\vec{\omega}$ or denoted
simply $\omega$ by
$$
\oint\nolimits_{A_I}\omega_J=\delta_{IJ}\,,\qquad\qquad\qquad
I,J=1,\ldots,h\,\,.\eqno{(5.24)}
$$
The {\it period matrix} is then defined by
$$
\oint\nolimits_{B_I}\omega_J=\Omega_{IJ}\eqno{(5.25)}
$$
and the Jacobian is a complex torus of dimension $h$:
$$
J(\Sigma)=\dbC^h/(\dbZ^h+\Omega\,\dbZ^h)\,\,.
\eqno{(5.26)}
$$
The intersection from $\#(\,\,,\,\,)$ is left invariant
under the action of the {\it modular group} on the
homology cycles
$$
\pmatrix{B\cr A\cr}_I\longrightarrow
\pmatrix{B'\cr A'\cr}_I=
\underbrace{\pmatrix{a &b\cr c &d\cr}_{IJ}}_{\in\,\Sp(2h;\,\dbZ)}
\quad \pmatrix{B\cr A\cr}_J
\eqno{(5.27)}
$$
under which
$$
\Omega\to\Omega'=(a\Omega+b)(c\Omega+d)^{-1}\,\,.
\eqno{(5.28)}
$$
The Riemann bilinear relations guarantee that $\Omega$
lies in the {\it Siegel upper half space}:
$\Omega\in\scrH_h$, where
$$
\scrH_h=\{h\times h\hbox{ complex }\Omega;\,
\Omega_{IJ}=\Omega_{JI};\,\im\,\Omega>0\}\,\,.
\eqno{(5.29)}
$$
In fact, Torelli's theorem guarantees that to each
Riemann surface in $\scrM_h$ there corresponds a unique
point in the fundamental domain of the modular group
$$
\scrA_h\equiv \scrH_h/\Sp(2h;\,\dbZ)\,\,,
\eqno{(5.30)}
$$
(the converse holds for the torus, but not for $h\Ge 2$.)

$\scrH_h$ carries a $\Sp(2h;\,\dbZ)$-invariant metric
$$
ds^2=\Tr((\im\,\Omega)^{-1}d\Omega(\im\,\Omega)^{-1}
d\Omega^*)\,\,.
\eqno{(5.31)}
$$
Since $\omega_I$ depends holomorphically on $\scrM_h$, so
does $\Omega_{IJ}$; in fact, we have an explicit formula
for the variation of $\Omega_{IJ}$ with holomorphic
Beltrami differentials $\mu_{\zbar}^z$ associated with a
Riemann surface deformation:
$$
\delta\Omega_{IJ}=-i\int\nolimits_{\Sigma}d^2z\,
\mu_{\zbar}^z\,\omega_{Iz}\,\omega_{Jz}
\eqno{(5.32)}
$$
The induced complex structure on $\scrM_h$ is the one
associated with the metric $ds^2$ of (5.31).

The {\it Jacobi $\vartheta$-function}, with
characteristics $\delta=(\delta',\delta'')\in[0,1]^{2h}$,
is defined as a function on $\dbC^h\times\scrH_h$:
$$
\vartheta[\delta](\zeta,\Omega)\equiv\sum\limits_{n\in\dbZ^h
+\delta'}
e^{\pi\,i n^T\Omega n+2\pi i\,n^T(\zeta+\delta'')}\,\,.
\eqno{(5.33)}
$$
It obeys, for any $M,N\in\dbZ^h$:
$$
\vartheta[\delta](\zeta+M+\Omega N,\Omega)=
\vartheta[\delta](\zeta,\Omega)
e^{-i\pi\,N^T\Omega\,N-2\pi i\,N^T(\zeta+\delta'')
+2\pi i\delta' M}\,\,,
\eqno{(5.34)}
$$
and this property may be used to define $\vartheta$ as
the unique (up to constant factor) holomorphic section of
a holomorphic line bundle on $J(\Sigma)$, with the above
transition functions.
One also often uses $\vartheta(\zeta,\Omega)\equiv
\vartheta[0](\zeta,\Omega)$.

Using the Jacobi $\vartheta$-functions, and the Riemann
vanishing theorem (which yields the zeros of $\vartheta$
as a function of $\zeta$, but which we shall not discuss
here), one may uniquely (up to constant factor)
reconstruct a meromorphic function on $\Sigma$ from its
divisor $D=z_1+\cdots+ z_d-w_1-\cdots- w_d$:
$$
f_D(z)=\prod\limits_{i=1}^d \vartheta\left(\zeta+
\int\nolimits_{z_i}^z \omega,\Omega\right)\vartheta
\left(\zeta+\int\nolimits_{w_i}^z
\omega,\Omega\right)^{-1}\,\,.
\eqno{(5.35)}
$$
Here $\zeta$ is such that $\vartheta(\zeta,\Omega)=0$,
but so that the above $\vartheta$ functions do not vanish
identically; $f_D(z)$ is then independent of $\zeta$.

Spin bundles on $\Sigma$ belong to $\Pic_{h-1}(\Sigma)$,
and may be described in terms of their spin structure, of
which there are $2^{2h}$.
A choice of homology basis singles out a particular spin
bundle $S_0$, and all other spin structures may be
labeled by points in the Jacobian, as indicated earlier:
$$
S=S_0\otimes\left[\matrix{\delta'\cr \delta''\cr}\right]
\qquad\qquad\qquad\qquad
\delta''+\Omega\delta'\in J(\Sigma)\,\,,
\eqno{(5.36)}
$$
where $\delta'$ and $\delta''$ are half-points, i.e. 
$\delta'_I,\,\delta''_I\in\left\{0,\,{1\over 2}\right\}$.
Spin structures are partitioned into {\it even} or {\it
odd spin structures}, according to the parity of the
$\vartheta$-function
$$
\vartheta[\delta](-\zeta,\Omega)=(-)^{4\delta'\cdot\delta''}
\vartheta[\delta](\zeta,\Omega)\,\,.
\eqno{(5.37)}
$$
Thus, there are $2^{h-1}(2^h\pm 1)$ even/odd spin
structure with $4\delta'\cdot\delta''$ even/odd, and the
number of holomorphic sections is then even/odd as well,
in fact generally $0/1$.
The generic holomorphic section $h_\delta(z)$
 for an odd spin structure
$\delta$ is obtained from the square root of an Abelian
differential with exactly $h-1$ double zeros, (this
$h_\delta(z)$ is a Dirac zero mode)
$$
h_\delta(z)\equiv\biggl(\sum\limits_{I}
\partial_I\vartheta[\delta](0,\Omega)\quad
\omega_I(z)\biggr)^{1/2}\,\,.
\eqno{(5.38)}
$$

Finally, we define one of the fundamental objects that
enter Green functions and determinants in string theory.
The {\it prime form} is a function on the universal cover
of $\Sigma\times\Sigma$ and $\scrH_h$, given by
$$
E(z,w)\equiv{\vartheta
[\delta]\left(\int_w^z\omega,\Omega\right)
\over h_\delta(z)\,\,h_\delta(w)}
\eqno{(5.39)}
$$
where $\delta$ is any {\it odd} half characteristic.
$E(z,w)$ is a holomorphic $\left(-{1\over 2},0\right)$
differential form in both $z$ and $w$, with a single zero
at $z=w$.
As such, $E(z,w)$ generalizes the function
 $z-w$ on $\dbC$ to higher
genus surfaces.
On the torus, we have the well-known expression
$$
E(z,w)={\vartheta_1(z-w,\Omega)\over
\vartheta'_1(0,\Omega)}\,\,.
\eqno{(5.40)}
$$
The monodromies of $E(z,w)$ around $A_I$ homology
cycles are simple:
$E(z,w)$ picks up a $-$ sign; the monodromy as $z$ moves
around $B_I$ once is:
$$
E(z,w)\to -\exp\left[-i\pi\Omega_{II}-2\pi i\int\nolimits_z^w
\omega_I\right]E(z,w)\,\,.
\eqno{(5.41)}
$$
Under modular transformations (represented by their
action on the homology cycles as in (5.27)), we have
$$
E(z,w)\to \exp\left[\pi i\int\nolimits_z^w
\omega^T(c\Omega+d)^{-1}c\int\nolimits_z^w\omega\right]
E(z,w)\,\,.
\eqno{(5.42)}
$$
Meromorphic Abelian differentials (meromorphic sections
of $K$) are readily obtained
$$
\eqalign{
\omega_{xy}(z) &=\partial_z\ln\,{E(z,x)\over E(z,y)}\,dz\cr
\omega_w(z) &=\partial_z\partial_w\ln\,E(z,w)dz\,\,.\cr}
\eqno{(5.43)}
$$
Meromorphic ${1\over 2}$ differentials (meromorphic
sections of $S_\delta$) are
$$
\eqalign{
\hbox{$\delta$ even} \qquad
S_\delta(z,w) &= {1\over E(z,w)}\quad {\vartheta[\delta]
  \left(\int\nolimits_z^w \omega,\Omega\right)\over
  \vartheta[\delta](0,\Omega)}\cr
\noalign{\bigskip}
\hbox{$\delta$ odd}\qquad
S_\delta(z,w) &= {1\over E(z,w)}\quad {\partial_I
\vartheta[\delta]\left(\int\nolimits_z^w \omega,\Omega\right)
  \omega_I(y)\over \partial_I\vartheta[\delta](0,\Omega)
\omega_I(y)}\cr}
\eqno{(5.44)}
$$
For $\delta$ even, $S_\delta$ is the Szeg\"o kernel,
while for $\delta$ odd, $\S_\delta$ is not single-valued on
$\Sigma$ as defined.  

\bigskip
\Item{E)} {\bf Green Functions, Determinants and Chiral
Splitting}

The Green function of scalar fields on $\Sigma$ is simply
expressed in terms of the prime form and Abelian
differentials
$$
G(z,w)=-\ln\,\vert
E(z,w)\vert^2+2\pi\,\im\int\nolimits_w^z\,\omega_I
(\im\,\Omega)_{IJ}^{-1}\,\im\int\nolimits_w^z \omega_J
\eqno{(5.45)}
$$
The second term on the right hand side is required to
render $G(z,w)$ single-valued on $\Sigma\times\Sigma$, as
well as to render it modular invariant.
This expression is readily seen to generalize the
$1$-loop result (and of course the $0$-loop result as
well.)

We use this result to prove the Chiral Splitting theorem
for the case of vertex operators for tachyon states.
(Higher mass states and their vertex operators may be
obtained by factorization or equivalently by using the
OPE for exponential operators.)
To restrict the loop momenta and bring out an integration
over them, we can insert the identity (as we did in
\S{III} for one loop amplitudes)
$$
1=\prod\limits_{I}
\int\limits_{\dbR^{26}}dp_I\,\delta\left(p_I^\mu
-{1\over 2\pi i}\int\limits_{A_I}dz\,\partial_z
x^\mu\right)
\eqno{(5.46)}
$$
into the correlation function
$\left<W_1(z_1,\zbar_1)\ldots W_N(z_N,\zbar_N)\right>_g$.
In fact, here, we may use a short-cut, yielding the same
result, carried out as follows.
If $W_i(z_i,\zbar_i)=\exp ik_i\cdot x(z_i,\zbar_i)$
then we have
$$
\eqalign{
\Bigl<\prod\limits_{i=1}^N e^{ik_i\cdot x(z_i,\zbar_i)}
\Bigr>_g &=\delta(k)\left({\Det'\Delta_{(0)}\over
\int\nolimits_{\Sigma}d\mu_g}\right)^{-13}
\prod\limits_{i<j}e^{-k_i\cdot k_jG(z_i,z_j)}\cr
&=\delta(k) Z_{(0)}(g)^{-13}\prod\limits_{i<j}
\vert E(z_i,z_j)\vert^{2k_i\cdot k_j}X\cr}
\eqno{(5.47)}
$$
where
$$
X=\det(\omega_I,\omega_J)^{-13}\prod\limits_{i<j}
e^{-2\pi k_i\cdot k_j\,\im\int\nolimits_w^z \omega\,
(\im\,\Omega)^{-1}\,\im\int\nolimits_w^z \omega}\,\,.
\eqno{(5.48)}
$$
Using the Riemann bilinear relations, we have
$$
\det(\omega_I,\omega_J)=\det(2\im\,\Omega_{IJ})
\eqno{(5.49)}
$$
and we may now represent $X$ by a single Gaussian
integral over real parameters $p_I^\mu$, which are just
the internal loop momenta:
$$
X=\prod\limits_{I}\int\limits_{\dbR^{26}}d^{26}p_I
\bigl\vert e^{i\pi p_I^\mu\Omega_{IJ}p_J^\mu+2\pi ip_I^\mu
k_i^\mu\int\nolimits_P^{z_i}\omega_I}\bigr\vert^2\,\,.
\eqno{(5.50)}
$$

It is now straightforward to read off the function
$\scrF$ that enters the Chiral Splitting Theorem:
$$
\scrF(z_i,m_j;\,p_I,k_i)=\prod\limits_{i<j}
E(z_i,z_j)^{k_i\cdot k_j}\cdot\exp
\left[i\pi p_I^\mu\Omega_{IJ}p_j^\mu+
2\pi i p_I^\mu k_i^\mu\int\nolimits_P^{z_i}
\omega_I\right]
\eqno{(5.51)}
$$
The point $P$ is arbitrary in view of the momentum
conservation $\delta(k)$-function factor.
$\scrF$ above is determined only up to phases. 
In particular, $\scrF$ is multiple-valued, with
monodromies that may be easily computed from the
monodromies of $E(z,w)$, as given in (5.41).
$\scrF$ also has branch cuts along the homology
cycles: the effect of letting vertex insertion point
$z_i$ cross a cycle $A_I$ is to add the external
momentum $k_i$ to the loop momentum $p_I$.
Modular transformation properties of $\scrF$ are also
non-trivial, and may be read off from (5.42) and
(5.28).

It remains to express the functional determinant
combinations $\scrZ_{(n)}^- (m_j)$ in terms of
$\vartheta$-functions and differentials.
This may be achieved by a variety of techniques, inducing
bosonization and integration over $\scrM_h$ of the stress
tensor.
We shall not develop these techniques here but limit
ourselves to stating the result for $\scrZ_{(-1)}^-$:
$$
\scrZ_{(-1)}^-(m_j)=
{\vartheta(3\Delta_w\vert\Omega)\over
\det\partial_w^i\phi_j(w)}\left(
{\partial_w^h\vartheta(\Delta_w\vert\Omega)\over
\det\partial_w^I\omega_J(w)}\right)^{-9}\,\,.
$$
Here, $w$ is an arbitrary point on $\Sigma$,
$\partial_w^\alpha$ stands for the derivative of order
$\alpha$ in $w$ and $\Delta_w$ is the Riemann vector
$$
\Delta_w^I=\half-\half\Omega_{II}+\sum\limits_{J\not= I}
\oint\nolimits_{A_J}\omega_J(z)\int\nolimits_w^z\omega_I\,\,.
$$

\bye



