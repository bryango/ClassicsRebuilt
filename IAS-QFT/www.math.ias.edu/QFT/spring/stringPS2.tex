% I am sending you a revised copy of Problem Set 2, where
% I simplified the question in Problem 2 section b), first
% sentence. Can you replace the one on the homepage with
% this version ? Thanks,


%%%%%%%%%%%%%%%%%%%%%%%%%%%%%%%%%%%%%%%%%%%%%%%%%%%%%%%%%%%%%%%%%%%%%%%%%%%
%%%%
%%%%    STRING THEORY : Problem Set 2, January 30, 1997
%%%%
%%%%%%%%%%%%%%%%%%%%%%%%%%%%%%%%%%%%%%%%%%%%%%%%%%%%%%%%%%%%%%%%%%%%%%%%%%%


\magnification=\magstep1
\overfullrule=0pt
\baselineskip=17pt

\def\12{{1 \over 2}}

\centerline{{\bf STRING THEORY}}
\centerline{ Problem Set \# 2}
\centerline{January 30, 1997}

\bigskip
\bigskip

The purpose of this problem set is to establish that the singular
parts of the OPE of two stress tensors 
$$
T(z) T(w) \sim {c/2 \over (z-w)^4} + {2 \over (z-w)^2} T(w)
+ {1 \over z-w} \partial _w T(w).
\eqno (1)
$$
and of a stress tensor and a primary field,
are valid for any worldsheet metric $g$ in 
any complex coordinate system in which $g=2g_{z\bar z} |dz|^2$, 
and are thus covariant under analytic
diffeomorphisms of the complex coordinates. 
(The issue of covariance of the OPE was raised by Pierre Deligne.)

1) Using the fact that $T(z)$ transforms as a projective connection,
establish directly that the OPE in (1) transforms covariantly
under analytic diffeomorphisms.

2) An alternative derivation of the covariance of (1), and in 
addition of the validity of (1) on any surface with any metric, may be 
obtained by retracing the derivation of the OPE using 
Diff($\Sigma$) and Weyl($\Sigma$) Ward
identities. The derivation was carried out in Gawedski's lectures, 
but the actual OPE seems to have been derived only for
locally flat metrics. 
Here, we propose to complete the analysis for general metric. 
For the sake of clarity, I shall go through all the steps
of the derivation.


Let $\phi _i $ be a set of primary fields, with conformal 
weights $(h_i, \bar h_i)$, $i=1, \cdots, n$, and consider their
(unnormalized) correlation functions 
$\langle \phi _1  \cdots \phi _n  \rangle _g$
for worldsheet metric $g$.
We define the (covariant) quantum stress tensor $T_{mn} ^{cov}$ by
$$
\delta _g \langle \phi _1  \cdots \phi _n  \rangle _g
= {1 \over 4 \pi}
\int _{\Sigma} d \mu _g \delta g ^{mn} (w,\bar w) 
\langle T_{mn} ^{cov}
 (w. \bar w) \phi _1  \cdots \phi _n  \rangle _g
$$
and the variation of the correlation function on the r.h.s is given by
$$
\delta _g \langle T_{mn} ^{cov}
 (w, \bar w) \phi _1  \cdots \phi _n  \rangle _g
= {1 \over 4 \pi}
\int _{\Sigma} d \mu _g \delta g ^{pq} (z, \bar z) 
\langle T_{pq} ^{cov} (z, \bar z) T_{mn} ^{cov}
 (w, \bar w) \phi _1  \cdots \phi _n  \rangle _g
$$

We define a conformal field theory on any surface 
$\Sigma$, and for any  metric $g$ by the following requirement 
on its full (covariant) quantum stress tensor $T_{mn} ^{cov}$
$$
\nabla ^m T_{mn} ^{cov} = 0 
\qquad \qquad 
T ^{cov} {} _{m} {}^{m}  = - {c \over 3} R_g
$$
where $\nabla$ is the covariant derivative with respect to the 
affine connection. This definition is identical to the axiomatic
definition used in Gawedski's lectures. 

a) Use the covariance of the correlation function 
$\langle \phi _1  \cdots \phi _n  \rangle _g $ 
under general infinitesimal diffeomorphisms (not just analytic ones)
to derive a first order differential equation relating 
$\langle T_{ww} ^{cov}
 (w) \phi _1  \cdots \phi _n  \rangle _g$
 to 
$\langle \phi _1  \cdots \phi _n  \rangle _g$.
Deduce the OPE of $T_{ww} ^{cov} $ with $\phi _i (z)$ from this equation
by solving for its singular part as $z \to w$.

b) By evaluating the variation of $\langle T_{mn} ^{cov} (w,\bar w)
  \phi _1  \cdots \phi _n  \rangle _g$ under a change $\delta g^{zz}$,
 derive an
 equation for $\langle T_{zz} ^{cov} (z,\bar z) T_{ww} ^{cov}
 (w, \bar w) \phi _1  \cdots \phi _n  \rangle _g $.
 Deduce the OPE $T_{zz} ^{cov}  T_{ww} ^{cov} $ from this
 equation by solving for its singular part as $z \to w$.
 
c) Show that there exists a quantity $Q_{zz}$ such that $T(z)$ defined by
 $$
 T (z) = T_{zz} ^{cov} (z, \bar z) + Q_{zz} (z, \bar z)
 $$
is analytic and obeys the standard OPE in (1).
Determine $Q$, and compute it explicitly when $g _{z \bar z} =1$.

d) Deduce from the results of c) for general metric that $T(z)$
transforms as a projective connection under analytic diffeomorphisms.



\end
--2690_6e99-7f7b_16b0-7049_286--

