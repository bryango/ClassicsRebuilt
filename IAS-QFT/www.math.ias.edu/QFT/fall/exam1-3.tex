\input amstex
\magnification=1200
\documentstyle {amsppt}
\pagewidth{6.5 true in}
\pageheight{8.9 true in}
\nologo
\input pictex

\noindent
{\bf Fall Term Exam, N$^{\text{0}}$. 3}\qquad\qquad\qquad
\qquad\qquad\qquad\qquad\qquad(solution by D. Freed)
\smallskip
\hbox to \hsize{\hrulefill}
\bigskip
\noindent
{\bf Problem:}
\medskip
Determine the asymptotic behavior of perturbation theory in
soft or cutoff scalar field theories, as follows.
(Fermions and gauge fields introduce different behavior,
not to mention other types of theories such as sigma
models.)
\roster
\item"{(a)}"  Consider the integral
$$
Z(g)=\int\nolimits^\infty_\infty
d\phi\,\exp\left(-\frac{1}{2}\,\phi^2-g\phi^4\slash4!\right).
\tag1
$$
\endroster
\noindent
Expand in an asymptotic series in $g$
$$
Z(g)=\sum\limits_ka_kg^k\tag2
$$
and determine the large $k$ behavior of $a_k$.  Hint:
write an integral representation for $a_k$ and find a
method of computing the integral for large $k$, for
instance, Where is the maximum of the integrand?
\roster
\item"{(b)}"  Interpret the answer to (a) as a statement
about the number of graphs of a particular kind, weighted
according to their automorphism groups.
\smallskip
\item"{(c)}"  Now consider the path integral
$$
Z(g)=\int D\phi\,e^{-L}\tag3
$$
in a scalar field theory in $n$ dimensions, with
$$
L=\int
d^nx\,\left(\frac{1}{2}|d\phi|^2+\frac{1}{2}m^2\phi^2+\frac{g}
{4!}\phi^4\right).\tag4
$$
\endroster
\noindent
Write down, again, an integral representation for the
coefficients $a_k$ in the asymptotic expansion of $Z(g)$
and determine the large $k$ behavior, at least at a
physical level of rigor, by finding the maximum of the
integrand.
\medskip
You should get a non-linear differential equation, and you
should be able to prove that it has a spherically
symmetric solution; it is believed that this solution does
determine the large $k$ behavior of the $a_k$ in a cutoff
$\phi^4$ theory or in dimensions below four.  In four
dimensions, the much more complicated renormalization,
which you've presumably ignored in the computation above,
changes things ($L$ must be redefined by incorporation of
counterterms and the expansion in powers of $g$ is not made
as simply as you've presumably done above).
\vfill\eject

\noindent
{\bf Solution:}
\medskip
(a)  We write
$$
Z(g)=\int\limits^{\,\,\,\,\,\,\,\,\,\,\infty}_{-\infty}d\phi
\,\,e^{-\phi^2\slash_2-g\phi^4\slash_{4!}}\sim\sum\limits^
\infty_{k=0}a_kg^k
$$
so that by Taylor's formula
$$
a_k=\frac{(-1)^k}{k!(4!)^k}\quad\int\limits^{\,\,\,\,\,\,\,
\,\,\,\infty}_{-\infty}d\phi\,\,e^{-\phi^2\slash_2}\,\phi^{4k}.
\tag1
$$
We can evaluate the integral in (1) exactly:  set
$$
\align
F(J)&=\int\limits^{\,\,\,\,\,\,\,\,\,\,\infty}_{-\infty}d\phi\,\,e^{-\phi
^2\slash_2+J\phi}\\
&=\int\limits^{\,\,\,\,\,\,\,\,\,\,\infty}_{-\infty}d\phi\,\,e^{-(\phi+J)^
2\slash_2}\,\,e^{J^2\slash_2}\\
&=\sqrt{2\pi}\,\,e^{J^2\slash_2}.
\endalign
$$
The integral in (1) is the derivative
$$
F^{(4k)}(J=0)
=\sqrt{2\pi}\,\,\,\frac{(4k)!}{(2k)!2^{2k}},\tag2
$$
so from (1) we have
$$
a_k=\frac{(-1)^k(4k)!\,\,\sqrt{2\pi}}{k!(24)^k(2k)!\,2^{2k}}.
\tag3
$$
Now we apply Sterling's formula
$$
n!\sim\sqrt{2\pi}\,\,\left(\frac{n}{e}\right)^n\,\sqrt n
$$
to deduce
$$
a_k\sim(-1)^k\,\,\left(\frac{2k}{3e}\right)^k\,\,\sqrt{\frac{2}
{k}}.\tag4
$$
\medskip
Alternatively, write the integral in (1) as
$$
\int\limits^{\,\,\,\,\,\,\,\,\,\,\infty}_{-\infty}d\phi\,\,\phi
^{4k}\,\,e^{-\phi^2\slash_2}=\int\limits^{\,\,\,\,\,\,\,\,\,\,
\infty}_{-\infty}d\phi\,\,e^{-\phi^2\slash_2+2k\,\ell
n\phi^2}\tag5
$$
and find the maximum of the exponent at
$\phi=\pm\sqrt{2k}$. At these points the second diverative of
the exponents is $-2$, so the integral to a Gaussian
approximation at either of these maxima is
$$
\align
\int\limits^{\,\,\,\,\,\,\,\,\,\,\infty}_{-\infty}d\phi\,\,e^{
-\phi^2\slash_2+2k\,\ell
n\phi^2}&\sim(4k)^{2k}\,\,e^{-2k}\,\,\int\limits^{\,\,\,\,\,\,\,
\,\,\,\infty}_{-\infty}d\tilde\phi\,\,e^{-2\tilde\phi^2\slash
_2}\\
&=(4k)^{2k}\,\,e^{-2k}\,\,\sqrt\pi.
\endalign
$$
Since there are two such maxima we find from (1) that
$$
a_k\sim\frac{(-1)^k\,2(4k)^{2k}\,\,e^{-2k}\,\sqrt\pi}{k!(24)^
k}.
$$
Using Sterling's formula we obtain agreement with (4).
\medskip
(b)  According to the Feynman calculus
$|a_k|\slash_{\raise 2pt\hbox{$\scriptstyle {\sqrt{2\pi}}$}}$ 
counts the number of
equivalence classes of graphs with the $k$ 4-valent
vertices, each graph $\Gamma$ counted with a factor of
$^1\slash_{\raise 2pt\hbox{$\scriptstyle {\sharp\text{Aut}
\Gamma}$}}$.  
We can check this directly by
counting.  Start with $k$ 4-valent vertices, with the
vertices numbered $1,2,...,k$ and the edges of each vertex
labeled 1,2,3,4.  Then each edge has a number $ab$
attached, where $a$ is the vertex number and $b$ the label
of the edge at that vertex.  Now there is an obvious
ordering of all edges (11, 12, 13, 14, 21, 22, 23, 24,
etc.).  We must attach the edges.  For the first edge there
are $(4k-1)$ choices, for the next available edge $(4k-3)$
choices, etc.  So the total number of labeled graphs is
$$
1\cdot3\cdot5\cdots\cdots(4k-1)=\frac{(4k)!}{(2k)!\,2^{2k}},
$$
and since there are $k!$ automorphisms of the vertex
numberings and $(4!)^k$ automorphisms of the edge labelings, our
counting is consistent with (1).
\medskip
(c)   Now for the path integral we have, analogously to
(1), 
$$
a_k=\frac{(-1)^k}{k!(4!)^k}\,\,\int
D\phi\,\,e^{-\frac{1}{2}\int d^nx(|d\phi|^2+m^2\phi^2)}\,\,
\left(\int\,d^nx\,\phi^4\right)^k.\tag6
$$
\medskip
As in (5) we rewrite the integrand in (6) as
$$
\exp\biggl\{-\frac{1}{2}\,\int d^nx\,(|d\phi|^2+m^2\phi^2)+
k\,\,\ell n\,\,\int\,d^nx\,\,\phi^4\biggr\}.
$$
To find the maximum of the exponent we differentiate
at
$\phi$ in the direction $\dot\phi$ and set the result to
zero:
$$
0=\int\,d^nx\,(-\Delta\phi-m^2\phi,\,\,\,\dot\phi)+\frac{4k\,
\int d^nx\,\phi^3\dot\phi}{\int d^nx\,\phi^4}
$$
which implies
$$
(\Delta+m^2)\phi=\frac{4k\,\phi^3}{\int d^nx\,\phi^4}.\tag7
$$
(Our Laplacian $\Delta$ is nonnegative.)
\medskip
We search for a spherically symmetric solution
$\phi=\phi(r)$ for $r$ the radial variable, so
$$
\Delta\phi=-\frac{1}{r^{n-1}}\,\,\,\frac{d}{dr}\,\,\,\left(r
^{n-1}\,\,\frac{d\phi}{dr}\right).
$$
Thus (7) becomes
$$
-\frac{d^2\phi}{dr^2}-\frac{n-1}{r}\,\,\,
\frac{d\phi}{dr}+m^2\phi
=\frac{k\phi^3}{\pi^2\,\int\nolimits^\infty_0\phi^4(r)r^3dr}.
$$
By rescaling $\phi$ it suffices to solve the equation
$$
-\frac{d^2\phi}{dr^2}-\frac{n-1}{r}\,\,\,\frac{d\phi}{dr}+m^2
\phi=k\phi^3;\tag8
$$
namely, we replace the solution $\phi$ to (8) by
$$
\bigg[\pi^2\,\int\nolimits^\infty_0\phi^4(r)r^3dr\bigg]
^{-2}\phi.
$$
Since we want $\int\nolimits^\infty_0\phi^4(r)\,r^3dr$ to
be finite, we require
$$
\mathop{\lim}\limits_{r\rightarrow\infty}\phi(r)=0.\tag9
$$
Also, to make the first derivative term meaningful at
$r=0$, we require
$$
\mathop{\lim}\limits_{r\rightarrow0}\phi'(r)=0.\tag10
$$
We now argue that a solution to (8) satisfying (9) and (10)
exists.
\medskip
For $n=1$ equation (8) is Newton's equation
$$
\frac{d^2\phi}{dr^2}=-V'(\phi)\tag11
$$
for the potential
$$
V(r)=\frac{k}{4}\,\phi^4-\frac{m^2}{2}\,\phi^2.
$$
%\input frr.tex
%%% beginning of file frr.tex
%\input pictex 
%\font\thinlinefont=cmr5
%
%\begingroup\makeatletter\ifx\SetFigFont\undefined
% extract first six characters in \fmtname
%\def\x#1#2#3#4#5#6#7\relax{\def\x{#1#2#3#4#5#6}}%
%\expandafter\x\fmtname xxxxxx\relax \def\y{splain}%
%\ifx\x\y   % LaTeX or SliTeX?
%\gdef\SetFigFont#1#2#3{%
%  \ifnum #1<17\tiny\else \ifnum #1<20\small\else
%  \ifnum #1<24\normalsize\else \ifnum #1<29\large\else
%  \ifnum #1<34\Large\else \ifnum #1<41\LARGE\else
%     \huge\fi\fi\fi\fi\fi\fi
%  \csname #3\endcsname}%
%\else
%\gdef\SetFigFont#1#2#3{\begingroup
%  \count@#1\relax \ifnum 25<\count@\count@25\fi
%  \def\x{\endgroup\@setsize\SetFigFont{#2pt}}%
%  \expandafter\x
%    \csname \romannumeral\the\count@ pt\expandafter\endcsname
%    \csname @\romannumeral\the\count@ pt\endcsname
%  \csname #3\endcsname}%
%\fi
%\fi\endgroup
$$
\hbox{\beginpicture
\setcoordinatesystem units < 1.000cm, 1.000cm>
%\unitlength= 1.000cm
\linethickness=1pt
%\setplotsymbol ({\makebox(0,0)[l]{\tencirc\symbol{'160}}})
%\setshadesymbol ({\thinlinefont .})
\setlinear
%
% Fig POLYLINE object
%
\linethickness= 0.500pt
%\setplotsymbol ({\thinlinefont .})
\putrule from  4.413 23.527 to  4.413 19.082
%
% Fig POLYLINE object
%
\linethickness= 0.500pt
%\setplotsymbol ({\thinlinefont .})
\putrule from  1.715 20.987 to  6.953 20.987
%
% Fig TEXT object
%
\put{$V(\phi)$} [lB] at  4.255 23.527
%
% Fig TEXT object
%
\put{$0$} [lB] at  4.255 21.622
%
% Fig TEXT object
%
\put{$\phi$} [lB] at  7.588 20.987
%
% Fig TEXT object
%
\put{$-\phi_0$} [lB] at  2.032 21.304
%
% Fig TEXT object
%
\put{$\phi_0$} [lB] at  6.318 21.304
%
% Fig TEXT object
%
\put{$-\phi_1$} [lB] at  3.302 21.304
%
% Fig TEXT object
%
\put{$\phi_1$} [lB] at  5.207 21.304
%
% Fig TEXT object
%
\put{$1$} [lB] at  2.826 20.828
%
% Fig TEXT object
%
\put{$1$} [lB] at  5.842 20.828
\linethickness=0pt
\putrectangle corners at  1.715 24.035 and  7.588 19.082
\endpicture}
$$
%%% end of file frr.tex

There is a solution $\phi(r)$ to (11) with the desired
boundary conditions where $\phi(0)=\phi_0$ is the positive
zero of $V$.  Picture a ball released in this potential at
$\phi=\phi_0$.  It rolls down and back up to $\phi=0$,
where it is at rest.  For $n>1$, we have
$$
\frac{d^2\phi}{dr^2}=-V'(\phi)-\frac{n-1}{r}\,\,\frac{d\phi}
{dr}.
$$
In terms of our mechanical model, the ball feels an
additional acceleration when moving to the left and an
additional deceleration when moving to the right.  The
effect dies off with time $(r)$ and is proportional to the
speed.  Also, the effect is bounded for small time if we
have zero initial velocity (as prescribed by (9)).  So now
if we release the ball at $\phi=\phi_0$, it overshoots
$\phi=0$.  Of course, if we release the ball at
$\phi=\phi_1$, the minimum of the potential, it stays at
rest.  By continuity there is some value $\bar\phi$ between
$\phi$, and $\phi_0$ where the ball released at
$\phi=\bar\phi$ will roll exactly up to $\phi=0$ as
desired.
\bye



