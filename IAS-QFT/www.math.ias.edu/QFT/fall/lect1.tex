
%% This is an AMS-TeX file.
%% The command to compile it is: amstex <file>
%%
\input amstex
\documentstyle{amsppt}
\loadeusm
\magnification=1200
\pagewidth{6.5 true in}
\pageheight{8.9 true in}

\catcode`\@=11
\def\logo@{}
\catcode`\@=13

\NoRunningHeads

\font\boldtitlefont=cmb10 scaled\magstep1
\font\bigboldtitlefont=cmb10 scaled\magstep2

\def\spinact{s}
\def\spinel{\tau}

\def\dspace{\lineskip=2pt\baselineskip=18pt\lineskiplimit=0pt}
\def\plus{{\sssize +}}  \def\pr{{\text{\rm pr}}}
 \def\const{\text{\rm const}}
\def\oplusop{\operatornamewithlimits{\oplus}\limits}
\def\Piop{\operatornamewithlimits{\Pi}\limits}
\def\w{{\mathchoice{\,{\scriptstyle\wedge}\,}
  {{\scriptstyle\wedge}}
  {{\scriptscriptstyle\wedge}}{{\scriptscriptstyle\wedge}}}}
\def\Le{{\mathchoice{\,{\scriptstyle\le}\,}
  {\,{\scriptstyle\le}\,}
  {\,{\scriptscriptstyle\le}\,}{\,{\scriptscriptstyle\le}\,}}}
\def\Ge{{\mathchoice{\,{\scriptstyle\ge}\,}
  {\,{\scriptstyle\ge}\,}
  {\,{\scriptscriptstyle\ge}\,}{\,{\scriptscriptstyle\ge}\,}}}
\def\vrulesub#1{\hbox{\,\vrule height7pt depth5pt\,}_{#1}}
\def\rightsubsetarrow#1{\subset\kern-6.50pt\lower2.85pt
     \hbox to #1pt{\rightarrowfill}}
\def\mapright#1{\smash{\mathop{\,\longrightarrow\,}\limits^{#1}}}
\def\arrowsim{\smash{\mathop{\to}\limits^{\lower1.5pt
  \hbox{$\scriptstyle\sim$}}}}

%\def\Mbar{\overline{M}}
%\def\Qbar{\overline{Q}}
\def\Ebar{\overline{E}}
\def\fbar{\overline{f}}
\def\gbar{\overline{g}}
\def\Sbar{\overline{S}}
\def\Vbar{\overline{V}}
\def\scrRbar{\overline{\scrR}}
\def\scrSbar{\overline{\scrS}}

\def\Stil{\widetilde{S}}
\def\scrStil{\widetilde{\scrS}}
\def\scrTtil{\widetilde{\scrT}}
\def\rhotil{\tilde{\rho}}

%\def\Stab{\text{\rm Stab}}
\def\Vbarspace{\overline {V}_{space}}
\def\Vspace{ V_{space}}
\def\SO{\text{\rm SO}} 
\def\Spin{\text{\rm Spin}}
\def\Res{\text{\rm Res}}
\def\Aut{\text{\rm Aut}}
\def\Out{\text{\rm Aut}}
\def\End{\text{\rm End}}
\def\supp{\text{\rm supp}}
\def\spin{\text{\rm spin}}
\def\QFT{\text{\rm QFT}}
%\def\SU{\text{\rm SU}} \def\Lie{\text{\rm Lie}}
%\def\SL{\text{\rm SL}} \def\Sym{\text{\rm Sym}}

\def\dbC{{\Bbb C}} 
\def\dbR{{\Bbb R}}
\def\dbZ{{\Bbb Z}} 

\def\gr#1{{\fam\eufmfam\relax#1}}

%Euler Fraktur letters (German)
\def\grA{{\gr A}}	\def\gra{{\gr a}}
\def\grB{{\gr B}}	\def\grb{{\gr b}}
\def\grC{{\gr C}}	\def\grc{{\gr c}}
\def\grD{{\gr D}}	\def\grd{{\gr d}}
\def\grE{{\gr E}}	\def\gre{{\gr e}}
\def\grF{{\gr F}}	\def\grf{{\gr f}}
\def\grG{{\gr G}}	\def\grg{{\gr g}}
\def\grH{{\gr H}}	\def\grh{{\gr h}}
\def\grI{{\gr I}}	\def\gri{{\gr i}}
\def\grJ{{\gr J}}	\def\grj{{\gr j}}
\def\grK{{\gr K}}	\def\grk{{\gr k}}
\def\grL{{\gr L}}	\def\grl{{\gr l}}
\def\grM{{\gr M}}	\def\grm{{\gr m}}
\def\grN{{\gr N}}	\def\grn{{\gr n}}
\def\grO{{\gr O}}	\def\gro{{\gr o}}
\def\grP{{\gr P}}	\def\grp{{\gr p}}
\def\grQ{{\gr Q}}	\def\grq{{\gr q}}
\def\grR{{\gr R}}	\def\grr{{\gr r}}
\def\grS{{\gr S}}	\def\grs{{\gr s}}
\def\grT{{\gr T}}	\def\grt{{\gr t}}
\def\grU{{\gr U}}	\def\gru{{\gr u}}
\def\grV{{\gr V}}	\def\grv{{\gr v}}
\def\grW{{\gr W}}	\def\grw{{\gr w}}
\def\grX{{\gr X}}	\def\grx{{\gr x}}
\def\grY{{\gr Y}}	\def\gry{{\gr y}}
\def\grZ{{\gr Z}}	\def\grz{{\gr z}}


\def\scr#1{{\fam\eusmfam\relax#1}}

\def\scrA{{\scr A}}   \def\scrB{{\scr B}}
\def\scrC{{\scr C}}   \def\scrD{{\scr D}}
\def\scrE{{\scr E}}   \def\scrF{{\scr F}}
\def\scrG{{\scr G}}   \def\scrH{{\scr H}}
\def\scrI{{\scr I}}   \def\scrJ{{\scr J}}
\def\scrK{{\scr K}}   \def\scrL{{\scr L}}
\def\scrM{{\scr M}}   \def\scrN{{\scr N}}
\def\scrO{{\scr O}}   \def\scrP{{\scr P}}
\def\scrQ{{\scr Q}}   \def\scrR{{\scr R}}
\def\scrS{{\scr S}}   \def\scrT{{\scr T}}
\def\scrU{{\scr U}}   \def\scrV{{\scr V}}
\def\scrW{{\scr W}}   \def\scrX{{\scr X}}
\def\scrY{{\scr Y}}   \def\scrZ{{\scr Z}}

\def\gr#1{{\fam\eufmfam\relax#1}}


%Euler Fraktur letters (German)
\def\grg{{\frak g}}
\def\grp{{\frak p}}


%\NoBlackBoxes
\document

\centerline{\bigboldtitlefont Lecture 1.}

\centerline{\boldtitlefont Wightman axioms.}

\bigskip
\centerline{David Kazhdan}

\subhead Introduction\endsubhead
In this series of lectures we will present  Wightman axioms and show how 
this set of axioms can be used to derive deep and unexpected results on  
behavior of  quantum field theories. These axioms constitute a consistent 
mathematical theory which can serve for a mathematician as a gateway to the 
strange world of quantum field theories. In spite of the failure to 
describe the structure of gauge theories    Wightman axioms are very useful 
for developing of an intuition on quantum field theories.  


\subhead{1.0} {Setup and notations}\endsubhead
Let $V$ be a pseudo-Riemannian manifold of signature
$(1,d-1)$.
In these lectures we will only consider the case
$V=\dbR^{1,d-1}$ with constant quadratic form.
We denote by $(\quad,\quad)\colon\,V\times V\to\dbR$ the
corresponding quadratic form and will write $v^2$ instead
of $(v,v)$ for $v\in V$.
We will sometimes use decomposition
$V=\dbR\times\dbR^{d-1}$, where $\dbR$ (time) and $\dbR^{d-1}$ (space) are
 orthogonal subspaces 
equipped with constant positive (respectively negative) definite
metrics.
This decomposition is fixed throughout the lectures.
We denote: $V_{space}$ $=\{v\in V\vert v^2<0\}$;
and $\Vbarspace$ is the closure of $V_{space}$.
The complement $V-\Vbarspace$ consists of $2$ connected components, $V_+$
and $V_-$ (where
$V_{\pm}\subset\dbR^{\pm}\times\dbR^{d-1})$.

For $m\in \dbR$ we put $\scrO_m^\pm=\{v\in\Vbar_\pm\vert
v^2=m^2,\,v\not=0\}$.
Let $dv$ be the standard measure on $V$ and let $\mu_m=\Res 
\vrulesub{\scrO_m^+}\,\frac{dv}{v^2-m^2}$ be the
invariant volume form on $\scrO_m^+$.

Let $G=\Spin(V)$ be the double (universal for $d>3$)
cover of connected component of the group $\SO(1,d-1)$.
The semidirect product $P=G\ltimes V$  is called the Poincare group.
It acts on $V$ by isometries.
We denote by $\spinel\in G$ the nontrivial element of the
kernel of the map $G\to\SO(1,d-1)$.

We will use the quadratic form to identify $V^*$ with $V$,
but we will distinguish between vectors and covectors by
reserving the letters $v$, $w$, etc. for vectors, and $p$,
$q$, etc. for covectors.

Note that the cone $V_+$ is self-dual (i.e. $\{p\in V\vert
(p,v)>0\,\,\forall\,v\in V_+\}=V_+$).
For a function $f$ on $V$ we will denote by $\scrF(f)$ its
Fourier transform; so we have:
$$
\scrF(f)(p)=\frac{1}{(2\pi)^{d/2}}\int\limits_{V}
e^{-i(p,v)}f(v)dv.
$$

Let $U\colon\,V\to\Out(\scrH)$ be a unitary representation
of the vector group $V$ in a Hilbert space $\scrH$.

\definition{Definition}
Representation $U$ is called {\it positive} if support of the
corresponding spectral measure lies in $\Vbar_+$.
\enddefinition

\medskip
If $U\colon\,P\to\Aut(\scrH)$ is a unitary representation,
then we say that $U$ is positive if $U\vrulesub{V}$ is
positive.

For a $G$-module $H$ we will write
$H=H^{even}\oplus H^{odd}$, where $H^{even}$ (respectively
$H^{odd}$) is the space of invariants (respectively antiinvariants)
 of $\spinel$. We will endow $H$ with
the structure of superspace given by this decomposition.

Let $R=R^{even}\oplus R^{odd}$ be a finite dimensional real
representation of $G$.
Since $P$ acts transitively on $V$ and stabilizer of $0$ is
$G$, we can associate to $R$ a $P$-equivariant vector bundle
$\scrR$ on $V$.

Let $\scrS^R=\scrS_{even}^R\oplus\scrS _{odd}^R$ be the space of
Schwartz sections of this bundle.
(Note that our bundle is trivialized equivariantly with respect to
translations,  and
$\scrS^R$ is identified with the space of smooth rapidly
decreasing $R$-valued functions on $V$.)
We will often abbreviate $\scrS=\scrS^R$.

Finally we need some preparations on signs. 

We say that a complex superspace $H = H^{even} \oplus H ^{odd}$
is a superHermitian space if it is endowed with an even $\dbR$-bilinear pairing
$(\quad,\quad):H\times H \to \dbC$ such that 
$(\lambda x,y)=(x, \overline {\lambda}y)=\lambda (x,y)$ and
$(x,y)=(-1)^{|x|\cdot |y|} \overline {(y,x)}$.
 Note that since $(\quad,\quad)$ is even we have
 $(x,y)=0$ for   $x\in H ^{even},\, y\in H^{odd}$. 

$H$ is called superHilbert space if ($H^{even}, (\quad,\quad)$)
and ($H^{odd} , -i(\quad,\quad)$) are Hilbert spaces. 
For $T\in \End(H)$ the (super)adjoint operator $T^*$ is defined by:
$(Tx,y)=(-1)^{|x|\cdot |T|}$ 
$ (x,T^*y)$. An operator is superHermitian
symmetric if $T^*=T$. We will often omit the word super.

Now we are ready to formulate:

\subhead{1.1} {Wightman axioms} \endsubhead

\noindent
{\bf Data:}\enspace

\roster
\runinitem"1)"
A finite dimensional real representation
$\rho\colon\,G\to\Aut\,R$.

\smallskip
\item"2)"
A positive unitary representation $U\colon\,P\to\
\Aut(\scrH)$ in a (super)Hilbert space $\scrH$.

\smallskip
\item"3)"
A dense $P$-invariant subspace $\scrD\subset\scrH$, and a
$P$-invariant vector $\Omega\in \scrD$,
$\Vert\Omega\Vert=1$.

\smallskip
\item"4)"
A linear map $\phi\colon\,\scrS ^R\to\End(\scrD)$, called 
{\it the field map} such that:
\endroster

\bigskip
\roster
\item"$\alpha)$"
$\phi$ is $P$-equivariant where $P$ acts on $\End(\scrD)$ by
conjugation.

\smallskip
\item"$\beta)$"
$\phi(f)$ is superHermitian-symmetric.

\smallskip
\item"$\gamma)$"
For any $\psi_1,\psi_2\in\scrD$ the map
$f\to\left<\psi_1,\phi(f)\psi_2\right>$ from $\scrS$ to $\dbC$
is continuous 
($\scrS=\scrS ^R$ is equipped with the Schwartz topology).

\smallskip
\item"$\delta)$"
The space $\scrD$ is generated by vectors
$\phi(f_1)\phi(f_2),\dotsc,\phi(f_n)\Omega$, for 
$f_1,\dotsc,f_n\in\scrS$.

\smallskip
\item"$\varepsilon)$" {\rm (space-locality)}
If $f_1,f_2\in\scrS$ are such that for any
$v_1\in\supp(f_1)$, $v_2\in\supp(f_2)$ we have
$(v_1-v_2)\in V_{space}$ (such $f_1$, $f_2$ are called {\it
space-like separated}), then $[\phi(f_1),\phi(f_2)]=0$
where $[\quad,\quad]$ stands for supercommutator.
\endroster

\remark{Remark}
It is often assumed that the space of $V$-invariants in
$\scrH$ is $1$-dimensional (uniqueness of vacuum).
\endremark

Such set of data satisfying the above properties 
is called QFT in Wightman axioms (or simply Wightman QFT).


\subhead{1.2} {Wightman's functions} \endsubhead
Let us denote: $V_n=V\times V\times\dots\times V$ ($n$
times);
$\scrR_n=\scrR^{\boxtimes n}$ (this is a bundle on $V_n$).
The space of Schwartz sections of $\scrR_n$ will 
be denoted by $\scrS_n$.

For $n\geq 1$ the Schwartz theorem and the property $(\gamma)$ imply the
existence of unique  tempered $\dbC$-valued
distribution $\scrW_n\in\scrS'_n$
such that
$$
\scrW_n(f_1\times f_2\times\dots\times f_n)=\left<\Omega,
\phi(f_1)\phi(f_2)\dots \phi(f_n)\Omega\right>
$$
for $f_1,\dotsc,f_n\in\scrS$. 

These $\scrW_n$ are called {\it Wightman functions}, or correlation
 functions of the theory. (We will use the
word ``function'' for a section of an appropriate vector
bundle when it causes no confusion). 



\subhead{1.3} {Reconstruction of QFT from Wightman functions} \endsubhead
We are going to show that a QFT is uniquely determined by its
Wightman functions.
So we assume that a representation $\rho$ of $G$ and 
a collection  of distributions $\scrW_n\in\scrS_n$
are  given, and that $\scrW_n$ are Wightman functions of some 
Wightman QFT $(\rho;\,\scrH, \scrD, \Omega ; \, \phi)$.
We want to reconstruct  $\scrH, \scrD, \Omega$ and $\phi$.


Let us put $\scrStil=\oplusop_{n}\scrS_n$, $\scrW=\Piop_{n}
\scrW_n\in\scrStil'$  (where we use the standard notation 
 $L'$ for the topological dual of  a topological vector space $L$).
The space $\scrStil$ carries an algebra structure: $f_1\cdot
f_2=f_1\boxtimes f_2$.
For $f\in\scrS_n$ we define $f^*\in\scrS_n$ by:
$f^*(v_1,\dotsc,v_n)=\pm f(v_n,\dotsc,v_1)$ 
(the sign is determined by the super-rule as usually).
Consider a bilinear form $\{\quad,\quad\}$ on $\scrStil$ given
by:
$$
\{f_1,f_2\}=\scrW(f_1^*\cdot f_2).
$$


\definition{Claim}
$\{f_1,f_2\}=\left<\phi(f_1)\Omega,\phi(f_2)\Omega\right>$.
\enddefinition

\demo{Proof}
Clear.
\enddemo

We  can recover $\scrH$ as the Hilbert space completion of
$(\scrStil\otimes\dbC,\{\quad,\quad\}')$,
where $\{\quad,\quad\}'$ is the unique Hermitian scalar
product on $\scrStil\otimes\dbC$, such that
$\{\quad,\quad\}'\vrulesub{\scrStil}=\{\quad,\quad\}$.
Then we can take $\scrD$ to be the image of the obvious map
$\scrStil\otimes\dbC\to\scrH$ and define $\Omega$ as the
image of $1\in\scrStil$ under this map.
Since $P$ acts on $\scrStil$ preserving $\{\quad,\quad\}$ this
action defines an action of $P$ on $\scrH$.
Finally, we get an algebra map
$\phi\colon\,\scrStil\to\End(\scrD)$.

\medskip

Before we formulate the next Claim let us fix some notations.
Let $\Delta \colon V\to V_n$ be the diagonal embedding,
and $\Vbar_n=V_n/\Delta$ be the quotient space.
Since $\scrR_n$ is $P$-equivariant it defines a bundle
$\scrRbar_n$ on $\Vbar_n$.
Let $\scrSbar_n$ be the space of Schwartz sections of
$\scrRbar_n$.
For $f\in \scrS_n$ let $\fbar\in\scrSbar_n$ be its averaging
over $\Delta(V)$, and let $pr^*$ be the dual map on distributions.

We identify $\Vbar_n\arrowsim V_{n-1}$  by:
$$
(v_1,v_2,\dotsc,v_n)\to (v_2-v_1,\dotsc,v_n-v_{n-1}).
$$

\proclaim{Claim}
$\scrW_n$ satisfy the following properties:

\smallskip
\roster
\item"1)"
 $\scrW_n$ is $P$-invariant for all $n$.

\noindent
In particular, $\scrW_n$ is $\Delta(V)$-invariant;
hence $\scrW_n=pr^*(W_n)$ for unique $W_n\in \scrSbar_n'$.


\smallskip
\item"2)"
$\supp\,\scrF(W_n)\subset(\Vbar_+)^{n-1}$.

\smallskip
\item"3)"
$\scrW_n(f^*)=\overline{\scrW_n(f)}$.

\smallskip
\item"4)" 
$\scrW(v_1,\dotsc,v_i ,v_{i+1},\dotsc,v_n)=
\pm\scrW(v_1,\dotsc,
v_{i+1},v_i,\dotsc,v_n)$ if $v_i-v_{i+1}\!\!\in\! V_{space}$.
(Here the sign comes from the $\dbZ/2$ grading on the space
of test function $\scrS_n=\scrS_n^+\oplus \scrS_n^-$.)

\smallskip
\item"5)"
$\scrW(f^*\cdot f)\Ge 0$.
\endroster
\endproclaim

\demo{Proof} 1) and 3)--4) are clear. Let us prove 2). 


We have to check that $W_n(f)=0$ if for some $i$
$\supp(\scrF(f))$ lies in
the domain $\{(v_1,..,v_n)$ $\in V_n\vert v_i\in V-\Vbar_+\}$.
For such $f$ we can find $g\in\scrS_{n+1}$ with $\gbar=f$
satisfying: $\supp\,\scrF(g)\subset\{(v_1,\dotsc,v_{n+1})
\in V_{n+1}\vert v_1+\cdots+v_{i+1}\in V-\Vbar_+\}$.
So we have to show that for such $g$: $\scrW_n
(g)=0$.
We can assume $g=g_1\otimes g_2$ for $g_1\in \scrS_{i+1},g_2\in
\scrS_{n-i}$, and $\supp\,\scrF(g_1)\subset
\{(v_1,\dotsc,v_{i+1})\in V_{i+1}\vert v_1+\cdots+v_{i+1}
\subset V-\Vbar_+\}$.
We have $\scrW_n(g)=\langle 
\phi(g_1)^*\Omega,\phi(g_2)\Omega \rangle$.
But the spectral support of $\phi(g_1^*)\Omega$ is contained
in the set $\{v_1+\cdots + v_{i+1}\vert (v_1,\dotsc,v_{i+1})\in
\supp\,\scrF(g_1)\}$.
On the other hand, it is contained in $\Vbar_+$ by the
positivity condition.
Hence, it is empty, and $\phi(g_1^*)\Omega=0\Rightarrow
\scrW_n(g)=0$.
\enddemo

\remark{Remark} By a standard  theorem  from analysis property 2) is
equivalent to the following: 

$2')$ $W_n$ is a boundary value of an analytic function on $V_{n-1}+i (V_+)
^{n-1}$.
\endremark

It is not hard to see the following

\proclaim{Theorem}
The above constructions provide a bijection between the
following sets of data:

\smallskip
\roster
\item"{\rm a)}"
$\rho$; $\scrH\supset \scrD\owns \Omega$; $\phi$ satisfying
Wightman's axioms;

\smallskip
\item"{\rm b)}"
representation $\rho$ of $G$ and a collection of
distributions $\{\scrW_n\}$ satisfying properties 1) -- 5)
above.
\endroster
\endproclaim

\remark{Remark}
Sometimes one considers theories satisfying weaker axioms.
Namely, let $\scrS_n^0\subset \scrS_n$ be the subspace of sections
which vanish on diagonals together with all partial
derivatives.
A QFT satisfying weak Wightman's Axioms is (a
representation $\rho$ together with) a collection of
$\scrW_n\in(\scrS_n^0)'$ satisfying 1), $2')$, 3), 4) above.

Note that an essential property 5) can not be formulated
if $\scrW_n$  are defined on $\scrS _n^0$ only  because
$f^*\cdot f$ does not lie in $\scrS ^0$.
\endremark

\subhead{1.4} {Spin-statistics Theorem}\endsubhead
This is a statement that our convention on the parity of the basic
superspace $R$ is the only right one. More precisely, we have:
\proclaim{Theorem} Consider the Wightman axioms with the
sign in 
the axiom of space-locality changed. Any theory satisfying this modified
set of axioms is trivial. (The field map $\phi=0$.) 
\endproclaim

\demo{Proof}
Let us take a 2-dimensional subspace $U\subset V$ of signature (1,1). Let
$\sigma'$ be the element of $\SO (d-1,1)$ which acts by $-1$ on $U$ and by
1 on the orthogonal complement; let $\sigma$ be an element in the
complexification of $\Spin(d-1,1)$ which projects to $\sigma'$.
 Then $\sigma ^2=\spinel$ is the spin element.
Also note that the eigenspaces of $\sigma$ in the space of
$R\otimes \dbC$ are defined over $\dbR$. If $\spinel$ acts trivially in $R$
then this is obvious because $\sigma$ is real;
 otherwise $\spinel$ acts by $-1$ and we use the identity $\bar \sigma=
\spinel \sigma$ to see that the $\pm i$ eigenspace of $\spinel$ is invariant
under the conjugation. 

In the next lecture (\S {2.2}) we will show that $W_n$ extends to an
analytic  $G_\dbC$-equivariant function on a certain open domain 
in the complexification $\Vbar_n\otimes \dbC$. In particular  $W_2$  is
analytic in the domain $\scrTtil := \{v\,|\, v^2\not \in \dbR^{\geq
0}\}$. This 
analytic function is also denoted $W_2$.



Let now $r\in R$ be a $\sigma$ eigenvector; let $v$ be a point in 
$U\otimes \dbC\cap
\scrTtil $. We claim that $\left<
r\otimes r, W_2(v)\right>=0$. Indeed we have  $\left<
r\otimes r, W_2(v)\right>= \left<
\sigma(r)\otimes \sigma(r), W_2(-v)\right>=\spin(R)\cdot  \left<
r\otimes r, W_2(-v)\right>$ where $\spin (R)$ is the $\spinel$ eigenvalue on
$R$; here the first equality follows from $\sigma$-equivariance, and the
second is obvious from $\sigma^2=\spinel$.

 On the other hand  by our assumption $  \left<
r_1\otimes r_2, W_2(v)\right>= -\spin (R)\cdot \left<
r_2\otimes r_1, W_2(-v)\right>$ for $v\in V^{space}$. Since
$V^{space}\subset \scrTtil$, we get
 $\left<
r\otimes r, W_2(v)\right>=0$ for $v\in V^{space}$, and hence for all $v\in
\scrTtil$. 

For any real valued Schwartz function $f$ on $V$ we have $\left<
\phi(f\cdot r)\Omega, 
\phi(f\cdot r)\Omega \right>=\int  \left<f\cdot r\times f\cdot r,
\scrW_2\right> = \int  \left<\scrF(f)\cdot r\times \scrF(f)\cdot r, 
\scrF(\scrW_2)\right>=\int \scrF(f)\cdot\overline{\scrF(f)}
 \left< r \otimes r,\scrF(W_2)\right>$. This shows that the distribution
$\left< r\otimes r, \scrF(W_2)\right>$ is nonnegative. 

Now let us take $v\in  V^+$. Then we get   
$$
0=\left<
r\otimes r, W_2(i v)\right>=\int \exp(-v,p)\left< r\otimes r,
\scrF(W_2)\right>.
$$
 Since $\exp(-v,p)$ is positive this implies that 
$\left< r\otimes r,\scrF(W_2)\right>=0$, hence $(\phi (f\cdot
r)\Omega)^2=0$, so the theory is trivial.
\enddemo

We finish by some elementary   
properties of Wightman functions, and their relation to properties of the
field map.

\subhead{1.5} {Mass spectrum of a theory}\endsubhead
Let us denote
$\Delta_\plus(p,m^2)=\scrF(\delta_{\scrO_m^\plus})(p)$.

\proclaim{Claim}
For any scalar field theory
$\phi\colon\,\scrS(V)\to\End(\scrD)$ there exists a positive
measure $d\rho$ on $[0,\infty)$ such that:
$$
W_2(p)=\int\Delta_\plus (p,m)d\rho(m)\,\,.
$$
\endproclaim

\demo{Proof} We have to show that $\scrF (W_2)=\int d\rho(m) \delta
_{\scrO_m ^+}$ for a positive measure $d\rho$.
 Since $\scrO_m^+$ is an orbit of the group $G$
it is obvious that any $G$-invariant distribution supported in $\sqcup
\scrO_m^+ = \overline {V_+}$ can be written down in such form for some
distribution $\rho$. In order to see that $\rho$ is positive we have to show 
that $\scrF(W_2)$ is positive.

For any $f\in \scrS$ we have $\scrW_2(f\boxtimes f)=\left<\phi(f)\Omega,
\phi(f) \Omega \right> \Ge 0$. Hence for  $g=\scrF(f)$ we have
$ \scrF(\scrW_2 )\left( g\boxtimes g\right)\Ge0$.

 We know that 
$ \scrF(\scrW_2 )$ is concentrated on the anti-diagonal and 
 $$ \scrF(\scrW_2 )(p,-p)=\scrF(W_2)(p)$$ Now note that $f\in \scrS$
iff $g=\scrF(f)\in \scrS^\dbC$ and $g(-p)=\overline {g(p)}$.
Thus for any such $g$ 
$$\scrF(W_2 )(g\cdot \overline g )= \scrF(\scrW_2 )(g\boxtimes g)
\Ge 0$$
which means that $\scrF(W_2) $ is positive.

\enddemo

\definition{Definition}
The {\it mass spectrum} of a $\QFT$ is support of the measure
$d\rho$.
The {\it mass} of the theory is the infinum of its mass spectrum
with 0 removed.
\enddefinition

\subhead{1.6} {Asymptotics of Wightman functions}\endsubhead
The next statement is a baby example of relation between analytical
behavior of Wightman functions and  properties of the field map. 
Estimates of this kind will be used essentially 
in the lecture on scattering theory (see lecture 4, 
\S 4.3, Proposition 1 and Claim 1).


\proclaim{Claim} 

a) Uniqueness of vacuum in a $\QFT$ is equivalent to the
equality:
$$\lim\limits_{\lambda\to\infty}\scrW_n(v_1,\dotsc,v_i,v_{i+1}
+\lambda
a,\dotsc,v_n+\lambda a)=\scrW_i(v_1,\dotsc,v_i)
\scrW_{n-i}(v_{i+1},\dotsc,v_n)$$ being true for all $i$,
$n$ and $a$ such that $a^2<0$.

b) \ Assume that mass of the theory is $m_0>0$.
Then $W_2(v)\sim e^{-m_0\sqrt{-v^2}}$ as $v^2\to-\infty$.
\endproclaim

\demo{Proof}
 a)  For any $f_1,\dotsc, f_n \in \scrS$ we have
$$\scrW_n\left(f_1,\dotsc,f_i,T_{\lambda a} (f_{i+1})
,\dotsc,T_{\lambda a}(f_n)\right) = \langle \phi (f_i)
\dotsc \phi (f_1) \Omega ,
\phi (T_{\lambda a}(f_{i+1}))
% \phi (T_{\lambda a}(f_{i+2})) 
\dotsc  \phi (T_{\lambda a}(f_n))\Omega \rangle$$
$$= \left< \phi (f_i)
\dotsc \phi (f_1) \Omega \, ,\, U(T_{\lambda a}) (\phi (f_{i+1})
 \dotsc  \phi (f_n)\Omega) \right>$$
where $T_{\lambda a}$ stands for translation by $\lambda a$.
From the positivity condition it follows that the weak  limit
 $ \lim\limits_{\lambda\to\infty} U(T_{\lambda a})$ exists as a bilinear
form (i.e. all matrix coefficients converge),
and equals to the projection on $\scrH ^V$.

Hence $$\lim\limits_{\lambda\to\infty} \left< \phi (f_i)
\dotsc \phi (f_1) \Omega , U(T_{\lambda a}) \left( \phi (f_{i+1})
 \dotsc  \phi (f_n)\Omega \right) \right> = \left<   \pr (\phi (f_i)
\dotsc \phi (f_1) \Omega) ,
 \pr (\phi (f_{i+1})
\dotsc \phi (f_ n) \Omega) \right>$$
where $\pr$ is the  projection on $\scrH ^V$. The statement follows.

 b) We have
$$W_2( v)= \int\limits _V \scrF(W_2)(p) e^{i( v,p)} dp =
 \int\limits_{0}^\infty d\rho(m) \int\limits
 _{\scrO _m^+} e^{(i
v,p)} d\mu_m (p)$$ where $\mu_m$ is the invariant measure on
$\scrO _m^+$. For $v\in \dbR^{d-1}-\{0\}$ it can be rewritten as:
$$ \const
+\int\limits_{m_0}^\infty d\rho(m) \int\limits _{\dbR^{d-1}} e^{i( v,q)} dq
/ \sqrt {m^2-q^2}$$
(Remember that $q^2=(q,q)$ is negative for $q\in \dbR^{d-1}-\{0\}$).

The integrand is
 analytic in the region $Re(m_0^2-q^2)>0$; it also 
  decreases as $(\vec{p})^2\to -\infty $ for fixed $p_0$, where  $p=(
 \vec{p},p_0)$ is decomposition into space and time components. Hence 
we can shift the
contour of integration from $\dbR^{d-1}$ to $\Gamma\subset \dbC^{d-1}$ where 
$\Gamma$ is $\dbR^{d-1}+im_0 v/\sqrt{-v^2}$ squeezed slightly towards
$\dbR^{d-1}$ in a small neighborhood
of the singular point $im_0 v/\sqrt{-v^2}$ to avoid the singularity.
Then the estimate becomes obvious. 



\enddemo

We will  usually assume that uniqueness of vacuum holds for the theory
under consideration.  

\enddocument


