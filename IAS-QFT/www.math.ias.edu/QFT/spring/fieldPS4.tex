%Date: Thu, 13 Feb 1997 09:38:58 -0500
%From: Edward Witten <witten@sns.ias.edu>

\input harvmac
\def\bar{\overline}
Term 2, Problem Set Four

(1) In the two-dimensional sigma model with target space a sphere $S^n$,
for large $n$, we introduced a Lagrange multiplier field $\sigma$ in
order to quantize the theory.

(a) By using the equations of motion, identify $\sigma$ as an operator in 
terms of the original variables.

(b) What is the dimension of the $\sigma$ field and therefore what
would you expect to be the large $q$ behavior (with Euclidean $q$)
of $\langle \sigma(q)\sigma(-q)\rangle$.  Compare to the actual behavior.

(2) Consider in two dimensions a $U(1)$ gauge theory with {\it two}
species of fermion, $\psi_i$, $i=1,2$, in the same representation of
$U(1)$, of bare masses $m_i$.

The Lagrangian is thus

\eqn\aa{L_\psi(\psi_1,m_1)+L_\psi(\psi_2,m_2)+
\int\left({F^2\over 4e^2}+i{F\over 2\pi}
\right)}
where $L_\psi$ is the Lagrangian that we considered in class:
\eqn\bb{L_\psi(\psi,m)=\int\left(\bar\psi_+D_-\psi_++\bar\psi_-D_+\psi_-
+m\bar\psi_-\psi_++\bar m \bar\psi_+\psi_-\right)d^2x.}


(a) Show that the $\theta$ dependence disappears if either of the $m_i$
vanishes.

(b) Show that in the limit that $m_1$ becomes very large keeping $m_2$
fixed (or letting $m_2$ go to zero while $m_1$ becomes large) 
the theory reduces to the one that we studied in class, but
with a possibly shifted effective value of $\theta$.  

(c) Describe qualitatively the behavior when both of the $m_i$
are large.

(d) Describe the behavior when both of the $m_i$ are small.

(e) Show that if the $m_i$ are equal, the theory has an $SU(2)$
symmetry.  Now, by considering the case that the $m_i$ are equal and small,
deduce that the two-dimensional sine-Gordon model, that is the model
of a scalar field with potential $V(\phi)=\mu^2\cdot\cos(\beta\phi)$,
has an $SU(2)$ symmetry  for a certain value of $\beta$ which you should
identify.

(f) Taking the limit as $\mu$ goes to zero, where the sine-Gordon theory
mentioned in (e) becomes a massless conformally invariant free field theory,
deduce that the two-dimensional theory of a free massless scalar of the
correct radius (or equivalently
of fixed radius $2\pi$ and the correct coefficient $k$
of the kinetic energy in $L=k\cdot \int d^2x  \,|d\phi|^2$)
has an $SU(2)$ symmetry.  Because of the conformal invariance
this theory actually represents  the affinization of $SU(2)$ (more
exactly, two copies of it, one for left-movers and one for right-movers)
and not only the finite-dimensional group $SU(2)$.  This is called
the vertex construction of the affine Lie group of $SU(2)$.

To obtain  it, we didn't really need to go via the gauge theory.
One could, more generally, simply consider several massless free
fermions and, by applying the bose-fermi correspondence to them,
get the vertex construction for $SO(2n)$ and (with a slight modification)
for $SU(n)$.                                                 

Hint: a reference for most of Problem 2 (except the last paragraph)
is Coleman's ``More On The Massive
Schwinger Model.''

\end


