%% This is a plain TeX file
%%
\magnification=1200
\input epsf

\font\dotless=cmr10 %for the roman i or j to be
                    %used with accents on top.
                    %(\dotless\char'020=i)
                    %(\dotless\char'021=j)
\font\itdotless=cmti10
\def\itumi{{\"{\itdotless\char'020}}}
\def\itumj{{\"{\itdotless\char'021}}}
\def\umi{{\"{\dotless\char'020}}}
\def\umj{{\"{\dotless\char'021}}}
\font\smaller=cmr5
\font\boldtitlefont=cmb10 scaled\magstep2
\font\ninerm=cmr9
\font\dun=cmdunh10 scaled\magstep1

\footline={\hfil {\tenrm I.\folio}\hfil}

\def\eps{{\varepsilon}}
\def\kap{{\kappa}}
\def\lam{{\lambda}}

\def\undertext#1{$\underline{\vphantom{y}\hbox{#1}}$}
\def\nspace{\lineskip=1pt\baselineskip=12pt%
     \lineskiplimit=0pt}
\def\dspace{\lineskip=2pt\baselineskip=18pt%
     \lineskiplimit=0pt}

\def\w{{\mathchoice{\,{\scriptstyle\wedge}\,}
  {{\scriptstyle\wedge}}
  {{\scriptscriptstyle\wedge}}{{\scriptscriptstyle\wedge}}}}
\def\plus{{\scriptscriptstyle +}}
\def\xdot{\dot{x}}

\def\Diff{\rm Diff} \def\Area{\hbox{\rm Area}}
\def\Map{\rm Diff}  \def\Met{\rm Met}
\def\Maps{\rm Maps} \def\Vol{\rm Vol}
\def\SU{\rm SU}

\def\fbar{\bar{f}}
\def\kbar{\bar{k}}
\def\zbar{\bar{z}}
\def\Abar{\bar{A}}
\def\Sbar{\bar{S}}

%These two files (in this order!!) are necessary
%in order to use AMS Fonts 2.0 with Plain TeX

\input amssym.def
\input amssym.tex

%Capital roman double letters(Blackboard bold)
\def\db#1{{\fam\msbfam\relax#1}}

\def\dbA{{\db A}} \def\dbB{{\db B}}
\def\dbC{{\db C}} \def\dbD{{\db D}}
\def\dbE{{\db E}} \def\dbF{{\db F}}
\def\dbG{{\db G}} \def\dbH{{\db H}}
\def\dbI{{\db I}} \def\dbJ{{\db J}}
\def\dbK{{\db K}} \def\dbL{{\db L}}
\def\dbM{{\db M}} \def\dbN{{\db N}}
\def\dbO{{\db O}} \def\dbP{{\db P}}
\def\dbQ{{\db Q}} \def\dbR{{\db R}}
\def\dbS{{\db S}} \def\dbT{{\db T}}
\def\dbU{{\db U}} \def\dbV{{\db V}}
\def\dbW{{\db W}} \def\dbX{{\db X}}
\def\dbY{{\db Y}} \def\dbZ{{\db Z}}

\font\teneusm=eusm10  \font\seveneusm=eusm7 
\font\fiveeusm=eusm5 
\newfam\eusmfam 
\textfont\eusmfam=\teneusm 
\scriptfont\eusmfam=\seveneusm 
\scriptscriptfont\eusmfam=\fiveeufm 
\def\scr#1{{\fam\eusmfam\relax#1}}


%Upper-case Script Letters:
%\def\scr#1{{\fam\eusmfam\relax#1}}

\def\scrA{{\scr A}}   \def\scrB{{\scr B}}
\def\scrC{{\scr C}}   \def\scrD{{\scr D}}
\def\scrE{{\scr E}}   \def\scrF{{\scr F}}
\def\scrG{{\scr G}}   \def\scrH{{\scr H}}
\def\scrI{{\scr I}}   \def\scrJ{{\scr J}}
\def\scrK{{\scr K}}   \def\scrL{{\scr L}}
\def\scrM{{\scr M}}   \def\scrN{{\scr N}}
\def\scrO{{\scr O}}   \def\scrP{{\scr P}}
\def\scrQ{{\scr Q}}   \def\scrR{{\scr R}}
\def\scrS{{\scr S}}   \def\scrT{{\scr T}}
\def\scrU{{\scr U}}   \def\scrV{{\scr V}}
\def\scrW{{\scr W}}   \def\scrX{{\scr X}}
\def\scrY{{\scr Y}}   \def\scrZ{{\scr Z}}



\def\scr#1{{\fam\eusmfam\relax#1}}

\def\scrH{{\scr H}}
\def\scrL{{\scr L}}

\line{\dun --- DRAFT ---\hfill{\rm IASSNS-HEP-97/72}}

\bigskip\bigskip
\centerline{\boldtitlefont Lecture I}
\bigskip
\centerline{\boldtitlefont Point Particles 
 {\ninerm vs} Strings}

\medskip
\centerline{Eric D'Hoker}

\frenchspacing

\dspace
\bigskip
In physics, the structure of matter is understood in
terms of the dynamics of smaller constitutents.
Atoms are composed of nuclei and electrons, nuclei are
made up of protons and neutrons, which in turn are
bound states of quarks.
To the present day limit of distance scales up to
approximately
$10^{-15}$ cm, electrons and quarks do not show any
substructure, and are viewed as ``elementary
particles''.
(Which particles are elementary in this way may change
over time, since at smaller distance scales
substructure may be found in terms of more elementary
particles.)
In quantum field theory, such elementary particles are
described as {\it points} in space.

Forces between particles result from exchanges between
them of other elementary particles, and the
fundamental interactions occur in a {\it local} way,
at coincident time at the same point in space.
For example electro-magnetic interaction forces
between charged particles result from the exchange of
photons, with local couplings to the charged
particles.
$$
\vbox{\epsfxsize=2in\epsfbox{fig1.eps}}
$$

\noindent
The photon thereby transfers momentum and produces a
force, the Coulomb force.

Quantum field theory of point particles (QFT) has been
extremely successful in describing quantitatively
three out of the four fundamental forces of Nature by
{\it Yang-Mills theory}: electro-magnetic, weak and
strong interactions.
However, no consistent quantum field theory
description appears to be available (or to exist?) for
the force of gravity.
One thus seeks an enlargement of the framework of QFT
in which all forces of nature fit consistently with
the known physical principles of quantum mechanics and
(special and general) relativity.

String theory to date seems to be the only such
candidate.
In fact my task will be to show that consistent string
theories automatically contain gravity, supersymmetry
and under certain circumstances Yang-Mills.

\bigskip\noindent
A. {\bf Strings}

In string theory, particles are not described as
points, but instead as strings: one-dimensional extended
objects.
If $M_s$ and $M(\sim\dbR\times M_s)$ denote the space and
space-time manifolds respectively, then we picture
strings as follows:

\centerline{\null\qquad\epsfxsize=5.25in\epsfbox{fig2.eps}}

While the point particle sweeps out a one-dimensional {\it
worldline}, the string sweeps out a {\it worldsheet},
i.e. a $2$-dim real surface.
For a {\it free string}, the topology of the
worldsheet is a {\it cylinder} (in the case of a
closed string) or a {\it sheet} (for an open string).

Roughly, different elementary particles correspond to
different vibration modes of the string jus as
different minimal notes correspond to different
vibrational modes of musical string instruments.

It will turn out that the physical size of strings is
set by gravity, more precisely the Planch length 
$\ell_P=\sqrt{G_N\hslash c^{-3}}\sim 10^{-33}$ cm.
This scale is so small that we effectively only see
point particles at our distance scales.
Thus, for length scales much larger than $\ell P$, we
expect to recover a quantum field theory description
of point particles, plus typical string corrections
that represent physics at the Planck scale.

Extended objects arise in many areas of physics.
For examples {\it flux tubes} arise in superconductivity, 
and are used to describe the physics of
cosmic strings and vortices.
However, those objects do have a transverse size, and
are only {\it approximately one dimensional}.
String theory considered here will always involve
strictly one-dimensional objects, suitable for
the problems of quantum gravity and particle physics.
In certain instances, it may be possible to use the
strictly one-dimensional string modules as
approximations to the dynamics of flux tubes.

\bigskip\noindent
B. {\bf Interactions}

While the string itself is extended, the fundamental
interactions are {\it local}, just as for point
particles.
The interaction takes place when strings overlap in
space at the same time.

For {\it closed} strings
$$
\vbox{\epsfxsize=4in\epsfbox{fig3.eps}}
$$

For {\it open} strings, end points can join
$$
\vbox{\epsfxsize=4in\epsfbox{fig4.eps}}
$$

%\vfill\supereject

For consistency, as soon as an interacting theory
contains open strings, it must also contain closed
strings, since end points can join.
Such theories are ``{\it open string theories}''.
$$
\vbox{\epsfxsize=4.5in\epsfbox{fig5.eps}}
$$
On the other hand, it is consistent to have a theory
of only closed strings, called  {\it closed string
theories}.
Other interactions result from combining the interactions
defined above
$$
\vbox{\epsfxsize=4.5in\epsfbox{fig6.eps}}
$$

In point particle theories, the fundamental
interactions are read off from the QFT Lagrangian.
An interaction occurs at a {\it geometrical point},
where the worldlines join and cease to be a manifold.
In Lorentz invariant theories (with $M$ flat Minkowski
space-time), the interaction point is Lorentz
invariant.
To specify how the point particles interact,
additional data must be supplied at the interaction
point, giving rise to {\it many possible} distinct
quantum field theories.

\centerline{\null\qquad\epsfxsize=5.0in\epsfbox{fig7.eps}}

In string theory, the interaction point {\it depends}
upon the Lorentz frame chosen to observe the process.
In the figure above, equal time slices are indicated
from the point of view of two different Lorentz
frames, schematially indicated by $t$ and $t'$.
The closed string interaction, as seen from frames $t$
and $t'$, occurs at times $t_2$ and $t'_2$ and at
(distinct) points $P$ and $P'$ respectively.

Lorentz invariance of interaction forbids that any 
point on the worldsheet be singled out as interaction
point. 
Instead, the interaction results purely from the
joining and splitting of strings.
While free closed
strings are characterized by their topology being
that of a cylinder, interacting strings are
characterized by the fact that their associated
worldsheet is connected to at least 3 strings,
incoming and/or outgoing.

As a result, the free string determines the nature of
the interactions completely, leaving only the string
coupling constant undetermined.

\vfill\eject

\noindent
%\medskip\noindent
{\it Additional structure}

\item{$\scriptstyle\bullet$}
Oriented strings: \ all worldsheets are assumed to be 
orientable.

\medskip
\item{$\scriptstyle\bullet$}
Non-oriented strings: \ worldsheets that are
non-orientable such as the M\"obius strip, Klein
bottle,  are included as  possible
configurations, as well. 

\medskip\noindent
We shall almost exclusively deal with {\it oriented}
strings only.

\bigskip\noindent
C. {\bf Loop expansion -- Topology of closed surfaces}

For simplicity consider closed oriented strings only,
so that the associated worldsheet is also oriented.
A general string configuration describing the process
in which $M$ incoming strings interact and produce $N$
outgoing strings looks at the topological level like
a closed surface with $M+N=E$ boundary components and
any number of handles.
$$
\vbox{\epsfysize=2.0in\epsfbox{fig8.eps}}
$$

\bigskip
The internal loops may arise when virtual particle
pairs are produced, just as in quantum field theory.
For example, a Feynman diagram in quantum field theory
that involves a loop is shown below together with its
 corresponding in  string theory:
$$
\vbox{\epsfysize=1.0in\epsfbox{fig9.eps}}
$$

Surfaces associated with closed oriented strings have
two topological invariants: the {\it number of
boundary components $E$} (as we shall see later,
boundary components may be shrunk to punctures, under certain
conditions), and the number $h$ of {\it handles} on
the surface, which equals the {\it genus}.

When $E=0$, we just have the topological
classification of compact oriented surfaces without
boundary:
$$
\vbox{\epsfysize=1.5in\epsfbox{fig10.eps}}
$$

\noindent
Rendering $E>0$ is achieved by removing $E$ discs
from the surface.

Recall that in quantum field theory an expansion in
powers of Planck's constant $\hslash$ yields an
expansion in the number of {\it loops} of the
associated Feynman diagram, for a given number of
external states:
$$
\hslash^{E+h-1}
\left\{ \eqalign{
\hslash &\quad \hbox{for every propagator}\cr
\hslash^{-1} &\quad \hbox{for every vertex}\cr
-1 &\quad \hbox{for overall momentum conservation.}\cr}
\right.
$$
Thus, in string theory we expect that, for a given
number of external strings $E$, the topological
expansion genus by genus should correspond to a {\it
loop} expansion as well.
(See Kazhdan's 4-th lecture.)

In quantum field theory, there are in general many
Feynman graphs that correspond to an amplitude with a
given number of external particles and a given number
of loops,
e.g. for $E=4$
 external particles and $h=1$ loop in $\phi^3$ theory
we have
$$
\vbox{\epsfysize=1.5in\epsfbox{fig11.eps}}
$$
The same process in string theory (for closed oriented
strings) is described by just a single diagram
$$
\vbox{\epsfysize=1.0in\epsfbox{fig12.eps}}
$$
Later on in the lectures we will show that the single
string diagram reproduces all the field theory
diagrams.
The different field theory diagrams arise from
different cells in a {\it cell-decomposition of the
moduli space of surfaces}.

\bigskip\noindent
D. {\bf Transition Amplitudes for Strings}

The fundamental quantities we are interested in
examining in a quantum field theory are the transition
amplitudes for processes in which a number of $M$
incoming particles scatter to produce $N$ outgoing
particles.
(The square modulus of the amplitude yields the
probability for this process to take place.)

The only way we have today to define string theory is
by giving a {\it rule}  for the evaluation of
transition amplitudes, order by order in the loop
expansion, i.e. genus by genus.
The rule is to assign a relative weight to a 
given configuration and then to sum over all
configurations.
To make this more precise, we first describe the space
of configurations.
$$
\vbox{\epsfysize=1.75in\epsfbox{fig13.eps}}
$$
\bigskip
We assume that $\Sigma$ and $M$ are differentiable
manifolds, of dimensions $2$ and $D$ respectively, and
that $x$ is a continuous map from $\Sigma$ to $M$.
If $\xi^m$, $m=1,2$, are local coordinates on $\Sigma$
and $x^\mu$, $\mu=1,\ldots,D$, are local coordinates on
$M$ then the map $x$ may be described by functions
$x^\mu(\xi^m)$ which are continuous.

To each configuration is associated a weight
$$
e^{-S[x,\Sigma,M]},\qquad\qquad S\in\dbC
$$
and the transition amplitude for specified external
strings (incoming and outgoing) is obtained by summing
over all surfaces $\Sigma$ and all possible maps
$x$.
$$
A=\sum\limits_{{\rm surfaces}~\Sigma} \sum\limits_{x}
e^{-S[x,\Sigma,M]}
$$

We now need to specify each of these ingredients:

\medskip\noindent
1) \
We assume $M$ to be a $D$-dimensional  Riemannian manifold,
with metric $(\,\,,\,\,)_G$.
A special case is flat Euclidean space-time $\dbR^D$.
The space-time metric is assumed {\it fixed}.
$$
ds^2=(dx,dx)_G=\mathop{\Sigma}\limits_{\mu,\nu=1}^D
G_{\mu\nu}(x)dx^\mu\otimes dx^\nu\eqno{(1.1)}
$$

\medskip
The physical world is of course not described by a Riemann
manifold, but rather by a pseudo-Riemann space, such
as Minkowski space-time.
We shall always take as a starting definition the
Riemannian situation, and analytically continue to
pseudo-Riemannia spaces to obtain physical transition
amplitudes.
For flat Euclidean space-time, this continuation is
analogous to the correspondence between Schwinger and
Wightman functions in QFT (see Kazhdan's lectures).

\medskip\noindent
2) \
The metric $G$ on $M$ induces a metric on $\Sigma$:
$\gamma=x^*(G)$
$$
\eqalign{
\gamma &=\gamma_{mn}d\xi^m\otimes d\xi^n\cr
\gamma_{mn} &=G_{\mu\nu}{\partial x^\mu\over\partial
\xi^m}{\partial x^\nu\over\partial\xi^n}\cr}
$$
This metric is non-negative, but depends upon $x$.
It is advantageous to introduce an intrinsic
Riemannian metric $g$ on
$\Sigma$, independently of $x$; in local coordinates,
we have
$$
g=g_{mn}(\xi)d\xi^m\otimes d\xi^n.\eqno{(1.2)}
$$

\medskip\noindent
3) \ 
The worldsheet action $S[x,\Sigma,M]$ will in general
depend on the map $x$.
We choose $S$ to be intrinsic, with the following
properties
$$
\cases{
\Diff^{\plus}(\Sigma) \quad\,\,\,\hbox{\rm invariant} &\cr
\Diff(M) \qquad\hbox{\rm invariant} &\cr
\hbox{\rm local in its dependence on } x,g, 
\hbox{ \rm and } G &\cr
\hbox{\rm renormalizable as a QFT.}\cr}
$$

A natural intrinsic candidate for $S$ is the area of
$x(\Sigma)$, which gives the {\it Nambu-Goto action}
$$
\Area\left(x\left(\Sigma\right)\right)=
\int\nolimits_{\Sigma}d\mu_\gamma=\int\nolimits_{\Sigma}
d^2\xi\sqrt{\det\,\gamma_{mn}}\eqno{(1.3)}
$$
It depends only upon $G$ and $x$, but not on $g$.
The difficulty with this action is that it is
non-polynomial in $x$ and its derivatives, even in flat
space-time.
Its quantization is difficult to define unambiguously.

Instead, we take as starting point the {\it Polyakov
action}
$$
\eqalign{
S[x,g,G] &=\kap\int\nolimits_{\Sigma}(dx,*dx)_G\cr
&=\kap\int\nolimits_{\Sigma}d\mu_g
g^{mn}\partial_m x^\mu \partial_n
x^\nu G_{\mu\nu}(x)\cr}
\eqno{(1.4)}
$$
Here $\kap$ is the {\it string tension}, which is a
positive constant with dimension of inverse length
square.
The stationary points of $S$ with respect to $g$ are
at $g^0=e^\phi \gamma$ for some function $\phi$ on
$\Sigma$,
and thus $S[x,g^0,G]\sim\Area\left(x(\Sigma)\right)$.

The Polyakov action leads to well-defined transition
amplitudes, obtained by integration over the space 
$\Met(\Sigma)$ of all positive metrics on $\Sigma$ 
for a given topology, as well as over the space of all
maps $\hbox{Map }(\Sigma,M)$: We {\it define}
$$
A=\sum\limits_{{\scriptstyle\rm topologies}\atop
{\scriptstyle \Sigma}}\int\nolimits_{{\rm
Met}(\Sigma)}{1\over\scrN(g)}\int\nolimits_{{\rm
Map}(\Sigma,M)}Dx\,e^{-S[x,g,G]},\eqno{(1.5)}
$$
and make the following remarks.

\medskip
\item{$\scriptstyle\bullet$}
$\scrN$ is a normalization factor, to be determined
later.

\smallskip
\item{$\scriptstyle\bullet$}
The measures $Dg$ and $Dx$ are constructed from 
$\Diff^{\plus}(\Sigma)$ and $\Diff(M)$ invariant $L^2$
norms on $\Sigma$ and $M$.

\smallskip
\item{$\scriptstyle\bullet$}
For fixed metric $g$, the action $S$ is
well-known: \ its stationary points are the harmonic
maps $x\colon\,\Sigma\to M$.
Here, however, $g$ varies and in fact is to be
integrated over.

\item{}(This is a reflection of the difference between
quantum field theory and strings.)

\smallskip
\item{$\scriptstyle\bullet$}
For a general metric $G$, the action $S$ defines a
{\it non-linear sigma model}, which is renormalizable
because the dimension of $\Sigma$ is $2$.
It would not in general be renormalizable in dimension
higher than $2$, which is usually regarded as an
argument against the existence of fundamental membrane
theories.

\bigskip\noindent
E. {\bf Weyl Invariance and Vertex Operator
Formulation}

The action $S$ is also invariant under {\it Weyl
rescalings} of the metric $g$ by a positive function
on $\Sigma$:
$$
g\to e^{2\sigma}g\qquad\qquad
\sigma\colon\,\Sigma\to\dbR.
$$
In general, Weyl invariance of the full amplitude
may be spoiled by anomalies.
We shall establish later on that Weyl invariance is a
crucial ingredient to obtain a consistent theory.

Assuming Weyl invariance of the full amplitude, the
integral defining $A$ may be {\it simplified} in two ways.

\medskip\noindent
1)\enspace
The integration over ${\rm Met}(\Sigma)$ effectively
collapses to an integration over the {\it moduli space
of surfaces}, which is finite dimensional, for each
genus $h$.

\smallskip\noindent
2)\enspace
The boundary components of $\Sigma$ --- characterizing
external string states --- may be mapped to regular
points on an underlying compact surface without
boundary by conformal transformations.
The {\it data}, such as momenta and other quantum
numbers of the external states, are mapped into {\it
vertex operators}.
The amplitudes are now
$$
A=\sum\limits_{h=0}^\infty\int\nolimits_{\Met(\Sigma)}
Dg{1\over\scrN(g)}\int\nolimits_{\Maps(\Sigma,M)}
Dx\,V_1\ldots V_N e^{-S}\eqno{(1.6)}
$$
for suitable vertex operators $V_1,\ldots V_N$.
We shall construct such vertex operators explicitly in
Lecture 3.

\bigskip\noindent
F. {\bf More general actions}

The action $S$ given above is not the most general one
that satisfies the criteria we listed.
Generalizations are possible when $M$ carries extra
structure.

\medskip
\item{1)}
$M$ carries a $2$-form $B\in\Omega^{(2)}(M)$, the
so-called {it anti-symmetric tensor field}.
The resulting contribution to the action is also that
of a ``non-linear sigma model''"
$$
S_B[x,B]=\int\nolimits_{\Sigma}x^*(B)=
\int\nolimits_{\Sigma}dx^\mu\w dx^\nu B_{\mu\nu}(x)
\eqno{(1.7)}
$$

\smallskip
\item{2)}
$M$ may carry a {\it dilaton field}
$\Phi\in\Omega^{(0)}(M)$ so that
$$
S_\Phi[x,\Phi]=\int\nolimits_{\Sigma}d\mu_g R_g\Phi(x).
\eqno{(1.8)}
$$
where $R_g$ is the Gaussian curvature of $\Sigma$ for
the metric $g$.

\smallskip
\item{3)}
There may be a {\it tachyon field}
$T\in\Omega^{(0)}(M)$ contributing
$$
S_T[x,T]=\int\nolimits_{\Sigma}d\mu_g T(x).
\eqno{(1.9)}
$$

\bigskip\noindent
G. {\bf Transition Amplitude for a single point particle}

The transition amplitude for a single point particle
could in fact be obtained in a  way analogous to how
we prescribed string amplitudes.
Let space-time be again a Riemannian manifold $M$, with
metric $(\,\,,\,\,)_G$.
The prescription for the transition amplitude of a
particle travelling
from a point $y\in M$ to a point $y'$ to $M$ is 
expressible in terms
of a sum over all (continuous) paths connecting $y$
to $y'$:
$$
A(y,y')=\sum\limits_{{{\scriptstyle\rm paths}}\atop
\scriptstyle{{\rm joining}~y~{\rm and}~y'}}e^{-S {\rm
[path]}}
$$
Paths may be parametrized by maps from $C=[0,1]$ into $M$
with $x(0)=y$, $x(1)=y'$.
A simple worldline action for a massless
particle is obtained by introducing a metric $g$ on
$[0,1]$
$$
S[x,g]={1\over 2}\int\nolimits_{C}d\tau\,g(\tau)^{-1}
\xdot^\mu\xdot^\nu G_{\mu\nu}(x)
\qquad\qquad\xdot={dx\over d\tau}
$$
which is invariant under $\Diff^{\plus}(C)$ and
$\Diff(M)$.

The analogous prescription for the point particle
transition amplitude is
$$
A(y,y')=\int\limits_{\Met(C)}Dg{1\over\scrN}
\int\limits_{\Map(C,M)}Dx\,e^{-S[x,g]}
$$

\medskip
Notice that for string theory, we had a prescription for
transition
amplitudes valid for all topologies of the worldsheet.
For point particles, there is only the topology of the
interval $C$, and we can only describe a single point
particle, but not interactions with other point
particles.
To put those in, we would have to supply additional
information.

\medskip\noindent
It is very instructive to work out the amplitude $A$
by carrying out the integrations.
The only $\Diff^{\plus}(C)$ invariant of $g$ is the
length $L=\int_0^1d\tau\,g(\tau)$; all else is
generated by $\Diff^{\plus}(C)$, and may be gauge-fixed 
following the standard Faddeev-Popov procedure.
Defining the normalization factor to be the volume
of $\Diff(C)$: 
$\scrN=\Vol(\Diff(C))$ we have $Dg=Dv\,dL$ and the
transition amplitude becomes
$$
\eqalign{A(y,y') &=\int\nolimits_0^\infty dL\int
  Dx\,e^{-{1\over 2L}\int\nolimits_0^1
d\tau(\xdot,\xdot)_G}\cr
&=\int\nolimits_0^\infty dL\Bigl<y'\vert
e^{-L\square}\vert y\Bigr>=\Bigl<y'\vert
{1\over\square}\vert y\Bigl>.\cr}
$$
Thus, the amplitude is just the Green function at
$(y,y')$ for the Laplacian  (for Euclidean space-time)
or the d'Alembertian (in the Minkowski case),
and corresponds to the propagation of a massless particle.

\bigskip\noindent
H. {\bf Generalized Point Particle Propagation}

The manifold $M$ may carry structure in addition to
the metric $(\,\,,\,\,)_G$, which is reflected in the
point particle action.

\medskip
\item{1.}
An $U(1)$ line bundle on $M$ with connection $A$ gives rise to
$$
S_A=ie\int\nolimits_{C}x^*(A)=ie\int\nolimits_{C}
d\tau\,\xdot^\mu\,A_\mu(x),
\eqno{(1.10)}
$$
i.e.  point particle with charge $e$ in the presence
of a $U(1)$ gauge field.

\smallskip
\item{2.}
An $\SU(N)$ vector bundle on $M$ with connection $A$.
To describe a particle in the defining
representation $\rho$ of $\SU(N)$, introduce worldline
fermions $\lam^i(\tau)$ $i=1,\ldots,N$ and the action
$$
S_{A_\rho}=\int\nolimits_{C}\bar{\lam}(d+x^*(A_\rho))\lam
\eqno{(1.11)}
$$

\smallskip
\item{3.}
The spin bundle over $M$ with spin connection $\partial$,
describes space-time spinors of $M$.
Introduce worldline fermions $\psi$ and the action
$$
S_\omega=\int\nolimits_{C}(\psi,\partial\psi)_G
\eqno{(1.12)}
$$
The total particle worldline action $S+S_\omega$ has
worldline supersymmetry.
(See Witten's problem sets.)



\bye

