%From: David Kazhdan <kazhdan@math.ias.edu>
%Date: Thu, 20 Feb 1997 15:31:47 -0500

\input amstex
\documentstyle{amsppt}
\magnification 1200
\NoRunningHeads
\NoBlackBoxes
\document

\def\h{\frak h}
\def\tW{\tilde W}
\def\Aut{\text{Aut}}
\def\tr{{\text{tr}}}
\def\ell{{\text{ell}}}
\def\Ad{\text{Ad}}
\def\u{\bold u}
\def\m{\frak m}
\def\O{\Cal O}
\def\tA{\tilde A}
\def\qdet{\text{qdet}}
\def\k{\kappa}
\def\RR{\Bbb R}
\def\be{\bold e}
\def\bR{\overline{R}}
\def\tR{\tilde{\Cal R}}
\def\hY{\hat Y}
\def\tDY{\widetilde{DY}(\g)}
\def\R{\Bbb R}
\def\h1{\hat{\bold 1}}
\def\hV{\hat V}
\def\deg{\text{deg}}
\def\hz{\hat \z}
\def\hV{\hat V}
\def\Uz{U_h(\g_\z)}
\def\Uzi{U_h(\g_{\z,\infty})}
\def\Uhz{U_h(\g_{\hz_i})}
\def\Uhzi{U_h(\g_{\hz_i,\infty})}
\def\tUz{U_h(\tg_\z)}
\def\tUzi{U_h(\tg_{\z,\infty})}
\def\tUhz{U_h(\tg_{\hz_i})}
\def\tUhzi{U_h(\tg_{\hz_i,\infty})}
\def\hUz{U_h(\hg_\z)}
\def\hUzi{U_h(\hg_{\z,\infty})}
\def\Uoz{U_h(\g^0_\z)}
\def\Uozi{U_h(\g^0_{\z,\infty})}
\def\Uohz{U_h(\g^0_{\hz_i})}
\def\Uohzi{U_h(\g^0_{\hz_i,\infty})}
\def\tUoz{U_h(\tg^0_\z)}
\def\tUozi{U_h(\tg^0_{\z,\infty})}
\def\tUohz{U_h(\tg^0_{\hz_i})}
\def\tUohzi{U_h(\tg^0_{\hz_i,\infty})}
\def\hUoz{U_h(\hg^0_\z)}
\def\hUozi{U_h(\hg^0_{\z,\infty})}
\def\hg{\hat\g}
\def\tg{\tilde\g}
\def\Ind{\text{Ind}}
\def\pF{F^{\prime}}
\def\hR{\hat R}
\def\tF{\tilde F}
\def\tg{\tilde \g}
\def\tG{\tilde G}
\def\hF{\hat F}
\def\bg{\overline{\g}}
\def\bG{\overline{G}}
\def\Spec{\text{Spec}}
\def\tlo{\hat\otimes}
\def\hgr{\hat Gr}
\def\tio{\tilde\otimes}
\def\ho{\hat\otimes}
\def\ad{\text{ad}}
\def\Hom{\text{Hom}}
\def\hh{\hat\h}
\def\a{\frak a}
\def\t{\hat t}
\def\Ua{U_q(\tilde\g)}
\def\U2{{\Ua}_2}
\def\g{\frak g}
\def\n{\frak n}
\def\hh{\frak h}
\def\sltwo{\frak s\frak l _2 }
\def\Z{\Bbb Z}
\def\C{\Bbb C}
\def\d{\partial}
\def\i{\text{i}}
\def\ghat{\hat\frak g}
\def\gtwisted{\hat{\frak g}_{\gamma}}
\def\gtilde{\tilde{\frak g}_{\gamma}}
\def\Tr{\text{\rm Tr}}
\def\l{\lambda}
\def\I{I_{\l,\nu,-g}(V)}
\def\z{\bold z}
\def\Id{\text{Id}}
\def\<{\langle}
\def\>{\rangle}
\def\o{\otimes}
\def\e{\varepsilon}
\def\RE{\text{Re}}
\def\Ug{U_q({\frak g})}
\def\Id{\text{Id}}
\def\End{\text{End}}
\def\gg{\tilde\g}
\def\b{\frak b}
\def\S{\Cal S}
\def\L{\Lambda}
\def\tl{\tilde\lambda}

\topmatter
\title Lecture II-3: Infrared behavior and the S-matrix of
the 2-dimensional sigma-model  with target space $S^{N-1}$
\endtitle
\author {\rm {\bf Edward Witten} }\endauthor
\endtopmatter

\centerline{Notes by Pavel Etingof and David Kazhdan}

\vskip .1in

{\bf 3.1. Infrared behaviour of 2-dimensional sigma-models with target space 
$S^{N-1}$.}

Consider the theory of $N$ scalar bosons, with the Lagrangian
$$
\Cal L=\int d^dx(\frac{1}{2}(\nabla\phi)^2+\frac{g}{4!}(\phi^2-a^2)^2),\tag 3.1
$$
where $\phi:\R^d\to\R^N$. 
In Lecture II-1 we saw that above $d=2$ this theory has symmetry breaking, 
from $SO(N)$ to $SO(N-1)$. In particular, for any 
point $y$ on the sphere $\phi^2=a^2$ there exists a realization 
$\Cal H_y$ of this theory, with $\<\phi\>=y$. 
We also saw that the low energy effective theory for (2.1) is the  
sigma-model with target space $S^{N-1}$. 

In 2 dimensions the first statement fails: there is no symmetry breaking, and 
there is only one realization, with an action of $SO(N)$. However, the
second statement remains valid: the low energy effective theory is
the sigma-model. 

In $d>2$, we know that the sigma-model is infrared free (in the zero 
approximation); more precisely, it converges in the infrared limit 
to a free theory of N-1 massless scalars (Goldstone bosons).
This is not the case in 2 dimensions. Indeed, in 2 dimensions 
the sigma-model is renormalizable, so it has marginal 
interactions (i.e. interactions with a 0-dimensional coupling).
On the other hand, it was shown in Gawedzki's lecture
on sigma-models that the $\beta$-function of the 2-dimensional sigma model 
into a space of positive curvature is negative (the model is asymptotically
free in the UV limit). Therefore, the marginal interactions are 
relevant in the infrared, and apriori
we cannot conclude that the model is not infrared free, even 
in the zero approximation. In fact, what happens (as we will see today)
is that instead of $N-1$ massless particles that we had in $d>2$,
we will have $N-1$ massive particles, so the infrared limit 
is trivial (the correlation functions are analytic at the origin in the 
momentum space, and decay exponentially at infinity in position space). 

In the first half of the lecture, 
we will show that the 2-dimensional sigma-model is indeed infrared 
trivial for large $N$. Namely, we will show that 
all coefficients of the expansion
of the correlation functions in a series in $1/N$ are analytic 
at the origin in momentum space, and the first term of this expansion 
gives the correlation functions for $N$ free massive particles.  

{\bf Remark 1.} Since the $\beta$-function of the 2-dimensional 
sigma-model into the sphere is negative, it is believed that this theory 
actually exists. So we will now study the infrared behavior of 
a (probably) actual quantum field theory.  

{\bf Remark 2.} Since the theory defined by the quartic Lagrangian (3.1) 
exists rigorously in $d=2$, it can be regarded as an ultraviolet
cutoff of the sigma-model. The characteristic momentum scale $\Lambda$ 
of this cutoff is the mass of the radial component of $\phi$. 
This cutoff is called the linear sigma-model, as opposed to the nonlinear 
sigma-model, which is the model of maps to the sphere. 

\vskip .1in

{\bf 3.2. Computation of the infrared behavior in the $N\to\infty$ limit.}

Now we will compute the infrared behavior of the sigma model in the limit 
$N\to\infty$ using the saddle point approximation in the path integral. 
We will operate with path integrals formally, without worrying 
whether they exist or not. 

The path integral which defines the generating series for
 the Euclidean correlation functions 
of the sigma-model is given by the formula
$$
Z(J)=\int_{\phi:\phi^2=1} D\phi e^{-\frac{1}{2\l}\int d^2x |d\phi|^2 
+Z_J\int d^2x J\phi },\tag 3.2
$$
where $\phi,J:\R^2\to \R^N$, and $\l$ is a coupling constant. 
The factor $Z_J$ is added because 
the operator $\phi$ has anomalous dimension 
(see Gross' lecture 3) and requires multiplicative renormalization. 
(As we know, $Z_J$ is cutoff-dependent, and diverges 
as the cutoff goes to infinity.)
 
It is more convenient to integrate over maps to a linear space
than over maps to a curved manifold. Therefore, it is useful to rewrite
integral (3.2) in the form
$$
Z(J)=\int_{\phi:\R^2\to\R^N} D\phi 
\prod_{x\in \R^2}\delta(\phi^2(x)-1)e^{-\frac{1}{2\l}\int d^2x |d\phi|^2 
+Z_J\int d^2x J\phi }.\tag 3.3
$$
We will now use the formula
$$
\delta(a)=\frac{1}{2\pi}\int_{-\infty}^{\infty}e^{ia\sigma}d\sigma.\tag 3.4
$$
Substituting (3.4) into (3.3), we get (up to a factor):
$$
Z(J)=\int D\phi\int D\sigma 
e^{i\int d^2x(\frac{\sigma}{2}(\phi^2-1)) -\frac{1}{2\l}\int d^2x |d\phi|^2 
+Z_J\int d^2x J\phi }.\tag 3.5
$$
Thus, the use of formula (3.4) led to introduction of an auxiliary scalar field
$\sigma$.

Now we will change the order of integration and integrate with respect to 
$\phi$. Our goal is to reduce the integral to the form 
in which $N$ (the dimension of the target space) enters analytically. 
We are lucky that the sphere is a quadric, so the integral with 
respect to $\phi$ is Gaussian. So we can compute it explicitly, and get 
$$
Z(J)=\int D\sigma \text{det} \biggl(-\frac{\Delta}{\l}-i\sigma\biggr)^{-N/2}
e^{-i\int d^2x\frac{\sigma}{2} -\frac{1}{2}Z_J^2\int d^2x
J\cdot (\Delta/\l+i\sigma)^{-1}J }\tag 3.6
$$
(the power $-N/2$ of the determinant appears because we are doing $N$ 
independent Gaussian integrals in the components $\phi_i$ of $\phi$). 
It is useful to raise the determinant into the exponential:
$$
Z(J)=\int D\sigma e^{-\frac{N}{2}\text{Tr}\ln(-\frac{\Delta}{\l}-i\sigma)
-i\int d^2x\frac{\sigma}{2} -\frac{1}{2}Z_J^2\int d^2x
J\cdot (\Delta/\l+i\sigma)^{-1}J }.\tag 3.7
$$
Now we want to bring integral (3.6) to the form where there is a factor of $N$
in front of all terms in the exponential (except for the J-term, which 
serves to expand the answer in the formal series in $J$), and then evaluate 
the integral using the saddle-point approximation. 
For this purpose we have to make the change of variable
$\rho=\sigma/N$ and rescale the coupling constant by $\tl
=\l N$. After these changes the integral, up to a factor, looks like
$$
Z(J)=\int D\sigma e^{-\frac{N}{2}[\text{Tr}\ln(-\Delta-i\tl\rho)
+i\int d^2x\frac{\rho}] -\frac{\tl}{2N}Z_J^2\int d^2x
J\cdot (\Delta+i\tl\rho)^{-1}J }.\tag 3.8
$$
In this form, we can already apply the saddle point approximation.
(This is really saddle point and not stationary phase, as the function in the 
exponential is complex). 

The most natural thing is to look for a Poincare invariant saddle point, since
the integral we are considering is Poincare-invariant. Thus, we should look 
for a saddle point $\rho(x)$ which is a constant, $\rho(x)=\rho$. 
The value of $\rho$ is found from the saddle point equation
$$
\frac{d}{d\rho}
[\text{Tr}\ln(-\Delta-i\tl\rho)
+i\rho\int d^2x]=0.\tag 3.9
$$
(at this point we are making an IR cutoff $L$, i.e. assume that the spacetime 
is a torus of size $L\times L$). Let us rewrite (3.9) using the momentum space
reprsentation of the operator $-\Delta-i\tl\rho$,  
and send $L$ to infinity. Then we get
$$
\tl\int \frac{d^2k}{(2\pi)^2}\frac{1}{k^2-i\tl\rho}
=1.\tag 3.10
$$ 
The integral on the left hand side is logarithmically UV divergent. 
So we should introduce an UV cutoff $\L$ in momentum space
(i.e. consider the integral over the ball $|k|<\L$). Then 
(3.10) becomes 
$$
\frac{\tl}{4\pi}\ln\frac{\L^2}{-i\tl\rho}=1.\tag 3.11
$$
This shows that
$$
\rho=\frac{i\L^2}{\tl}e^{-\frac{4\pi}{\tl}}.\tag 3.12
$$ 
In Gross' lecture 3 it was shown that in an asymptotically free theory, 
the effective coupling constant at scale $\L$ depends on the scale 
according to the formula
$$
\tl(\L)=\frac{1}{A\ln \L}+o(\frac{1}{\ln \L}), \L\to\infty,\tag 3.13
$$
where $A$ is minus the coefficient of the 1-loop beta-function. 
In the sigma-model it turns out that $A=(2\pi)^{-1}\frac{N-2}{N}$,
so it equals to $(2\pi)^{-1}$ 
modulo $1/N$ (see the end of Gawedzki lecture 3). 
It can be shown, that modulo $1/N$, the beta-function equals 
to its 1-loop approximation. This implies that  
there exists a limit $M^2=\lim_{\L\to\infty}(-i\tl\rho)$, which is positive. 

{\bf Remark.} 
Thus, in the quantum theory we have a characteristic momentum scale
defined by $M$, despite the fact that the classical theory 
is conformally invariant, and the Lagrangian has only a dimensionless coupling
$\l$ and defines no particular momentum (or length) scale.
This phenomenon is called ``dimensional transmutation''.
It gives a very vivid demonstration of the fact (which we already know), that 
there may be no canonical way of renormalizing a 
theory defined by a scale-invariant Lagrangian. 

Recalling the form of the J-term in the integral $Z(J)$, we conclude 
that the $J^2$ term of the expansion of $Z(J)$ has order $1/N$.
Thus, we should rescale the fields $\phi$, $\phi\to N^{1/2}\phi$,  
to get a nonzero limit of correlation functions as 
$N\to\infty$. After this change it turns out that 
the 2-point functions of $\phi_i$ have a nonzero limit, 
while the connected $4,6,...$-point functions vanish modulo $1/N$.  
 Thus,  
in the first order approximation in $1/N$, our theory is free.
Moreover, the propagator of this first order theory 
is $\frac{1}{k^2+M^2}$, where $M$ is defined as above. 
Thus, the limiting theory is the theory of $N$ independent free 
massive scalars of mass $M$. 

The connected 4-point function is of order $1/N$. Therefore, the effective 
coupling constant between the massive scalars is of order $1/N$. 

The fact that the fields $\phi_i$ decouple in the large $N$ limit 
and become massive scalar fields, has the following probabilistic 
explanation. 
   
Consider the theory of $N$ free massive bosons $a_i$, described by the 
Lagrangian  
$$
\Cal L=\int (\prod Da_i) e^{-N\sum_i(|da_i|^2+M^2a_i^2)}.\tag 3.14
$$
Consider the expectation value 
$\<a_i^2\>$ of the operator $a_i^2$ in this theory for a fixed $i$. 
This value requires renormalization: it is given by the
integral 
$$
\int \frac{d^2k}{(2\pi)^2}\frac{1}{N(k^2+M^2)},\tag 3.15 
$$
which is UV divergent. So we should introduce an UV cutoff $\L$.
Then we have $\<a_i^2\>= 
\frac{1}{4\pi N}\ln \L^2/M^2$. 
 Therefore, 
$\<\sum_i a_i^2\>=\frac{1}{4\pi}\ln \L^2/M^2$. 
On the other hand, by the law of large numbers, 
if $d$ is the dispersion of $a_i^2$, then the dispersion of
$\sum a_i^2$ is $d/\sqrt{N}$, so for large $N$ the vector 
$(a_1,...,a_N)$ stays close to the the sphere $\sum x_i^2=
\frac{1}{4\pi}\ln \L^2/M^2$. 
Thus, the theory of $N$ independent massive fields for large $N$ 
 behaves like the sigma-model for the sphere, whose radius 
varies logaritmically with the cutoff. 

{\bf 3.3. Computation of the S-matrix.}

Now we will study the sigma-model into the sphere in a totally different way.
Namely we will show that this model is integrable even for finite $N$, 
which means that its S-matrix can be computed explicitly. 

Recall the Coleman-Mandula theorem (Bernstein's lecture 1): in a field theory 
of dimension $d>2$ with a mass gap and a nondegenerate 
S-matrix, any even infinitesimal symmetry of the S-matrix is  
a linear combination of an element 
of the Poincare Lie algebra and a symmetry which commutes with the Poincare
Lie algebra. In other words, if there exists a ``forbidden'' symmetry
(not having such a decomposition), then the S-matrix, under some technical 
conditions, equals 1, i.e. the theory is free. 

In two dimensions, the Coleman-Mandula theorem is false. That is, there 
exist 2-dimensional 
quantum field theories with $S\ne 1$  which have an infinitesimal 
symmetry forbidden by the Coleman-Mandula theorem. 
In fact, we will see that the 2-dimensional sigma-model 
has this property. 

Thus, in 2 dimensions the information that there exists a ``forbidden''
symmetry is not enough to conclude that $S=1$. However, 
this information still allows to compute the S-matrix. In this section we will
show how to do it for the sigma-model into $S^{N-1}$. 

We will work on a flat Minkowski spacetime with coordinates 
$x_+,x_-$ and the metric $dx_+dx_-$. Let 
$$
\Cal L=\frac{1}{2\l}\int d^2x(\d_+\phi\d_-\phi), \phi^2=1.\tag 3.16
$$ 
be the Lagrangian of
the sigma-model with target $S^{N-1}$ ($\phi\in\R^N$). 
Since this Lagrangian can be written naturally on a curved spacetime,
it has a stress-energy tensor $T=\frac{\delta \Cal L}{\delta g}$. 
This is a symmetric, rank 2 tensor, so it has 3 components in coordinates 
$x_+,x_-$: $T_{++},T_{--},T_{+-}$.  

Let us consider some properties of the stress-energy tensor. 
Classically, we have 
$$
T_{++}=(\d_+\phi)^2, T_{--}=(\d_-\phi)^2.\tag 3.17
$$
Since the sigma-model is classically conformally invariant, 
we have $T_{+-}=0$. However, 
we should expect that quantum mechanically 
conformal invariance is broken (in fact, we know this from the large $N$ 
limit), so quantum mechanically, $T_{+-}\ne 0$; in fact, we will  
see that it is proportional to 
$\d_+\phi\d_-\phi$ (see below). Still, the
current conservation equations 
$$
\d_-T_{++}+\d_+T_{+-}=0, \d_+T_{--}+\d_-T_{+-}=0,\tag 3.18
$$
(which classically reduce to $\d_-T_{++}=0, \d_+T_{--}=0$, because
of vanishing of $T_{+-}$) are still satisfied quantum mechanically, 
if the operator $T_{+-}$ is suitably renormalized (this follows from the fact 
that the Poincare symmetry exists in the quantum theory). 

Let us give another demonstration of why the equation
$$
\d_-T_{++}+\d_+T_{+-}=0.\tag 3.19
$$ 
is satisfied under a suitable renormalization 
of $T_{+-}$. For this purpose recall that classically operators 
in our theory have a bigrading $(d_+,d_-)$ (with respect to 
dilations of $x_+$, $x_-$). In quantum theory, 
since conformal invariance is broken, 
only the diagonal grading $d_+-d_-$ survives, and the 
bigrading becomes a (bi)filtration. 

Let us classify $SO(N)$-invariant  operators of various bidegrees
(under $(d_+,d_-)$ we will list operators of bidegree $d_+-n,d_--n$, 
$n\ge 0$). 

(0,0): 1

(0,1): none

(1,0): none 

(2,0): $(\d_+\phi)^2=T_{++}$

(0,2): $(\d_-\phi)^2=T_{--}$

(1,1): $\d_+\phi\d_-\phi, 1$

(2,1): $\d_+^2\phi\d_-\phi$

This table shows that (3.19) has to be satisfied, 
under a suitable renormalization of $T_{ij}$, for dimensional
reasons (to be more specific, because any operator of
bidegree (2.1) is $\d_+$ of an operator of bidegree (1,1), and
there is no operators of bidegree (1,0). 

{\bf Remark.} In fact, it can be shown that $T_{+-}=\beta\d_+\phi\d_-\phi$,
where $\beta$ is the beta-function. Thus, the beta-function measures the 
failure of conformal invariance. 

We will now construct a symmetry forbidden by the Coleman-Mandula theorem
using the (quantum) stress-energy tensor. The idea of constructing this 
symmetry is the following. Suppose that we found two 
local operators $X_+,X_-$ such that $\d_+X_-=-\d_-X_+$, but 
$X_-\ne  \d_+ Y$ for any $Y$. In this case the 
operator-valued form $J=X_+dx_+-X_-dx_-$ is a nontrivial conserved current, 
and the charge operator $Q=\int_CJ$, where $C$ is a spacelike cycle, is
an infinitesimal symmetry. So we should construct $X_+,X_-$. 

Classically, $X_+=T_{++}^2$ satisfies the above condition, as
$\d_-(T_{++})^2=0$. Quantum mechanically, however, this 
equation is not satisfied. 
So we will again use dimension counting to demonstrate the existence
of a conservation law.

We want to show that there exist 
operators $X_+,X_-$ of degrees (4,0) and (3,1), 
such that $\d_-X_++\d_+X_-=0$, and $X_+\ne  \d_+ Y$. 
To do this, let us extend 
the above table of operators:

(3,0): $\d_+\phi\d_+^2\phi$

(3,1): $(\d_-\phi\d_+\phi)(\d_+\phi)^2$, 
$\d_-\phi\d_+^3\phi$, $(\d_+\phi)^2$. 

(4,0): $((\d_+\phi)^2)^2$, $(\d_+^2\phi)^2$, $\d_+^3\phi\d_+\phi$

(4,1): $\d_-\phi\d_+^4\phi$, 
$(\d_-\phi\d_+\phi)(\d_+\phi\d_+^2\phi)$, $(\d_-\phi\d_+^2\phi)(\d_+\phi)^2$, 
$\d_+\phi\d_+^2\phi$. 

Let $H_{d_+,d_-}$ be the space of operators of bidegree $d_+-n,d_--n$
for all $n\ge 0$.
We have: $\text{dim}H_{3,0}=1$, $\text{dim}H_{3,1}=3$,
$\text{dim}H_{4,0}=3$, $\text{dim}H_{4,1}=4$.
We have two maps $\d_+: H_{3,1}\to H_{4,1}$, $\d_-: H_{4,0}\to H_{4,1}$. 
For dimensional reasons, there is a 2-dimensional subspace $Z$ in
$H_{4,0}$ such that for $X\in Z$ $\d_-X\in \text{Im}\d_+$. 
On the other hand, the image $B$ of $\d_+$ in $H_{4,0}$ is 1-dimensional.
Therefore, the ``cohomology'' group $H=Z/B$ is not zero (it is 1-dimensional).
This ``cohomology'' group represents the conservation law that we are looking 
for.     

{\bf Remark.} It is easy to define the complex for which $Z$ are cocycles,
$B$ are coboundary, and $H$ are cohomology. 

Denote the charge operator corresponding to this conservation law by $Q$. 
It is easy to show that $[Q,P_\pm]=0$, and $[K,Q]=3Q$, where
$K$ is the infinitesimal boost operator of the Poincare group (the 
generator of $\frak{so}(1,1)$), normalized in such a way that 
$[K,P_\pm]=\pm P_{\pm}$, where $P_\pm$ are infinittesimal translations of 
$x_+,x_-$. 

Thus, $Q$ is an infinitesimal symmetry violating the 
conclusion of the Coleman-Mandula 
theorem.

{\bf Remark.} Of course, by interchanging $+$ and $-$ in the definition of 
$Q$, we can obtain another ``forbidden'' conserved charge $Q'$,
with $[K,Q']=-3Q'$.  

Now let us see how we can use $Q$. Let $H_m\subset \Cal H$ be a 
subrepresentation of the Poincare group which represents a particle
of mass $m$ ($H=L^2(\O_m^+)$). Let $|k\>$ denote the state of this particle 
in which its momentum is $k=(k_+,k_-)$, where $k_+k_-=m^2$. (Of course, 
$|k\>\notin H_m$, but for an $L^2$-function $f$ on $\O_m^+$, 
$\int f(k)|k\>dk\in H_m^+$.) We have $P_\pm |k\>=k_\pm|k\>$, and 
$Q|k\>=ck_+^3|k\>$, where $c$ is a $k$-independent constant. 

Let $\l$ be the parameter on the hyperbola $k_+k_-=m^2$ defined by the 
equations $k_\pm=me^{\pm \l}$.  
As we know, a 1-particle state is represented by a wave function,
which satisfies the Klein-Gordon equation:
$$
\psi(x_+,x_-)=\int_{-\infty}^{\infty}\hat\psi(\l)e^{i(k_+x_++k_-x_-)}d\l,
\tag 3.20
$$

The operator $e^{i\gamma P_\pm}$ 
of the Poincare group acts on the wave function by translating 
$x_+,x_-$ by $\gamma$. 
Let us now compute how the ``forbidden'' symmetry $e^{i\delta Q}$ 
acts on the wave function:
$$
e^{i\delta Q}\psi(x_+,x_-)=
\int_{-\infty}^{\infty}\hat\psi(\l)e^{i(\delta ck_+^3+k_+x_++k_-x_-)}d\l\tag 3.21
$$
Thus, the operator $e^{i\delta Q}$ transforms the weight function of a particle
with momentum $k$ by translating $x_+$ by the effective amount 
$\gamma_{\text{eff}}=\delta c k_+^2$. 

Suppose now that we have a particle whose wave function is localized near 
a classical worldline $x=x_0+vt$ ($t+x=x_+,t-x=x_-$). Then the symmetry 
$e^{i\delta Q}$ shifts the worldline in a way dependent on the velocity $v$. 
(This property is characteristic for all symmetries which don't commute 
with the Poincare group, and don't belong to it).

Now it is easy to explain the idea of proof of the Coleman-Mandula theorem,
and why it fails in 2 dimensions. 

In dimension $d>2$, two lines 
generically don't intersect. So, in a theory with a ``forbidden'' symmetry,  
if we have two particles heading 
for a collision (i.e. their worldlines intersect), we can apply the 
``forbidden'' symmetry and obtain two other worldlines which don't intersect,
and are arbitrarily far from each other.
Since the S-matrix is invariant under the symmetry, this argument shows
that it equals $1$. 

In two dimensions, however, two lines generically do intersect, so we cannot 
conclude from the above argument that the S-matrix is 1. Still, since three
lines generically do not intersect at the same point, 
the above argument shows that scattering 
reduces to a successive 2-particle collisions, in which 
two particles are produced. 

The presence of symmetry $Q$ also shows that 
in the 2-2 scattering, the momenta (and hence masses) of incoming particles 
are equal to those of outgoing particles.
Indeed, let $1,2$ label incoming particles, and $3,4$ label outgoing 
particles. The charge $Q$ acts on the state of the i-th particle by 
by multiplication by $C_i(k_i)_+^3$. So by charge conservation
$C_1(k_1)_+^3+C_2(k_2)_+^3=C_3(k_3)_+^3+C_4(k_4)_+^3$.
On the other hand, by momentum conservation 
$(k_1)_++(k_2)_+=(k_3)_++(k_4)_+$. So if $k_1,k_2$ are known, 
$(k_3,k_4)$ can take only a discrete set of values. 
This implies that $(k_3,k_4)=(k_1,k_2)$ or $(k_3,k_4)=(k_2,k_1)$. 

Now let us return to the sigma-model with target space $S^{N-1}$.
In this case, as we know for large $N$ (and believe for all $N\ge 3$),  
we have $N$ massive particles of the same mass $m$, which form 
an $N$-dimensional representaion $V$ of $SO(N)$
(i.e. the space of 1-particle states is $V\o L^2(\O_m^+)$). 
The above arguments allow us to make the following conclusions about 
the scattering matrix.

(i) The scattering matrix is in this case an honest matrix, i.e. 
a function of the form $S(\l_1-\l_2)$ with values in $\End_{SO(N)}(V\o V)$, 
where $\l_i$ are defined by $k_i=(me^{\l_i},me^{-\l_i})$, and 
$k_i$ are the momenta of the incoming particles.
Thus, 
$$
S(\theta)_{ij}^{kl}=F(\theta)\delta_i^k\delta_j^l+
G(\theta)\delta_i^l\delta_j^k+H(\theta)\delta_i^j\delta_k^l,\tag 3.22
$$
where $F,G,H$ are complex-valued functions. 

(ii) $S(\l)$ satisfies the quantum Yang-Baxter equation:
$$
S_{12}(\l_1-\l_2)S_{13}(\l_1-\l_3)S_{23}(\l_2-\l_3)=
S_{23}(\l_2-\l_3)S_{13}(\l_1-\l_3)S_{12}(\l_1-\l_2).\tag 3.23
$$

(iii) $S(\l)$ is unitary. 
  
(iv) Crossing symmetry: $S$ continues 
meromorphically to the complex 
plane, and $S_{ij}^{kl}(\l)=S_{li}^{jk}(\pi i-\l)$.

Property (i) we have already explained. Property (ii) follows 
from the fact that the 3-3 scattering matrix factors as a product 
of three 2-2 scattering matrices, and the order in which 
these three 2-2 collisions occur can be reversed by the ``forbidden'' 
symmetry. Property (iii) is satisfied for the S-matrix of any field theory. 
Property (iv), roughly,
 follows from the fact that the 2-2 S-matrix is the residue
of the pole of the 4-point function, which extends to the complex 
values of momenta, so the S-matrix has to be invariant under the 
discrete part of the Poincare group, which permutes past ant future.

A.Zamolodchikov and Al.Zamolodchikov showed in 1979 
(Annals of Physics, vol. 120) that the S-matrix
is uniquely determined by these conditions for $N\ge 3$, 
up to a scalar $\theta$-dependent factor, provided it is nontrivial
(which we can get from the large $N$ expansion). 
The functions $F,G,H$ for this S-matrix is given by 
an easy explicit formula: 
$$
F(\theta)=1, G(theta=-\frac{2\pi i}{N-2}\theta, H(\theta)=
-\frac{2\pi i}{(N-2)(\pi i-\theta)}.\tag 3.24
$$
This solution of the quantum Yang-Baxter equation was one of the main examples
from which the theory of quantum groups has originated. 
The quantum group corresponding to it is now called 
``the Yangian of $\frak{so}(N)$'' (see Drinfeld's talk ``Quantum groups''
at the International Congress of Mathematicians at Berkeley).

As we indicated, the S-matrix is defined by (3.24) up to 
multiplication by a scalar function 
$f(\theta)$, which satisfies the conditions
$f(\theta)f(-theta)=1$, and $f(\pi i-\theta)=f(\theta)$. 
Thus, the function $f$ is $2\pi i$-periodic. 
Which periodic function corresponds to the actual S-matrix of
the theory is a nontrivial question. A perturbative calculation indicates
that this function should be chosen in such a way that the S-matrix has 
the smallest possible number of singularities. There are two such choices,
and one of the is believed to be realized in the sigma-model. 
The other choice is believed to be realized similarly in the Gross-Neveu 
model. 

Thus, we have shown the integrability of the sigma-model into 
$S^{N-1}$. 

One can show in a similar way the the Gross-Neveu model introduced
in Gross' lecture 4 is also integrable, in the sense that
its S-matrix and 1-particle matrix elements of composite operators 
can be computed. The S-matrix turns out to be the same as above, 
up to the scalar factor. 

But these two models are an exception. Already in the next lecture 
we will consider a slightly more complicated sigma-model -- with target 
space $\C P^N$, for which the explicit form of the S-matrix 
is unknown. 
 
\end
  
Also define the space and time coordinates 
$t,x$ by
$t=\frac{1}{2}(x_++x_-)$, $x=\frac{1}{2}(x_+-x_-)$.

At $t=0$ we have
$$
\psi(x,0)=\int_{-\infty}^{\infty}\hat\psi(\l)e^{i(k_+-k_-)x}d\l,
\tag 3.21
$$
and the wave function is completely determined by its value at $x=0$,
via the Klein-Gordon equation, and the energy positivity condition. 

