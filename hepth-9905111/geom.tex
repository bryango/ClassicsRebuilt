
\section{Anti-de Sitter Space} 
\label{adsgeom}

\subsection{Geometry of Anti-de Sitter Space}

In this section, we will review some geometric facts about 
anti-de Sitter space. One of the important facts is
the relation between the conformal compactifications of $AdS$
and of flat space. In the case of the Euclidean signature
metric, it is well-known that the flat
space $\IR^n$ can be compactified to the $n$-sphere $S^n$ by 
{\it adding a point at infinity}, and a conformal
field theory is naturally defined on $S^n$. On the other hand, 
the $(n+1)$-dimensional hyperbolic space, which is the Euclidean
version of $AdS$ space, can be conformally
mapped into the $(n+1)$-dimensional disk $D_{n+1}$.
Therefore the boundary
of the compactified hyperbolic space is the compactified
Euclid space. A similar correspondence holds in the 
case with the Lorentzian signature metric, as we will see below. 

%\bigskip

%\noindent
\subsubsection{Conformal Structure of Flat Space}
\label{confflatspace}

%\medskip
One of the basic features  of the {\it AdS}/CFT correspondence
is the identification of the isometry group of \adsp\ with the conformal
symmetry of flat Minkowski space $\IR^{1,p}$. Therefore, it would be
appropriate to start our discussion by reviewing the conformal structure of 
flat space. 

\bigskip
\noindent $\circ$ $\IR^{1,1}$

\medskip

We begin with two-dimensional Minkowski
space $\IR^{1,1}$:
\beq
  ds^2 = - dt^2 + dx^2,~~~~~~~~(-\infty < t, x < + \infty).
\label{flat1}
\eeq
This metric can be rewritten by the following coordinate
transformations
\ber
  ds^2 & = & - du_+ du_- ~,~~~~~~~~~(u_\pm = t \pm x) \nonumber \\
          & = & \frac{1}{4\cos^2\tilde{u}_+\cos^2 \tilde{u}_-}
            (-d\tau^2 + d\theta^2) ~,~~~~~
 (u_\pm = \tan \tilde{u}_\pm; ~
\tilde{u}_\pm  = (\tau \pm \theta)/2).
\label{flat2}
\eer
In this way, the Minkowski space is conformally mapped into 
the interior of the compact region, $| \tilde u_\pm| < \pi/2$, 
as shown in figure \ref{F1}. Since light ray  trajectories are 
invariant under a conformal rescaling of the metric, this
provides a convenient way to express the causal structure of
$\IR^{1,1}$. The new coordinates $(\tau, \theta)$ are well
defined at the asymptotical regions of the flat space. Therefore,
the conformal compactification is used to give a
rigorous definition of {\it asymptotic flatness} of spacetime
--- a spacetime is called asymptotically flat if it has the same
boundary structure as that of the flat space after conformal
compactification. 
\begin{figure}[htb]
\begin{center}
\epsfxsize=2.5in\leavevmode\epsfbox{hfig1.eps}
\end{center}
\caption{Two-dimensional Minkowski space is conformally mapped
into the interior of the rectangle.}
\label{F1}
\end{figure} 
\begin{figure}[htb]
\begin{center}
\epsfxsize=4in\leavevmode\epsfbox{hfig2.eps}
\end{center}
\caption{The rectangular region can be embedded in
a cylinder, with $\theta=\pi$ and $\theta=-\pi$ being identified.}
\label{F2}
\end{figure} 

The two corners of the rectangle at
$(\tau, \theta) = (0, \pm \pi)$ correspond
to the spatial infinities $x=\pm \infty$
in the original coordinates. By identifying these
two points, we can embed the rectangular image of $\IR^{1,1}$ 
in a cylinder $\IR \times S^1$ as shown in figure \ref{F2}.
It was proven by L\"uscher and Mack  \cite{Luscher:1974ez} that
correlation functions of
a conformal field theory (CFT) on $\IR^{1,1}$ can be analytically
continued to the entire cylinder. 

As we saw in section \ref{cft}, 
the global conformal symmetry of $\IR^{1,1}$ is
$SO(2,2)$, which is
generated by the 6 conformal Killing vectors
$\partial_\pm, u_\pm \partial_\pm, u_\pm^2 \partial_\pm$.
The translations 
along the cylinder $\IR \times S^1$ are expressed as their linear
combinations
\beq
 \frac{\partial}{\partial \tau} \pm
  \frac{\partial}{\partial \theta}=   
\frac{\partial}{\partial \tilde{u}_\pm} = 
 (1 + u_\pm^2) \frac{\partial}{\partial u_\pm}.
\label{cylindertranslation}
\eeq
In the standard form of $SO(2,2)$ generators, $J_{ab}$,
given in section \ref{cft}, they 
correspond to $J_{03}$ and $J_{12}$, 
and generate the maximally compact subgroup 
$SO(2) \times SO(2)$ of $SO(2,2)$. 


\bigskip
\noindent $\circ$ $\IR^{1,p}$ with $p \geq 2$

\medskip
It is straightforward to extend
the above analysis to higher dimensional Minkowski
space:
\beq
  ds^2 = -dt^2 + dr^2 + r^2 d\Omega_{p-1}^2,
\label{flat3}
\eeq
where $d\Omega_{p-1}$ is the line element on the unit
sphere $S^{p-1}$. A series of coordinate changes transforms
this as
\ber
  ds^2 & = & - du_+ du_- + \frac{1}{4}(u_+-u_-)^2 d\Omega_{p-1}^2
~,~~~~~~~~ (u_\pm = t \pm r) \nonumber \\
  & = & \frac{1}{\cos^2\tilde{u}_+\cos^2\tilde{u}_-}
                  \left(- d\tilde{u}_+d\tilde{u}_-
   + \frac{1}{4} \sin^2(\tilde{u}_+ - \tilde{u}_-) d\Omega_{p-1}^2 \right)
           ~,~~~~~~~(u_\pm = \tan \tilde{u}_\pm) \nonumber \\
     & = & \frac{1}{4\cos^2\tilde{u}_+\cos^2\tilde{u}_-}
            (-d\tau^2 + d\theta^2 + 
            \sin^2 \theta d \Omega_{p-1}^2), ~~~~~~~~
(\tilde{u}_\pm = (\tau \pm \theta)/2).
\label{flat4}
\eer

\begin{figure}[htb]
\begin{center}
\epsfxsize=2.0in\leavevmode\epsfbox{hfig3.eps}
\end{center}
\caption{The conformal transformation maps the 
$(t,r)$ half plane into a triangular region in
the $(\tau,\theta)$ plane. }
\label{F3}
\end{figure} 

As shown in figure \ref{F3}, 
the $(t,r)$ half-plane (for a fixed point on $S^{p-1}$)
is mapped into a triangular region in the $(\tau,\theta)$ plane.
The conformally scaled metric 
\beq
ds'^2 = -d\tau^2 + d\theta^2 + \sin^2 \theta d \Omega_{p-1}^2
\label{flat5}
\eeq 
can be analytically continued outside
of the triangle, and the maximally extended space with
\beq
   0 \leq \theta \leq \pi, ~~-\infty < \tau < + \infty,
\eeq
has the
geometry of $\IR \times S^p$ (Einstein static universe),
where $\theta=0$ and $\pi$ corresponds
to the north and south poles of $S^p$. This is a natural
generalization of the conformal embedding of $\IR^{1,1}$ into
$\IR \times S^1$ that we saw in the $p=1$ case.  
  
Since
\beq
  \frac{\partial}{\partial \tau} = \frac{1}{2} (1 + u_+^2) 
\frac{\partial}{\partial u_+}
      + \frac{1}{2}(1+u_-^2) \frac{\partial}{\partial u_-},
\eeq
the generator $H$ of the global time translation on $\IR \times S^p$
is identified with the linear combination
\beq
  H = \frac{1}{2}(P_0 + K_0) = J_{0,p+2},
\label{globaltime}
\eeq
where $P_0$ and $K_0$ are translation and special conformal
generators,
\beq
  P_0: {1 \over 2}\left( \frac{\partial}{\partial u_+}
 + \frac{\partial}{\partial u_-} \right), ~~
  K_0:  {1 \over 2}\left( u_+^2 \frac{\partial}{\partial u_+}
 + u_-^2 \frac{\partial}{\partial u_-} \right)
\eeq
on $\IR^{1,p}$ defined in section \ref{cft}. 
The generator $H = J_{0,p+2}$ corresponds to the $SO(2)$ part 
of the maximally compact subgroup $SO(2) \times SO(p+1)$ 
of $SO(2,p+1)$. Thus the subgroup $SO(2) \times
SO(p+1)$ (or to be precise its universal cover)
of the conformal group $SO(2,p+1)$ can be identified
with  the isometry of the Einstein static universe $\IR \times S^p$.
The existence of the generator $H$ also guarantees
that a correlation function of a CFT on $\IR^{1,p}$ can
be analytically extended to the entire Einstein static universe
$\IR \times S^p$. 


\begin{figure}[htb]
\begin{center}
\epsfxsize=2.5in\leavevmode\epsfbox{hfig35.eps}
\end{center}
\caption{\adsp\ is realized as a hyperboloid in $\IR^{2,p+1}$.
The hyperboloid has closed timelike curves along the $\tau$
direction. To obtain a causal space, we need to unwrap the
circle to obtain a simply connected space.}
\label{F3.5}
\end{figure} 


%\bigskip
%\noindent
\subsubsection{Anti-de Sitter Space}

%\medskip

The $(p+2)$-dimensional anti-de Sitter space (\adsp)
can be represented as the hyperboloid 
\beq
   X_{0}^2 + X_{p+2}^2 - \sum_{i=1}^{p+1} X_i^2  = R^2, 
\label{hyperboloid}
\eeq 
in the flat $(p+3)$-dimensional space with metric
\beq
  ds^2 = - dX_{0}^2 - dX_{p+2}^2 + \sum_{i=1}^{p+1} dX_i^2.
\label{metric1}
\eeq 
By construction, the space has the isometry $SO(2,p+1)$,
and it is homogeneous and isotropic. 

Equation (\ref{hyperboloid}) can be solved by setting
\ber
   X_{0} &=& R \cosh \rho\ \cos \tau, 
 ~~~~~ X_{p+2} = R \cosh \rho\ \sin \tau, 
  \nonumber \\
   X_i &=& R \sinh \rho\ \Omega_{i} ~~~(i=1,\cdots,p+1; 
 \sum_i \Omega_i^2 = 1).
\label{globalcoord}
\eer
Substituting this into
(\ref{metric1}), we obtain the metric on \adsp\ as
\beq
  ds^2 = R^2 ( -\cosh^2\rho\ d\tau^2 + d\rho^2 + \sinh^2\rho\
                    d\Omega^2 ).
\label{metric2}
\eeq
By taking $0 \leq \rho$ and $0 \leq \tau < 2 \pi$
the solution (\ref{globalcoord}) covers the entire hyperboloid
once. Therefore, $(\tau,\rho, \Omega_i)$ are called the global
coordinates of \ads.
Since the metric behaves near $\rho = 0$ as
$ds^2 \simeq R^2 ( -d\tau^2 + d \rho^2 + \rho^2\ d\Omega^2)$,
the hyperboloid has the topology of $S^1 \times \IR^{p+1}$, with
$S^1$ representing closed timelike curves in the $\tau$ direction.
To obtain a causal spacetime, we can simply unwrap the circle $S^1$
(i.e. take $-\infty < \tau < \infty$ with no identifications)
and obtain the universal covering of the hyperboloid without closed
timelike curves. In this paper, when we refer to \adsp ,
we only consider this universal covering space. 

The isometry group $SO(2,p+1)$ of \adsp\ has the maximal compact subgroup
$SO(2) \times SO(p+1)$.  
From the above construction, it is clear that the $SO(2)$ part 
represents
the constant translation in the $\tau$ direction, and
the $SO(p+1)$ gives rotations of $S^p$.


\begin{figure}[htb]
\begin{center}
\epsfxsize=3in\leavevmode\epsfbox{hfig4.eps}
\end{center}
\caption{\ads$_3$ can be conformally mapped into one half
of the Einstein static universe $\IR \times S^2$.}
\label{F4}
\end{figure} 

To study the causal structure of \adsp, it is convenient to introduce
a coordinate $\theta$ related to $\rho$ by $\tan \theta = \sinh \rho$
($0 \leq \theta < \pi/2$). The metric (\ref{metric2}) then takes
the form
\beq
   ds^2 = \frac{R^2}{\cos^2 \theta}
        ( - d\tau^2 + d\theta^2 + \sin^2 \theta\ d\Omega^2).
\label{metric3}
\eeq
The causal structure of the spacetime does not change by a conformal
rescaling on the metric. Multiplying the metric by 
$R^{-2} \cos^2 \theta$, it
becomes 
\beq
 ds'^2 =  - d\tau^2 + d\theta^2 + \sin^2 \theta\ d\Omega^2.
\label{metric4}
\eeq
This is the metric of the Einstein static universe, which
also appeared, with the dimension lower by one, in
the conformal compactification of $\IR^{1,p}$ (\ref{flat5}). 
This time, however,  
the coordinate $\theta$ takes values in $0 \leq \theta < \pi/2$, rather
than $0 \leq \theta < \pi$ in (\ref{flat5}). Namely, \adsp\ can be 
conformally mapped
into {\it one half} of the Einstein static universe; the spacelike
hypersurface of constant $\tau$ is a $(p+1)$-dimensional
hemisphere.  The equator at $\theta = \pi/2$ is a boundary of the space
with the topology of $S^p$, as shown in figure \ref{F4} in the case of
$p=1$. (In the case of $AdS_2$, the coordinate $\theta$ ranges
$-\pi/2 \leq \theta \leq \pi/2$ since $S^0$ consists of two points.)
As in the case of the flat space discussed earlier, the
conformal compactification is a convenient way to describe the
asymptotic regions of \ads . In general, if a spacetime can
be conformally compactified into a region which has the same
boundary structure as one half of the Einstein static universe,
the spacetime is called {\it asymptotically AdS}. 

Since the boundary extends in the timelike
direction labeled by $\tau$, 
we need to specify a boundary condition on
the $\IR \times S^p$ at $\theta = \pi/2$ in order
to make the Cauchy problem well-posed on \ads\ \cite{Avis:1978yn}. 
It turns out that the boundary of \adsp , or to be precise the
boundary of the conformally compactified \adsp , is identical
to the conformal compactification of the $(p+1)$-dimensional
Minkowski space. This fact plays an essential role
in the \adsp/CFT$_{p+1}$ correspondence.

\begin{figure}[htb]
\begin{center}
\epsfxsize=3in\leavevmode\epsfbox{hfig5.eps}
\end{center}
\caption{\ads$_2$ can be conformally mapped into $\IR \times [-\pi/2,\pi/2]$.
The $(u,t)$ coordinates cover the triangular region.}
\label{F5}
\end{figure} 

In addition to the global parametrization (\ref{globalcoord})
of \ads, there is another set of coordinates $(u,t,\vec{x})$ ($0 < u,
\vec{x} \in \IR^p$) which will be useful
later. It is defined by
\ber
   X_{0} &=& \frac{1}{2u}\left( 1 + u^2 (R^2 + \vec{x}^{~2} - t^2)
   \right), ~~~~~ 
   X_{p+2} = Rut, \nonumber \\
  X^{i} &=& Ru x^i~~~(i=1,\cdots,p), \nonumber \\
   X^{p+1} &=& \frac{1}{2u}\left( 1 - u^2(R^2 - \vec{x}^{~2} +
     t^2)    \right)  .
\label{poincarecoord}
\eer
These coordinates cover one half of the hyperboloid
(\ref{hyperboloid}), as shown in figure \ref{F5} in the case
of $p=0$. Substituting this into (\ref{metric1}), we obtain
another form of the \adsp\ metric
\beq
   ds^2 = R^2 \left( \frac{du^2}{u^2} + u^2 ( -dt^2 + d \vec{x}^{~2})
   \right).
\label{metric5}
\eeq
The coordinates $(u, t, \vec{x})$ are called the Poincar\'e
coordinates. In this form of the metric, the subgroups 
$ISO(1,p)$ and $SO(1,1)$ of the $SO(2,p+1)$ isometry are manifest,
where $ISO(1,p)$ is the Poincar\'e transformation on $(t, \vec{x})$
and $SO(1,1)$ is 
\beq
    (t, \vec{x}, u) \rightarrow (c t, c \vec{x}, c^{-1} u), ~~~
     c>0.
\eeq
In the \ads/CFT correspondence,
this is identified with the dilatation $D$ in the conformal symmetry
group of $\IR^{1,p}$. 



\begin{figure}[htb]
\begin{center}
\epsfxsize=1.8in\leavevmode\epsfbox{hfig6.eps}
\end{center}
\caption{The timelike Killing vector $\partial_t$ is depicted
in the $AdS_2$ case. The vector $\partial_t$
becomes a null vector at $u=0$.}
\label{F6}
\end{figure} 

It is useful to compare the two expressions, (\ref{metric2}) and
(\ref{metric5}), for the metric
of \adsp. In (\ref{metric2}), the norm of the timelike 
Killing vector $\partial_\tau$ is everywhere non-zero. In particular,
it has a constant norm in the conformally rescaled metric
(\ref{metric3}). For this reason, $\tau$ is called the global time
coordinate of \ads. On the other hand, the timelike Killing vector
$\partial_t$ in (\ref{metric5}) becomes null at $u=0$ (Killing
horizon), as depicted in figure \ref{F6} in the $AdS_2$ case. 

%\bigskip
%\noindent
\subsubsection{Euclidean Rotation}

%\medskip

Since \adsp\ has the global time coordinate $\tau$ and
the metric (\ref{metric2}) is static with respect to $\tau$,
quantum field theory on \adsp\ (with an appropriate boundary condition
at spatial infinity) allows the Wick rotation
in $\tau$, $e^{i\tau H} \rightarrow e^{- \tau_E H}$. From
(\ref{globalcoord}), one finds that
the Wick rotation $\tau \rightarrow \tau_E = -i\tau$ is expressed
in the original coordinates $(X_{0},  \vec{X}, X_{p+2})$ on the hyperboloid
as $X_{p+2} \rightarrow X_E = -i X_{p+2}$, and the space becomes
\ber
    & & X_{0}^2 - X_E^2 - \vec{X}^2 = R^2, \nonumber \\ 
    & & ds_E^2 = -dX_{0}^2 + dX_E^2 + d \vec{X}^2 .
\label{eads}
\eer    

We should point out that the same space is obtained by
rotating the time coordinate $t$ of the Poincar\'e coordinates
(\ref{poincarecoord}) as $t \rightarrow t_E = -i t$, even
though the Poincar\'e coordinates cover only a part of the entire
\ads\ (half of the hyperboloid).  This is analogous to 
the well-known fact in
flat Minkowski space that the Euclidean rotation of the time 
coordinate $t$ in the Rindler space $ds^2 = -r^2 dt^2 + dr^2$ gives 
the flat Euclidean plane $\IR^2$, even though the Rindler coordinates
$(t,r)$ cover only a $1/4$ of the entire Minkowski space $\IR^{1,1}$.

 In the coordinates
$(\rho, \tau_E, \vec{\Omega}_p)$ and $(u, t_E, \vec{x})$,
the Euclidean metric is expressed as
\ber
   ds_E^2 & = & R^2 \left( \cosh^2 \rho\ d\tau_E^2 + d\rho^2
              + \sinh^2 \rho \ d \Omega_p^2 \right) \nonumber \\
     & = & R^2 \left( \frac{du^2}{u^2} + u^2 (dt_E^2 + d\vec{x}^2 )
     \right).
\label{eads2}
\eer
In the following, we also use another, trivially equivalent,
form of the metric, obtained from the above by
setting $u =1/y$ in (\ref{eads2}), giving
\eqn{poincusual}{
ds^2 = R^2 \left( { dy^2 + dx_1^2 + \cdots + dx^2_{p+1} \over y^2 } \right).
}


\begin{figure}[htb]
\begin{center}
\epsfxsize=4.5in\leavevmode\epsfbox{hfig7.eps}
\end{center}
\caption{The Euclidean $AdS_2$ is the upper half plane with
the Poincar\'e metric. It can be mapped into a disk, where
the infinity of the upper half plane is mapped to a point
on the boundary of the disk.}
\label{F7}
\end{figure} 

The Euclidean \adsp\ is useful for various practical
computations in field theory. For theories on flat space,
it is well-known that 
correlation functions $\langle \phi_1 \cdots \phi_n \rangle$
of fields on the Euclidean space are related,
by the Wick rotation, to the $T$-ordered 
correlation functions $\langle 0 | T (\phi_1 \cdots \phi_n) | 
0 \rangle$ in the Minkowski space. 
The same is true in the anti-de Sitter space if the theory has
a positive definite Hamiltonian with respect to the global
time coordinate $\tau$. Green functions of free fields on 
\adsp\ have been computed in  
\cite{Burgess:1985ti,Inami:1985wu} using this method.

The Euclidean \adsp\ can be mapped into a $(p+2)$-dimensional
disk. In the coordinates $(u, t_E, \vec{x})$,
$u=\infty$ is the sphere $S^{p+1}$
at the boundary with one point removed. The full boundary
sphere is recovered by adding a point corresponding to 
$u=0$ (or equivalently $\vec{x} = \infty$). This
is shown in figure \ref{F7} in the case of \ads$_2$,
for which $z=t_E + i/u$ gives a complex coordinate on
the upper-half plane. By adding a point at infinity,
the upper-half plane is compactified into a disk. 
 In the Lorentzian case,
$u=0$ represented the Killing horizon giving the boundary
of the $(u,t,\vec{x})$ coordinates. Since the $u=0$ plane
is null in the Lorentzian case, it shrinks to a point
in the Euclidean case. 


\subsection{Particles and Fields in Anti-de Sitter Space}
\label{pfinads}

Massive particles, moving along geodesics,
 can never get to the boundary of $AdS$. 
On the other hand, since the Penrose diagram of $AdS$ is a cylinder,
 light rays
can go to the boundary and back in finite time, as observed by
an observer moving along a geodesic in AdS.
More precisely, the light ray will reflect if suitable boundary 
conditions are set for the fields propagating in $AdS$. 

Let us first consider the case of a scalar field propagating in 
\adsp. The 
field equation
\beq
  \left(\Delta - m^2 \right) \phi = 0
\label{kg}
\eeq
has stationary wave solutions
\beq
    \phi = e^{i\omega \tau} G(\theta) Y_l(\Omega_p),
\eeq
where $Y_l(\Omega_p)$ is a spherical harmonic, which 
is an eigenstate of the Laplacian on $S^p$ with
an eigenvalue $l(l+p-1)$, and  
$G(\theta)$ is given by the hypergeometric function
\beq
   G(\theta) = (\sin \theta)^l (\cos \theta)^{\lambda_\pm}
 ~_2F_1 \left(a,b,c; \sin \theta\right),
\label{hyper}
\eeq
with
\ber
   a & = & \frac{1}{2}(l+\lambda_\pm - \omega R), \nonumber \\
   b & = & \frac{1}{2}(l+\lambda_\pm + \omega R), \nonumber \\
   c & = & l + \frac{1}{2}(p+1),
\eer
and
\beq
   \lambda_\pm = \frac{1}{2} (p+1) \pm \frac{1}{2}\sqrt{(p+1)^2 + 4
     (m R)^2 }.
\label{branch}
\eeq
The energy-momentum tensor
\beq
   T_{\mu\nu} = 2 \partial_\mu \phi \partial_\nu \phi
     -g_{\mu\nu}\left( (\partial \phi)^2 + m^2  \phi^2 \right)
   + \beta(g_{\mu\nu} \Delta - D_\mu D_\nu + R_{\mu\nu}) \phi^2
\eeq
is conserved for any constant value of $\beta$. The value of
$\beta$ is determined by the coupling of the scalar curvature
to $\phi^2$, which on \ads\ has the same effect as  the mass term in the
wave equation (\ref{kg}). The choice of $\beta$ for
each scalar field depends on the theory we are considering.
The total energy $E$ of the scalar field fluctuation,
\beq
   E = \int d^{p+1} x \sqrt{-g} T^0_0,
\eeq
is conserved only if the energy-momentum flux through the 
boundary at $\theta=\pi/2$ vanishes,
\beq
\int_{S^p} d\Omega_p \sqrt{g} n_i T^i_{0|\theta = \pi/2} 
= 0.
\eeq
This requirement reduces to the boundary condition
\beq
   (\tan \theta)^p \left[ (1-2\beta) \partial_{\theta}
   + 2\beta \tan \theta \right] \phi^2 \rightarrow 0
 ~~~(\theta \rightarrow \pi/2).
\eeq
Going back to the stationary wave solution (\ref{hyper}), 
this is satisfied if and only if either $a$ or $b$ in (\ref{hyper})
is an integer. If we require the energy $\omega$
to be real, we find
\beq
  |\omega| R = \lambda_\pm + l + 2n,~~~~(n=0,1,2,\cdots).
\label{energyquanta}
\eeq  
This is possible only when $\lambda$ defined by (\ref{branch}) is
real. Consequently, the mass is bounded from below as
\beq
  - \frac{1}{4}(p+1)^2 \leq m^2 R^2  .
\label{posbound} \eeq
This is known as the Breitenlohner-Freedman bound 
\cite{Breitenlohner:1982bm,Breitenlohner:1982jf}. Note
that a negative (mass)$^2$ is allowed to a certain extent.
The Compton wavelength for these possible tachyons is comparable
to the curvature radius of $AdS$. 
If $m^2>-(p-1)(p+3)/4R^2$, we should choose 
$\lambda_+ $ in (\ref{energyquanta})
since this solution is normalizable while the solution with $\lambda_-$
is not.
If $m^2\leq -(p-1)(p+3)/4R^2$, both solutions are
normalizable and there are two different
quantizations of the scalar field on $AdS$ space.
% leading to two different quantum field theories. 
Which quantization to choose is often determined by
requiring symmetry. See 
\cite{Breitenlohner:1982jf,Hawking:1983m,Mezincescu:1984iu}
for discussions of boundary conditions in supersymmetric
theories. 
In general, all solutions to the wave equation form a single 
$SO(2,p+1)$ highest weight representation. The highest weight state
is the lowest energy solution \cite{Heidenreich:1982rz}. 
Since $SO(2,p+1)$ acts on $AdS$ as
isometries, the action of its generators 
on the solutions is given by 
first order differential operators. 





%%%%%




\subsection{Supersymmetry in Anti-de Sitter Space}
\label{susyinads}


The $SO(2,p+1)$ isometry group of \adsp\ has a supersymmetric 
generalization called an \ads\ supergroup. To understand the supersymmetry on 
\ads , it would be useful to start with the simple supergravity 
with a cosmological constant $\Lambda$. 
In four dimensions, for example, 
the action of the ${\cal N}=1$ theory is \cite{Townsend:1977qa}
\beq
  S = \int d^4 x \left(
      - \sqrt{g} ({\cal R} - 2\Lambda) + \frac{1}{2}
      \epsilon^{\mu\nu\rho\sigma} \bar{\psi}_\mu \gamma^5
   \gamma_\nu \tilde{D}_\rho \psi_\sigma \right),
\label{simpleaction}
\eeq
where 
\beq
 \tilde{D}_\mu = D_\mu + \frac{i}{2} \sqrt{\frac{\Lambda}{3}} \gamma_\mu
\eeq
and $D_\mu$ is the standard covariant derivative. The local
supersymmetry transformation rules for the vierbein $V_{a\mu}$
and the gravitino $\psi_\mu$ are
\ber
  \delta V_{a\mu} & = & -i \bar{\epsilon}(x) \gamma_a \psi_\mu,
  \nonumber \\
  \delta \psi_\mu & = & \tilde{D}_\mu \epsilon(x).
\label{localsusy}
\eer 

A global supersymmetry of a given supergravity background is
determined by requiring that the gravitino variation is annihilated,
$\delta \psi_\mu = 0$. The resulting condition on $\epsilon(x)$,
\beq
  \tilde{D}_\mu \epsilon = \left(D_\mu + \frac{i}{2} 
\sqrt{\frac{\Lambda}{3}} \gamma_\mu
  \right) \epsilon = 0,
\label{killing}
\eeq
is known as the Killing spinor equation. The integrability of
this equation requires
%\ber
% [\tilde{D}_\mu, \tilde{D}_\nu ] \epsilon
% &=& \frac{1}{2} (R_{\mu\nu\rho\sigma} \sigma^{\rho\sigma}
%        - \frac{2}{3} \Lambda \sigma_{\mu\nu}) \epsilon = 0 \nonumber \\
%& & ~~~\sigma_{\mu\nu} = \frac{1}{2} \gamma_{[\mu,} \gamma_{\nu]}.
%\label{integrable}
%\eer
\beq
 [\tilde{D}_\mu, \tilde{D}_\nu ] \epsilon
 = \frac{1}{2} ({\cal R}_{\mu\nu\rho\sigma} \sigma^{\rho\sigma}
        - \frac{2}{3} \Lambda \sigma_{\mu\nu}) \epsilon = 0,
\label{integrable}
\eeq
where
\beq
\sigma_{\mu\nu} = \frac{1}{2} \gamma_{[\mu,} \gamma_{\nu]}.
\eeq
Since \ads\ is maximally symmetric, the curvature obeys
\beq
   {\cal R}_{\mu\nu\rho\sigma} = {1 \over R^2} (g_{\mu\rho}g_{\nu\sigma}
        - g_{\mu\sigma}g_{\nu\rho}) ~,
\eeq
where $R$ is the size of the hyperboloid defined by (\ref{hyperboloid}).
Thus, if we choose the curvature of \ads\ to be $\Lambda = 3/R^2$ 
(this is necessary for \ads\ to be a classical solution of
(\ref{simpleaction})), the integrability condition (\ref{integrable})
is obeyed for any spinor $\epsilon$. Since the Killing spinor equation
(\ref{killing}) is a first order equation, this means that there
are as many solutions to the equation as the number of independent
components of the spinor. Namely, \ads\ preserves as many supersymmetries
as flat space. 

The existence of supersymmetry implies that, with an appropriate set
of boundary conditions, the supergravity theory on \ads\ is stable
with its energy bounded from below. The supergravity theories
on \ads\ typically contains scalar fields with negative (mass)$^2$.
However they all satisfy the bound (\ref{posbound}) 
\cite{Mezincescu:1984iu,Mezincescu:1985ev}. 
The issue of the boundary condition and
supersymmetry in \ads\ was further studied in \cite{Hawking:1983m}.  
A non-perturbative 
proof of the stability of $AdS$ is given in \cite{Gibbons:1983aq},
 based on a generalization of 
Witten's proof \cite{Witten:1981mf} of the positive energy theorem
in flat space \cite{Schoen:1982re}. 


\subsection{Gauged Supergravities and Kaluza-Klein Compactifications}
\label{kkcatalogue}

Extended supersymmetries in \adsp\ with $p = 2,3,4,5$ are 
classified by Nahm \cite{Nahm:1978tg} (see also \cite{Kac:1977em}) as
\ber
   AdS_4 &:& ~~ OSp({\cal N} | 4), ~~ {\cal N} = 1,2,\cdots 
        \nonumber \\
   AdS_5 &:& ~~ SU(2,2|{\cal N}/2), ~~ {\cal N}=2,4,6,8  \nonumber \\
   AdS_6 &:& ~~ F(4) \nonumber \\
   AdS_7 &:& ~~OSp(6,2 |{\cal N}), ~~ {\cal N}=2,4 .
\eer
For \adsp\ with $p>5$, there is no simple \ads\ supergroup.
These extended supersymmetries are realized as global symmetries of 
gauged supergravity on \adsp . The AdS/CFT correspondence identifies
them with the superconformal algebras discussed in section \ref{superconfalg}.
Gauged supergravities are supergravity
theories with non-abelian gauge fields in the 
supermultiplet of the graviton. Typically the cosmological
constant is negative and \adsp\ is a natural background geometry.
Many of them are related to Kaluza-Klein compactification of
the supergravities in 10 and 11 dimensions. A complete catalogue of
gauged supergravities in dimensions $\leq 11$ is found in 
\cite{Salam:1989fm}. Here we list some of them.

\medskip
\noindent
$\circ$ $\underline{AdS_7}$

\smallskip
The gauged supergravity in 7 dimensions
has global supersymmetry $OSp(6,2|{\cal N})$. 
The maximally supersymmetric case of ${\cal N}=4$
constructed in \cite{Pernici:1984xx} 
contains a Yang-Mills field with a gauge group $Sp(2) \simeq SO(5)$. 
The field content of this theory can be derived from a
truncation of the spectrum of
the Kaluza-Klein compactification of the 11-dimensional
supergravity to 7 dimensions,
\eqn{seventofour}{
\IR^{11} \rightarrow AdS_7 \times S^4.}
%
The 11-dimensional supergravity has the Lagrangian
%
\eqn{elevensugra}{{\cal L} = \sqrt{g} \left( \frac{1}{4} R 
  - \frac{1}{48} F_{\mu\nu\rho\sigma} F^{\mu\nu\rho
\sigma} \right) + \frac{1}{72} A \wedge F \wedge F
+ {\rm fermions},}
%
where $A$ is a 3-form gauge field and $F = dA$.
It was pointed out by Freund and Rubin \cite{Freund:1980xh}
that there is a natural way  to ``compactify''
the theory to 4 or 7 dimensions. 
We have put the word ``compactify'' in quotes since we will
see that typically the size of the compact dimensions is comparable
to the radius of curvature of the non-compact dimensions. 
To compactify the theory to 7 dimensions, 
the ansatz of Freund and Rubin sets
the 4-form field strength $F$ to be proportional 
to the volume element on a 4-dimensional subspace $M_4$. 
The Einstein equation, which includes the contribution of $F$ to the
energy-momentum tensor,
implies a positive curvature on $M_4$ and a
constant negative curvature on the non-compact dimensions,
$i.e.$ they are $AdS_7$. 

The maximally symmetric case is obtained by considering
$M_4=S^4$. Since there is no cosmological constant
in 11 dimensions, the radius $R$ of $S^4$ is
proportional to the curvature radius of $AdS_7$. 
By the Kaluza-Klein mechanism, 
the $SO(5)$ isometry of $S^4$ becomes the
gauge symmetry in 7 dimensions. The spherical harmonics
on $S^4$ give an infinite tower of Kaluza-Klein particles 
on \ads$_7$. 
%It is natural to expect that the consistent
%truncation of the Kaluza-Klein spectrum gives rise
%to the ${\cal N}=4$ $SO(5)$ gauged supergravity on \ads$_7$. 
A truncation of this spectrum to include only the graviton
supermultiplet gives the spectrum of the $\cn=4$ $SO(5)$ gauged
supergravity on \ads$_7$. It has been believed 
that this is a consistent truncation of the full theory, and 
very recently it was shown in \cite{vN:99ct} that this
is indeed the case. In general, there
are subtleties in the consistent truncation procedure, 
which will be discussed
in more detail in the next subsection. There are also 
other ${\cal N}=4$ theories with non-compact gauge groups $SO(p,q)$ with
$p+q=5$ \cite{Pernici:1985zw}.

The seven dimensional ${\cal N}=2$ gauged supergravity with gauge
group $Sp(1) \simeq SU(2)$ was constructed in \cite{Townsend:1983kk}.
In this case, one can have also a matter theory with possibly another
gauge group $G$. It is not known whether a matter theory of arbitrary
$G$ with arbitrary coupling constant can be coupled to gauged
supergravity. The Kaluza-Klein compactification of 10-dimensional
${\cal N}=1$ supergravity, coupled to ${\cal N}=1$ super Yang-Mills,
on $S^3$ gives a particular example. In this case, ten dimensional
anomaly cancellation requires particular choices of $G$.

\medskip
\noindent
$\circ$ $\underline{AdS_6}$

The 6-dimensional anti-de Sitter supergroup $F(4)$ is
realized by the ${\cal N}=4$ gauged supergravity with
gauge group $SU(2)$.
It was predicted to exist in \cite{DeWitt:1982wm} and
constructed in \cite{Romans:1986tw}.
It was
conjectured in \cite{Ferrara:1998gv} to be related
to a compactification of the ten dimensional massive type IIA supergravity theory.
The relevant compactification
of the massive type IIA supergravity is constructed as
a fibration
of  \ads$_6$ over $S^4$ \cite{us:1999ma}. The form of the ten dimensional space is
called a 
{\it warped product}
\cite{vanNieuwenhuizen:1985ri} and it is the most general one that has the 
$AdS$
isometry group \cite{van:1983xx}.
The $SU(2)$ gauge group of  
the 6-dimensional ${\cal N}=4$ gauged supergravity is associated with an $SU(2)$ subgroup
of the  $SO(4)$ isometry group of the compact part of the ten dimensional
space.



\medskip
\noindent
$\circ$ $\underline{AdS_5}$

In 5 dimensions, there are ${\cal N} = 2,4,6$ and $8$ gauged
supergravities with supersymmetry $SU(2,2|{\cal N}/2)$.
The gauged ${\cal N}=8$ supergravity was constructed 
in \cite{Pernici:1985ju,Gunaydin:1986cu}. 
It has the gauge group $SU(4) \simeq SO(6)$ and
the global symmetry $E_6$. 
This theory can be
derived by a truncation of the
compactification of $10$-dimensional type IIB
supergravity on $S^5$ using the Freund-Rubin ansatz, i.e.
setting the self-dual 5-form field strength $F^{(5)}$ 
to be proportional to the volume form of $S^5$
\cite{Schwarz:1983qr,Gunaydin:1985fk,Kim:1985ez}. By the Einstein equation,
the strength of $F^{(5)}$ determines the radius of $S^5$
and the cosmological constant $R^{-2}$ of \ads$_5$.

This case is of particular interest; as we will see below,
the $AdS$/CFT correspondence claims that 
it is dual to the large $N$ (and large $g_{YM}^2 N$) limit of 
${\cal N}=4$ supersymmetric $SU(N)$ gauge theory in four dimensions. 
The complete Kaluza-Klein mass spectrum of
the IIB supergravity theory on $AdS_5 \times S^5$ was obtained in 
\cite{Gunaydin:1985fk,Kim:1985ez}.
One of the interesting features
of the Kaluza-Klein spectrum (in this case as well as in the other
cases discussed in this section) is that the frequency $\omega$ of
stationary wave solutions is quantized. For example, the masses of the
scalar fields in the Kaluza-Klein tower are all of the form
$(m R)^2  = \tilde{l}(\tilde{l} + 4)$, where $\tilde{l}$ is
an integer bounded from below. Substituting this into (\ref{branch})
with $p=3$, we obtain
\beq
 \lambda_\pm = 2 \pm |\tilde{l} + 2|.
\eeq
Therefore, the frequency $\omega$ given by (\ref{energyquanta})
takes values in integer multiples of $1/R$: 
\beq
  |\omega| R = 2 \pm | \tilde{l} + 2| + l + 2n, ~~~(n=0,1,2,\cdots).
\eeq
This means that all the scalar fields in the supergravity
multiplet are periodic in $\tau$ with the period $2\pi$, i.e.
the scalar fields are single-valued on the original hyperboloid
(\ref{hyperboloid}) before taking the universal covering. 
This applies to all other fields in the supermultiplet as well,
with the fermions obeying the Ramond boundary condition around
the timelike circle. 

The fact that the frequency $\omega$ is quantized has its origin
in supersymmetry. The supergravity particles in 10 dimensions
are BPS objects and preserve one half of the supersymmetry.
This property is preserved under the Kaluza-Klein
compactification on $S^5$. The notion of the BPS particles in the case
of \ads\ supergravity is clarified in \cite{Freedman:1984na} 
and it is shown, in the context of theories in
4 dimensions, that it leads to the
quantization of $\omega$. In the \ads/CFT correspondence, 
this is dual to the fact that chiral primary operators
do not have  anomalous dimensions. 

On the other hand, energy levels of other states, such as stringy
states or black holes, are not expected to be quantized as 
the supergravity modes are. Thus, the full string theory does not make
sense on the hyperboloid but only on its universal cover
without the closed timelike curve.

The ${\cal N}=4$ gauged supergravity with gauge group $SU(2) \times
U(1)$ was constructed in
\cite{Romans:1986ps}. Various ${\cal N}=2$ theories were
constructed in \cite{D'Auria:1982yi,Gunaydin:1984bi,Gunaydin:1984nt,
Gunaydin:1985ak}. 


\medskip
\noindent
$\circ$ $\underline{AdS_4}$

In four dimensions, some of the possible $AdS$ supergroups are
$OSp({\cal N}|4)$ with ${\cal N}=1,2,4$ and $8$. $\cn=8$ is the
maximal supergroup that corresponds to a supergravity theory.
The ${\cal N}=8$ gauged supergravity with $SO(8)$
gauge group was constructed in \cite{deWit:1982eq,
deWit:1982ig}. This theory (like the other theories discussed in
this section) has a highly
non-trivial potential for scalar fields, whose extrema
were analyzed in \cite{Warner:1983vz,Warner:1984du}. It was
shown in \cite{deWit:1985iy}  that the extremum with
${\cal N}=8$ supersymmetry corresponds to a truncation of the
compactification of  11-dimensional supergravity
on $AdS_4\times S^7$. Some of the other extrema can also be
identified with truncations of
compactifications of the 11-dimensional theory. 
For a review of the 4-dimensional compactifications
of 11-dimensional supergravity, see
\cite{Duff:1986hr}. 


\medskip
\noindent
$\circ$ $\underline{AdS_3}$

Nahm's classification does not include this case since
the isometry group $SO(2,2)$ of $AdS_3$ is not a simple group
but rather the direct product of two $SL(2,\IR)$ factors.
The supergravity theories associated with  the $AdS_3$
supergroups $OSp(p|2) \times OSp(q|2)$
were constructed in \cite{Achucarro:1989gm}
and studied more recently in \cite{Nishimura:1998ud}. They
can be regarded as the Chern-Simons gauge theories of gauge group
$OSp(p|2) \times OSp(q|2)$. Therefore, they are topological
field theories without local degrees of freedom. The case
of $p=q=3$ is obtained, for example, by a truncation of
the Kaluza-Klein
compactification of the 6-dimensional $\cn=(2,0)$ supergravity
on $S^3$. In addition to $OSp(p|2)$, several other
supersymmetric extensions of $SL(2,\IR)$ are known, such as:
\eqn{extensions}{
SU({\cal N}|1,1),~G(3),~F(4),~D(2,1,\alpha).}
%
Their representations are studied extensively in the context
of two-dimensional superconformal field theories. 



\subsection{Consistent Truncation of Kaluza-Klein Compactifications}
\label{ConsistentTruncation}

Despite the fact that the equations of motion for type~IIB
supergravity in
ten dimensions are known, it turns out to be difficult to extract any
simple form for the equations of motion of fluctuations around
its five-dimensional Kaluza-Klein compactification on $S^5$.
The
spectrum of this compactification
is known from the work of \cite{Kim:1985ez,Gunaydin:1985fk}.
It
is a general feature of compactifications involving anti-de Sitter
space
that the positively curved compact part has a radius of curvature on
the
same order as the negatively curved anti-de Sitter part.  As a result,
the
positive $\hbox{(mass)}^2$ of Kaluza-Klein modes is of the same order
as
the negative $\hbox{(mass)}^2$ of tachyonic modes.  Thus there is no
low-energy limit in which one can argue that all but finitely many
Kaluza-Klein harmonics decouple.  This was a traditional worry for all
compactifications of eleven-dimensional supergravity on squashed
seven-spheres.

However, fairly compelling evidence exists (\cite{deWit:1987iy} and
references therein) 
that the reduction of eleven-dimensional supergravity on $S^7$ can be
{\it
consistently truncated} to four-dimensional ${\cal N}=8$ gauged
supergravity.  This is an exact statement about the equations of
motion,
and does not rely in any way on taking a low-energy limit.  Put
simply, it
means that any solution of the truncated theory can be lifted to a
solution
of the untruncated theory.  Charged black hole metrics in anti-de
Sitter
space provide a non-trivial example of solutions that can be lifted to
the
higher-dimensional theory
\cite{Chamblin:1999tk,Cvetic:1999ne,Cvetic:1999xp}.  There is a belief
but
no proof that a similar truncation may be made from ten-dimensional
type~IIB supergravity on $S^5$ to five-dimensional ${\cal N}=8$
supergravity.  To illustrate how radical a truncation this is, we
indicate
in figure~\ref{figCssg} the five-dimensional scalars that are kept
(this is
a part of one of the figures in \cite{Kim:1985ez}).  Note that not all
of
them are $SO(6)$ singlets.  Indeed, the fields which are kept are
precisely the
superpartners of the massless graviton under the supergroup
$SU(2,2|4)$,
which includes $SO(6)$ as its R-symmetry group.

\begin{figure}
      \vskip0cm
   \centerline{\psfig{figure=figCssg.eps,width=2in}}
   \vskip0cm
 \caption{The low-lying scalar fields in the Kaluza-Klein reduction of
type~IIB supergravity on $S^5$.  The filled dots indicate fields which
are
kept in the truncation to gauged supergravity. We also indicate
schematically the ten-dimensional origin of the
scalars.}\label{figCssg}
  \end{figure}

The historical route to gauged supergravities was as an elaboration of
the
ungauged theories, and only after the fact were they argued to be
related
to the Kaluza-Klein reduction of higher dimensional theories on
positively
curved manifolds.  In ungauged $d=5$ $\cn=8$
supergravity, the scalars parametrize
the
coset $E_{6(6)}/USp(8)$ (following \cite{Cremmer:1980gs} we use here
$USp(8)$ to denote the unitary version of the symplectic group with a
four-dimensional Cartan subalgebra).  The spectrum of gauged
supergravity
is almost the same: the only difference is that twelve of the vector
fields
are dualized into anti-symmetric two-forms.  Schematically, we write
this
as
  \eqn{SpectrumSplit}{
\begin{picture}(150,70)(-15,-40)
\thicklines
\put(0,0){$g_{\mu\nu} \qquad \psi^a_\mu \qquad A^{ab}_\mu \qquad
  \chi^{abc} \qquad
 \phi^{abcd}$}
\put(-3,15){$1$}
\put(37,15){$8$}
\put(73,15){$27$}
\put(114,15){$48$}
\put(156,15){$42$}
\put(81,-5){\line(1,-1){10}}
\put(81,-5){\line(-1,-1){10}}
\put(58,-30){$A_{\mu\,IJ}$}
\put(91,-30){$B^{\,I\alpha}_{\mu\nu}$}
\put(53,-45){$15$}
\put(106,-45){$12$}
\end{picture}
  }
 Lower-case Roman indices are the eight-valued indices of the
 fundamental
of $USp(8)$.  Multiple $USp(8)$ indices in \SpectrumSplit\ are
antisymmetrized and the symplectic trace parts removed.  The upper-case
Roman indices $I$, $J$ are the six-valued indices of the vector
representation of $SO(6)$, while the index $\alpha$ indicates a
 doublet of
the $SL(2,\IR)$ which descends directly from the $SL(2,\IR)$
 global
symmetry of type~IIB supergravity.  These groups are embedded into
$E_{6(6)}$ via the chain
  \eqn{EmbedSO}{
   E_{6(6)} \supset SL(6,\IR) \times SL(2,\IR) \supset
    SO(6) \times SL(2,\IR) \ .
  }

The key step in formulating gauged supergravities is to introduce
minimal
gauge couplings into the Lagrangian for all fields which are charged
under
the subgroup of the global symmetry group that is to be gauged.  For
instance, if $X_I$ is a scalar field in the vector representation of
$SO(6)$, one makes the replacement
  \eqn{GaugeIt}{
   \partial_\mu X_I \to D_\mu X_I = \partial_\mu X_I - g A_{\mu\,IJ}
   X^J
  }
 everywhere in the ungauged action.  The gauge coupling $g$ has
 dimensions
of energy in five dimensions, and one can eventually show that $g=2/R$
where $R$ is the radius of the $S^5$ in the $AdS_5 \times S^5$
geometry.
The replacement \GaugeIt\ spoils supersymmetry, but it was shown in
\cite{Gunaydin:1986cu,Pernici:1985ju} that a supersymmetric Lagrangian
can
be recovered by adding terms at $O(g)$ and $O(g^2)$.  The full
Lagrangian
and the supersymmetry transformations can be found in these
references.  It
is a highly non-trivial claim that this action, with its beautiful
non-polynomial structure in the scalar fields, represents a consistent
truncation of the reduction of type~IIB supergravity on $S^5$.  This
is not
entirely implausible, in view of the fact that the $SO(6)$ isometry of
the
$S^5$ becomes the local gauge symmetry of the truncated theory.
Trivial
examples of consistent truncation include situations where one
restricts to
fields which are invariant under some subgroup of the gauge group.
For
instance, the part of ${\cal N}=8$ five-dimensional supergravity
invariant
under a particular $SU(2) \subset SO(6)$ is ${\cal N}=4$ gauged
supergravity coupled to two tensor multiplets \cite{Freedman:1999gp}.
A similar trunction to ${\cal N}=6$ supergravity was considered in
\cite{Ferrara:1998zt}.

The $O(g^2)$ term in the Lagrangian is particularly interesting: it is
a
potential $V$ for the scalars.  $V$ is an $SO(6) \times SL(2,\IR)$
invariant function on the coset manifold $E_{6(6)}/USp(8)$.  It
involves
all the $42$ scalars except the dilaton and the axion.  Roughly
speaking,
one can think of the $40$ remaining scalars as parametrizing a
restricted
class of deformations of the metric and 3-form fields on the
$S^5$, and of $V$ as measuring the
response of
type~IIB supergravity to these deformations.  If the scalars are
frozen to
an extremum of $V$, then the value of the potential sets the
cosmological constant in five dimensions.  The associated conformal
field theories were discussed in 
%theory on the boundary is regarded as a phase of the ${\cal N}=4$
%theory
\cite{Distler:1998gb,Girardello:1998pd,Khavaev:1998fb}.  The known
extrema can be classified by the subset of the $SO(6)$ global
R-symmetry group that is preserved.








