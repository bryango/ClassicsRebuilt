
 

%% This is an AMS-TeX file.
%% The command to compile it is: amstex <file>
%%
\input amstex
\documentstyle{amsppt}
\loadeusm
\magnification=1200
\pagewidth{6.5 true in}
\pageheight{8.9 true in}

\catcode`\@=11
\def\logo@{}
\catcode`\@=13

\NoRunningHeads

\font\boldtitlefont=cmb10 scaled\magstep1
\font\bigboldtitlefont=cmb10 scaled\magstep2

\def\spinact{s}
\def\spinel{\tau}

\def\dspace{\lineskip=2pt\baselineskip=18pt\lineskiplimit=0pt}
\def\SO{\text{\rm SO}} 
 \def\udot{\Dot{u}}
\def\vdot{\Dot{v}}
\def\fdot{\Dot{f}}
\def\fdotbar{\Bar{\fdot}}

\def\const{\text{\rm const}}
\def\im{\text{\rm Im}}

\def\plus{{\sssize +}}
\def\otimesop{\operatornamewithlimits{\otimes}\limits}
\def\Le{{\mathchoice{\,{\scriptstyle\le}\,}
  {\,{\scriptstyle\le}\,}
  {\,{\scriptscriptstyle\le}\,}{\,{\scriptscriptstyle\le}\,}}}
\def\Ge{{\mathchoice{\,{\scriptstyle\ge}\,}
  {\,{\scriptstyle\ge}\,}
  {\,{\scriptscriptstyle\ge}\,}{\,{\scriptscriptstyle\ge}\,}}}
\def\plusmin{{\mathchoice{\,{\scriptstyle\pm}\,}
  {\,{\scriptstyle\pm}\,}
  {\,{\scriptscriptstyle\pm}\,}{\,{\scriptscriptstyle\pm}\,}}}
\def\vrulesub#1{\hbox{\,\vrule height7pt depth5pt\,}_{#1}}

\def\fbar{\Bar{f}}
\def\gbar{\Bar{g}}
\def\ftil{\tilde{f}}
\def\htil{\tilde{h}}
\def\htilbar{\Bar{\tilde{h}}}
\def\psitil{\tilde{\psi}}
\def\etatil{\tilde{\eta}}

\def\plus{{\sssize +}}  
\def\oplusop{\operatornamewithlimits{\oplus}\limits}
\def\otimesop{\operatornamewithlimits{\otimes}\limits}
\def\Piop{\operatornamewithlimits{\Pi}\limits}
\def\w{{\mathchoice{\,{\scriptstyle\wedge}\,}
  {{\scriptstyle\wedge}}
  {{\scriptscriptstyle\wedge}}{{\scriptscriptstyle\wedge}}}}
\def\Le{{\mathchoice{\,{\scriptstyle\le}\,}
  {\,{\scriptstyle\le}\,}
  {\,{\scriptscriptstyle\le}\,}{\,{\scriptscriptstyle\le}\,}}}
\def\Ge{{\mathchoice{\,{\scriptstyle\ge}\,}
  {\,{\scriptstyle\ge}\,}
  {\,{\scriptscriptstyle\ge}\,}{\,{\scriptscriptstyle\ge}\,}}}
\def\vrulesub#1{\hbox{\,\vrule height7pt depth5pt\,}_{#1}}
\def\rightsubsetarrow#1{\subset\kern-6.50pt\lower2.85pt
     \hbox to #1pt{\rightarrowfill}}
\def\mapright#1{\smash{\mathop{\,\longrightarrow\,}\limits^{#1}}}
\def\arrowsim{\smash{\mathop{\to}\limits^{\lower1.5pt
  \hbox{$\scriptstyle\sim$}}}}

\def\eps{{\varepsilon}}
\def\kap{{\kappa}}

\def\pvec{\vec{p}}

\def\Ebar{\overline{E}}
\def\fbar{\overline{f}}
\def\gbar{\overline{g}}
\def\Sbar{\overline{S}}
\def\Vbar{\overline{V}}
\def\scrRbar{\overline{\scrR}}
\def\scrSbar{\overline{\scrS}}
\def\grSbar{\overline{\grS}}

\def\ftil{\widetilde{\varphi}}
\def\xtil{\widetilde{\chi}}
\def\Stil{\widetilde{S}}
\def\scrStil{\widetilde{\scrS}}
\def\scrTtil{\widetilde{\scrT}}
\def\rhotil{\tilde{\rho}}

%\def\Stab{\text{\rm Stab}} \def\det{\text{\rm det}}
\def\space{\text{\rm space}} \def\const{\text{\rm const}}
\def\Vbarspace{\overline{V}_{\space}}
\def\Vspace{V_{\space}} \def\Spec{\text{\rm Spec}}
\def\SO{\text{\rm SO}}  \def\tr{{\text{\rm tr}}} 
 \def\pr{{\text{\rm pr}}} 
\def\Spin{\text{\rm Spin}}
\def\Res{\text{\rm Res}}
\def\Aut{\text{\rm Aut}}
\def\Out{\text{\rm Aut}}
\def\End{\text{\rm End}}
\def\supp{\text{\rm supp}}
\def\QFT{\text{\rm QFT}} 
%\def\SU{\text{\rm SU}} \def\Lie{\text{\rm Lie}}
%\def\SL{\text{\rm SL}} \def\Sym{\text{\rm Sym}}

\def\dbC{{\Bbb C}} 
\def\dbR{{\Bbb R}}
\def\dbZ{{\Bbb Z}} 

\def\gr#1{{\fam\eufmfam\relax#1}}

%Euler Fraktur letters (German)
\def\grA{{\gr A}}	\def\gra{{\gr a}}
\def\grB{{\gr B}}	\def\grb{{\gr b}}
\def\grC{{\gr C}}	\def\grc{{\gr c}}
\def\grD{{\gr D}}	\def\grd{{\gr d}}
\def\grE{{\gr E}}	\def\gre{{\gr e}}
\def\grF{{\gr F}}	\def\grf{{\gr f}}
\def\grG{{\gr G}}	\def\grg{{\gr g}}
\def\grH{{\gr H}}	\def\grh{{\gr h}}
\def\grI{{\gr I}}	\def\gri{{\gr i}}
\def\grJ{{\gr J}}	\def\grj{{\gr j}}
\def\grK{{\gr K}}	\def\grk{{\gr k}}
\def\grL{{\gr L}}	\def\grl{{\gr l}}
\def\grM{{\gr M}}	\def\grm{{\gr m}}
\def\grN{{\gr N}}	\def\grn{{\gr n}}
\def\grO{{\gr O}}	\def\gro{{\gr o}}
\def\grP{{\gr P}}	\def\grp{{\gr p}}
\def\grQ{{\gr Q}}	\def\grq{{\gr q}}
\def\grR{{\gr R}}	\def\grr{{\gr r}}
\def\grS{{\gr S}}	\def\grs{{\gr s}}
\def\grT{{\gr T}}	\def\grt{{\gr t}}
\def\grU{{\gr U}}	\def\gru{{\gr u}}
\def\grV{{\gr V}}	\def\grv{{\gr v}}
\def\grW{{\gr W}}	\def\grw{{\gr w}}
\def\grX{{\gr X}}	\def\grx{{\gr x}}
\def\grY{{\gr Y}}	\def\gry{{\gr y}}
\def\grZ{{\gr Z}}	\def\grz{{\gr z}}


\def\scr#1{{\fam\eusmfam\relax#1}}

\def\scrA{{\scr A}}   \def\scrB{{\scr B}}
\def\scrC{{\scr C}}   \def\scrD{{\scr D}}
\def\scrE{{\scr E}}   \def\scrF{{\scr F}}
\def\scrG{{\scr G}}   \def\scrH{{\scr H}}
\def\scrI{{\scr I}}   \def\scrJ{{\scr J}}
\def\scrK{{\scr K}}   \def\scrL{{\scr L}}
\def\scrM{{\scr M}}   \def\scrN{{\scr N}}
\def\scrO{{\scr O}}   \def\scrP{{\scr P}}
\def\scrQ{{\scr Q}}   \def\scrR{{\scr R}}
\def\scrS{{\scr S}}   \def\scrT{{\scr T}}
\def\scrU{{\scr U}}   \def\scrV{{\scr V}}
\def\scrW{{\scr W}}   \def\scrX{{\scr X}}
\def\scrY{{\scr Y}}   \def\scrZ{{\scr Z}}

\def\gr#1{{\fam\eufmfam\relax#1}}

%Euler Fraktur letters (German)
\def\grg{{\frak g}}
\def\grp{{\frak p}}


\NoBlackBoxes
\document

\centerline{\boldtitlefont Lecture 4.}

\medskip

\centerline{\boldtitlefont Scattering theory.}

\medskip

\centerline{David Kazhdan}


\dspace
%\bigskip\bigskip

\smallskip
\subhead 4.1
Introduction
\endsubhead

This important branch of functional analysis is not so
popular as it should be; one of the reasons is that the
phenomena studied in scattering theory has no
finite-dimensional analog.

The situation is roughly as follows.

Let $A_0$ be a self-adjoint operator in a Hilbert space $H$
(the function space).
Let $\lambda$ be a point in the {\it continuous} spectrum of
$A_0$ and let $f_\lambda^0$ be the
corresponding ``eigenfunction'' (quotation marks are put to
remind that $f_\lambda^0$ lies not in the Hilbert space $H$,
but rather in an appropriate larger function space, i.e.
$f_\lambda$ is a linear functional on a dense subspace in
$H$.)

We consider a ``small'' perturbation $A=A_0+B$ of our
self-adjoint operator $A_0$, where $B$ is a self-adjoint operator such that
$(\sqrt{1+A_0^2})^{-1} B$ is compact,
 and we look for a $\lambda$-eigenfunction $f_\lambda$ of the
operator $A$.

\remark{Remark} This problem makes sense only if $\lambda$ lies in
the continuous spectrum of $A_0$; in particular it never
makes sense in finite dimensions.
Indeed, the spectrum of an operator moves as we perturb
the operator; so an isolated point of the spectrum in general does not
lie in the spectrum of  perturbed operator.
\endremark

We look for an eigenfunction of the form $f_\lambda=f_\lambda^0+g$.
We have:
$$
\align
Af_\lambda &=\lambda f_\lambda \Leftrightarrow
(A-\lambda)g=Bf_\lambda^0\quad\text{so formally}\\
g &=(A-\lambda)^{-1}B(f_\lambda^0)\,\,.
\endalign
$$
Of course the last equality makes no sense since
$(A-\lambda)$ is not invertible. Nevertheless in
 some cases one can show existence of the limits $g_\pm =
\lim\limits _{\varepsilon \to \pm 0} (A-\lambda\pm i\varepsilon
)^{-1}B(f_\lambda^0)$. Then we put $f_\pm^\lambda = f_\lambda ^0 +g_\pm$.

\medskip

Let us consider a concrete example of the above situation.

In this example one can actually show existence of the limits 
$\lim\limits _{\varepsilon \to \pm 0} (A-\lambda\pm i\varepsilon
)^{-1}B(f_\lambda^0)$. 
 We will present another way to obtain the two eigenfunctions  $f^\lambda_\pm$.
We leave it as an exercise to the reader to check  that  the two procedures
lead to the same answer. 

\medskip

We take $H=L^2(E)$, where $E$ is a $d$-dimensional Euclidean
space.
Let $A_0=-\Delta+m^2$, and let $B$ be multiplication by a
function $b\in C_c^\infty(E)$.



\proclaim{Claim}
For $f\in C_c^\infty(E)$ we have: 
$$
\Vert Be^{itA_0}f\Vert\sim |t|^{-d/2}\qquad\text{for}\qquad
t\to\pm\infty\,\,
$$
\endproclaim

\proclaim{Corollary}
For any $f\in C_c^\infty(E)$ the $L^2$-limits
$\lim\limits_{t\to\pm\infty}e^{itA}e^{itA_0}f$ exist  provided $d>2$.
\endproclaim

\demo{Proof of Corollary}
Denote $f^t=e^{itA}e^{-itA_0}f$.
Then we have:
$$
\align
\left\Vert \tfrac{\partial f^t}{\partial t}\vrulesub{t_0}
\right\Vert &=\Vert iAe^{it_0A}e^{-it_0A_0}f-ie^{it_0A}A_0
e^{-it_0A_0}f\Vert\\
&=\Vert e^{it_0A}Be^{-it_0A_0}f\Vert=\Vert
Be^{-t_0A_0}f\Vert\,\,.
\endalign
$$
So by the previous claim $f_\plus=\int\limits_0^\infty
\left(\tfrac{\partial f^t}{\partial t}\right)dt$ converges
if $d>2$.
The proof for $f_-$ is analogous.
\enddemo

Obviously: \ $\Vert f_\pm\Vert_{L^2}=\Vert f\Vert_{L^2}$,
hence there exist well-defined operators $W_\pm\colon\,
H\to H$, such that $W_\pm(f)=f_\pm$.
Moreover, we see that:
$$
\Vert(AW_\pm-W_\pm A_0)(f)\Vert=\lim\limits_{t\to\pm\infty}
\Vert e^{itA}Be^{-itA_0}(f)\Vert=0\,\,,
$$
hence $AW_\pm=W_\pm A_0$.

If the image of $W_\plus$ is equal to the image of $W_-$, we
define the {\it $S$-matrix} by: \ $S=W_\plus^*W^-$.
Thus $S\in\End(H)$, and $S$ commutes with $A_0$.


\subhead{4.2} {System of $n$ particles (potential scattering)}\endsubhead
This is a generalization of the last problem.
We take $H=L^2(E^n)$, $A=-\sum\limits_{i=1}^n\Delta_i+\sum
B_{ij}$, where $B_{ij}$ is multiplication by a function
$b(v_i-v_j)$, for $b\in C_c^\infty(E)$, and $\Delta_i$ is
the Laplace operator along the $i^{\text{th}}$ copy of $E$.

Let $\pi$ be a partition  of $\{1,\dotsc,n\}$ in a union of
$k$ disjoint sets $\pi_1,\dotsc,\pi_k$ (``clusters'').

Consider the decomposition $E^{\pi_\ell}\simeq E\times
\Ebar_{\pi_\ell}$, where
$\Ebar_{\pi_\ell}=\{(v_{i_1},\dotsc,v_{i_{\vert\pi_\ell\vert}})\in
E^{\pi_\ell}\vert \sum v_i=0\}$, and the isomorphism is given by
$(v_{i_1},\dotsc,v_{i_{\vert\pi_\ell\vert}})\to
(w_\ell=\frac{1}{\vert\pi_\ell\vert}\sum v_i$;
$(v_{i_1}-w_\ell,v_{i_2}-w_\ell,\dotsc
v_{i_{\vert\pi_\ell\vert}}-w_\ell))$.

We have obvious decompositions $E=\Piop_{\ell=1}^k
E^{\pi_\ell}$, and the corresponding decomposition of the
function space:
$$
H=L^2(E^n)=\otimesop_{\ell=1}^k L^2(E^{\pi_\ell})=
\otimesop_{\ell=1}^k \left(L^2(E)\otimes
L^2(\Ebar_{\pi_\ell})\right)
$$

The operator $A_0$ obviously decomposes as: \
$A_0=\sum\limits_{\ell=1}^k A_\ell$, where $A_\ell$ acts
along $E^{\pi_\ell}$; and furthermore
$A_\ell=A'_\ell+A''_\ell$, where $A'_\ell$ is minus the
Laplace operator acting along $E$, and $A''_\ell$ acts along
$\Ebar_{\pi_\ell}$.
Let $H_\ell=\oplus H_\ell^\lambda\subset
L^2(\Ebar_{\pi_\ell})$ be the sum of  eigenspaces of
$A''_\ell$ where $\lambda$ runs over the discrete spectrum
of $A''_\ell$.
Put $H_0(\pi)=\otimesop_{\ell=1}^k(L^2(E)\otimes
H_\ell)\subset L^2(E^n)=H$.
Let $J_\pi\colon\,H_0(\pi)\hookrightarrow H$ denote the
natural isometric inclusion; thus $J_\pi(H_0(\pi))$ is an
$A_0$-invariant subspace in $H$.
We denote $J=\oplus J_\pi\colon\, \oplusop_{\pi}H_0(\pi)\to
H$.

\proclaim{Theorem 1} 
\roster
\runinitem"{\rm a)}"
There exist strong limits $W_\pm=\lim\limits_{t\to\pm\infty}
e^{itA}Je^{-itA_0}$.

\item"{\rm b)}"
$W_\pm$ is an isometric imbedding and we have: \
$A\circ W_\pm=W_\pm\circ A_0$.

\item"{\rm c)}"
$W_\pm$ is an isomorphism.
\endroster
\endproclaim

We omit the proof.
Just remark that the proof of a) and b) is routine, while c)
is a difficult theorem, the full proof of which was obtained
only recently.

%\medskip
%\line{\hrulefill}

\smallskip\noindent


\subhead{4.3} {Haag-Ruelle theory}\endsubhead
 We now  apply the method of scattering
theory to QFT.


Claims and methods  of scattering theory  discussed below
 apply only to Minkowski, not to the Euclidean picture.


We will describe the following construction.

Let $(U,\scrH\supset\scrD\owns\Omega_i)$ be a scalar QFT.
Suppose that:

\roster
\item"a)"
$\scrH^V=\dbC\Omega$ (uniqueness of vacuum).

\item"b)"
There exists a $P$-invariant Hilbert subspace
$H\subset\scrH$ with a $P$-equivariant isomorphism
$H\simeq L^2(\scrO_m^\plus)$, $m>0$.

\item"c)" Mass spectrum of
${\scrH/H\oplus\dbC\Omega}$ does
not contain $0$ and $m$.

\item"d)"
The span of $\left<\phi(f)\Omega\right>$, $f\in\scrS$
is not contained in $H^\perp$.
\endroster

\noindent
Let $(U_0,\scrH_0\supset\scrD_0,\phi_0)$ be the free
field theory of mass $m$.
Then one can construct two $P$-equivariant isometric
imbeddings $W_{\plusmin}\colon\,\scrH^0\hookrightarrow\scrH$.

\remark{Remark}
We cannot hope that $W_\plusmin$ is an isomorphism, because
$\square\vrulesub{\scrH}$ can have discrete spectrum outside
of $\{0,\,m^2\}$ while for free theory it does not.
An ideal theorem would assert that a free theory
constructed on  discrete part of the spectral decomposition of 
$\square$ on $\scrH$ maps
isomorphically to $\scrH$ (compare with  potential scattering
described in the previous section).
Unfortunately there is
no known way to formulate a precise theorem of this sort. 
\endremark


\remark{Remark} The construction can be carried out under weaker
assumptions. Namely one can omit condition d), and replace condition 
c) by requirement that the theory has positive mass, and
 $m^2$ is an  isolated point of the discrete spectrum of $\square$.

However the assumption that $0$ is an isolated point of
$\Spec(U\vrulesub{V})$ is essential.
Application of scattering to a massless QFT confronts
much more serious difficulties than in the massive case.
\endremark

\remark{Remark}
The construction does not work for small $d$.
Consider the case $d=1$.
Then QFT reduces to quantum mechanics.
The spectrum of Hamiltonian of the free theory
$\Spec\,H_0=m\dbZ_{\plus}$ is totally discrete, thus
scattering theory is not applicable.
One can prove a slightly weaker statement  in the case $d=2$.
Our argument will work for $d\Ge 4$.
\endremark



We now turn to construction of imbeddings $W_\pm$.
We try to imitate the construction of 
the previous section  (potential scattering).

Let $\Lambda\subset V^{\plus}$ be a $G$-invariant
neighborhood of $\scrO_m^{\plus}$, such that $\Lambda\cap
\Spec\,U\vrulesub{H^\perp}=\emptyset$.
For any function $f$ on $V$ such that
$\supp\,\scrF(f)\subset \Lambda$ and $t\in\dbR$ let
$f^t\in\scrS(V)$ be the function such that:
$$
\scrF(f^t)(p_0,\vec{p})=
e^{it\left(p_0-\sqrt{m^2-\vec{p}^2}\,\right)}\scrF(f)(p_0,
\vec{p}\,)\,.
$$
Note that $f^t=f$ provided
$\supp(\scrF(f))\subset\scrO_m^{\plus}$.

In the rest of the lecture we prove the following

\proclaim{Theorem 2}
\roster
\runinitem"a)"
If $d\Ge 4$ then for any $f_1,\dotsc,f_n\in\scrS(V)$ such
that $\supp(\scrF(f_i))\subset\Lambda$ the limits
$\lim_{t\to\plusmin\infty}
\phi(f_1^t)\ldots\phi(f_n^t)\Omega$ exist.

\item"b)"
There exists a unique isometric imbedding
$W_{\plusmin}\colon\,\scrH_0\hookrightarrow\scrH$, such that
for any $f_1,\dotsc,f_n\in\scrS(V)$ with
$\supp(\scrF(f))\subset\lambda$ we have:
$$
W_{\plusmin}(\phi_0(f_1)\ldots\phi_0(f_n)\Omega)=
\lim\limits_{t\to\plusmin\infty}\phi(f_1^t)\ldots
\phi(f_n^t)\Omega
$$
The imbeddings $W_\pm$ are $P$-equivariant.
\endroster
\endproclaim

\remark{Remark}
For a free field theory the vector
$\phi_0(f_1^t)\ldots\phi_0(f_n^t)\Omega$ does not
depend on $t$.
Indeed, the field operator $\phi_0(f)$ for a free
field theory depends on
$\phi(f)\vrulesub{\scrO_m^{\plus}}$ only, and
$\scrF(f^t)\vrulesub{\scrO_m^{\plus}}$ does not depend
on $t$.
\endremark

\remark{Remark}
Let $L$ be the space of functions satisfying the conditions of Lemma 1.
Fourier transform identifies $L$ with the space of
Schwartz functions on $\scrO_m^{\plus}$.
Let $\chi$ be a Schwartz function of one variable such
that $\scrF(\chi)$ is supported in a small neighborhood of
$0$.
For $f\in L$ let us put $f_\chi(x)=\chi (x_0)\cdot f(x)$
where $x=(x_0,\vec{x}\,)$.
Then $f_\chi\in\scrS(V)$, and the Fourier transform of $f_\chi$ is
concentrated in a neighborhood of $\scrO_m^{\plus}$ being  the
convolution of the function $\scrF (f)$ concentrated on $\scrO _m^+$
and a function concentrated in a small neighborhood of 0 on  the time-axis.


This way we can fix an initial imbedding $j$ of (a dense subspace of)
$\scrH_0$ to $\scrH$ such that $j(\phi_0(f_1)\cdots \phi_0(f_k)\Omega)=
\phi(f_{1\chi})\cdots \phi(f_{k \chi})\Omega$ for any functions
$f_1,..,f_k\in L$. Then one checks directly that:
$$\phi(f_{1\chi}^t)\cdots \phi(f_{k \chi}^t)\Omega = U\left(
(t,\vec{0}\,)\right) j 
(U_0\left( (-t,\vec{0}\,)\right)\phi_0(f_1)\cdots \phi_0(f_k) \Omega).$$
 
Thus the situation is quite analogous to the potential scattering
considered in the previous section.  
\endremark

\demo{Proof of the Theorem}
We start with two auxiliary estimates needed in the proof. 
The first is a standard fact from analysis.


\proclaim{Lemma 1 {\rm (locality for free fields)}}
Let $f\in C^\infty(V)$ be a solution of the Klein-Gordon
equation, i.e. $\supp\,\scrF(f)\subset\scrO_m^{\plus}$.
Assume that $f(0,\vec{x}\,)$ and $\frac{\partial
f}{\partial x_0}(x_0,\vec{x})\vrulesub{x_0=0}$ are Schwartz functions
on $\dbR^{d-1}$.
Then for some constant $C$ we have:

{\rm a)} $
\sup\limits_{\vec{x}\in\dbR^{d-1}}\vert
f(x_0,\vec{x}\,)\vert\Le C\,(1+\vert x_0\vert)^{-(d-1)/2}$

{\rm b)} $\int\limits _{\dbR^{d-1}}  |f(x_0,\vec{x})|d\vec{x}\Le
C(1+|x_0|)^{(d-1)/2} $

\endproclaim

We omit the proof; it is an exercise in stationary phase method (see
lecture 1, \S 1.6,  Claim 1b) for a similar argument). 

The second ingredient is the following estimate on truncated Wightman
functions (see lecture 3, \S 3.5 for the definition).



\proclaim{Proposition 1}  There exist a differential operator with constant
coefficients $D$,  
a $
C^0$ function of tempered growth $F$, and a polynomial function
$P$ on $V_n$  such that
 $\scrW_n^\tr = D(F)$ and the following estimate holds. 


 Let $[1..n]=I \cup J$ be a partition of $[1..n]$ into union of 2 disjoint
subsets; 
let $a_1,\dotsc,a_n\in\dbR^{d-1}\subset V$ be
such that  $a_i \not =
a_j$ for  $i\in I,\,j\in J$. Then for all $t_1,...,t_n \in \dbR \subset
V$ we have:
$$
|F(t_1+ a_1,\dotsc,t_n+ a_n)|
 <|P(t_1+ a_1,\dotsc,t_n+ a_n)|
  e^{-m  \min_{i\in I, j\in J}
({\sssize\sqrt{-(a_i-a_j)^2}}-|t_i-t_j|)}
.$$



\endproclaim

\remark{Remark} The particular case $n=2$ of the Proposition 
is implied by lecture 1, \S 1.6,  Claim 1b). Also
   part a) of the same Claim guarantees that at least $\scrW_n^\tr
 (v_1,..,v_i,v_{i+1}+\lambda a, ..,v_n+\lambda a) \to 0$ for any 
$v_1, \ldots, v_n$ provided $a^2<0$, which obviously follows from
 the Proposition.
\endremark

The statement we will actually need is the  following immediate

\proclaim{Corollary}
Let $i_t $ denote the embedding of $(\dbR^{d-1})^n$ to $V_n$,
 $\prod \vec{v}_i
\to \prod(t,\vec{v}_i)$, and let $\pi:(\dbR^{d-1})^n\to (\dbR^{d-1})^{n-1}=
(\dbR^{d-1})^n/\Delta$ be the projection to quotient modulo diagonal. 
For a Schwartz function $h$ on $V_n$ the restriction 
 $i_t^*(h*\scrW_n^{\tr})$ is a smooth function 
invariant under diagonal translations, hence 
$i_t^*(h*\scrW_n^{\tr})=\pi^*(F_t^h)$ for some function $F^t_h$ on $
 (\dbR^{d-1})^{n-1}$. Then 
$F^t_h\in L^1\left( (\dbR^{d-1})^{n-1}\right)$ for any $t$, and
 $\left| F_t^h \right|
 _{L^1}$ 
 is bounded by a constant independent of $t$. 

\endproclaim

\demo{Proof of the Proposition}
We start with an auxiliary result.
Recall from the first lecture that the Fourier transform
of Wightman functions has the property:
$\scrF(\scrW_n(p_1,\dotsc,p_n))=0$ if $p_1+\cdots+p_k\notin
\Vbar_\plus$ for some $k$.
One can  easily show:


\proclaim{Lemma 2}
Let $V_\plus(m)$ be $\{p\in V_\plus\vert p^2>m\}$.
Then we have $\scrF(\scrW_n^{\tr}(p_1,\dotsc,p_n))=0$
provided $p_1+\cdots+p_k\notin\overline{V_\plus(m)}$ for some $k$ where
$m$ is the mass of the theory.
Also $\scrF(\scrW_n^{\tr}(p_1,\dotsc,p_n))$ is concentrated
on the hyperplane $\sum\limits_{i=1}^n p_i=0$.
\endproclaim


\medskip

We are ready to prove  the Proposition.

Let us denote: $\scrW'(v_1,\dotsc,v_n)=\scrW^{\tr}(v_{i_1},\dotsc,
v_{i_k}; v_{j_1},v_{j_2},\dotsc,v_{j_{n-k}})$\break
and
$\scrW''(v_1,\dotsc,v_n)=\scrW^{\tr}(v_{j_1},\dotsc,v_{j_{n-k}};
v_{i_1},\dotsc,v_{i_k})$
 where $i_1<i_2<\ldots<i_k$ are the
elements of $I$, and $j_1<\ldots<j_{n-k}$ are the elements of
$J$.
From the Lemma it follows that
$$
\alignat2
&\scrF(\scrW')(p_1,\dotsc,p_n)=0 &\quad &\text{if}\quad
  p_{i_1}+\cdots+p_{i_k}\notin
  \overline{V_\plus(m)};\\
&\scrF(\scrW'')(p_1,\dotsc,p_n)=0 &&\text{if}\quad
-(p_{i_1}+\cdots+p_{i_k})\notin \overline{V_\plus(m)}
\endalignat
$$

Let $\xtil$ be a smooth function of one variable, such that
$\xtil(x)=1$ for $x>m$, $\xtil(x)=-1$ for $x<-m$, and its
Fourier transform has asymptotics $\ftil = \scrF(\xtil(t))\sim
e^{-m\vert t\vert}$ as $t\to\pm\infty$. [It is a standard
fact from calculus that such a function exists].

Let us put $\chi(p_1,\dotsc,p_n)=\xtil(p_{i_1}^{(0)}\plus\cdots
\plus p_{i_k}^{(0)})$, where $p_i=(p_i^{(0)},\pvec_i)$ for $p_i^{(0)}\in
\dbR,\, \pvec_i \in \dbR^{d-1}$ is the usual decomposition.
Then we see that $\chi\scrF(\scrW')=\scrF(\scrW')$,
$\chi\scrF(\scrW'')=-\scrF(\scrW'')$.
Let $\varphi $ be Fourier transform of $\chi$; so $\varphi$
is concentrated on a 1-dimensional subspace.
Then we have $\varphi*\scrW'=\scrW'$,
$\varphi*\scrW''=-\scrW''$, where $*$ stands for
convolution.

Since $\scrW_n$ is a tempered distribution it can be represented as
$\scrW_n= D(F)$ where $F$ is a continuous function of polynomial growth
and $D$ is a differential operator with constant coefficients. 
We can assume that $D$ is invariant under the action of symmetric group  
$\Sigma _{n+1}$. Then the functions
$F'=F(v_{i_1},..,v_{i_k};v_{j_1},..,v_{j_{n-k}} )$ and
$F''=F(v_{j_1},..,v_{j_{n-k}} ;v_{i_1},..,v_{i_k})$ satisfy: $D(F')=\scrW'$
and $D(F'')=\scrW''$. Then obviously: $\varphi*F'=F'$ and
$\varphi*F''=-F''$


The space-locality property implies that:
$F'(v_1,\dotsc,v_n)=F''(v_1,\dotsc,v_n)=F
(v_1,\dotsc,v_n)$ provided $(v_i-v_j)^2<0$ for $i\in I$,
$j\in J$.


So for $|(a_i-a_j)^2|>(t_i-t_j)^2$ we have:
$$
\align
&F(t_1 + a_1,\dotsc,t_n+a_n) =\tfrac12
(F'(t_1+ a_1,\dotsc,t_n+ a_n)
+F''(t_1+ a_1,\dotsc,t_n+ a_n))\\
&=\tfrac12\varphi*(F'-F'')=\tfrac12
  \int\limits_{-\infty}^\infty (F'-F'')
  ((t_1+ a_1)^t,(t_2+
a_2)^t,\dotsc,(t_n+a_n)^t)\ftil(t)dt
\endalign
$$
where we used the notation: $(t_i+ a_i)^t=t_i+
a_i-t$ if $i\in I$, and $(t_j+ a_i)^t=t_j+
a_j$ for $j\in J$.
The expression in the integral vanishes if
$t<\min (\sqrt{-(a_i-a_j)^2} -|t_i-t_j|)$.
So since $F$ is of polynomial growth 
we get the desired upper bound for the integral.
\enddemo



\enddemo


The crucial step of the proof is the following


\proclaim{Lemma 3}
$$\vert \scrW_n^{\tr}(g  _1^t,\dotsc ,g_n^t)\vert
\Le C (1+\vert t\vert)^{(1-d)(n-2)/2}$$
 for some $C$ and all
$t$, where $g_j^t=f_j^t$ or $g_j^t=\frac{\partial
f_j^t}{\partial t}$ or complex conjugate for each $j$.
\endproclaim

\demo{Proof of the Lemma}
Let us work in Fourier coordinates; let $pr$ be the projection $pr:V_n \to
(\dbR^{d-1})^n $. 

Of course $ \scrW_n^{\tr}(g  _1^t,\dotsc ,g_n^t)=
\int  \scrF(\scrW_n^{\tr})(-p) \scrF (g_1^t \times \cdots \times g  _n^t)
 =  \int pr_*(\scrF(W^\tr_n)(-p)\cdot \scrF (g_1^t \times \cdots \times g
_n^t))$.
Let us look more closely at the last integral. 

 For this purpose it is convenient to write $\scrF(f_i)$ (respectively
$\scrF(\fbar_i)$) in the form
$ \psitil_i \cdot \etatil_i $ where $\psitil_i$ is a Schwartz
function with support in $\Lambda$ (respectively in $-\Lambda$),
 and $\etatil_i$ is the pull-back of a 
Schwartz function under the  projection on $\dbR^{d-1}$.

Let $\eta _i$ be given by 
$\scrF(\eta_i)=\delta_{\scrO_m^+}\cdot \etatil_i$
(respectively  $\scrF(\eta_i)=\delta_{\scrO_m^-}\cdot \etatil_i$).

Using the projection formula we can rewrite:
 $$pr_*\left(\scrF(\scrW_n)(-p)\cdot \scrF (\prod g_i^t)\right) =
 pr_*[T_t\left(
\scrF(\scrW_n)(-p)\cdot \prod \htil_i \right)]\cdot 
pr_*[T_{-t}\left( \scrF(\prod \eta_i)\right)]$$

Here $T_t(f)(p)=\exp(it\cdot p_0)f(p)$ and $\htil_i=\psitil_i$ 
if $g_i^t=f_i^t$ or complex conjugate,
and $\htil_i=\psitil_i\cdot \pm i (p_0 - \sqrt{m^2-\vec{p}^2})$ if $g_i^t =
\frac{\partial f^t_i}{\partial t}$ or complex conjugate.

Let $h$ denote $\scrF(\prod \htil_i)$.  Then the integral we need to
estimate equals to
$$
\int\limits_{(\dbR^{d-1})^n}\kern-10pt  pr_*\!\!\left[T_t\left(
\scrF(\scrW_n)(-p)\cdot \scrF \left(\prod \htil_i\right)\right)\right]\cdot
pr_*\!\!\left[T_{t}\left(
\overline{\scrF\Bigl(\prod \eta_i}\Bigr)\right)\right] 
=\kern-10pt\int\limits_{(\dbR^{d-1})^n}\kern-12pt i_t^*(F*h)\cdot
i_{-t}^* \left(\prod \eta_i\right)
$$


 

By Corollary to the Proposition $i_t^*(\scrW_n*h)=\pi^*(F_t^h)$, for some
function $F_t^h \in L^1( (\dbR^{d-1})^{n-1})$, and $\left|
F^t_h\right|_{L^1} < C$ where $C$ does not depend on $t$.

 Thus we have:
$$\int\limits _{(\dbR^{d-1})^n} |pr^*i_t^*(\scrW_n*h)\cdot pr_1
i_{-t}^*(\eta _1)| \Le C
\cdot  \int\limits _{\dbR^{d-1}} | j_t^*(\eta _1)| \Le C \cdot
(1+|t|)^{(d-1)/2} $$
by Lemma 1b), where $j_t$ is the imbedding
$\dbR^{d-1}\hookrightarrow V$, $\vec{v}\to (t,\vec{v})$), and
$pr_i:(\dbR^{d-1})^n\to \dbR^{d-1}$ is the
projection to the i-th component.
 Hence
$$
\split
\int\limits_{(\dbR^{d-1})^n} i_t^*(\scrW_n*h)\cdot
&i_{-t}^* \left(\,\prod \eta_i\right) 
\Le C (1+|t|)^{(d-1)/2}\\
&\cdot\max\limits
_{(\dbR^{d-1})^n} \prod \limits _{k=2}^n pr_k^* j_{-t}^*(\eta _j) \Le C
(1+|t|)^{-(n-2)(d-1)/2} 
\endsplit
$$
where the last inequality follows from Lemma 1a).  Lemma 3 is proved.








\enddemo

We are now ready to prove the Theorem.
Denote $v^t=\phi(f_1^t)\ldots\phi(f_n^t)\Omega$.
To show that the limit $\lim\limits_{t\to\infty}v^t$
exists it suffices to check that the integral
$\int\limits_0^\infty\left\vert\frac{\partial
v^t}{\partial t}\right\vert\,dt$ converges.
We have:
$$
\split
\left(\tfrac{\partial v^t}{\partial
t}\right)^2=\biggl(\sum\limits_{i} &\phi(f_1^t)\ldots
  \phi(\fdot_i^t)\ldots\phi(f_n)\Omega,\,
\sum\limits_{j}\phi(f_1^t)\ldots\phi(\fdot_j^t)
\ldots\phi(f_n)\Omega\biggr)\\
&=\sum\limits_{i,j}\scrW_{2n}\left(\fbar_n^t,\fbar_{n-1}^t,
\dotsc,\fdotbar{^{t}_i},\dotsc,\fbar_1^t,f_1^t,\fdot_j^t,
\dotsc,f_n\right)
\endsplit
$$

Let us decompose $\scrW_{2n}$ as sum of products of
truncated Wightman functions, and consider the
corresponding decomposition of each summand in the last
sum.

Each factor containing $\scrW_1^{\tr}$  obviously
vanishes.
By the previous lemma each factor containing
$\scrW_i^{\tr}$ is bounded by $\const\cdot (1+\vert
t\vert)^{(i-2)(1-d)/2}$; in particular each factor is
bounded.

Let us look at a factor of the form
$\scrW_k^{\tr}(g_1^t,...,g_k^t)$ where at least one of
$g_i^t$ is a time derivative of $f_\ell^t$ (or complex
conjugate).
If $k=2$ then such a factor vanishes.
Indeed, if say $g_1^t=\overline{f_i^t}$ and
$g_2^t=\fdot_j^t$ then $\scrW_2^{\tr}(g_1^t,g_2^t)=
\left<\phi(f_i^t)\Omega,\phi(\fdot_j^t)\Omega\right>=0$
since $\phi(f_j^t)\Omega$ does not depend on $t$.
(Here we use that
$\supp\,\scrF(f_j)\cap\supp\,U\vrulesub{V}=\scrO_m^{\plus}$.)
The other cases are absolutely parallel.

We now see that each non-zero summand contains a factor of the
form \break
$\scrW_k^{\tr}(g_1^t,\dotsc,g_k^t)$ with $k\Ge 3$.
Furthermore, it either contains at least $2$ factors
$\scrW_k^{\tr}$ with $k\Ge 3$ or at least one factor
$\scrW_k$ with $k\Ge 4$, because the term has the form
$\scrW_{k_1}^{\tr}\ldots\scrW_{k_\ell}^{\tr}$ with
$k_1+\cdots+k_\ell=2n$ being even, and each $k_i\Ge 2$.
In any case it is bounded by $\const\cdot(1+\vert
t\vert)^{(1-d)/2\cdot\sum(k_i-2)}\le\const\cdot(1+\vert
t\vert)^{1-d}$.
Thus $\vert\vdot^t\vert\Le\const\cdot (1+\vert t\vert)^{(1-d)/2}$
and $\int\limits_{-\infty}^\infty \vdot^t\,dt$ converges
absolutely if $d\Ge 4$.

\demo{Proof of {\rm b)}}
Let us check that if
$v^t=\phi(f_1^t)\ldots\phi(f_k^t)\Omega$,
$w^t=\phi(g_1^t)\ldots\phi(g_\ell^t)\Omega$ for
$f_i$, $g_j$ as above then
$\lim\limits_{t\to\plusmin\infty}(v^t,w^t)=(\phi_0(f_1)\ldots
\phi_0(f_k)\Omega,
\phi_0(g_1)\ldots\phi_0(g_\ell)\Omega)$.
Let us again decompose
$(v^t,w^t)=\scrW_{\ell+k}(\fbar_k^t,\fbar_{k-1}^t,\dotsc,
\fbar_1^t,g_1^t,\dotsc,g_\ell^t)$ into sum of products of
truncated Wightman functions.
From the above arguments we see that every term
containing $\scrW_1^{\tr}$ vanishes, and every term
containing $\scrW_k^\tr$ for $k\Ge 3$ tends to $0$ as
$t\to\plusmin\infty$.
Besides,
$\scrW_2^{\tr}(\fbar,g)=(\phi(f)\Omega,\phi(g)\Omega)=
(\phi_0(f)\Omega,\phi_0(g)\Omega)$.
Thus we have
$$
\gather
\lim\limits_{t\to\plusmin\infty}(\phi(f_1^t)\ldots\phi
(f_k^t)\Omega,\phi(g_1^t)\ldots\phi(g_\ell^t)
\Omega)=\lim\limits_{t\to\plusmin\infty}
(\phi_0(f_1^t)\ldots\phi_0(f_k^t)\Omega,\\
\phi_0(g_1^t)\ldots\phi_0(g_\ell^t)\Omega)=
(\phi_0(f_1)\ldots\phi_0(f_k)\Omega,
\phi_0(g_1)\ldots\phi_0(g_\ell)\Omega)
\endgather
$$
We can now finish the proof.
For free field theory the space $\scrD$ is generated by
the vectors $\phi_0(f_1)\ldots\phi_0(f_n)\Omega$ for
$f_i$ as above, because $\phi_0(f)x$ depends only on
$\scrF(f)\vrulesub{\scrO_m^{\plus}}$ for $f\in\scrS$,
$x\in\scrH$.

For $x\in\scrH$ let us choose functions $f_i$ such that
$x=\sum\phi_0(f_{i_1})\ldots\phi_0(f_{i_s})\Omega$,
and consider the limit $\lim\limits_{t\to\plusmin\infty}
\sum\phi(f_{i_1}^t)\ldots\phi(f_{i_s}^t)\Omega$.
We proved already that for any $y\in \scrH$, and functions $g_j$ such that 
$y=\sum\phi_0(g_{j_1})\ldots\phi_0(g_{j_r})\Omega$ we have
$$
\left(\lim\limits_{t\to\plusmin\infty}
\sum\phi(f_{i_1}^t)\ldots\phi(f_{i_s}^t)\Omega,\lim\limits_{t\to\plusmin\infty}
\sum\phi(g_{j_1}^t)\ldots\phi(g_{j_r}^t)\Omega \right)=(x,y).
$$
Hence the limits do not depend on the choice of $f_i$, $g_j$, and give
 correctly defined isometric imbedding. 
The Theorem is proved.
\enddemo

\subhead{4.4} {Scattering matrix}\endsubhead
\proclaim{Proposition 2}
We have: $\im(W_\plus)=\im(W_-)$.
\endproclaim

Thus we can define the $S$-matrix $S=W_\plus^*W_-
\in\End(\scrH_0)$.
It is interesting that $S$ can be described in terms of the
Wightman functions of the theory.


The precise formulation is as follows. Let $\tilde f$ denote the
partial Fourier
transform of $f$ in space variables. Vector $\vec v\in \dbR^{d-1}$ is called
a velocity of $f$ if  we have $\vec v = \vec p /\sqrt{\vec p ^2+m^2}$ for
$\vec p$ such that $(p_0, \vec p)\in \supp (\tilde f)$ for some real number
$p_0$. 


\proclaim{Theorem} {\rm (Lehmann-Symanzik-Zimmerman)}

\roster
\runinitem"a)"
The residue $\scrF(\scrW_n^T)\vrulesub{(\scrO_m^\plus)^n}$
is a well-defined distribution on $V_n$ supported on
$(\scrO_m^\plus)^n$, where $\scrW_n^T$ is the
time-ordered Wightman function (see lecture 2, \S 2.4 for the definition).

\item"b)"
Assume $f_i\in\scrS$ and the set of velocities of $f_i$ and $f_j$
does not intersect for $i\not = j$.
Then:
$$
\split
\sum\limits_{I\subset(1,\dotsc,n)}(-1)^{\vert I\vert}
&\left<W_-\left(\prod\limits_{i\in I}\phi_0(f_i)\Omega\right),
W_\plus\left(\prod\limits_{j\notin
I}\phi_0(f_j)\Omega\right)\right>=\\
&=\left<\Res(\scrF
(\scrW_n^T)),\scrF(f_1)\times\ldots\times
\scrF(f_n)\right>\,.
\endsplit
$$
\endroster
\endproclaim

It is an elementary combinatorial problem to deduce an
expression for matrix coefficients of the $S$-matrix:
$\left<W_-\prod\phi_0(f_1)\Omega,W_\plus\prod\phi_0
(f_j)\Omega\right>$ in terms of $\scrF(\scrW^T)$ from the
last formula.


\centerline{\bf References}

\medskip
\ref
\by M. Reed, B. Simon
\book Methods of Modern Mathematical Physics, vol. 4. Analysis of
operators; \S XI.16 
\publ Academic Press
\yr 1978
\endref


\medskip
\ref
\by J. Glimm, A. Jaffe
\book Quantum Physics. (A functional integral point of view); 
\S\S 13.5--14.2
\publ Springer-Verlag
\yr 1987
\endref






 







\enddocument





