
%%%%%%%%%%%%%%%%%%%%%%%%%%%%%%%%%%%%%%%%%%%%%%%%%%%%%%%%%%%%%%%%%%%%%%%%%
%%%%%%%%%%%                                                  %%%%%%%%%%%%
%%%%%%%%%%%        Lectures on Conformal Field Theory        %%%%%%%%%%%%
%%%%%%%%%%%                       by                         %%%%%%%%%%%%
%%%%%%%%%%%               Krzysztof Gawedzki                 %%%%%%%%%%%%
%%%%%%%%%%%                                                  %%%%%%%%%%%%
%%%%%%%%%%%                 IAS, fall 1996                   %%%%%%%%%%%%
%%%%%%%%%%%                                                  %%%%%%%%%%%%
%%%%%%%%%%%       lectures 1, 2 and 3, latex file            %%%%%%%%%%%%
%%%%%%%%%%%                                                  %%%%%%%%%%%%
%%%%%%%%%%%%%%%%%%%%%%%%%%%%%%%%%%%%%%%%%%%%%%%%%%%%%%%%%%%%%%%%%%%%%%%%%
\documentstyle[12pt]{article}
\setlength{\textwidth}{17.3cm}
\setlength{\textheight}{21.82cm}
\hoffset -20mm
\topmargin= -0.45cm
\raggedbottom
\raggedbottom
\newcommand{\be}{\begin{eqnarray}}
\newcommand{\en}{\end{eqnarray}\vs 0.5 cm}
\newcommand{\non}{\nonumber}
\newcommand{\no}{\noindent}
\newcommand{\vs}{\vskip}
\newcommand{\hs}{\hspace}
\newcommand{\e}{\'{e}}
\newcommand{\ef}{\`{e}}
\newcommand{\dd}{d\hspace{-0.16cm}{}^-}
\newcommand{\hh}{h\hspace{-0.23cm}{}^-}
\newcommand{\p}{\partial}
\newcommand{\ep}{\epsilon}
\newcommand{\pp}{-\hspace{-0.28truecm}-\hspace{-0.67truecm}o}
\newcommand{\un}{\underline}
\newcommand{\Nt}{{{\bf t}}}
\newcommand{\Ga}{{{\Gamma}}}
\newcommand{\ga}{{{\gamma}}}
\newcommand{\Nn}{{{\bf n}}}
\newcommand{\NR}{{{\bf R}}}
\newcommand{\NA}{{{\bf A}}}
\newcommand{\NP}{{{\bf P}}}
\newcommand{\NC}{{{\bf C}}}
\newcommand{\NT}{{{\bf T}}}
\newcommand{\NZ}{{{\bf Z}}}
\newcommand{\NQ}{{{\bf Q}}}
\newcommand{\NE}{{{\bf E}}}
\newcommand{\NH}{{{\bf H}}}
\newcommand{\NN}{{{\bf N}}}
\newcommand{\Nx}{{{\bf x}}}
\newcommand{\Ny}{{{\bf y}}}
\newcommand{\NX}{{{\bf X}}}
\newcommand{\NY}{{{\bf Y}}}
\newcommand{\NW}{{{\bf W}}}
\newcommand{\NV}{{{\bf V}}}
\newcommand{\Nk}{{{\bf k}}}
\newcommand{\Nm}{{{\bf m}}}
\newcommand{\Nz}{{\bf z}}
\newcommand{\Ng}{{\bf g}}
\newcommand{\Nh}{{\bf h}}
\newcommand{\Nj}{{\bf j}}
\newcommand{\Nr}{{\bf r}}
\newcommand{\qq}{\begin{eqnarray}}
\newcommand{\de}{\bar\partial}
\newcommand{\da}{\partial}
\newcommand{\ee}{{\rm e}}
\newcommand{\Ker}{{\rm Ker}}
\newcommand{\qqq}{\end{eqnarray}}
\newcommand{\llambda}{\mbox{\boldmath $\lambda$}}
\newcommand{\aalpha}{\mbox{\boldmath $\alpha$}}
\newcommand{\xx}{\mbox{\boldmath $x$}}
\newcommand{\xxi}{\mbox{\boldmath $\xi$}}
\newcommand{\kk}{\mbox{\boldmath $k$}}
\newcommand{\tr}{\hbox{tr}}
\newcommand{\ch}{\hbox{ch}}
\newcommand{\ad}{\hbox{ad}}
\newcommand{\Lie}{\hbox{Lie}}
\newcommand{\w}{{\rm w}}
\newcommand{\R}{{\bf R}}
\newcommand{\CA}{{\cal A}}
\newcommand{\CB}{{\cal B}}
\newcommand{\CC}{{\cal C}}
\newcommand{\CD}{{\cal D}}
\newcommand{\CE}{{\cal E}}
\newcommand{\CF}{{\cal F}}
\newcommand{\CG}{{\cal G}}
\newcommand{\CH}{{\cal H}}
\newcommand{\CI}{{\cal I}}
\newcommand{\CJ}{{\cal J}}
\newcommand{\CK}{{\cal K}}
\newcommand{\CL}{{\cal L}}
\newcommand{\CM}{{\cal M}}
\newcommand{\CN}{{\cal N}}
\newcommand{\CO}{{\cal O}}
\newcommand{\CP}{{\cal P}}
\newcommand{\CQ}{{\cal Q}}
\newcommand{\CR}{{\cal R}}
\newcommand{\CS}{{\cal S}}
\newcommand{\CT}{{\cal T}}
\newcommand{\CU}{{\cal U}}
\newcommand{\CV}{{\cal V}}
\newcommand{\CX}{{\cal X}}
\newcommand{\CY}{{\cal Y}}
\newcommand{\CW}{{\cal W}}
\newcommand{\CZ}{{\cal Z}}
\newcommand{\s}{\hspace{0.05cm}}
\newcommand{\m}{\hspace{0.025cm}}
\newcommand{\La}{\Lambda}
\newcommand{\la}{\lambda}
\newcommand{\hf}{{_1\over^2}}
\newcommand{\hslash}{{h\hspace{-0.23cm}^-}}
\newcommand{\Di}{{\slash\hs{-0.21cm}\partial}}
\pagestyle{plain}
\renewcommand{\baselinestretch}{0.3}
\begin{document}
\
\vskip 2cm
\begin{center}
{\Large{\bf{LECTURES
\vskip 0.3cm
on
\vskip 0.3cm
CONFORMAL FIELD THEORY}}
\vskip 2cm}

Krzysztof Gaw\c{e}dzki
\vskip 0.5cm

C.N.R.S., I.H.E.S., 91440 Bures-sur-Yvette, France
\end{center}
\vskip 2.1cm


\noindent{\large{\bf Introduction}}
\vskip 0.8cm

\no Over the last decade and a half, conformal field theory (CFT)
has been one of the main domains of interaction between
theoretical physics and mathematics. The present review is designed
as an introduction to the subject aimed at mathematicians.
Its scope is limited to certain simple aspects
of the theory of conformally invariant quantum fields in two
space-time dimensions. The two-dimensional CFT experienced
an explosive developement following the seminal 1984 paper
of Belavin, Polyakov and Zamolodchikov, although many
of its concepts were introduced before that date. It still plays
a very important role in numerous recent developments concerning
higher-dimensional quantum fields. From the mathematical point
of view, CFT may be defined as a study of Virasoro algebra (or
algebras containing it), of its representations and of their
intertwiners. The theory defies, however, such narrowing
definitions which obstruct the much wider view that it opens
and into which we offer here only some glimpses. In four lectures
we discuss:
\vskip 0.3cm

\no- \ conformal free fields,
\vskip 0.3cm

\no- \ axiomatic approach to conformal field theory,
\vskip 0.3cm

\no- \ perturbative analysis of two-dimensional sigma models,
\vskip 0.3cm

\no- \ exact solutions of the Wess-Zumino-Witten and coset theories.
\vskip 0.35cm

\no To signal the omissions, whose full list
would be much longer, let us point out that almost no mention
is made of lattice models whose critical points are described
by CFT's, of the perturbative approach to string theory, based
on the two-dimensional CFT, of superconformal theories. The modest
goal of these lectures is to make the physical literature on CFT,
both the original papers and the textbooks
(e.g. ``Conformal Field Theory'' by Di Francesco-Mathieu-S\'{e}n\'{e}chal,
Springer 1996) more accessible to mathematicians.
\eject

\
\vskip 0.8cm

\noindent{\large{\bf Lecture 1.\ \ Simple functional integrals}}
\vskip 0.8cm

\no\un{Contents\s}:
\vskip 0.5cm

\no 1. \ What is quantum field theory (for me)?

\no 2. \ Euclidean free field and Gaussian functional integrals

\no 3. \ Feynman-Kac formula

\no 4. \ Massless free field with values in $S^1$: the partition
functions

\no 5. \ Toroidal ``compactifications'' of 2-dim.
free fields, baby T-duality and mirror symmetry

\no 6. \ Compactified 2d free fields: the correlators
\vskip 1.5cm


\noindent{\bf 1.\ \s What is quantum field theory?}
\vskip 0.4cm

\no Field theory deals with maps $\phi$
between space $\Sigma$ (the space-time) and space $M$ (the target).
These spaces come with additional structure, e.g. they may be
Riemannian or pseudo-Riemannian manifolds. The case of
Minkowski signature on $\Sigma$ is the one of field theory
proper whereas the Euclidean signature corresponds to
static (equilibrium) situations. In many cases, however,
(for example for flat $\Sigma$) the passage from one signature
to the other may be obtained by analytic continuation
in the time variable (``Wick rotation'' $t\mapsto it$)
and both situations may be studied interchangeably, the
Euclidean setup being sometimes more convenient.
\vskip 0.3cm

An important datum of the field theory is the action, a
local functional of $\phi$. For example, one may consider
$S(\phi)=\int_{_\Sigma}|d\phi|^2 dv$ (where the metric
structures on $\Sigma$ and $M$ and a volume on $\Sigma$
must be used to give sense to the right hand side).
\vskip 0.3cm

In the classical field theory one studies the extrema of
the action functional i.e. maps $\phi_{cl}$ satisfying
\qq
\delta S(\phi_{cl})=0\s.
\non
\qqq
The extremality condition is a PDE for $\phi_{cl}$, e.g.
the wave or Laplace equation or the Maxwell, Yang-Mills,
or Einstein ones, to mention only the most famous cases.
One should bear in mind that non-linear PDE's is
a complicated subject where our ignorance exceeds our
knowledge.
\vskip 0.3cm

Following an extremely intuitive reformulation of quantum
field theory (QFT) by Feynman, the latter consists
in studying functional integrals
\qq
\int\limits_{Map(\Sigma,M)} F(\phi)
\ \ee^{-{1\over\hh}S(\phi)}\ D\phi
\label{FFI}
\qqq
where $F(\phi)$ is a functional of $\phi$ (an ``insertion'')
and $D\phi$ stands for a local product $\prod_{_{
x\in\Sigma}}d\mu(\phi(x))$ of measures on $M$. The above expression
is formal and one of the aims of these lectures
is to show that it may be given sense and even calculated
in some simple situations. More generally, however,
the functional integral written above should be considered
as an approximate expression for structures which live their
own lives, different and in some aspects more interesting
then the lives of the objects whose symbols appear in
the integral. Still, although formal and approximate,
the functional integral language proved extremely useful
in studying the QFT structures. It also made the relation
of QFT to classical field theory quite intuitive: unlike in
the case of the latter, all maps $\phi$ (called often
but somewhat abusively classical field configurations or
classical fields) give contributions to QFT, each with
the probability amplitude $p(\phi)\s\sim\s
\ee^{-{1\over\hh}S(\phi)}$ (in Minkowski case,
these are not probabilities since $S(\phi)$ should
be taken imaginary, but never mind). The classical physics
corresponds to the stationary phase or saddle point approximation
in which all contributions to the integral (\ref{FFI}) but those
of the stationary points of $S$ are discarded. Such an approximation
is justified when the Planck constant $\hh$ may be treated as very
small (as in the usual macro-scale physics but not for example in
superfluid helium).
\vskip 0.3cm

It should be stressed that, in general, the relation of quantum
to classical is not one to one (as the formulation (\ref{FFI})
could suggest) and not even many to many, except for special
situations, e.g. with lots of symmetries
on both levels. Such situations are of special interest
for mathematicians because they allow to reduce QFT to
the more familiar classical structures. The QFT approach
allowed in many such cases new insights, recall, for example,
the use of topological or quasi-topological field theories as
factories of invariants. This motivates the utilitarian interest
of mathematicians in QFT. The very difficulty in making
sense out of the integrals (\ref{FFI}) is at the origin of a deeper
source of mathematical interest of QFT, namely in the mathematics
of the new structures carried by QFT. The mathematical structures
in integrable or conformal two-dimensional field theories or
in four-dimensional SUSY gauge theories just emerging provide here
the examples (not speaking about mathematics which promises to underlie
the panoply of string theory dualities).
\vskip 0.3cm

It may be good to remind briefly what do physicists use QFT for.

i/. It provides a relativistic theory of interactions of
elementary particles. And so Quantum Electrodynamics (QED)
describes interactions of electrons, their anti-particles
positrons and photons, the Glashow-Weinberg-Salam theory
of electro-weak interactions describes at the same time
the beta decays and QED, Quantum Chromodynamics (QCD) deals
with the strong interactions (quarks forming nucleons).
The latter two build what is called the standard model
of particle physics.

ii/. QFT in its Euclidean version describes critical phenomena
at the $2^{\m\rm nd}$ order phase transition points like that
in H$_2$O at temperature $T_c\cong 374^\circ\m$C and pressure
$p_c\cong2.2\times 10^7$ pascals ($\sim 20\m$atm), or that
in Fe at $T_c\cong 770^\circ\m$C or in the Ising model at its
critical temperature. The criticality is characterized by
slow decay with the distance of statistical correlations
whose asymptotics is described by Euclidean QFT.

iii/. In string theory aiming at unification of gravity
with the other interactions (from point i/.) two-dimensional
conformal QFT provides the classical (and perturbative)
solutions and the the quantum string theory proper
(still to be non-perturbatively defined) may be considered
as a deformation of quantum field theory where particles
are replaced by strings (the typical size of the string being
the deformation parameter).

iv/. Finally, many QFT techniques are used in the theory of
non-relativistic condensed matter.
\vskip 1cm

\no{\bf 2.\ \s Euclidean free field and Gaussian functional integrals}
\vskip 0.4cm

\no In the rest of this lecture we shall describe how one may
give sense to functional integral (\ref{FFI}) in the simplest case of
free field. This is the case where the space of maps $Map(\Sigma,M)$
is a vector space (inheriting the linear structure from that of $M$)
or is a union of affine spaces and where the action functional
$S$ is quadratic. The corresponding functional integral
is Gaussian plus, in the $2^{\m \rm nd}$ case, an easy but interesting
decoration (in theta functions). The adjective ``free'' refers to the
absence of particle interactions which is related to linearity
of the classical equations.
\vskip 0.3cm

Let $(\Sigma,\gamma)$ be a Riemannian, $(d+1)$-dimensional,
oriented, compact manifold and let $M=\NR$ (we consider
the Euclidean case). The action functional is taken as
\qq
S(\phi)={_\beta\over^{4\pi}}\int\limits_{\Sigma}(|d\phi|^2
+m^2\phi^2)\s dv\s\equiv\s{_1\over^2}\m(\phi,\m G^{-1}\phi)_{_{L^2}}
\label{Am}
\qqq
where $dv$ is the Riemannian volume and the operator ${\beta\over 2\pi}\m G
=(-\Delta+m^2)^{-1}$ is often called the (Euclidean) propagator of
free field of mass $m$.
\vskip 0.3cm

The simplest functional integral of the type
(\ref{FFI}) is the one with trivial insertion $F=1$ giving what
is called ``statistical sum'' or ``partition function''
and denoted traditionally by $Z$:
\qq
Z\ =\s\int\limits_{Map(\Sigma,\NR)}\ee^{-S(\phi)}\ D\phi
\s\ =\s\s\int\limits_{Map(\Sigma,\NR)}\ee^{-{1\over 2}\m(\phi,
\m G^{-1}\phi)}\ D\phi
\non
\qqq
(we have put $\hh=1$ for simplicity). The names come from
statistical physics: the integral sums the probabilities
(probability amplitudes) $p(\phi)$ of all microscopic states
$\phi$ of the system. The space $Map(\Sigma,\NR)$ may be considered
as the Hilbert space with the $L^2$ scalar product using the metric
volume $dv$ on $\Sigma$. Were this space finite dimensional,
the integral would give
\qq
Z\s=\s (\det\s G)^{1\over 2}
\non
\qqq
if we normalized $D\phi=\prod_{_i}{d\phi_i\over\sqrt{2\pi}}$
where $(\phi_i)$ are any orthonormal coordinates on the Hilbert
space of maps. It is sensible to maintain the above
formula as a definition of the formal functional integral for
the partition function even in the infinite-dimensional case.
It is necessary then to give sense to the determinant
of the positive operator $G$ whose (discrete) eigenvalues
$\la_n$, $n=1,2,\dots$, behave as $\CO(n^{-2/(d+1)})$. A convenient
(but nonunique) way to do it is by the zeta-function
regularization defining
\qq
\det\s G\s=\s\ee^{-\zeta'_G(0)}
\non
\qqq
where $\zeta_G(s)$, given as \s$\sum_n\la_n^{-s}\s$ for
$\s{\rm Re}\s s<-{d+1\over2}$, extends to a meromorphic function analytic
in the vicinity of zero. We shall stick to this definition
of infinite determinants throughout the present lectures.
\vskip 0.3cm

The next functional integrals we may like to compute
are the ones for the correlations functions depending
on a sequence $(x_i)_{i=1}^{\s n}$ of points in $\Sigma$
\qq
\langle\phi(x_1)\s\cdots\s\phi(x_n)\rangle\
\equiv\s{\int\limits_{Map(\Sigma,\NR)}
\phi(x_1)\s\cdots\s\phi(x_n)\ \ee^{-S(\phi)}\ D\phi\over
\int\limits_{Map(\Sigma,\NR)}\ee^{-S(\phi)}\ D\phi}\s.
\non
\qqq
Again mimicking the case of finite-dimensional Gaussian integrals,
we may define the formal functional integral on the right hand side
by setting
\qq
\langle\phi(x_1)\s\cdots\s\phi(x_n)\rangle\ =\ \cases{
\hbox to 8cm{$0$\hfill}{\rm for}\ n\ {\rm odd},\cr\cr
\hbox to 8cm{$G(x_1,x_2)$\hfill}{\rm for}\ n=2,\cr\cr
\hbox to 8cm{$G(x_1,x_2)\s G(x_3,x_4)+G(x_1,x_3)\s G(x_2,x_4)$\hfill}\cr
\hbox to 8cm{$+G(x_1,x_4)\s G(x_2,x_3)$\hfill}{\rm for}\ n=4,\cr\cr
\hbox to 8cm{$\sum\limits_{{\rm pairings}\atop\{(i_+,i_-)\}}
\prod\limits_{(i_+,i_-)}G(x_{i_+},\m x_{i_-})$\hfill}
{\rm for}\ n\ {\rm even}}
\non
\qqq
where $G(x,y)$ denotes the kernel of operator $G$
which is smooth for $x\not=y$ and exhibits a coinciding points
singularity $\sim\s\ln {\rm dist}(x,y)$
for $d=1$ and $\sim\s{\rm dist}(x,y)^{-d+1}$ for $d>1$.
Such a definition of the correlation functions is additionally
substantiated by the fact that there exists a probability
measure $d\mu_{_G}$ on the space of distributions $\CD'(\Sigma)$ such
that
\qq
\langle\phi(x_1)\s\cdots\s\phi(x_n)\rangle\ =\
\int\limits_{\CD'(\Sigma)}\phi(x_1)\s\cdots\s\phi(x_n)\ d\mu_{_G}(\phi)
\non
\qqq
where the equality is understood in the sense of distributions.
$\phi(x)$ may be then considered as a random distribution.
\vskip 1cm

\no{\bf 3.\ \s Feynman-Kac formula}
\vskip 0.4cm

\no Some people in the audience may wonder what it all has
to do with Minkowski space field theory involving Hilbert space
$\CH$, quantum Hamiltonian $H$ and quantum field operators
acting in $\CH$ since the (Euclidean) functional integral
scheme has led us to entirely commutative structures
as the Euclidean random distributions $\phi(x)$ which may, at
most, be considered as a distribution with values in commuting
multiplication operators acting in $L^2(\CD'(\Sigma),d\mu_{_G})$.
The relation of the two schemes is provided by the so called
{\bf Feynman-Kac formula}. Let us start from a simple
quantum mechanical example.
\vskip 0.5cm

\no{\bf Example 1}. \ $d=0$, \ $\Sigma=[0,L]$ with the periodic
identification of the ends and the standard metric.
In this case, $d\mu_{_G}$ is supported by the space of continuous
functions $\CC_{per}([0,L])$ and is essentially a version of the Wiener
measure. More exactly, it differs from the latter by the density
\s$\sim\s\ee^{-{\beta m^2\over 2\pi}\int_0^L\phi(x)^2}$. Suppose that
$0\leq x_1\leq x_2\leq\s\dots\s\leq x_n\leq L$. Then
\qq
\int\limits_{\CC_{per}([0,L])}\phi(x_1)\s\cdots\s\phi(x_n)\ d\mu_{_G}(\phi)
\ =\ {\tr\ \ee^{-x_1 H}\varphi\s\s\ee^{(x_1-x_2)H}\varphi\s\cdots\s
\varphi\s\s\ee^{(x_n-L)H}\over\tr\ \ee^{-LH}}
\label{FK}
\qqq
where
\qq
H\s=\s-\m{_\pi\over^\beta}\s{_{d^2}\over^{d\varphi^2}}\s+\s
{_{\beta m^2}\over^{4\pi}}\s\varphi^2\s-{_m\over^2}\s=\s
m\left(-\sqrt{{_\pi\over^{\beta m}}}\s{_d\over^{d\varphi}}\s+\s
\sqrt{{_{\beta m}\over^{4\pi}}}\s\varphi\right)
\left(\sqrt{{_\pi\over^{\beta m}}}\s{_d\over^{d\varphi}}\s+\s
\sqrt{{_{\beta m}\over^{4\pi}}}\s\varphi\right)\s\equiv\s
m\s a^* a
\non
\qqq
is the Hamiltonian of a harmonic oscillator acting in
$L^2(\NR,d\varphi)$. $a$ and its adjoint $a^*$
satisfy the canonical commutation relation
\qq
[a,a^*]=1\s.
\non
\qqq
The ground state $\Omega$ of $H$ is proportional to $\ee^{-{\beta m
\over 4\pi}\m\varphi^2}$ and corresponds to the zero eigenvalue.
$\Omega$ is annihilated by $a$ and the higher $H$-eigenstates are
obtained by aplying powers of $a^*$ to $\Omega$, each $a^*$ raising
energy (i.e. eigenvalue of $H$) by $m$
($a$ is called the annihilation and $a^*$ the creation operator).
Hence the spectrum
of $H$ is $\{0,m,2m,\dots\}=m\NZ_+$. With the use of the orthonormal
basis \s$({1\over\sqrt{n!}}(a^*)^n\Omega)_{n=1}^{\s\infty}$ composed,
up to normalizations, from the Hermite polynomials $H_n(\sqrt{{\beta
m\over\pi}}\s\varphi)$ times $\Omega$, $L^2(\NR,d\varphi)$ may be
identified with (the Hilbert-space completion of) the symmetric
algebra $S\NC$ (the bosonic Fock space over $\NC$). Note that
\qq
\varphi\s=\s\sqrt{_\pi\over^{\beta m}}\s\s(a+a^*)\s.
\non
\qqq
\vskip 0.3cm

\no{\bf Problem 1}. \ Consider formula (\ref{FK}) for
the 2-point function.
\vskip 0.1cm
\no (a). \ Use the Fourier transform to write the left hand side.
Show that its $L\to\infty$ limit $G_\infty(x_1,x_2)$ exists.
\vskip 0.1cm
\no (b). \ Prove that for $x_1,\dots, x_n\m>\m0$ and complex
numbers $\lambda_1,\dots,\lambda_n$,
\qq
\sum\limits_{k,l=1}^n\bar\lambda_k\m\lambda_l\, G_\infty
(-x_k,\m x_l)\ \geq\ 0\m.
\qqq
\vskip 0.1cm
\no (c). \ What is the $L\to\infty$ limit of the right hand side
of Eq.\,\,(\ref{FK}) for $n=2$?
\vskip 0.1cm
\no (d). \ Show that both sides of Eq.\,\,(\ref{FK}) with $n=2$
coincide at $L=\infty$. Prove (b) using this result.
\vskip 0.1cm
\no (e). \ Prove relation (\ref{FK}) for $n=2$ and finite $L$.
\vskip 0.5cm

\no It may be more natural to read the Feynman-Kac formula from
the right to left. $\ee^{-xH}(\phi,\phi')$ is nothing else but
the transition probability to pass from $\phi$ to $\phi'$
in time $x$ which may be used to define the Markov process
$\phi(x)$ with the measure on the space of continuous realizations
coinciding with $d\mu_{_G}$.
\vskip 0.7cm

\no{\bf Example 2}. \ $d>0$, \ $\Sigma=[0,L]^{d+1}$ with periodic
identifications. Now $d\mu_{_G}$ is carried by genuinely distributional
$\phi$'s. Let $(x_i=(x_i^0,\Nx_i))_{i=1}^n$ be s. t. $0<x_1^0<x_2^0<
\dots<x_n^0<L$. The Feynman-Kac formula now takes the form
\qq
\int\limits_{\CD'([0,L]^{d+1})}\phi(x_1)\s\cdots\s\phi(x_n)\ d\mu_{_G}(\phi)
\ =\ {\tr\ \ee^{-x_1^0 H}\varphi(\Nx_1)\s\s\ee^{(x_1^0-x_2^0)H}
\varphi(\Nx_2)\s\cdots\s
\varphi(\Nx_n)\s\s\ee^{(x_n^0-L)H}\over\tr\ \ee^{-LH}}
\label{FK1}
\qqq
where the quantum Hamiltonian $H$ is a positive self-adjoint operator
in the Hilbert space $\CH$, a tensor product of an infinite number of
harmonic oscillators, one for each Fourier mode $\varphi_\Nk$
of the classical time zero field \s$\varphi_\Nk\s=\int_{[0,L]^d}
\ee^{\m i\Nk\cdot\Nx}\phi(0,\Nx)\s d\Nx$\s:
\qq
\CH\s=\s\bigotimes\limits_{\pm\Nk\in{2\pi\over L}\NZ^d}
L^2(\NC,\m d^2\varphi_\Nk)\s.
\non
\qqq
The annihilation operators
\qq
a_\Nk\s=\s\sqrt{_{\pi L^d}\over^{\beta k_0}}\s{_d\over^{d\varphi_{\Nk}}}
\s+\s\sqrt{_{\beta k_0}\over^{4\pi L^d}}\m\s\varphi_{-\Nk}\s,
\non
\qqq
where $k_0=\sqrt{\Nk^2+m^2}$, and the creation operators
$a_\Nk^*$ adjoint to them satisfy the canonical commutation relations
\qq
[a_\Nk,\m a^*_{\Nk'}]\s=\s\delta_{\Nk,\Nk'}
\non
\qqq
with all the other commutators vanishing.
The quantum (time zero) field is
\qq
\varphi(\Nx)\s=\s\sum\limits_\Nk\sqrt{{_\pi\over^{\beta k_0 L^d}}}
\s\s\ee^{-i\m \Nk\cdot\Nx}\s\s(a_{-\Nk}+\m a_{\Nk}^{\s*})\s.
\non
\qqq
The Hamiltonian
\qq
H\s=\s\sum\limits_\Nk k_0\s\m a_\Nk^* a_\Nk
\non
\qqq
has \s$\Omega\sim\ee^{\sum_\Nk{\beta k_0\over4\pi L^d}\s|\varphi_\Nk|^2}\s$
as the ground state annihilated by all $a_\Nk$. The spectrum
of $H$ is $\sum_\Nk k_0\NZ_+$.
\vskip 0.3cm

\no There are three natural ways to look at the Hilbert space $\CH$:
\vskip 0.1cm
i/. $\CH$ is an infinite tensor product of oscillator spaces
$L^2(\NC,\m d^2\varphi_\Nk)$ (how should it be defined?);
\vskip 0.1cm
ii/. $\CH$ is the Hilbert space completion of the symmetric
algebra $S(l^2({2\pi\over L}\NZ^d))\s\cong\s S(L^2([0,L]^d))$;
this is the Fock space picture;
\vskip 0.1cm
iii/. $\CH$ is a space of functionals of variables $\varphi_\Nk$
or of the time zero classical field $\phi(0,\Nx)$ obtained
by acting by creation operators $a_\Nk^{\m*}$ on the vacuum
functional $\Omega$.
\vskip 0.3cm

One may introduce the (Minkowski) time dependence of the
quantum field by defining
\qq
\varphi(t,\Nx)\s=\s\ee^{\m itH}\s\varphi(\Nx)\s\s\ee^{-itH}\s.
\non
\qqq
\vskip 0.3cm

\no{\bf Problem 2}. \ Show that
\qq
\varphi(t,\Nx)\s=\s\sum\limits_\Nk\sqrt{{_\pi\over^{\beta k_0 L^d}}}
\s\s(\m\ee^{-i\m t\m k_0+i\m \Nk\cdot\Nx}\s\m a_{\Nk}\s+\s
\ee^{i\m t\m k_0-i\m \Nk\cdot\Nx}\s\m a_{\Nk}^{\s*}\m)\s.
\label{mff}
\qqq
\vskip 0.3cm

The infinite volume limit $L\to\infty$ of the formulae for $H$ and
$\varphi(t,\Nx)$ may be easily taken if we introduce operators
$a(\Nk)= L^{d/2}a_\Nk$. In the limit one obtains the operator-valued
distributions $a(k)$ and their adjoints $a^*(k)$ acting in
the Fock space ${\overline{S(L^2(\NR^d,\dd\Nk))}}$ ($\dd\Nk\equiv
d\Nk/(2\pi)^d$) and satisfying the commutation relations
\qq
[a(\Nk),a^*(\Nk')]\s=\s(2\pi)^d\s\delta(\Nk-\Nk')\s.
\non
\qqq
By identifying $L^2(\NR^d,\dd\Nk)$ (by multiplication by $\sqrt{2k_0}$)
with the space of functions on the upper mass hyperboloid
$\{(k_0,\Nk)\}$ square-integrable with the Lorentz-invariant
measure ${\dd\Nk\over2k_0}$, one obtains the Minkowski
scalar free field of mass $m$ constructed in more abstract
and explicitly Poincare-covariant way in David Kazhdan's
lectures (check it!).
\vskip 1cm

\no{\bf 4.\ \s Massless free field with values in $S^1$}
\vskip 0.4cm

\no Let us pass to the next case where $(\Sigma,\gamma)$ is
again a general compact $(d+1)$-dimensional manifold, $M=\NR/2\pi\NZ
\cong S^1$ and the action functional is that of the massless
free field: \s$S(\phi)={\beta\over 4\pi}\int_{_\Sigma}|d\phi|^2\s
dv\m$ (note that $\beta$ has a natural interpretation
of square of the radius $\rho$ of the circle if we rewrite the classical
action as ${_1\over^{4\pi}}\int|d\phi|^2\s dv$ but use the metric
$\rho^2d\phi^2$ on the target). This is the case of a conformal
(invariant) field theory with the conformal group acting
(projectively) in the corresponding
Hilbert space of states, transforming covariantly field operators.
We shall get there slowly discussing in more detail the
$d=1$ case where the conformal group is infinite-dimensional,
essentially $=\s Diff(S^1)\times Diff(S^1)$. Let us start with
an elementary treatment of the functional-integral.
\vskip 0.2cm


How should we view the space of maps from $\Sigma$ to $S^1$?
A convenient way is to define
\qq
Map(\Sigma,\NR/2\pi\NZ)\ =\ \bigcup\limits_{\chi\m\in\m
Hom(\pi_1(\Sigma),2\pi\NZ)}Map(\tilde\Sigma,\NR)_\chi\s/\s2\pi\NZ
\non
\qqq
where \s$\phi_\chi\s\in\s Map(\tilde\Sigma,\NR)_\chi\s$ is a
a function on the universal cover $\tilde\Sigma$ of $\Sigma$
equivariant with respect to the action of the fundamental
group:
\qq
\phi_\chi(ax)\s=\s\phi_\chi(x)\s+\s\chi(a)\quad\quad
{\rm for}\s\ a\in\pi_1(\Sigma)\s.
\non
\qqq
Note that this definition makes sense for maps of arbitrary class
(smooth continuous or distributional).
$Hom(\pi_1(\Sigma),2\pi\NZ)\cong H^1(\Sigma,2\pi\NZ)$ with
$\chi$ given by the periods of $\alpha\in H^1(\Sigma,2\pi\NZ)$.
Each $\phi_\chi\in Map_\chi$
may be uniquely decomposed according to
\qq
\phi_\chi\s=\s\int_{x_0}^x\alpha_h\s+\s\psi\ \equiv\ \phi_h\s+\s
\psi
\label{sepa}
\qqq
where $\alpha_h$ is the harmonic representative of $\alpha\in H^1$
corresponding to $\chi$, $x_0$ is the base point of $\Sigma$
and $\psi$ is a univalued function on $\Sigma$.
For the free field action we obtain
\qq
S(\phi_\chi)\s=\s{_{\beta}\over^{4\pi}}
\s\Vert\alpha_h\Vert^2_{_{L^2}}
\s+\s{_\beta\over^{4\pi}}\s\m(\psi\m,\m-\Delta
\psi)_{_{L^2}}
\non
\qqq
(there is no mixed term, why?). This suggests the following definition
of the functional integral for the partition function of the system:
\qq
Z\s\ =\s\int\limits_{Map(\Sigma,S^1)}\ee^{-S(\phi)}\ D\phi
\ &=&\sum\limits_{\alpha\in H^1(\Sigma,2\pi\NZ)}\ee^{-{\beta\over
4\pi}\m\Vert\alpha_h\Vert^2_{L^2}}\s\int\limits_{Map(\Sigma,\NR)}
\ee^{-{\beta\over 4\pi}\s(\psi\m,\m-\Delta\psi)_{L^2}}
\s\s D\psi\cr&=&\sum\limits_{\alpha\in H^1(\Sigma,2\pi\NZ)}
\ee^{-{\beta\over4\pi}\m\Vert\alpha_h\Vert^2_{L^2}}\s\left(
{_{2\pi\s{\rm vol}_\Sigma}\over^{\det'(-{\beta\over 2\pi}\Delta)}}
\right)^{{\hs{-0.06cm}1/2}}
\label{inst}
\qqq
where in $\det'$ the zero mode should be omitted (it contributes
the factor $\sqrt{2\pi\s{\rm vol}_\Sigma}$, where ${\rm vol}_\Sigma=
\int_\Sigma dv$, to the functional integral, why?).
\vskip 0.7cm

\no{\bf Example 3}. \ $d=0$, \ $\Sigma=[0,L]_{per}$. \s In this case,
\s$\alpha_h={2\pi\over L}\m n\s dx\s$ and an easy
calculation (see Problem 4 below) shows that
\qq
{\det}'(-{_{\beta}\over^{2\pi}}\m{_{d^2}\over^{dx^2}})
\s=\s 2\pi L^2/\beta\s.
\non
\qqq
Hence the $d=0$ partition function
\qq
Z\s=\s\sum\limits_{n\in\NZ}\ee^{-\pi\beta L^{-1} n^2}\left({_{2\pi L}
\over^{\det'(-{\beta\over 2\pi}\m{d^2\over dx^2})}}
\right)^{{\hs{-0.06cm}1/2}}\s{\smash{\mathop{=}
\limits_{\rm {Poisson\atop resummation}}}}\
\s\sum\limits_{n\in\NZ}
\ee^{-\pi\beta^{-1}Ln^2}\s=\s\tr\ \ee^{-LH}
\non
\qqq
where now \s$H=-\m{\pi\over\beta}\s{d^2\over d\varphi^2}\s$
is the operator acting in $L^2(\NR/2\pi\NZ,\m d\varphi)$
with the eigenvectors $\ee^{in\varphi},\ n\in\NZ$, corresponding to
eigenvalues $\pi n^2/\beta$.
\vskip 0.5cm

\no{\bf Problem 3}. \ Prove for $0\leq x_1\leq\dots\leq x_n\leq L$ and
$q_i\in\NZ$ the Feynman-Kac formula
\qq
\int\prod\limits_{i=i}^n\ee^{iq_i\phi(x_i)}
\ \ee^{-{\beta\over 4\pi}\int_0^L(d\phi/dx)^2}\ D\phi
\ =\ \tr\ \ee^{x_1 H}\s\ee^{iq_1\varphi}\s\m
\ee^{(x_1-x_2)H}\s\ee^{iq_2\varphi}
\s\cdots\s\ee^{iq_n\varphi}\s\m\ee^{(x_n-L)H}
\non
\qqq
where the functional integral over ${Map([0,L],S^1)}$
on the left hand side is computed as the one for the partition function
$Z$ treated above. Infer that the left hand side may be also expressed
as the expectation \s$\langle\m\ee^{iq_1\phi(x_1)}\s\cdots\s
\ee^{iq_n\phi(x_n)}\m\rangle\s$ w. r. t. the Wiener measure
on the periodic paths on $S^1$ constructed from the transition
probabilities $\ee^{-xH}(\varphi,\varphi')$.
\vskip 0.7cm

Let us discuss in greater detail the case $d=1$ when $(\Sigma,\gamma)$
is a Riemann surface of genus $h_\Sigma$ with a fixed metric $\gamma$
(inducing the complex structure of $\Sigma$). Let us chose a marking
of $\Sigma$ (a symplectic bases $(a_i,b_j)_{_{i,j=1}}^{^{\ h_\Sigma}}$
of $H_1(\Sigma,\NZ)$) with the corresponding  basis $(\omega^i)_{_{i=1}
}^{^{h_\Sigma}}$ of holomorphic 1,0-forms,
\s$\int_{_{a_i}}\omega^j=\delta^{ij}\s$,
$\s\int_{_{b_i}}\omega^j=\tau^{ij}\s$. The imaginary part
$\tau_2$ of the period matrix $\tau=(\tau^{ij})$ is positive.
The equation
\qq
\alpha_h\s=\s{_\pi\over^i}(\bar\tau\Nm+\Nn)^t\tau_2^{\s-1}\omega\s+\s
c.c.
\non
\qqq
for $\Nm,\Nn\in\NZ^g$ gives the harmonic forms in $H^1(\Sigma,2\pi\NZ)$
(with $a_i$-periods $-2\pi m_i$ and $b_j$-periods $2\pi n_j$).
\qq
\Vert\alpha_h\Vert^2_{_{L^2}}\s=\s(2\pi)^2\s(\bar\tau\Nm
+\Nn)^t\tau_2^{\s-1}(\tau\Nm+\Nn)
\non
\qqq
and the sum over $\alpha$ in eq.\s\s(\ref{inst}) may be done
explicitly. After Poisson resummation over $\Nn$, one obtains
the following result
\qq
\sum\limits_{\alpha\in H^1(\Sigma,2\pi\NZ)}
\ee^{-{\beta\over4\pi}\m\Vert\alpha_h\Vert^2_{L^2}}\s=\s
\beta^{-h_\Sigma/2}\s(\det\s\tau_2)^{1/2}\s\m
\vartheta_{_{Q_\beta}}(\tau,\bar\tau)
\label{inst1}
\qqq
where the ``theta function'' $\vartheta_{_{Q_\beta}}$ is defined as follows.
Let $\NE_s$ be the $s$-dimensional Euclidean space. Let $Q$ be a
lattice in $\NE_{s_+,s_-}=\NE_{s_+}\oplus\NE_{s_-}$ considered with
the indefinite metric $|\cdot|^{\s2}_{{\NE_{s_+}}}
-\s|\cdot|^{\s2}_{{\NE_{s_-}}}$. Then
\qq
\vartheta_{_{Q}}(\tau,\bar\tau)\s
=\s\sum\limits_{(q_+,q_-)\s\in\s Q^{^{h_\Sigma}}}
\ee^{\m\pi i\s(q_+,\m\tau\m q_+)\s-\s\pi i\s(q_-,\m\bar\tau\m q_-)}\s.
\non
\qqq
where the decomposition $q=(q_+,q_-)$ is according to that of
$\NE_{s_+,s_-}$. Above,
\qq
Q_\beta\s=\s\{\ (\s{_{\beta^{1/2}m+\beta^{-1/2}n}\over^{\sqrt{2}}}\s,
\s{_{\beta^{1/2}m-\beta^{-1/2}n}\over^{\sqrt{2}}})\ |\ m,n\in\NZ\ \}\
\subset\ \NR\oplus\NR\s.
\non
\qqq
Inserting the relation (\ref{inst1}) into eq.\s\s(\ref{inst}), we obtain
\qq
Z\ \equiv\ Z_\beta\ =\ \ee^{\m(-6\ln{2\pi}+11\ln{\beta}/2)(h_\Sigma-1)}
\ \vartheta_{_{Q_\beta}}(\tau,\bar\tau)
\left({_{{\rm vol}_\Sigma\s\s\det{\tau_2}}
\over^{\det'(-\Delta)}}\right)^{{\hs{-0.06cm}1/2}}\s.
\non
\qqq
{}From eq.\s\s(\ref{inst}) it is obvious that the right hand side
is marking-independent. Technically, this is due to the fact that,
in the indefinite scalar product, the lattice $Q_\beta$ is even
(scalar-products are integers, scalar squares are even) and self-dual.
\vskip 0.3cm

We may discard from $Z$ any factor of the form $({\rm const.}
)^{h_\Sigma-1}$ by the addition to the action of a term proportional
to the integral of the scalar curvature $r$ of $\Sigma$
since $\int_{_\Sigma} r\s dv=4\pi(1-h_\Sigma)$. Doing that we discover
a somewhat miraculous equality
\qq
Z_\beta\ =\ Z_{1/\beta}\s,
\label{T}
\qqq
a consequence of the obvious identity \s$Q_\beta=Q_{1/\beta}\s$.
More directly, identity (\ref{T}) follows from the Poisson
resummation formula applied to the left hand side of
eq.\s\s(\ref{inst1}) and the fact that the lattice
$H^1(\Sigma,2\pi\NZ)$ with the $L^2$ scalar product
is isomorphic to its dual (the isomorphism  is induced
by the intersection form). Eq.\s\s(\ref{T}) is the simplest
manifestation of the so called $T$-{\bf duality}
which states that the $1+1$-dimensional massless free fields with
values in the circles of radius $\rho$ and of radius $\rho^{-1}$ are
indistinguishable. This identification
of inverse radia of free field compactification has a deep
meaning in the string theory context and we shall return to it
in a later discussion.
\vskip 0.3cm

On the genus 1 curve $T_\tau=\NC/(\NZ+\tau\NZ)$ with $\tau$ in
the upper half plane and the standard metric,
\qq
{\det}'(-\Delta)\ =\ \tau_2^2\s|\eta(\tau)|^4
\non
\qqq
where $\eta(\tau)=\ee^{\m\pi i\tau/12}\s\prod\limits_{n=1}^\infty
(1-\ee^{\m 2\pi in\tau})$ is the Dedekind eta function.
\vskip 0.5cm

\no{\bf Problem 4}\s\s\ (a relatively complex calculation, going
back to Kronecker $~$1890).
\vskip 0.2cm

\no (a). \ Using the identity \s$\la^{-s}=\Gamma(s)^{-1}\int_0^\infty
t^{s-1}\s\ee^{-\la\m t}\s dt\s$ show that \s$\zeta(0)=-{_1\over^2}$
annd \s$\zeta(-1)=-{_1\over^{12}}\s$
where $\zeta$ is the Riemann zeta function \s$\zeta(s)=
\sum\limits_{n=1}^\infty n^{-s}\s$ (for \s${\rm Re}\m s>1$,
analytically continued elsewhere).
\vskip 0.2cm

\no (b). \ For $\tau=\tau_1+i\tau_2$ with $\tau_i$ real, $\tau_2>0$
show using the identity from (a) and the Poisson resummation
that for ${\rm Re}\m s$ sufficiently large
\qq
\sum\limits_{n=-\infty}^\infty|\tau+n|^{-2s}\ =\
{\sqrt{_\pi}\over^{\Gamma(s)}}\left(\sum\limits_{n\not=0}
\ee^{\m2\pi in\tau_1}\int_0^\infty t^{s-3/2}\s\m\ee^{-\tau_2^2 t
-\pi^2n^2/t}\s\s dt\ +\ \tau_2^{-2s+1}\s\Gamma(s-1/2)\right)\s.
\non
\qqq
Note that the right hand side is analytic in $s$ around $s=0$.
Using (easy) relations \s$\Gamma(s)^{-1}=s+\CO(s^2)\s,$\ \s$
\Gamma(-\m{_1\over^2})=-2\sqrt{\pi}\s$ and
\s$\int_0^\infty t^{-3/2}\s\ee^{-x(t+t^{-1})}\s dt\s=\s\sqrt{\pi}\s
x^{-1/2}\s\ee^{-2x}\s$ obtain:
\qq
{_d\over^{ds}}\bigg|_{_{s=0}}\sum\limits_n|\tau+n|^{-2s}\ =\ -\s
\ln{|1-q|^2}\s-\s 2\pi\tau_2
\non
\qqq
with the standard notation \s$q\equiv\ee^{\m 2\pi i\tau}\s$.
\vskip 0.1cm

\no (c). \ By taking $\tau\to 0$ in the last formula show that
 \s$\zeta'(0)=-\m{_1\over^2}\s\ln(2\pi)\s$.
\vskip 0.2cm

\no (d). \ Prove that for the periodic b.c. operator
\s${{d^2}\over{dx^2}}\s$ on $[0,L]$ the zeta-regularized
determinant
\qq
{\rm det}'(-\m{_\beta\over^{2\pi}}\s{_{d^2}\over^{dx^2}})\s=\s2\pi L^2/\beta\s.
\non
\qqq
\vskip 0.2cm

\no (e). \ Show that the spectrum of the Laplacian $\Delta_\tau$
on the torus $\NC/(\NZ+\tau\NZ)$ in the metric $|dz|^2$ is given
by \s$\la_{m,n}=-({2\pi\over\tau_2})^2\s|\tau m+n|^2\s$ for $n,m\in\NZ$.
\vskip 0.2cm

\no (f). \ Proceeding as in (b) decompose
\qq
\sum\limits_{(m,n)\not=(0,0)}|\tau m+n|^{-2s}\s=\s
\sum\limits_{m\not=0,\ n}|\tau m+n|^{-2s}\s+\s 2\m\zeta(2s)\hs{3cm}\cr
=\s{_{\sqrt{\pi}}\over^{\Gamma(s)}}\left(\sum\limits_{m,n\not=0}
\ee^{\m 2\pi imn\tau_1}\int_0^\infty t^{s-3/2}\s
\m\ee^{-m^2\tau_2^2 t-\pi^2 n^2/t}\s\m dt\s+\s\sum\limits_{m\not=0}
m^{-2s+1}\s\tau_2^{-2s+1}\s\Gamma(s-1/2)\right)\s+\s 2\m\zeta(2s)
\non
\qqq
and show that (after analytic continuation)
\qq
\zeta_{-\Delta'_\tau}(s)\s\equiv\s
\sum\limits_{(m,n)\not=(0,0)}(-\la_{m,n})^{-s}\s=\s
-1\s-\s2s\s\ln{|\prod\limits_{m=1}^\infty(1-q^m)|^2}
\s-2s\s\ln{\tau_2}\s+\s{_1\over^3}\m\pi\s\tau_2\m s\s.
\non
\qqq
Infer that
\qq
\zeta_{-\Delta'_\tau}(0)\s=\s-1\s,\quad\quad\zeta'_{-\Delta'_\tau}(0)=
-2\s\ln{|\prod\limits_{m=1}^\infty(1-q^m)|^2}
\s-2\s\ln{\tau_2}\s+\s{_1\over^3}\m\pi\s\tau_2\s,&\label{2}
\non
\qqq
and that
\qq
{\det}'(-\Delta_\tau)\s=\s\tau_2^2\s|\eta(\tau)|^4\label{3}
\non
\qqq
where the Dedekind eta function \s$\eta(\tau)=q^{\m 1/24}
\m\prod\limits_{n=1}^\infty(1-q^n)\s$.
\vskip 0.5cm

\no Hence in the genus 1 case,
\qq
Z\ \equiv\ Z_\beta(\tau)\ =\ \vartheta_{_{Q_\beta}}(\tau,\bar\tau)
\ |\eta(\tau)|^{-2}\ =\ Z_{1/\beta}(\tau)\s.
\label{genu1}
\qqq
The marking independence (together with the independence
of $Z_\beta$ on the normalization of the flat metric on $T_\tau$,
see below) implies that $Z(\tau)$ is a {\bf modular invariant}
function
\qq
Z_\beta(\tau)\ =\ Z_\beta({_{a\tau+b}\over^{c\tau+d}})
\non
\qqq
for \s$(\matrix{a&b\cr c&d})\s\in\s SL_2(\NZ)\s$.
\vskip 1cm



\no{\bf 5.\ \s Toroidal compactifications: the partition functions}
\vskip 0.4cm

\no The above discussions may be easily generalized to the case
of ``toroidal compactifications'' i.e. to the case of massless
free field on $(\Sigma,\gamma)$ with values in the $N$-dimensional
torus $T^N=(\NR/2\pi\NZ)^N$. \s Fix a constant metric
$g\m=\sum_{ij}g_{ij}\m d\phi^i d\phi^j$ and a constant 2-form
$\omega\m=\sum_{ij}b_{ij}\m d\phi^i\wedge d\phi^j$ on $T^N$ and define
the classical action of the field $\phi:\Sigma\rightarrow T^N$
as
\qq
S(\phi)\ =\ {_1\over^{4\pi}}\s(\m\Vert d\phi\Vert^{\m2}_{_{L^2}}\s+\s
i\int_{_\Sigma}\phi^*\omega\m)\s.
\label{act}
\qqq
Applying the same method as before (do it!) results in the formula
\qq
Z\ \equiv\ Z_{d}\ =\ \vartheta_{_{Q_d}}(\tau,\bar\tau)\s
\left({_{{\rm vol}_\Sigma\s\s\det{\tau_2}}
\over^{\det'(-\Delta)}}\right)^{{\hs{-0.06cm}N/2}}\ =\ Z_{d^{-1}}
\non
\qqq
($T$-duality again!) where $d=(d_{ij}=g_{ij}+b_{ij})$ and the lattice
\qq
Q_d\s=\s Q_{d^{-1}}\s=\s\{\ (\s{_{dm+n}\over^{\sqrt{2}}}\s,
\s{_{d^tm-n}\over^{\sqrt{2}}})\ |\ m,n\in\NZ^N\ \}\ \subset\
\NR^N\oplus\NR^N
\non
\qqq
is an even self-dual lattice in $\NR^N\oplus\NR^N$ with the indefinite
scalar product $|(x,y)|^2=(x,\m g^{-1}x)-(y,\m g^{-1}y)$.
At genus 1
\qq
Z\ \equiv\ Z_d(\tau)\ =\ \vartheta_{_{Q_d}}(\tau,\bar\tau)
\s\s|\eta(\tau)|^{-2N}
\ =\ Z_{d^{-1}}(\tau)\ =\ Z_d({_{a\tau+b}\over^{c\tau+d}})\s.
\non
\qqq
\vskip 0.5cm

\no{\bf Example 4}. \ Let $T$ be the Cartan torus of a simply-laced,
simple, simply-connected Lie group (the compact form of the $A,\ D,\ E$
groups). By spanning the Lie algebra of $T$ by the coroots $\alpha^\vee_i$,
we may identify $T$ with $T^N$ where $N$ is the rank of the group.
Let $g_{ij}={_1\over^2}\s\tr\s\m\alpha^\vee_i\alpha^\vee_j$ where $\tr$
is the Killing form normalized so that $g_{ii}=1$. We may write
\qq
2\m g_{ij}=d_{ij}+d_{ji}
\non
\qqq
for some integers $d_{ij}$ and set
\qq
2\m b_{ij}=d_{ij}-d_{ji}
\non
\qqq
so that $d_{ij}=g_{ij}+b_{ij}$. The corresponding action of the
toroidal compactification coincides (mod $2\pi i$) with
the action of the WZW model with fields taking values in the
corresponding simple group (the $\sim\int\phi^*\omega$ term
is the remnant of the topological WZ term). Defining for
$p^\vee\in(P^\vee)^{h_\Sigma}$, where $P^\vee$ is the coweight
lattice dual to the root lattice,
\qq
\vartheta_{{_{Q^\vee,p^\vee}}}(\tau)\ =\ \sum\limits_{q^\vee\s\in\s
(Q^\vee)^{h_\Sigma}}\ee^{\m\pi i\s\tr\s\s(p^\vee+q^\vee)^t,\s\tau\m(
p^\vee+q^\vee)}
\non
\qqq
one obtains
\qq
\vartheta_{_{Q_b}}(\tau,\bar\tau)\ =\ \sum\limits_{[p^\vee]\in(P^\vee
/Q^\vee)^{h_\Sigma}}|\vartheta_{_{Q^\vee,p^\vee}}(\tau)|^2\s.
\non
\qqq
In particular at genus 1.
\qq
Z_d(\tau)\ =\ \sum\limits_{[p^\vee]\in P^\vee/
Q^\vee}\bigg|{\vartheta_{_{Q^\vee,p^\vee}}(\tau)\over\eta(\tau)^N}\bigg|^2
\s=\s\sum\limits_{[p^\vee]\in P^\vee/Q^\vee}|\ch^1_{[p^\vee]}(\tau)|^2
\label{ADE}
\qqq
where $\s\ch^1_{[p^\vee]}(\tau)=
{\vartheta_{_{Q^\vee,p^\vee}}(\tau)\over\eta(\tau)^N}\s$ runs through
the characters of the level 1 representations of the corresponding
Kac-Moody algebra. In particular, for the $E_8$ case $P^\vee=Q^\vee$
and the partition function is the absolute value squared
of $\ch^1_0(\tau)$ which is a cubic root of the modular
invariant function $j(\tau)$. In general, the right hand side of
eq.\s\s(\ref{ADE}) coincides with
the genus 1 partition function of the level 1 WZW model.
This remains true for higher genera and for the complete
CFT's which is another miraculous coincidence
of field theories with fields taking values in quite different
target spaces (e.g. the $SU(2)$ WZW model at level 1 is equivalent
to the free field with values in $S^1$ of radius 1).
\vskip 0.3cm

The fact that the toroidal partition function
(\ref{ADE}) is a finite sesqui-linear combination of expressions
holomorphic in $\tau$ is a characteristic feature
of rational conformal theories.
\vskip 0.5cm

\no{\bf Problem 5}. \ Show that the free field compactified on a
circle of rational radius squared ($=\s\beta$) is rational
in the above sense.
\vskip 0.5cm

\no For free fields with values in the Cartan tori of simply laced
groups described above, the general partition functions
are hermitian squares with respect to Quillen-like metric
of holomorphic sections of projectively flat vector bundles
over the moduli spaces of curves. We shall return to these issues
during a more detailed discussion of the WZW models.

\vskip 0.5cm

\no{\bf Example 5}. \ Consider the toroidal compactification
to $T^2$ equipped with the complex structure induced by the
complex variable $\psi=\phi^1+T\phi^2$ ($T$ is in the
upper half plane) and with a constant K\"{a}hler metric
$g\m=\m(R_2/T_2)\m d\psi\m d\bar\psi$ with $R_2>0$ and
a constant 2-form $\omega\m=\m i(R_1/T_2)\s d\psi\wedge
d\bar\psi$. Set $R=R_1+iR_2$. The partition function
of the corresponding free field is
\qq
Z\ \equiv\ Z_{R,T}\ =\ \vartheta_{_{Q_{R,T}}}(\tau,\bar\tau)\s
\left({_{{\rm vol}_\Sigma\s\s\det\s\tau_2}\over^{{\det}'(-\Delta)}}
\right)
\non
\qqq
where
\qq
Q_{R,T}\ =\ \{\ (\s{_{R\m m^1+TR\m m^2+T\m n^1-n^2}\over^{\sqrt{
2R_2T_2}}}\s,\s\m{_{\bar R\m m^1+T\bar R\m m^2+T\m n^1-n^2}\over^{\sqrt{
2R_2T_2}}}\s)\ |\ m^i,\m n^i\in\NZ\ \}\ \subset\ \NC\oplus\NC
\non
\qqq
with the indefinite quadratic form $|(z_1,z_2)|^2=|z_1|^2-|z_2|^2$.
Note that $Q_{T.R}$ may be obtained from $Q_{R,T}$ by complex
conjugation on the second $\NC$. We infer that
\qq
Z_{R,T}\ =\ Z_{T,R}
\label{ms}
\qqq
which is the simplest instance of {\bf mirror symmetry} claiming
identity of CFT's with fields in two
different Calabi-Yau manifolds with the role of modular
parameters of complex and (polarized) K\"{a}hler structures
interchanged.
\vskip 0.3cm

\no{\bf Problem 6}. \ Show that, besides the relation (\ref{ms}),
the partition function satisfies the identities
\qq
Z_{R,T}(\tau)\s=\s Z_{R+1,T}(\tau)\s=\s Z_{R, T+1}(\tau)\s
=\s Z_{R,-T^{-1}}(\tau)\s=\s Z_{-R^{-1},-T^{-1}}(\tau)\s,
\non
\qqq
which imply the separate $SL_2(\NZ)$ invariance in $R$ and $T$.
\vskip 0.5cm

\no In general, the moduli space of $N$-dimensional
toroidal compactifications is a double coset
\qq
O(N)\times O(N)\s\bigg\backslash\s O(\m\NR^N\oplus\NR^N,({_{
{I}}\atop^0}\s\m{_0\atop^{-{I}}})\s\m;\m\s\NR^N\oplus\NR^N,
({_0\atop^{I}}\s\m{_{I}\atop^0})\m)\s\bigg/\s
O(\NR^N\oplus\NR^N,({_0\atop^{I}}\s\m{_{I}\atop^0})\s
|\s\NZ\m)
\non
\qqq
(in a, hopefully, self-explanatory notations) which coincides with
the moduli of even self-dual lattices in $\NR\oplus\NR$ with the indefinite
scalar product.
\vskip 0.4cm

The action functional (\ref{act}) of the (compactified)
two-dimensional massless free field uses only the conformal class
of the metric $\gamma$ on $\Sigma$. The regularization
of the free field determinants reintroduces however the dependence
on the conformal factor of the metric, an effect called
{\bf conformal anomaly}. More exactly, one has
\qq
{_{\delta}\over^{\delta\sigma(x)}}\bigg|_{_{\sigma=0}}
\ln{\left({_{{\det}'(-\Delta)}\over^{{\rm vol}_\Sigma}}\right)}
\ =\ -\m{{_1\over^{12\pi}}}\s\m r(x)\s,
\label{coa}
\qqq
where $r$ is the scalar curvature of $\Sigma$, or, denoting
the metric dependence of the partition function by the subscript,
\qq
{_{\delta}\over^{\delta\sigma(x)}}\bigg|_{_{\sigma=0}}\s
Z_{\ee^\sigma\gamma}\ =\ {_N\over^{24\pi}}\s\m r(x)\ Z_\gamma\s.
\label{coa1}
\qqq
\vskip 0.3cm

\no{\bf Problem 7}. \ Prove the relation (\ref{coa})
using the identity \s$\zeta_{-\Delta}(s)={\Gamma(s)}^{-1}
\int_0^\infty t^{s-1}\s\m\tr\s\s\ee^{t\Delta}\s$
and the short time expansion of the heat kernel of $-\Delta$:
\qq
\ee^{\m t\Delta}(x,x)\s=\s{_1\over^{4\pi t}}
\s+\s{_1\over^{12\pi}}\s r(x)\s+\s\CO(t)
\non
\qqq
\vskip 0.3cm

\no{\bf Problem 8}. \ Prove that the infinitesimal relation
(\ref{coa1}) is equivalent to the global one
\qq
Z_{\ee^\sigma\gamma}\ =\ \ee^{\s{N\over 96\pi}\s(\m\Vert d\sigma\Vert^2_{L^2}
\s+\s4\int_\Sigma\sigma\m r\s dv\m)}\ Z_\gamma
\label{gcoa}
\qqq
\vskip 0.9cm



\no{\bf 6.\ \s Toroidal compactifications: the correlation functions}
\vskip 0.4cm

\no Besides the functional integrals for the  partition
functions, we would like to study the ones for the correlation
functions of the massless field with values in $S^1$ of the type
\qq
\int \prod\limits_{i=1}^n\ee^{\m i\m q_i\m\phi(x_i)}
\ \ee^{-{ \beta\over 4\pi}\int_\Sigma|d\phi|^2\m dv}\ D\phi\s,
\non
\qqq
for integer $q_i$, see Problem 3 for the $d=0$ example. This
may be attempted by the same strategy as before by separating
the field into the harmonic and univalued part,
as in eq.\s\s(\ref{sepa}), and then summing over the first
and integrating over the second. This gives the expression
\qq
\sum\limits_{\alpha\in H^1(\Sigma,2\pi\NZ)}
\ee^{-{\beta\over4\pi}\m\Vert\alpha_h\Vert^2_{L^2}\s
+\s i\sum_i\m q_i\m\phi_h(x_i)}\ \int\ee^{-{\beta\over4\pi}\s
(\psi,-\Delta\psi)_{L^2}\s
+\s i\sum_iq_i\m\psi(x_i)}\ D\psi\s.
\non
\qqq
The sum over $H^1$ may be expressed by partial Poisson
resummation as an explicit theta-function.
As for the functional integral, it may be formally performed
by mimicking the finite-dimensional formulae:
\qq
\int\ee^{-{\beta\over4\pi}\s
(\psi,-\Delta\psi)_{L^2}\s
+\s i\sum_iq_i\m\psi(x_i)}\ D\psi\
=\ \delta_{_{\sum_iq_i,\m 0}}\
\ee^{\m{\pi\over \beta}\sum_{i,j=1}^n q_iq_j\s G(x_i,x_j)}
\s\left({_{2\pi\s{\rm vol}_\Sigma}\over^{\det'(-{\beta
\over 2\pi}\Delta)}}\right)^{{\hs{-0.06cm}1/2}}
\non
\qqq
where the Kronecker delta is contributed by the integral
over the constant mode of $\psi$ and $G(x,y)=G(y,x)$ is a
Green function of $\Delta$ satisfying $\Delta_{x}G(x,y)=\delta(x,y)-
{1\over{\rm vol}_\Sigma}$
(the constant ambiguity should drop out above due to the
vanishing of $\sum q_i$). The obvious problem
with the above formula is that
\qq
G(x,y)\s=\s {_1\over^{2\pi}}\s\ln\m{{\rm dist}(x,y)}\
+\ {\rm finite}
\non
\qqq
when $y\to x$ so that $G(x,x)$ is not defined. This is a
standard problem with the short-distance singularities
due to distributional character of typical configurations
in the Gaussian free field measure. A possible treatment is
to renormalize the above expression by replacing
the divergent contributions by their finite parts
\qq
\tilde G(x,x)\s=\s \lim\limits_{y\to x}\s G(x,y)\s-\s
{_1\over^{2\pi}}\s\s\ln\m{{\rm dist}(x,y)}
\label{ren}
\qqq
Upon division by the partition function, all this leads to
a well defined renormalized expression for the correlation
functions which we shall denote by
\qq
\langle\s:\ee^{\m i\m q_1\m\phi(x_1)}:\s\cdots\s
:\ee^{\m i\m q_n\m\phi(x_n)}:\s\rangle_\gamma
\non
\qqq
where the colons remind the renormalization procedure
(which is closely related to the Wick ordering discussed
in Kazdan's lectures). The correlations depend on the
``charges'' $q_i$ and positions $x_i$ but also on the metric
$\gamma$ on $\Sigma$ which is signaled by the subscript.
In particular, on $\NC\NP^1$ with $H^1=0$, one may take
$G(x,y)={_1\over^{2\pi}}\s\ln{|z(x)-z(y)|}$ in the standard
complex variable and for the metric $g=\ee^{\sigma}\s dz\m d\bar z$
with a conformal factor $\ee^\sigma$, we obtain, setting
$z_i\equiv z(x_i)$,
\qq
\langle\s:\ee^{\m i\m q_1\m\phi(x_1)}:\s\cdots\s
:\ee^{\m i\m q_n\m\phi(x_n)}:\s\rangle_\gamma\ =\
\delta_{_{\sum_iq_i,\m 0}}\
\ee^{-\sum_i{q_i^2\over 4\beta}\m\sigma(x_i)}\s\prod\limits_{i<j}
|z_i-z_j)|^{^{q_iq_j\over\beta}}.
\label{cor1}
\qqq
The dependence on the conformal factor of the metric is solely
due to the renormalization (\ref{ren}) and persists in general:
\qq
\langle\s:\ee^{\m i\m q_1\m\phi(x_1)}:\s\cdots\s
:\ee^{\m i\m q_n\m\phi(x_n)}:\s\rangle_{\ee^{\sigma}\gamma}\ =\
\ee^{-\sum_i\Delta_i\m\sigma(x_i)}\s
\langle\s:\ee^{\m i\m q_1\m\phi(x_1)}:\s\cdots\s
:\ee^{\m i\m q_n\m\phi(x_n)}:\s\rangle_\gamma
\label{corf}
\qqq
where $\Delta_i={_{q_i^2}\over^{4\beta}}$ are the conformal
dimensions of the (Euclidean) fields
$:\ee^{\m i\m q_i\m\phi}:\s$.
\s\s The generalization to the toroidal compactifications
is straightforward. The conformal dimensions of fields
$:\ee^{\m i\m q\m\phi}:$
where $q\in\NZ^N$ is now ${_1\over^4}\m(q,g^{-1}q)$.
\vskip 0.5cm

The operator picture of the the free field compactified
on $S^1$ is as follows. The quantum Hibert space is
\qq
\CH\s=\s L^2(S^1,d\varphi_0)^\NZ\otimes\CF\otimes\tilde\CF\s.
\non
\qqq
Above $L^2(S^1,d\varphi_0)^\NZ$ is the infinite sum of
copies of $L^2(S^1)$, each labeled by an integer (``winding number'')
$w$. \s The Fock space $\CF$ is generated
by applying operators $\alpha_n$, $n=-1,-2,\dots,$ to a
vector annihilated by $\alpha_n$ with $n=1,2,\dots$,
\qq
[\alpha_n,\alpha_m]\s=\s n\m\delta_{n,-m}\s,
\quad\ \ \alpha_n^*=\alpha_{-n}\s,
\non
\qqq
and $\tilde\CF$ is another copy of $\CF$.
Let \s$|p,w\rangle\s$ denote the function \s${1\over^{\sqrt{2\pi}}}
\s\ee^{\m ipx}\s$ in the $w^{\m\rm th}$ copy of
\s$L^2(S^1,d\varphi_0)^\NZ$. \s Integer $p$ is the eigenvalue
of the operator \s$p_0={1\over{i}}\m{d\over d\varphi_0}\m$.
\s$\CH$ may be generated by applying operators
\s$\alpha_n,\s\tilde\alpha_n\s$ with negative $n$ to
vectors \s$|p,w\rangle\m$. \s The (multivalued)
free field operator (with time dependence) is given by
\qq
\varphi(t,x)\s=\s\varphi_0\s+\s\beta^{-1} p_0\m t\s+\s w\m x
\s+\s{_i\over^{\sqrt{2\beta}}}
\sum\limits_{n\not=0}(\s{_{\alpha_n}\over^n}\s
\ee^{-i(t+x)n}\s-\s{_{\tilde\alpha_n}\over^n}\s\ee^{-i(t-x)n})\s.
\non
\qqq
The relation to the
$m\to0$ limit of the massive free field on periodic interval
of length $L=2\pi$ given
by eq.\s\s(\ref{mff}) should be evident. Modulo the constant
and winding modes, for $n=1,2,\dots,$
\qq
\alpha_n=-i\sqrt{n}\s a_{-n}\s,\quad\alpha_{-n}=i\sqrt{n}\s a_{-n}^*\s,
\quad\tilde\alpha_n=i\sqrt{n}\s a_{n}\s,\quad\tilde\alpha_{-n}=-i
\sqrt{n}\s a_{n}^{\s*}\s.
\non
\qqq
The relabeling allows to separate the left-moving part (involving
$\alpha_n$) from the right-moving one (containing $\tilde\alpha$).
The one-handed parts of $\varphi(t,x)$ are called chiral
fields (the zero modes can be separated too).
For the uncompactified massless free field, the $\Nk=0$ mode
contributes to the Hamiltonian the term $\sim\m-d^2/d\varphi_0^2$
acting on $L^2(\NR,d\varphi_0)$ which has continuous spectrum.
The compactification of the field (and consequently of its zero mode),
restores the discreteness of the energy spectrum.
\vskip 0.3cm

Hilbert space $\CH$ carries a representation of two
commuting copies of the Virasoro algebra with generators
$L_n$ and $\tilde L_n$. Explicitly,
\qq
L_n\s=\s{_1\over^2}\s\sum\limits_{m\in\NZ}:\alpha_m\alpha_{n-m}:
\non
\qqq
(creator to the left of annihilators)
and similarly for $\tilde L_n$ where we have set $ \alpha_0=
{1\over\sqrt{2}}\m(\beta^{1/2}w+\beta^{-1/2}p_0)\s$ and $\tilde\alpha_0=
{1\over\sqrt{2}}\m(\beta^{1/2}w-\beta^{-1/2}p_0)\s$\s.
The quantum Hamiltonian $H=L_0+\tilde L_0\m$. \s$P=L_0-\tilde L_0\s$
generates the space translations. \s$\Omega=|0,0\rangle\s$ is
the ground state of $H$ (it is also annihilated by $P$).
\vskip 0.5cm

\no{\bf Problem 9}. \ Show that $L_n$'s indeed satisfy the Virasoro
algebra relations (with unit central charge)
\qq
[L_n,\m L_m]\s=\s(n-m)\s L_{n+m}\s\s+\s{_1\over^{12}}\s(n^3-n)\s\m
\delta_{n,-m}\s.
\non
\qqq
\vskip 0.5cm

Let us introduce the ``vertex operators''
\qq
V_q(t,x)\ =\ :\ee^{\m i\m q\m\varphi(t,x)}:
\non
\qqq
defined as the (formal) power series in $\alpha_n$ and $\tilde\alpha_n$
reordered by putting the creation operators with negative $n$ indices
to the left of the annihilation operators corresponding to
positive $n$ and also $\varphi_0$ operators to the left of $p_0$.
\vskip 0.5cm

\no{\bf Problem 10}. \ Using the operator relation
$\ee^A\m\ee^B=\ee^B\m\ee^A\m\ee^{[A,B]}$ holding if $[A,B]$ commutes
with $A$ and $B$, show that for $t_1<t_2<\cdots<t_n$,
\qq
(\s\Omega\s,\s\m V_{q_1}(it_1,x_1)\ \cdots\
V_{q_n}(it_n,x_n)\s\Omega\s)\ =\ \delta_{_{\sum_iq_i,\m 0}}\s\s
\prod\limits_{i<j}|z_i-z_j|^{^{q_iq_j\over\beta}}
\non
\qqq
where $z_j=\ee^{-t_j+ix_j}$. This, together with eq.\s\s(\ref{cor1}),
provides a spherical version of the Feynman-Kac formula.
\vskip 0.5cm

\no In variables $z=\ee^{\m i\m(t+x)}$ and $\tilde z=
\ee^{\m i\m(t-x)}$, the commutation relations of $L_n$'s and
$\tilde L_n$'s with the vertex operators take the form
\qq
&&[\m L_n\m,\s V_q(z,\tilde z)\m]\ =\ (n+1)\s\Delta\s\m z^n
\s\m V_q(z,\tilde z)\s+\ z^{n+1}\s\m\da_z\s\m
V_q(z,\tilde z)\s,\\
&&[\m \tilde L_n\m,\s V_q(z,\tilde z)\m]
\ =\ (n+1)\s\Delta\s\m {\tilde z}^n
\s\m V_q(z,\tilde z)\s+\ {\tilde z}^{n+1}
\m\s\da_{\tilde z}\s\m V_q(z,\tilde z)\s
\non
\qqq
where above $\Delta={_{q^2}\over^{4\beta}}$ or $\Delta={1\over^4}
\s(q,\m g^{-1}q)$ stands for the conformal
dimension of the operator. Later on, we shall see that these relations
essentially follow from eq.\s\s(\ref{corf}) and the general covariance
of the corresponding correlation functions.
In the professional jargon, the fields
satisfying such commutation relations are called primary Virasoro
operators.
\vskip 0.3cm

On the level of the Hilbert space $\CH\equiv\CH_\beta$,
$T$-duality becomes the unitary transformation
\s$U_T:\ \CH_\beta\s\rightarrow\s\CH_{1/\beta}\s$
such that
\qq
U_T\m|u,w\rangle\s=\s(-1)^{uw}\m|w,u\rangle
\quad\ {\rm and}\quad\ U_T\m\alpha_n\s=\s\alpha_n\m U_T\s,\quad
U_T\m\tilde\alpha_n\s=\s-\m\tilde\alpha_n\m U_T\s,\quad
\non
\qqq
where, for $u,w\in\NZ$,
\s$|u,w\rangle\s$ denotes the function \s${1\over{\sqrt{2\pi}}}
\s\ee^{\m iu\phi_0}\s$ in the $w$-component of
\s$L^2(S^1,d\phi_0)^\NZ\s$. \s$U_T$ intertwines the action
of the Virasoro algebras in $\CH_\beta$ and $\CH_{1/\beta}$
and maps the vertex operators in $\CH_\beta$ to new
operators which should be considered on the equal
footing with the original ones.
\vskip 0.2cm

Up to now, we have considered only $S^1$-valued free fields
with periodic boundary conditions. In string theory aplications,
one also considers fields on space with
boundaries and with fixed boundary conditions like the Neumann
ones (open strings). Quantizing such fields
on space which is an interval $[0,\pi]$ \s($d=1$) one
obtains quantum field
\qq
\varphi^N(t,x)\s=\s\varphi_0\s+\s\beta^{-1} p\m t\s+\s
{_i\over^{\sqrt{2\beta}}}
\sum\limits_{n\not=0}(\s{_{\alpha^N_n}\over^n}\s
\ee^{-i(t+x)n}\s+\s{_{\alpha^N_n}\over^n}\s\ee^{-i(t-x)n})
\non
\qqq
which may be realized in the subspace of the periodic b.c.
Hilbert space $\CH_\beta$ generated by applying operators
\s${1\over\sqrt{2}}(\alpha_n-\tilde\alpha_n)=\alpha^N_n\s$
with negative $n$ to vectors \s$|u,0\rangle\s$. \s The $T$-duality
maps this field into the one corresponding to the Dirichlet
boundary conditions
\qq
\varphi^D(t,x)\s=\s\varphi_0\s+\s w\m x
\s+\s{_i\over^{\sqrt{2\beta}}}\sum\limits_{n\not=0}(\s{_{\alpha^D_n}
\over^n}\s\ee^{-i(t+x)n}\s-\s{_{\alpha^D_n}\over^n}\s\ee^{-i(t-x)n})
\non
\qqq
where $\varphi_0$ is fixed modulo $\pi\NZ$. \s$\varphi^D(t,x)\s$
acts in the subspace of $\CH_{1/\beta}$ generated by applying
\s${1\over\sqrt{2}}(\alpha_n+\tilde\alpha_n)=\alpha^D_n\s$
to vectors \s$|0,w\rangle\s$. In toroidal compactifications
with more dimensions of the target, one may have mixed
''$D$-brane''-type boundary conditions with some coordinates
of the field fixed to prescribed values at the ends of
the space-interval $[0,\pi]$.
\vskip 0.5cm

\no{\bf Problem 11}\s\s\ (Massless fermions on Riemann surface).
\vskip 0.3cm

\no Let \s$(\Sigma,\gamma)\s$ be a Riemann surface.
Spin structure on \s$\Sigma\s$ may be identified with
the square root \s$L\s$ of the canonical
bundle $K={T^*}^{1,0}(\Sigma)\s$. \s A Dirac spinor
\s$\Psi=(\psi,\tilde\psi)\s$ is an element of
\s$\Gamma(L\oplus\bar L)\s$ where $\bar L$ is the bundle
complex conjugate to $L$. The conjugate spinor is
\s$\bar\Psi=(\tilde\chi,\chi)\in\Gamma(\bar L\oplus L)\s$
and in the euclidean Dirac theory it should be treated as an
independent field
($\chi=\psi,\ \tilde\chi=\tilde\psi\s$ for Majorana fermions).
Denote by \s$\bar\da_L\s$ the $\bar\da$ operator of $L$
and by \s$\da_{\bar L}\s$ its complex conjugate
which may be naturally identified with \s$\de_L^{\s*}\s$.
The action is a function on the odd vector space
\s$\Pi(\Gamma(L\oplus\bar L)\oplus\Gamma(\bar L\oplus L))\s$:
\qq
S(\Psi,\bar\Psi)\s=\s-\m{_1\over^\pi}\int_{_\Sigma}(\chi\bar\da_{L}
\psi\s+\s\tilde\chi\da_{\bar L}\tilde\psi)\s
\non
\qqq
(note that the integrand is naturally a 2-form).
Partition functions of the Dirac fermions are given by
the formal Berezin integral
\qq
Z_L\s=\s\int\ee^{-S(\Psi,\bar\Psi)}\s D\bar\Psi\s D\Psi
\s=\s\det(\da_{\bar L})\s\det(\bar\da_L)\s=\s\det(\bar\da_L^{\s*}
\bar\da_L)\s.
\non
\qqq
The last determinant may be
zeta-regularized giving a precise sense to the partition function
\s$Z_L\s$ of the Dirac field on $\Sigma$.
\vskip 0.2cm

On the elliptic curve \s$\NC/(\NZ+\tau\NZ)\s$ with $\tau$
in the upper half-plane,
the canonical bundle \s$K\s$ may be trivialized by the section \s$dz\s$
and spin structures correspond to the choice of periodic or
anti-periodic boundary conditions under
\s$z\to z+1\s$ and \s$z\to z+\tau\s$:
\qq
L\s=\s pp,\ pa,\ ap,\ aa\s.
\non
\qqq
\vskip 0.1cm

\no (a). \ Show that the eigenvalues of \s$\bar\da_L^{\s*}
\bar\da_L\s$ are
\qq
\lambda_{m,n}\s=\s({_\pi\over^{\tau_2}})^2\m|\tau m+n|^2
\non
\qqq
with
\qq
\hbox to 5cm{\hspace{2cm}$m\in\NZ\s,$\hfill}
\hbox to 2cm{$n\in\NZ$\hfill}
\quad\quad{\rm for}\quad L=pp\s,\cr
\hbox to 5cm{\hspace{2cm}$m\in\NZ\s,$\hfill}
\hbox to 2cm{$n\in\NZ+{_1\over^2}$\hfill}
\quad\quad{\rm for}\quad L=pa\s,\cr
\hbox to 5cm{\hspace{2cm}$m\in\NZ+{_1\over^2}\s,$\hfill}
\hbox to 2cm{$n\in\NZ$\hfill}
\quad\quad{\rm for}\quad L=ap\s,\cr
\hbox to 5cm{\hspace{2cm}$m\in\NZ+{_1\over^2}\s,$\hfill}
\hbox to 2cm{$n\in\NZ+{_1\over^2}$\hfill}
\quad\quad{\rm for}\quad L=aa\s.
\non
\qqq
\vskip 0.1cm

\no (b). \ Infer that
\qq
Z_{pp}(\tau)\s=\s 0\s.
\non
\qqq
\vskip 0.1cm

\no (c). \ Show that
\qq
\zeta_{\bar\da_{pa}^{\s*}\de_{pa}}(s)\s=\s2^{2s}\s(\m
\zeta_{-\Delta'_{2\tau}}(s)
\s-\s\zeta_{-\Delta'_\tau}(s)\m)\s.
\non
\qqq
Infer from Eq.\s\s(\ref{2}) that
\qq
Z_{pa}(\tau)\s=\s4\m|q^{\m 1/24}\prod\limits_{n=1}^\infty
(1+q^n)|^4\s.
\non
\qqq
In the Hilbert space picture
\qq
Z_{pa}(\tau)\s=\s\tr_{\CH_R\otimes\tilde\CH_R}
\s\s q^{L_0-1/24}\bar q^{\tilde L_0-1/24}\s.
\qqq
The ``Ramond sector'' Hilbert space
is \s$\CH_R\otimes\tilde\CH_R\s$ with
\qq
\CH_R\s=\s\NC^2\otimes\left(
\wedge(\mathop{\oplus}\limits_{n=1}^\infty\NC)
\right)^{\otimes 2}
\qqq
and \s$\tilde\CH_R\s$ is another copy of \s$\CH_R\s$.
\s$L_0$ acts in the first copy. It has eigenvalue
\s${1\over 8}\s$ on \s$\NC^2\s$ (the ``Ramond ground states'')
and the occupied $n^{\s\rm th}$ mode in the fermionic Fock space
adds \s$n\s$ to it.
\vskip 0.1cm

\no The periodic partition function is
\qq
Z_{pp}(\tau)\s=\s\tr_{\CH_R\otimes\tilde\CH_R}
\s\s(-1)^{F+\tilde F}\s\s q^{L_0-1/24}\bar q^{\tilde L_0-1/24}
\ \equiv\ {\rm str}_{\CH_R\otimes\tilde\CH_R}
\s\s\s\s q^{L_0-1/24}\bar q^{\tilde L_0-1/24}
\qqq
where \s$(1,0),(0,1)\in\NC^2\s$ correspond to the eigenvalues
\s$+1,-1\s$ of \s$(-1)^F\s$
and each occupied fermionic Fock space mode
adds \s$1\s$ to \s$F\s$. \s$Z_{pp}(\tau)\s$ vanishes since
modes with odd and even Fermi numbers are paired.
\vskip 0.2cm

\no (d). \ Show that
\qq
\zeta_{\bar\da_{ap}^{\s*}\de_{ap}}(s)\s=\s2^{4s}\m
\zeta_{-\Delta'_{\tau/2}}(s)
\s-\s2^{2s}\m\zeta_{-\Delta'_\tau}(s)\ \ \quad
{\rm and}\quad\ \
Z_{ap}(\tau)\s=\s|q^{-1/48}\prod\limits_{n=0}^\infty
(1-q^{n+1/2})|^4\s.
\non
\qqq
The Hilbert space interpretation is
\qq
Z_{ap}(\tau)\s={\rm str}_{\CH_{NS}\otimes\tilde\CH_{NS}}
\s\s\s\s q^{L_0-1/24}\bar q^{\tilde L_0-1/24}
\qqq
where the ``Neveu-Schwarz sector'' Hilbert space is
\qq
\CH_{NS}\s=\s\left(
\wedge(\mathop{\oplus}\limits_{n=0}^\infty\NC)
\right)^{\otimes 2}\s.
\qqq
The ``Neveu-Schwarz ground state'' has eigenvalue zero of $L_0$
and the \s$n^{\s\rm th}\s$ occupied zero mode contributes
\s$(n+{1\over2})\s$ to it. The fermion number of the NS-ground state
vanishes and each occupied fermionic mode adds \s$1\s$ to it.
\vskip 0.2cm

\no (e). \ Show that
\qq
\zeta_{\bar\da_{aa}^{\s*}\de_{aa}}(s)\s=\s2^{2s}\m
\zeta_{\bar\da_{pa}^{\s*}\de_{pa}}(s)\bigg|_{_{\tau/2}}
\s-\s\m\zeta_{\bar\da_{pa}^{\s*}\de_{pa}}(s)\bigg|_{_{\tau}}\ \
\quad{\rm and}\quad\ \
Z_{aa}(\tau)\s=\s|q^{-1/48}\prod\limits_{n=0}^\infty
(1+q^{n+1/2})|^4\s
\non
\qqq
and that
\qq
Z_{aa}(\tau)\s={\rm tr}_{\CH_{NS}\otimes\tilde\CH_{NS}}
\s\s\s\s q^{L_0-1/24}\bar q^{\tilde L_0-1/24}\s.
\qqq
\vskip 0.1cm

\no (f). \ Prove the modular properties:
\qq
Z_{pa}(\tau+1)\s\m=\s Z_{pa}(\tau)\s,\quad\quad Z_{ap}(\tau+1)\m\s=\s
Z_{aa}(\tau)\s,\quad\quad Z_{aa}(\tau+1)\s=\s Z_{ap}(\tau)\s,\cr
Z_{pa}(-1/\tau)\s=\s Z_{ap}(\tau)\s,\quad\quad Z_{ap}(-1/\tau)\s=\s
Z_{pa}(\tau)\s,\quad\quad Z_{aa}(-1/\tau)\s=\s Z_{aa}(\tau)\s.
\non
\qqq
\eject
\vskip 0.5cm

\no{\bf Problem 12}\s\s\ {{Bosonization}}.
\vskip 0.3cm

\no
The spin structure is called even (odd)
if the dimension of the kernel of $\de_L$ is even (odd).
Denote by $\sigma(L)\s$ the parity of $L$.
The bosonization formula asserts that
\qq
{_1\over^2}
\sum\limits_L(-1)^{\,\sigma(L)}\s\s Z_L\s=\s C^{h_{_\Sigma}-1}\s Z_{1/2}
\label{4}
\qqq
where on the right hand side we have the partition function
of the bosonic free field with values in the circle of radius
squared $1\over 2$, $C$ is a constant and \s$h_{_\Sigma}$ the genus of
the Riemann surface $\Sigma$. These equalities extend to correlations.
For example, the fermionic fields \s$(\psi\tilde\psi)(x)\s$
correspond to bosonic fields \s$:\ee^{\m i\phi(x)}:\s$
and \s$(\tilde\chi\chi)(x)\s$ to \s$:\ee^{-i\phi(x)}:\s$.
What are their conformal weights?
Prove identity (\ref{4}) for \s$\Sigma=\NC/(\NZ+\tau\NZ)\s$
using the expression (\ref{genu1}) for \s$Z_{1/2}(\tau)\s$ and
the classical product expressions for the theta functions
\qq
\vartheta(z|\tau)\equiv\sum\limits_{n\in\NZ}\ee^{\m\pi i\tau n^2\s+\s
2\pi in z}\s=\s\prod\limits_{n=1}^\infty(1-q^n)\s(1+\ee^{\m 2\pi i z}
\s q^{\m n-1/2})\s(1+\ee^{-2\pi i z}\m q^{\m n-1/2})\s.
\non
\qqq
What is the Hilbert space interpretation of the left hand side
of Eq.\s\s(\ref{4}) on the elliptic curve?
\vskip 0.5cm

\no{\bf In summary}, by ``calculating'' the functional integrals
for compactified massless free fields we have constructed models
of two-dimensional CFT specified
by giving the partition functions and correlation functions
on general Riemann surfaces. In the next lecture(s), we shall
examine the emerging CFT structure on a more abstract level.
\vskip 0.8cm

\no{\bf References}
\vskip 0.4cm

A set of Gaussian integration formulae may be found
e.g. in "Quantum Field Theory and Critical Phenomena"
by J. Zinn-Justin (Sects. 1.1 and 1.2). See also Sects. 2.0-2.2
and 2.5 for the discussion  of of the path integral and
the Feynman-Kac formula. Of course, for the latter topic,
"Quantum Mechanics and Path Integral" by Feynman-Hibbs is
the physics classic. Rigorous theory of infinite-dimensional
Gaussian integrals may be found e.g. in the 4$^{\m\rm th}$
volume of Gelfand-Vilenkin. See also Simon's "The Euclidean
$P(\phi)_2$ Quantum Field Theory".
\vskip 0.3cm

The free fields with values in $S^1$ are discussed
briefly e.g. in the Ginspargs contributions to Les Houches 1988 School
(Session XLIV "Fields, Strings and Critical Phenomena", eds.
Brezin-Zinn-Justin) or in Drouffe-Itzykson:
"Th\'{e}orie Statistique des Champs", InterEditions 1989,
(Sects. 3.2 and 3.6). \m For the case of fields
with values in a torus see the paper by Narain-Sarmadi-Wittem
in Nucl. Phys. B 279 (1987) p. 369 and for the case of complex torus
read Vafa's contribution to "Essays on Mirror Symmetry", ed. S.-T. Yau,
International Press, Hong Kong 1992.
\vskip 0.3cm

Free fermions and bosonization on a Riemann surface are discussed
in the paper by Alvarez-Gaum\'{e}-Bost-Moore-Nelson-Vafa
in Commun. Math. Phys. 112 (1987), p. 503.

\eject



\
\vskip 0.8cm

\noindent{\large{\bf Lecture 2.\ \ {Axiomatic
approaches to conformal field theory}}}
\addtocounter{equation}{-27}
\vskip 0.8cm


\no\un{Contents\s}:
\vskip 0.5cm

\no 1. \ Conformal field theory data

\no 2. \ Conformal Ward identities

\no 3. \ Physical positivity and Hilbert space picture

\no 4. \ Virasoro algebra and its primary fields

\no 5. \ Highest weight representations of $Vir$

\no 6. \ Segal's axioms and vertex operator algebras
\vskip 1.5cm


\no{\bf 1.\ \s Conformal field theory data}
\vskip 0.4cm

\no In the first lecture, we have discussed
a functional-integral construction of the simplest models of CFT:
the toroidal compactifications (of massless free fields).
In this lecture we shall present a more general
approach to CFT which, although not overly formalized,
will be axiomatic in spirit using only the most general
properties of the free field models. We shall assume that
the basic data of a CFT model specify for each compact
Riemann surface $(\Sigma,\gamma)$ its partition function
\s$Z_\gamma>0\s$ and a set of its correlation
functions \s$\langle\s\phi_{l_1}(x_1)\s\cdots\s\phi_{l_n}(x_n)
\s\rangle_{\gamma}\s$ of the ``primary fields'' from a fixed
set \s$\{\phi_l\}\s$. The correlation functions are
symmetric in the pairs of arguments \s$(x_i,l_i)\s$,
are defined for non-coincident insertion points
\s$x_i\in\Sigma\s$ and are assumed smooth. We shall also need
later some knowledge of their short distance singularities.
The dependence of both the partition and the correlation
functions on the Riemannian metric \s$\gamma\s$
will be assumed regular enough to assure existence
of distributional functional derivatives of arbitrary order.
The basic hypothesis are the following symmetry properties:
\renewcommand{\labelenumi}{(\bf\roman{enumi})}
\begin{enumerate}
\item {\bf diffeomorphism covariance}
\qq
  Z_{\gamma}&=&Z_{D^*\gamma}\ ,\label{PFDI}\\
\langle\s\phi_{l_1}(D(x_1))\s\cdots\s\phi_{l_n}(D(x_n))
\s\rangle{_{\gamma}} &=&
\langle\s\phi_{l_1}(x_1)\s\cdots\s\phi_{l_n}
(x_n)\s\rangle_{D^*\gamma} \ ,\label{GFDI}
\qqq
\item {\bf local scale covariance}
      \qq
&\hspace{1.5cm}Z_{\ee^{\sigma}\gamma}\s=\s\ee^{\m\frac{{c}}{{96\pi}}\s(
\Vert d\sigma\Vert^2_{L^2}\s+\s4\int_\Sigma\sigma\s r\s dv)}
\s\s Z_\gamma\ ,&\label{PFWI}\\
&\langle\s\phi_{l_1}(x_1)\s\cdots\s\phi_{l_n}(x_n)
\s\rangle_{_{\ee^\sigma \gamma}}\s=\s
\prod\limits_{i=1}^n\ee^{-\Delta_{l_i}\m\sigma(x_i)}
\s\s\langle\s\phi_{l_1}(x_1)\s\cdots\s\phi_{l_n}(x_n)
\s\rangle_{_\gamma}\ \s&\label{GFWI} \qqq
\end{enumerate}
where \s$c\s$ is the central charge of the theory.
In \s${\it{({\bf i})}}\s$, \s we limit
ourselves to orientation preserving
diffeomorphism assuming that under the change of orientation
of the surface,
\qq
&\m Z_\gamma\m\mapsto\m Z_\gamma\ ,&\\
&\s\s\s\langle\s\phi_{l_1}(x_1)\s\cdots\s\phi_{l_n}(x_n)
\s\rangle{_\gamma}\s\mapsto\s{\overline{\langle
\s\phi_{\m\bar l_1}(x_1)\s
\cdots\s\phi_{\m\bar l_n}(x_n)
\s\rangle{_\gamma}}}\ ,&
\label{OrCh}
\qqq
where \s$\phi_l\mapsto\phi_{\bar l}\s$
is an involution of the set of primary
fields preserving their conformal weights
($:\ee^{\m iq\phi}:\s\s\mapsto\s\s:\ee^{-iq\phi}:\s$
for the toroidal compactifications).
In what follows we shall first explore the implications of the
above identities which we shall, jointly,
call {\bf conformal symmetries}.
Other important properties of the correlation functions,
for example those responsible for the Hilbert space
interpretation of the theory, will be introduced and analyzed later.
\vs 0.4cm

Let us define new correlation functions with
insertions of {\bf energy-momentum tensor}\footnote{called also
stress tensor in a static view
of the Euclidean field theory} by setting
\qq
   &\langle\s T_{\mu_1 \nu_1}(y_1)\s\cdots\s T_{\mu_m
\nu_m}(y_m)\s\m \phi_{l_1}(x_1)\s
\cdots\phi_{l_n}(x_n)\s\rangle{_\gamma}&\cr\cr\cr
&=\s{Z_\gamma}^{\hs{-0.03cm}-1}\s\m{(4\pi)^m\s\delta^{m}\over\delta
\gamma^{\mu_1\nu_1}(y_1) \s\cdots\s\delta \gamma^{\mu_m\nu_m}(y_m)}\s\s
Z_\ga\s\m\langle\s\phi_{l_1}
(x_1)\s\s\cdots\s\s\phi_{l_n}(x_n)\m\rangle{_\gamma} \ ,
\label{EMT}
\qqq
where $\gamma^{\mu\nu}\m\da_\mu\da_\nu\equiv \ga^{-1}$
is the inverse metric. In complex coordinates,
energy-momentum tensor has the components
\qq
  T_{zz}\s=\s{\overline{T_{\bar{z} \bar{z}}}}\ \ \ \ \ {\rm and}\ \ \
\ \ T_{z \bar{z}}\s=\s T_{\bar{z}z }\s=\s{\overline{T_{z \bar{z}}}}\ .
\non
\qqq
By definition, the correlation functions \s\s$\langle\s T_{\mu_1 \nu_1}
(y_1)\s\cdots\s T_{\mu_m
\nu_m}(y_m)\s\m \phi_{l_1}(x_1)\s
\cdots\phi_{l_n}(x_n)\s\rangle{_\gamma}\s\s$ are
distributions in their dependence
on \s$y_1,\s\dots,\s y_m\s$. As we shall see
below, they are given by smooth functions for
non-coincident arguments and away from \s$x_i$'s,
but we shall also have to study their distributional behavior
at coinciding points.
\vs 1cm



\no{\bf 2.\ \s Conformal Ward identities}
\vskip 0.4cm

\no Symmetries in QFT are expressed as {\bf Ward identities} between
correlation functions. Eqs.\s\s(\ref{GFDI}) and (\ref{GFWI})
are examples of such
relations for group-like conformal symmetries.  It is often useful
to work out also Ward identities corresponding to infinitesimal,
Lie algebra version of symmetries.  We shall
do this here for the infinitesimal
conformal symmetries.  The resulting formalism was the starting point
of the 1984 Belavin-Polyakov-Zamolodchikov's paper.
The approach presented here
is close in spirit to the 1987 article by Eguchi-Ooguri
(to some extend also to Friedan's 1982 Les Houches lecture notes).
The general strategy is to expand the global symmetry identities
to the second order in infinitesimal symmetries.
This will be a little bit technical so you might
wish to see first the results listed at the end of
this section.
\vs 0.4cm

Let us start by exploring the infinitesimal version of the local
scale covariance (\ref{PFWI}).  Using the definition (\ref{EMT}),
we obtain the relation
\qq
      4\pi\s Z_\gamma^{-1}\s\frac{_{\delta}} {^{\delta
\sigma}}\bigg|_{\sigma=0} Z_{e^{\sigma}\ga}\s=\s -\ga^{zz}\s\m \langle\m
T_{zz}\m\rangle_{\ga}\m -\m 2\m \ga^{z\bar{z}}\m\s\langle\m T_{z
\bar{z}}\m\rangle_{\ga}\m -\m \ga^{\bar{z}\m\bar{z}}\m\s \langle\m
T_{\bar{z}\m\bar{z}}\m\rangle_{\ga}\s=\s \frac{_c}{^6}\s r\s\m .
\label{hallo7}
\qqq
Note that if \s$\ga=|dz|^2\s$ then
\s$\ga_{zz}=\ga_{\bar z\m\bar z}=
\ga^{zz}=\ga^{\bar z\m\bar z}=0\s$, \s$\ga_{z\bar z}={1\over 2}\s$ and
\s$\ga^{z\bar z}=2\s$. \s Besides, the scalar curvature of \s$\ga\s$
vanishes.  In such a metric, eq.\s\s(\ref{hallo7}) reduces to
the equality
\qq
\langle\m T_{z\bar z}\m\rangle\s=\s0
\label{Traceless0}
\qqq
which states that
energy-momentum tensor in a CFT is traceless (in the flat metric,
\s$\tr\s\m T_{\mu\nu}\s=\s4\m T_{z\bar z}\s$)\m.
It is the first example of Ward identities expressing the
infinitesimal conformal invariance on the quantum level.  We shall see
further identities of this type below.
\vs 0.4cm

Notice that if \s$\ga\mapsto\ee^\sigma \ga\s$ with \s$\sigma=1\s$ around
the insertion points then the correlation functions do not change. \s Let us
fix the complex structure of \s$\Sigma\s$ and holomorphic complex
coordinates around the insertion points of a correlation function. Call a
metric \s$\ga\s$ locally flat if it is compatible with the complex
structure of \s$\Sigma\s$ and of the form \s$|dz|^2\s$ around the
insertions. For such a choice of \s$\ga\s$ we shall drop the subscript
``\s$\ga\s$'' in the notation for the correlation functions, like in
eq.\s\s(\ref{Traceless0}).  We may restore the full dependence on the
conformal factor by using the covariance relations (\ref{PFWI})
and (\ref{GFWI}).  For example, for \s$\langle\m
T_{zz}\m\rangle_{\ga}\s$, we obtain
\qq
\langle\m T_{zz}\m\rangle_{_{\ee^\sigma\ga}}\s=\s
\langle\m T_{zz}\m\rangle_\ga\s+\s
{_{c}\over^{24}}\s{_{\delta}\over^{\delta \ga^{zz}}}\s
(\m\Vert\da\sigma\Vert^2_{L^2}\m+\m 4\int_{_\Sigma}\sigma\m r\s dv\m)
\ .
\label{ChA}
\qqq
In order to compute the functional derivative on the right hand side,
we shall need the following
\vs 0.3cm

\no{\bf Lemma.} \ Let \s$\ga^{z \bar{z} }=\ga^{\bar{z} z}=2\s$. \s To
the first order in \s$\ga^{zz}\s$,
\qq
r\s=\s-\frac{_1}{^2}\m(\partial_z^2\m \ga^{zz}+\m
\partial_{\bar{z}}^2\m \ga^{\bar{z}\m\bar{z} })\ .
\label{hallo9}
\qqq

\no{\bf {Proof}}. \ Consider the inverse metric \s$\ga^{-1}\s=\s \ga^{zz}\s
\partial_z^2 +\m 4
\partial_z \partial_{\bar{z}}\m+\m \ga^{\bar{z} \bar{z}}
\partial_{\bar{z}}^2\s.$ \s To compute the curvature
to the first order in $\ga^{zz}$ we shall change the variables to $z'=z+
\zeta(z,\bar z)$ so that in the new coordinate the metric is
$(4+\rho)\m\partial_{z'}
\partial_{\bar{z}'}\s.$ \s Since
\qq
\partial_z=(1+ \partial_z \zeta)\m\da_{z'}+ (\partial_z
\bar{\zeta})\m \partial_{\bar{z}'}\ ,\hs{0.7cm}
\partial_{\bar{z}}=(\partial_{\bar{z}} \zeta)\m
\partial_{z'}+(1+\partial_{\bar{z}}\bar{\zeta})
\partial_{\bar{z}'}
\nonumber
\qqq
then, retaining only the terms of the first order in \s$\ga^{zz}\s$,
\s$\zeta\s$ (and their complex conjugates), we obtain
\qq
\ga^{-1}&=&(\ga^{zz} -(\partial_x \ga^{zz}) \zeta -(\partial_{\bar{z}}
\ga^{zz}){\bar{\zeta}}+2\ga^{zz} \partial_z \zeta +4\partial_{\bar{z}}
\zeta)\s\partial_{z'}^2\cr\cr
&+&(4+4 \partial_z \zeta +4 \partial_{\bar{z}} {\bar{\zeta}}+2\ga^{zz}
\partial_z {\bar{\zeta}}+2\ga^{\bar{z}\m
\bar{z}} \partial_{\bar{z}} \zeta)\s\partial_{z'}
\partial_{\bar{z}'}\cr\cr
&+&(\ga^{\bar{z}\m\bar{z}}-(\partial_{\bar z} \ga^{\bar{z}\m
\bar{z}})\bar\zeta-(\partial_{{z}} \ga^{\bar{z}\m
\bar{z}}){{\zeta}}+2\ga^{\bar{z}\m
\bar{z}}\partial_{\bar{z}}{\bar{\zeta}}
+4\partial_z{\bar{\zeta}})\s\partial_{\bar{z}'}^2
\label{ChM}
\qqq
(we have kept more terms then needed for the Lemma for a future use).
The requirement that \s$\ga^{-1}=\m
(4+\rho)\m\partial_{z'}\partial_{{\bar z}'}\s$
means in the leading order
that
\s$\partial_{\bar{z}} \zeta\s=\s-\frac{1}{4}\m\ga^{zz}\s$.
\s Hence to the first order in $\ga^{zz}$
\qq
r'v'\s=\s-i\s\de'\da'\s \log{(1+\partial_{z}
\zeta + \partial_{\bar{z}}
{\bar{\zeta}})}&=&i\s\partial_{\bar z}
\partial_{{z}} (\partial_z
\zeta+\partial_{\bar{z}}
\bar{\zeta})\s dz \wedge d\bar{z}\ \cr
&=&-\frac{_1}{^2}\s(\partial_{z}^2\ga^{zz}+ \partial_{\bar{z}}^2
\ga^{\bar{z}\m\bar{z}})\s v'
\non
\qqq
where $v'$ is the new volume form equal to ${i\over^2}\m dz\wedge d\bar z$
in the $0^{\m\rm th}$ order.
\vs 0.5cm

\no Using the Lemma and the relation
\s\s$\Vert d\sigma\Vert^2_{L^2}
\s=\s\int_{_\Sigma}(\da_\mu\sigma)
\m(\da_\nu\sigma)\s \ga^{\mu\nu}\s dv\s,\s$
we obtain the relation
\qq
{_\delta\over^{\delta \ga^{zz}}}
\s(\m\Vert d\sigma\Vert^2_{L^2}\m+\m4\int_{_\Sigma}\sigma\m r\s dv\m)
\s=\s-2\s \da_z^2\sigma+(\da_z\sigma)^2
\label{tocom}
\qqq
which, substituted into eq.\s\s(\ref{ChA}), gives
the dependence on the conformal factors of the expectation value
of \s$T_{zz}\s$:
\qq
\langle\m T_{zz}\m
\rangle_{_{\ee^\sigma dzd\bar z}}\s=\s
\langle\m T_{zz}\m\rangle\s
-\s{_c\over^{12}}\s(\m\da_z^2\m\sigma\s-
\s{_1\over^2}\m(\da_z\sigma)^2\m)\ .
\label{2.68}
\qqq
What are the transformation properties of \s$\langle\m
T_{zz}\m\rangle\s$ under holomorphic changes \s$z\m\mapsto\m
z'=f(z)\s$ of the local coordinate? Under such replacements the notion
of a locally flat metric changes accordingly. By the diffeomorphism
covariance and eq.\s\s(\ref{2.68}), we have
\qq
&&({_{dz'}\over^{dz}})^2\s\m
\langle\m T_{z'z'}\m\rangle
\s=\s\langle\m T_{zz}\m\rangle_{_{ |dz'/dz|^2 dzd\bar z}}\cr
&&=\s\langle\m T_{zz}\m\rangle\s-\s{_c\over^{12}}
\left(\da_z^2\s\log{(dz'/dz)}\m-\m{_1\over^2}
\m(\da_z\s\log{(dz'/dz)}\m)^2\right)\cr
&&=\s\langle\m T_{zz}\m\rangle\s-\s{_c\over^{12}}
\left({_{d^3z'/dz^3}\over^{dz'/dz}}\s-\s{_3\over^2}
\m(\m{_{d^2z'/dz^2}\over^{dz'/dz}}
\m)^{^{\hs{-0.05cm}2}}\right)\s\equiv
\s\langle\m T_{zz}\m\rangle\s-\s{_c\over^{12}}\s\{z';\m z\}
\ .
\label{ProjCon}
\qqq
The function \s$\{z';\m z\}\s$ is the {\bf Schwarzian derivative} of
the change of variables. As we see, in the correlation functions with
locally flat metric, \s$T_{zz}\s$ does not transform as a pure
quadratic differential under general holomorphic changes of variables.
The transformation law
(\ref{ProjCon}) defines what is called a {\bf projective connection} on
\s$\Sigma\s$.
\vs 0.5cm

\no{\bf Problem 1}.\ \ Show that the Schwarzian derivative
\s$\{z';\m z\}\s$ vanishes iff \s$z'={az+b\over cz+d}\s$
with \s\s$(\matrix{_a&_b\cr^c&^d})\s\in$
\vskip -0.4cm
\no$SL(2,\NC)\s$, \s i.\s e. for the M\"{o}bius transformations.
\vs 0.6cm

The further information about the correlation functions with energy-momentum
tensor insertions will be obtained by studying deviations of the
metric from the locally flat one.  Applying to eq.\s\s(\ref{hallo7})
the operator
\s$\frac{\pi}{Z_\ga
}\s\frac{\delta}{\delta\ga^{ww}}\s Z_\ga\s$ at $\ga$ locally
flat\footnote{\s$w\s$ and \s$z\s$ refer to the complex coordinates of
two nearby insertions taken in the same holomorphic chart} and using
eq.\s\s(\ref{hallo9}), we obtain
\qq
 \pi\s\delta^{(2)}(z-w)\s\langle\m T_{zz}\m\rangle\s+\s\langle \m
T_{ww}\s T_{z \bar{z}}\m\rangle\s=\s\frac{_{\pi c}}{^{12}}\s\m
\partial^2_z\m \delta^{(2)}(z-w)\ , \label{hallo10}
\qqq
where \s$\delta^{(2)}\s$ stands for the two-dimensional
\s$\delta$-function. Let
us explore now the implications of the diffeomorphism covariance
(\ref{PFDI}) and (\ref{GFDI}). Under an infinitesimal transformation
\qq
  D(z)\s=\s z+ \zeta(z, \bar{z})\s\equiv\s z'\ ,
\label{hallo12}
\qqq
the change in the inverse metric \s$\delta
\ga^{-1}\s={\ga'}^{-1}-\ga^{-1}\s$,
\s where \s$D^*\ga'=\ga\s$, \s may be read from eq.\s\s(\ref{ChM}):
$$
\begin{array}{l}
\delta \ga^{zz}\s=\s-(\partial_z\ga^{zz})\m\zeta\s-
\s(\partial_{\bar{z}} \ga^{zz})
\m\bar{\zeta}\s+\s2\m \ga^{zz}\m \partial_z \zeta\s
+\s4\m\partial_{\bar{z}}\zeta\ ,
\\[3ex]
\delta \ga^{z\bar{z}
}\s=\s2\m\partial_z \zeta\s+\s2\m\partial_{\bar z}
\bar{\zeta}\s+\s \ga^{zz}\m\partial_z {\bar{\zeta}}\s+\s
\ga^{\bar{z}\m\bar{z}}\m\partial_{\bar{z}} \zeta\ ,
\\[3ex]
\delta \ga^{\bar{z}\m\bar{z}}\s=\s-(\partial_{\bar z}
\ga^{\bar{z}\m\bar{z}})\m\bar\zeta\s-\s(\partial_{{z}}
\ga^{\bar{z}\m\bar{z}})\m {\zeta}\s+\s2\m
\ga^{\bar{z}\m\bar{z}}\m\partial_{\bar{z}}
\bar\zeta\s+\s4\m \partial_z \bar{\zeta}\ .
\end{array}
$$ The diffeomorphism covariance implies that
$$\int_{_\Sigma}\left(\langle\m T_{zz}\m\rangle_{\ga}\s
\delta \ga^{zz}\s+\s2\m\langle
\m T_{z\bar{z}}\m\rangle_{\ga}\s\delta
\ga^{z\bar{z}}\s+\s\langle\m T_{\bar{z}\m\bar{z}}\m
\rangle_{\ga}\s\delta
\ga^{\bar{z}\m\bar{z}}\right)\m dv\s=\s0\ .$$ Inserting the expressions
for \s$\delta \ga^{-1}\s$,
\s stripping the resulting equation from the arbitrary function
\s$\zeta\s$ and retaining only the first order terms in $\ga^{zz}$ around
a locally flat metric, we obtain
\qq
  &(\partial_z \ga^{zz})\s\langle\m T_{zz}\m\rangle_{\ga} \s+\s2\m
\partial_z\m(\m \ga^{zz}\m\langle\m
T_{zz}\m\rangle_{\ga})\s+\s4\m\partial_{\bar{z}}\m \langle\m
T_{zz}\m\rangle_{\ga}\ \ &\cr\cr &+\s4\m\partial_z\s\langle\m
T_{z\bar{z}}\m\rangle_{\ga} \s+\s2\m\partial_{\bar{z}}\s(\m
\ga^{\bar{z}\m\bar{z}}\m \langle\m
T_{z\bar{z}}\m\rangle\m)\s+\s(\m\partial_z \m
\ga^{\bar{z}\m\bar{z}}\m)\m\langle\m
T_{\bar{z}\m\bar{z}}\m\rangle\s=\s0\ .& \label{hallo13}
\qqq
Specializing to $\ga^{zz}=0$, \s we infer that
\qq
  \partial_{\bar{z}}\s\langle\m T_{zz}\m\rangle\s=\s0\s=\s
\partial_z\langle\m T_{\bar{z}\m\bar{z}}\m\rangle\ .  \label{anal0}
\qqq
More generally, the component \s$T_{zz}\s$ (\s$T_{\bar z\m\bar z}\s$)
of energy-momentum tensor \s is analytic (anti-analytic) in correlation
functions in a locally flat metric and away from other insertions.
Eq.\s\s(\ref{anal0}) is another conformal Ward identity.
\vs 0.4cm

At coinciding points, the correlation functions of  energy-momentum
tensor give rise to singularities which we shall study now.
Application of \s\s$\frac{ \pi}{Z_\ga}\s \frac {\delta}{\delta \ga^{ww}}
\s Z_\ga\s\s$ at \s$\ga\s$ locally flat to eq.\s\s(\ref{hallo13})
gives: $$\pi \left(\partial_z\m\delta^{(2)} (z-w) \right)\m\langle\m
T_{zz}\m\rangle\s+\s2\m\pi\m
\partial_z\m \delta^{(2)}(z-w)\s\langle\m T_{ww}\m\rangle
\s+\s\partial_{\bar{z}}\s\langle\m T_{zz}\s T_{ww}\m\rangle
\s+\s\partial_z\m
\langle\m T_{z\bar{z}}\s T_{ww}\s\rangle\s=\s0\ .$$
Using eq.\s\s(\ref{hallo10}) differentiated with respect \s$z\s$ in
order to replace
\s$\partial_z\m\langle\m T_{z\bar{z}}
\s T_{ww}\m\rangle\s$ in the last relation, we obtain
\begin{eqnarray*}
   \partial_{\bar{z}}\s\langle\m T_{zz}\s T_{ww}\m\rangle &=&-\pi
\left(\partial_z\m \delta^{(2)}(z-w) \right)\m\langle\m
T_{zz}\m\rangle\s-\s \pi\s\partial_z\m\delta^{(2)}(z-w) \m\langle\m
T_{ww}\m\rangle\s-\s\frac{_{\pi c}}{^{12}} \s
\partial_z^3\m\delta^{(2)}(z-w)\ \cr &=&-\frac{_{\pi
c}}{^{12}}\s\partial_z^3\m\delta^{(2)}(z-w) \s-\s2\m\pi\s \partial_z\m
\delta^{(2)} (z-w)\s\langle\m T_{ww}\m\rangle\s
+\s\pi\s\delta^{(2)}(z-w)\s\partial_w\m\langle\m T_{ww}\m\rangle\ .
\end{eqnarray*}
This is a distributional equation. Since \s$\delta^{(2)}(z-w)\s=
\s\frac{1}{\pi}\m\partial_{\bar{z}}\s\frac{1}{z-w}\s$
in the sense of distributions, it follows that
\qq
\da_{\bar z}\s\langle\m T_{zz}\s T_{ww}\m\rangle\s=\s
\da_{\bar z}\left(\frac{c/2}{(z-w)^4}\s
+\s\frac{2}{(z-w)^2}\s\langle\m T_{ww}\m\rangle\s+\s\frac{1}{z-w}
\s\partial_w\m \langle\m T_{ww}\m\rangle\right)\ ,
\label{CWId}
\qqq
which is still another conformal Ward identity.
Since the only solutions of
the distributional equation
\s$\da_{\bar z}\m f=0\s$ are analytic functions, one may rewrite
the identity (\ref{CWId}) as a short distance
expansion
encoding the ultraviolet properties of the CFT:
\qq
\langle\m T_{zz}\s T_{ww}\m\rangle
\s=\s\frac{c/2}{(z-w)^4}\s+\s\frac{2}{(z-w)^2}\s\langle\m
T_{ww}\m\rangle\s+\s\frac{1}{z-w}
\s\partial_w\m \langle\m T_{ww}\m\rangle\s\s+\s\s\m.\ \s.\ \s.\ \s\s,
\label{OPE0}
\qqq
where ``\s$.\ \s.\ \s.\s$'' stands for terms analytic in \s$z\s$
around \s$z=w\s.$ \s The complex conjugation of
eq.\s\s(\ref{OPE0}) gives the singular terms of
\s\s$\langle\s T_{\bar z\m\bar z}\s T_{\bar w\s\bar w}
\s\rangle\s$, \s this time, up to anti-analytic terms.
Expansions of the type (\ref{OPE0}) are usually
called the {\bf operator product expansion} (OPE)
in accordance with the operator interpretation of correlation
functions to be discussed in the next section.
We shall follow this terminology.
\vs 0.5cm

What about the mixed insertions? Differentiating
(\s$Z_\ga\s\s\times\s\m)$ \s eq.\s\s(\ref{hallo13})
with respect to \s$\ga^{\bar w\m\bar w}\s$ at \s$\ga\s$ locally
flat, we obtain
\qq
\da_{\bar z}\m\langle\s
T_{zz}\s\m T_{\bar w\m\bar w}\s\rangle
\s+\s\da_z\m\langle\s T_{z\bar z}\s\m
T_{\bar w\m\bar w}\s\rangle
\s+\s\pi\s(\m\da_z\s\delta^{(2)}(z-w)\m)\s\m\langle\s
T_{\bar z\m\bar z}\s\rangle\s=\s0\ .
\non
\qqq
With the use of the complex conjugate version of
eq.\s\s(\ref{hallo10}) to eliminate \s$\langle\s T_{z\bar z}\s\m
T_{\bar w\m\bar w}\s\rangle\s$, \s this reduces to
\qq
\da_{\bar z}\m\langle\s
T_{zz}\s\m T_{\bar w\m\bar w}\s\rangle\s=\s
-\m{_{\pi c}\over^{12}}\s\da_z\s\da_{\bar z}^{\m 2}\s\m
\delta^{(2)}(z-w)
\nonumber
\qqq
which, stripped of \s$\da_{\bar z}\s$, \s gives
\qq
\langle\s
T_{zz}\s\m T_{\bar w\m\bar w}\s\rangle\s=\s
-\m{_{\pi c}\over^{12}}\s\da_z\s\da_{\bar z}\s\m
\delta^{(2)}(z-w)\s+\s\ .\ \s.\ \s.\ \ ,
\label{contact}
\qqq
i.\s e. a contact term with support at \s$z=w\s$
plus a function analytic in \s$z\s$
and anti-analytic in \s$w\s$.
\vs 0.5cm

The other source of singular contributions to the correlation functions
of \s$T_{zz}\s$ or \s$T_{\bar z\m\bar z}\s$
are insertions of the (primary) fields
\s$\phi_l(x)\s$. \s Let us compute these singularities.
Proceeding similarly as before, we apply
\s\s$\frac{{\pi}}{{Z_\ga}}\s\frac{\delta}{{\delta \sigma}}\s\s$
to \s\s$Z_{e^{\sigma}\ga}\s\m\langle\m \phi_l(x)\m
\rangle_{_{e^{\sigma}\ga}}\s\s$ at \s$\sigma=0\s$
and $\ga$ locally flat obtaining with the help of eqs.\s\s(\ref{PFWI})
and (\ref{GFWI}) the relation
\qq
\langle\m T_{z\bar{z}}\s\s \phi_l(w,\bar w)\m\rangle\s =\s
\pi\s
\Delta_l\s\m \delta^{(2)}(z-w)\s\s\langle\s \phi_l(w,\bar w)\s\rangle
\label{hallo14}
\qqq
(we have replaced the point \s$x\s$ in the argument of
\s$\phi_l\s$ by its local coordinate \s$w\s$
and its complex conjugate to stress
the non-holomorphic dependence on \s$x\s$
of the \s$\phi_l(x)\s$ insertion).
Next we exploit the diffeomorphism covariance.  For \s$D(z)
= z+\zeta(z,\bar z)\equiv z'\s$
and \s$\ga=D^* \ga'\s$, $$ \langle\s
\phi_l(w',{\bar w}')\s\rangle_{{\ga'}}\s\m Z_{\ga'}\s=\s
\langle\s\phi_l(w,\bar w)\s\rangle_{{\ga}}\s\m Z_{{\ga}}\ .$$
Since for \s$\ga=|dz|^2\s$ $$\delta \ga^{-1}\m
\equiv\s (\ga')^{-1}-\ga^{-1}\s=\s
4\m(\partial_{\bar{z}}\zeta)\m
\partial^2_z\s+\s4\m(\m\partial_z \zeta\m+
\m\partial_{\bar{z}}\bar\zeta\m)\m
\partial_z \partial_{\bar{z}}\s+\s4\m
(\m\partial_{z}\bar{\zeta})\m
\partial^2_{\bar{z}},
$$ to the first order in \s$\zeta\s$, \s see eq.\s\s(\ref{ChM}), we infer
that $$ \pi\s \delta^{(2)}(z-w)\s\s \partial_w\s
\langle\s\phi_l(w,\bar w)\s\rangle\s - \s
\partial_{\bar{z}}\s\langle\s
T_{zz}\s \phi_l(w,\bar w)\s\rangle
\s-\s \partial_z\s\langle\s T_{z {\bar{z}}}
\s\s\phi_l(w,\bar w)\s\rangle\s=\s0\ .
$$ Using the last equation to eliminate \s$\partial_z
\s\langle\s T_{z{\bar{z}}}\s\s\phi_l(w,\bar w)\s\rangle\s$
from eq.\s\s(\ref{hallo14}) acted upon by
\s$\partial_z\s$, \s
we obtain the relation $$
\partial_{\bar{z}}\s\langle\s T_{zz}\s\phi_l(w,\bar w)
\s\rangle\s =\s- \pi\s\Delta_l\s \delta^{(2)}
(z-w)\s\langle\s\phi_l(w,\bar w)\s\rangle\s + \s\pi\s
\delta^{(2)}(z-w)\s\m\partial_w\m\langle\m\phi_l(w,\bar w)\s
\rangle$$ which may be conveniently rewritten as an
OPE of the product of the
\s$T_{zz}\s$ component of
energy-momentum tensor with a primary field:
\qq
\langle\s T_{zz}\s\phi_l(w,\bar w)\s\rangle
  \s=\s\left( \frac{{\Delta_l}}{{(z-w)^2}}\s+
\s\frac{1}{{z-w}}\s \partial_w
  \right)\m\langle\s \phi_l(w,\bar w)\s\rangle\s\s+\s\s\m.\ \s.\
\s.\
\s\s.
\label{SHEXP2}
\qqq
\vs 0.3cm

Finally note, that under the holomorphic change of
the local coordinate
\s$z\s\mapsto\s z'=f(z)\s$
\qq
\langle\s\phi_l(z',{\bar z}')\s\rangle\s=\s
\langle\s\phi_l(z,\bar z)\s
\rangle_{_{|dz'/dz|^2dzd\bar z}}\s\s=\s|{_{dz'}
\over^{dz}}|^{^{-2\m\Delta_l}}\s\m\langle\s\phi_l(z,\bar z)
\s\rangle
\non
\qqq
\vs -0.3cm
\no or
\vs -0.55cm
\qq
\langle\s\phi_l(z',{\bar z}')\s\rangle\s\m
(dz')^{^{\Delta_l}}(d{\bar z}')^{^{\Delta_l}}
\s=\s\langle\s\phi_l(z,\bar z)\s\rangle\s\m
(dz)^{^{\Delta_l}}(d{\bar z})^{^{\Delta_l}}
\non
\qqq
so that \s$\phi_l\s$ behaves like a \s$(\Delta_l,\Delta_l)$-form in
the correlation functions with locally flat metric.
One often needs to consider also primary fields
with weights
\s$(\Delta_l,\tilde\Delta_l)\s$ and
\s$\Delta_l-\m\tilde\Delta_l\s$ integer
(or half-integer).
\s$d_l=\Delta_l+\tilde\Delta_l\s$ is the scaling
dimension of such a field and
\s$s_l=\Delta_l-\tilde\Delta_l\s$ its spin.
Geometrically, the correlation functions of such fields are sections of the
\s$s^{\m{\rm th}}\s$ power of the sphere subbundle in the cotangent
bundle \s$T^*\Sigma\s$.
\vs 0.6cm

Let us collect the relations obtained in this section for low point
insertions in the
correlation functions. Since all the considerations were local, the same
equalities hold in correlation functions with other insertions as long as
their points stay away from the insertions taken together.
Adding also
the relations involving the complex conjugate components of
energy-momentum tensor and introducing
simplified notation \s$T\equiv T_{zz}\s$,
\s$\bar T=T_{\bar z\m\bar z}\s$,
\s we obtain:
\vs 0.4cm

\no\hbox to 1cm{i/.\hfill}{\bf identities}
\qq
&&T_{z\bar z}\s=0=\s T_{\bar z z}\ ,
\label{Tracecalss}\\
\cr
&&\partial_{\bar{z}}\m T\s=0=\s
\partial_z\m \bar T\ ,
\label{anal}
\qqq
\vs 0.3cm
\no\hbox to 1cm{ii/.\hfill}{\bf operator product expansions}
\qq
&&T(z)\s\m T(w)\s=\s\frac{c/2}{(z-w)^4}\s
+\s\frac{2}{(z-w)^2}\s T(w)\s
+\s\frac{1}{z-w}\s\partial_w\m
T(w)\s+\hs{0.1cm}\s\s\s.\ \s.\ \s.\ \s\s,
\label{OPE1}
\cr
&&\bar T(\bar z)\s\m
\bar T(\bar w)
\s=\s\frac{c/2}{(\bar z-\bar w)^4}\s
+\s\frac{2}{(\bar z-\bar w)^2}\s\bar T(\bar w)\s
+\s\frac{1}{\bar z-\bar w}\s\partial_{\bar w}\s
\bar T(\bar w)\s+\hs{0.1cm}\s\s\s.\ \s.\ \s.\ \s\s,
\label{OPE2}
\cr
&&T(z)\s\m \bar T(\bar w)\s=\s
-\m{{\pi c}\over{12}}\s\m\da_z\s\da_{\bar z}\s\m
\delta^{(2)}(z-w)\s+\s\s\s.\ \s.\ \s.\ \s\s,
\label{OPEC}\\
&&T(z)
\s\s\phi_l(w,\bar w)\s=\s\left(\frac{{\Delta_l}}{{(z-w)^2}}+
\frac{1}{{z-w}}\s\partial_w \right)
\phi_l(w,\bar w)\s+\s\s\s.\ \s.\ \s.\ \s\s,
\label{OPE3}\\
\cr
&&\bar T(\bar z)\s\s
\phi_l(w,\bar w)\s=\s\left(\frac{{\Delta_l}}
{{(\bar z-\bar w)^2}}\s+\s
\frac{1}{{\bar z-\bar w}}\s\partial_{\bar w}\right)
\phi_l(w,\bar w)\s+\s\s\s.\ \s.\ \s.\ \s\s,
\label{OPE4}
\qqq
\vs 0.3cm

\no\hbox to 1cm{iii/.\hfill}{\bf transformation laws}
\qq
&&T(z')\s (dz')^2
\s=\s T(z)\s(dz)^2\s-
\s{c\over{12}}\s\{z';\m z\}
\s(dz)^2\ ,\ \
\label{transT}\\
\cr
&&\bar T({\bar z}')\s (d{\bar z}')^2
\s=\s\bar T(\bar z)\s(d\bar z)^2\s-
\s{c\over{12}}\s{\overline{\{z';\m z\}}}
\s(d\bar z)^2\ ,\ \
\label{transbarT}\\
\cr
&&\phi_l(z',{\bar z}')\s
(dz')^{^{\Delta_l}}(d{\bar z}')^{^{\tilde\Delta_l}}
\s=\s\phi_l(z,\bar z)\s
(dz)^{^{\Delta_l}}(d{\bar z})^{^{\tilde\Delta_l}}\s.\
\label{transvarp}
\qqq
\vs 1cm



\no{\bf 3.\ \s Physical positivity and Hilbert space picture}
\vskip 0.4cm

\no Up to now we have analyzed abstract conformal
fields in the Euclidean formalism, probabilistic in its
nature and distinct from the traditional operator
approach. The operator formalism of QFT fits into
the general quantum mechanical scheme with the
\begin{enumerate}
\item
Hilbert space of states,
\item
representation of the symmetry group
or algebra,
\item
distinguished family of operators
\end{enumerate}
as its basic triad. This is a fundamental fact of QFT
that the passage between the Euclidean and the operator
formalisms, which we have discussed already for free fields,
may be done in quite general circumstances.
This fact is responsible for the deep relation between
critical phenomena and quantum fields and it has strongly marked
the developments of QFT. CFT, which is not an exception in
this respect, has largely profited from the unity of two
approaches. In the present section we shall discuss
how the operator picture may be recovered from
the Euclidean formulation of CFT presented
above assuming  the {\bf physical} (or Osterwalder-Schrader)
{\bf positivity} formulated as a condition on correlation functions
on the Riemann sphere \s$\NC P^1\s$.
Analysis of the genus zero situation will allow
to recover the Hilbert space of states
and to translate the operator product
expansions of the last section into an action
of the Lie algebra of conformal symmetries and of
the primary field operators in the space of states.
Later we shall describe the operator formalism
on higher genus Riemann surfaces which permits
to relate naturally CFT in different space-time
topologies.
\vs 0.5cm

Let us consider the map \s$\vartheta:\NC P^1\rightarrow\NC P^1\s$,
\s$\vartheta(z)={\bar z}^{\m-1}\s$. \s$\vartheta\s$ interchanges
the disc \s$D=\{\s|z|\leq 1\s\}\s$ with \s$D'\equiv\{\s|z|\geq 1\s\}\s$
and leaves invariant their common boundary \s$\{\s|z|=1\s\}\s$.
\s Suppose that we are given a Riemannian metric \s$\ga\s$
on \s$D\s$, compatible with the complex structure,
which is of the form \s$|z|^{-2}|dz|^2\s$ around \s$\da D\s$
(we shall call such a metric flat at boundary). \s$\vartheta^*\ga\s$
is a metric on \s$D'\s$ and it glues smoothly with \s$\ga\s$ on \s$D\s$
to the metric \s$\vartheta^*\ga\vee \ga\s$ on \s$\NC P^1\s$. \s
Consider formal expressions
\qq
X\s=\s\prod\limits_i\s\phi_{l_i}(z_i,\m \bar z_i)
\label{X}
\qqq
for distinct
points \s$z_i\s$ in the interior of \s$D\s$
(with the empty product case included).
\s Denote
\qq
\Theta\phi_l(z,\bar z)\s\equiv\s
(-{\bar z}^{\m-2})^{\Delta_l}\m
(-z^{-2})^{\tilde\Delta_l}
\s\m\phi_{\m\bar l\m}({_1\over^{\bar z}},\m
{_1\over^{z}})\
\label{Theta1}
\qqq
where \s$l\mapsto\bar l\s$ is the same involution
that appeared in eq.\s\s(\ref{OrCh}).
For \s$X\s$ as above, we set
\qq
\Theta X\s=\s\prod\limits_i\s\Theta\phi_{l_i}
(z_i,\m \bar z_i)\ .
\label{ThetaX}
\qqq
The physical positivity
requires that for each family \s$(\lambda_\alpha)\s$ of
complex numbers, each family \s$(X_\alpha)\s$ of expressions
(\ref{X}) and each family \s$(\ga_\alpha)\s$ of metrics on \s$D\s$
flat at boundary
\qq
\sum\limits_{\alpha_1,\s\alpha_2}\s\bar\lambda_{\alpha_2}\s
\lambda_{\alpha_1}
\s\m Z_{\vartheta^*
\ga_{\alpha_2}\hs{-0.03cm}\vee \ga_{\alpha_1}}
\s\s\langle\s(\Theta X_{\alpha_2})\s\m X_{\alpha_1}\s
\rangle_{_{\vartheta^*\ga_{\alpha_2}\hs{-0.03cm}
\vee \ga_{\alpha_1}}}
\ \geq\ 0\ .
\label{PhysPosX}
\qqq
These properties hold for the free field compactifications.
The condition (\ref{PhysPosX}) may be rewritten using
correlation functions with energy-momentum insertions
and a fixed metric. Set
\qq
\Theta T(z)\s\equiv\s{\bar z}^{\m-4}\s\m
T({_1\over^{\bar z}})\ ,\label{Theta2}\\
\Theta \bar T(\bar z)\s\equiv\s{z}^{-4}\s\m
\bar T({_1\over^{z}})\ \ \s\label{Theta3}
\non
\qqq
and extend the definition (\ref{X}) to expressions
\qq
Y\s=\s\prod\limits_m\m T(z_m)\s
\prod\limits_n\m \bar T
(\bar z_n)\s\prod\limits_i\phi_{l_i}(z_i,\m \bar z_i)
\label{Y}
\qqq
(with all points in the interior of \s$D\s$ and distinct)
for which
\qq
\Theta Y\s=\s\prod\limits_m\m \Theta T(z_m)
\s\prod\limits_n\m\Theta\bar T
(\bar z_n)\s\prod\limits_i\m\Theta
\phi_{l_i}(z_i,\m \bar z_i)\ .
\label{ThetaY}
\qqq
One may infer from the property (\ref{PhysPosX}) that
\qq
\sum\limits_{\alpha_1,\s\alpha_2}\s\bar\lambda_{\alpha_2}\s
\lambda_{\alpha_1}
\s\m\langle\s(\Theta Y_{\alpha_2})\s Y_{\alpha_1}
\s\rangle\ \geq\ 0
\label{PhysPosY}
\qqq
where \s$\langle\ \cdots\ \rangle\s$ denotes the correlation
functions in a locally flat metric.
\vs 0.5cm

\no{\bf Problem 2}.\ \ Show that (\ref{PhysPosX})
implies (\ref{PhysPosY}).
\vs 0.5cm

\no The construction of the Hilbert space \s${\CH}\s$ of states
is now simple. The expression
\qq
\sum\limits_{\alpha,\s\beta}
\m\bar\lambda'_\beta\s\lambda_\alpha\s\m\langle\s(\Theta Y'_\beta)
\s Y_\alpha\s\rangle
\non
\qqq
\vs -0.1cm
\no defines a hermitian form on the space \s$V_D\s$ of formal
linear combinations of products (\ref{Y}). \s Due to
(\ref{PhysPosY}), this form is positive and becomes positive
definite on the quotient by its null subspace
\s$V_D^{\rm null}\s$. \s One sets
\qq
{\CH}\s=\m{\overline{V_D/V_D^{\rm null}}}\s.
\label{HilbSp}
\qqq
We shall denote by \s$\iota\s$ the canonical map
from $V_D$ to $\CH$, by \s$\CH_0\s$ its image \s$\subset\CH\s$
and by \s$\CY\s$ the image \s$\iota(Y)\s$ of $Y$.
The empty product in (\ref{Y}) gives rise to the
``{\bf vacuum vector}'' \s$\Omega\s$. The scalar product is given by
\qq
(\s\CY'\m,\s\CY\s)\s=\s\langle\s(\Theta Y')\s Y\s\rangle\s.
\non
\qqq
$\CH\s$ carries an anti-unitary
involution \s$\CI\s$ mapping vector \s$\CY\s$
to \s$\bar\CY\s$ where \s$\CY\s$ corresponds to
\qq
\bar Y\s=\s\prod\limits_m\m T(\bar z_m)\s
\prod\limits_n\m \bar T
(z_n)\s\prod\limits_l\phi_{\bar l_i}
(\bar z_i,\m z_i)\ .
\label{barY}
\qqq
\vs 1.1cm



\no{\bf 4.\ \s Virasoro algebra and its primary fields}
\vskip 0.4cm

\no Define the action of dilations by
\s$q\in\NC\m,\ 0<|q|\leq 1\m,\s$ on the fields
by setting
\qq
S_q\m T(z)= q^2\m T(qz)\m,\hs{0.25cm}
S_q\m \bar T(\bar z)= {\bar q}^2\m
T(\bar q\m\bar z)\m,\hspace{0.25cm}
S_q\m\phi_l(z,\bar z)= q^{\Delta_l}\m{\bar q}^{\bar
\Delta_l}\m\phi_l(qz,\bar q\m\bar z)\s.\hs{1cm}
\non
\qqq
For \s$Y\s$ given by (\ref{Y}), we put
\qq
S_q\m Y\s=\s\prod\limits_m\m S_q\m T(z_m)\s
\prod\limits_n\m S_q\m \bar T
(\bar z_n)\s\prod\limits_l\m S_q\m
\phi_{l_i}(z_l,\m \bar z_l)\ .
\non
\qqq
\vs 0.5cm

\no{\bf Problem 3}.\ \ Using the conformal symmetries of
the correlation functions, verify that
\qq
\langle\s(\Theta Y')\s S_q\m Y\s\rangle
\s=\s\langle\s(\Theta\s S_{\bar q}\m Y')\s Y\s\rangle\ .
\label{Selfad}
\qqq
\vs 0.7cm

\no In the Hilbert space, we may define the dilation
operator \s$\CS_q\s$ by the equality
\qq
\CS_q\m\CY\s=\s\iota(S_q\m Y)\s.
\non
\qqq
Note that eq.\s\s(\ref{Selfad}) implies that
\s$\CS_q\s$ is well defined on the dense invariant
domain \s$\CH_0\s$. \s In fact, the family of operators
\s$(\CS_q)\s$ forms a semigroup: \s$\CS_{q_1}\CS_{q_2}=\CS_{q_1q_2}\s$.
\s Applying many times the Schwartz inequality,
identity (\ref{Selfad}) and the semigroup property
of \s$\CS_q\s$, \s one obtains following
Osterwalder-Schrader:
\qq
&|\s(\s\CY'\m,\s\CS_q\m\CY\s)\s|\s\leq\s
\|\CY'\|\s\s\|\CS_q\m\CY\|\s=\s
\|\CY'\|\s\m(\s\CY\m,\s\CS_{\bar qq}\m\CY\s)^{1/2}\ &\cr\cr
&\s\leq\ \cdot\ \cdot\ \cdot\ \cdot\ \cdot\ \leq\s
\|\CY'\|\s\s\|\CY\|^{{1\over2}+\dots+{1\over2^{n-1}}}\s\s
(\s\CY\m,\s\CS_{(\bar qq)^{2^{n-1}}}\m\CY\s)^{\m1/2^n}\ .&
\label{estim}
\qqq
Assume now that for each \s$\epsilon>0\s$, there exists
a constant $C_\epsilon$ s.\s t.
\qq
|\s\langle\s(\Theta\m Y')\s S_t\s Y\s\rangle\s|\s\leq\s
C_\epsilon\s t^{-\epsilon}
\label{techni}
\qqq
when \s$t\rightarrow 0\s$. \s What it means
is that when the distances of a group
of insertions are uniformly shrunk to zero
the singularity of the correlation functions
is not stronger then the power law given by the overall scaling
dimension of the group. Using bound (\ref{techni}) on the right
hand side of (\ref{estim}) and taking \s$n\s$ to infinity,
we infer that
\qq
|\s(\s\CY'\m,\s\CS_q\m\CY\s)\s|\s\leq\s\|\CY'\|\s\|\CY\|\ ,
\non
\qqq
i.\s e. that the dilation
semigroup \s$\CS_q\s$ is composed
of contractions of \s$\CH\s$. \s Eq.\s\s(\ref{Selfad})
implies now that \s$\CS_q^*=\CS_{\bar q}\s$.
\s The weak continuity of the semigroup \s$(\CS_q)\s$ on
\s$\CH\s$ follows from that on \s$\CH_0\s$ which is evident.
By the abstract semigroup theory
\qq
\CS_q\s=\s q^{L_0}\s{\bar q}^{\tilde L_0}
\label{AbSemi}
\qqq
for strongly commuting self-adjoint operators
\s$L_0\s$ and \s$\tilde L_0\s$ s.\s t.
\s$L_0+\tilde L_0\s\geq\s0\s.\s$
Clearly, \s$\s\CH_0\s$ is inside the domain of
\s$L_0\s$ and of \s$\tilde L_0\s$ and
\qq
L_0\m\CY\s=\s \da_q|_{_{q=1}}\s\CS_q\m\CY\ ,\hs{0.5cm}
\tilde L_0\m\CY\s=\s \da_{\bar q}|_{_{q=1}}
\s\CS_q\m\CY\ .\hs{0.5cm}
\label{generat}
\qqq
It also follows that
\s$\CS_q\s\m\CH_0\s$ is dense in \s$\CH\s$ for all
\s$q\s$.
\vs 0.4cm

$L_0\s,\ \s\tilde L_0\s$ are only the
tip of an operator iceberg. To see more of it,
define operators \s$\CT(z)\s$,
\s${\bar\CT}(\bar z)\s$
and \s$\varphi_l(z,\bar z)\s$, \s with $\CS_{z}\CH_0$
as the (dense) domain ($|z|<1$),  by setting
\qq
\CT(z)\s\CY\s=\s\iota(T(z)\s Y)\s,\quad\ \
\bar\CT(\bar z)\s\CY\s=\s\iota(\bar T(\bar z)\s Y)\s,\quad\ \
\varphi_l(z,\bar z)\s\CY\s=\s\iota(\phi_l(z,\bar z)\s Y)\s.
\non
\qqq
It is easy to see that the operators
\s$\CT(z)\s$, \s$\bar\CT(\bar z)\s$
and \s$\varphi_l(z,\bar z)\s$ are well defined.
Note that for \s$Y\s$ given by eq.\s\s(\ref{Y}) and
with the absolute values of all insertion points different,
\qq
\CY\s=\s R\left(\m\prod\limits_m\m \CT(z_m)\s
\prod\limits_n\m {\bar\CT}
(\bar z_n)\s\prod\limits_i\varphi_{l_i}(z_i,\m \bar z_i)
\right)\m\Omega
\label{RQ}
\qqq
where \s$R\s(\m\cdots\m)\s$
reorders the operators so that they act
in the order of increasing \s$|z|\s$.
This is the reason why the operator scheme described here
is often called {\bf radial quantization}. Under
the conjugation by
the anti-unitary involution \s$\CI\s$ of \s$\CH\s$,
\qq
\CI\m\varphi_l(z,\bar z)\m\CI\s=\s\varphi_{\m\bar l\m}
(\bar z,\m z)\s,\ \ \s\CI\m\CT(z)\m\CI\s
=\s\CT(\bar z)\s,\ \ \s\CI\m\bar\CT
(\bar z)\m\CI\s=\s\bar\CT(z)\ .
\label{CII}
\qqq
\vs 0.3cm

It will be useful to introduce Fourier components
of the operators \s$\CT(z)\s$ and
\s$\bar\CT(\bar z)\s$:
\qq
  L_n &=& \frac{_1}{^{2 \pi i}}
  \oint_{|z|=r<1}\hs{-0.2cm}z^{n+1}\s\CT(z)\m\s dz\ ,
  \label{Ln}\\
  \tilde{L}_n &=& -\frac{_1}{^{2 \pi i}} \oint_{|z|=r<1}\hs{-0.2cm}
  {\bar{z}}^{n+1}\s\bar\CT(\bar z)
  \s\s d\bar z\ .
  \label{barLn}
\qqq
Since the insertion of \s$T(z)\s$
in the correlation functions \s$\langle\s\s\cdots\s\s\rangle\s$
is analytic in \s$z\s$ as long as the
other insertions are not met, the matrix elements
\s$(\s\CY'\m,\s L_n\m\CY\s)\s$ (and hence the vector \s$L_n\CY\s$ itself)
does not depend on \s$r\s$
as long as \s$r<1\s$ and the contour \s$|z|=r\s$ surrounds the
insertions of \s$Y\s$ (similarly for \s$\tilde L_n\s$)\m.
Notice that
\begin{eqnarray}
(\s\CY'\m,\s L_n\m\CY\s)&=&\frac{_1}{^{2\pi i}}\oint_{|z|=1-\epsilon}
	   \hs{-0.2cm}z^{n+1}\s\m\langle\s(\Theta Y')\s\s
	   T(z)\s\s Y\s\rangle\s dz\cr
       &=&\frac{_1}{^{2\pi i}}\oint_{|z|=1+\epsilon}\hs{-0.2cm}z^{n-3}
	   \s\m\langle\s\m(\s\Theta\m(Y'\s T({_1\over^{\bar z}})\s)
	   \s\s\m Y\s\m\rangle\s\m dz
	   \label{PartR}
\end{eqnarray}
where we have moved the integration contour slightly,
\s representing \s$T(z)\s$ with \s$|z|=1+\epsilon\s\s$ as
\s\s$z^{-4}\s\Theta\m T({_1\over^{\bar z}})\s$.
\s The right hand side is equal to
\qq
       &&\hs{1cm}(\s-\frac{_1}{^{2\pi i}}
	  \oint_{|z|=1+\epsilon}\hs{-0.2cm}\bar{z}^{n-3}
	  \s\s\hat{T}_{zz}(\frac{_1}{^{\bar{z}}})\s\s d\bar z\s\s\m
	   \CY'\s\m,\s\s\s\CY\s\m)\s\m \cr
       &&=\s(\s\m\frac{_1}{^{2\pi i}}\oint_{|w|=(1+\epsilon)^{-1}}
	 \hs{-0.2cm}w^{-n+1}\s\s T(w)\s\s dw\s\s
	   \CY'\s\m,\s\s\s \CY\s\m)\s=\s
	   (\m\s L_{-n}\m\CY'\s,\s\s\CY\s\m)\ .\nonumber
\end{eqnarray}
It follows, that operators \s$L_n\s$ (and \s$\tilde L_n\s$) \s
are closable\footnote{we keep
the same symbols for their closures} and their adjoints
satisfy
\qq
L_n^{\s*}\s=\s L_{-n}\ ,\ \ \ \
\tilde L_n^{\s*}\s=\s \tilde L_{-n}\ .
\label{Adjoints}
\qqq
$L_n\s$'s \s and \s$\tilde L_n\s$'s \s commute with the
anti-involution \s$\CI\s$ of \s$\CH\s$.
It will be convenient
to somewhat extend the domain of definition of
the operators introduced above. Let us
admit in expressions \s$Y\s$ of
(\ref{Y}) integrated insertions \s$\oint_{|z|=r}z^{n+1}\s
T(z)\s dz\s$ and similarly for \s$\bar T(\bar z)\s$.
Denote by \s$\CH_1\s$ the resulting subspace of \s$\CH\s$.
\s Of course \s$\CH_1\s$ contains \s$\CH_0\s$ and
is invariant under \s$L_n$'s and \s$\tilde L_n$'s. Operators
\s$\CT(z)\s$, \s$\bar\CT(\bar z)\s$
and \s$\varphi_l(z,\bar z)\s$ may be clearly
extended to \s$\CS_{z}\CH_1\s$ and
we shall assume below that this has been done\footnote{let
us remark that these operators
are not closable so their domains should be
handled with special care}.
\vs 0.5cm

The calculation which we shall do now is an example
of an argument which translates (certain) OPE's into commutation
relations and is used in CFT again and again.
A devoted student should memorize its idea once for all.
We start with the OPE (\ref{OPE1}) for
\s$T(z)\s$ which will give commutation relations
between \s$L_n\m$'s. \s Let us consider
the matrix element
\qq\lefteqn{
    (\s\CY'\m,\s[L_n\m,\s\CT(w)]\m\s\CY\s)}\\
  &=&\frac{_1}{^{2 \pi i}}\left( \oint_{|z|=|w|+\epsilon}
     \hs{-0.2cm}dz\s -\m
     \oint_{|z|=|w|-\epsilon}\hs{-0.2cm}dz\right)
     \m  z^{n+1}\s\s\langle\s(\Theta Y')\s
     \s T(z)\s\s T(w)\s\s Y\s\rangle\\
  &=&\frac{_1}{^{2 \pi i}}\s\oint_{|z-w|=\epsilon}\hs{-0.2cm}
    z^{n+1}\s dz\s\s\langle\s(\Theta Y')\s
    \s\s(\m\frac{_{c/2}}{^{(z-w)^4}}
    +\frac{_2}{^{(z-w)^2}}\s T(w)\m+\frac{_1}{^{z-w}}\s
    \partial_w T(w)\m)\s\s Y\s\rangle\ ,\ \
\label{ChC}
\qqq
where we have used the fact (\ref{RQ}) that the order of
operators is determined by the radial order of insertions
in the correlation function. In the last line
we have collapsed the contour of integration to a small
circle around $w$ and inserted the OPE (\ref{OPE1}).
Expanding \s$z^{n+1}\s$ around \s$z=w\s$
\qq
  z^{n+1}\s=\s\left( (z-w)+w \right)^{n+1}\s=\s
  \frac{_{n^3-n}}{^6}\s(z-w)^3\s w^{n-2}\s+\s{_{n^2+n}\over^2}
  \s(z-w)^2\s w^{n-1}\ \cr
  +\s(n+1)\s(z-w)\s
  w^n\s+\s w^{n+1}\s+\s\dots
  \nonumber
\qqq
and retaining only the terms which
contribute to the residue at \s$z=w\s$ in the last
integral of eq.\s\s(\ref{ChC}),  we obtain
\begin{eqnarray*}
(\s\CY'\m,\s[L_n\m,\s\CT(w)]\m\s\CY\s)\s=\s
\langle\m\s(\m\Theta\s Y'\m)\s\s\m\{\s\frac{_c}{^{12}}\s(n^3-n)\s
w^{n-2}\s+\s2\m(n+1)\s w^{n}\s T(w)\\
+\s w^{n+1}\s\m\partial_w\m T(w)\s\}\s\m Y\m\s\rangle
\end{eqnarray*}
\vs-0.3cm
\no which is the weak form of relations
\qq
[L_n\m,\s\CT(w)]\s=\s\frac{_c}{^{12}}\s(n^3-n)\s
w^{n-2}\s+\s2\m(n+1)\s w^{n}\s\CT(w)\s+\s
w^{n+1}\s\m\partial_w\m\CT(w)\ .\hs{0.5cm}
\label{CoMm}
\qqq
Similarly, the OPE (\ref{OPE2}) implies that
\qq
[\tilde L_n\m,\s{\bar\CT}(\bar w)]\s=
\s\frac{_c}{^{12}}\s(n^3-n)\s
{\bar w}^{n-2}\s+\s2\m(n+1)\s{\bar w}^{n}\s
{\bar\CT}(\bar w)\s+\s
{\bar w}^{n+1}\s\m\partial_{\bar w}\m
{\bar\CT}(\bar w)\ .\hs{0.5cm}
\label{barCoMm}
\qqq
By virtue of eq.\s\s(\ref{OPEC}), the mixed commutators
\s\s$[L_n\m,\s{\bar\CT}(\bar w)]\s\s$ and
\s\s$[\tilde L_n\m,\s\CT(w)]\s\s$ vanish.
\vs 0.6cm

Performing a contour integral over \s$w\s$ on both sides
of eq.\s\s(\ref{CoMm}) multiplied by \s$z^{m+1}\s$, we
obtain the commutation relations
\qq
  [L_n\m,\s L_m]\s=\s(n-m)\s L_{n+m}\s
  +\s\frac{_c}{^{12}}\m(n^3-n)\s\delta_{n+m,0}\ .
  \label{Viras}
\qqq
The (infinite-dimensional) Lie algebra with
generators \s$L_n\s$ and a central element
\s$\CC\s$ (called the {\bf central charge})
and with relations (\ref{Viras}), where \s$c\s$
is replaced replaced by \s$\CC\s$, \s
is known as the {\bf Virasoro algebra}.
We shall denote it by \s${Vir}\s$. \s
It is closely related to the (Witt) Lie algebra of
polynomial vector fields \s${Vect}(S^1)\s$ on
the circle \s$\{\m\s|z|=1\s\m\}\s$
with generators \s$l_n=-z^{n+1}\m\da_z\s$
and relations
\qq
[l_n\m,\s l_m]\s=\s(n-m)\s l_{n+m}\ .
\non
\qqq
More exactly, \s${Vir}\s$ is a central extension
of \s${Vect}(S^1)\s$, i.\s e. we
have an exact sequence of Lie algebras
\qq
0\ \longrightarrow\ \NC\ \longrightarrow\ {Vir}\
\longrightarrow\
{Vect}(S^1)\ \longrightarrow\ 0\ ,
\non
\qqq
where the second arrow sends \s$1\s$ to \s$\CC\s$ and the third
one maps \s$L_n\s$ to \s$l_n\s$.
\vs 0.5cm

Eq.\s\s(\ref{barCoMm}) gives rise to
another set of Virasoro commutation relations
\qq
  [\tilde L_n\m,\s\tilde L_m]\s=\s(n-m)\s\tilde L_{n+m}\s
  +\s\frac{_c}{^{12}}\m(n^3-n)\s\delta_{n+m,0}\ .
  \label{barViras}
\qqq
$L_n$'s and \s$\tilde L_m$'s commute. \s Both
(\ref{Viras}) and \s$(\ref{barViras})\s$
hold on the invariant dense domain \s$\CH_1\in\CH\s$,
\s As we see, the Hilbert space of
states \s$\CH\s$ of a CFT
carries a densely defined unitary
(i.\s e. with property (\ref{Adjoints})\m)
\m representation of the algebra
\s${Vir}\oplus{Vir}\s$ with central charges
acting as the multiplication by \s$c\s$.
\vs 0.5cm

The representation
theory of the Virasoro algebra has played
an important role in the construction of models of CFT. We shall
include for completeness a brief sketch of its elements in the next
section. But why did the Virasoro algebra
appear in CFT in the first place? As we have
mentioned, \s${Vir}\s$ is the central extension
of an algebra of vector fields
on the circle. But \s${Vect}(S^1)
\oplus{Vect}(S^1)\s$
may be identified with the Lie algebra
of (polynomial) conformal vector fields
on the two-dimensional cylinder \s$\{\s(t,x)\ \s
|\ \s x\s\m{\rm mod}\s\m 2\pi\s\}\s$
with the Minkowski metric
\s$\ga_M\equiv dt^{\m2}-dx^{\m2}\s$.
By definition, the conformal vector fields
\s$X\s$ satisfy \s$\CL_X\m \ga_M=f_X\m \ga_M\s$ for
some function \s$f_X\s$, \s where \s$\CL_X\s$ denotes
the Lie derivative w.r.t. $X\s$. \s The identification
assigns to generators \s$l_n\s$ and \s$\bar l_n\s$
the conformal vector fields \s$-z^{n+1}\m\da_z\s$
and \s$-{\bar z}^{\m n+1}\m\da_{\bar z}\s$, \s respectively,
with \s$z\equiv\ee^{\m i(t+x)}\s$ and \s$\bar z\equiv
\ee^{i(t-x)}\s$. \s In particular,
\s$i(l_0+\bar l_0)\s$ is the infinitesimal shift
of the Minkowski time \s$t\s$ and \s$i(l_0-\bar l_0)\s$
the infinitesimal shift of \s$x\s$. \s Hence
\s${Vect}(S^1)\oplus{Vect}(S^1)\s$
is the Lie algebra of Minkowskian conformal symmetries
and representations of \s${Vir}\oplus{Vir}\s$
describe its projective actions realizing such conformal
symmetries on the quantum-mechanical level (projective representations
correspond to genuine actions of symmetries on the
rays in the Hilberts space representing (pure) quantum states).
\s$H\equiv L_0+\tilde L_0\s$
is the quantum Hamiltonian\footnote{later, we shall see that
it is more natural to shift \s$H\s$ by a constant}
and \s$P=L_0-\tilde L_0\s$ is the quantum momentum operator.
The unitarity conditions (\ref{Adjoints}) correspond to the natural
real form of the algebra composed of real vector fields:
such vector fields are represented by skew-adjoint
operators so that the corresponding global conformal
transformations act by unitary operators. \s${Vect}(S^1)
\s$ may be viewed as the Lie algebra of the group
\s${Diff}_+(S^1)\s$ of orientation preserving
diffeomorphisms of the circle. Let \s$\widetilde{Diff}_+
(S^1)\s$ denote the group of diffeomorphisms of the line
commuting with the shifts by \s$2\pi\s$.
$$0\s\longrightarrow\s\NZ\s\longrightarrow\s
\widetilde{Diff}_+(S^1)\s\longrightarrow\s
{Diff}_+(S^1)\s\longrightarrow\s0\ .$$
The group \s$\CD\s\equiv\s(\widetilde{Diff}_+
(S^1)\times\widetilde{Diff}_+
(S^1))\s/\s\NZ_{\rm diag}\s$ (which acts
on the light-cone variables \s$x^\pm\equiv t\pm x\s$
is the group of conformal,
orientation and time-arrow preserving diffeomorphism
of the Minkowski cylinder. \s$Vir\oplus Vir$\s action
in \s$\CH\s$ integrates to the projective
unitary representation of \s$\CD\s$.
\vs 0.6cm


We shall need more information about
the representations of \s${Vir}\times{Vir}\s$
which appear in CFT. This may be obtained by studying
the behavior of the primary field operators
with respect to the Virasoro algebra action.
\vs 0.6cm

\no{\bf Problem 4}.\ \ (a).\ \s Show by employing the
contour integral technique that
\qq
  &&[L_n\m,\s{\varphi}_l(w,\bar{w})]\m
  \s=\s\Delta_l\s(n+1)\s
  \m w^n\s{\varphi}_l(w,\bar{w})\s+\s w^{n+1}\s\partial_w
  {\varphi}_l(w,\bar{w})\ ,\label{Prime}\\
  &&[\tilde L_n\m,\s{\varphi}_l(w,\bar{w})]\s=\s\bar
  \Delta_l\s(n+1)\m\s{\bar w}^n\s{\varphi}_l(w,\bar{w})\s
  +\s{\bar w}^{n+1}\s\partial_{\bar w}
  \m{\varphi}_l(w,\bar{w})\ .\label{barPrime}
\qqq
(b). \s Using the above relations and eqs.\s\s(\ref{CoMm})
and (\ref{barCoMm}) prove that the operators \s$L_0\s,$
\s$\tilde L_0\s$ given by (\ref{Ln}) and
(\ref{barLn}) coincide with the generators of the semigroup
\s$(\CS_q)\s$ introduced earlier.
\vs 0.6cm

\no Eqs.\s\s(\ref{Prime}) and (\ref{barPrime}) express
on the operator level the properties of the (Virasoro)
primary fields of conformal weights \s$(\Delta_l,\m
\tilde\Delta_l)\s$. \s Comparing them to the last two equations
of Lecture 1, we infer that operators $\varphi_l(w,\bar w)$
for $(w,\bar w)=(\ee^{-t+ix},\ee^{-t-ix})$ should be
interpreted as the imaginary time versions of Minkowski
fields. Note that the components \s$\CT(z)\s$
and \s${\bar\CT}(\bar z)\s$ fail to be
Virasoro primary fields of weights \s$(2,0)\s$
and \s$(0,2)\s$, \s respectively,
due to the anomalous term proportional
to \s$c\s$ in the relations
(\ref{CoMm}) and (\ref{barCoMm}).
\vs 0.4cm

Recall, that (as a generator of a self-adjoint semigroup
of contractions) the self-adjoint operator \s$H=L_0+\tilde L_0\s$
has to be positive. In Minkowskian QFT with
Poincare invariance the positivity of the Hamiltonian
implies the spectral condition \s$H\pm P\geq 0\s$
where \s$P\s$ is the momentum operator.
The same is true in CFT with its Hilbert space
corresponding to cylindrical Minkowski
space. The Virasoro commutation relations imply,
\qq
L_0\s\geq\s0\ ,\hs{1cm}\tilde L_0\s\geq\s0\ .
\label{SpectCon}
\qqq
Indeed. Let \s$E_{B}\s$ be a non-vanishing
joint spectral projector of \s$L_0\s$ and \s$\tilde L_0\s$
corresponding to eigenvalues in a small ball \s${B}\s$,
with the \s$L_0\s$ eigenvalues negative and
such that \s$E_{{B}-(1,0)}\s=\s0\s$.
\s Then, for any normalized vector \s$\psi\s$ with
\s$E_{B}\m\psi=\psi\s$, we have
$$L_1\m\psi=L_1\s E_{B}\m\psi=E_{{B}-(1,0)}\s
L_1\m\psi=0\ .$$
On the other hand,
$$0\leq(\m L_{-1}\m\psi\m,\s L_{-1}\m\psi\m)=(\m
\psi\m,\s L_1\m L_{-1}\m\psi\m)=(\m\psi\m,
[L_1,L_{-1}]\m\psi\m)=2\m(\m\psi\m,\s L_0\m\psi\m)<0$$
which shows that \s$L_0\s$ cannot have negative spectrum.
Similarly for \s$\tilde L_0\s$. \s Hence only {\bf
positive energy representations} of the Virasoro
algebra with \s$L_0\geq0\s$ (\s$\tilde L_0\geq 0\s$)
\s appear in CFT with the Hilbert space
interpretation.
The techniques of CFT apply, however, also to certain
scaling limits of statistical mechanical models
without physical positivity where a wider class
of Virasoro representations intervenes.
\vs 0.4cm


Relations (\ref{Prime}) and (\ref{barPrime}) provide
further spectral information about \s$L_0\s$ and \s$\tilde L_0\s$.
\s They imply the equalities
\qq
&&\hbox to 6cm{$L_n\s\Omega\s=\s0\ ,\ \ \ n\geq-1$\hfill}
\hbox to 6cm{$\tilde L_n\s\Omega\s=\s0\ ,\ \ \ n\geq-1$\hfill}\\
&&\hbox to 6cm{$L_n\s\varphi_l(0)\m\s\Omega\s=\s0\ ,
\ \ \ n>0\ ,$\hfill}\hbox to 6cm{$\tilde L_n\s\varphi_l(0)
\m\s\Omega\s=\s0\ ,\ \ \ n>0\ ,$\hfill}\\
&&\hbox to 6cm{$L_0\s\varphi_l(0)\m\s\Omega\s=
\s\Delta_l\s\varphi_l(0)\s\m\Omega\ ,$\hfill}
\hbox to 6cm{$\tilde L_0\s\varphi_l(0)\s\Omega\s=
\s\tilde\Delta_l\s\varphi_l(0)\s\m\Omega\ \s$\hfill}
%&&\hbox to 6cm{$L_1\s\varphi_l(0)\s\m\Omega\s=
%\s\da_w\m\varphi_l(0)\m\s\Omega\ $\hfill}
%\hbox to 6cm{$\tilde L_1\s\varphi_l(0)\m\s\Omega\s=
%\s\da_{\bar w}\m\varphi_l(0)\m\s\Omega\ ,$\hfill}
\non
\qqq
where, by definition, \s$\varphi_l(0)\m\s\Omega\s=
\s\lim\limits_{z\to 0}\ \varphi_l(z,\bar z)\s\m\Omega\s$.
In particular, it follows that the vacuum
vector \s$\Omega\s$ is an
eigenvector of \s$L_0\s$ and \s$\tilde L_0\s$ with
the lowest possible eigenvalues \s$0\s$.
We shall assume that it is a unique vector, up
to normalization, with this property (although there are
CFTs without this property). In fact, \s$\Omega\s$ is annihilated
by the \s$sl_2\times sl_2\s$ Lie subalgebra generated by
\s$L_0,\s L_{\pm1},\s\tilde L_0\s$ and \s$\tilde L_{\pm1}\s$ but not
by the entire symmetry algebra \s${Vir}\times{Vir}\s$
of the theory: the conformal symmetry is spontaneously broken.
\vs 0.4cm

$\varphi_l(0)\s\m\Omega\s$ are also eigenvectors
of \s$L_0\s$ and \s$\tilde L_0\s$, with eigenvalues
\s$(\Delta_l,\m\tilde\Delta_l)\s$ and it follows that
\s$\Delta_l,\ \tilde\Delta_l>0\s$.
\s Also \s$\CI\s\Omega=\Omega\s$ and \s$\CI\s\varphi_l(0)\s\Omega
=\varphi_{\m\bar l\m}(0)\s\Omega\s$. \s In fact, vectors
\s$\s\varphi_l(0)\s\m\Omega\s$ are annihilated by all
Virasoro generators with positive indices.
The eigenvectors of \s$L_0,\ \tilde L_0\s$ (and \s$\CC\s$)
with such property are called Virasoro highest weight (HW)
vectors.
\vs 0.4cm

Summarizing, \s we have shown that the Hilbert space
of states in a CFT carries a (densely defined)
positive energy representation
of two commuting copies of the Virasoro algebra
with the same central charge \s$c\s$.
The primary field operators \s$\varphi_l(z,\bar z)\s$
applied to the vacuum become in the limit \s$z\to 0\s$
HW vectors of the Virasoro representations.
\vs 1.1cm



\no{\bf 5.\ \s Highest weight representations of $Vir$}
\vskip 0.4cm

\no For completeness, we include a brief sketch
of representation theory of the Virasoro algebra.
\vskip 0.3cm

An important class of representations
of the Virasoro algebra is constituted by the
so called {\bf highest weight} (HW) representations.
Let \s$\theta=\NC L_0 \oplus\NC \CC\s$, \s$\CN_+ =\bigoplus
\limits_{n=1}^\infty \NC L_{n}\s$,
\s$\CN_- =\bigoplus\limits_{n=1}^\infty \NC L_{-n}\s$.
\s${Vir} = \CN_-
\oplus \theta\oplus \CN_+$ is the triangular decomposition
of the Virasoro algebra.
%By Poincare-Birkhoff-Witt theorem, the enveloping algebra
%\s$\CU({Vir})\cong\CU(\CN_-)\otimes\CU(\theta)\otimes
%\CU(\CN_+)\s$ and \s$\CU(\CB_\pm)\cong
%\CU(\CN_\pm)\cong\CU(\theta)\s$.
Let \s$\lambda\in\theta^*\s$, the
dual space to \s$\theta\s$, \s$\lambda(\CC)=c\m,$ \s$\lambda
(L_0)=\Delta\s$. \s A
${Vir}$-module (representation) \s$M_{c,\Delta}\s$ is
called a HW module of HW \s$\lambda\s$
if there exists a vector
\s$v_0\in V\s$ such that
\qq  \begin{array}{lcl}
     \CN_+\m v_0           & = & 0\ ,         \\
     {\cal U}(\CN_-)\m v_0 & = & M_{c,\Delta}\ ,        \\
     x\s v_0             & = & \lambda(x)\s
     v_0 \quad {\rm for}\quad x\in \theta\ \s
    \end{array}
\non
\qqq
where \s$\CU(\s\cdot\s)\s$ denotes the enveloping
algebra. $v_0\s$ is called the HW vector, \s$c\s$ the
central charge and \s$\Delta\s$ the
conformal weight of the HW representation.
It follows that \s$M_{c,\Delta}\s$ is the
linear span of the vectors \s$L_{-n_r}L_{-n_{r-1}}\cdots
L_{-n_1}v_0\s$ with \s$0<n_1\leq n_2\leq\ldots\leq n_r\s$,
but these vectors are not
necessarily linearly independent. \s$N=\sum\limits_{i=1}^r n_i\s$
is called the level
of the vector \s$L_{-n_r}L_{-n_{r-1}}\cdots L_{-n_1}v_0\s$.
A level \s$N\s$ vector is an
eigenvector of \s$L_0\s$ with eigenvalue \s$N+\Delta\s$.
We shall denote the subspace of the level \s$N\s$ vectors
by \s$M_{c,\Delta}^{(N)}\s$.
\s Clearly vectors of different
levels are linearly independent, thus we have
\qq
 M_{c,\Delta} = \bigoplus_{N=1}^\infty M_{c,\Delta}^{(N)}
\non
\qqq
with the \s$\mbox{dim}\m M_{c,\Delta}^{(N)}$$\leq p(N)\s$,
\s the partition number of \s$N\s$. \s A HW
module with \s$\mbox{dim}(M_{c,\Delta}^{(N)})$
$=\s p(N)\s$ for all
\s$N\s$, \s i.\s e. where all
the vectors of the form \s$L_{-n_r}L_{-n_{r-1}}\cdots
L_{-n_1}v_0\s$ (with ordered
\s$n_i\s$'s) are linearly independent is called the Verma module
\s$V_{c,\Delta}\s$. It exists for all complex
\s$c\s,\ \Delta\s$
and is unique up to isomorphisms. In the analysis of
the HW modules of the Virasoro algebra an
important role is played by {\bf singular vectors}.
A non-zero vector \s$v_s^{(N)}\s$ of level \s$N\s$
is called singular, if \s$L_n v_s^{(N)}
=0\s$ for all \s$n>0\s$. Any vector \s$v_s\s$ with
\s$L_n v_s =0\s$ for all \s$n>0\s$ is a
a sum of singular vectors (its non-zero homogeneous components).
Any singular vector \s$v_s^{(N)}\s$ generates a submodule
in \s$V_{c,\Delta}\s$
isomorphic to \s$V_{c,\Delta+N}\s$. We have:
\vs 0.4cm


 \renewcommand{\labelenumi}{(\roman{enumi})}
 \begin{enumerate}
 \item Any submodule of \s$V_{c,\Delta}\s$ is generated by
       its singular vectors.
 \item Any HW module \s$M_{c,\Delta}\s$ is isomorphic
       to a quotient of the Verma module \s$V_{c,\Delta}\s$.
 \item The factor module of the Verma module by the maximal proper
       submodule is the unique (up to isomorphisms)
       irreducible HW module \s$H_{c,\Delta}\s$.
 \end{enumerate}

\vs 0.4cm


A \s$Vir\s$-module \s$M\s$ is called unitary, if there exists
a (positive, hermitian) scalar product \s$(\s\cdot\s,\s\cdot\s)\s$
on  \s$M\s$ s.\s t.
\qq
 (v, L_n w)\s=\s(L_{-n} v,w)\ \ {\rm for\ all}\ n\in\NZ\ {\rm and\
for\ all}\ \s v,w \in V\ .
\label{Unita}
\qqq
It follows then that \s$( v, \CC w )\s =\s( \CC v,w)\s$
for all \s$v,w \in V\s$ and that the eigenvalues of \s$\CC\s$
and \s$L_0\s$ are real. \s On each Verma module
\s$V_{c,\Delta}\s$ with \s$c\s$ and \s$\Delta\s$ real
there exists a unique hermitian (Shapovalov) form
\s$\langle\, \cdot\, ,\, \cdot\,\rangle\s$ s.\s t.
\s$\langle v,L_n w\rangle =$$\langle L_{-n} v,w \rangle\s$
for all \s$v,w \in V_{c,\Delta}\s$ and that
\s$\langle v_0 ,v_0 \rangle =1\s$.
\s Define \s$N\!ull_{c,\Delta} =\s$ \s$\left\{ v\in
V_{c,\Delta}\s\m |\s\m \langle v ,w \rangle = 0\ \s{\rm for\ all}
\ \s w\in V_{c,\Delta}\right\}\s$.
\vs 0.5cm


\renewcommand{\labelenumi}{(\roman{enumi})}
 \begin{enumerate}
  \item \s$ \langle v^{(N)},v^{(N')} \rangle = 0 \ \s$ if
	 \ \s$v^{(N)}\s$ (\s$v^{(N')}\s\s$)
	 is a level \s$N\s$ (\s$N'\s$)
	 vector and \s$N\not= N'\s$,
  \item  any singular vector of positive level belongs
	 to \s$N\!ull_{c,\Delta}\s$.
  \item  \s$N\!ull_{c,\Delta}\s$ is the
	 maximal proper submodule
	 of \s$V_{c,\Delta}\s$, \s i.\s e.
	 \s$H_{c,\Delta} = V_{c,\Delta}/N\!ull_{c,\Delta}\s$
 \end{enumerate}

\vs 0.6 cm


Let us investigate the conditions under which
\s$H_{c,\Delta}\s$ is a unitary \s${Vir}$-module,
or, equivalently, under which the hermitian form
on \s$H_{c,\Delta}\s$ induced by the Shapovalov form
is positive. Since \s$\left<L_{-n} v_0 ,L_{-n} v_0\right>
=\s$ \s$2n\Delta + \frac{c}{12}(n^3-n)\s$,  necessarily
$c\geq 0\s$ and \s$\Delta\geq 0\s$.
Now consider the two \s$2n\s$ level
vectors \s$v = L_n^{\ 2}v_0\s$ and \s$w=L_{-2n}v_0\s$ and suppose
that \s$c=0\s$. The Gram
determinant
\qq
  \det\left(\begin{array}{cc}
     \left<v,v\right> & \left<v,w\right> \\
     \left<w,v\right> & \left<w,w\right>
 \end{array}\right) = -20\s n^4 \Delta^2+32\s n^3 \Delta^3\ .
\non
\qqq
Thus for \s$c=0\s$ a necessary (and sufficient)
condition is \s$\Delta=0\s$. \s$c,\Delta=0\s$ correspond to the trivial
one-dimensional representations.
So there exists no
non-trivial unitary HW representation of the Lie algebra
\s${Vect}(S^1)\s$ of the (polynomial) vector fields on
the circle. It is enough
to study the positivity of the Shapovalov form restricted
the the subspaces \s$H^{(N)}_{c,\Delta}\s$ of the
given level. Let \s$m^{(N)}_{c,\Delta}\s$ denote the
\s$p(N)\times p(N)\s$ matrix
with the entries \s$\langle\s L_{-n_r}L_{-n_{r-1}}\cdots
L_{-n_1}v_0\m,\s L_{-n'_r}L_{-n'_{r-1}}\cdots
L_{-n'_1}v_0\s\rangle\s$ where \s$(n_i)\s$ and \s$(n'_i)\s$
are ordered and \s$\sum n_i=\sum n'_i=N\s$.
Unitarity of \s$H_{c,\Delta}\s$ is equivalent to
the conditions  \s$m_{c,\Delta}^{(N)}\geq 0\s$ for all \s$N\s$.
In particular, \s$\det\m( m_{c,\Delta}^{(N)})\s$
has to be non-negative for all \s$N\s$ when \s$H_{c,\Delta}\s$
is unitary. Direct calculation for level \s$1\s$ and
\s$2\s$ gives
\qq \begin{array}{lcl}
     \det\m( m_{c,\Delta}^{(1)}) & = & 2\Delta\ , \\
     \det\m( m_{c,\Delta}^{(2)}) & = &
     4\Delta\left(\frac{1}{2}c+(c-5)\Delta+8\Delta^2\right)\ .
   \end{array}
\non
\qqq
A general formula for the determinant of \s$m_{c,\Delta}^{(N)}\s$
was given by Kac and was proven by Feigin and Fuchs:
\qq
 \det\m( m_{c,\Delta}^{(N)})
 \s\s=\s\s \kappa_N\prod_{{1\leq rs\leq N\atop r,s\in \NN}}
		\left(\Delta-\Delta_{r,s}(m)\right)^{p(N-rs)}
\non
\qqq
where \s$\Delta_{r,s}(m)=\frac{((m+1)r-ms)^2-1}{4m(m+1)}\s$,
\s\s$m\s$ is a root of the equation \s\s$c = 1-\frac{6}{m(m+1)}\s,
\s$ \s and \s\s$\kappa_N\s =
\prod\limits_{{1\leq rs\leq N\atop r,s\in \NN}}\left((2r)^s
s!\right)^{n(r,s)}\s$ with
\s$n(r,s)\s$ denoting the number of partitions
of \s$N\s$ in which \s$r\s$ appears \s$s\s$ times.
\vs 0.5cm

For \s$\Delta\to\infty\s$, \s\s$m_{c,\Delta}^{(N)}\s$ goes
to a diagonal matrix with positive entries. From the Kac formula
it follows that \s$\det\m( m_{c,\Delta}^{(N)})>0\s$
for \s$c>1\s$, \s$\Delta>0\s$. Therefore \s$m_{c,\Delta}^{(N)}\s$
is non degenerate and
positive for \s$c>1\s$, \s$\Delta>0\s$ and non-negative for
\s$c\geq 1\s$, \s$\Delta\geq 0\s$.
Thus \s$V_{c,\Delta}\s$ is irreducible for \s$c>1\s$,
\s$\Delta>0\s$
and \s$H_{c,\Delta}\s$ is unitary for \s$c \geq 1\s$,
\s$\Delta\geq 0\s$. \s Since, for \s$c=1\s$
\qq
 \det\m( m_{1,\Delta}^{(N)})\s =\s \kappa_N\prod_{1\leq rs\leq N}
	  \left(\Delta-\frac{(r-s)^2}{4}\right)^{p(N-rs)}\ ,
\non
\qqq
it follows that \s$V_{1,\Delta}\s$ is irreducible if and only
if \s$\Delta\neq \frac{n^2}{4}\s$,
\s$n\in\NZ\s$.
\vs 0.6cm

For $0\leq c< 1$, the
situation is more interesting providing the
first example of the selective power of conformal invariance.
\vs 0.4cm

\no{\bf Theorem}. \ For \s$(c,\Delta)\s$ with \s$0\leq c<1\s$
and \s$\Delta\geq 0\s$,
unitarity of the irreducible HW representation \s$H_{c,\Delta}\s$
requires that \s$(c,\Delta)\s$ belong to the following discrete
series of points:
\[ \left.
 \begin{array}{lcl}
 {\quad\s c}         &=& 1-\frac{6}{m(m+1)}  \\
	   & &                     \\
 \Delta_{r,s}(m)&=&\frac{((m+1)r-ms)^2-1}{4m(m+1)}
 \end{array} \right\}
		   \begin{array}{l} m=2,3,\ldots\ , \\
				 1\leq r\leq m-1\ , \\
				 1\leq s\leq r\ .
		   \end{array}
\]

\vs 0.4cm


\no The theorem was proven by Friedan-Qiu-Shenker
by a careful analysis of the geometry of lines
$(c(m),\ \Delta_{r,s}(m))$ in the \s$(c,\Delta)\s$ plane.
This, in conjunction with the Kac
determinant formula, allowed subsequent elimination
of portions of the \s$(c,\Delta)\s$ plane were negative norm
vectors appear, by an induction on \s$N\s$.
\s At the end, only the points listed above were left.
Goddard-Kent-Olive constructed explicitly unitary irreducible
HW representations of the Virasoro algebra
for the above series of \s$(c,\Delta)\s$
employing the so called "coset construction".
\vs 0.4cm

All unitary HW representations integrate to
projective unitary representations of \s$Diff_+(S^1)\s$
in the Hilbert space completion of \s$H_{c,\Delta}\s$.
\vs 0.4cm

All unitary representations \s$M\s$ of the Virasoro
algebra s.\s t. (the closure of) \s$L_0\s$ is a
positive self-adjoint operator with a discrete
spectrum of finite multiplicity in the Hilbert
space completion \s${\overline M}\s$ of \s$M\s$ is essentially
isomorphic to a direct sum of unitary representations
(the isomorphism may require a different choice
of the common invariant dense domain for \s$L_n$'s
in \s${\overline M}\s$)\m. \m We shall see in the next section
that in CFT the operators \s$\CS_q=q^{L_0}
\m\bar q^{\tilde L_0}\s$ should be trace class for \s$|q|<1\s$
from which it follows that \s$L_0\s$ and \s$\tilde L_0\s$
have a discrete spectrum of finite multiplicity.
Hence the Virasoro algebra representations which
appear in CFT are direct sums of the unitary HW
representations.
\vskip 0.3cm

The algebra \s$Vect(S^1)\oplus Vect(S^1)\s$, and hence
also \s$Vir\oplus Vir\s$ acts naturally on the space
\s$\Lambda^{\Delta\tilde\Delta}\s$
of \s$(\Delta,\tilde\Delta)$-forms \s$f\s(dz)^{\Delta}
(d\bar z)^{\tilde\Delta}\s$ on \s$D\setminus\{0\}\s$ by
Lie derivatives:
\qq
l_n\s f\s=\s-\Delta\m(n+1)\s z^n\m f\s-\s z^{n+1}\m\da_{z}f\s,
\quad\quad\bar l_n\s f\s=\s-\tilde\Delta\m(n+1)\s\bar z^n\m f\s
-\s\bar z^{n+1}\m\da_{\bar z}f\s.
\non
\qqq
The commutation relations (\ref{Prime}) and (\ref{barPrime})
express the fact that the operators $\s\varphi_l\s$
intertwine the action of \s$Vir\oplus Vir\s$ in \s$\CH_1\subset\CH\s$
and $\CH_1\otimes\Lambda^{\Delta_l\tilde\Delta_l}\s$. This
gives a representation theory interpretation of the primary
fields of CFT.
\vs 1.1cm




\no{\bf 6.\ \s Segal's axioms and vertex operator algebras}
\vskip 0.4cm

\no Up to now we have studied QFT on
closed Riemann surfaces.
Let us consider now a compact Riemann surface
\s$\Sigma\s$ (connected or disconnected)
with the boundary composed of the
connected components \s$\CC_i\s$, \s$i\in I\s$.
We shall parametrize each \s$\CC_i\s$
in a real analytic way
by the standard unit circle \s$S^1\subset\NC\s$.
Components \s$\CC_i\s$ may be divided
into ``in'' and ``out'' ones,
depending on whether the parametrization
disagrees or agrees with the orientation of \s$\CC_i\s$
inherited from \s$\Sigma\s$. This induces
the splitting \s$I=I_{\rm in}\cup I_{\rm out}\s$
of the set of indices \s$i\s$.
\s We may invert the orientation of \s$\CC_i\s$
by composing its parametrization
with the map \s$z\mapsto z^{-1}\s$ of
\s$S^1\s$. \s To \s$\Sigma\s$ with parametrized
boundary, we may uniquely assign a compact surface
\s$\hat\Sigma\s$ without boundary by gluing a copy
of disc \s$D\s$ to each boundary component \s$\CC_i\s$
with \s$i\in I_{\rm in}\s$ and
a copy of disc \s$D'\s$ to each
\s$\CC_i\s$ with \s$i\in I_{\rm out}\s$.
\s Conversely, given a closed Riemann surface
\s$\hat\Sigma\s$ with holomorphically
embedded disjoint discs \s$D\s$
and \s$D'\s$ (``local parameters''),
by removing their interiors we obtain the surface
\s$\Sigma\s$ with boundary parametrized by the
standard circles \s$\{\s\m|z|=1\s\}\s$.
\s We shall call a metric
\s$h\s$ on \s$\Sigma\s$ (compatible with
its complex structure) flat at boundary if,
for each \s$i\s$, \s it is of the
form \s$|z|^{-2}|dz|^2\s$ in the
local holomorphic coordinate
around \s$\CC_i\s$ extending its parametrization.
We shall consider only such metrics on surfaces
with boundary.
\vs 0.4cm

\s First, let us note that a metric \s$\ga\s$ on \s$\Sigma\s$
and metrics \s$\ga_i\s$ on \s$D\s$, all flat
at boundary, give rise by gluing to the metric
\addtocounter{subsection}{2}
\qq
\tilde \ga\s=\s(\vee_{i\in I_{\rm in}}\ga_i)\vee
\ga\vee(\vee_{i\in I_{\rm out}}\vartheta^*\ga_i)
\non
\qqq
on \s$\hat\Sigma\s$. It follows from the local scale
covariance formula (\ref{PFWI}) that
the combination of partition functions
\qq
Z_{\tilde\ga}\ \s\prod\limits_{i\in I}\s
(Z_{\vartheta^*\ga_i\vee\ga_i})^{-1/2}\s\ \equiv\ \s Z_{\ga}
\label{PFda}
\qqq
is independent of (the conformal factors) of \s$\ga_i\s$.
Besides, the transformation of \s$Z_\ga\s$
under \s$\ga\mapsto\ee^\sigma\ga\s$ with
\s$\sigma\s$ vanishing around the boundary
is still given by eq.\s\s(\ref{PFWI}).
It is sensible to call \s$Z_{\ga}\s$
the partition function of the Riemann surface
\s$\Sigma\s$ with boundary. Let
\s$\CY_i\in\CH_0\s$. Consider the matrix elements
defined by
\qq
(\s\otimes_{i\in I_{\rm out}}\CY_i\s\m,\ A_{_{\Sigma,\ga}}\s\m
\otimes_{i\in I_{\rm in}}\CY_i\s\s)\ =\
Z_\ga\s\s\langle\s\prod\limits_{i\in I_{\rm out}}(\Theta Y_i)\s\s
\prod\limits_{i\in I_{\rm in}}Y_i\s\rangle\ .
\label{MatEle}
\qqq
We shall postulate that the above formula defines operator
``amplitudes'' \s$A_{\Sigma,\gamma}\s$  mapping the tensor
products of the CFT Hilbert spaces associated to the boundary components
of \s$\Sigma\s$. In particular, for \s$\Sigma\s$ without boundary,
eq.\s\s(\ref{MatEle}) should be read as the identity
\s$A_{{\Sigma,\ga}}\s=\s Z_\ga\s$. \s Amplitudes \s$A_{\Sigma,\gamma}\s$
(or rather their holomorphic counterparts) were considered as the
defining data of CFT in the work of Segal. Let us state the Segal
axioms (adapted to the real setup).
\vs 0.3cm

\renewcommand{\labelenumi}{(\bf\roman{enumi})}
\begin{enumerate}
\item
We are given the Hilbert space of states \s$\CH\s$ with
an anti-unitary involution \s$\CI\s$ and for each
compact Riemann surface (with parametrized boundary
or without boundary,
connected or disconnected) the operator\footnote{the
empty tensor product should be interpreted
as \s$\NC\s$}
\qq
A_{_{\Sigma,\ga}}\s:\s\s
\bigotimes\limits_{i\in I_{\rm in}}\CH\ \longrightarrow\
\bigotimes\limits_{i\in I_{\rm out}}\CH\ ,
\label{Ax1}
\qqq
which we assume trace class.
\item
If \s$\Sigma\s$ is a disjoint union of \s$\Sigma_1\s$
and \s$\Sigma_2\s$, then
\qq
A_{_{\Sigma,\ga}}\s=\s A_{_{\Sigma_1,\ga}}\otimes A_{_{\Sigma_2,\ga}}\ .
\non
\qqq
\item
If \s$D:\Sigma_1\rightarrow\Sigma_2\s$ is a diffeomorphism
reducing to identity around the (parametrized) boundary then
\qq
A_{_{\Sigma_2,\ga}}\s=\s A_{_{\Sigma_1,D^*\ga}}\ .
\non
\qqq
\item
If \s$\bar\Sigma\s$ denotes the Riemann surface
with conjugate complex structure (and opposite
orientation) then
\qq
A_{_{\bar\Sigma,\ga}}\s=\s A_{_{\Sigma,\ga}}^{\dagger}
\non
\qqq
\item The inversion of boundary parametrization
acts on the amplitude \s$A_{_{\Sigma,\ga}}\s$
by the isomorphism \s$\CH\cong\CH^*\s$
induced by \s$\CI\s$ in the corresponding Hilbert space
factor.
\item
If \s$\Sigma'\s$ is obtained from
\s$\Sigma\s$ by gluing of the \s$\CC_{i_0}\s$ and
\s$\CC_{i_1}\s$ boundary components then
\qq
A_{_{\Sigma',\ga}}\s=\s\m \tr_{i_0,i_1}\ A_{_{\Sigma,\ga}}\ ,
\non
\qqq
where \s$\tr_{i_0,i_1}\s$ denotes the partial trace
in the \s$\CH\s$ factors corresponding
to $\CC_{i_0}\s$ and \s$\CC_{i_1}\s$.
\item
For any function \s$\sigma\s$ on \s$\Sigma\s$
vanishing around the boundary
\qq
A_{_{\Sigma,\ee^\sigma \ga}}\s=\s\s
\ee^{\m\frac{{c}}{{96\pi}}\s(\m\|d\sigma\|^2_{L^2}+4\int_\Sigma
\sigma\m r\s dv\m)}\ A_{_{\Sigma,\gamma}}
\non
\qqq
\end{enumerate}
\vs 0.4cm

In the approach in which we start from the data
specifying partition functions and correlation functions
on surfaces without boundaries, property ({\bf i}),
in conjunction with eq.\s\s(\ref{MatEle}), imposes certain
new regularity conditions on the correlation functions.
Property ({\bf ii}) may be viewed as a definition of
the amplitudes for disconnected surfaces.
All the remaining properties except for ({\bf{vi}}) follow
easily. The gluing axiom brings an essential novel element
allowing to compare the correlation functions of different
surfaces (with different complex structures and different
topologies). In the heuristic
approach employing functional integrals, it expresses locality
of the latter. Let us explain this point in more detail.
\vskip 0.3cm


We shall think about the partition function of a QFT
on a closed Riemann surface as being
given by a functional integral of the type
\qq
Z_\ga\s=\s\int\s\ee^{-S_\Sigma
(\phi)}\s\s D\phi
\label{FPF}
\qqq
where \s$S_\Sigma(\phi)\s$ is the action functional
and \s$D\phi=\prod_{x\in\Sigma}d\nu(\phi(x))\s$
is the formal volume element on the space of fields.
On a Riemann surface with boundary \s$\Sigma\s$,
we could consider an
analogous functional integral
with fixed boundary values
\s$\phi|_{\CC_i}\equiv\varphi_i\s$ of the fields:
\qq
A_{_{\Sigma,\ga}}((\varphi_i)_{i\in I})
\ \s=\s\int\limits_{\phi|_{\CC_i}=\varphi_i}
\hs{-0.3cm}\ee^{-S_\Sigma(\phi)}\s\s D\phi\ .
\label{Bint}
\qqq
It is a function of the field boundary
values. The space of functionals \s$\CF(\varphi)\s$
of the fields on the circle
with the formal \s$L^2\s$ scalar product may be
thought of as the Hilbert space \s$\CH\s$ of states of the system
(we have seen a concrete realization of this idea
for the free compactified field). We may then
interpret the functional \s$A_{_{\Sigma,\ga}}((\varphi_i)_{i
\in I})\s$ as the kernel of an operator
mapping the tensor product of state spaces,
one for each ``in'' component of the boundary,
to the similar product for the ``out'' components.
\vs 0.3cm

The space \s$\CH\s$ may be equipped with a formally
anti-unitary involution
\s$\CI\s$, \s\ $(\CI\CF)(\varphi)$ $\equiv
{\overline{\CF(\varphi^\vee)}}\s$
where \s$\varphi^\vee(z)
\equiv\varphi(z^{-1})\s$ which allows to turn the operators
in \s$\CH\s$ into bilinear
forms and vice versa and to identify the operator
amplitudes when we invert the orientation of some
boundary components.
\vs 0.3cm

The basic formal property of the functional
integral (\ref{Bint}) is that one should be able
to compute it iteratively. Suppose that the surface \s$\Sigma'\s$
is obtained by identifying a
(parametrized) ``out'' component \s$\CC_{i_0}\s$
with an ``in'' component
\s$\CC_{i_1}\s$ of a (connected or
disconnected) surface \s$\Sigma\s$.
\s The functional integral on \s$\Sigma'\s$
may be done by keeping first the value of the field on the
gluing circle fixed and integrating over it only
subsequently. In other words, the equality
\qq
\int\limits_{^{\phi|_{\CC_i}=\varphi_i,}_{i\not=i_0,i_1}}
\ee^{-S_{\Sigma'}(\phi)}
\s\s D\phi
\ =\ \int\bigg(\int\limits_{^
{\phi|_{\CC_{i}}=\varphi_i,\
i\not=i_0,i_1}_{\phi|_{\CC_{i_0}}
=\varphi_0=\phi|_{\CC_{i_1}}}}
\hs{-0.4cm}\ee^{-S_\Sigma(\phi)}
\s\s D\phi\bigg)
\s D\varphi_0
\non
\qqq
should hold and this is exactly a formal version
of the gluing property ({\bf vi}).
\vs 0.3cm

What is the functional integral interpretation of
eq.\s\s(\ref{MatEle})?
One should interpret vectors \s$\CY\in\CH\s$ as
corresponding to functions
\qq
\CF_Y(\varphi)\s=\s(Z_{\vartheta*\ga\vee \ga})^{-1/2}\s
\int\limits_{\phi|_{\da D}=\varphi}
Y\s\s\ee^{-S_D(\phi)}\s\s D\phi
\non
\qqq
of fields on the circle. The complex
conjugate functions are then
given by the \s$D'$-functional integrals
\qq
(Z_{\vartheta^*\ga\vee\ga})^{-1/2}\s
\int\limits_{\phi|_{\da D}=\varphi}
\s \Theta Y\m\s\s\ee^{-S_{D'}(\phi)}\s
\s D\phi\
\non
\qqq
so that the formal \s$L^2\s$ scalar product
for the functions of fields \s$\varphi\s$ on the
boundary circle gives
\qq
&{\displaystyle{\int}}
\m{\overline{\CF_Y(\varphi)}}\s\s\m\CF_Y(\varphi)\s\s
D\varphi\s=\s(Z_{\vartheta^*\ga\vee\ga})^{-1}\s
{\displaystyle{\int}}
\m(\Theta Y)\s\s Y\s\s\ee^{-S_{D\cup D'}(\phi)}\s
\s D\phi\ &\cr
&=\s\langle\s(\Theta Y)\s Y\s\rangle\s=\s
(\m\CY\m,\m\CY\m)\ &
\non
\qqq
(\m$\ga\s$ has been assumed locally flat above).
Formula (\ref{MatEle}) follows now by iterative
calculation of the functional integral
\qq
Z_{\tilde \ga}\s\s\langle\s\prod\limits_{i\in I_{\rm out}}
(\Theta Y_i)\s\s
\prod\limits_{i\in I_{\rm in}}Y_i\s\rangle\ =\
\int\prod\limits_{i\in I_{\rm out}}(\Theta Y_i)\s
\prod\limits_{i\in I_{\rm in}}Y_i\s\ \ee^{-S_{\hat\Sigma}
(\phi)}\s\s D\phi
\non
\qqq
by first fixing the values of \s$\phi\s$ on \s$\da\Sigma\s$
and then integrating over them.
\vs 0.3cm

Any two-dimensional QFT should give rise to operator amplitudes
with properties ({\bf i}) to ({\bf vi}).
In particular, the so called $P(\phi)_2$ theories
corresponding to the actions \s$S(\phi)\sim\|\phi\|^2_{L^2}+
\int P(\varphi)\s dv\s$, \s where \s$P\s$ is a bounded below
polynomial, give undoubtly rise to such structures.
The special property, which distinguishes the CFT models from
other two-dimensional QFT's is, of course, the
local scale covariance $({\bf vii})$ (conjecturally,
special limits of the \s$P(\phi)_2\s$ theories should
possess this property).
\vskip 0.4cm


As already stressed, if we consider eq.\s\s(\ref{MatEle}) as
the definition of the amplitudes \s$A_{\Sigma,\ga}\s$ then the properties
$({\bf i})$ to $({\bf vii})$ above become further requirements
on the correlation functions. The point of Segal was, however,
that the amplitudes \s$A_{{\Sigma,\ga}}\s$ satisfying
$({\bf i})$ to $({\bf vii})$ may be taken as the defining objects
of CFT from which the rest of the structure, like the Virasoro
algebra action, the primary fields and their correlation functions, etc.
may be extracted. In this sense, the requirements $({\bf i})$
to $({\bf vii})$ may be viewed as the basic axioms of CFT,
encapsulating its mathematical structure.
Let us briefly sketch Segal's arguments.
They require looking at the amplitudes \s$A_{{\Sigma,\ga}}\s$
for the simplest geometries.
\vskip 0.6cm

\no{\bf Discs}
\vs 0.4cm

For \s$\Sigma=D\s$, \s$A_{{D,\ga}}\in\CH\s$ and
\s\s$(Z_{\vartheta^*\ga\vee\ga})^{-1/2}\s A_{{D,\ga}}\s\s$
is a metric independent vector which, according to
eq.\s\s(\ref{MatEle}) we should interpret as
the vacuum vector:
\qq
(Z_{\vartheta^*\ga\vee\ga})^{-1/2}\s A_{_{D,\ga}}\s
\equiv\s\Omega\ .
\label{vac}
\qqq
Similarly, \s\s$A_{_{D',\vartheta^*\ga}}
\s\s$ is the linear functional on \s$\CH\s$,
\qq
(Z_{\vartheta^*\ga\vee\ga})^{-1/2}\s A_{_{D',\vartheta^*\ga}}
\s=\s(\s\Omega\s,\s\ \cdot\ \s)\ .
\non
\qqq
It follows from the properties $({\bf iv})$ and $({\bf v})$
that \s$\Omega=\CI\s\Omega\s$.
\vs 0.6cm

\no{\bf Annuli}
\vs 0.5cm

Let us pass to surfaces \s$\Sigma\s$
with annular topology. Under gluing of an ``out'' and an
``in'' boundary components of two such surfaces
their amplitudes \s$A_{{\Sigma,\ga}}\s$ form a semigroup
due to the property ({\bf vi}).
\no This is the semigroup which encodes
the Virasoro action on \s$\CH\s$. \s In particular,
for the annuli \s$\Sigma=\{\s
|q|\leq |z|\leq 1\s\}=
{C_q}\s$ with the ``out'' boundary
parametrized by \s$z\mapsto z\s$ and the ``in'' one
by \s$z\mapsto qz\s$, using the metric \s$\ga_0=|z|^{-2}|dz|^2\s$
on \s${C_q}\s$, \s we obtain a semigroup
almost identical to the semigroup \s$(\m\CS_q\m)\s$
considered before
\qq
(\s\CY'\m,\s A_{_{{C_q},\ga_0}}\s\CY\s)\s=\s
Z_{\ga_0}\s\s\langle\s(\Theta Y)\s\s S_qY\s\rangle
\s=\s Z_{\ga_0}\s\s(\s\CY'\m,\s\CS_q\s\CY\s)
\label{62}
\qqq
where \s$Z_{\ga_0}\s$ is given by eq.\s\s(\ref{PFda}).
\vs 0.5cm

\no{\bf Problem 5}.\ \ Show that \s$Z_{\ga_0}\s=\s(q\bar q)^{-c/24}\s$.
\label{63}
\vs 0.5cm


\no The immediate consequence of this relation and of
eqs.\s\s(\ref{62}) and (\ref{AbSemi}) is the identity
\qq
A_{_{{C_q},\ga_0}}\s=\s\s q^{L_0-c/24}\m\s\bar q^{\tilde L_0-c/24}\ .
\label{c/24}
\qqq
Note that the mapping \s$z\s\mapsto\s\ee^{\m i z}\s$
establishes a holomorphic diffeomorphism between
the finite cylinder \s$C_\tau\s=\s\{\s\m z\ |\ 0\leq{\rm Im}\s z
\leq 2\pi\m\tau_2\s\m\}/2\pi\NZ\s$ and \s${C_q}\s$ where
\s$q=\ee^{\m 2\pi i\tau}\s$. The pullback of the metric \s$\ga_0\s$
by \s$z\s\mapsto\s\ee^{\m i z}\s$
is \s$|dz|^2\s$. \s The amplitude
\s\s$A_{_{C_\tau\m,\m |dz|^2}}\s\s$
is equal to
\s\s$\ee^{-2\pi\tau_2\m H}\s\s\ee^{2\pi i\tau_1\m P}\s$,
\s\s where \s$\tau=\tau_1+i\tau_2\s$ and \s$H\s$ is
the Hamiltonian of the theory
quantized in periodic volume \s$\NR/2\pi\NZ\s$
and \s$P\s$ is the momentum operator. Comparison
with eq.\s\s(\ref{c/24}) yields
\qq
H\s=\s L_0+\tilde L_0-{_c\over^{12}}\ ,\hs{0.7cm}
P\s=\s L_0-\tilde L_0\ .
\non
\qqq
The requirement that the amplitudes be trace class
operators implies that \s$L_0\s$ and \s$\tilde L_0\s$ have
discrete spectrum with finite multiplicity,
so that their eigenvalue zero corresponding
to eigenvector \s$\Omega\s$ is isolated (with possible
multiplicity).
Note, that energy of the vacuum state becomes equal
to \s${c\over 12}\s$ now. If we work on the
space \s$\NR/L\NZ\s$ instead,
the energy spectrum is multiplied by \s${2\pi\over L}\s$
so that the lowest energy becomes \s${\pi c\over 6L}\s$.
This permits to calculate (following Cardy) the central charge
of the conformal models from the finite size dependence
of the ground state energy in the periodic interval.
\vs 0.3cm

Gluing together the boundaries of \s$C_\tau\s$,
one obtains the elliptic curve
\s$T_\tau\s=\s\NC/(\NZ+\tau\NZ)\s$ (with metric
$4\pi^2 |dz|^2$).
\s The properties $({\bf iii})$ and $({\bf vi})$ of the
operator amplitudes imply
then the following expression for the {\bf toroidal
partition function}
\qq
Z(\tau)\s\m\equiv\s A_{_{T_\tau,\m |dz|^2}}\s=
\s\m\tr\s\s\m q^{\m L_0-c/24}\s\s\bar q^{\m \tilde L_0-c/24}\ .
\label{TPFu}
\qqq
which is necessarily a modular invariant function (recall why?).
The modular invariance of the partition functions of CFT has
played an important role in the search and the classification
of models.
\vs 0.6cm



The amplitudes for other annular surfaces allow
to obtain the action of other generators of
the double Virasoro algebra in the Hilbert
space of states \s$\CH\s$. If \s$D\ni z\mapsto f(z)\in D\s$
is a holomorphic embedding of \s$D\s$ into its interior
preserving the origin then \s$\Sigma_f\s:=\s D\setminus
f({\rm int}(D))\s$ is in a natural way an annulus with parametrized
boundary (one ``in'' \s$f(\da D)\s$ and one ``out'' \s$\da D\s$
component). Such annuli form a semigroup. In particular,
for \s$f_{q,\alpha,n}(z)\s=\s\mapsto\s q^{z\da_z}\s
\ee^{\alpha\m z^{n+1}\da_z}\m z\s$
for \s$n>0\s$, \s$|q|<1\s$ and \s$|\alpha|\s$ sufficiently small,
\qq
A_{_{\Sigma_{f_{q,\alpha,n}}\m,\m\ga}}\s=
\s\s Z_\ga\s\s\s\ee^{\m\alpha L_n}\s q^{L_0}\s\m
\ee^{\m\bar\alpha \tilde L_n}\s{\bar q}^{\tilde L_0}\
\non
\qqq
encoding the action of the operators \s$L_n\s$, \s$\tilde L_n\s$
for \s$n>0\s$ (the complex-conjugate
annuli \s$\bar\Sigma_{f_{q,n,\alpha}}\s$ give amplitudes
encoding \s$L_n\s$, \s$\tilde L_n\s$ for \s$n<0\s$)\m.
\s The Virasoro HW vectors \s$\CX_l\in\CH\s$
of conformal weights \s$(\Delta_l,\tilde\Delta_l)\s$ may be
characterized by the equalities
\qq
Z_\ga^{-1}\s A_{_{\Sigma_f\m,\m \ga}}\s\CX_l\s=\s
({_{df(0)}\over^{dz}})^{\Delta_l}\s({_{d\bar f(0)}\over^
{d\bar z}})^{\tilde\Delta_l}\s\CX_l\s.
\label{HWV}
\qqq
In particular, \s$L_0\CX_l=\Delta_l\CX_l\s$, \s$L_n\CX_l=0\s$
for \s$n>0\s$ and similarly for \s$\tilde L_n$'s.
\vs 0.3cm

Each Virasoro HW vector gives rise to a primary field and
the correlation functions of the primary fields may be recovered
from the operator amplitudes by the following construction.
Let \s$(x_1,\s\dots\s,\m x_n)\s$ be a sequence of non-coincident
points in the surface \s$\hat\Sigma\s$ without boundary.
Specify local parameters at points \s$x_j\s$ by
embedding discs \s$D\s$ into \s$\hat\Sigma\s$ so that
their images centered at points \s$x_j\s$ do not intersect.
\s$\hat\Sigma\s$ may be then viewed as a surface \s$\Sigma\s$
with boundary capped with discs \s$D\s$.
The correlation functions of the primary fields \s$\varphi_{l_i}\s$
corresponding to the HW vectors \s$\CX_{l_i}\s$ may be
defined in a locally flat metric by the formula
\qq
\langle\s\phi_{l_1}(x_1)\s\cdots\s\phi_{l_n}(x_n)
\s\rangle\ =\ {_1\over^{Z_\ga}}\s\s A_{_{\Sigma,\ga}}\s\otimes_{i}
\hs{-0.07cm}\CX_{l_i}
\label{xxx}
\qqq
(in accordance with relation (\ref{MatEle})). The characteristic
property (\ref{HWV}) of the HW vectors assures that
the left hand side is unambiguously defined
as a \s$\Delta_{l_j},\tilde\Delta_{l_j}$-form
in variable \s$x_j\s$:
\vs 0.6cm

\no{\bf Problem 6}.\ \ Using relation (\ref{HWV}) and the
gluing axiom show that the right hand side of eq.\s\s(\ref{xxx})
picks only a factor \s$({_{df(0)}\over^{dz}})^{\Delta_{l_j}}
\s({_{d\bar f(0)}\over^{d\bar z}})^{\tilde\Delta_{l_j}}\s$
under the change \s$z\mapsto f(z)\s$ of the \s$j^{\rm \s th}\s$
local parameter of surface \s$\hat\Sigma\s$.
\vs 0.6cm


\no As we see, the relation between the primary fields
and the HW vectors \s$\phi_l\s\mapsto\s\varphi_l(0)\s\Omega\s$,
\s discovered before is one to one (if we include
among the primary fields the trivial ``identity'' field
whose insertions have no effect in correlation functions
and which corresponds to the HW vector \s$\Omega\s$).
\vs 0.6cm

\no{\bf Pants}
\vs 0.4cm

More generally, it is possible to associate to each vector \s$\CX\s$
in the Hilbert space of states \s$\CH\s$, \s in the domain
of \s${q_1}^{-L_0}\s{\bar{q_1}}^{\m-\tilde L_0}\s$ for some \s${q_1}
\s$ with \s$|q_1|<1\s$, \s an operator-valued
field \s$\varphi(\CX\m;\m w,\bar w)\s$ defined for \s$0<|w|<1\s$
in the following way. For
\s$0<|q|<|w|-|q_1|<1-2\m|q_1|\s$, consider the Riemann surface
\qq
P_{q,q_1,w}\s=\s\{\m\s|q|\leq|z|\leq 1,\ |z-w|\geq|q_1|\s\m\}
\non
\qqq
with the ``out'' boundary component \s$\{\s\m|z|=1\s\m\}\s$
parametrized by \s$z\mapsto z\s$ and the ``in'' ones by \s$
z\mapsto qz\s$ and by \s$z\mapsto w+q_1z\s$. \s Such a surface is
often called ``pants'' for obvious reasons.
The operator \s$\varphi\m(\CX\m;\m w,\bar w)\s$ will
be essentially defined as the amplitude of the pants.
More exactly, we shall put
\qq
\varphi\m(\CX\m;\m w,\bar w)\s\m\CY\
=\ {_1\over^{Z_\ga}}\s\s\m A_{_{P_{q,q_1,w}\m,\m \ga}}
\s\s(\s q^{-L_0}\s{\bar q}^{\m-\tilde L_0}\s\CY\s\otimes\s
{q_1}^{-L_0}\s{\bar{q_1}}^{\m-\tilde L_0}\s\CX\s)\ .
\non
\qqq
Note that \s$\varphi\m(\CX\m;\m w,\bar w)\m\s\CY\s$
is independent of \s$q_1\s$ and of \s$q\s$ as long as
\s$\CX\s$ is in the domain of
\s${q_1}^{-L_0}\s{\bar{q_1}}^{\m-\tilde L_0}\s$
and \s$\CY\s$ in the domain
of \s$q^{-L_0}\s{\bar q}^{\m-\tilde L_0}\s$.
\vs 0.6cm

\no{\bf Problem 7}.\ \ (a).\ \s Show that \s$\varphi\m(\CX\m;\m 0\m)
\s\s\Omega\s
\equiv\s\lim\limits_{w\to 0}\ \varphi\m(\CX\m;\m w,
\bar w)\s\s\Omega\s=\s\CX\s$.\hfill\\
(b).\ \s Show that for a HW vector \s$\CX_l\s$\m,
\s\ $\varphi\m(\CX_l;\m w,\bar w)\s\s$
coincides with the operator \s$\varphi_l(w,\bar w)\s$
assigned to the primary field \s$\phi_l\s$
corresponding to \s$\CX_l\s$.\hfill\\
(c).\ \s Show that\s$\ \ \
\varphi\m(L_{-2}\m\Omega\m;\m w,\bar w)
\s=\s\CT(w)\s\ $ \ and that\ \ \ \s$\varphi
\m(\tilde L_{-2}\m\Omega\m;\m w,\bar w)\s=\s
{\bar\CT}(\bar w)\s$.
\vs 0.6cm

\no The operators \s$\varphi\m(\CX\m;\m w,\bar w)\s$ satisfy
an important identity
\qq
\varphi\m(\CY\m;\m z,\bar z)\s\s\m\varphi\m(\CX\m;\m w,\bar w)
\ =\ \varphi\m(\m\varphi(\CY\m;\m z-w\m,\m\bar z
-\bar w)\s\m\CX\s;\s\m w,\bar w\m)
\label{OPEgl}
\qqq
holding for \s$0<|w|<|z|\s$, \s$0<|z-w|<1\s$.
\s Eq.\s\s(\ref{OPEgl}) follows from the two ways that one
may obtain the disc with three holes
by gluing two discs with two holes.
\vskip 0.6cm

\no{\bf Problem 8}.\ \ Prove eq.\s\s(\ref{OPEgl})
using the gluing property $({\bf vi})$
of the operator amplitudes and treating with care
the normalizing factors \s$Z_\ga^{\s-1}\s$.
\vs 0.6cm

\no The relation (\ref{OPEgl}) may be viewed as a
global form of the OPE's, the local forms following
from it by expanding the vector \s$\varphi\m(\CY\m;\m z-w\m,
\m\bar z-\bar w\m)\s\m\CX\s$ into terms homogeneous in \s$z-w\s$
and \s$\bar z-\bar w\s$.
\vs 0.53cm

Since any Riemann surface can be built from
discs, annuli and pants the general amplitudes
\s$A_{{\Sigma,\ga}}\s$ may be expressed
by the Virasoro generators and operators
\s$\varphi(\CX\m;\m w,\bar w)\s$.
This permits to formulate the basic mathematical
structure of CFT in an even more economic
(and more algebraic) way than through
the amplitudes \s$A_{{\Sigma,h}}\s$ with the properties
$({\bf i})$ to $({\bf vii})$, getting rid of
the Riemann surface burden.
Instead, one obtains a set of axioms for the action
of the \s${Vir}\times{Vir}\s$ algebra
and of the {\bf vertex operators}
\s$\varphi\m(\CX\m;\m w,\bar w)\s$ in the
Hilbert space \s$\CH\s$, \s with the relation
(\ref{OPEgl}), the main consistency condition,
playing a prominent role.
That was, essentially, the idea which, in the
holomorphic version of CFT, has led to the concept
of a {\bf vertex operator algebra} developed
by Frenkel-Lepowsky-Meurman and by Borcherds.
The latter, by studying the algebras arising
in the context of toroidal compactifications
(on Minkowki targets) was led to the concept
of generalized Kac-Moody or (Borcherds) algebras
which promise to play an important role in physics
and mathematics.
\vskip 1cm

\no{\bf References}
\vskip 0.5cm

The operator product expansion in the two-dimensional CFT
is a starting point of the seminal Belavin-Polyakov-Zamolodchikov
paper in Nucl. Phys. B 241 (1984), p. 333, see also
Eguchi-Ooguri: Nucl. Phys. B 282 (1987), p. 308.
\vskip 0.4cm

The treatment of the operator formalism based of the notion
of physical positivity was adapted from Osterwalder-Schrader:
Commun. Math. Phys. 31 (1973), p. 83 \s and 42 (1975), p. 281.
\vskip 0.4cm

For the theorem about the unitary representations of the Virasoro
algebras see Friedan-Qiu-Shenker, Phys. Rev. Lett. 52 (1984),
p. 1575 and Commun. Math. Phys. 107 (1986), p. 535 and also
Goddard-Kent-Olive, Commun. Math. Phys. 103 (1986), p. 105.
\vskip 0.4cm

For the Segal axiomatics of CFT see
G. Segal's contribution to "IXth International Congress
in Mathematical Physics", Simon-Truman-Davies (eds.),
Adam Hilger 1989 and G. Segal: "The definition of conformal field
theory," preprint of variable geometry. A general discussion of CFT
designed for the mathematical audience may be found in \s Gaw\c{e}dzki:
Asterisque 177/178 (1989), p. 95.
\vskip 0.4cm

Vertex operator algebras are the subject of Frenkel-Lepowsky-Meurman's
book ``Vertex Operator Algebras and the Monster'', Academic Press 1988,
see also a review by Gebert in Int. J. Mod. Phys. A8 (1993), p. 5441.

\eject


\
\vskip 0.8cm

\noindent{\large{\bf Lecture 3.\ \ Sigma models}}
\addtocounter{equation}{-78}
\vskip 0.8cm

\no\un{Contents\s}:
\vskip 0.5cm

\no 1. \ 1PI effective action and large deviations

\no 2. \ Geometric sigma models

\no 3. \ Regularization and renormalization

\no 4. \ Renormalization group effective actions

\no 5. \ Background field effective action

\no 6. \ Dimensional regularization

\no 7. \ Renormalization of sigma models to 1 loop

\no 8. \ Renormalization group analysis of sigma models
\vskip 1.7cm


\noindent{\bf 1.\ \s 1PI effective action and large deviations}
\vskip 0.5cm

\no It will be useful to describe another relation between
a field theoretic and a probabilistic concept.
Consider a positive measure
\qq
d\mu(\xi)\s=\s\ee^{-S(\xi)}\s\s D\xi\s.
\non
\qqq
on a (finite dimensional) euclidean space \s$\NE\s$.
We shall assume that \s$S(\xi)\s$ grows faster than linearly
at infinity and, for convenience, that the measure is normalized.
The characteristic functional of \s$d\mu\s$
\qq
&\ee^{\m W(J)}\s=\s\int\ee^{\m\langle\m\xi\m,\m J\m\rangle}
\s\s d\mu(\xi)&
\non
\qqq
is then an analytic function of \s$J\s$ in the complexified
dual of \s$\NE\s$. \s Let \s$N\zeta\s=\s\sum_1^N\xi_j\s$ be
the sum of $N$ independent random variables equally distributed
with measure \s$d\mu\s$. \s The probability distribution of
\s$N\zeta\s$
is
\qq
&P_N(\zeta)\ =\ \int\delta(N\zeta-\sum\limits_{j=1}^N\xi_j)
\s\prod\limits_{j=1}^Nd\mu(\xi_j)&\cr
&=\ \int DJ\int\ee^{-\langle\m N\zeta-\sum\xi_j\m,\m J\m\rangle}
\s\prod\limits_{j=1}^N d\mu(\xi_j)\
=\ \int\ee^{-N\m\langle\m\zeta\m,\m J\m\rangle\s+\s N\m W(J)}
\s\s DJ\s.&
\label{FoT}
\qqq
where the \s$J$-integration is over an imaginary section
of \s$\NE^*_\NC\s$ with \s$DJ\s$ denoting
the properly normalized Lebesque measure on it\footnote{another
reading of eq.\s\s(\ref{FoT}) says that the \s$N$-fold
convolution becomes the \s$N^{\rm\s th}$ \m power
in the Fourier language}. We shall be
interested in the behavior of \s$P_N\s$ for large \s$N\s$. \s
It is not difficult to see that
\qq
P_N(\zeta)\s=\s\ee^{\s N{\mathop{\rm inf}\limits_{J\in\NE^*}}\s
\m[\s-\langle\m\zeta\m,\m J\m\rangle\s+\s W(J)\s]\s\ +\ \s o(N)}\
\equiv\ \ee^{\m-\m N\m\Gamma(\zeta)\ \s+\ \s o(N)}\s.
\label{PN}
\qqq
In probability theory,
\qq
\Gamma(\zeta)\ =\ {\mathop{\rm sup}\limits_{J\in\NE^*}}\s
\m[\s\langle\m\zeta\m,\m J\m\rangle\s-\s W(J)\s]
\ =\ [\s\langle\m\zeta\m,\m J\m\rangle\s-\s W(J)\s]
\bigg|_{{\zeta=W'(J)}}\s
\label{ga}
\qqq
($W'\s$ denotes the derivative of \s$W\s$)
is called the ``large deviation (rate) function''.
It is the Legendre transform of \s$W(J)\s$ which is a strictly
convex function on \s$\NE^*\s$. \s It controls the regime where
\s$\sum\xi_j\s=\s\CO(N)\s$ as opposed to the central limit theorem
which probes \s$\sum(\xi_j-\langle\xi\rangle)=\CO(N^{1/2})\s$
where \s$\langle\xi\rangle\s$ is the mean value of \s$\xi_j\s$.
\s Since \s$\Gamma'(\zeta)=J\s$, the minimum of \s$\Gamma\s$
occurs at \s$\zeta=W'(0)=\langle\xi\rangle\s$. \s The central
limit theorem sees only the second derivative of \s$\Gamma\s$
at \s$\langle\xi\rangle\s$:
\qq
\lim\limits_{N\to\infty}\ P_N(\langle\xi\rangle+N^{-1/2}\psi)\
\sim\ \ee^{-{1\over^2}\s\Gamma''(\langle\xi\rangle)(\psi^2)}\s.
\non
\qqq
$W\s$ may be recovered from \s$\Gamma\s$ by the
inverse Legendre transform:
\qq
W(J)\ =\ {\mathop{\rm sup}\limits_{\zeta\in \NE}}\s
\m[\s\langle\m\zeta\m,\m J\m\rangle\s-\s \Gamma(\zeta)\s]
\ =\ [\s\langle\m\zeta\m,\m J\m\rangle\s-\s \Gamma(\zeta)\s]
\bigg|_{{J=\Gamma'(\zeta)}}\s.
\label{ilt}
\qqq
We may compute \s$W(J)\s$ and \s$\Gamma(\zeta)\s$ as formal
power series introducing a coefficient \s$\hh\s$
(the Planck constant), Taylor expanding \s$S(\xi)\s$
and separating the quadratic contribution to it:
\qq
\ee^{\m W(J)}\ =\ \int\ee^{\s{1\over{h\hspace{-0.23cm}{}^-}}
\m[\m\langle\m\xi\m,\m J\m\rangle
\s-\s S(\xi)\m]}\s\s D\xi\s\bigg|_{{\hh}=1}\hspace{8.5cm}\cr\cr\cr
=\ \bigg[\s\ee^{-{1\over\hh}\m S(0)}\s\s
\int\ee^{\s{1\over\hh}\m\langle\m\xi\m,\s
J-S'(0)\m\rangle\s-\s
{1\over 3!\hh}
\m S'''(0)(\xi^3)\s-\s\dots\dots}\ d\mu_{_{\hh\m S''(0)^{-1}}}(\xi)
\ \int\ee^{-{1\over 2\hh}\s S''(0)(\xi^2)}\s\s D\xi
\s\bigg]\bigg|_{{\hh}=1}
\label{form0}
\qqq
where the Gaussian measure
\qq
d\mu_{_{\hh\m S''(0)^{-1}}}(\xi)\
=\ {\ee^{-{1\over2\hh}\m S''(0)(\xi^2)}
\s\s D\xi\over\int\ee^{-{1\over 2\hh}\s S''(0)(\xi^2)}\s\s D\xi}\s.
\qqq
Expanding the exponential under the \s$d\mu\s$ integral
into the power series and performing the Gaussian integration,
we obtain an expansion in powers of \s$\hh\s$ which,
as discussed in Kazhdan's lectures, gives upon
exponentiation the relation
\qq
\ee^{\m W(J)}\ =\
\bigg[\s\ee^{-{1\over{h\hspace{-0.23cm}
{}^-}}\m S(0)\s\s+\sum\limits_{{\rm vacuum}\atop
{\rm graphs\ }G}\hs{-0.1cm}{{h\hspace{-0.23cm}{}^-}
}^{\m\#\{{\rm loops\s\m of\s\m}G\}\m-\m 1}\s\s
I_G(J,S)}\ \int\ee^{-{1\over{2\hh}}
\m S''(0)(\xi^2)}\s\s
D\xi\s\bigg]\bigg|_{{\hh}=1}\hspace{1.2cm}
\label{form}
\qqq
where, by definition, graphs \s$G\s$ are
connected graphs\footnote{recall that the exponential
of the sum over connected graphs is the sum over connected
and disconnected graphs}
made of 1-leg vertices \s$J\s$ or \s$-S'(0)\s$,
3-leg vertices \s$-S'''(0)\s$, 4-leg vertices
\s$-S^{(4)}(0)\s$ etc., with propagators \s$S''(0)^{-1}\s$
on the internal lines\footnote{we count lines ending at the
1-leg vertices as internal} and no propagators on the external lines.
The vacuum graphs are the ones without external lines.
The amplitudes
\s$I_G(J,S)\s$ are associated to the graphs in a natural way,
with the symmetry factors of the graphs included. If \s$S\s$
is a polynomial and $S'(0)=0$, then there is only a finite number of
graphs with a given number of $J$-vertices and a given number of loops
and we may view \s$W(J)\s$ as a formal series in \s$J\s$ and
in the number of loops. More exactly, comparing the left and the right
hand sides of eq.\s\s(\ref{form}), we infer that
\qq
&W(J)\ =\ -\m S(0)\ +\sum\limits_{{\rm vacuum}\atop
{\rm graphs}\s\m G}I_G(J,S)\s\m+
\m\s\ln{(\int\ee^{-{1\over 2}\s S''(0)(\xi^2)}
\s\m D\xi)}\s.&
\non
\qqq
As discussed by Kazdan, by cutting all the lines of the graphs
whose removal makes
the graph disconnected, we obtain the second representation
for \s$W(J)\s$:
\qq
W(J)\ =\ \s-\m \Gamma(0)\ +\ \sum
\limits_{{\rm vacuum}\atop{\rm trees}\s\m T}I_T(J,\Gamma)
\non
\qqq
where the ``1PI effective action''
\s$\Gamma(\zeta)\s$ is defined by its formal
Taylor series
\qq
&&\hbox to 11cm{$\Gamma(0)\s\ =\ S(0)\s\m-
\hs{-0.2cm}\sum\limits_{1{\rm PI\s\m vacuum}\atop
{\rm graphs}\s\m G}\hs{-0.2cm}I_G(S)\s\m-
\m\s\ln{(\int\ee^{-{1\over 2}\s S''(0)(\xi^2)}
\s\m D\xi)},$\hfill}
\Gamma'(0)\ =\ -\hs{-0.5cm}\sum\limits_{1{\rm PI\s\m graphs}\s\m G\atop
{\rm with\s\m 1\s\m external\s\m line}}
\hs{-0.2cm}I_G(S)\s,\hs{0.3cm}\cr
&&\hbox to 9cm{${_1\over^2}\s\Gamma''(0)\s\s=\ {_1\over^2}\s S''(0)\s
-\hs{-0.5cm}\sum\limits_{1{\rm PI\s\m graphs}\s\m G\atop
{\rm with\s\m 2\s\m external\s\m lines}}\hs{-0.3cm}I_G(S)\s,$\hfill}
{_1\over^{3!}}\s\Gamma'''(0)\s\m=\ -\hs{-0.5cm}
\sum\limits_{1{\rm PI\s\m graphs}\s\m G\atop
{\rm with\s\m 3\s\m external\s\m lines}}\hs{-0.3cm}
I_G(S)\s,\hs{0.3cm}\label{Pe}\\
&&\dots\dots\dots\dots\dots\dots\dots\dots\dots\dots\dots\dots\dots\dots
\hs{0.3cm}\cr
\non
\qqq
where ``1PI'' stands for (amputated) 1-particle irreducible
graphs without $J$-vertices. \s
Rewriting eq.\s\s(\ref{form}) with \s$S\s$
replaced by \s$\Gamma\s$, i.e. as an expansion for
\s$\int\ee^{\s{1\over\hh}
\m[\m\langle\m\zeta\m,\m J\m\rangle
\s-\s \Ga(\zeta)\m]}\s\s D\zeta\s$, \s but keeping
only the leading terms at \s${\hh}\s$ small, we obtain
finally the equality
\qq
{\mathop{\rm sup}\limits_{\zeta\in \NE}}\s
\m[\s\langle\m\zeta\m,\m J\m\rangle\s-\s \Gamma(\zeta)\s]
\ =\ \s-\m \Gamma(0)\ +\ \sum
\limits_{{\rm vacuum}\atop{\rm trees}\s\m T}I_T(J,\Gamma)
\ =\ W(J)\s.
\non
\qqq
Comparing this to eq.\s\s(\ref{ilt}) we see that eqs.\s\s(\ref{Pe})
provide a perturbative interpretation of the large deviation
function \s$\Gamma\s$.
\vskip 1cm


\noindent{\bf 2.\ \s Geometric sigma models}
\vskip 0.4cm

\no We have already discussed a simple way
to write down a conformal invariant action
for maps \s$\phi:\Sigma\rightarrow M\s$ where
\s$(\Sigma,\ga)\s$ is a Riemann surface and \s$(M,g)\s$
a Riemannian manifold. The functional
\qq
S_g(\phi)\ =\ {_1\over^{4\pi}}\m\|d\phi\|_{L^2}^{\s 2}\ =\
{_i\over^{2\pi}}\int_{_\Sigma}g_{ij}(\phi)
\s\s\da\phi^i\wedge\de\phi^j
\label{sm}
\qqq
(summation convention!) depends only on the conformal
class of \s$\ga\s$.
We could add to \s$S_g(\phi)\s$ also a ``topological'' term
\qq
S_{top}(\phi)\ =\ {_i\over^{4\pi}}\s\int_{_\Sigma}\phi^*\omega
\ =\ {_i\over^{2\pi}}\int_{_\Sigma}b_{ij}(\phi)
\s\s\da\phi^i\wedge\de\phi^j
\label{stop}
\qqq
involving a 2-form \s$\omega=b_{ij}(\phi)\s d\phi^i
\wedge d\phi^j\s$ on \s$M\s$ which does not
depend on \s$\ga\s$ but only on the orientation of \s$\Sigma\s$.
Renormalizability forces addition of two more terms
to the action which break classical conformal invariance:
\qq
S_{tach}(\phi)\s=\s{_1\over^{4\pi}}
\int_{_\Sigma}u\circ\phi\ dv\quad\quad
{\rm and}\quad\quad
S_{dil}\s=\s{_1\over^{4\pi}}\int_{_\Sigma}w\circ\phi\ r\s\m dv
\label{tach}
\qqq
where \s$u,w\s$ are functions on \s$M\s$ called, respectively, the
tachyon and the dilaton potentials\footnote{the names
come from the string theory context}, \s$dv\s$ stands
for the volume measure and \s$r\s$ for the scalar
curvature of \s$\Sigma\s$.
\vskip 0.4cm

One may also consider a supersymmetric version of the model
(see Problem set 3) with the action
\qq
S_g^{\rm SUSY}(\Phi)\ =\ {_1\over^{2\pi i}}\int g_{ij}(\Phi)
\s\s D\Phi\s\s\tilde D\Phi\ dz\wedge d\bar z\wedge
d\theta\wedge d\tilde\theta
\label{Sus}
\qqq
where the superfield
\qq
\Phi\s=\s\phi\s+\s\theta\psi\s+\s\tilde\theta\tilde\psi\s+\s
\theta\tilde\theta\s F\s.
\non
\qqq
and \s$D=\da_\theta+\theta\da_z\s$, \s$\tilde D=\da_{\tilde\theta}
+\tilde\theta\da_{\bar z}\s$\s. \s In components, after
elimination of the auxiliary field \s$F\s$ through its
equation of motion, one obtains
\qq
S_g^{\rm SUSY}(\phi,\psi,\tilde\psi)\ =\
{_i\over^{2\pi}}\int \bigg(g_{ij}(\phi)\s\s\da_z\phi^i\s\s\da_{\bar z}
\phi^j\hs{-0.13cm}&-&\hs{-0.13cm}
g_{ij}(\phi)\left(\psi^i\nabla_z\psi^j\s+\s
\tilde\psi^i\nabla_{\bar z}\tilde\psi^j\right)\hs{1.8cm}\cr
&&\hs{0.7cm}-\ {_1\over^2}\s
R_{ijkl}(\phi)\s\s\psi^i\psi^j\tilde\psi^k\tilde\psi^l\bigg)\m
dz\wedge d\bar z
\label{Susc}
\qqq
where \s$\nabla_{z}\psi^j=\da_z\psi^i+\Ga^j_{kl}
\s\m\da_z\phi^k\s\m\psi^l\s$ with \s$\Ga^j_{kl}=\{{j\atop kl}\}\s$
standing for the Levi-Civita connection symbols
and similarly for \s$\nabla_{\bar z}\tilde\psi^i\s$ and where
\s$R_{ijkl}\s$ denotes the curvature tensor of \s$M\s$.
\s Addition of the 2-form \s$\omega=b_{ij}(\phi)\s
d\phi^i\wedge d\phi^j\s$ term  corresponds to the change
\s$g_{ij}\mapsto g_{ij}+b_{ij}\s$ in eq.\s\s(\ref{Sus}).
In the component formula (\ref{Susc}) it results in the same
replacement of \s$g_{ij}\s$ and, additionally,
in the replacement of the Levi-Civita symbols in \s$\nabla_z\psi^j\s$
(\s$\nabla_{\bar z}\tilde\psi^j\s$) by symbols of a metric
connection with torsion \s$\{{j\atop kl}\}\pm
{3\over 2}\m g^{jm}\s H_{klm}\s$, \s respectively,
where \s$H_{klm}\s$ is the
antisymmetric tensor representing \s$d\omega\s$. \s The
curvature \s$R_{ijkl}\s$ in eq.\s\s(\ref{Susc}) becomes
that of the connection with the plus sign\footnote{it is equal to
\s$R_{klij}\s$ for the minus sign connection}.
\vskip 0.4cm

The two-dimensional field theory with action (\ref{sm})
is usually called a {\bf sigma model}. The stationary points
of \s$S(\phi)\s$ are harmonic maps from \s$\Sigma\s$ to \s$M\s$
and correspond to the classical solutions. Can one quantize sigma
models by giving sense to functional integrals
\qq
\int\limits_{Map(\Sigma,M)}F(\phi)\ \ee^{\m-\m S(\phi)}
\s\s D\phi
\label{fio}
\qqq
with \s$S=S_g+S_{top}+S_{tach}+S_{dil}\s$ where  for
\s$F(\phi)\s$ one may for example take
\s$\prod_ju_j(\phi(x_j))\s$ for some functions \s$u_j\s$
on \s$M\s$? \s
We have already seen that this was easily doable for \s$M\s$
a torus with a constant metric and a constant 2-form
\s$\omega\m$, with vanishing (or constant) tachyon and dilaton
potentials. \m The corresponding functional integral
was essentially Gaussian and the resulting theory was a little
decoration of the free massless field. Here we would like
to examine the case with an arbitrary topology and
geometry of the target by treating the functional integrals
of the type (\ref{fio}) in perturbation theory and also,
possibly, going beyond the purely perturbative considerations
employing powerful methods of the renormalization group
approach to quantum field theories.
\vskip 1.3cm


\noindent{\bf 3.\ \s Regularization and renormalization}
\vskip 0.4cm

\no We may anticipate problems with the definition
of functional integrals (\ref{fio}) even in a
perturbative approach. We shall attempt to remove
these problems by using freedom
to change the parameters of the theory,
namely the metric on \s$M$\s and the tachyon and dilaton potentials
(for the sake of simplicity, we shall discard the topological term
in the action). The strategy to make sense of functional integrals
of type (\ref{fio}) will then be as follows:
\vskip 0.1cm

1. ({\bf regularization}) \ we modify the theory introducing a
(short distance) cutoff \s$\La\s$ into it to make functional
integral exist;
\vskip 0.1cm

 2. ({\bf renormalization}) \ we try to choose the metric
\s$g\s$ and the tachyon and dilaton potentials entering
the action in a \s$\La$-dependent
way so that the cutoff versions of integrals
(possibly after further multiplication
by a \s$\La$-dependent factor) converge to a non-trivial
limit.
\vskip 0.3cm

There are many ways to introduce a short distance (ultra-violet)
cutoff into the theory. To simplify the problem further
let us assume that \s$\Sigma\s$ is the periodic box
\s$[-{0},{L}]^2\s$ (that will do away with the contribution
of the dilaton potential). One possibility to introduce
the UV cutoff is to consider the lattice version of the
sigma model. Let \s$\Sigma_{\Lambda}\subset[0,L]^2\s$ be composed
of points with coordinates in \s${1\over\La}\NZ\s$ where \s$\La L\s$
is a power of \s$2\s$. \s The lattice version of
\s$\phi:[0,L]^2\rightarrow
M\s$ is the map \s$\phi:\Sigma_\La\rightarrow M\s$ and for the cutoff
action we may put
\qq
S^\La_{g,u}(\phi)\ =\ {_1\over^{8\pi}}
\sum\limits_{{x,y\in \Sigma_\La\atop |x-y|=\La^{-1}}}
d_g^{\s 2}(\phi(x),\phi(y))\s+\s{_1\over^{4\pi}}\sum_{x\in\Sigma_\La}
\La^{-2} u(\phi(x))
\non
\qqq
where \s$d_g\s$ stands for the metric distance on \s$M\s$. \s
If \s$\phi\s$ is the restriction of a fixed smooth (periodic)
\s$M$-valued map on \s$[0,L]^2\s$ and \s$u\s$ is continuous
then in the limit \s$\La\to\infty\s$ we recover the value of
the original action \s$S_g+S_{tach}\equiv S_{g,u}\s$.
The cutoff version of the normalized integral
(\ref{fio}) with \s$F(\phi)=\prod u_j(\phi(x_j))\s$ becomes now
\qq
{\int\prod\limits_{j=1}^nu_j(\phi(x_j))\ \ee^{-S^\La_{g,u}(\phi)}
\s\s D_g\phi\over\int\ee^{-S^\La_{g,u}(\phi)}
\s\s D_g\phi}
\label{corff}
\qqq
where \s$D_g\phi=\prod\limits_{x\in\Sigma_\La}dv_g(\phi(x))\s$
with \s$dv_g\s$ denoting the metric volume measure on \s$M\s$.
The integral is finite e.\s g.\s\s for compact \s$M\s$
and \s$u_j\s$, say, continuous. The lattice sigma models
for \s$M=S^{N-1}\s$ with a metric proportional to that of
the unit sphere in \s$\NR^N\s$ are essentially well known
``spin'' models in classical (as opposed to quantum) statistical
mechanics (\s$N=1\s$ corresponds to the Ising model,
\s$N=2\s$ to a version of the XY model, \s$N=3\s$ to a slightly
modified classical Heisenberg model; \s$u\s$ proportional
to a coordinate in \s$\NR^N\s$ describes the coupling to the
magnetic field).
\vskip 0.4cm


We would like to study if, after renormalization,
the cutoff may be removed in the correlation
functions\footnote{from the point of view
of statistical mechanics which reformulates the problem
in terms of the system with a fixed lattice spacing, this is a question
about the large distance behavior of correlation functions} (\ref{corff}).
More precisely we would like to show that the limits
\qq
\lim\limits_{\La\to\infty}\ \s\m
{\int\prod\limits_{j=1}^n
Z(\La)u_j(\phi(x_j))\ \ee^{-S^{\s\La}_{g(\La),u(\La)}(\phi)}\ D_{g(\La)}
\phi\s\over\int\ \ee^{-S^{\s\La}_{g(\La),u(\La)}(\phi)}\ D_{g(\La)}
\phi\s}
\label{RP}
\qqq
exist for a cutoff-dependent linear map\footnote{recall that
even the free field case required a multiplicative
renormalization of the correlation functions
of exponents of field \s$\phi\s$} \s$Z(\La)\s$
on the space of functions on \s$M\s$ and for cutoff-dependent
choices of the metric \s$g(\La)\s$ and of the tachyon potential
\s$u(\La)\s$  on \s$M\s$. \s
We would also like to parametrize possible limits (\ref{RP})
defining the correlation functions of the quantum two-dimensional
sigma models.
\vskip 1cm


\noindent{\bf 4.\ \s Renormalization group effective actions}
\vskip 0.4cm

One could study the questions raised above first
by perturbative methods applied directly to the lattice
correlation functions (\ref{corff}). It is important, however
to set the perturbative scheme in the way that
does not destroy the geometric features of the model
(i.e. in a way covariant under diffeomorphisms
of \s$M\m$). One way to assure this is to study, instead of
correlation functions, objects known under the name of
renormalization group effective actions.
\vskip 0.3cm

Fix \s$\mu\s$ such that \s$\La/\mu\s$ is a power of 2
and for \s$y\in\Sigma_\mu\s$ denote by \s$B(y)\s$
the set of \s$x\in\Sigma_\La\s$ in the square
\s$y+[0,{\mu\over\La}L[^2\s$,
\s Call a point \s$\phi\in M\s$
a barycenter of a set of points \s$\phi_j\in M\s, j=1,\dots,N,\s$
if \s$\phi\s$ extremizes \s$\sum_jd_g^2(\phi,\phi_j)\s$.
\s Clearly, if \s$M\s$ is a euclidean space then
\s$\phi={1\over^N}{\sum_j\phi_j}\s$. \s
Suppose that we fix a map \s$\phi:\Sigma_\mu\rightarrow
M\s$ and compute the integral
\qq
\ee^{-S^{\s\mu}_{eff}(\phi)}\s\s D\phi\ =\ \int\prod\limits_{y\in\Sigma_\mu}
\delta({_1\over^2}\nabla_{\phi(y)}\hs{-0.2cm}
\sum\limits_{x\in B(y)}\hs{-0.2cm}
d_g^2(\phi(y),\varphi(x))\m)
\s\ \ee^{-S^\La_{g,u}(\varphi)}\ D_g\varphi\s.
\label{efac}
\qqq
The right hand side is naturally a measure
on \s$M^{\Sigma_\mu}\s$. \s It essentially computes the probability
distribution of the barycenters \s$\phi(y)\s$ of ``spins'' \s$\varphi(x)\s$
with \s$x\s$ in blocks \s$B(y)\s$\footnote{this would hold
if the barycenters were unique and with the normalizing
factor \s${1\over\int\delta({_1\over^2}\nabla_{\phi(y)}
\hs{-0.2cm}\sum\limits_{x\in B(y)}\hs{-0.2cm}
d_g^2(\phi(y),\varphi(x)))\s\s
D_g\phi}\s$ under the integral on the right
hand side of eq.\s\s(\ref{efac})}.
\s$S_{eff}^{\s\mu}(\phi)\s$, \s the logarithm of the
the density of the right hand side w.r.t. to
some reference measure \s$D\phi\s$ on \s$M^{\Sigma_\mu}\s$,
is called the (``block spin'') {\bf renormalization group}
(RG) {\bf effective action} on scale \s$\mu\s$.
\vskip 0.3cm

The renormalizability
problem may now be reformulated as the question about existence
of the \s$\La\to\infty\s$ limit of \s$S_{eff}^{\s\mu}\s$,
\s more exactly, of the normalized measure
\qq
d\nu_{eff}^{\s\mu}(\phi)
={\ee^{-S_{eff}^{\s\mu}(\phi)}\s\s D\phi\over
\int\ee^{-S_{eff}^{\s\mu}(\phi)}\s\s D\phi}
\label{renme}
\qqq
on \s$M^{\Sigma_\mu}\s$, \s  if we choose ``bare'' \s$g\s$ and \s$u\s$
in the \s$\La$-dependent way and keep \s$\mu\s$ fixed.
With the modification
of the definition of \s$S_{eff}^{\s\mu}\s$ described in
the footnote, one may show that the two formulations
of the renormalizability problem are essentially equivalent.
\vskip 0.3cm

The limiting measures \s$d\nu_{eff}^{\s\mu}\s$ may be viewed
as describing the \s$\La=\infty\s$ theory averaged over
variations of the fields on distance scales \s$\leq\mu^{-1}\s$.
Pictorially, they describe the system viewed from far away
when we do not distinguish details of length $\mathop{<}\limits_{^\sim}
\s{1\over\mu}\s$. \s
An important observation at the core of the RG analysis
is that this averaging may be done inductively by first
eliminating the variations on the smallest scales, then
on the larger ones, and so on until scale \s$\mu\s$ is
reached\footnote{this is not exactly the case for
our definition of \s$S_{eff}^{\s\mu}\s$ but ignore
this for a moment}. In the infinite volume
(\s$L=\infty\s$), the process
may be viewed as a repeated application of a map
on a space of unit lattice actions. If under the iterations
the effective actions are driven to a simple attractor
(like an unstable manifold of a fixed point) then the
renormalization consists of choosing the initial
``bare'' actions so that the \s$\La=\infty\s$ effective
actions \s$S_{eff}^{\mu}\s$ end up on the attractor.
In the vicinity of a fixed point this would be possible
if the family of the bare actions (parametrized by
bare couplings) crosses transversally the stable manifold.
The renormalized couplings parametrize then the unstable
manifold (a drawing would be helpful here).
This dynamical system view of renormalization developed
by K. G. Wilson is extremely important and will hopefully
be explained in much more details in future lectures.
\vskip 0.3cm


We have suppressed in the notation the dependence on the size \s$L\s$
of the box. If the choice of \s$g(\La)\s$ and \s$u(\La)\s$ involved
in the \s$\La\to\infty\s$ limit can be done in an \s$L\s$ independent
way, we automatically obtain a family of measures parametrized
by \s$\mu\s$ and \s$L\s$. The measures with the product \s$L\mu\s$ fixed
(to a power of 2) are related by the rescaling of
space-time distances\footnote{i.\s e.\s\s of the metric on \s$\Sigma\m$}.
\m If the infinite volume limit \s$L\to\infty\s$
of the theory exists, \s$\mu\s$ becomes a continuous
parameter. Suppose that the effective actions \s$S_{eff}^{\s\mu}\s$
of possible continuum limits (i.e. the attractor of the RG
map in the dynamical system view) may be parametrized
by (dimensionless) ``renormalized'' metrics \s$g\s$ and tachyon potentials
\s$u\s$. \s You should think that \s$S_{eff}^{\s\mu}\s$ is equal to
\s$S^{\s\mu}_{g,\m\mu^2u}\s$ plus less important (higher
dimension) terms separated by a precise rule. We would say then that
the theory is renormalizable by a metric and a tachyon potential
renormalization. This is the scenario realized in
perturbation theory, see below. In such a situations
the \s$\La,L=\infty\s$ theories are characterized
by the ``running'' metric \s$\mu\mapsto g(\mu)\s$
and tachyon potential \s$\mu\mapsto u(\mu)\s$ describing
\s$S_{eff}^{\s\mu}\s$ on different scales \s$\mu\s$
in the passive view of the scale-dependence of the
renormalized theory. In the active view, the \s$\mu$-dependence
of \s$g\s$ and \s$u\s$ is generated by the action of rescalings
of distances on the limiting theory. The infinitesimal scale
transformations generate a vector field
\s$\beta\da_g+\gamma\da_u\s$ in the space of \s$(g,u)\s$
defined by
\qq
\beta(g,u)=\mu{_\da\over^{\da\mu}}\m g\s,
\quad\quad\gamma(g,u)=\mu{_\da\over^{\da\mu}}\m u\s.
\label{RGE}
\qqq
$\beta(g,u)\s$ and \s$\gamma(g,u)\s$ are called in the
physicists jargon the RG ``{\bf beta}'' and ``{\bf gamma functions}''.
In the dynamical system language, \s$\beta\da_g+\gamma\da_u\s$
is a vector field on the attractor of the RG map
and it extends the map to a flow.
The importance of the RG functions lies in the fact that, computed
in perturbation expansion, they allow to go beyond it,
providing for example a consistency check
on the latter: by solving the RG eqs.\s\s (\ref{RGE})
with \s$\beta\s$ and \s$\gamma\s$ given by few perturbative
terms we may check whether the trajectories
\s$\mu\mapsto(g(\mu),u(\mu))\s$ stay or are driven out
for large \s$\mu\s$ (that is at short distances)
from the region of the \s$(g,u)$-space
where the perturbative calculation may be trusted.
It is clear that the zeros of the \s$(\beta,\gamma)\s$
vector field should play an important role. They correspond
to scale invariant (and hence conformal
invariant) field theories and, in the dynamical system picture,
to fixed points of the RG map (since they lie already
on the attractor).
\vskip 0.4cm

How to generate the perturbation expansion for the
RG effective actions \s$\ee^{-S_{eff}^{\s\mu}}\s$?
\s A helpful observation is that the delta-function
in the definition (\ref{efac}) can be rewritten
in simple terms if we use the exponential
map \s$\ee:T_\phi M\rightarrow M\s$.
\vskip 0.5cm


\no{\bf Problem 1}\s\s\ (geometric). \ Show that for
vectors \s$\xi_j\s$ in a small ball in
\s$T_\phi M\s$,
\qq
{_1\over^2}\nabla_{\phi}(\sum\limits_{j=1}^Nd^2_g(\phi,\ee^{\xi_j}
\phi)\ =\ \sum\limits_{j=1}^N\xi_j\s.
\non
\qqq
It follows that \s$\phi\s$ is a barycenter of the set of points
\s$\{\ee^{\xi_j}\phi\}\s$ iff \s$\sum\xi_j=0\s$.
\vskip 0.5cm


\no Substituting in eq.\s\s(\ref{efac}) \s$\varphi(x)
=\ee^{\xi(x)}\phi(y)\s$ for
\s$\xi(x)\in T_{\phi(y)}M\s$ if \s$x\in B(y)\s$,
or, in a shorthand notation, \s$\varphi=\ee^\xi\tilde\phi\s$
where \s$\tilde\phi(x)=\phi(y)\s$ for \s$x\in B(y)\s$,
we obtain\footnote{the fact that the exponential
parametrization may work only locally does not
impede the perturbative analysis}
\qq
\ee^{-S_{eff}^{\s\mu}(\phi)}\ =\
\bigg[\int\prod\limits_{y\in\Sigma_\mu}\delta(\hs{-0.2cm}\sum
\limits_{x\in B(y)}\hs{-0.2cm}\xi(x))\ \s
\ee^{-{1\over{\hh}}\m S^\La_{g,u}
(\ee^\xi\tilde\phi)}\ D_g\ee^\xi\tilde\phi\s\bigg]\bigg|_{\hh=1}\s.
\label{efact}
\qqq
Note that the lattice field \s$\xi\s$ takes values in
a vector space \s$\{\s\xi\s\s|\s\s\xi(x)\in T_{\phi(y)}M\s,\hs{-0.1cm}
\sum\limits_{x\in B(y)}\hspace{-0.2cm}\xi(x)=0\s\}\m$.
\s The loop expansion for
\s$S_{eff}^{\s\mu}\s$ may just be generated in the standard way
by expanding in powers of \s$\xi\s$ on the right hand side
of eq.\s\s(\ref{efact}) all terms except for the quadratic
contribution to \s$S_{g,u}\s$ which is used to produce a Gaussian
measure. At each loop order the result will be invariant
under the simultaneous action of diffeomorphisms of \s$M\s$
on \s$\phi\s$, \s$g\s$ and \s$u\s$. When \s$\La\to\infty\s,$
\s divergences will appear in the perturbative expressions.
The perturbative renormalizability of the theory may be studied
by replacing the ``bare'' \s$g\s$ and \s$u\s$ on the right hand side
of eq.\s\s(\ref{efact}) by
\qq
g(\La)\ =\ g\s+\s\sum\limits_{n=1}^\infty
{\hh}^n\s\m\delta g_n(g,u,\La/\mu)\s,\quad\quad
u(\La)\ =\ \mu^2\left(u\s+\s\sum\limits_{n=1}^\infty
{\hh}^n\s\m\delta_n u(g,u,\La/\mu)\right)\s.
\label{renn}
\qqq
We may attempt to fix the above series
by choosing some way to extract the renormalized metric
\s$g\s$ and the renormalized potential
\s$u\s$ from effective actions \s$S_{eff}^{\s\mu}\s$.
We would then like to show that the above substitution
cancels the \s$\La\to\infty\s$ divergences in each loop order of
\s$S_{eff}^{\s\mu}\s$ resulting in a  family of perturbative
RG effective actions parametrized by ``running'' metric
\s$g(\mu)\s$ and tachyon potential \s$u(\mu)\s$.
Differentiation of the series (\ref{renn}) over \s$\ln{\mu}\s$
with \s$g(\La),\s u(\La)\s$ fixed would then
produce in the \s$\La\to\infty\s$ limit the loop expansion for
the beta and gamma RG functions.
\vskip 1.2cm





\noindent{\bf 5.\ \s Background field effective action}
\vskip 0.5cm





In practice, the lattice perturbative calculations are prohibitively
complicated. It would be easier to work with continuum
regularization and renormalization which allow to calculate the diagram
amplitudes by momentum space integrals and to make use of rotational
invariance. We have seen in Witten's lectures on perturbative renormalization
of the scalar field theories with the \s$\phi^3\s$ or \s$\phi^4\s$
interactions that it was convenient to analyze directly the
``1PI effective action'' \s$\Ga\s$ given by the Legendre transform
of the free energy functional \s$W\s$. The latter was defined
as the logarithm of the integral of type of (\ref{fio}),
with \s$F(\phi)=\ee^{\m\langle\m\phi\m,\m J\m\rangle}\s$. \s
The definition of both \s$W\s$ and \s$\Gamma\s$, \s however,
as well as their perturbative analysis, used heavily the linear
structure in the space of maps from the space-time
to the target \s$M$\s, inherited from the linear structure
of \s$M\m$. \m Such structure is missing if \s$M$\s
is a general manifold. It is possible, nevertheless, to introduce
for sigma models an effective action mimicking
the construction of the large deviations function
(see eq.\s\s(\ref{PN})\m) and somewhat similar in spirit to
the RG effective actions for the lattice version of
sigma models discussed in the previous section.
Instead of fixing the block barycenters in a single
lattice theory, we shall take \s$N\s$ independent
copies of continuum theories with fields \s$\phi_j\s$
and shall fix for each \s$x\s$ the barycenters
\s$\phi(x)\s$ of \s$\phi_j(x)\s$ defining the functional
\qq
P_N(\phi)\ =\ \int\prod\limits_x\delta({_1\over^2}\nabla_{\phi(x)}
\sum\limits_{j=1}^N d_g^2(\phi(x),\phi_j(x)))\ \prod\limits_j
\ee^{-S_{g,u}(\phi_j)}\ D_g\phi_j
\label{PNN}
\qqq
by a formal functional integral. Note that the right hand side
reduces to a well defined integral for a lattice version of
the theory. For a map \s$\phi:\Sigma\rightarrow M\s$ and for
a section \s$\xi\s$ of the pullback \s$\phi^*TM\s$
of the bundle tangent to \s$M\s$, denote by \s$\ee^\xi\phi\s$
the map from \s$\Sigma\s$ to \s$M\s$ whose value at
point \s$x\s$ is obtained by applying the exponential
map to \s$\xi(x)\in T_{\phi(x)}M\s$. \s Reparametrizing
in eq.\s\s(\ref{PNN}) \s$\phi_j=\ee^{\xi_j}\phi\s$, \s we
obtain
\qq
P_N(\phi)\ =\ \int\delta(\sum\limits_{j=1}^N\xi_j)
\s\s\prod\limits_j\s\ee^{\m-S(\ee^{\xi_j}\phi)}
\s\s D(\ee^{\xi_j}\phi)\s.
\label{PN1}
\qqq
We may try to extract the ``{\bf background field
effective action}'' \s$\Ga_{b}(\phi)\s$ from the leading
contribution to \s$P_N\s$ at large \s$N\s$:
\qq
P_N(\phi)\ =\ \ee^{\m N\s\Ga_{b}(\phi)\ +\ \s o(N)}\s.
\label{efab}
\qqq
It should be clear that \s$\Gamma_{b}(\phi)\s$ coincides
then with the effective action \s$\Ga_\phi(\zeta=0)\s$
of the \s$\xi$-field theory (depending on \s$\phi\s$ as
a parameter) corresponding to the functional integral
\qq
\int\ -\ \s\ee^{-S(\ee^\xi\phi)}\ D(\ee^\xi\phi)\s.
\label{non}
\qqq
Fields \s$\xi\s$ take values in a vector space of sections
of \s$\phi^*TM\s$ so that the perturbative treatment
of the \s$\xi$-theory is more standard. Note that only
the geometric structure on \s$M\s$ was used in the formal
definition of \s$\Ga_{b}(\phi)\s$.
\vskip 0.5cm


We shall reformulate the renormalizability problem
(\ref{RP}) for the second time
as the question about existence
of the \s$\La\to\infty\s$ limit of the regularized
version of the background-field effective action
\s$\Gamma_{b}^{\La}(\phi)\s$ for a cutoff-dependent
theory with the action \s$S_{g(\La),u(\La)}(\phi)\s$.
We could regularize the functional integral
(\ref{PN1}) by putting fields \s$\xi_j\s$
on a lattice with spacing \s${1\over\La}\s$ while
keeping \s$\phi\s$ as a continuum field. This would not
produce a big computational gain in comparison to
the perturbative calculation of the RG effective
actions. It is possible, however, to regularize the
loop expansion of the background field effective
action just by introducing an ultraviolet cutoff in
the momentum space integrals for the 1PI vacuum
amplitudes in the \s$\xi$-field theory whose fields
form a vector space of sections of \s$\phi^*TM\s$.
\s$\Ga_{b,\La}(\phi)\s$ obtained this way will be
covariant under the diffeomorphisms
of \s$M\s$ in each order of the loop expansion.
\s The perturbative renormalization
will consist of choosing the ``bare'' parameters
of the theory in a cuttoff-dependent way
as in eqs.\s\s(\ref{renn}) and such that
the \s$\La\to\infty\s$ limit of \s$\Ga_{b}^\La(\phi)\s$
exists order by order in the loop expansion.
The perturbative limits will be parametrized
by the ``running'' metric \s$g(\mu)\s$ and potential \s$u(\mu)\s$
on \s$M\s$, \s with the change of \s$\mu\s$ induced
by rescaling of distances on \s$\Sigma\s$.
\vskip 1.2cm


\noindent{\bf 6.\ \s Dimensional regularization}
\vskip 0.4cm

\no We shall prove the perturbative
renormalizability of the background field
effective action in the 2$D$ sigma model
only in the leading order of the loop expansion,
concentrating instead on the discussion
of the renormalization group aspects of the 1 loop
result. To avoid calculational
(but not conceptual) difficulties, we shall work
in the flat euclidean space-time \s$\Sigma=\NE^2\s$.
We shall also use a specific scheme for regularization
of divergent diagrams: the {\bf dimensional regularization}
and a particular way to renormalize the theory
(i.\s\s e. to chose \s$g(\La)\s$ and \s$u(\La)\m)\m$:
\s the {\bf minimal subtraction}. Briefly, the idea
is to
\vskip 0.2cm

1. regularize the momentum space integrals by
rewriting them as integrals into which the space-time
dimension \s$D\s$ enters as an analytic (complex) parameter,
then
\vskip 0.2cm

2. to calculate the integral for the values of \s$D\s$
where it converges and, finally,
\vskip 0.2cm

3. to analytically continue
to the physical values of \s$D\s$ extracting the pole parts
of the result at the physical dimension as the divergence
to be removed by the renormalization.
\vskip 0.6cm


In order to gain some practice let us compare how the
simplest divergent diagram of the 4-dimensional \s$\phi^3\s$
theory \s${\Large{-\hs{-0.1cm}{\rm O}\hs{-0.1cm}-}}\s$ is regularized
and renormalized first in the more conventional momentum
space regularization used in Witten's lecture and then
in the dimensional regularization - minimal subtraction scheme.
The momentum space amplitude \s$\hat I(k)\s$
corresponding to the (amputated) graph
was given by the integral
\qq
\hat I_D(k)\ =\ {_{g^2}\over^4}
\int{\dd q\over(q^2+m^2)\m((q-k)^2+m^2)}\ =\
{_{g^2}\over^4}
\int_0^1d\alpha\int{\dd q\over (q^2+\alpha(1-\alpha)k^2+m^2)^2}
\label{ampel}
\qqq
where \s$k\s$ is the external momentum, \s$q\s$ is that
of the loop (both euclidean) and \s$\dd q \equiv{d^Dq\over{(2\pi)^D}}\s$.
In space-time dimension \s$D=4\s$ the \s$q$-integral
diverges logarithmically. It may be regularized by restricting
the integration to \s$|q|\leq\La\s$.
\qq
\hat I_4^\La(k)\ =\ {_{g^2}\over^4}
\int_0^1d\alpha\int\limits_{|q|\leq\La}
{\dd q\over (q^2+\alpha(1-\alpha)k^2+m^2)^2}\ =\
{_{g^2}\over^{32\pi^2}}\s\m\ln{{_\La\over^\mu}}\s+\s
\hat K^\La_4({_{k^2}\over^{\mu^2}},{_m\over^\mu})
\non
\qqq
where
\qq
\lim\limits_{\La\to\infty}
\hat K^\La_4({_{k^2}\over^{\mu^2}},{_m\over^\mu})
\ =\ -{_{g^2}\over^{64\pi^2}}\left(\int_0^1 d\alpha
\s\s\ln{[\alpha(1-\alpha)\m{_{k^2}
\over^{\mu^2}}\m+\m{_{m^2}\over^{\mu^2}}]}
\s+\s1\right)
\label{finite}
\qqq
The renormalization idea is then to substitute
\qq
&g(\La)\ =\ \mu\la\s,\quad\quad
m^2(\La)\ =\ \mu^2\left(r\s+\s\sum\limits_{n=1}^\infty{\hh}^n\s
\delta r_n(\la,r,{_\La\over^\mu})\right)&
\label{momsc}
\qqq
for the coupling constant and the mass squared in the initial
action (recall that the theory does not need renormalization
of \s$g\s$ and only 1 loop counterterm would do).
The powers of \s$\mu\s$ make \s$\la\s$
and \s$r\s$ dimensionless. With a choice
\qq
\delta r_1\s=\s {_{\la^2}\over^{16\pi^2}}\s
\ln{{_\La\over^\mu}}
\non
\qqq
the contribution to the 1 loop amplitude
\s$I_4^\La(k)\s$ diverging when \s$\La\to\infty\s$
is canceled resulting in the renormalized
value of the amplitude
\qq
\hat I_{4,ren}^\La(k)\ =\
-{_{\mu^2\m\la^2}\over^{64\pi^2}}\left(\int_0^1 d\alpha
\s\s\ln{[\alpha(1-\alpha)\m{_{k^2}
\over^{\mu^2}}\m+\m{r}]}
\s+\s1\right)
\non
\qqq
The RG functions \s$\beta(\la,r)=\mu{d\over d\mu}\m\la\s,
\ \s\gamma_2(\la,r)=\mu{d\over d\mu}\m r\s$ describing the
scale-dependence of the renormalized couplings are obtained
by differentiating eq.\s\s(\ref{momsc}) over \s$\ln{\mu}\s$
with \s$\La,g(\La)\s$ and \s$m^2(\La)\s$ held fixed:
\qq
0&=&\mu\m\la\s+\s\mu^2{_{d\m\la}\over^{d\mu}}\s,\cr
0&=&\mu{_d\over^{d\mu}}\s[\m\mu^2 r\s+\s\hh\m{_{g^2}\over^{16\pi^2}}
\s\ln{_\La\over^\mu}\s\m+\s\s\CO({\hh}^2)\m]\ =\
2\m\mu^2\m r\s+\s\mu^3{_{d\m r}\over^{d\mu}}\s-\s\hh
\s\mu^2\m{_{\la^2}\over^{16\pi^2}}\s\m+\s\s\CO({\hh}^2)
\non
\qqq
so that
\qq
\beta(\la,r)\s=\s-\la\s,\quad\quad\gamma_2(\la,r)\s=\s
-2 r\s+\s\hh\s{_{\la^2}\over^{16\pi^2}}\s\m+\s\s\CO({\hh}^2)\s.
\label{RGmom}
\qqq
\vskip 0.4cm


Let us see how the same problem is treated
in the dimensional regularization - minimal subtraction
scheme. Using the relation \s$\int_0^\infty\ee^{-a\sigma}\s\m\sigma
\s\m d\sigma=\s a^{-2}\s$, \s we may rewrite the integral
for \s$\hat I_D(k)\s$
in the form\footnote{for those who do not remember
Feynman's famous formula (I don't),
we could have used twice the identity \s$\int_0^\infty
\ee^{-a_i\sigma_i}\s\m d\sigma_i
=\s a_i^{-1}\s$ in the original expression for \s$\hat I_D(k)\s$
changing then the variables to \s$(\alpha,\sigma)\s$ where
\s$\sigma_1=\alpha\sigma\s,\ \sigma_2=(1-\alpha)\sigma\s$}
\qq
\hat I_D(k)\ =\ {_{g^2}\over^4}
\int_0^1d\alpha
\int_0^\infty d\sigma\int\sigma\s\m\ee^{-[q^2+\alpha
(1-\alpha)k^2+m^2]\m\sigma}
\s\m\dd q\s.
\label{iNt}
\qqq
Performing the \s$q$-integral first, we obtain
\qq
\hat I_D(k)\ =\ {_{g^2}\over^{2^{D+2}\pi^{D/2}}}
\int_0^1d\alpha
\int_0^\infty d\sigma\s\s\sigma^{1-D/2}\s\s\ee^{-[\alpha
(1-\alpha)k^2+m^2]\m\sigma}\s.
\non
\qqq
The latter integral converges for any complex \s$D\s$ with
\s${\rm Re}\m D<4\s$. \s It gives explicitly
\qq
\hat I_D(k)\ =\ {_{g^2}\over^{2^{D+2}\pi^{D/2}}}\s\s\Ga(2-{_D\over^2})
\int_0^1d\alpha\s\s[\alpha(1-\alpha)\m k^2+m^2]^{{D\over 2}-2}\s.
\non
\qqq
The divergence in four dimensions manifests itself as a pole
in the expression at \s$D=4\s$;
\qq
\hat I_D(k)\ =\ {_{g^2}\over^{32\pi^{2}}}\s{_1\over^{4-D}}
\s-\s{_{g^2}\over^{64\pi^{2}}}\left(\int_0^1d\alpha\s\s\ln{[
\alpha(1-\alpha)\m k^2+m^2]}\s+\s\ln{4\pi}\s+\s C\right)
\s\s+\ \CO(4-D)
\non
\qqq
where \s$C=-\Ga'(1)\s$ is the Euler constant.
In the minimal subtraction renormalization scheme,
the amplitude is renormalized by substituting
for the original coupling and mass squared
\qq
&g\ =\ \mu^{3-D/2}\m\la\s,\quad\quad
m^2\ =\ \mu^2\left(r\s+\s\sum\limits_{n=1}^\infty{\hh}^n
\s\delta r_n(\la,r)\right)&
\label{msub}
\qqq
with the pure pole dependence of \s$\delta r_n\s$
on the dimension:
\qq
\delta r_n=\sum\limits_{j=1}^{k_n}\delta r_{n,j}
(\la,r)\s{_1\over^{(4-D)^{j}}}
\non
\qqq
chosen to cancel exactly the pole part of the dimensionally
regularized amplitudes. In reality, only the 1 loop amplitude
has a pole which is simple. With
\qq
\delta r_1\ =\ {_{\la^2}\over^{16\pi^2\m(4-D)}}\s,
\non
\qqq
we obtain the renormalized amplitude
\qq
\hat I_{4,ren}^\La(k)\ =\ -\s{_{\mu^2\m\la^2}\over^{64\pi^{2}}}
\left(\int_0^1d\alpha\s\s\ln{[
\alpha(1-\alpha)\m {_{k^2}\over^{\mu^2}}+r]}\s
+\s\ln{(4\pi)}\s+\s C\right)
\non
\qqq
(note how \s$\mu\s$ has entered under the logarithm).
The difference between the two renormalizations
may be absorbed into a finite redefinitions of the renormalized
parameters \s$\la,\s r\s$, \s see Problem 2 below.
\vskip 0.4cm

Now the RG functions \s$\beta(\la,r),\s\m\gamma_2(\la,r)\s$
are obtained by differentiating eqs.\s\s(\ref{msub})
with respect to \s$\ln{\mu}\s$
while keeping \s$g\s$ and \s$m^2\s$ fixed:
\qq
0&=&(3-{_D\over^2})\m\mu^{3-D/2}\m\la\s+\s\mu^{4-D/2}\m{_{d\m\la}
\over^{d\mu}}\s,\cr
0&=&\mu{_d\over^{d\mu}}\s[\m\mu^2\m r\s+\s\hh\s{_{\mu^2\m\la^2}
\over^{16\pi^2(4-D)}}\m]\ =\
\mu{_d\over^{d\mu}}\s[\m\mu^2\m r\s+\s\hh\s{_{\mu^{D-4}\m g^2}
\over^{16\pi^2(4-D)}}\m]\cr
&=&2\m\mu^2\m r\s+\s\mu^3{_{d\m r}\over^{d\mu}}\s-\s\hh
\s\mu^2\s{_{\la^2}\over^{16\pi^2}}
\non
\qqq
so that
\qq
\beta(\la,r)\s=\s({_D\over^2}-3)\m\la\s,
\quad\quad\gamma_2(\la,r)\s=\s-2r\s
+\s\hh\s{_{\la^2}\over^{16\pi^2}}\s.
\label{RGms}
\qqq
The gain is that we have obtained formulae for
\s$\beta\s$ and \s$\gamma_2\s$ in general dimension.
They may serve as an indication of how the field
theory behave in smaller or larger dimension
then the one considered. For example, the vanishing of
the linear contribution to \s$\beta(\la)\s$ in 6 dimensions
signals that the theory becomes only renormalizable there.
Note that eqs.\s\s(\ref{RGms})
reduce to (\ref{RGmom}) at \s$D=4\s$. \s This did
not have to be the case since we have changed the parametrization
of the limiting theories and what is geometrically defined
is the vector field \s$\beta\da_\la+\gamma_2\da_r\s$.
\vskip 0.6cm

\no{\bf Problem 2}. \ Find to the 1 loop order the transformation
between the coordinates \s$(\la,r)\s$ of the renormalized
4-dimensional \s$\phi^3\s$ theory corresponding to
the passage between the two renormalization schemes
discussed above. Show that it preserves the form
of the vector field \s$\beta\da_\la+\gamma_2\da_r\s$.
\vskip 0.6cm

\no{\bf Problem 3}. \ Find the running couplings \s$\la(\mu),\s r(\mu)\s$
using the 1 loop approximations to the \s$\beta,\s\gamma_2\s$
functions. What can one tell about the effect of higher loop
corrections to the large \s$\mu\s$ (UV) behavior of the running
couplings?
\vskip 0.6cm

\no The four-dimensional \s$\phi^3\s$ theory, in spite of its
super-renormalizability (only a finite number of divergent
1PI graphs) and self-consistency of its perturbative calculations,
has a non-perturbative stability problem related to the
lack of a lower bound for the cubic polynomial. This
should serve as a warning that even the RG improved
perturbative analysis is not enough to assure existence
of a renormalized QFT.
\vskip 1.2cm


\noindent{\bf 7.\ \s Renormalization of the sigma models
to 1 loop}
\vskip 0.5cm

\no As mentioned above, the background field effective
action \s$\Gamma_{b}(\phi)\s$ of the sigma model
is equal to the \s$\zeta=0\s$ value of the effective
action of the \s$\xi$-theory with the action \s$S_\phi(\xi)\s$
given by the relation
\qq
\ee^{-S_{g,u}(\ee^{\xi}\phi)}\
D_g(\ee^{\xi}\phi)\ =\
\ee^{-S_\phi(\xi)}\ D\xi\s.
\non
\qqq
The 1$^{\rm\s st}\s$ of eqs.\s\s(\ref{Pe}) implies then
that in the perturbation expansion
\qq
\Ga_{b}(\phi)\ =\ S_\phi(0)\s\m-
\hs{-0.2cm}\sum\limits_{1{\rm PI\s\m vacuum}\atop
{\rm graphs}\s\m G}\hs{-0.2cm}I_G(S_\phi)\s\m-
\m\s\ln{(\int\ee^{-{1\over 2}\s S_\phi''(0)(\xi^2)}
\s\m D\xi)}\s.
\label{imin}
\qqq
One of the simplifying features of the dimensional
regularization is that we may disregard the terms
in \s$S_\phi(\xi)\s$ coming from the logarithm
of the ``Radon-Nikodym derivative''
\s${D_g(\ee^{\xi}\phi)\over D\xi}\s$. \s Formally,
they are proportional to \s$\delta^{(2)}(0)=\int\dd q\s$
which vanishes in the dimensional
regularization.
\vskip 0.5cm

\no{\bf Problem 4}. \ (a). \ (for Pasha). \ Show by calculating
the $D$-dimensional integral \s$\int\ee^{-q^2}\s\m dq\s$
in two ways that the volume of the unit sphere in
\s$D\s$ dimensions is equal to
\s$2\pi^{{D\over2}}/\Ga({_D\over^2})\s$.
\vskip 0.2cm

\no (b). \ Show that in the radial variables the
\s$D$-dimensional integral \s$\int\limits_{|q|\geq\epsilon}dq\s$
converges for \s${\rm Re}\s D<0\s$:
\qq
\int\limits_{|q|\geq\epsilon}\dd q\s=\s-\s{_{2^{1-D}\pi^{-D/2}}\over
^{D\s\Ga(D/2)}}\s\s\ep^D\s.
\non
\qqq
Defining the value of the integral by analytic
continuation for integer \s$D\geq 0\s$ and taking
\s$\ep\s$ to zero we infer that \s$\int\dd q\s$
vanishes in positive dimensions
in the dimensional regularization.
\vskip 0.5cm

\no Expanding in local coordinates
\qq
(\ee^{\xi}\phi)^i\ =\ \phi^i+\xi^i\s-\s{_1\over^2}
\s \Ga^i_{jk}(\phi)\s\xi^j\xi^k\s+\s\CO(\xi^3)
\non
\qqq
where \s$\Ga^i_{jk}=\{{i\atop jk}\}\s$ is the Levi-Civita symbol,
we obtain after a little calculation (you may do it!)
\qq
&&S_{g,u}(\ee^\xi\phi)\ =\ {_1\over^{4\pi}}\int
\bigg(g_{ij}(\phi)\s\s\da_\nu\phi^i\s\s
\da_\nu\phi^j\ +\  u(\phi)\
+\ 2\s g_{ij}(\phi)\s\s\da_\nu\phi^i\s\s\nabla_\nu\xi^j\
+\ \da_iu(\phi)\s\s\xi^i\s\cr\cr
&&+\ g_{ij}(\phi)\s\s\nabla_\nu\xi^i\s\s\nabla_\nu\xi^j
\ -\ R_{ijkl}(\phi)\s\s\da_\nu\phi^i\s\s\da_\nu\phi^k
\s\s\xi^j\s\s\xi^l\ +\ {_1\over^2}\nabla_i\da_j
u(\phi)\s\s\xi^i\m\xi^j\bigg)\m dx\s\s+\s\ \CO(\xi^3)\s.
\label{exP}
\qqq
Above, \s$\nabla_i\s$ denotes the covariant derivative over
the \s$i^{\rm \s th}\s$ coordinate and
\qq
\nabla_\nu\xi^i\s=\s\da_\nu\xi^i\s+\s\Ga^i_{jk}(\phi)\s\s
\da_\nu\phi^j\s\s\xi^k\s.
\non
\qqq
The vector bundle \s$\phi^*TM\s$ has a natural metric given by
\s$\phi^*g\s$. \s It will be convenient to choose a global
orthonormal frame \s$(e_a)_{a=1}^{\s d}\s$ of \s$\phi^*TM\s$
(\m$d\s$ denotes the dimension on \s$M\m$). \s In coordinates,
\s$e_a=e_a^i\da_{\phi^i}\s$. \s Different choices of \s$e_a\s$
are related by gauge transformations \s$e'_a=\la_a^b\m e_b\s$
with \s$\la\s$ an \s$SO(d)$-valued function
on the plane \s$\NE^2$. \m Of course \s$e_a^i\s$ depend
also on field \s$\phi\s$. \s It will be more convenient to rewrite
\qq
\xi=\xi^a\m e_a\quad\quad{\rm or}\quad\quad
\xi^i=\xi^a\m e_a^i
\non
\qqq
with \s$(\xi^a)\s$ a sequence of functions on \s$\NE^2\m$.
\m The expansion (\ref{exP}) becomes then
\qq
&&S_{g,u}(\ee^\xi\phi)\ =\ {_1\over^{4\pi}}\int\bigg(g_{ij}(\phi)
\s\s\da_\nu\phi^i\s\s\da_\nu\phi^j\ +\  u(\phi)\
+\ \s 2\s e^a_i\s\s\da_\nu\phi^i\nabla_\nu\xi^a\
+\ e_a^i\s\da_iu(\phi)\s\s\xi^a\s\cr\cr
&&+\ \s\s(\nabla_\nu\xi^a)^2
\ -\ R_{iakb}(\phi)\s\s\da_\nu\phi^i\s\s\da_\nu\phi^k\s\s\xi^a
\s\s\xi^b\ +\ {_1\over^2}\s e^i_a\s e^j_b\s\nabla_i\da_j
u(\phi)\s\s\xi^a\m\xi^b\bigg)\m dx\s\s+\ \s\CO(\xi^3)\s.
\label{exP1}
\qqq
where \s$(e^a_i)\s$ is the matrix inverse to \s$(e_a^i)\s$,
\s$R_{iakb}=e_a^je_b^l\m R_{ijkl}\s\s$ and
\qq
\nabla_\nu\xi^a\s=\s\da_\nu\xi^a\s+\s A^{\s a}_{b\m\nu}
\m\xi^b\quad\quad
{\rm with}\quad\quad A^{\s a}_{b\m\nu}\s=\s
e^a_i\left(\da_\nu e_b^i\m+\m \Ga^i_{jk}\m\da_\nu\phi^j\m e_b^k\right)\s.
\non
\qqq
Clearly, \s$A^{\s a}_{b\m\nu}\s dx^\nu\s$ transforms as an \s$SO(d)\s$
connection form under the gauge transformations
\s$e_a\mapsto \la_a^b\m e_b\s$.
The perturbation expansion (\ref{imin}) for \s$\Ga_b(\phi)\s$
becomes now to the 1 loop order:
\qq
&&\hs{-1.25cm}\Ga_{b}(\phi)\ =\ {_1
\over^{4\pi{\hh}}}\int\left(g_{ij}(\phi)\s\s
\da_\nu\phi^i\s\s\da_\nu\phi^j\s+\s u(\phi)\right) dx\s\s\cr\cr
&&-\ \s\ln\left(\int\ee^{-{1\over4\pi}\int\left(
(\nabla_\nu\xi^a)^2
\ -\ R_{iakb}(\phi)\s\s\da_\nu\phi^i\s\s\da_\nu\phi^k\s\s\xi^a
\s\s\xi^b\ +\ {_1\over^2}\s e_a^ie_b^j\nabla_i\da_j u
(\phi)\s\s\xi^a\m\xi^b\right)\s dx}\s\s D\xi\right)
\s\s+\ \CO({\hh})\hs{0.4cm}
\label{gab}
\qqq
(the terms linear in \s$\xi\s$
on the right hand side of (\ref{exP}) do not contribute
to \s$\Ga_b\s$). \s The functional integral gives
a determinant and we could use the zeta-function prescription
to make sense out of it in an \s$SO(d)$-gauge-invariant way.
We shall be, however, more interested
in the divergent part of the determinant than in its renormalized
value. The dimensional
regularization will allow to extract the divergence in a convenient
(\s$SO(d)$-gauge-invariant) manner and,
besides, it works to all orders.
\vskip 0.3cm


We shall obtain the expression for the 1 loop contribution
to \s$\Ga_b(\phi)\s$ expanded around a constant value
\s$\phi_0\s$ of \s$\phi\s$ in the form (in coordinates
around \s$\phi_0\s$)
\qq
[\Ga_b(\phi)]^{1\s{\rm loop}}\s=\
\sum\limits_n\int K_{n,D}(x_1,\dots,x_n;\phi_0)\s\prod\limits_{j=1}^n
(\phi(x_j)-\phi_0)\s\s dx_j
\label{kere}
\qqq
with the translationally invariant kernels \s$K_{n,D}\s$
regularized dimensionally, i.\s e.\s\s meromorphically
dependent on \s$D\s$, with possible poles at
\s$D=2\s$ and with \s$\phi-\phi_0\s$ vanishing fast
at infinity. In order to generate the expansion (\ref{kere})
for the 1 loop contribution to \s$\Ga_b(\phi)\s$, we shall separate
the term
\qq
{_1\over^{4\pi}}\int\left((\da_\nu\xi^a)^2
\s+\s{_1\over^2}\s(e_a^i e_b^j\m\nabla_i\da_ju)|_{\phi_0}\s\xi^a\m
\xi^b\right)
\m dx
\non
\qqq
in the action as producing the Gaussian
measure\footnote{we assume that the matrix
\s$(\m(e_a^i e_b^j\m\nabla_i\da_ju)|_{\phi_0})\s$
is positive which is the case, for example, in the vicinity
of a minimum of \s$u\s$} from
\qq
{_1\over^{4\pi}}\int\bigg(2\s A^{\s a}_{b\m\nu}
\s\da_\nu\xi^a\m\s\xi^b
\ +\ A^{\s a}_{b\m\nu}\s A^{\s a}_{c\m\nu}\s\m\xi^b\s\m\xi^c\s
+\s{_1\over^2}\left(e_a^i e_b^j\m\nabla_i\da_ju(\phi)\s-
\s(e_a^i e_b^j\m\nabla_i\da_ju)|_{\phi_0}\right)
\m\xi^a\m\xi^b\ \ \cr
-\s R_{iakb}(\phi)\s\s\da_\nu\phi^i\s\s\da_\nu\phi^k\s\s\xi^a\s\m\xi^b
\bigg)\m dx
\non
\qqq
treated as an interaction. Now it is easy to enumerate
the divergent graphs. First there are logarithmically
divergent contributions coming from the graphs
\qq
\smash{\mathop{{\Large{{\rm O}}}\hs{-0.232cm}{_{{\Large\bullet}}}}
\limits_{\atop A_\nu\s A_\nu}}\quad\quad\quad
\quad\quad\quad\quad\quad\quad\quad\quad\quad
_{^{A_\nu}}\s\s^{_\bullet}\hs{-0.12cm}{\Large{\rm O}}
\hs{-0.09cm}^{_\bullet}\s\s_{^{A_\la}}\ \ .
\non
\qqq
\vskip 0.2cm
\no They cancel each other (there are no divergences in
2D gauge theory). The only divergent terms we are left with are
\qq
\smash{\mathop{{\Large{{\rm O}}}\hs{-0.232cm}{_{{\Large\bullet}}}}
\limits_{\atop-\m R\s\m\da\phi\s\da\phi}}\ ,
\quad\quad
\smash{\mathop{{\Large{{\rm O}}}\hs{-0.232cm}{_{{\Large\bullet}}}}
\limits_{\atop{1\over^2}e\m e\m\nabla\da\m u\s-\s{1\over^2}
(e\m e\m\nabla\da\m u)|_{\phi_0}}}
\quad\quad{\rm and}\quad\quad\quad
{_1\over^2}\s\s\ln\ \det{\m{_1\over^{2\pi}}\left(
-\delta_{ab}\s\Delta\s+\s{_1\over^2}\s
(e_a^i e_b^j\nabla_i\da_j\m u)|_{\phi_0}\right)}\ .
\label{ppin}
\qqq
All other contributions are easily checked to
have finite limits at \s$D=2\s$.
\s Since in the dimensional regularization
\qq
&\int{{\dd q}\over{q^2+m^2}}\ =\ \int_0^\infty d\sigma
\int\ee^{-\sigma\m(q^2+m^2)}\s\s\dd q\ =\
2^{-D}\m\pi^{-D/2}\int_0^\infty d\sigma\s\s\sigma^{-D/2}
\s\s\ee^{-\sigma\m m^2}&\cr\cr\cr
&=\ 2^{-D}\m\pi^{-D/2}\s\s m^{D-2}
\s\s\Ga(1-D/2)\ =\ {1\over{2\pi}}\s\s{1\over{2-D}}
\ +\ {\rm part\ regular\ at\ }D=2\s,&
\non
\qqq
the pole part of the loops in (\ref{ppin}) is equal to
\qq
{_1\over^{4\pi}}\s\m{_1\over^{2-D}}\int\left(-\m R_{iaka}(\phi)
\s\s\da_\nu\phi^i\s\s\da_\nu\phi^k\s+\s{_1\over^2}
(\Delta_gu(\phi)-\Delta_gu(\phi_0))\right)\m dx\s
\non
\qqq
(since \s$e_a^ie_a^j\m\nabla_i\da_j\m u\s=\s g^{ij}
\nabla_i\da_j\m u\s=\s\Delta_gu\s$
where \s$\Delta_g\s$ denotes the Laplacian on \s$M\m$).
\s Similarly,
\qq
&{1\over2}\s\s\ln\ \det{\m{1\over{2\pi}}\left(
-\s\Delta\m\delta_{ab}\s
+\s{1\over2}\s(e_a^ie_b^j\s\nabla_i\da_j\m u)|_{\phi_0}
\right)}\ =\ {1\over2}\int dx\int\tr\s\s\ln\s{1\over2\pi}
\left(q^2\s+\s{1\over2}\s e\s e\m\nabla\da\m u\m|_{\phi_0}\right)
\s\dd q&\cr\cr\cr
&=\ {\rm const}.
\ +\ {1\over2}\int dx\int_0^1 dt\s\s\tr\s\s{1\over{q^2
\s+\s{t\over2}\s e\s e\m\nabla\da\m u\m|_{\phi_0}}}
\s\s{1\over2}\s e\s e\m\nabla\da\m u\m|_{\phi_0}\s\s\dd q&\cr\cr\cr
&=\ {\rm const}.\ +\ {1\over{8\pi}}\s\m{1\over{2-D}}
\int \Delta_gu(\phi_0)\s\s dx\s.&
\non
\qqq
We infer that the pole part of \s$\Gamma_g(\phi)\s$ to the 1 loop
order is
\qq
[\Ga_{b}(\phi)]_{_{\rm div}}^{^{1\ {\rm loop}}}\ =\ \
-\m{_1\over^{4\pi}}\s\m
{_1\over^{2-D}}\int R_{ij}(\phi)\s\s \da_\nu\phi^i\s\s\da_\nu\phi^j
\s\s dx\ +\ {_1\over^{8\pi}}\s\m{_1\over^{2-D}}\int\Delta_g
u(\phi)\s\s dx
\non
\qqq
where \s$R_{ij}=R_{iaja}\s$ is the Ricci tensor on \s$M\s$.
\s$[\Ga_{b}(\phi)]_{_{\rm div}}^{^{1\ {\rm loop}}}\s$
is an integral of dimension 2 and dimension 0 operators
and this result, in accord with a simple power counting,
remains true at higher orders.
\vskip 0.3cm

The minimal subtraction renormalization scheme
adds counterterms to bare metric \s$g\s$ and bare potential \s$u\s$
which cancel the above poles. More exactly, one substitutes
in the initial action
\s$S_{g_0,u_0}(\phi)\s$ of the model with the bare metric
\s$g_0\s$ and bare tachyon potential \s$u_0\s$,
\qq
g_0\s&=&\s\mu^{D-2}\m(\m g\s+\s{_{\hh}\over^{2-D}}\s\m\delta g_1
\s\s+\s\s\CO({\hh}^2)\s)\s,\label{br1}\\\cr
u_0\s&=&\s\mu^D\m(\m u\s+\s{_{\hh}\over^{2-D}}\s\m\delta u_1
\s\s+\s\s\CO({\hh}^2)\s)\s.
\label{br2}
\qqq
The added 1 loop counterterms change the effective action by
\qq
\delta\Ga_b(\phi)\ =\ {_1\over^{4\pi}}
\s\m{_{\mu^{D-2}}\over^{2-D}}\int\left(\delta {g_{1}}_{ij}(\phi)
\s\s\da_\nu\phi^i\s\s\da_\nu\phi^j\ +\ \mu^2\s\s\delta u_1(\phi)
\right)\m dx\ +\ \s\CO(\hh)
\non
\qqq
and we put
\qq
\delta {g_1}_{ij}\ =\ R_{ij}\s,\quad\quad\quad \delta u_1\ =\
-\m{_1\over^2}\s\m\Delta_g u\s.
\non
\qqq
canceling the poles at \s$D=2\s$. \s This proves the renormalizability
of the sigma model (background field effective action) to 1 loop.
\vskip 1cm


\noindent{\bf 8.\ \s Renormalization group analysis of sigma models}
\vskip 0.4cm

\no Let us compute the vector field \s$\beta\da_g+\ga\da_u\s$
on the space
of metrics and potentials as given by the minimal subtraction
version of eqs.\s\s(\ref{RGE}):
\qq
\beta(g,u)\ =\ \mu{_{\da}\over^{d\mu}}
\s g\s\bigg|_{g_0={\rm const.}\atop u_0={\rm const.}}\ ,
\quad\quad\quad
\gamma(g,u)\ =\ \mu{_{\da}\over^{d\mu}}
\s u\s\bigg|_{g_0={\rm const.}\atop u_0={\rm const.}}\ .
\label{rgms}
\qqq
Applying the derivative \s${d\over d\ln{\mu}}\s$ to
eq.\s\s(\ref{br1}), we obtain
\qq
&0\ =\ \mu\m{{d}\over{d\mu}}\bigg[\mu^{D-2}\left(g_{ij}\s+\s{{\hh}
\over{2-D}}\s\s R_{ij}\s\s+\ \CO({\hh}^2)\right)\bigg]
\ =\ \mu\m{{d}\over{d\mu}}\bigg[\mu^{D-2}\left(g_{ij}\s+\s{{\hh}
\over{2-D}}\s\s {R_0}_{ij}\s\s+\ \CO({\hh}^2)\right)\bigg]&\cr
&=\ \mu^{D-2}\bigg[\beta_{ij}(g)\s-\s(2-D)\m g_{ij}\s-\s\hh\s R_{ij}
\s\s+\ \CO({\hh}^2)\bigg]&
\non
\qqq
from which we infer that
\qq
\beta_{ij}(g)\ =\ (2-D)\m g_{ij}\s+\s\hh\s\m R_{ij}\s\s+\ \CO({\hh}^2)\s.
\label{Beta}
\qqq
Similarly,
\qq
&\mu\m{{d}\over{d\mu}}\bigg[\mu^{D}\left(u\s-\s{{\hh}
\over{2(2-D)}}\s\m\Delta_g u\s\s+\ \CO({\hh}^2)\right)\bigg]
\ =\
\mu\m{{d}\over{d\mu}}\bigg[\mu^{D}u\s-\s\mu^{D-2}\s\m{{\hh}
\over{2(2-D)}}\s\m\Delta_{g_0} u_0\s\s+\ \CO({\hh}^2)\bigg]&\cr
&=\ \mu^D\bigg[\gamma(u)\s+\s D\m u\s+\s{{\hh}\over2}\s\Delta_gu
\s\s+\ \CO({\hh}^2)\bigg]&
\non
\qqq
so that
\qq
\ga(u)\ =\ -\ D\m u\ -\ {_1\over^{2}}\s\hh\s\Delta_g u\s\s
+\ \CO({\hh}^2)
\label{Pote}
\qqq
\vskip 0.4cm

The vector field \s$\beta\s$ in the space of metrics
may be used to find out in which situations we may expect
the perturbative calculation to be self-consistent.
The condition is that the running metric \s$g(\mu)\s$
satisfying the RG equation
\qq
\mu\m{_d\over^{d\mu}}\m g\ =\ \beta(g)
\label{runn}
\qqq
stays on all scales \s$\mu\geq\mu_0\s$ in the perturbative
regime. Let us illustrate
this on the example where \s$M=S^{N-1}\s$
with the metric \s${1\over\alpha'}\s$ times the
induced metric \s$g\s$ of the unit sphere in \s$\NR^{N}\s$.
Due to the rotational symmetry, the renormalized
metric is \s${1\over\alpha'(\mu)}\m g\s$ and the eq.\s\s(\ref{runn})
for its \s$\mu$-dependence reduces in \s$D=2\s$ to
\qq
\mu\m{_d\over^{d\mu}}\s\alpha'\ =\ \hh\s(2-N)
\s(\alpha')^2\s+\ \CO({\hh}^2)\s.
\non
\qqq
Clearly, for \s$N>2\s$, \s$\alpha'\s$ is driven to zero
for large \s$\mu\s$
approximately as \s$\CO({1\over\ln{\mu}})\s$.
\s The perturbative regime corresponds to small \s$\alpha'\s$
so that the perturbative expansion is self-consistent for
the sigma model with \s$M=S^{N-1}\s$ for \s$N>2\s$.
The phenomenon is called the {\bf asymptotic freedom} of the spherical
sigma model since \s$\alpha'=0\s$ corresponds to a free theory.
It permits to expect that the theory may be constructed
non-perturbatively, at least in finite volume.
Such a non-perturbative theory would break the conformal invariance
of the classical sigma model. In fact, there are strong reasons to believe
that its infinite volume version is massive
(an exact expression for its \s$S$-matrix is, conjecturally, known).
\vskip 0.3cm

The property of asymptotic freedom is shared by all the
sigma models which have compact symmetric spaces as their
targets (and also, more importantly, by the non-abelian
4-dimensional gauge theories with not too many fermion species,
like Quantum Chromodynamics (QCD) describing the strong
interactions of quarks, mediated by \s$SU_3\s$ gauge fields).
\vskip 0.5cm


\no{\bf Problem 5}. \ Consider the flow
\qq
\mu{_{d}\over^{d\mu}}\m\alpha\s=\s-\alpha^2\s,\quad\quad
\mu{_{d}\over^{d\mu}}\m u\s=\s-2\m u\s+\s u\m\alpha
\qqq
in \s$\NR^2\s$. Show that there exist only one perturbative
solution
\qq
u\s=\s\sum\limits_{n=0}^\infty a_n\m\alpha^n
\qqq
for the invariant manifold of the flow. Study the
(non-perturbative) invariant manifolds.
\vskip 0.5cm


Specially interesting cases correspond to manifolds with
vanishing Ricci curvature. The \s$N=2\s$ spherical
sigma model is the simplest example
(coinciding with free field with values in \s$S^1\s$).
As we know, it corresponds to a CFT.
One may then read from the \s$\gamma\s$ function the
dimensions equal to \s$d_q={1\over^2}\m q^2\s$ of the composite
operators given by the exponential functions \s$\ee^{\m iq\phi}\s$
on \s$S^1\s$. \s$d_g\s$ are equal to the eigenvalues
of \s$-{1\over^2}\Delta_g\s$ (the tree contribution to
the \s$\gamma\s$ comes from the fact that we have considered
integrated insertions of the composite operator into
the action). For Ricci flat targets there is no
renormalization of the metric in the 1 loop order and,
in the supersymmetric versions, up to 4 loops (4 loops
excluded). No renormalization of the metric means that
the beta function vanishes and the scale invariance
is preserved (to 4 loops). One may then argue in perturbation
theory (studying the Hessian of the 1 loop \s$\beta\s$)
that in the K\"{a}hlerian case, the Ricci flat metric may
be perturbed as to give rise to a scale invariant
quantum sigma model with \s$N=2\s$ superconformal
symmetry (as discussed by Ed Witten during the lecture).
Thus Calabi-Yau (\s$\cong\s$ K\"{a}hler, Ricci flat) manifolds
should correspond to superconformal \s$N=2\s$ field theories.
This observation resulted in a conjectured mirror symmetry
between Calabi-Yau manifolds, now established in many instances.
\vskip 0.2cm

In the case of SUSY sigma models with hyper-K\"{a}hler targets
(i.e. with the \s$N=4\s$ supersymmetry), \s$\beta\s$ vanishes to
all orders of the loop expansion.
\vskip 0.3cm


The inclusion of the 2-form term (\ref{stop})
into the action of the sigma models
modifies the above results. In the 1
loop order, the beta function $\mu{d\over d\mu}(g_{ij}+b_{ij})\s$
is given by the Ricci
curvature of the metric connection with torsion
\s\s$\Ga^i_{jk}=\{{i\atop{jk}}\}
+{3\over 2}\m g^{il}\m H_{jkl}\s\s$ where
the antisymmetric tensor \s$H_{jkl}\s$ corresponds
to \s$\pm d\omega\s$. In models in which the connections
with torsion are globally flat, the beta function vanishes to
all orders (even without supersymmetry). The WZW model
of CFT, which we shall discuss in the next lecture, corresponds
to such a situation. Addition of the 2-form
which is closed does not modify the \s$\beta\s$ function
but may change the long-distance behavior of the model
(that seems to happen for the sigma model with
\s$S^2\s$ target where the addition of the term
with \s$\omega\s$ equal to \s$\pi\s$ times
the volume form of the unit sphere should render
the model massless).
\vskip 0.4cm

As for the renormalization of the potentials whose
scale-dependence is governed by the RG equation
\qq
\mu\m{_d\over^{d\mu}}\s u\ =\ -2u\s-\s{_1\over^2}\s\hh\s\Delta_g u
\s\s+\ \CO({\hh}^2)
\non
\qqq
note that, on a symmetric space, \s$u\s$ (approximately)
reproduces itself up to a normalization if it belongs to
an eigen-subspace of the Laplacian.
The RG analysis allows then to predict the
short distance behavior of the correlation functions
involving insertions of the corresponding composite operators
(somewhat similarly as for \s$M=S^1\s$).
\vskip 0.8cm

\no{\bf References}
\vskip 0.4cm

For the rudiments of the perturbative approach to functional
integrals, Feynman graphs etc. see again the book by Zinn-Justin,
Sects. 5.1-5.3 and Kazhdan's, Witten's and Gross' lectures in the
present series.
\vskip 0.3cm

The original reference to the renormalization of
geometric sigma models is Fiedan's thesis published with few
years delay in Ann. Phys. 163 (1985), p. 318. The SUSY case
is discussed in Alvarez-Gaume-Freedman-Mukhi, Ann. Phys. 134 (1981)
p. 85, with further developments in Alvarez-Gaum\'{e}-Ginsparg,
Commun. Math. Phys. {102} (1985), p. 311, Alvarez-Gaum\'{e}-S.
Coleman-Ginsparg, Commun. Math. Phys. {103} (1986), p. 423 and
Grisaru-Van De Ven-Zanon, Nucl. Phys. {B 277} (1986), p. 388 and p. 409
(the last papers discovered 4$^{\m\rm th}$ order contributions
to the supesymmetric beta function). For the case of sigma models
with a 2-form in the action see Braaten-Curtright-Zachos, Nucl. Phys.
{B 260} (1985), p. 630 and also Callan-Friedan-Martinec-Perry, Nucl.
Phys. B 262 (1985), p. 593.
\eject


\
\vskip 0.8cm
\no{\large{\bf{Lecture 4.\ \ Constructive conformal
field theory}}}
\addtocounter{equation}{-46}
\vskip 0.8cm
\no{\un{Contents}}:
\vskip 0.5cm

1. \ WZW model

2. \ Gauge symmetry Ward identities

3. \ Scalar product of non-abelian theta functions

4. \ KZB connection

4. \ Coset theories

5. \ WZW factory

\vs 1.7cm

Let us recall the logical structure of this course.
In the first lecture we studied the free field
examples of CFT's. In the second one,
we analyzed the scheme of (two-dimensional)
CFT from a more abstract, axiomatic
point of view. In the third one, we searched
perturbatively among geometric sigma models for non-free
examples of CFT's. Finally, in the present lecture
compressed due to lack of time, we shall analyze a specially important
sigma model, the Wess-Zumino-Witten (WZW) one, whose
correlation functions may be constructed non-perturbatively,
with a degree of explicitness comparable to that attained
for toroidal compactifications of free fields
(constituting the simplest examples of WZW theories).
The WZW model appears to be a generating theory
of a vast family of CFT's whose
correlations can be expressed in terms of the WZW ones.
The comparison of the non-perturbative models obtained
this way with the perturbative constructions of sigma models
allows for highly non-trivial tests of differences between
the geometry of Ricci flat (Einstein) spaces and that
of CFT's, replacing the Einstein geometry
in the stringy approach to gravity.
\vskip 1cm

\no{\bf{1. \ WZW model}}
\vskip 0.6cm

The target space of the WZW sigma model is a compact Lie group
manifold $G$ and the two-dimensional theory may be considered
as a generalization of quantum mechanics of a particle moving
on $G$. In the latter case the (Euclidean) action functional is
\qq
S(g)\s=\s-\m{_1\over^2}\int\tr\s(g^{-1}\da_xg)^2\s\s dx
\qqq
where \s$"\tr"\s$ denotes the Killing form\footnote{normalized
so that the long roots have length squared 2}.
Let \s$R\s$ denote an (irreducible) unitary representation
\s$g\mapsto g_{_R}\s$ of \s$G\s$ in a (finite dimensional) Hilbert
space $V_{_R}$. \s The path integral for the quantum-mechanical
particle on $G$, corresponding to the Wiener measure on $G$,
may be solved with the use of the Feynman-Kac
formula taking the form
\qq
\int\limits_{Map([0,L]_{per},\m G)}\hs{-0.5cm}
{\mathop{\otimes}\limits_{i=1}^n}
g_{_{R_i}}(x_i)\ \ee^{-k\m S(g)}\s\prod\limits_x dg(x)\ \bigg/
\hs{-0.3cm}\int\limits_{Map([0,L]_{per},\m G)}\hs{-0.5cm}
\ee^{-k\m S(g)}\s\prod\limits_x dg(x)\cr
=\ {\rm Tr}\ \ee^{-x_1H}\s g_{_{R_1}}\s
\ee^{\m(x_2-x_1)H}\s g_{_{R_2}}\s\dots\s g_{_{R_n}}\s
\ee^{\m(L-x_n)H}\ \bigg/\ {\rm Tr}\ \ee^{-LH}
\label{FKG}
\qqq
where \s$0\leq x_1\leq x_2\leq\cdots\leq x_n\leq L\s$,
\s$2k\m H\s$ is the Laplacian on $G$ and $g_{_R}$ is viewed as
a matrix of multiplication operators, both acting in \s$L^2(G,dg)\s$
($dg$ is the Haar measure). Compare Problem 3 in Lecture 1
dealing with the case \s$G=S^1\m$. \m The theory possesses
the $G\times G$ symmetry
which may be used to solve it: the right hand side of (\ref{FKG})
is calculable in terms of the harmonic analysis on $G$.
\vs 0.7cm


\no{\bf Problem 1}.\ \ {\it Compute explicitly the 1-,2- and
3-point functions in} (\ref{FKG}).
\vs 0.8cm

The Euclidean action of the WZW model is a functional
on $Map(\Sigma,G)$ where $\Sigma$ is a compact Riemann
surface. If, for simplicity, we assume $G$ to be connected
and simply connected then
\qq
S(g)\ =\ -{_i\over^{4\pi}}\int\limits_\Sigma\tr\s\s g^{-1}\da g
\wedge g^{-1}\de g\s+\s{_i\over^{4\pi}}\int\limits_\Sigma g^*\omega
\label{actWZW}
\qqq
where the 2-form \s$\omega\s$ is defined on open subsets
\s$\CO\subset G\s$ with $H_2(\CO)=0$
and satisfies there \s$d\omega=-{1\over^3}\s\tr\s\m
(g^{-1}dg)^{\wedge3}\s$. \s The dependence
of the second term of \s$S(g)\s$ on the choice of \s$\omega\s$
makes \s$S(g)\s$ defined modulo \s$2\pi i\NZ\s$ so that
\s$\ee^{-k\m S(g)}\s$ is well defined for integer $k$.
To have the energy bounded below, we shall take \s$k\m$,
called the level of the WZW model, positive.
\vs 0.7cm


\no{\bf Problem 2}.\ \ {\it Assuming, more generally, that}
\s$g\s$ {\it takes values in the complexified group} \s$G^\NC\m$
{\it find the equations
for stationary points of} \s$S(g)\m$.
\vs 0.7cm


\no The correlation functions of the WZW model
are formally given by the functional integrals:
\qq
\langle\s\mathop{\otimes}\limits_{i=1}^ng_{_{R_i}}(x_i)\s\rangle
\ \s = \int\limits_{Map(\Sigma,G)}
\hs{-0.2cm}\mathop{\otimes}\limits_{i=1}^n g_{_{R_i}}(x_i)
\ \ee^{-k\m S(g)}\ Dg\ \bigg/\hs{-0.2cm}
\int\limits_{Map(\Sigma,G)}
\hs{-0.2cm}\ee^{-k\m S(g)}\ Dg
\ \ \ \in\ \s\s\mathop{\otimes}\limits_{i=1}^n
{\rm End}(V_{_{R_i}})\hs{0.5cm}
\label{cOc}
\qqq
where $Dg$ stands for the formal product of the Haar
measures \s$dg(x)\s$ over \s$x\in\Sigma\s$.
\vs 0.3cm

As we have mentioned at the end of Lecture 3,
the renormalization group beta function computed
for the WZW sigma model vanishes
to all orders due to the flatness of the connections
with torsion generated from the metric and
the 2-form $\omega$ on $G$. Thus the model is conformally
invariant and does not
need renormalization of the action (\ref{actWZW})
in perturbation theory. The conformal invariance
holds, in fact, also non-perturbatively due to
the \s$LG\times LG\s$ symmetry of the theory where
\s$LG\s$ denotes the loop group \s$Map(S^1,\m G)\s$
of \s$G\m$. \s The WZW model may be solved exactly by

1. \ harmonic analysis on \s$LG\s$

\no or by

2. \ exact functional integration.

\no As we shall see, the (matrix-valued) composite operators
\s$g_{_R}(x)\s$ need multiplicative renormalization
and acquire scaling dimensions \s${2\m c_{_{R}}\over k+h^\vee}\s$
where \s$c_{_R}\s$ denotes
the quadratic Casimir of $R$ and \s$h^\vee\s$
stands for the dual Coxeter number of $G$ (the quadratic
Casimir of the adjoint representation).
\vskip 1.1cm



\no{\bf{2. \ Gauge symmetry Ward identities}}
\vskip 0.6cm


We shall sketch here the functional integral approach
to the WZW theory. It will be convenient to extend a little
the model by coupling it to an external gauge
field\footnote{in general, we shall not assume the unitarity
\s$A=-A^*\s$ of the gauge field}
\s$A=A^{10}+A^{01}\m$, \s a 1-form with values in
the complexified Lie algebra \s${\bf g}^{\NC}\s$ of \s$G\m$.
Define\footnote{a more standard definition subtracts
also \s$A^{10}\wedge A^{01}\s$ inside \s$[\s\dots\s]\s$}
\qq
S(g,A)\ =\ S(g)\s+\s{_{ik}\over^{2\pi}}\int\limits_\Sigma
\tr\s\s[\m A^{10}\wedge g^{-1}\de g\s+\s g\m\da
g^{-1}\wedge A^{01}\s+\s g\m A^{10}g^{-1}\wedge A^{01}]\s.\ \
\qqq
Under the ("chiral") gauge transformations corresponding
to maps \s$h_{1,2}:\s\Sigma\rightarrow G^\NC\s$
the action \s$S(g,A)\s$ transforms according to the
Polyakov-Wiegmann formula
\qq
S(h_2 g\m h_1^{-1},\s{}^{h_1}\hs{-0.14cm}A^{10}\m+
{}^{h_2}\hs{-0.14cm}A^{01})\ =\ S(g\m,\s A^{10}+A^{01})\s-\s
S(h_1,\s A^{10})\s-\s S(h_2^{-1},\s A^{01})
\qqq
where \s${}^{h_1}\hs{-0.14cm}A^{10}\m=\m h_1 A^{10}h_1^{-1}+
h_1\da h_1^{-1}\s$ and \s${}^{h_2}\hs{-0.14cm}A^{01}\m=\m
h_2 A^{01}h_2^{-1}+h_2\de h_2^{-1}\s$.
\vs 0.7cm


\no{\bf Problem 3}.\ \ {\it Prove the Polyakov-Wiegmann formula}.
\vs 0.7cm


\no In the presence of the external gauge field $A$,
the partition function of the WZW theory will be formally
defined as
\qq
Z_{_A}\ =\ \int\limits_{Map(\Sigma,G)}
\hs{-0.2cm}\ee^{-k\m S(g,A)}\ Dg
\label{PFW}
\qqq
and the correlation functions
\s$\langle\s\otimes g_{_{R_i}}(x_i)\s\rangle_{_A}\s$
by eq.\s\s(\ref{cOc}) with \s$S(g)\s$
replaced by \s$S(g,A)\s$ (no functional integration over $A$).
Using the formal extension to functional
integrals of the simple invariance property
\qq
\int\limits_G f(h_2 g\m h_1^{-1})\s\s dg\ =\
\int\limits_G f(g)\s\s dg
\non
\qqq
holding for \s$h_{1,2}\in G^\NC\s$ if \s$f\s$ is an
analytic function on \s$G^\NC\s$, we obtain
\qq
Z_{_{
\s{}^{^{h_1}}\hs{-0.14cm}A^{10}\m+
{}^{^{h_2}}\hs{-0.14cm}A^{01}}}\
\langle\s\otimes g_{_{R_i}}(x_i)\s\rangle_{_{
\s{}^{^{h_1}}\hs{-0.14cm}A^{10}\m+
{}^{^{h_2}}\hs{-0.14cm}A^{01}}}\
=\ \int\mathop{\otimes}\limits_{i=1}^n(h_2 g\m
h_1^{-1})_{_{R_i}}(x_i)
\ \ee^{-k\m S(h_2 g\m h_1^{-1},
\s{}^{^{h_1}}\hs{-0.12cm}A^{10}\m+\s
{}^{^{h_2}}\hs{-0.12cm}A^{01})}\ Dg\s\m\cr\cr
=\ \ee^{\m k\m S(h_1,\m A^{10})}\
\ee^{\m k\m S(h_2^{-1},\m A^{01})}\
\mathop{\otimes}\limits_i(h_2)_{_{R_i}}(x_i)\
Z_{_A}\ \langle\s\otimes g_{_{R_i}}(x_i)\s\rangle_{_A}
\ \mathop{\otimes}\limits_i(h_1)^{-1}_{_{R_i}}(x_i)\s.
\label{CWI}
\qqq
This is the chiral gauge symmetry Ward identity for
the correlation functions (recall the diffeomorphism
group and the local rescaling Ward identities discussed
in Lecture 2).
\vskip 0.3cm

The identity (\ref{CWI}) factorizes into a holomorphic
($A^{01}$-dependent) and an anti-holomorphic ($A^{10}$-dependent)
ones. Hence in order to study the \s$A^{01}$-dependence
of the correlation functions it is enough to look for holomorphic
maps on a Sobolev space of 0,1-forms\footnote{what follows does
not depend on the assumed degree of smoothness of
forms provided it is high enough}
\s$A^{01}\s$ with values in \s$\Ng\s$
\qq
\Psi\m :\ \CA^{01}\s\longrightarrow\s
\mathop{\otimes}\limits_{i=1}^n V_{_{R_i}}\ \equiv\ V_{_\NR}
\non
\qqq
satisfying the "factorized" Ward identity
\qq
\Psi({}^h\hs{-0.14cm}A^{01})\ =\ \ee^{\m k\m S(h^{-1},\m A^{01})}
\ \mathop{\otimes}\limits_{i=1}^n h_{_{R_i}}(x_i)\
\Psi(A^{01})\s.
\label{HWI}
\qqq
The relation (\ref{HWI}) describes the behavior of \s$\Psi\s$
along the orbits of the group \s$\CG^\NC\s$ of complex
(Sobolev-class) gauge transformations in \s$\CA^{01}\m$.
\s The orbit space \s$\CA^{01}/\CG^\NC\s$ is the moduli
space of holomorphic \s$G^\NC\s$ bundles which, upon
restriction to semi-stable bundles and appropriate
treatment of semi-stable but not stable ones, becomes
a compact variety \s$\CN$ of complex dimension \s 0\m,
\s${\rm rank}(G)\s$ and \s${\rm dim}(G)\s(h_{_\Sigma}-1)\s$
for genus \s$h_\Sigma\s$ equal to 0, 1 and $>1$, respectively.
The space \s$W(\Sigma,{\bf x},{\bf R},k)\s$
of \s$\Psi\m$'s \m coincides with the space
\s$H^0(\CV)\s$ of holomorphic sections of a vector bundle
\s$\CV\s$ over $\CN\s$ with typical fiber \s$V_{_{\NR}}\s$
(\m$\CV=\CA^{01}\times_{_{\CG^\NC}}V_{_{\NR}}\s$
essentially). In another description,
\s$W(\Sigma,{\bf x},{\bf R},k)=H^0(\CL)\s$ where \s$\CL\s$
is a line bundle over the moduli space of
holomorphic \s$G^\NC$-bundles with parabolic structures
at points \s$x_i\m$ and \s$\Psi\m$'s \m may be interpreted
as a non-abelian generalization of theta functions.
The essential implication of these
identifications is that \s$W(\Sigma,{\bf x},{\bf R},k)\s$
is a finite-dimensional space. Its dimension depends, in fact,
only on \s$h_{_\Sigma},\m k\s$ and \s${\bf R}\s$
and is given by the celebrated Verlinde formula.
\s$W(\Sigma,{\bf x},{\bf R},k)\s$ may be also identified
with the space of quantum states of the Chern Simons theory.
\vskip 0.3cm

Out of the global Ward identities (\ref{CWI})
one may extract the infinitesimal ones
by taking \s$h_i=\ee^{\Lambda_i}\s$
and Taylor-expanding in \s$\Lambda_i\s$ similarly as we analyzed
the infinitesimal consequences of the diffeomorphism
and rescaling Ward identities in Lecture 2. Define
the insertions of currents into the correlations by
\qq
&&\langle\s J^a_z\s\dots\s\rangle_{_A}\ =\ -\m\pi\s
{1\over Z_{_A}}\s\s{\delta\over\delta A^a_{\bar z}}\s\s Z_{_A}\s
\langle\s\dots\s\rangle_{_A}\s,\cr
&&\langle\s J^a_{\bar z}\s\dots\s\rangle_{_A}\ =\ -\m\pi\s
{1\over Z_{_A}}\s\s{\delta\over\delta A^a_{z}}\s\s Z_{_A}\s
\langle\s\dots\s\rangle_{_A}\s
\non
\qqq
(the subscript "a" corresponds to a basis \s$(t^a)\s$ of
the Lie algebra \s$\Ng\m$ s.t. $\tr\s\s t^a\m t^b={1\over 2}
\s\delta^{ab}\m$).
Denote by \s$J(z)\s$ (\s$\bar J(\bar z)\s$) \s the insertions
of \s$J_z\s$ (\s$\bar J_{\bar z}\s$) \s into correlations
with \s$A\s$ vanishing around the insertion point and the metric
locally flat. \s$J(z)\s$ (\s$\bar J(\bar z)\s$) \s depends
holomorphically (anti-holomorphically) on \s$z\s$ away from other
insertions. Expanding to the second order, one obtains
from the Ward identities (\ref{CWI}) the operator product expansions
\qq
&&J^a(z)\s\s J^b(w)\ =\ {\delta^{ab}\s k/2\over(z-w)^2}\s
+\s{i\s f^{abc}\over z-w}\s J^c(w)\ \s
+\ \ .\ .\ .\ \ ,\label{cur1}\\
&&\bar J^a(\bar z)\s\s \bar J^b(\bar w)\
=\ {\delta^{ab}\s k/2\over(\bar z-\bar w)^2}\s
-\s{i\s f^{abc}\over \bar z-\bar w}\s\bar J^c(\bar w)
\ \s+\ \ .\ .\ .\ \ ,\label{cur2}\\
&&J^a(z)\s\s \bar J^b(\bar w)\ =\ \s.\ .\ .
\label{cur3}
\qqq
which imply for the modes of the corresponding
Hilbert space operators \m$\CJ(z)=\sum_{_n}J_n\m z^{-n-1}$,
\s$\bar\CJ(\bar z)=\sum_{_n}\bar J_n\s{\bar z}^{-n-1}\s$
the Kac-Moody algebra relations
\qq
[J^a_n,\s J^b_m]\s=\s i\s f^{abc}\s J^c_{n+m}\s+\s{_1\over^2}
\s k\m n\s\m\delta^{ab}\s\delta_{n+m,0}
\non
\qqq
and similarly for \s$\bar J_n\s$ with the commutators
between \s$J_n\s$ and \s$\bar J_m\s$ vanishing.
\vs 0.5cm

\no{\bf Problem 4}.\ \ {\it Prove the operator product expansions}
(\ref{cur1}-\ref{cur3}).
\vs 0.7cm


Subtraction of the singular part from the expression
\s$\tr\s\s J(z)\s\m J(w)\s$ gives the Sugawara construction
of the energy-momentum tensor of the WZW theory:
\qq
T(w)\ =\ {_2\over^{k+h^\vee}}\ \lim\limits_{z\to w}\ \left(\tr
\s\s J(z)\s\m J(w)\s-\s{_{{\rm dim}(G)\s\m k}\over^{4\m(z-w)^2}}
\right)
\non
\qqq
and similarly for \s$\bar T(\bar w)\s$. \s In modes, this becomes
\qq
L_n\s=\s{_2\over^{k+h^\vee}}
\sum\limits_{m=-\infty}^\infty\tr\s\s: J_{n-m}\m J_{m}:
\non
\qqq
where the normal ordering puts \s$J_p\s$ with positive \s$p\s$
to the right of the ones with negative \s$p\s$.
\vskip 1cm



\no{\bf{3. \ Scalar product of non-abelian theta functions}}
\vskip 0.6cm


Since the \s$A^{10}$-dependence of the unnormalized
correlation functions
\s$Z_{_A}\s\s\langle\s\otimes g_{_{R_i}}(x_i)\s\rangle_{_A}
\s$ coincides with that
of \s$\overline{\Psi(-(A^{10})^*)}\s$ (recall that the
complex conjugate space \s$\overline{V_{_R}}
\m\cong\m V^*_{_R}\s$)\m, \s we must have
\qq
Z_{_A}\s\s\langle\s\otimes g_{_{R_i}}(x_i)\s\rangle_{_A}
\ \ \in\ \
W(\Sigma,{\bf x},{\bf R},k)
\otimes\overline{W(\Sigma,{\bf x},{\bf R},k)}
\non
\qqq
as a function of $A$ or, more explicitly,
\qq
Z_{_A}\s\s\langle\s\otimes g_{_{R_i}}(x_i)\s\rangle_{_A}
\ =\ H^{\alpha\beta}\ \Psi_\alpha(A^{01})
\otimes\overline{\Psi_\beta(-(A^{10})^*)}
\label{mainf}
\qqq
where \s$(\Psi_\alpha)\s$ is a basis of
\s$W(\Sigma,{\bf x},{\bf R},k)\s$, \s$(H^{\alpha\beta})\s$
is an \s${\bf x}$-dependent matrix and the summation convention
is assumed. From formal reality properties of the functional
integral defining \s$Z_{_A}\s\s
\langle\s\otimes g_{_{R_i}}(x_i)\s\rangle_{_A}\s$ one
may see that \s$(H^{\alpha\beta})\s$ should be a hermitian
matrix. In fact one may argue that the inverse matrix
\s$(H_{\s\beta\alpha})\s$ corresponds to the
scalar product on the space \s$W(\Sigma,{\bf x},{\bf R},k)\m$
of non-abelian theta functions:
\qq
H_{\beta\alpha}\ =\ (\Psi_\beta,\m\Psi_\alpha)
\label{tbS}
\qqq
where \s$(\m\s\cdot\s\m,\s\cdot\s\m)\s$ is formally given by
\qq
(\Psi,\m\Psi')\ =\
\int(\m\Psi(A^{01}),\s
\Psi'(A^{01})\m)_{_{V_{_{\NR}}}}
\s\s\ee^{-{k\over2\pi}
\s\Vert A\Vert^2_{_{L^2}}}\ DA
\label{scPro}
\qqq
with the integration over the unitary gauge fields
\s$A\s$ with \s$A^{10}=-(A^{01})^*$. \s The scalar
product (\ref{scPro}) is exactly the one which gives
the probability amplitudes between the states of
the Chern-Simons theory. Expressions (\ref{mainf}) and (\ref{tbS})
for the correlation functions may be expressed in a basis-independent
way as follows. Let \s$e_{{A^{01}}}\s$ denote the evaluation
map
\qq
W(\Sigma,{\bf x},{\bf R},k)\s\s\ni\ \Psi\ \s
\mathop{\longrightarrow}\limits^{e_{{A^{01}}}}\ \s\Psi(A^{01})
\ \in\s\s V_{_{\NR}}\s.
\non
\qqq
$e_{{A^{01}}}\s$ may be considered as an element of
\s$V_{_{\NR}}\otimes W(\Sigma,{\bf x},{\bf R},k)^*\s$
and using the scalar product dual to (\ref{scPro})
on the second factor, we obtain the equality
\qq
Z_{_A}\s\s\langle\s\otimes g_{_{R_i}}(x_i)\s\rangle_{_A}
\ =\ \overline{(e_{{A^{01}}},
\m\s e_{{-(A^{10})^*}})}
\label{mafA}
\qqq
viewed as a relation between the \s$V_{_{\NR}}\otimes\overline{
V_{_{\NR}}}$-valued
functionals of $A$.
\vskip 0.4cm

Let us present a functional integral proof
of the relation (\ref{scPro}).
Denote \s$Z_{_A}\s
\langle\m\otimes g_{_{R_i}}(x_i)\m\rangle_{_A}$
$\equiv\s\Gamma(A)\s$. \s Consider the integral
over the unitary gauge fields \s$B\s$
\qq
\int\Gamma(B^{10}+A^{01})
\ \Gamma(A^{10}+B^{01})
\ \ee^{-{k\over2\pi}
\s\Vert B\Vert^2_{_{L^2}}}\ DB\hs{7.3cm}\cr\cr
=\ \int\mathop{\otimes}_{i=1}^n(g_1g_2)_{_{R_i}}(x_i)\
\ee^{-k\m S(g_1)\s-\s k\m S(g_2)}\
\cdot\ \ee^{-\m{ik\over2\pi}\int\tr\s\s[\m A^{10}\wedge
g_2^{-1}\de g_2\s+\s g_1\da g_1^{-1}\wedge A^{01}]}\hs{2.8cm}\cr
\cdot\ \ee^{-\m{ik\over2\pi}\int\tr\s\s[\m
B^{10}\wedge g_1^{-1}\de g_1\s+\s g_1 B^{10} g_1^{-1}
\wedge A^{01}\s+\s g_2\da g_2^{-1}\wedge B^{01}\s+
\s g_1A^{10} g_2^{-1}\wedge B^{01}\s-\s B^{10}\wedge B^{01}]}
\ Dg_1\s\s Dg_2\s\s DB\cr\cr
=\ \int\mathop{\otimes}_{i=1}^n(g_1g_2)_{_{R_i}}(x_i)\
\ee^{-k\m S(g_1)\s-\s k\m S(g_2)\s-\s
{ik\over2\pi}\int\tr\s\s[\m A^{10}\wedge
g_2^{-1}\de g_2\s+\s g_1\da g_1^{-1}\wedge A^{01}]}\hs{3.4cm}\cr
\cdot\ \ee^{-{ik\over2\pi}\int\tr\s\s[\m
(g_2\da g_2^{-1}\m+\m g_2 A^{10} g_2^{-1})\wedge
(g_1^{-1}\de g_1\m+\m g_1^{-1} A^{01} g_1)]}
\ Dg_1\s\s Dg_2\hs{1cm}\cr\cr
=\ \int\mathop{\otimes}_{i=1}^n(g_1g_2)_{_{R_i}}(x_i)\
\ee^{-k\m S(g_1g_2,\m A)}\ Dg_1\s\s Dg_2\ =\
\Gamma(A)\hs{1.8cm}
\non
\qqq
where the $2^{\m\rm nd}$ equality is obtained
by a straightforward Gaussian integration over $B$.
Upon the substitution of relations (\ref{mainf})
and (\ref{scPro}), the last identity becomes
\qq
H^{\alpha\beta}\s\m H^{\gamma\delta}\ (\Psi_\beta,\m\Psi_\gamma)
\ \Psi_\alpha(A^{01})\otimes\overline{\Psi_\delta(-(A^{10})^*)}
\ =\ H^{\alpha\delta}\
\Psi_\alpha(A^{01})\otimes\overline{\Psi_\delta(-(A^{10})^*)}
\non
\qqq
or \s\s$H^{\alpha\beta}\s\m H^{\gamma\delta}\
(\Psi_\beta,\m\Psi_\gamma)\ =\ H^{\alpha\delta}\s\s$
from which the relation (\ref{tbS}) follows if we also assume that
\s$(H^{\alpha\beta})\s$ is an invertible matrix.
\vskip 0.4cm


The above expressions reduce the calculation
of the correlation functions of the WZW model to that
of the functional integral (\ref{scPro}).
The latter appears easier to calculate then the original
functional integral (\ref{cOc}). In the first step,
the integral (\ref{scPro}) may be rewritten
by a trick resembling the Faddeev-Popov treatment of
gauge theory functional integrals. The reparametrization
of the gauge fields
\qq
A^{01}\s=\s{}^{h^{-1}}\hs{-0.16cm}A^{01}(n)
\label{chva}
\qqq
by chiral gauge transforms of a (local) slice
\s$n\mapsto A^{01}(n)\s$ in \s$\CA^{01}\s$ cutting each
\s$\CG^\NC$-orbit once\footnote{in genus 0 and 1, \s$h\in\CG^\NC\s$
should be additionally restricted} gives
\qq
\Vert\Psi\Vert^2\ =\ \int(\m\Psi(A^{01}(n),
\otimes(hh^*)_{_{R_i}}^{-1}
\Psi(A^{01}(n)\m)_{_{V_{_{\NR}}}}\ \ee^{(k+2h^\vee)
\m S(hh^*,\m A(n))}\ D(hh^*)\ d\mu_{_Q}(n)\s.\hs{0.6cm}
\label{HWZW}
\qqq
The term \s$2h^\vee\m S(hh^*)\s$ in the action
comes from the Jacobian of the change of variables (\ref{chva})
contributing also to the measure \s$d\mu_{_Q}(n)\s$.
The latter is defined as follows. Denote by \s$S$
the composition of the derivative of the map
\s$n\mapsto A^{01}(n)\s$ with the canonical projection
of \s$\CA^{01}\s$ onto the cokernel of
\s$\de+[A^{01}(n),\s\cdot\s\s]\s$. \s Then the volume
form \s$\mu_{_Q}(n)\s$ on the slice is the composition
of the determinant (\m$\equiv\m$
the maximal exterior power) of \s$S\s$ with the Quillen
metric on the determinant bundle of the family
\s$(\de+[A^{01}(n),\s\cdot\s\s])\s$ of
\s$\de$-operators\footnote{again, the cases
of genus 0 or 1 require minor modifications}.
\vskip 0.3cm

Unlike in the standard Faddeev-Popov setup, the integral
over the group of gauge transformations did not
drop out since the integrand in (\ref{scPro}) is invariant
only under the \s$G$-valued gauge transformations. Instead
we are left with a functional integral (\ref{HWZW})
similar to the one (\ref{cOc}) for the original
correlation functions, except
that it is over fields \s$hh^*\s$ which may be considered
as taking values in the hyperbolic space \s$G^\NC/G\m$.
\s$D(hh^*)\s$ is the formal product of \s$G^\NC$-invariant
measures on \s$G^\NC/G\m$. \s The gain is that the
functional integral (\ref{HWZW})  may be reduced to
an explicitly doable iterative Gaussian integral.
For example for \s$G=SU(2)\s$ and at genus 0 where
we may take \s$A^{01}(n)\equiv 0\s$,
\qq
S(hh^*)\ =\ -\m{_i\over^{2\pi}}\int\da\phi\wedge\de\phi\s
-\s{_i\over^{2\pi}}\int(\da+\da\phi)\bar v\wedge(\de+\de\phi)v
\non
\qqq
in the Iwasawa parametrization
\s\s$
h\m=\m(\m\matrix{\ee^{\phi/2}&0\cr 0&\ee^{-\phi/2}}\m)
\s(\m\matrix{1&v\cr 0&1}\m)\s u\s\s$ of the 3-dimensional
hyperboloid
\s$SL_2(\NC)/SU_2\s$ by \s$\phi\in\NR\s$ and \s$v\in\NC\s$
(\s$u\in SU_2\m$). \s Field \s$v$ enters quadratically into
the action and polynomially into insertions.
Hence the \s$v$-integral is Gaussian and its explicit
calculation requires the knowledge of the determinant
of the operator \s$(-\da+\da\phi)(\de+\de\phi)=
-\ee^\phi\da\ee^{-2\phi}\de\ee^{\phi}\s$
and of the propagator
\qq
((-\da+\da\phi)(\de+\de\phi))^{-1}(z_1,z_2)\ \ \sim\
\ \ee^{-\phi(z_1)-\phi(z_2)}\int{\ee^{2\phi(y)}\s\m d^2y
\over (\bar z_1-\bar y)(y-z_2)}\s.
\label{scrch}
\qqq
The \s$\phi$-dependence of \s$\ln\det((-\da+\da\phi)
(\de+\de\phi))\s$ is given by the chiral anomaly
(or local index theorem) and is the sum of a local quadratic
and a linear term. The resulting \s$\phi$-field integral
appears to be also Gaussian (of the type
encountered in functional-integral representations
of a $2$-dimensional Coulomb gas correlation functions
in statistical mechanics). Similar iterative procedure
based on the Iwasawa parametrization of \s$G^\NC/G\s$
works for arbitrary \s$G\s$ and also at higher genera.
A result becomes a finite-dimensional integral over
parameters \s$y_a\in\Sigma$ in the expressions
of the type (\ref{scrch}) for the \s$v$-field propagators
(positions of the "screening charges" in the Coulomb
gas interpretation) and, at genus \s$h_{_\Sigma}>0\m$, \m over
(a part of) the moduli parameters \s$n\s$.
\vskip 0.4cm

At genus 0, the \s$\CG^\NC$-orbit of \s$A^{01}=0\s$
is dense in \s$\CA^{01}$. \m As a result \s$\Psi\s\in
\s W(\NC P^1,{\bf x},{\bf R},k)\s$ is fully determined
by \s$\Psi(0)\m\in\m (V_{_{\NR}})^G\m$, \s
the \m$G$-invariant subspace of \s$V_{_{\NR}}\m$.
Hence
\qq
W(\NC P^1,{\bf x},{\bf R},k)\ \subset\ (V_{_{\NR}})^G
\non
\qqq
canonically. For \s$G=SU_2\s$ the representations \s$R_i\s$
are labeled by integer or half-integer spins \s$j_i\s$
and the representation spaces \s$V_{j_i}\s$ are spanned
by vectors \s$(f^lv_{j_i})_{_{l=0,1,\dots,2j_i}}\s$ where
\s$v_{j_i}\s$ is the highest weight (HW) vector annihilated
by \s$e\s$, with \s$(e,f,h)\s$ the usual basis
of \s$sl_2\s$. One has, using the standard complex
variable \s$z\s$ on \s$\NC P^1\s$ to label the insertion
points,
\qq
W(\NC P^1,{\bf z},{\bf j},k)\ =\ \{\s v\in(V_{_{\Nj}})^{SU_2}
\ \vert\ (\m\otimes v_{j_i}\s,\s\m{\prod}_{_i} e_i^{n_i}\s
\ee^{\m z_ie_i}\s v\s)=0\s\ {\rm if}\s\ N\leq J-k-1
\s\m\}
\non
\qqq
where \s$e_i=1\otimes1\cdots\otimes1\otimes\mathop{e}\limits_{\hat i}
\otimes1\otimes\cdots\otimes1\s$, \s$N\equiv\sum_{_i}n_i\s$ and
\s$J\equiv\sum_{_i}j_i\m$. \s In particular, for 2 or 3 points,
\qq
W(\NC P^1,{\bf z},{\bf j},k)\ =\ \cases{\hbox to 2.5cm{$(
V_{_{\Nj}})^{SU_2}$\hfill}{\rm if}\ \ J\leq k\s,\cr
\hbox to 2.5cm{$\hspace{0.5cm}\{0\}$\hfill}{\rm if}\ \ J> k}
\non
\qqq
and does not depend on \s${\bf z}\m$. \s The scalar product
(\ref{scPro}) is given by
\qq
\Vert v\Vert^2\ =\ f(\sigma,{\bf z},{\bf j},k)
\int\limits_{\NC^J}\big\vert\s(\s v\m,\s\omega(\Nz,\Ny)\s)\s\s
\ee^{-{1\over k+2}\m U(\Nz,\Ny)}\m\big\vert^2\s\s\prod\limits_{a=1}^J
d^2y_a
\label{scP0}
\qqq
where
\qq
f(\sigma,{\bf z},{\bf j},k)\ =\ \ee^{\sum_{_i}{j_i(j_i+1)\over k+2}
\m\sigma(z_i)\s+\s{1\over 16\pi(k+2)}\m\Vert d\sigma\Vert_{L^2}^2}
\s\left({_{{\det}'(-\Delta)}\over^{{\rm area}_{\NC P^1}}}\right)^{3/2}
\non
\qqq
carries the dependence on the metric \s$\ee^{\sigma}\vert dz\vert^2\s$
on \s$\NC P^1\m$, \s$\Ny=(y_1,\dots,y_J)\s$,
\s$\omega(\Nz,\Ny)\s$ is a meromorphic \s$V_{_{\Nj}}$-valued
function
\qq
\omega(\Nz,\Ny)\ =\ \prod\limits_{a=1}^J\sum\limits_{i=1}^n
{_1\over^{y_a-z_i}}\s f_i\s\s
\mathop{\otimes}\limits_{i=1}^n v_{j_i}
\non
\qqq
and \s$U(\Nz,\Ny)\s$ is a multivalued function
\qq
{_1\over^2}\s U(\Nz,\Ny)\ =\ \sum\limits_{i<i'}
j_i j_{i'}\s\m\ln(z_i-z_{i'})
\s-\s\sum\limits_{i,a}j_i\s\m\ln(z_i-y_a)\s+\s\sum\limits_{a<a'}
\ln(y_a-y_{a'})\s.
\non
\qqq
Integral (\ref{scP0}) is over a positive density with
singularities at coinciding \s$y_a\s$ and the question
arises as to whether it does converge.
A natural conjecture is that the integral
is convergent if and only if \s$v\in
W(\NC P^1,{\bf x},{\bf R},k)\subset(V_{_{\NR}})^G\s$
(the only if part is easy). For 2- or 3-point functions
the integrals can indeed be computed explicitly confirming
the conjecture. Numerous other special cases have been
checked. However, the general case of the conjecture
remains to be verified. Note that the dependence of the
scalar product (\ref{scP0}) on the conformal factor
\s$\sigma\s$ agrees with the value \s${3k\over k+2}\s$
of the central charge of the \s$SU_2\s$ WZW theory and
with the values \s$\Delta_j=\bar\Delta_j={j(j+1)\over k+2}\s$
of the conformal dimensions of field \s$g_j(x)\s$
(it is the inverse of \s$f(\sigma,{\bf x},{\bf j},k)\s$
which enters the WZW correlation functions).
\vskip 0.4cm

Explicit finite-dimensional integral formulae for the scalar
product (\ref{scPro}) have been also obtained for general
groups and at genus 1 and, for \s$G=SU_2\m$, \m for higher
genera\footnote{it is clear that the case of general group
and genus $>$1 could be treated along the same lines}.
The proof of the convergence of the corresponding integrals
is the only missing element in the explicit construction
of all correlation functions of the WZW theory although several
special cases have been settled completely.
\eject
\vskip 1cm



\no{\bf{4. \ KZB connection}}
\vskip 0.6cm

The spaces \s$W(\Sigma,\Nx,\NR,k)\s$ of non-abelian
theta functions depend on the complex structures
of the surface \s$\Sigma\s$ and on the insertion points.
The complex structures $J\in\Gamma({\rm End}\m T\Sigma)$,
$J^2=-1$, form a complex (infinite dimensional Fr\'{e}chet)
manifold on which the group of Diffeomorphisms of
$\Sigma$ acts naturally. The holomorphic tangent vectors
to the quotient moduli space \s$\delta J=
\delta\mu\s$ correspond to sections of
\s${\rm End}\m T^\NC\Sigma\s$
satisfying \s$J\m\delta\mu=-\delta\mu\m J=i\m\delta\mu\m$.
\m Locally, \s$\delta\mu\s$ may be represented as
\s$\delta\mu^z_{\bar z}\s\da_z\otimes d\bar z\s$ in
\m$J$-complex coordinates. The family of spaces
\s$W(\Sigma,\Nx,\NR,k)\s$ forms a complex finite-dimensional
bundle \s$\CW(\NR,k)\s$ over the space of complex structures
and \m$n$-tuples of noncoincident points \s$\Nx\m$
in \s$\Sigma\s$.
\vskip 0.3cm

The bundle \s$\CW(\NR,k)\s$ may be supplied with a natural
(w.r.t.\s\s the action of diffeomorphisms of \s$\Sigma\m$)
\m connection \s$\nabla\m$ provided that we choose (smoothly)
for each \s$J\s$ a compatible metric on \s$\Sigma\m$.
\m The connections for different choices
of the metric are related by the conformal anomaly.
If \s$(J,\Nx,A)\mapsto\Psi(J,\Nx,A)\s$
depending holomorphically on \s$A^{01}=A(1+iJ)/2\s$
(\m$A=-A^*$ is assumed) represents a local section
of \s$\CW(\NR,k)\s$, then
\qq
&&\nabla_{_{\overline{\delta\mu}}}\s\Psi\ =\
d_{_{\overline{\delta\mu}}}\s\Psi\s+\s
{_k\over^{8\pi}}\left(\int\tr\s\s A^{01}\wedge A^{01}
{\overline{\delta\mu}}\right)\m\Psi\s,\label{KZ1}\\
&&\nabla_{_{\bar z_i}}\s\Psi\ =\ \de_{_{\bar z_i}\s}\Psi\s
+\s (A_{\bar z_i})_{_i}
\s\Psi\s,\label{KZ2}\\
&&\nabla_{_{\delta\mu}}\s\Psi\ =\ d_{_{\delta\mu}}\s\Psi
\s-\s{_1\over^{2\pi i}}\left(\int T(z)\s\delta\mu^z_{\bar z}
\s\m d^2z\right)\m\Psi\s,\label{KZ3}\\
&&\nabla_{_{z_i}}\s\Psi\ =\ \da_{_{z_i}}\s+\s\lim\limits_{z\to z_i}
\s\s{_{2\m t_i^a}\over^{k+h^\vee}}\left(J^a(z)\s+\s{_{t^a_i}
\over^{z-z_i}}\right)\m\Psi\s.
\label{KZ4}
\qqq
Above \s$z\s$ denotes a \m$J$-complex coordinate on \s$\Sigma\m$
and \s$d_{_{\delta\mu}}\m\Psi\s$ or \s$d_{_{\overline{\delta\mu}}}
\m\Psi\s$ stands for the directional
derivatives of \s$\Psi\s$ when the points \s$\Nx\s$ and \s$A\s$
are kept constant. The first two equations equip \s$\CW(\NR,k)\s$
with a structure of a holomorphic vector bundle. In the last
2 equations, the metric on \s$\Sigma\s$ is assumed
for simplicity to satisfy
\s$\gamma^{z\bar z}=2\s$, \s$\delta\gamma^{zz}={_2\over^i}
\m\delta\mu^z_{\bar z}\s$ and \s$A\s$ is taken vanishing
around the support of \s$\delta\mu\s$ or around the insertion
point \s$x_i\m$ and $\delta\mu=\CO((z-z_i)^2)$.
\vskip 0.3cm

In the genus 0 case, \s$\CW(\NR,k)\s$ is a subbundle
of the trivial bundle with the fiber \s$(V_{_{\NR}})^G\s$
and the connection \s$\nabla\s$ extends to the bigger bundle
and is given by
\qq
\nabla_{_{\bar z_i}}\ =\ \da_{_{\bar z_i}}\s,\quad\quad
\nabla_{_{z_i}}\ =\
\da_{_{z_i}}\s+\s{_2\over^{k+h^\vee}}\sum\limits_{i'\not= i}
{_{t_i^a\m t^a_{i'}}\over^{z_{i'}-z_{i}}}\ \ \s\equiv\ \
\da_{_{z_i}}\s+\s{_1\over^{k+h^\vee}}\s H_i(\Nz)
\non
\qqq
for the metric flat around the insertions. The commuting
operators \s$H_i(\Nz)\in{\rm End}(V_{_{\NR}})\s$ are
known as the Gaudin Hamiltonians. The corresponding
flat connection appeared (implicitly) in the work of
Knizhnik-Zamolodchikov on the WZW theory.
The higher genus generalizations
of the KZ connection were first studied by Bernard.
We shall call the connection
defined by eqs.\s\s(\ref{KZ1}-\ref{KZ4}) the KZB connection.
In general, it is only projectively flat.
\vskip 0.3cm


One of the basic open questions concerning the KZB connection
is whether it is unitarizable. In other words, whether there
exists a hermitian structure on the bundle \s$\CW(\NR,k)\s$
preserved by \s$\nabla\m$. It was conjectured
that the answer to this question is positive and that it is
exactly the scalar product on spaces \s$W(\Sigma,\Nx,\NR,k)\s$
discussed above that provides the required hermitian structure.
Note that a 0,1 unitary connection on a holomorphic
hermitian vector bundle is uniquely determined.
Recall that the scalar product, given formally by the
gauge field functional integral (\ref{scPro}), may be
reduced to a finite-dimensional integral which, if
convergent, defines a positive hermitian form on
\s$W(\Sigma,\Nx,\NR,k)\m$ and determines the unitary connection
(and the energy momenstum tensor of the WZW theory).
For genus 0 and \s$G=SU_2\m$,
\m where the scalar product is given by integral (\ref{scP0}),
the unitarity of the KZ connection requires that
\qq
\da_{z_i}(\m v\m,\s v\m)\ = \ (\m v\m,\s(\da_{z_i}-{_1\over^{k+2}}
\m H_i)\m v\m)
\label{thelA}
\qqq
for a holomorphic family \s${\bf z}\mapsto v({\bf z})\in
W(\NC P^1,{\bf z},{\bf j},k)\subset V_{\Nj}^{SU_2}\m$.
\m Assuming the convergence of the integrals permitting
to differentiate under the integral and to integrate by parts,
the above is a consequence of the relation
\qq
\left(\da_{z_i}+\m{_1\over^{k+2}}\m H_i(\Nz)\right)
\left(\omega(\Nz,\Ny)\s\s\ee^{-{1\over{k+2}}\m U(\Nz,\Ny)}
\right)\ =\ \da_{y_a}\left(\eta_{i,a}\s\m\ee^{-{1\over{k+2}}
\m U(\Nz,\Ny)}\right)
\label{alA}
\qqq
where
\qq
\eta_{i,a}(\Nz,\Ny)\ =\ {_1\over^{z_i-y_a}}\s
f_i\prod\limits_{a'\not=a}\sum\limits_{i'=1}^n
{_1\over^{y_{a'}-z_{i'}}}\s f_{i'}\s\s
\mathop{\otimes}\limits_{i=1}^n v_{j_i}\s.
\non
\qqq
Identity (\ref{alA}) is equivalent to two relations:
\qq
&&\da_{z_i}\m\omega(\Nz,\Ny)\s
=\s\da_{y_a}\m\eta_{i,a}(\Nz,\Ny)\s,\cr\cr
&&\da_{z_i}U(\Nz,\Ny)\s\s\omega(\Nz,\Ny)\s
-\s\da_{y_a}U(\Nz,\Ny)\s\s\eta_{i,a}(\Nz,\Ny)\s
-\s H_i(\Nz)\s\s\omega(\Nz,\Ny)\s=\s0\s.
\label{bapr}
\qqq
The first one is immediate whereas the second, more involved
one implies that
\qq
H_i(\Nz)\s\s\omega(\Nz,\Ny)\
=\ \da_{z_i}U(\Nx,\Ny)\ \omega(\Nz,\Ny)
\quad\quad{\rm if}\quad\ \da_{y_a}U(\Nx,\Ny)\s=\s0
\non
\qqq
i.e.\s\s the {\bf Bethe Ansatz} diagonalization of the Gaudin
Hamiltonians \s$H_i(\Nz)\s$: \s vectors \s$\omega(\Nz,\Ny)\s$
are common eigenvectors of \s$H_i(\Nz)\s,\ i=1,\dots,n\s$
with eigenvalues \s$\da_{z_i}U(\Nz,\Ny)\s$ provided
that \s$\Ny\s$ satisfies the Bethe Ansatz equations
\s$\da_{y_a}U(\Nx,\Ny)\s=\s0\m$. \s The relations
between the Bethe Ansatz and the limit of the KZB connection
when \s$k\to-h^\vee\s$ appear in the context of Langlands
geometric correspondence. These relations seem also to be at
the heart of the question about the unitarity of the KZB
connection at positive integer \s$k\m$.
\vskip 1cm



\no{\bf{5. \ Coset theories}}
\vskip 0.6cm

There is a rich family of CFT's
which may be obtained from the WZW models by a simple
procedure known under the name of a {\bf coset construction}.
On the functional integral level,
the procedure consists of coupling the \s$G$-group WZW
theory to a subgroup \s$H\subset G\s$ unitary gauge
field \s$B\s$ which is also integrated over with
gauge-invariant insertions.
Let us assume, for simplicity, that \s$H\s$ is
connected and simply connected, as \s$G\m$. \m Let
\s$t_i\in({\rm Hom}(V_{_{R_i}},V_{_{r_i}}))^H\s$ be
intertwiners of the action of \s$H\s$ in the irreducible
\s$G$- and \s$H$-representation spaces, respectively.
The simplest correlation functions of the \s$G/H\s$ coset
theory take the form
\qq
\langle\s\s\prod\limits_{i=1}^n \tr\s\m t_i g_{_{R_i}}
(x_i)\m t_i^*\s\rangle\s
%\hs{8cm}\cr
=\s\int\s
\prod\limits_{i=1}^n
\tr_{_{V_{_{r_i}}}}\hs{-0.1cm}t_i\m g_{_{R_i}}
(x_i)\s t_i^*\ \m\ee^{-k\m S(g,B)}\s\s Dg\s\m DB\s\bigg/
\int\ee^{-k\m S(g,B)}\s\s Dg\s\m DB\s.\hs{0.3cm}
\label{cfGH}
\qqq
Note that the \s$g$-field integrals are the ones of the
WZW theory and are given by eq.\s\s(\ref{mainf}).
Denoting \s\s$Z_{_{G/H}}\m
=\m\int\ee^{-k\m S(g,B)}\s\s Dg\s\m DB\m$,
we obtain
\qq
Z_{_{G/H}}\s\s\langle\s\s\prod\limits_{i=1}^n
\tr\s\m t_i g_{_{R_i}}(x_i)\m t_i^*\s\rangle\s\
=\ H^{\alpha\beta}\int(\m\otimes t_i
\Psi_\beta(B^{01})\m,\s\otimes t_i\Psi_\alpha
(B^{01})\s)_{_{V_{_\Nr}}}\ \ee^{-{k\over 2\pi}\m\Vert B
\Vert_{L^2}^2}\s\s DB\s.\hs{0.4cm}
\label{aneq}
\qqq
For \s$\Psi\in W(\Sigma,\Nx,\NR,k)\m$, \m
the map \s$B^{01}\mapsto\otimes t_i\Psi_\alpha(B^{01})
\in V_{_{\Nr}}\s$ is a group \s$H\s$ non-abelian theta
function belonging to \s$W(\Sigma,\Nx,\Nr,\tilde k)\s$
(the normalization of the Killing forms of \s$G\s$ and
\s$H\s$ may differ, hence the replacement \s$k\rightarrow
\tilde k\s$). Denote by \s$T\s$ the corresponding map
from \s$W(\Sigma,\Nx,\NR,k)\s$ to
\s$W(\Sigma,\Nx,\Nr,\tilde k)\m$.
\m Eq.\s\s(\ref{aneq}) may be rewritten as
\qq
Z_{_{G/H}}\s\s\langle\s\s\prod\limits_{i=1}^n
\tr\s\m t_i g_{_{R_i}}(x_i)\m t_i^*\s\rangle\s\
=\ H^{\alpha\beta}\ (\m T\Psi_\beta\m,\s T\Psi_\alpha\m)
\ =\ {\rm Tr}\s\s T^*T\s,
\label{GHcf}
\qqq
or choosing a basis \s$(\psi_\lambda)\s$ of
\s$W(\Sigma,\Nx,\Nr,\tilde k)\m$, \m
\qq
Z_{_{G/H}}\s\s\langle\s\s\prod\limits_{i=1}^n
\tr\s\m t_i g_{_{R_i}}(x_i)\m t_i^*\s\rangle\s\
=\ H^{\alpha\beta}\s\s\overline{T^\lambda_\beta}\s\s
h_{\lambda\nu}\s\s T^\nu_\alpha
\label{GHce}
\qqq
where \s$(T^\lambda_\alpha)\s$ is the ("branching") matrix
of the linear map \s$T\s$ in bases \s$(\Psi_\alpha)\m$,
\s$(\psi_\lambda)\s$ and \s$h_{\lambda,\nu}=(\psi_\lambda,
\m\psi_\nu)\s$.
Since the above formula holds also for the partition function
itself, it follows that the calculation of the coset theory
correlation functions (\ref{cfGH}) reduces to that of the scalar
products of group \s$G\s$ and group \s$H\s$ non-abelian theta
functions, both given by explicit, finite-dimensional
integrals.
\vskip 0.3cm

Among the simplest examples of the coset theories
is the case with \s$G=SU_2\times SU_2\s$ with level
\s$(k,1)\s$ (for product groups, the levels may
be taken independently for each group) and with \s$H\s$
being the diagonal \s$SU_2\s$ subgroup. The resulting
theories coincide with the unitary "minimal" series of
CFT's with (Virasoro) central charges
\s$c=1-{6\over{(k+2)(k+3)}}\s$ first considered by
Belavin-Polyakov-Zamolodchikov.
The Hilbert spaces of these theories are built
from the unitary heighest weight representations
of the Virasoro algebras with \s$0<c<1\s$ discussed
in Lecture 2. The simpliest one of them with
\s$k=1\s$ and \s$c={1\over2}\s$
is believed to describe the continuum limit of the
Ising model at critical temperature or the scaling limit
of the massless \s$\phi^4_2\s$ theory. In particular,
in the continuum limit the spins in
the critical Ising model are represented
by fields \s$\tr\s\s g_{_{1/2}}(x)\s$ where \s$g\s$ takes
values in the first \s$SU_2\m$. \s The corresponding correlation
functions may be computed as above. One obtains this way
for the 4-point function an explicit expression in terms
of hypergeometric functions.
\vskip 0.3cm

Similar coset theories but at level \s$(k,2)\s$ give rise
to the supersymmetric $N=1$ minimal unitary series of
CFT's, the simplest one with \s$k=1\s$
(appearing also at \s$k=2\s$ in the previous series)
corresponds to the so called 3-critical Ising model.
\vskip 0.3cm

The \s$G/H\s$ coset theory with \s$H=G\s$ is a prototype
of a two-dimensional topological field theory. As follows from
eq.\s\s(\ref{GHcf}), the correlation functions of fields
\s$\tr\s\s g_{_R}(x)\s$ are equal to the dimension of
spaces \s$W(\Sigma,\Nx,\NR,k)\s$, normalized by the dimension
of \s$W(\Sigma,\emptyset,\emptyset,k)\s$ (and are given by the Verlinde
formula). In particular, they do not depend on the position
of the insertion points.
%\eject
\vskip 1cm



\no{\bf{6. \ WZW factory}}
\vskip 0.6cm

As we have seen above, the coset construction
allows to obtain new soluble CFT's
from the WZW models. Let us briefly discuss further refinements
which permit a chain production of conformal models
whose partition functions and correlation
functions may be computed exactly, at least in principle.
The most interesting cases of such models correspond
to situations when two different constructions give rise to the same
CFT, as in \s$T$-duality, mirror
symmetry and other numerous instances.
\vskip 0.4cm

\no{\bf 1.} \ If the group \s$G\s$ is {\bf not simply connected},
the original definition (\ref{actWZW}) of the action
of the WZW model requires a modification. The result
is possible further restrictions on the levels and
the appearance, in some cases, of different quantizations
of the same classical theory ("$\theta$-vacua" or "discrete
torsion"). The models are still exactly soluble
although only the partition functions and the correlations
of "untwisted" fields have been worked out in detail
for general \s$G\m$.
\vskip 0.4cm

\no{\bf 2.} \ Let \s$H\subset G\s$ and \s$Z\subset H\s$
be a subgroup of the center \s$Z_{_G}\s$ of \s$G\m$.
\m Let \s$P_{_{H'}}\s$ be a principal \s$H'$-bundle
where \s$H'=H/Z\m$ and \s$Q_{_G}=P_{_{H'}}\hs{-0.06cm}
\times_{_{{\rm Ad}_{_{H'}}}}\hs{-0.2cm}G\s$ be the \s$G$-bundle
associated to \s$P_{_{H'}}\s$
via the adjoint action of \s$H'\s$ on \s$G\m$. \m For
appropriate \s$k\m$, \m and for
a section \s$g\s$ of \s$Q_{_G}\s$ and
a connection \s$B\s$ on \s$P_{_{H'}}\s$ one may define
the amplitude \s$\ee^{-k\m S(g,B)}\m$. \m The (unnormalized)
correlation functions of the coset \s$G/H'$-model
may then be obtained by integrating
gauge invariant insertions, weighted with
\s$\ee^{-k\m S(g,B)}\m$, \m over \s$g\s$ and \s$B\s$
and summing the result over inequivalent \s$H'$-bundles
\s$P_{_{H'}}\m$. \m Hence, for given \s$H\subset
G\s$, there are as many coset theories as subgroups
of \s$H\cap Z_{_G}\s$
(some of them might have a non-unique vacuum).
\vskip 0.4cm

\no{\bf 3.} \ If \s$H\s$ is a discrete subgroup of \s$G\m$,
then \s$P_{_H}\s$ carries a unique canonical flat connection
and is given by a conjugation class of homomorphisms
of the fundamental group of is \s$\Sigma\s$ into \s$H\m$.
\m The construction from the preceding point gives rise to the
{\bf orbifolds} of the WZW models.
\vskip 0.4cm

\no{\bf 4.} \ {\bf Supersymmetric WZW models}. \ One adds to the
\s$G$-valued field \s$g\s$ the (anticommuting) Majorana
Fermi fields \s$\psi,\s\tilde\psi\m$ in the adjoint representation
(i.e. sections of \s$L\otimes\Ng\s$ and \s$\bar L\otimes\Ng\m$,
\m respectively, where \s$L\s$ is a square root of the canonical
bundle of \s$\Sigma\m$) and one considers the action
\qq
S(g,\psi,\tilde\psi,A)\ =\ k\m S(g,A)\s-\s{_2\over^{\pi}}
\int\tr\s\left(\psi\m(\de_L+[A^{01},\s\cdot\s\m]\m)\m\psi\s+\s
\tilde\psi\m(\da_{\bar L}
+[A^{10},\s\cdot\s\m]\m)\m\tilde\psi\right)
\label{SUWZW}
\qqq
with the external, group \s$G\s$ gauge field \s$A\m$. \m
The fermionic part of the theory is free and the complete
theory may be easily solved.
\vskip 0.4cm

\no{\bf 5.} \ {\bf Supersymmetric coset models}. \ The action
is as in eq.\s\s(\ref{SUWZW}) except that \s$A\s$ is replaced by
a group \s$H\s$ gauge field \s$B\s$ and the Majorana fields
\s$\psi,\m\tilde\psi\s$ are taken with values in \s$\Ng/\Nh\s$
rather than in \s$\Ng\m$.
Both the supersymmetric WZW models and the supersymmetric
coset models possess the \s$N=1\s$ superconformal symmetry.
\vskip 0.4cm

\no{\bf 6.} \ $N=2\s$ {\bf coset models}. \ If
\s$G/H\s$ is a K\"{a}hler symmetric space then the supersymmetric
\s$G/H\s$ coset model possesses the \s$N=2\s$ superconformal
symmetry including the \m$U(1)\m$ loop group symmetry.
The simplest examples are provided by the
\s$SU(2)/U(1)\s$ models which, at level \m$k$, \s give
rise to the minimal \s$N=2\s$ superconformal theory with
central charge \s$c={{3k}\over{k+2}}\m$.
\vskip 0.4cm

\no{\bf 7.} \ {\bf Orbifods of tensor products} of conformal
field theories may give rise to essentially new models.
The famous example are the \s$(\NZ/5\NZ)^3\s$ orbifolds
of product of five \s$k=3\s$ minimal \s$N=2\s$ superconformal
models. Two different orbifolds may give equivalent conformal
sigma models corresponding to a mirror pair of Calabi-Yau
quintic targets.
\vskip 0.4cm

\no{\bf 8.} \ The theories with \m$U(1)\m$ loop group symmetries
like the \s$N=2\s$ supersymmetric coset models may be
{\bf twisted} by considering their fields as taking values
in bundles associated with the sphere subbundle of the spin
bundle \s$L\s$ and coupled to the spin connection. Such
twisting of \s$N=2\s$ superconformal models may be done in two
essentially different ways ($A$- and $B$-twist) and it produces
topological field theories. The genus 0 correlations
of the \s$A$-twisted \s$N=2\s$ sigma models compute
the quantum cohomology of the target.
\vskip 0.4cm

\no{\bf 9.} \ For {\bf non-compact groups} \s$G\m$, \m the WZW
action \s$S(g)\s$ is not bounded below but one may try
to stabilize the Euclidean functional integral by
analytic continuation or/and coset-type gauging of subgroups
of \s$G\m$. \m Such stabilization procedures may however destroy
the physical positivity (Hilbert-space picture) of the theory.
The best studied models of the non-compact type correspond
to finite coverings of \s$SL_2(\NR)\s$ with the \s$U(1)\s$
subgroup twisted and the nilpotent subgroup gauged away
(the construction of minimal models {\it \`{a} la} Drinfeld-Sokolov),
the \s$SL_N(\NR)\s$ generalizations
thereof, the Liouville and Toda theories
and the \s$SL_2(\NR)/U(1)\s$ black hole model.
Our knowledge of non-compact WZW models
is certainly much less complete than that of the compact
case (note that this is true also on the level of quantum
mechanics where we know everything about harmonic analysis
of compact Lie groups but the harmonic analysis of non-compact
ones has still open problems).
\vskip 0.8cm


\no{\bf References}
\vskip 0.4cm

The basic papers on the Wess-Zumino-Witten model are:
Witten: Commun. Math. Phys 92 (1984), p. 455,
\m Polyakov-Wiegmann: Phys. Lett. B 141, (1984), p. 223,
\m Knizhnik-Zamolodchikov: Nucl. Phys. B 247 (1984), p. 83,
\m Gepner-Witten: Nucl. Phys. B 278 (1986), p. 493,
\m E. Verlinde: Nucl. Phys. B 300 (1988), p. 360,
\m Tsuchiya-Kanie: Adv. Stud. Pure Math. 16 (1988),
p. 297, \s Tsuchiya-Ueno-Yamada: Adv. Stud. Pure Math. 19
(1989), p. 459.
\vskip 0.3cm

The present lecture follows a geometric approach developed
in author's paper in Nucl. Phys. B 328 (1989),
p. 733 where the relations (\ref{mainf}) between the scalar
product of non-abelian theta functions and WZW
correlations as well as the conjecture about
the unitarity of the KZ connection were first
formulated. See also ``Functional Integration, Geometry
and Strings'', Haba, Sobczyk (eds.),
Birkh\"{a}user 1989, p. 277. The scalar product
formulae for arbitrary Lie group were
obtained at genus 0 in Falceto-Gaw\c{e}dzki-Kupiainen:
Phys. Lett. B 260 (1991), p. 101 \s and at genus 1
in Falceto-Gaw\c{e}dzki: hep-th/9604094.
The case $G=SU_2$, genus $>$1 was studied in Gaw\c{e}dzki:
Commun. Math. Phys. 169 (1995), p. 329 \s and Lett. Math. Phys.
33 (1995), p. 335.
\s The functional integral proof of eq.\s\s(\ref{mainf})
was borrowed from Witten: Commun. Math. Phys. 144 (1992),
p. 189.
\vskip 0.3cm

The KZ connection at higher genera appeared in
Bernard: Nucl. Phys. B 303 (1988), p. 77 and B 309 (1988) p. 145
and was further discussed in Hitchin: Commun. Math. Phys.
131 (1990), p. 347, \s in Axelrod-Della Pietra-Witten:
J. Diff. Geom. 33 (1991) p. 787 and by other authors.
For the relations between the KZ connection,
Bethe Ansatz and geometric Langlands correspondence
see Feigin-E.\m Frenkel-Reshetikhin: Commun. Math. Phys. 166 (1994),
p. 27 and E. Frenkel's contribution to ``XI$^{\rm\m th}$
International Congress of Mathematical Physics'', Iagolnitzer (ed.),
International Press 1995.
\vskip 0.3cm

The original papers on the coset models are:
Goddard-Kent-Olive in Phys. Lett. B 152 (1985), p. 88
(short version) and Commun. Math. Phys. 103 (1896), p. 105.
The functional integral approach discussed above follows
Gaw\c{e}dzki-Kupiainen: Nucl. Phys. B 320 (1989), p. 625,
see also Bardakci-Rabinovici-S\"{a}ring, Nucl. Phys. B 299 (1988),
p. 157 \s and Karabali-Schnitzer: Nucl. Phys. B 329 (1990), p. 649
\vskip 0.3cm

General WZW models with non-simply connected groups
were discussed in Felder-Gaw\c{e}dzki-Kupiainen: Commun. Math. Phys.
117 (1988), p. 127, \m see also Gaw\c{e}dzki: Nucl. Phys. B (Proc. Suppl.)
18B (1990), p. 78.
\vskip 0.3cm

The extension of the coset construction mentioned in Sect.\s\s4.2
is due to Hori: hep-th/9411134, see also
Fuchs-Schellekens-Schweigert
hep-th/9612093 for the representation theory counterpart.
\vskip 0.3cm

For orbifolds of WZW models see Kac-Todorov: hep-th/9612078.
The relation of tensor products of $N=2$
CFT's to Calabi-Yau sigma models was
discovered in Gepner: Phys. Lett. B 199 (1987), p. 380.
The mirror pair of quintics appeared in this context in
Greene-Plesser: Nucl. Phys. B 338 (1990), p. 15.
\vskip 0.3cm

The $N=1$ supersymmetric coset models are discussed in
Schnitzer: Nucl. Phys. B 324 (1989), p. 412, \m the
$N=2$ ones in Kazama-Suzuki: Nucl. Phys. B321 (1989), p. 232.
\vskip 0.3cm

Twisting of $N=2$ superconformal field theories is discussed
in Witten's contribution to ``Essays on Mirror Symmetry'',
Yau (ed.), International Press 1992.
\vskip 0.3cm

The  $SL_n(\NR)$ WZW models are considered e.g.
in Bershadsky-Ooguri, Commun. Math. Phys. 126 (1989) p. 49.
See Frenkel-Kac-Wakimoto: Commun. Math. Phys. 147 (1992), p. 295
for the representation theory aspects. For Toda theories,
see e.g.\s\s Bilal-Gervais: Nucl. Phys. B 318
(1989), p. 579. The $SL_2(\NR)/U(1)$ black hole
is the subject of Witten: Phys. Rev. D 44 (1991), p. 314.


\end{document}
