\input amstex
\documentstyle{amsppt}
\magnification=1200
\loadbold
\loadeusm

%\font\dotless=cmr10 %for the roman i or j to be
                    %used with accents on top.
                    %(\dotless\char'020=i)
                    %(\dotless\char'021=j)
%\font\itdotless=cmti10
%\def\itumi{{\"{\itdotless\char'020}}}
%\def\itumj{{\"{\itdotless\char'021}}}
%\def\umi{{\"{\dotless\char'020}}}
%\def\umj{{\"{\dotless\char'021}}}
%\font\thinlinefont=cmr5
%\font\smaller=cmr5
\font\boldtitlefont=cmb10 scaled\magstep2

\NoRunningHeads
\pagewidth{6.5 true in}
\pageheight{8.9 true in}

\catcode`\@=11
\def\logo@{}
\catcode`\@=13

\def\eps{{\varepsilon}}

\def\undertext#1{$\underline{\vphantom{y}\hbox{#1}}$}
\def\nspace{\lineskip=1pt\baselineskip=12pt%
     \lineskiplimit=0pt}
\def\dspace{\lineskip=2pt\baselineskip=18pt%
     \lineskiplimit=0pt}

\def\upvee{{\sssize \vee}}
\def\w{{\mathchoice{\,{\scriptstyle\wedge}\,}
  {{\scriptstyle\wedge}}
  {{\scriptscriptstyle\wedge}}{{\scriptscriptstyle\wedge}}}}
\def\Le{{\mathchoice{\,{\scriptstyle\le}\,}
{\,{\scriptstyle\le}\,}
{\,{\scriptscriptstyle\le}\,}{\,{\scriptscriptstyle\le}\,}}}
\def\Ge{{\mathchoice{\,{\scriptstyle\ge}\,}
{\,{\scriptstyle\ge}\,}
{\,{\scriptscriptstyle\ge}\,}{\,{\scriptscriptstyle\ge}\,}}}
\def\mapright#1{\smash{\mathop{\,\longrightarrow\,}%
     \limits^{#1}}}
\def\rmapdown#1{\Big\downarrow\kern-1.0pt\vcenter{
     \hbox{$\scriptstyle#1$}}}
\def\bfs{\bold{s}}
\def\arrowsim{\smash{\mathop{\,\longrightarrow\,}%
   \limits^{\lower1.5pt\hbox{$\scriptstyle\sim$}}}}
\def\leftarrowsim{\smash{\mathop{\,\longleftarrow\,}%
   \limits^{\lower1.5pt\hbox{$\scriptstyle\sim$}}}}

\def\Sym{\text{\rm Sym}} 
\def\Hom{\text{\rm Hom}} 
\def\Tor{\text{\rm Tor}} \def\Gr{\text{\rm gr}}
\def\Ext{\text{\rm Ext}} \def\ev{\text{\rm ev}}
\def\I{\text{\bf I}} \def\Ber{\text{\rm Ber}}
\def\II{\text{\bf II}} \def\ber{\text{\rm ber}}


%Script letters:
\def\scr#1{{\fam\eusmfam\relax#1}}
\def\scrT{{\scr T}} 


\topmatter
\title\nofrills
{\boldtitlefont Appendix: signs}
\endtitle
\author
P. Deligne
\endauthor
\endtopmatter

\NoBlackBoxes
\parindent=20pt
\frenchspacing
\document
\bigskip
\dspace
Some classical constructions give objects which are
graded, and which can profitably be viewed as ``super'',
i.e. subject to Koszul's sign rule.
Examples: cohomology groups of all kind ($H^*$,
$\Tor_*$, $\Ext^*\ldots$), the de Rham complex and
other standard complexes,$\,\ldots\,\,\,$.

To handle their analogue in the super world, two points
of view have been used.
As we will see, they are basically equivalent.
However, they lead to different sign conventions.

\subhead
I
\endsubhead
One considers them as graded $\mod 2$ graded (i.e.
super) objects.
The grading which is the analogue of the grading in the
classical case will be called the cohomological grading.
The commutativity isomorphism for the tensor product of
graded $\mod 2$ graded vector spaces is defined as
follows:
for $v$ of bidegree $(p,n)$: parity $p$ and
cohomological degree $n$, and $w$ of bidegree $(q,m)$,
$$
\sigma_{\I}\colon\, V\otimes W\to W\otimes V
\quad\text{is}\quad
v\otimes w\longmapsto (-1)^{pq+nm}w\otimes v.
$$

\subhead
II
\endsubhead
One considers that the classical construction already
lives in the super world, with parity $=$ cohomological
degree $\mod 2$.
In the super world, one continues to obtain super
objects; they also have a cohomological degree, with no
influence on signs: the commutativity isomorphism of
graded $\mod 2$ graded vector spaces is defined as
$$
\sigma_{\II}\colon\, V\otimes W\to W\otimes V\colon\,
v\otimes w\to (-1)^{pq}w\otimes v
$$
for $v$ of parity $p$ and $w$ of parity $q$.
Of course, one could consider modules instead of vector
spaces.

Here are advantages of the point of view {\bf I}.

\medskip\noindent
{\bf (A)}\enspace
For any tensor category $\scrT$, say additive with
``associative and commutative'' tensor product, one can
define a new tensor category $\scrT^{\Gr}$ of graded
objects of $\scrT$, with the commutativity of $\otimes$
being given by:
$$
\sigma^{\Gr}\colon\,V\otimes W\to W\otimes V
$$
induces on $V^n\otimes W^m$ the map $V^n\otimes
W^m\to W^m\otimes V^n$ of $\scrT$, multiplied by
$(-1)^{nm}$.
For $\scrT$ the tensor category of super vector spaces,
$\scrT^{\Gr}$ is the category $\I$ of graded $\mod 2$
graded vector spaces.
The point of view $\I$ hence allows for the following
way of reasoning: to prove a statement in superworld,
prove it first in any tensor category (keeping in mind
the case of vector spaces).
At the end, specialize to the category of super vector
spaces.
In the general categorical story, no sign of super
origin are there to confuse us: they are hidden in the
compatibilities obeyed by the associativity and
commutativity isomorphisms for $\otimes$.

\example{Example {\rm (of this kind of reasoning)}}
Suppose we want to see the following for finite
dimensional super vector spaces.
Known: if $W$ is the dual of $V$, then $V$ is the dual
of $W$.
As $V$ is the dual of $W$, repeating ``known'' for $W$
and $V$, we get that: $W$ is the dual of $V$.
Wished: the duality map is the same as the one we
started with.
\endexample

The dual of $V$ is well defined up to unique isomorphism
as an object $W$ provided with a morphism $\ev\colon\,
W\otimes V\to 1$ with suitable properties, for $1$ the
unit object.
To give a map $W\otimes V\to 1$ is the same thing as
giving a map $V\otimes W\to 1$, by the isomorphism
$\sigma$ of $W\otimes V$ with $V\otimes W$.

For super vector spaces, the structural map
$\ev\colon\,W\otimes V\to k$ identifies $W$ with the space
of linear forms on $V$, the evaluation map $\ev$ becoming
$w\otimes v\mapsto w(v)$.
If one prefers to say that it is the map $V\otimes W\to
k$ which makes of $W$ the dual of $V$, one is rather led
to identify the dual $W$ of $V$ with the space of linear
forms on $V$ by $w\mapsto\text{ linear form }(v\mapsto
\text{ image of $v\otimes w$)}$.
This gives two ways to identify the dual (in the
categorical sense) with the space of linear forms on
$V$, those two ways differ by a sign on the odd part,
and one easily gets confused.
In the general categorical setting, no sign occurs.

\medskip\noindent
{\bf (B)}\enspace
One has not to decide early on whether an object should
be seen as having a cohomological degree.

\example{Example}
Let $A$ be a commutative super $k$-algebra.
The point of view $\I$ suggest to define the module 
$\Omega_A^1$ of K\"ahler differential as being an
$A$-module (hence bimodule) $\Omega$, provided with a
morphism of super $k$ vector spaces
$d\colon\,A\to\Omega$ such that
$$
d(ab)=a.db+ da.b,\tag{B.1}
$$
which is universal.
Later, when considering the de Rham complex, one may
decide that $\Omega_A^1$ is of cohomological degree $1$.
In the point of view $\II$, deciding that in the
classical case (purely even $A$) $\Omega_A^1$ is odd,
requires $d$ to be an odd map, and that (B.1) be
replaced by
$$
d(ab)=(-1)^{p(a)}a.db+da.b\tag{B.2}
$$
\endexample

\medskip\noindent
{\bf (C)}\enspace
The point of view $\I$ minimizes the use of the parity
change functor $\Pi$.
Using this functor leads to nightmares of signs, for the
following reasons.

\medskip\noindent
(i)\enspace
Let $1^-$ be $k$, viewed as an odd $k$-vector space.
The functor $\Pi$ is best viewed as being the tensor
product with $1^-$.
One has to decide whether it is $V\mapsto 1^-\otimes V$
or $V\mapsto V\otimes 1^-$.
The two are canonically isomorphic, but lead to
different sign conventions.

\smallskip\noindent
(ii)\enspace
One has natural isomorphisms $(\Pi V)\otimes W\arrowsim
(V\otimes W)$ and $(V\otimes \Pi W)\arrowsim
\Pi(V\otimes W)$, exchanged by the commutativity of
$\otimes$.
The diagram
$$
\CD
\Pi V\otimes\Pi W @>>> \Pi(V\otimes\Pi W)\\
@VVV @VVV\\
\Pi(\Pi V\otimes W) @>>> \Pi\Pi(V\otimes W)
\endCD
$$
is anticommutative, rather than commutative.

\bigskip
Here are advantages of the point of view $\II$.

\medskip\noindent
{\bf (A)}\enspace
One has only one parity to consider, for applying the
sign rule.

\medskip\noindent
{\bf (B)}\enspace
Some constructions are more natural.
For example, if $D^-$ is the standard odd line
(coordinate ring $k[\theta]$, $\theta$ odd, $\theta^2=0$), the
de Rham complex of a super manifold $M$ is (up to a
completion) the space of functions on the super manifold
$$
\underline{\Hom}(D^-,M).
$$
In the point of view $\I$, one has to apply to the de
Rham complex the functor ``associated simply graded
object'' explained below.

I prefer the point of view $\I$.
Bernstein prefers the point of view $\II$.

The points of view $\I$ and $\II$ are equivalent, in the
sense that the tensor categories $\scrT_{\I}$ and
$\scrT_{\II}$ of graded $\mod 2$ graded vector spaces
introduced in $\I$ and $\II$ are equivalent:
we have an equivalence of categories
$$
\widetilde{\bold{s}}\colon\,\scrT_{\I}\longrightarrow
\scrT_{\II}
$$
and an isomorphism of functors
$$
\alpha\colon\,
\widetilde{\bold{s}}(V\otimes_{\I}W)\arrowsim
\widetilde{\bold{s}}(V)\otimes_{\II}\widetilde{\bold{s}}(W)
$$
compatible with the associativity and commutativity
isomorphisms for $\otimes_{\I}$ and $\otimes_{\II}$.

The functor $\widetilde{\bold{s}}$ is the regrading functor:
$\widetilde{\bold{s}}(V)$ has the same underlying vector space
as $V$, with $v$ of parity $p$ and cohomological degree
$n$ acquiring the parity $p+n$, and keeping its
cohomological degree.

On the underlying vector space, $\otimes_{\I}$ and
$\otimes_{\II}$ are both the usual tensor product.
For $v$ in $V^{p,n}$ and $w$ in $W^{q,m}$, $\alpha$ is
defined as
$$
\alpha\colon v\otimes w\longmapsto (-1)^{nq}v\otimes w.
$$

The compatibility with the commutativity isomorphisms is
the commutativity of the diagram
$$
\CD
 @>{(-1)^{nq}}>> \\
@V{(-1)^{pq+nm}}VV @VV{(-1)^{(p+n)(q+m)}}V\\
 @>{(-1)^{mp}}>> 
\endCD
\qquad\qquad\qquad \lower25pt\hbox{.}
$$

The compatibility with the associativity isomorphisms
comes from the identity
$$
n_1(p_2+p_3)+n_2p_3=n_1p_2+(n_1+n_2)p_3\,\,.
$$

The functor ``{\it associated simply $(\mod 2)$
graded}'' $\bfs$ is the composite of $\widetilde{\bold{s}}$
with the functor ``forgetting the cohomological degree''
to super vector spaces.

\example{Example}
Let $A$ be a graded $\mod 2$ graded algebra (point of
view $\I$).
It is given by a multiplication
$$
\lower2pt\hbox{$\boldsymbol{\cdot}$}\,\,\colon\, 
A\otimes A\to A.
$$
Applying $\bfs$, we obtain a super algebra, with $a$ of
parity $p$ and cohomological degree $n$ becoming of
parity $p+n$.
The new product
$$
\gather
*\colon\, \bfs(A)\otimes\bfs(A)
\leftarrowsim \bfs(A\otimes
A)@>{\bfs(\lower1.5pt\hbox{$\boldsymbol{\cdot}$})}>>
\bfs(A)\\
\intertext{is}
x*y=(-1)^{nq}x.y
\endgather
$$
for $x$ in $A^{p,n}$ and $y$ in $A^{q,m}$.

The same applies to graded $\mod 2$ graded modules.
\endexample

\noindent
Example:\enspace
the already mentioned case of the K\"ahler differential
of $A$.

\example{Example: {\rm the Berezinian}}
For $V$ a super vector space, considering $V$ as a space
leads to consider the super algebra $\Sym(V^*)$ of
functions on ``$V$, viewed as a space''.
More generally, let $V$ be a free super module of finite
rank $(p,q)$ on a commutative super ring $A$.
One has an augmentation $\Sym_A^*(V)\to A$, and
$$
\Ber(V):= \underline{\Ext}_{\Sym_A(V^*)}^*
(A,\Sym_A^*(V^*)).
$$

Those $\Ext$ are best considered as given by the derived
functor of the functor $\underline{\Hom}
(A,\,\,\,)$, identified with
the functor from $\Sym_A^*(V)$-modules to $A$-modules:
$M\mapsto$ the sub-$A$-module of elements of
$M$ annihilated by the augmentation ideal.

The only non zero $\underline{\Ext}$ is $\underline{\Ext}^p$.
It is a free $A$-module of dimension $(1,0)$ for $q$
even, $(0,1)$ for $q$ odd.
Cohomological degree is $p$.

\smallskip\noindent
Basic examples:\enspace
For $\Sym_A(V^*)=A[T]$, $\underline{\Ext}$ 
can be computed using the
following resolution of $A[T]$, pictured as a
quasi-isomorphism of complexes
$$
\spreadmatrixlines{1\jot}
\matrix
A[T] &&A[T] &\\
\rmapdown{\equiv} &\colon &\Big\downarrow &\\
K^* &&A[T,T^{-1}] &\to A[T,T^{-1}]/A[T]\,\,.
\endmatrix
$$

This resolution is not injective, but it is good enough:
$\underline{\Ext}^i(A,K^j)=0$ 
for $i>0$, as can be seen by using the
projective resolutions $A[T]\mapright{T} A[T]$ of the
$A[T]$-module $A$, and this ensures that for any
injective resolution $K'$ and any morphism of resolutions
$\alpha\colon\,K\to K'$, $\alpha$ induces an
isomorphism, which is independent of $\alpha$,
$$
H^*\underline{\Hom}_{A[T]}(A,K)\longrightarrow
H^*\underline{\Hom}_{A[T]}(A,K')\,.
$$
A basis of $\underline{\Ext}^1$ is the image of $T^{-1}\in
A[T,T^{-1}]/A[T]$.
For $\Sym_A(V^*)=A[\theta]$, no resolution is necessary.
A basis of $\underline{\Ext}^0$ is $\theta$.
\endexample

For $(V_i)_{i\in I}$ a finite family of free modules, by
the associativity and commutativity isomorphisms for
$\oplus$, one can define the direct sum of the $V_i$
without having first to choose an ordering of $I$.
Similarly, by the associativity and commutativity
isomorphisms for $\otimes$, for the tensor product of
the $\Ber(V_i)$, viewed as graded super modules (point
of view $\I$). 
A K\"unneth formula gives a canonical isomorphism
$$
\Ber(\oplus V_i)\simeq \otimes\Ber(V_i)\,.
$$
If $V$ is filtered, a splitting of the filtration gives
a direct sum decomposition $V\arrowsim \oplus \text{Gr}_F^i
(V)$ and the resulting isomorphism
$$
\Ber(V)\arrowsim \otimes \Ber(\text{Gr}_V^i(V))
$$
does not depend on the choice of the splitting.

The compatibilities to which those constructions obeys
involve no sign.
Trouble begins when one wants to name an element of
$\otimes\Ber(V_i)$: a family of $b_i\in \Ber(V_i)$
define an element $\otimes b_i$ of $\otimes\Ber(V_i)$
only once an ordering of $I$ has been chosen, and for
different orderings, $\otimes b_i$ maps to different
elements, according ot the sign rule.
In other words: trouble begins when one starts writing
formuli, as formuli name elements.
Written text is linear, and it is the usage that the
linear order of a formula gives the linear order of $I$
which is used.

Applying to $\Ber$ the functor ``associated simply $\mod
2$ graded'' $\bfs$, one obtains\break
 $\ber(V):=\bfs \Ber(V)$.
It is a free $A$-module of rank $(1,0)$ or $(0,1)$,
depending on the parity of $p+q$;, and it obeys the same
compatibility formalism.

\example{Example: {\rm densities}}
On a supermanifold $W$ of dimension $(p,q)$, the line
bundle of densities is the tensor product of the
orientation local system with $\Ber(T^{\upvee})$, for
$T$ the tangent bundle.
It is natural to give the orientation local system
cohomological degree $-p$: the line bundle of densities
has dimension $(0,1)$ or $(0,1)$, depending on the
parity of $q$, and its cohomological degree zero.

With these conventions, the integration of densities with
compact support is an even map.
\endexample




\enddocument



