%Date: Fri, 09 May 1997 09:18:25 -0400
%From: Edward Witten <witten@sns.ias.edu>

\input harvmac

Homework, May 9
\def\bar{\overline}

These exercises are concerned with twisted versions of $N=2$ supersymmetric
theories in two dimensions.  Recall that in such a twisted theory one
considers only operators that commute with a $Q$ such that 
$Q=Q_++Q_-$ or $Q=\bar Q_++Q_-$ (or the hermitian conjugate of one of these).
If one works on a flat ${\bf R}^2$, $Q$ commutes with $S+S'$ where
$S$ is the generator of rotations of ${\bf R}^2$ and $S'$ is a sum or 
difference
of left and right moving $R$-symmetries.

(1) Consider a theory of chiral superfields $\Phi_i$ with some polynomial
superpotential
$W(\Phi_i)$.  We assume that $W$ has only finitely many critical points
(perhaps degenerate).

The Lagrangian is
\eqn\harry{L=\int d^2xd^4\theta \bar\Phi_i\Phi_i+\left(\int d^2xd^2\theta 
W(\Phi)
+c.c.\right).}

We recall that this theory can be obtained by dimensional reduction of a 
four-dimensional theory with coordinates $x^1,x^2,\dots ,x^4$; the reduction
to two dimensions is made by taking fields to be independent of $x^3,x^4$.
Let $S$ be the generator of rotations of the $1-2$ plane and $S'$ the
generator of rotations of the $3-4$ plane. $S'$ is a symmetry for any $W$.
Twist by $S'$ to get a topological field theory in which the $Q$-invariant
operators are chiral superfields (plus their $q$-form descendants for $q=1,2$);
this theory can be defined for any $W$ as $S'$ is a symmetry for any $W$.

Let ${\cal O}_i$ be chiral superfields (note that one can take these to be
polynomial functions of
the $\Phi_i$ only, without derivatives) 
and compute in genus zero the path integral
\eqn\juggu{\langle \prod_{i=1}^s{\cal O}_i(x_i)\rangle.}
As a preliminary reduction, note that by using the chiral ring you can
reduce to the case $s=1$.  

To compute the path integral in genus zero, the main step is to reduce
to a finite dimensional integral by throwing away the non-zero modes of all
fields.  If you don't know how to do or interpret the resulting finite 
dimensional
integral, practice with the case of only one chiral superfield.

(2) Consider first of all an $N=2$ gauge theory in two dimensions with gauge
group $G$.  The curvature is the twisted chiral superfield 
$\Sigma=\sigma+\dots$.
This theory also can come by dimensional reduction from four dimensions.
In four dimensions there is an $R$ symmetry $S''$.  Upon reduction to
two dimensions, we made a twisted topological theory in which rotations
were generated by $S+S''$ and the topological symmetry is $\bar Q_++Q_-$;
observables in this theory are twisted chiral superfields (and not chiral
superfields, as in the previous problem).

We showed the following:

(a) Invariant polynomials in $\sigma$ (and their $q$-form descendants) gave
the topological observables.

(b) Such a polynomial $P$ corresponds to a cohomology class $w_P$ on
the moduli space of flat connections.

(c) Correlation functions are given by integrals over moduli space of
a product of such cohomology classes.  (It is important to consider the
$q$-form descendants since otherwise everything vanishes, a statement first
conjectured by Newstead for $G=SU(2)$.)

Now modify this by including chiral superfields $\Phi_i$ in some representation
of $G$.  Note that (in contrast to problem (1) where we made a twist that
allows an arbitrary superpotential) here we are making the twist by
$S+S''$ that forbids a generic superpotential.\foot{If one wants to include
a superpotential, (i) it must have weight 2 under some ${\bf C}^*$ action
on the $\Phi_i$, (ii) one must add to $S''$ the generator of this
${\bf C}^*$, thus modifying how the geometrical $R$ symmetry acts on the
chiral superfields.}
Because of this, take the superpotential to be zero.

There is no change in (a) and (b).  However, (c) will be modified.  The 
correlation
functions of the operators $P^{(q)}$ (integrated over $q$-cycles in $\Sigma$)
 in  the twisted theory with the chiral superfields
will be the integral over the moduli space of flat connections of
the same product of cohomology classes that we had before times a new
class that depends on the chiral superfields.  Can you identify this class?







\end


