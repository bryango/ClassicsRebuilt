%Date: Sat, 4 Apr 1998 16:14:21 -0500 (EST)
%From: Pavel Etingof <etingof@abel.math.harvard.edu>

\input amstex
\documentstyle{amsppt}
\magnification 1200
\NoRunningHeads
\NoBlackBoxes
\input epsf
\document

\def\tW{\tilde W}
\def\Aut{\text{Aut}}
\def\tr{{\text{tr}}}
\def\ell{{\text{ell}}}
\def\Ad{\text{Ad}}
\def\u{\bold u}
\def\m{\frak m}
\def\O{\Cal O}
\def\tA{\tilde A}
\def\qdet{\text{qdet}}
\def\k{\kappa}
\def\RR{\Bbb R}
\def\be{\bold e}
\def\bR{\overline{R}}
\def\tR{\tilde{\Cal R}}
\def\hY{\hat Y}
\def\tDY{\widetilde{DY}(\g)}
\def\R{\Bbb R}
\def\h1{\hat{\bold 1}}
\def\hV{\hat V}
\def\deg{\text{deg}}
\def\hz{\hat \z}
\def\hV{\hat V}
\def\Uz{U_h(\g_\z)}
\def\Uzi{U_h(\g_{\z,\infty})}
\def\Uhz{U_h(\g_{\hz_i})}
\def\Uhzi{U_h(\g_{\hz_i,\infty})}
\def\tUz{U_h(\tg_\z)}
\def\tUzi{U_h(\tg_{\z,\infty})}
\def\tUhz{U_h(\tg_{\hz_i})}
\def\tUhzi{U_h(\tg_{\hz_i,\infty})}
\def\hUz{U_h(\hg_\z)}
\def\hUzi{U_h(\hg_{\z,\infty})}
\def\Uoz{U_h(\g^0_\z)}
\def\Uozi{U_h(\g^0_{\z,\infty})}
\def\Uohz{U_h(\g^0_{\hz_i})}
\def\Uohzi{U_h(\g^0_{\hz_i,\infty})}
\def\tUoz{U_h(\tg^0_\z)}
\def\tUozi{U_h(\tg^0_{\z,\infty})}
\def\tUohz{U_h(\tg^0_{\hz_i})}
\def\tUohzi{U_h(\tg^0_{\hz_i,\infty})}
\def\hUoz{U_h(\hg^0_\z)}
\def\hUozi{U_h(\hg^0_{\z,\infty})}
\def\hg{\hat\g}
\def\tg{\tilde\g}
\def\Ind{\text{Ind}}
\def\pF{F^{\prime}}
\def\hR{\hat R}
\def\tF{\tilde F}
\def\tg{\tilde \g}
\def\tG{\tilde G}
\def\hF{\hat F}
\def\bg{\overline{\g}}
\def\bG{\overline{G}}
\def\Spec{\text{Spec}}
\def\tlo{\hat\otimes}
\def\hgr{\hat Gr}
\def\tio{\tilde\otimes}
\def\ho{\hat\otimes}
\def\ad{\text{ad}}
\def\Hom{\text{Hom}}
\def\hh{\hat\h}
\def\a{\frak a}
\def\t{\hat t}
\def\Ua{U_q(\tilde\g)}
\def\U2{{\Ua}_2}
\def\g{\frak g}
\def\n{\frak n}
\def\hh{\frak h}
\def\sltwo{\frak s\frak l _2 }
\def\Z{\Bbb Z}
\def\C{\Bbb C}
\def\d{\partial}
\def\i{\text{i}}
\def\ghat{\hat\frak g}
\def\gtwisted{\hat{\frak g}_{\gamma}}
\def\gtilde{\tilde{\frak g}_{\gamma}}
\def\Tr{\text{\rm Tr}}
\def\l{\lambda}
\def\I{I_{\l,\nu,-g}(V)}
\def\z{\bold z}
\def\Id{\text{Id}}
\def\<{\langle}
\def\>{\rangle}
\def\o{\otimes}
\def\e{\varepsilon}
\def\RE{\text{Re}}
\def\Ug{U_q({\frak g})}
\def\Id{\text{Id}}
\def\End{\text{End}}
\def\gg{\tilde\g}
\def\b{\frak b}
\def\S{\Cal S}
\def\L{\Lambda}

\topmatter
\title Lecture II-2, Part II: Spontaneous breaking of gauge symmetry 
\endtitle
\author {\rm {\bf Edward Witten} }\endauthor
\endtopmatter
\centerline{Notes by Pavel Etingof and David Kazhdan}

In this lecture we will consider gauge symmetry breaking. 

{\bf 2.1. Gauge symmetry.} 
Recall what gauge symmetry is. We have a spacetime $X=\R_{time}\times
X_0$. We have a compact gauge group $G$. 
We have a field theory where a field configuration
is a connection is some principal bundle over $\R^d$ and possibly 
some matter fields. 

Recall the Hamiltonian approach to gauge theory. 
Let $\tilde M_0$ be the space of solutions to the classical 
equations of motion. On $\tilde M_0$ we have an action 
of the group $\hat G$ of gauge transformations. 
Let $M_0\subset \tilde M_0$ be the space of all solutions 
where the G-bundle is trivialized in the time direction, and 
the connection is trivial in that direction. Such solutions 
as usual are completely determined by the pair $A(t_0),\frac{dA}{dt}(t_0)$, 
where $t=t_0$ is a space cycle, and initial data for the matter fields. 
It is clear that any element of $\tilde M_0$ 
can be brought to $M_0$ by a gauge transformation, so 
$M_0$ still contains all solutions up to gauge transformations.      

Suppose that $X_0=\R^{d-1}$. In this case we may consider only trivial 
bundles, and connections which vanish at spatial infinity. 
In other words, if $\Cal A$ 
is the space of connections $A$ on the trivial $G$-bundle over $\R^{d-1}$
which vanish at $\infty$ then $M_0$ for pure gauge theory is 
$T^*\Cal A$. If matter fields are present, then $M_0$ is a product of 
$T^*\Cal A$ with some other space. 

Define $\tilde G$ to be 
the group of elements of $Maps(\R^{d-1},G)$ 
which have a limit at infinity, and $\tilde G_0$ to be the subgroup of
$\tilde G$ consisting of functions which tend to 1 at $\infty$. 
We have $\tilde G/\tilde G_0=G$. This quotient group is called the group 
of constant gauge transformations at $\infty$ and called $G_\infty$. 

The group $\tilde G$ acts symplectically on $M_0$. 
The physical phase space in gauge theory is the symplectic quotient
$M=M_0//\tilde G_0$. Note that we only divide by $\tilde G_0$ and not by
the whole group $\tilde G$, so that we inherit an action 
of the quotient $G_\infty$ on $M$. It is this symmetry group whose
breaking we will discuss. 

{\bf 2.2. Breaking of gauge symmetry 
and charges at infinity.}

{\bf Definition.} Suppose we have a (classical) gauge theory, and 
let $s\in M$ be its vacuum state. Let $H\subset G=G_\infty$ be 
the stabilizer of $s$. In this case  
we will say that at the vacuum state $s$  the gauge 
symmetry is broken from $G$ to $H$.

Thus, by symmetry breaking we mean essentially the same thing as for 
global symmetry: there is a 
symmetry of the Poisson algebra of functions on $M$ which 
does not fix a particular vacuum state. 

{\bf Important remark.} The above expression ``the same thing ''
should be taken with great care. There are some fundamental differences 
between the two situations, which will become clear below. 
They come from the fact that in the situation we are considering here, 
(unlike Lecture II1) the physical observables, being gauge invariant 
by definition, automatically commute with $G$ and therefore do not, in 
general, separate points on $M$; i.e. not every function on $M$ is 
``observable''. In other words, the action of 
$G$ on the ``theory'' (in the sense of Lecture II1) is trivial 
from the beginning. 
 
Let us now compute the action of $G$ (classically).  
First of all, we have a moment map $\mu:M_0\to \tilde \g_0^*$, where 
$\tilde\g_0$ is the Lie algebra of $\tilde G_0$ -- the algebra of 
functions from $\R^{n-1}$ to the Lie algebra $\g$ of $G$ which vanish at 
infinity. Thus, for any $\e\in \tilde\g_0^*$ we have a Hamiltonian 
$Q(\e)\in C^\infty(M_0)$ defined by $Q(\e)(X)=\mu(X)(\e)$. 

In fact, it is easy to compute $Q(\e)$ using Noether formalism. Namely, 
$$
Q(\e)=\int_{\R^{d-1}}Tr(\frac{\d A}{dt}\nabla_A \e)d^{d-1}x+
\text{ matter terms },\tag 2.1
$$

On $M$, $Q(\e)=0$ if $\e$ vanishes at infinity. Thus, on $M$ we have 
$\nabla_A^*\frac{dA}{dt}=
\text{ matter terms }$. In particular, in pure gauge theory 
$\nabla_A^*\frac{dA}{dt}=0$. 

Taking this into account, we see that on $M$ 
$$
Q(\e)=\int_{\R^{d-1}}Tr(\nabla_A (\e\frac{dA}{dt}))d^{d-1}x.\tag 2.2
$$

Using Stokes' formula, we can rewrite (2.2) as   
$$
Q(\e)=\lim_{r\to\infty}\int_{S^{n-2}(r)}
*_{d-1}Tr(\e\frac{dA}{dt})=\lim_{r\to\infty}
\int_{S^{n-2}(r)}*_dTr(\e F), \tag 2.3
$$
where $F$ is the curvature of the spacetime connection
corresponding to the given point of $M_0$, and $S^k(r)$ is the k-sphere of 
radius $r$. This formula defines the hamiltonians 
for the action of $G=G_\infty$ on $M$. 

This formula shows that $Q(\e)$ vanishes for all gauge transformations 
(not necessarily vanishing at infinity) on a particular state 
if $F=o(r^{2-n})$, $r\to\infty$ on that state. However, if 
this is not the case, then $Q(\e)$ may be nonzero for a constant $\e$.  

{\bf Example.} Consider a $U(1)$ gauge theory with a charged 
complex scalar. The fields are a connection $A$ on a hermitian line bundle 
and a section $\phi$ of this bundle. The Lagrangian is 
$$
\Cal L=\frac{1}{4e^2}\int F^2+\int |D_A\phi|^2d^4x+\int\frac{\l}{8}
(|\phi|^2+v^2)^2d^4x.\tag 2.3 
$$
This is the most general renormalizable Lagrangian in these fields in 
4 dimensions. Here $e,\l,v$ are parameters and $e^2,\l$ are positive while 
$v^2$ can be positive or negative. For simplicity we assume first 
that $v^2\ne 0$. 

This theory is not believed to exist in the UV, but we will regard it 
 as an effective theory for some more fundamental theory. 

Classically (and quantum mechanically for $e^2,\l<<1$) we have two cases. 

1. $v^2>0$; the potential has a single minimum. 

2. $v^2<0$; the potential has a circle of minima. 

Let us consider how in these two cases the theory behaves in the infrared. 

\centerline{\epsfxsize=2in\epsfbox{plot2.eps}}
\centerline{Figure 1. The potential for $v^2>0$.}

\bigskip

{\bf Case 1.} $v^2>0$. In this case the minimum of energy is attained when 
$\phi=0$. First consider the case when the gauge coupling vanishes: 
$e^2=0$. In this case our theory is a direct product of a pure (free) 
abelian gauge theory and the $\phi^4$ theory. Therefore, it has a unique 
vacuum, and    
the particles which occur at the lower part of the spectrum are 
a massless vector, or gauge boson 
(coming from gauge theory) and two massive real scalars
(coming for $\phi^4$ theory). 

If we turn on small $e^2$ the situation should remain the same. 
Indeed, certainly nothing can happen to the massive scalars 
(the part of the Hilbert space 
with the nonzero charge, where these scalars are, has a mass gap, and 
massiveness is an open condition); moreover, their masses must be equal since 
there is a $U(1)$ symmetry at infinity (the $Q(\e)$ for constant $\e$)
which prohibits the masses to differ. The fact that $Q(\e)\ne 0$ 
is clear since this is so at $e^2=0$, when $Q(\e)$ represents the $U(1)$ 
global symmetry. 

Furthermore, the massless vectors cannot become massive. 
Indeed, recall that a massless vector means an irreducible 
representation of $SO(3,1)$ with $p^2=0$ and spin 1, i.e. the space of 
sections of a 2-dimensional equivariant vector bundle over the light cone.
This vector bundle cannot be deformed to an equivariant vector bundle over 
the hyperboloid, since the stabilizer group $SO(3)$ of a point on the 
hyperboloid does not have an irreducible 2-dimensional representation. 
 Thus, the quantum theory for small coupling 
will have the same particles -- two massive scalars 
(the real and imaginary part of $\phi$) and a massless vector
(the gauge boson).  


{\bf Remark.} The above argument on non-deformability of a massless vector
fails in 3 and 2 dimensions. For example, in 3 dimensions, the 
massless vector is just the space of functions over the cone, which 
can be successfully deformed into the space of functions over a hyperboloid. 
This actually happens when in pure $U(1)$ gauge theory one introduces 
a Chern-Simons term $c\int A\wedge dA$. The theory remains free but 
becomes massive, yielding one massive scalar. In the theory we are 
considering (for 3 dimensions), this cannot happen dynamically since 
the Chern-Simons term is odd under change of orientation, but in other 
theories this could happen. 

In fact, quantum mechanically 
the operator $Q(\e)$ (for a suitable normalization of $\e$)
has integer eigenvalues, and thus defines (in quantum theory) 
a $\Z$-grading of the corresponding Hilbert space. In particular, 
since $Q(\e)\ne 0$, there are sectors of the Hilbert space which 
cannot be reached from the vacuum by applying local operators. 
This shows that we have a fundamental violation of Wightman axioms: 
the representation of the operator algebra in the physical Hilbert space 
is not irreducible. However, the theory still has one vacuum only: 
the minimal energy in the sectors with nonzero charge $Q(\e)$ is positive.   

\centerline{\epsfxsize=2in\epsfbox{plot1.eps}}
\centerline{Figure 2. The potential for $v^2<0$.}

\bigskip

{\bf Case 2.} $v^2<0$. Let $v^2=-b^2$. Then classically we have a minimum of 
energy on the circle $|\phi|=b$. This implies 
that any finite energy configuration has the property 
$\phi=be^{i\theta_0}$ at infinity, where $\theta_0$ is a constant. 
Therefore, by a gauge transformation which has a finite limit 
at infinity, we can arrange 
that $\phi$ is real and positive: $\phi=b+w$ where $w$ is a new real 
variable.  
Writing the Lagrangian in terms of the new variables, we will get 
something with the following quadratic part: 
$$
L_{quadratic}=\frac{1}{4e^2}\int F^2+\int d^4x((dw)^2+M^2w^2)+\int d^4xb^2A^2.
\tag 2.4
$$
It is seen from (2.4) than now all fields are massive. Of course, 
Lagrangian (2.4) is not gauge invariant for $A$, since we have already 
``spent'' the gauge symmetry on making $\phi$ real. 

Thus, infrared limit of the corresponding quantum theory is trivial
for small values of the couplings. In particular, there are no massless gauge 
bosons: they have been ``eaten'' by the $\phi$-field. This situation 
is called Higgs phenomenon, or spontaneous breaking of gauge symmetry.  

Note that in spite of the presence of a circle of zero energy states, 
our theory has only one vacuum. In other words, all points of the circle are 
regarded as the same state, on the grounds that they are gauge equivalent 
to each other and therefore define equivalent realizations
(i.e. give the same expectation values of gauge invariant local operators)
This is a fundamental difference between gauge 
and global symmetry breaking. In global symmetry breaking, the points of 
the circle represent different vacua (realizations) of the theory, 
since there exist non-symmetric operators which have different 
expectation values at different point of the circle. 

Note also 
that the operator $Q(\e)$ doesn't act in the Hilbert space of states, 
since classically $Q(\e)$ generates a group which rotates the circle 
and permutes the zero energy states. In particular, in this case 
local operators act irreducibly in the Hilbert space, and there are 
no sectors  
which cannot be reached from the vacuum. This is the difference between 
case 2 and case 1: in case 1, as you remember, $Q(\e)$ acts in the Hilbert 
space nontrivially and defines a splitting into sectors. 

{\bf Remark.} If one tries to compute $Q(\e)$ in Case 2 (when the symmetry 
is broken) using formula (2.3), the answer will be zero since the integrand 
dies rapidly at infinity. 
 
The particles which are found in the infrared in the situation 
of Case 2 are, according to (2.4),  a massive vector ($A$) and a massive 
scalar ($\phi$). There is only one scalar since $\phi$ is now real.  
Thus, at the level of representation theory the Higgs phenomenon arising 
in Case 2 boils down to a deformation of representations of the Poincare 
group: a massless vector plus a massless scalar is deformed to a massive 
vector. Recall for comparison that a massless vector separately 
cannot be deformed into a massive representation. 

Finally, consider the special case $v^2=0$. In this case classically 
we have no symmetry breaking as for $v^2>0$, and the particles 
are a massless vector and two massless scalar. However, it is not 
expected to be the quantum answer, since this configuration is not stable 
under perturbations. 

{\bf Remark.} If the Lagrangian we start with is not IR free 
(say, it is the Lagrangian of an asymptotically free gauge theory) 
then the classical analysis we discussed above does not apply
in quantum theory. In this case the infrared behavior of the theory 
is difficult to determine. In particular, it could happen that in the 
infrared the gauge group of the ultraviolet theory will be replaced with 
some completely different group, which is not even a subgroup in the original 
group.  

{\bf 2.3. Symmetry breaking and gauging.}
In conclusion, let us discuss the connection between global symmetry and gauge 
symmetry breaking. Suppose we have a Lagrangian $L$ of a field theory 
(say in 4 dimensions) which has a global $U(1)$ symmetry.
A typical example is when the theory contains some scalar fields $\phi_j$ 
which are sections of hermitian vector bundles,  
and $U(1)$ acts by multiplication of these sections 
by $e^{in_j\theta}$ This $U(1)$ symmetry can be gauged, by introducing 
a $U(1)$ gauge field $A$ and new Lagrangian 
$$
L_{gauged}=\frac{1}{4e^2}\int F_A^2+L_A, \tag 2.5
$$
where $L_A$ is $L$ in which all derivatives of $\phi_j$ are replaced 
by covariant derivatives. 

The statement is that 
if $L$ is infrared free 
then for small gauge couplings the symmetry breaking behavior 
of the theories defined by $L$ and $L_{gauged}$ is usually the same. 
Namely, if there is breaking of global symmetry for $L$ 
then there is breaking of gauge symmetry for $L_{gauged}$ and vice versa. 

Indeed, let us consider both cases.

{\bf Case 1}. No global symmetry breaking. In this case 
classically the minimum of energy is at $\phi_j=0$, and thus there is a 
$U(1)$-invariant vacuum. In quantum theory,  
$U(1)$ acts in the Hilbert 
space, and there are no massless particles (Goldstone bosons) corresponding 
to $U(1)$. 
In this case, for small gauge coupling $e$, the matter part 
$L_A$ of the Lagrangian almost decouples from the gauge part; 
so classically we get a massless gauge boson. 

To consider the quantum mechanical situation, 
we assume that there are no massless particles 
in the ungauged theory. In this case, the above classical answer 
is also quantum mechanical for small couplings, by the non-defomability 
of representations from massless to massive. However, if massless particles
(say Goldstone bosons corresponding to other symmetries which are broken)
are present, this answer may not be true. 

{\bf Case 2.} Global symmetry breaking. In this case at the minimum of energy 
some of the $\phi_j$ is not zero. There is no invariant vacuum, and 
there is a Goldstone boson corresponding to this symmetry breaking. 
In this case, pick a vacuum state and 
a component $\phi_j^{(n)}$ which is not zero at this vacuum. 
In the gauged theory, we can perform a gauge transformation 
which will make this component real and positive. 
This shows that if there are no other 
massless particles (in particular, no other broken global symmetries), 
all fields in the theory will become massive. This happens classically, due 
to Higgs mechanism as in Case 2 above, and also quantum mechanically for small 
couplings. Thus, we have breaking of gauge symmetry. 

  
\end

