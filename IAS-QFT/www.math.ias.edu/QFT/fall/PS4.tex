%Date: Thu, 24 Oct 1996 15:32:29 EDT
%From: Edward Witten <witten@sns.ias.edu>

\input harvmac

{\it ``Problem Set Four''}



(1) Consider a simple harmonic oscillator in one dimension.
The Lagrangian is
\eqn\normo{ L= \int dt\left({m\over 2}\left({dx\over dt}\right)^2
-{1\over 2}m\omega^2x^2\right).}
The Hamiltonian is accordingly
\eqn\pippo{H={p^2\over 2m}+{1\over 2}m\omega^2x^2}
where
\eqn\ippo{p=-i{d\over dx}}
acting on ${\bf L}^2({\bf R})$.


Calculate the kernel of the operator $e^{-itH}$, that is
the matrix element
\eqn\nopo{ \langle b |e^{-iHt}|a\rangle,}
using path integrals.

That is, find the classical path $x_0(t)$ that starts at $a$ at time
0 and ends at $b$ at time $t$.  Write $x(t)=x_0(t)+\tilde x(t)$.
By substituting $x(t)$ in $L$, write the action as
a quadratic function of $\tilde x(t)$ plus a known function
of $a,b$ and $t$.  Do the path integral over $\tilde x(t)$ by
interpreting it as a determinant which you must study somehow.

In fact after dividing out formally by an infinite but $t$-independent
constant - or if you prefer, upon using something like a zeta
function definition of the determinant - you should be able
to reduce to an infinite sum that will look something like
\eqn\gippo{\prod_{n=1}^\infty\left(1+{p^2\over n^2}\right)}
for some $p$.  This product in fact defines a certain elementary
function of $p$.  

(2) Consider the scalar field theory with Lagrangian
\eqn\goppo{L=\int d^nx\left({1\over 2}|d\phi|^2
+{1\over 2}m^2\phi^2+{g\over 3!}\phi^3\right)}
in $n$-dimensional Euclidean space.

Calculate the contribution of order $g^3$ to the
three-point function $\langle\phi(x)\phi(y)\phi(z)\rangle.$
(It is best to work in momentum space, that is to set $z=0$
and take the Fourier transform with respect to $x$ and $y$.)

(a) Demonstate that the result is finite in $n\leq 6$, and
for $n=6$, describe the precise form of the divergence -- that
is, introduce a cutoff $\Lambda$, and describe just what sort
of $\Lambda$-dependence you get for large $\Lambda$.

Renormalize the result for $n=6$.  That is, carry out the
necessary renormalization and write down the renormalized,
finite correction to the three point function.


(b) For $n<6$ make the result as explicit as you can.
Use Feynman's identity
\eqn\loopo{{1\over ABC}=\int_0^1d\alpha\,d\beta\,d\gamma\,
\delta(\alpha+\beta+\gamma-1)
{1\over (\alpha A+\beta B+\gamma C)^3}.}
After completing the square, you should be able to reduce the
integral over the loop momentum $q$ to
\eqn\obbo{\int d\alpha\,d\beta\,d\gamma\,\delta(\alpha+\beta+\gamma
-1)\int d^nq\,{1\over (q^2+b^2)^3}}
for some $b$ (which is a polynomial in the three external momenta
and in $\alpha$ and $\beta$).
Set $n=4$ and carry out the $q$ integral explicitly.

For extra credit, get as far as you reasonably can with the $\alpha$
and $\beta$ integrals.

(3) Now consider a theory in $n$-dimensional Euclidean space
of a scalar $\phi$ and a spinor $\psi$ with Lagrangian
\eqn\jogo{L=\int d^nx\left({1\over 2}|d\phi|^2+{1\over 2}M^2\phi^2
+{i\over 2}
\left(\psi, i(D+m)\psi\right) +{g\over 2}\phi (\psi,\psi)\right).}
($\psi$ takes values in a representation of $Spin(n)$ that
is associated with a real representation of $Spin(1,n-1)$.  The
precise details depend a bit on $n$, but this should not
affect the calculation you will have to do. $(~,~)$ is a quadratic
form on the spin representation. $D$ is the Dirac operator.)

The free propagator of the $\psi$ field is
in momentum space $1/(\Gamma\cdot k+im)$, where $\Gamma\cdot k$
is physicist's notation for Clifford multiplication by $k$.  


The free propagator of $\phi$ is $1/(k^2+M^2)$.

$L$ is written so that there is only one interaction vertex,
a cubic vertex with two $\psi$'s and one $\phi$.
The factor to be included at such a vertex in  a Feynman
diagram is just $-g$ (times the identity matrix or natural
bilinear form $(~,~)$ on the fermion $\psi$, depending on how you
look at it).

Compute the correction of order $g^2$ to the fermion
two point function $\langle \psi(x) \psi(0)\rangle$.  (Again,
it is best to take the Fourier transform.  Also, it is best
to write the two point function as $1/(\Gamma\cdot k+im+i\Sigma)$
and think of computing a correction to $\Sigma$, which will
be given by an ``amputated one loop diagram.''  $\Sigma$
will be of the form $w(k^2) \Gamma\cdot k +iv(k^2)$ and the problem
is to study both functions $w$ and $v$.)  

(a) Show that the result is finite for $n<4$. 

(b) Show
that (for $n<4$) the result is analytic in $k^2$ for $k^2>-(m+M)^2$.
(We are writing the formulas in Euclidean signature here;
in Lorentz signature with metric $+-----$ this would
read $k^2<(m+M)^2$.)  How does this compare to what you would
have expected?  In particular, your result is holomorphic
near $k^2=-m^2$, so that the correction changes the value
of the fermion mass (by an amount that is small if $g$ is small)
but not the existence of a discrete state.


(c) For $n=4$, what renormalizations are necessary to remove the
infinity that you have found?  Write a formula for the renormalized,
finite, correction to the two point function, that is, to $\Sigma$.
For $n=4$, does the renormalized two point function have
the same analytic behavior found in (b) for $n<4$?
\end


