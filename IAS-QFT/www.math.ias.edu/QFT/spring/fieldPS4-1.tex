%From: Elly Gustafsson <gustafss@math.ias.edu>
%Date: Wed, 26 Mar 1997 09:05:26 -0500

\input amstex
\documentstyle{amsppt}
\magnification=1200
\pagewidth{6.5 true in}
\pageheight{8.9 true in}
\loadeusm
\NoPageNumbers
\input pictex

\def\operatorname{ln}

\catcode`\@=11
\def\logo@{}
\catcode`\@=13

\NoRunningHeads


\font\boldtitlefont=cmb10 scaled\magstep2

\def\lam{{\lambda}}
\def\eps{{\varepsilon}}

\def\psibar{\bar{\psi}}
\def\Lbar{\bar{L}}
\def\Sbar{\bar{S}}

\def\Ghat{\hat{G}}

\def\halfspace{\lineskip=1pt\baselineskip=15pt
     \lineskiplimit=0pt}
\def\vrulesub#1{\hbox{\vrule height7pt depth5pt\,}_{#1}}
\def\Ge{{\mathchoice{\,{\scriptstyle\ge}\,}
  {\,{\scriptstyle\ge}\,}
  {\,{\scriptscriptstyle\ge}\,}{\,{\scriptscriptstyle\ge}\,}}}

\def\eps{{\varepsilon}}
\def\Lam{{\Lambda}}


\def\dbC{{\Bbb C}}

\def\Tr{\text{\rm Tr}} \def\ln{\text{\rm ln}}
\def\End{\text{\rm End}} \def\Sym{\text{\rm Sym}}
\def\I{\text{\rm I}} \def\free{\text{\rm free}}
\def\II{\text{\rm II}}
\def\III{\text{\rm III}}


\def\scr#1{{\fam\eusmfam\relax#1}}

\def\scrL{{\scr L}}


\document

\line{{\boldtitlefont Witten's
Problems}, Set Four N$^{\text {o}}$. 1
\hfill(solution written by D.~Freed)}
\hbox to \hsize{\hrulefill}

\bigskip

\noindent
{\bf Problem:}
\smallskip

(1) In the two-dimensional sigma model with target
space a sphere $S^n$, for large $n$, we introduced a
Lagrange multiplier field $\sigma$ in order to
quantize the theory.

(a) By using the equations of motion, identify
$\sigma$ as an operator in terms of the original
variables.

(b) What is the dimension of the $\sigma$ field and
therefore what would you expect to be the large $q$
behavior (with Euclidean $q$) of $\langle \sigma
(q)\sigma (-q)\rangle$.
Compare to the actual behavior.
\bigskip\bigskip

\noindent
{\bf Solution:}
\smallskip
(a) This model has two bosonic fields: a map
$\phi:\Bbb R^2 \rightarrow \Bbb R^{n+1}$
and a ``Lagrange multiplier'' $\sigma:\Bbb R^2
\rightarrow \Bbb R$.
The action is
$$
\Cal L =\frac 1{2\lambda} \, |d\phi|^2 -\frac
{i\sigma}{2} (|\phi|^2 -1).
$$
It is easy to compute the classical equations of
motion
$$
\align
|\phi|^2&= 1\\
\frac {\Delta}{\lambda} \phi &=i\sigma \phi.
\endalign
$$
Our Laplacian '$\Delta$' is nonnegative.
The first says that $\phi$ maps into the $n$-sphere
$S^n$ and the second that the Laplacian of $\phi$ is
perpendicular to $S^n$.
This means that $\phi$ is a harmonic map into
$S^n$.
We express $\sigma$ in terms of $\phi$ as
$$
\sigma =\frac {1}{i\lambda} (\Delta \phi, \phi).\tag
1
$$
Up to a total derivative
 $\sigma = \frac 1{i\lambda} |d\phi|^2$.

\smallskip
(b) According to equation (1) the $\sigma$ field has
dimension 2, since $\phi$ is a scalar field in two
dimensions and so has dimension $0$.
We expect the behaviour of the 2-point function in
position space to be
$$
\langle \sigma(x) \sigma(0)\rangle
\sim \frac 1{|x|^4} \text { as } x\rightarrow 0,
$$
or equivalently in momentum space to be
$$
\langle\sigma(q) \sigma(-q)\rangle \sim q^2 \quad
\text { on } \quad q\rightarrow \infty,
\tag 2
$$
with possible logarithmic connections due to
anomalous dimensions.
We could integrate out the $\sigma$ field first,
thereby obtaining the $\sigma$-model into $S^n$, and
since this theory is asymptotically free the 2-point
function of $|d\phi|^2$ does exhibit the behaviour
(2) (up to logarithms).
Instead, we integrate out $\phi$ first and compute
the effective action for $\sigma$, at least in the
large $n$ limit.

The computation of the effective action appears in
Witten's lecture II-3:
$$
\Gamma (\tilde\sigma)= \frac n2 \{Tr \, \ln \, (\Delta
-i \tilde \sigma\tilde \lambda)+ i\tilde
\sigma\}.\tag 3
$$
where $\tilde\sigma=\sigma/n$ and $\tilde
\lambda=\lambda n$.
We know that we should expand around the value $-i
\tilde \sigma_0 \tilde \lambda=M^2$ for a positive
constant $M^2$.
The $\tilde \sigma$ propagator is the inverse of the
second derivative of the action (3), which we
evaluate at the constant $\tilde\sigma_0=
iM^2/\tilde\lambda$:
$$
\aligned
\frac {\delta^2\Gamma}{\delta\tilde
\sigma^2}\bigg|_{\tilde \sigma_0} &=
\frac {n\tilde \lambda^2}2
Tr \big((\Delta+M^2)^{-1} \delta \tilde\sigma
(\Delta+M^2)^{-1} \delta \tilde\sigma\big)
\\
&= \frac {n\tilde \lambda^2}{2} \int dxdy \, G(x, y)
\, \delta \tilde \sigma(y) \, G(y, x) \, \delta
\tilde\sigma(x)\\
&=\frac {n\tilde\lambda^2}{2} \int dxdy \, G(x, y)^2
\, \delta \tilde\sigma(x) \, \delta \tilde\sigma (y),
\endaligned
\tag 4
$$
where $G$ is the kernel of the Greens function
$(\Delta +M^2)^{-1}$.
So the kernel of the inverse $\tilde
\sigma$-propagator is $\frac {n\tilde \lambda^2}{2}
G(x, y)^2$.
In momentum space the square goes over to convolution,
and so the kernel of the inverse
$\tilde\sigma$-propagator is
$$
\align
K(q, -q)&= \frac {n\tilde\lambda^2}{2} \int \frac
{d^2p}{(2\pi)^2} \  \frac 1{p^2+M^2} \ \frac 1
{(q-p)^2+M^2}.\tag 5\\
&=\frac {n\tilde\lambda^2}{8\pi^2} \int_0^1 d\alpha
\int d^2p \ \frac
{1}{[p^2+M^2+\alpha(1-\alpha)q^2]^2}\\
&=\frac {n\tilde \lambda^2}{8\pi} \int_0^1 \, \frac
{d\alpha}{M^2+\alpha (1-\alpha)q^2}\\
&= \frac {n\tilde \lambda^2}{4\pi} 
\ \frac
{\ln \bigg(\frac{\sqrt{1+4M^2/q^2}+1}{\sqrt
{1+4M^2/q^2}-1}\bigg)}{q^2\sqrt{1+4M^2/q^2}}\\
&\sim \frac {n\tilde\lambda^2}{2\pi} \ \frac{\ln
(q^2/M^2)}{q^2} \quad \text { as } \quad q
\rightarrow \infty.
\endalign
$$
So the $\tilde\sigma$-propagator behaves as
$$
\langle \tilde\sigma(q) \tilde \sigma(-q)\rangle \sim
\frac {2\pi}{n\tilde \lambda^2} \ \frac {q^2}{\ln
(q^2/M^2)} \quad \text { as } \quad
q\rightarrow\infty.
$$
Since $\tilde\sigma =\sigma/n$, we have finally
$$
\langle \sigma(q) \sigma(-q)\rangle \sim \frac {2\pi
n}{\tilde \lambda^2} \, \frac {q^2}{\ln (q^2/M^2)}
\quad \text { as } \quad q\rightarrow \infty.\tag 6
$$
Notice that the behaviour in $n$ could have been
predicted from (1), since the 2-point function of
 $(\Delta \phi, \, \phi)$ is $n+1$ times the 2-point
function of a single scalar field.
The large $q$ behavior is as predicted in (2),
but corrected by a logarithm.

Notice that the inverse propagator (5) can be computed
by diagrams.
The effective action obtained by integrating our
$\phi$ is computed by diagrams with external $\sigma$
lines and internal $\phi$ lines.
We should expand around $\sigma = i M^2/\lambda$, of
course.
For the quadratic term there is a single diagram, as
shown, where the solid line
%\input circ.tex
%%% beginning of file circ.tex
%\input pictex
\font\thinlinefont=cmr5
$$
\hbox{\beginpicture
\setcoordinatesystem units < 1.000cm, 1.000cm>
%\unitlength= 1.000cm
\linethickness=1pt
%\setplotsymbol ({\makebox(0,0)[l]{\tencirc\symbol{'160}}})
\setshadesymbol ({\thinlinefont .})
\setlinear
%
% Fig ELLIPSE
%
\linethickness= 0.500pt
\setplotsymbol ({\thinlinefont .})
\ellipticalarc axes ratio  0.953:0.953  360 degrees 
	from  6.001 24.479 center at  5.048 24.479
%
% Fig POLYLINE object
%
\linethickness= 0.500pt
\setplotsymbol ({\thinlinefont .})
\setdots < 0.0953cm>
\plot  6.159 24.479  7.588 24.479 /
%
% Fig POLYLINE object
%
\linethickness= 0.500pt
\setplotsymbol ({\thinlinefont .})
\plot  3.937 24.479  2.508 24.479 /
\linethickness=0pt
\putrectangle corners at  2.508 25.432 and  7.588 23.527
\endpicture}
$$

%%% end of file circ.tex
represents $\phi$ and the dotted line represents
$\sigma$.
Note the $\phi$ propagator is massive due to the
shift in $\sigma$.
This diagram is evaluated by (5).

$$
$$



\end


