%From: Pierre Deligne <deligne@math.ias.edu>
%Date: Thu, 23 Jan 1997 17:51:44 -0500
%Subject: set6-1.tex

\input amstex
\documentstyle{amsppt}
\magnification=1200
\pagewidth{6.5 true in}
\pageheight{8.9 true in}
\loadeusm

\catcode`\@=11
\def\logo@{}
\catcode`\@=13

\NoRunningHeads

\input pictex.tex

\font\boldtitlefont=cmb10 scaled\magstep2
\font\bigsym=cmsy10 at 14.40pt 
\font\biggersym=cmsy10 scaled \magstep3

\def\dspace{\lineskip=2pt\baselineskip=18pt\lineskiplimit=0pt}
\def\wedgeop{\operatornamewithlimits{\wedge}\limits}
\def\w{{\mathchoice{\,{\scriptstyle\wedge}\,}
  {{\scriptstyle\wedge}}
  {{\scriptscriptstyle\wedge}}{{\scriptscriptstyle\wedge}}}}
\def\Le{{\mathchoice{\,{\scriptstyle\le}\,}
  {\,{\scriptstyle\le}\,}
  {\,{\scriptscriptstyle\le}\,}{\,{\scriptscriptstyle\le}\,}}}
\def\Ge{{\mathchoice{\,{\scriptstyle\ge}\,}
  {\,{\scriptstyle\ge}\,}
  {\,{\scriptscriptstyle\ge}\,}{\,{\scriptscriptstyle\ge}\,}}}
\def\vrulesub#1{\hbox{\vrule height7pt depth5pt\,}_{#1}}
\def\mapright#1{\smash{\mathop{\,\longrightarrow\,}%
     \limits^{#1}}}
\def\plus{{\sssize +}}
\def\upvee{{\sssize\vee}}
\def\underSim#1{\mathop{\vtop{\ialign{##\crcr
  \hfil {\bigsym\char'030}\hfil\crcr
  \noalign{\nointerlineskip}${\ssize #1}$\crcr
  \noalign{\nointerlineskip}\crcr}}}}

\def\eps{{\varepsilon}}
\def\Lam{{\Lambda}}
\def\mynabla{{\nabla\!}}

\def\xtil{\widetilde{x}}

\def\Adot{\Dot{A}}
\def\Bdot{\Dot{B}}
\def\Xdot{\Dot{X}}
\def\xdot{\Dot{x}}
\def\psidot{\Dot{\psi}}

\def\dbR{{\Bbb R}}

%\def\GL{\text{\rm GL}} \def\Hom{\text{\rm Hom}}
%\def\aff{\text{\rm aff}} \def\red{\text{\rm red}}
%\def\Map{\text{\rm Map}} \def\Lie{\text{\rm Lie}}
%\def\Diff{\text{\rm Diff}} \def\Vol{\text{\rm Vol}}
%\def\Tr{\text{\rm Tr}}
\def\I{\text{\rm I}}
\def\II{\text{\rm II}}
\def\III{\text{\rm III}}


\def\scr#1{{\fam\eusmfam\relax#1}}

\def\scrA{{\scr A}}   \def\scrB{{\scr B}}
\def\scrC{{\scr C}}   \def\scrD{{\scr D}}
\def\scrE{{\scr E}}   \def\scrF{{\scr F}}
\def\scrG{{\scr G}}   \def\scrH{{\scr H}}
\def\scrI{{\scr I}}   \def\scrJ{{\scr J}}
\def\scrK{{\scr K}}   \def\scrL{{\scr L}}
\def\scrM{{\scr M}}   \def\scrN{{\scr N}}
\def\scrO{{\scr O}}   \def\scrP{{\scr P}}
\def\scrQ{{\scr Q}}   \def\scrR{{\scr R}}
\def\scrS{{\scr S}}   \def\scrT{{\scr T}}
\def\scrU{{\scr U}}   \def\scrV{{\scr V}}
\def\scrW{{\scr W}}   \def\scrX{{\scr X}}
\def\scrY{{\scr Y}}   \def\scrZ{{\scr Z}}


\document
\line{{\boldtitlefont Witten's Problems}, Set Six --- 
N$^{\text{o}}$. 1 
\hfill (solution written by D. Freed)}
\smallskip
\hbox to \hsize{\hrulefill}

\bigskip\noindent
{\bf Problem:} 

\smallskip
(1) In free scalar field theory in four dimensions,
with mass $m$, 
calculate the operator product expansion
$\phi^2(x)\phi^2(x')|_{x\to x'}$ up to and including
all operators of dimension four or less.  (This means,
relative to what we did in class, working out contributions
of some operators with derivatives.  This should not be a long
exercise.)


\bigskip\bigskip\noindent
{\bf Solution:}

\smallskip
We represent $\phi^2(x)$ by $\bigwedge\limits^{{\overset
x\to{\sssize\otimes}}}$ 
and then there are 3 diagrams in $\phi^2(x) \phi^2(x')$:
$$
\spreadmatrixlines{1\jot}
\matrix
\operatornamewithlimits{\hbox{\biggersym\char'136}}
\limits^{{\overset x\to{\ssize\otimes}}}\quad
\operatornamewithlimits{\hbox{\biggersym\char'136}}
\limits^{{\overset \,\,x'\to{\ssize\otimes}}}
&\qquad \vbox{\beginpicture
\setcoordinatesystem units < .50cm, .50cm>
\setlinear
%
% Fig POLYLINE object
%
\linethickness= 0.500pt
\plot  8.922 19.494  9.207 20.130 /
%
% Fig POLYLINE object
%
\linethickness= 0.500pt
\putrule from  9.557 20.320 to 10.859 20.320
%
% Fig POLYLINE object
%
\linethickness= 0.500pt
\plot 11.113 20.066 11.398 19.431 /
%
% Fig TEXT object
%
\put{\null\kern-2pt$\ssize\otimes$} [lB] at 10.986 20.161
%
% Fig TEXT object
%
\put{$\ssize\otimes$} [lB] at  9.112 20.193
%
% Fig TEXT object
%
\put{$\ssize x$} [lB] at  9.207 20.574
%
% Fig TEXT object
%
\put{\null\kern-2pt$\ssize x'$} [lB] at 11.081 20.574
\linethickness=0pt
\putrectangle corners at  8.922 21.082 and 11.398 19.431
\endpicture}
&\qquad \vbox{\beginpicture
\setcoordinatesystem units < .50cm, .50cm>
\setlinear
%
% Fig CIRCULAR ARC object
%
\linethickness= 0.500pt
\circulararc 65.352 degrees from 11.303 18.796 center at 10.430 17.435
%
% Fig CIRCULAR ARC object
%
\linethickness= 0.500pt
\circulararc 65.352 degrees from  9.557 18.224 center at 10.430 19.586
%
% Fig TEXT obje-t
%
\put{\null\kern-3pt $\ssize\otimes$} [lB] at  9.493 18.383
%
% Fig TEXT object
%
\put{$\ssize\otimes$} [lB] at 11.113 18.383
%
% Fig TEXT object
%
\put{\null\kern-1pt $\ssize x$} [lB] at  9.081 18.351
%
% Fig TEXT object
%
\put{$\ssize x'$} [lB] at 11.494 18.383
\linethickness=0pt
\putrectangle corners at  9.081 19.114 and 11.494 17.939
\endpicture}\\
\I &\qquad \II &\qquad \III
\endmatrix
$$


\noindent
The first diagram is the ordinary Taylor series, which to
operators of dimension four is simply
$$
I=\phi(x)^4+\cdots
$$
The second diagram occurs 4 times and is
$$
\align
\II &=4G(x-x')\phi(x)\phi(x')\\
&=4G(x-x')\Bigl[\phi(x)^2+(x'-x)^\mu\phi(x)\partial_\mu
  \phi(x)+\tfrac12(x'-x)^\mu(x'-x)^\nu
  \phi(x)\partial_\mu\partial_\nu
  \phi(x)\\
&\null\kern10.0 true cm +\cdots\Bigr].
\endalign
$$
The third diagram occurs twice and is simply
$$
\III=2G(x-x')^2.
$$
Since $G(x-x')\sim\frac{1}{\vert x-x'\vert^2}$ as $x'\to
x$, we have the operator product expansion up to
operators of dimension four:
$$
\align
\phi^2(x)\phi^2(x') \underSim{x'\to x}
\tfrac{2}{\vert x-x'\vert^4} &+\tfrac{4\phi(x)^2}{\vert
x-x'\vert^2}+4\tfrac{(x-x')^\mu}{\vert x-x'\vert^2}
  \phi(x)\partial_\mu\phi(x)\\
&+\left\{\phi(x)^4+4\tfrac{(x-x')^\mu(x-x')^\nu}{\vert
x-x'\vert^2}\phi(x)\partial_\mu\partial_\nu\phi(x)\right\}\\
&+\cdots
\endalign
$$

\enddocument




