%% This is a plain TeX file
%%
\magnification=1200
\hsize=6.5 true in
\vsize=8.7 true in
\input epsf.tex

\input amssym.def
\input amssym.tex

\font\dotless=cmr10 %for the roman i or j to be
                    %used with accents on top.
                    %(\dotless\char'020=i)
                    %(\dotless\char'021=j)
\font\itdotless=cmti10
\def\itumi{{\"{\itdotless\char'020}}}
\def\itumj{{\"{\itdotless\char'021}}}
\def\umi{{\"{\dotless\char'020}}}
\def\umj{{\"{\dotless\char'021}}}
\font\smaller=cmr5
\font\boldtitlefont=cmb10 scaled\magstep2
\font\smallboldtitle=cmb10 scaled \magstep1
\font\ninerm=cmr9
\font\dun=cmdunh10 %scaled\magstep1
\font\Rfont=cmss10

\footline={\hfil {\tenrm VIII.\folio}\hfil}

\def\eps{{\varepsilon}}
\def\Eps{{\epsilon}}
\def\kap{{\kappa}}
\def\lam{{\lambda}}
\def\Lam{{\Lambda}}
\def\mynabla{{\nabla\!}}

\def\underNS{\underline{\NS}}
\def\underR{\underline{\R}}

\def\Bmu{{B_{\mu\nu}}}
\def\Gmu{{G_{\mu\nu}}}

\def\xdot{{\dot x}}
\def\xddot{{\ddot x}}

\def\undertext#1{$\underline{\vphantom{y}\hbox{#1}}$}
\def\nspace{\lineskip=1pt\baselineskip=12pt%
     \lineskiplimit=0pt}
\def\dspace{\lineskip=2pt\baselineskip=18pt%
     \lineskiplimit=0pt}

\def\half{\raise4.5pt\hbox{{\vtop{\ialign{##\crcr
  \hfil\rm $1$\hfil\crcr
   \noalign{\nointerlineskip}--\crcr
   \noalign{\nointerlineskip\vskip-1pt}$2$\crcr}}}}}
\def\third{\raise4.5pt\hbox{{\vtop{\ialign{##\crcr
  \hfil\rm $1$\hfil\crcr
  \noalign{\nointerlineskip}--\crcr
  \noalign{\nointerlineskip\vskip-1pt}$3$\crcr}}}}}
\def\fourth{\raise4.5pt\hbox{{\vtop{\ialign{##\crcr
  \hfil\rm $1$\hfil\crcr
  \noalign{\nointerlineskip}--\crcr
  \noalign{\nointerlineskip\vskip-1pt}$4$\crcr}}}}}
\def\sixth{\raise4.5pt\hbox{{\vtop{\ialign{##\crcr
  \hfil\rm $1$\hfil\crcr
  \noalign{\nointerlineskip}--\crcr
  \noalign{\nointerlineskip\vskip-1pt}$6$\crcr}}}}}
\def\eighth{\raise4.5pt\hbox{{\vtop{\ialign{##\crcr
  \hfil\rm $1$\hfil\crcr
  \noalign{\nointerlineskip}--\crcr
  \noalign{\nointerlineskip\vskip-1pt}$8$\crcr}}}}}

\def\oplusop{\mathop{\oplus}\limits}
\def\w{{\mathchoice{\,{\scriptstyle\wedge}\,}
  {{\scriptstyle\wedge}}
  {{\scriptscriptstyle\wedge}}{{\scriptscriptstyle\wedge}}}}
\def\Le{{\mathchoice{\,{\scriptstyle\le}\,}
{\,{\scriptstyle\le}\,}
{\,{\scriptscriptstyle\le}\,}{\,{\scriptscriptstyle\le}\,}}}
\def\Ge{{\mathchoice{\,{\scriptstyle\ge}\,}
{\,{\scriptstyle\ge}\,}
{\,{\scriptscriptstyle\ge}\,}{\,{\scriptscriptstyle\ge}\,}}}
\def\plus{{\hbox{$\scriptscriptstyle +$}}}
\def\xdot{\dot{x}}
\def\Condition#1{\item{#1}}
\def\Firstcondition#1{\hangindent\parindent{#1}\enspace
     \ignorespaces}
\def\Proclaim#1{\medbreak
  \medskip\noindent{\bf#1\enspace}\it\ignorespaces}
  %the way to use this is:
  %"\Proclaim{Theorem 1.1.}" for instance.
\def\finishproclaim{\par\rm
     \ifdim\lastskip<\smallskipamount\removelastskip
     \penalty55\medskip\fi}
\def\Item#1{\par\smallskip\hang\indent%
  \llap{\hbox to\parindent {#1\hfill\enspace}}\ignorespaces}
\def\ItemItem#1{\par\indent\hangindent2\parindent
     \hbox to \parindent{#1\hfill\enspace}\ignorespaces}
\def\vrulesub#1{{\,\vrule height7pt depth5pt}_{\,#1}}
\def\underbrake#1#2{\mathop{#1}\limits_{\raise3pt
  \hbox{%
\vrule height 3pt depth 0pt
  %\kern.1pt
  \hbox to #2{\hrulefill}
  \kern-3.4pt
  \vrule height 3pt depth 0pt}}}

\def\ominus{{$-$\kern-9pt $\bigcirc$}}
\def\Oplus{{+\kern-9pt $\bigcirc$}}
\def\ssbullet{{\scriptstyle\bullet\,\,\,}}

%\def\im{{\rm Im}}  
\def\A{{\rm A}}
\def\P{{\rm P}}
\def\EL{{\rm L}}
\def\Open{{\rm open}}
\def\osc{{\rm osc}}
%\def\Pic{{\rm Pic}} 
\def\Sp{{\rm Sp}}
\def\R{{\rm R}}  \def\NS{{\rm NS}}
\def\RNS{{\rm RNS}}
\def\Diff{{\rm Diff}}  \def\expt{{\rm expt}}
\def\cm{{\rm cm}}  \def\annulus{{\rm annulus}}
\def\cylinder{{\rm cylinder}}
\def\Closed{{\rm closed}}
\def\Map{{\rm Map}}  
%\def\spurious{{\rm spurious}}
\def\Met{{\rm Met}} 
\def\Spin{{\rm Spin}}
\def\spin{{\rm spin}} 
\def\phys{{\rm phys}}
\def\diag{{\rm diag}}  
%\def\Vir{{\rm Vir}}
%\def\Res{{\rm Res}}
\def\Null{{\rm null}} 
\def\mass{{\rm mass}}
\def\SO{{\rm SO}} 
\def\GSO{{\rm GSO}}
\def\Tr{{\rm Tr\,}}
%\def\SU{{\rm SU}} 
\def\tr{{\rm tr}}
\def\Weyl{{\rm Weyl}} 
\def\Lorentz{{\rm Lorentz}}
\def\Ker{{\rm Ker}}
\def\Range{{\rm Range}} 
%\def\SL{{\rm SL}} \def\prim{{\rm primitive}}
\def\Det{{\rm Det}} 
%\def\re{{\rm Re}}
%\def\dist{{\rm dist}} \def\PSL{{\rm PSL}}
%\def\Vol{{\rm Vol}} \def\ghosts{{\rm ghosts}}
%\def\Fock{{\rm Fock}} \def\BRST{{\rm BRST}}
%\def\Weil-Peterson{{\rm Weil-Peterson}}
\def\IIA{{\rm II~A}}
\def\IIB{{\rm II~B}}

%\def\fbar{\bar{f}}  \def\mubar{\bar{\mu}}
%\def\barh{\bar{h}}  \def\gammabar{\bar{\gamma}}
%\def\kbar{\bar{k}}  
\def\lambar{\bar{\lambda}}
\def\mbar{\bar{m}}   \def\Lambar{\bar{\Lambda}}
%\def\phibar{\bar{\phi}}
\def\wbar{\bar{w}}  
%\def\etabar{\bar{\eta}}
%\def\vbar{\bar{v}}  \def\partialbar{\bar{\partial}}
%\def\xbar{\bar{x}}  
%\def\cbar{\bar{c}}
%\def\bbar{\bar{b}}
%\def\epsbar{\bar{\epsilon}}
\def\Gammabar{\bar{\Gamma}}
\def\rbar{\bar{r}}
\def\zbar{\bar{z}}  
%\def\Abar{\bar{A}}
\def\Gbar{\bar{G}}
\def\Kbar{\bar{K}}
%\def\Pbar{\bar{P}}
\def\Sbar{\bar{S}}
\def\Tbar{\bar{T}}

%\def\scrFbc{{\scrF^{(bc)}}}
%\def\scrFbcbar{{\scrF^{(\bbar\cbar)}}}

%\def\ghat{\hat{g}}
%\def\muhat{\hat{\mu}}

\def\Ftil{\widetilde{F}}
\def\atil{\tilde{a}}
\def\btil{\tilde{b}}
\def\dtil{\tilde{d}}
\def\xtil{\tilde{x}}
%\def\htil{\tilde{h}}
%\def\Ctil{\widetilde{C}}
%\def\Dtil{\widetilde{D}}
\def\Ltil{\tilde{L}}
\def\Ntil{\widetilde{N}}
\def\Ptil{\widetilde{P}}
\def\betatil{\tilde{\beta}}
%\def\Ttil{\widetilde{T}}
\def\scrFtil{\widetilde{\scrF}}
%\def\epstil{\tilde{\eps}}
%\def\psitil{\tilde{\psi}}

\def\dbR{{\Bbb R}}
%\def\dbZ{{\Bbb Z}}

%These two files (in this order!!) are necessary
%in order to use AMS Fonts 2.0 with Plain TeX

\input amssym.def
\input amssym.tex

%Capital roman double letters(Blackboard bold)
\def\db#1{{\fam\msbfam\relax#1}}

\def\dbA{{\db A}} \def\dbB{{\db B}}
\def\dbC{{\db C}} \def\dbD{{\db D}}
\def\dbE{{\db E}} \def\dbF{{\db F}}
\def\dbG{{\db G}} \def\dbH{{\db H}}
\def\dbI{{\db I}} \def\dbJ{{\db J}}
\def\dbK{{\db K}} \def\dbL{{\db L}}
\def\dbM{{\db M}} \def\dbN{{\db N}}
\def\dbO{{\db O}} \def\dbP{{\db P}}
\def\dbQ{{\db Q}} \def\dbR{{\db R}}
\def\dbS{{\db S}} \def\dbT{{\db T}}
\def\dbU{{\db U}} \def\dbV{{\db V}}
\def\dbW{{\db W}} \def\dbX{{\db X}}
\def\dbY{{\db Y}} \def\dbZ{{\db Z}}

\font\teneusm=eusm10  \font\seveneusm=eusm7 
\font\fiveeusm=eusm5 
\newfam\eusmfam 
\textfont\eusmfam=\teneusm 
\scriptfont\eusmfam=\seveneusm 
\scriptscriptfont\eusmfam=\fiveeufm 
\def\scr#1{{\fam\eusmfam\relax#1}}


%Upper-case Script Letters:

\def\scrA{{\scr A}}   \def\scrB{{\scr B}}
\def\scrC{{\scr C}}   \def\scrD{{\scr D}}
\def\scrE{{\scr E}}   \def\scrF{{\scr F}}
\def\scrG{{\scr G}}   \def\scrH{{\scr H}}
\def\scrI{{\scr I}}   \def\scrJ{{\scr J}}
\def\scrK{{\scr K}}   \def\scrL{{\scr L}}
\def\scrM{{\scr M}}   \def\scrN{{\scr N}}
\def\scrO{{\scr O}}   \def\scrP{{\scr P}}
\def\scrQ{{\scr Q}}   \def\scrR{{\scr R}}
\def\scrS{{\scr S}}   \def\scrT{{\scr T}}
\def\scrU{{\scr U}}   \def\scrV{{\scr V}}
\def\scrW{{\scr W}}   \def\scrX{{\scr X}}
\def\scrY{{\scr Y}}   \def\scrZ{{\scr Z}}

\def\gr#1{{\fam\eufmfam\relax#1}}

%Euler Fraktur letters (German)
\def\grA{{\gr A}}	\def\gra{{\gr a}}
\def\grB{{\gr B}}	\def\grb{{\gr b}}
\def\grC{{\gr C}}	\def\grc{{\gr c}}
\def\grD{{\gr D}}	\def\grd{{\gr d}}
\def\grE{{\gr E}}	\def\gre{{\gr e}}
\def\grF{{\gr F}}	\def\grf{{\gr f}}
\def\grG{{\gr G}}	\def\grg{{\gr g}}
\def\grH{{\gr H}}	\def\grh{{\gr h}}
\def\grI{{\gr I}}	\def\gri{{\gr i}}
\def\grJ{{\gr J}}	\def\grj{{\gr j}}
\def\grK{{\gr K}}	\def\grk{{\gr k}}
\def\grL{{\gr L}}	\def\grl{{\gr l}}
\def\grM{{\gr M}}	\def\grm{{\gr m}}
\def\grN{{\gr N}}	\def\grn{{\gr n}}
\def\grO{{\gr O}}	\def\gro{{\gr o}}
\def\grP{{\gr P}}	\def\grp{{\gr p}}
\def\grQ{{\gr Q}}	\def\grq{{\gr q}}
\def\grR{{\gr R}}	\def\grr{{\gr r}}
\def\grS{{\gr S}}	\def\grs{{\gr s}}
\def\grT{{\gr T}}	\def\grt{{\gr t}}
\def\grU{{\gr U}}	\def\gru{{\gr u}}
\def\grV{{\gr V}}	\def\grv{{\gr v}}
\def\grW{{\gr W}}	\def\grw{{\gr w}}
\def\grX{{\gr X}}	\def\grx{{\gr x}}
\def\grY{{\gr Y}}	\def\gry{{\gr y}}
\def\grZ{{\gr Z}}	\def\grz{{\gr z}}

\overfullrule=0pt

\parindent=25pt
\line{\dun --- DRAFT --- \hfill{\rm IASSNS-HEP-97/72}}

\bigskip\bigskip
\centerline{\boldtitlefont Lecture 13}
\bigskip
\centerline{\smallboldtitle VIII. Heterotic Strings}

\medskip
\centerline{Eric D'Hoker}

\frenchspacing

\dspace
\bigskip
So far, the only method we have discussed for
obtaining massless Yang-Mills states was via the
Chan-Paton rules in open string theory, where gauge
charges were appended to the end points of open
strings.

Heterotic string theories utilize a second method for
obtaining massless Yang-Mills states.
Heterotic string theories are constructed from {\it
closed oriented strings} only. 
Internal degrees of
freedom --- required to obtain Yang-Mills states ---
are provided by degrees of freedom that live in the
bulk of the worldsheet.
The heterotic string construction relies
on the independence of the
oscillator modes of left and right moving sectors on
closed oriented string worldsheets (the
left and right momenta obey certain relations that couple
left and right degress of freedom). 
The left moving sector of the heterotic string is
built out of the (left-moving sector of the) Type II
superstring, while the right moving sector of the
heterotic string is built out of the (right-moving
sector of the) bosonic string.

At first sight, this kind of construction would appear
inconsistent.
The Type II superstring part (left movers) exhibits
$\scrN=1$ worldsheet supersymmetry, and $\scrN=1$
super-Virasoro invariance and has critical
space-time dimension $D=10$, while the bosonic string
part (right-movers) exhibits only Virasoro invariance,
and has critical space-time dimension $D=26$.
This mismatch of dimensions is put to excellent
use in the following way.
We let the heterotic string live in $D=10$ space-time
dimensions, and we let the remaining $16$ dimensions
of right movers emerge as internal degrees of freedom,
which are to be metamorphosed into Yang-Mills degrees of
freedom.

This system of $16$ free internal right movers forms a
conformal field theory of central charge $16$.
It is a unitary CFT, since the $16$ scalar fields
emerge out of this construction with positive definite
internal metric.
(We shall later establish that the 16 scalars actually
live on a torus, with constant internal metric.)
An alternative way of seeing that this CFT must be
unitary is by the fact that in the right moving sector,
the Virasoro algebra has already been used up to
eliminate the negative norm states from the
$10$-dimensional space-time part of the $x$-fields.
Consistency of the construction will only be possible
if the remaining internal part of the $x$-fields 
is unitary all by itself.

Poincar\'e invariance is required to hold only in the
$10$-dimensional subspace of the bosonic right movers
which is naturally complemented 
by the $10$-dimensional space of the bosonic left movers
from the superstring.
The internal degrees of freedom in the right moving
sector are invariant under this $10$-dimensional
Poincar\'e symmetry, and are thus not constrained to be
free fields on $\dbR^{16}$.
In fact, it is consistent with conformal invariance in
the right moving sector to consider a system of
internal degrees of freedom described by a general,
unitary conformal field theory of central charge $16$.

On worldsheets of general topology, the separation of
left and right movers, which is used in the very
definition of the heterotic string, leads to anomalies
in each sector separately under diffeomorphisms of
$\Sigma$.
Actually, the subgroup of diffeomorphisms connected to
the identity, $\Diff_0(\Sigma)$, is anomaly free as
long as the combined central charge of matter and
ghosts vanishes in each sector.
However, global diffeomorphisms, belonging to the
modular group $\Sp(2h;\dbZ)$ in general lead to global
anomalies.
Unitarity of the heterotic string (on worldsheets of
genus $h\Ge 1$) will require that such anomalies
cancel, and this will impose severe restrictions on
the allowed $c=16$ CFT's that can govern the right
moving sector of internal degrees of freedom.
In fact, only gauge groups $E_8\times E_8$ and
$\Spin(32)/\dbZ_2$ will be allowed, as we shall show.

The benefits of the heterotic string construction
should be quite evident.
Due to space-time supersymmetry of the left movers, we
automatically have a string theory with space-time
fermions, but without a tachyon!
In addition, the theory will possess large --- and
phenomenologically interesting --- gauge symmetries.
We now begin the construction of the heterotic string.

First, we shall investigate unitary $c=16$ conformal
field theories involving free fields only.
Using Bose-Fermi correspondence, free fields may be
formulated either in terms of $16$ real scalar fields
$x^I$, $I=1,\ldots,16$, which map $\Sigma$ into a flat
torus $\dbR^{16}/\Lam$, for some lattice $\Lam$, with
Euclidean signature metric, or in terms of $32$
Majorana spinors $\lam^\alpha$, $\alpha=1,\ldots,32$.
Of course, we shall have to retain only the
right-moving sector of these degrees of freedom.
At the level of conformal field theory operators, one
may simply retain the right moving operators, while at
the level of transition amplitudes, we shall make use of
a generalization of the
chiral splitting theorem, established in \S{VI}.

The actions in local complex coordinates $z$, $\zbar$
are given by
$$
\eqalignno{
S_I[x^I,g] &\equiv{1\over 4\pi}\int\nolimits_{\sigma}
d^2 z\partial_z x^I\partial_{\zbar}x^I&(8.1)\cr
S_I[\lam^\alpha,g] &\equiv{1\over 4\pi}\int\nolimits_{\Sigma}
d^2z(\lam_+^\alpha\partial_{\zbar}\lam_+^\alpha+
\lam_-^\alpha\partial_z \lam_-^\alpha)&(8.2)\cr}
$$
Here, we represent Majorana spinors by a complex Weyl
spinor $\lam_+^\alpha$ and its complex conjugate
$\lam_-^\alpha=(\lam_+^\alpha)^*$.
Summation over repeated indices $I$ and $\alpha$ is
assumed throughout.

\bigskip\noindent
\Item{\bf A)} {\bf Free Fermion Realization of
Internal Degrees of Freedom}

We start by analyzing the local properties (on
$\Sigma$) of the conformal field theory described by
$S_I[\lam,g]$.
We shall discuss the global issues, such as the
spin structure of $\lam^\alpha$, shortly.
We restrict attention to $\lam_-^\alpha$, since
$\lam_+^\alpha$ may be gotten by complex
conjugation.

The field equations $\partial_{z}\lam_-^\alpha=0$
imply that $\lam_-^\alpha(\zbar)$ is anti-holomorphic, 
and the OPE is given as usual.
(We assume that all $\lam_-^\alpha$ have the same spin
structure.)
$$
\lam_-^\alpha(\zbar)\lam_-^\beta(\wbar)
\sim{\delta^{\alpha\beta} \over \zbar-\wbar}
\eqno{(8.A.1)}
$$
The stress tensor is
$$
T_{\zbar\zbar}=\half\,\partial_{\zbar}
\lam_-^\alpha\lam_-^\alpha
\eqno{(8.A.2)}
$$
The action $S_I$ is also invariant under $\SO(32)$
rotations of the spinor $\lam_-^\alpha$.
We denote the structure constants of $\SO(32)$ by
$f^{abc}$ and the representation matrices in the
fundamental ($32$-dimensional) representation of $\SO(32)$
by $t^a$, $a,b,c=1,\ldots,\dim\,\SO(32)=496$:
$$
[t^a,t^b]=if^{abc}t^c\qquad\qquad
\tr\,t^at^b=+\,\half\,\delta_{ab}\,\,.
\eqno{(8.A.3)}
$$
The $\SO(32)$ Noether current $J_{\zbar}^a$ is 
anti-analytic and given by
$$
J_{\zbar}^a(\zbar)
=(t^a)_{\alpha\beta}\lam_-^\alpha\lam_-^\beta(\zbar)
\eqno{(8.A.4)}
$$
and satisfies the following OPE:
$$
J_{\zbar}^a(\zbar)J_{\zbar}^b
(\wbar)\sim{k\delta^{ab}\over (\zbar-\wbar)^2}+
i{f^{abc}\over \zbar-\wbar}\,J_{\zbar}^c(\wbar)\,\,.
\eqno{(8.A.5)}
$$
Here, we have
 $k\delta^{ab}\equiv 2\,\tr\,t^at^b$ and thus $k=1$.
Notice that this OPE is precisely the one encountered
in Problem Set \#8.
Here again, the stress tensor of (8.A.2) may be expressed in
terms of the $\SO(32)$ current by the Sugawara
construction:
$$
T_{\zbar\zbar}={1\over 2\kap}\,: J_{\zbar}^aJ_{\zbar}^a :
\eqno{(8.A.6)}
$$
with $\kap=+{1\over 2} C_2(G)+k$, where $C_2(G)$ is the
quadratic Casimir of the group $G$.
The central charge is given by
$$
c={2k.\dim\,G\over 2k+C_2(G)}\,\,.\eqno{(8.A.7)}
$$
For $G=\SO(32)$, $\dim\,G=496$, $C_2(G)=60$ and we
have $k=1$.
(Witten established that Wess-Zumino-Witten models
with $k=1$ are equivalent to free fermions for
general $G$.)

We now make use of this free fermion model to
introduce internal degrees of freedom in the
right-moving sector of the heterotic string.
Thus, we retain from $\lam^\alpha$ only
$\lam_-^\alpha$.
We begin by investigating free heterotic strings, and
work on a worldsheet with the topology of an annulus
(equivalently a cylinder), where two spin structures
are possible.
We denote these by $\P$ and $\A$, and define them in
analogy with $\R$ and $\NS$ spin structures:
$$
\matrix{\displaystyle
 \P:\quad  
  \lam_-^\alpha &\hbox{ periodic on cylinder}\hfill
  &\qquad \hbox{(cf. $\R$-sector)}\hfill\cr
\noalign{\medskip}
\A:\quad \lam_-^\alpha
  &\hbox{ anti-periodic on cylinder}\hfill
   &\qquad  \hbox{(cf. $\NS$-sector)}\hfill\cr}
\eqno{(8.A.8)}
$$

A priori, not all $\lam_-^\alpha$ have to carry the
same spin structures.
If all $\lam_-^\alpha$ carry the same spin
structure, then the free fermion theory will be
invariant under $\SO(32)$.
More generally, one may consider spinors $\lam_-^\alpha$
belonging to different groups, where the spin structure
in each group is independently chosen.
For example if $\lam_-^\alpha$ fall into 2 groups of
size $k$ and $32-k$, with $1\Le k\Le 31$,
respectively, then the remaining symmetry will be
$\SO(k)\times\SO(32-k)$, with four possible
combinations of spin structures; $\P\P$, $\A\P$, $\P\A$,
$\A\A$.
We shall investigate these possibilities shortly.

The mode expansions of $\lam_-^\alpha(\zbar)$ are
given in terms of integer $(\P)$ or integer
$+{1\over 2}\,\,(\A)$ modes:
$$
\matrix{\displaystyle
\P: &\qquad\qquad\qquad \lam_-^\alpha(\zbar)
=\sum\limits_{n\in\dbZ}\lam_n^\alpha\,\,\zbar^{-n-1/2}\hfill\cr
\noalign{\medskip}
\A: &\qquad\qquad\qquad
  \lam_-^\alpha(\zbar) =\sum\limits_{r\in{1\over 2}\,+\dbZ}
  \lam_r^\alpha\,\,\zbar^{-r-1/2} &\qquad\qquad\qquad & \hfill\cr}
\eqno{(8.A.9)}
$$
with
$$
\matrix{\displaystyle
\null\,\,\,\P: &\qquad\qquad \{\lam_m^\alpha,\lam_n^\beta\}
=\delta^{\alpha\beta}\delta_{m+n,0} &m,n\in\dbZ
&\qquad\qquad&\cr
\noalign{\medskip}
\null\,\,\,\A: &\qquad\qquad 
  \{\lam_r^\alpha,\lam_s^\beta\} =\delta^{\alpha\beta}
  \delta_{r+s,0} &r,s\in{1\over 2}+\dbZ
&\qquad\qquad&\hfill\cr}
\eqno{(8.A.10)}
$$

As is expected in analogy with the $R$ sector, there
are zero modes in the $\P$ sector, generated by 
$\lam_0^\alpha$ and
obeying the Clifford algebra of $\SO(32)$.
As a result, the $\P$-ground state is a $\Spin(32)$
spinor, which we shall denote by 
$\left.\vert\sigma;\P\right>_R$,
with $\sigma$ running over the weights of the
$\Spin(32)$ spinor representation (we shall see later
that this spinor must be Weyl), and
$\lam_n^\alpha\left.\vert\sigma;\P\right>_R=0$, for
all $n\in\dbN$.
(If several independent groups of $\lam_-^\alpha$'s
arise, $\SO(32)$ is reduced accordingly.)
The $\P$ sector produces a Hilbert space
$$
\scrF^\P\equiv\left\{\{\lam_{-n}^\alpha\}_{n\in\dbN}
\left.\vert\sigma;\P\right>_R\right\}\,\,.
\eqno{(8.A.11)}
$$
The $\A$ sector has a unique ground state denoted
by $\left.\vert0,\A\right>_R$ and obeying
$\lam_r^\alpha\left.\vert0,A\right>_R=0$ for all
$r\in-\half\,+\dbN$, and produces a Hilbert space
$$
\scrF^\A\equiv\left\{\{\lam_{-r}^\alpha\}_{r\in-{1\over 2}
  +\dbN}
\left.\vert0,A\right>_R\right\}\,\,.
\eqno{(8.A.12)}
$$

The Virasoro generators are given as usual, and we
shall only need $L_0$ and $\Ltil_0$ of these.
We define those generators by normal ordering as
usual:
$$
L_0=\half\,p^2+N;\qquad\qquad
\Ltil_0=\half\,p^2+\Ntil
\eqno{(8.A.13)}
$$
where the number operators are given by
$$
\matrix{\displaystyle
\R: &\qquad\qquad
    N\equiv\sum\limits_{n=1}^\infty(x_{-n}^\mu x_{n\mu}
  +nd_{-n}^\mu d_{n\mu})\hfill \cr
\noalign{\medskip}
\NS: &\qquad\qquad
    N\equiv\sum\limits_{n=1}^\infty x_{-n}^\mu x_{n\mu}
  +\sum\limits_{r={1\over 2}}^\infty rb_{-r}^\mu b_{r\mu}
   \hfill\cr}\eqno{(8.A.14)}
$$ 
and
$$
\matrix{\displaystyle
\P: &\qquad\qquad
   \Ntil =\sum\limits_{n=1}^\infty (\xtil_{-n}^\mu
  \xtil_{n\mu}+n\lam_{-n}^\alpha\lam_n^\alpha)\hfill\cr
\noalign{\medskip}
\A: &\qquad\qquad
   \Ntil =\sum\limits_{n=1}^\infty \xtil_{-n}^\mu
  \xtil_{n\mu}+\sum\limits_{r=1/2}^\infty r\lam_{-r}^\alpha
  \lam_r^\alpha\hfill \cr}\eqno{(8.A.15)}
$$
It is understood in the above that the summations over
the $\alpha$-index run over the number of
$\lam^\alpha$'s that belong either to the $\P$ or $\A$
sectors.
We shall denote the eigenvalues in each sector by
$N_{\R}$, $N_{\NS}$, $N_{\P}$, $N_{\A}$.

\bigskip\noindent
\Item{\bf B)} {\bf Free Fermion Realization of the 
$\Spin(32)/Z_2$ Heterotic String}

We now assume that $S_I$ has $\SO(32)$ symmetry, so
that there are only $2$ sectors: either
all $\lam_-^\alpha$ are $\P$ or all $\lam_-^\alpha$
are $\A$.
The Virasoro conditions are easily worked out
$$
(L_0-a)\left.\vert\psi\right>_L=0\qquad\qquad
(\Ltil_0-\atil)\left.\vert\psi\right>_R=0
\eqno{(8.B.1)}
$$
The constants $a$ and $\atil$ are given
by:
$$
a=\left\{\eqalign{
0\,\,\, &\qquad \R\cr
1/2 &\qquad \NS\cr}\right.\qquad\qquad\qquad
\atil=\left\{\eqalign{\,\,1 &\qquad \A\cr
-1 &\qquad \P\cr}\right.
\eqno{(8.B.2)}
$$
The values for $a$ were obtained in \S{VII}.
The values for $\atil$ follow in complete analogy
with the corresponding values of $a$ in the
$\R$, $\NS$ sectors, namely $1/24$ for each
physical bosonic $x$, $-1/24$ for each integer moded
physical fermionic $\psi$ or $\lam$, and $+1/48$ for
each half integer moded physical fermionic $\psi$ or
$\lam$.

We are now in a position to determine the massless spectrum.
We have the following states
$$
\left(\eqalign{&N_\R=0\cr
&N_{\NS}=1/2\cr}\right)_{\!\!\!\! L}\otimes
\left(\eqalign{&N_\A=1\cr
&\hbox{no $\P$ states allowed}\cr}\right)_{\!\!\!\! R}
\eqno{(8.B.3)}
$$
This may be seen from the expression for $M^2$ in each
sector by combining (8.A.13--15) and (8.B.2)
$$
\matrix{M^2 &\!\!\!=2N_\R=2N_{\NS}-1\hfill
  &\kern2.0true cm\hbox{left-movers}\hfill\cr
\noalign{\bigskip}
&\!\!\!=2N_\A-2=2N_\P+2 &\kern2.0true cm
  \hbox{right-movers}\cr}
\eqno{(8.B.4)}
$$

For  reasons that will become fully apparent at string
loop amplitudes only, a GSO projection must be carried
out on the $\lam^\alpha$'s, just as it was required in
the $\R$ and $\NS$ sectors.
In the functional integral formulations, the GSO
summation will again correspond to a summation over
the spin structures of the worldsheet spinor
$\lam^\alpha$.

In the $\P$ sector, the projection is carried out with
the help of the $(-1)^F$ operator, defined by
$$
\eqalign{
(-1)^F &\equiv \lambar_0  (-1)^{F_P}\qquad
F_P\equiv\sum\limits_{n=1}^\infty 
\lam_{-n}^\alpha\lam_n^\alpha\cr
\lambar_0 &\equiv
\lam_0^1\lam_0^2\ldots\lam_0^{32}\cr}
\eqno{(8.B.5)}
$$
which anti-commutes with all $\lam_n^\alpha$.
We define the $\GSO$ projection on the Fock space of
$\lam_-^\alpha$ by requiring that only one eigenvalue
of $(-1)^F$ be retained.
Which eigenvalue is chosen is immaterial.
$$
\scrFtil_{\GSO}^\P\equiv\left\{\left.\vert\psi\right>_\R
\in\scrFtil^p\quad\hbox{such that}\quad
(-1)^F\left.\vert\psi\right>_\R=\left.\vert\psi\right>_\R
\right\}\,\,.
\eqno{(8.B.6)}
$$
In the $\A$ sector, the projection uses
$$
(-1)^F\equiv(-1)^{F_A}\qquad
F_A\equiv\sum\limits_{r=1/2}^\infty
\lam_{-r}^\alpha \lam_r^\alpha
\eqno{(8.B.7)}
$$
which also anti-commutes with all $\lam_{-r}^\alpha$.
We define the $\GSO$ projection on the Fock space of
$\lam_-^\alpha$ by requiring that only the $+$
eigenvalue of $(-1)^F$ is retained.
This guarantees that the $A$-ground state is retained
by the $\GSO$ projection.
$$
\scrFtil_{\GSO}^\A\equiv\left\{\left.\vert\psi\right>_\R
  \in\scrFtil^\A\quad\hbox{such that}\quad
  (-1)^F\left.\vert\psi\right>_\R=\left.\vert\psi
  \right>_\R\right\}\,\,.
\eqno{(8.B.8)}
$$
Notice that in the $\NS$ and $\A$ sectors, the $\GSO$
projection is compatible with the physical state conditions.
In the $\R$ and $\P$ sectors, the $\GSO$ projection
simply determines the chirality.

It is now easy to list the massless states.
In parentheses, we list the number of physical states
in each sector.

$$
\kern3.0true cm
\vbox{\offinterlineskip
\halign{\null\kern-50pt#\hfill &\qquad\qquad\qquad #\cr 
\hbox{Bosons} &$\NS$ sector only\qquad
  $N_{\NS}=1/2,\,\,N_A=1$\hfill\cr
\noalign{\bigskip\medskip (1)\enspace $b_{-1/2}^\mu
  \xtil_{-1}^\nu \left.\vert0,k;\NS\right>_{\EL}\otimes
  \left.\vert0,k\right>_\R
  \otimes\left.\vert0,\A\right>_\R$\bigskip}
&$\matrix{- &\hbox{graviton}\hfill &G_{\mu\nu} &(35)\cr
- &\hbox{anti-symmetric tensor}\hfill &B_{\mu\nu} &(28)\cr
- &\hbox{dilaton}\hfill &\Phi &(\hfill1)\cr}$ \cr
\noalign{\bigskip\medskip (2)\enspace $b_{-1/2}^\mu
  \lam_{-1/2}^\alpha\lam_{-1/2}^\alpha\left.\vert
  0,k;\NS\right>_{\EL}\otimes\left.\vert0,k\right>_\R
  \otimes \left.\vert0,\A\right>_\R$\bigskip}
&$\matrix{- &\SO(32) \hbox{ Yang-Mills field}
  &&(8\times 496)\cr
  &\hbox{(adjoint representation)} &&\cr}$\cr
\noalign{\bigskip}
\hbox{Fermions} &$\R$ sector only\qquad \qquad
     $\!\!\!N_\R=0$, $N_\A=1$\hfill\cr
\noalign{\bigskip\medskip (1)\enspace $\xtil_{-1}^\nu\left.
  \vert0,\alpha;\R\right>_{\EL}\otimes\left.\vert
  0,k\right>_\R\otimes\left.\vert0;\A\right>_\R$\bigskip}
&$\matrix{- &\hbox{gravitino}\hfill 
  &&\qquad\qquad\chi_\mu^\alpha(56)\cr
- &\hbox{dilatino}\hfill 
  &&\qquad\qquad\lam_\alpha(\hfill8)\cr}$\cr
\noalign{\bigskip (2)\enspace
$\lam_{-1/2}^\alpha\lam_{-1/2}^\beta
  \left.\vert0,\alpha;k;\R\right>_{\EL}\otimes\left.
  \vert0,k\right>_\R\otimes
\left.\vert0;\A\right>_\R$\hfill\bigskip}
&$\matrix{- &\SO(32) \hbox{ gaugino}\hfill 
     &\psi^\alpha(8\times 496)\cr
&\hbox{(adjoint representation)}\hfill &\cr}$\cr}}
$$

\medskip
Next, we examine the first massive level, obtained for
$M^2=2$:
$$
\left(\eqalign{
&N_\R=1\cr
&N_{\NS}=3/2\cr}\right)_{\!\!{\EL}}\otimes
\left(\eqalign{
&N_\A=2\cr
&N_\P=0\cr}\right)_{\!\!\R}
\eqno{(8.B.9)}
$$
The $N_\R=1$ multiplet of left-movers has the structure
$$
\eqalign{
&x_{-1}^\mu\left.\vert0,k,\alpha,\R\right>_{\EL}\cr
&d_{-1}^\mu\left.\vert0,k,\alpha;\R\right>_{\EL}\cr}
\eqno{(8.B.10)}
$$
and spans a massive Majorana fermion multiplet, with
$2\times 8\times 8=128$ physical states.
The $N_{\NS}=3/2$ multiplet of left-movers has the
following states
$$
\matrix{
x_{-1}^\mu b_{-1/2}^\nu\left.\vert0,k,\NS\right>_{\EL}\hfill
  &(64)\cr
\noalign{\bigskip}
b_{-1/2}^\mu b_{-1/2}^\nu
b_{-1/2}^\rho\left.\vert0,k;\NS\right>_{\EL}\hfill 
&(56)\cr
\noalign{\bigskip}
b_{-3/2}^\mu\left.\vert0,k;\NS\right>_{\EL}\hfill &(8)\cr}
\eqno{(8.B.11)}
$$
and has $128$ physical bosonic states.
Thus, the number of fermionic and bosonic physical
states in the left-moving sector are equal, as is
expected from space-time supersymmetry, as discussed
also in \S{VII},
The $N_\A=2$ multiplet of right-movers has the
following states
$$
\matrix{
\xtil_{-1}^\mu \xtil_{-1}^\nu\left.\vert0,k;\A\right>_\R\hfill
  &(36)\cr
\noalign{\bigskip}
\xtil_{-2}^\mu\left.\vert0,k;\A\right>_\R\hfill &(8)\cr
\noalign{\bigskip}
\xtil_{-1}^\mu\lam_{-1/2}^\alpha\lam_{-1/2}^\beta\left.\vert
  0,k;A\right>_\R\hfill &(8\times 496)\cr
\noalign{\bigskip}
\lam_{-1/2}^\alpha \lam_{-1/2}^\beta
\lam_{-1/2}^\gamma \lam_{-1/2}^\delta\left.\vert
 0,k;\A\right>_\R\hfill &(32\times 31\times 30\times 29/24)\cr
\noalign{\bigskip}
\lam_{-1/2}^\alpha \lam_{-3/2}^\beta\left.\vert
  0,k;\A\right>_\R\hfill &(32\times 32)\cr}
\eqno{(8.B.12)}
$$
with a total of 41016 states.
Finally, the novel sector here is the $\P$ sector,
whose analogue did not occur at the massless level.
It produces a single Weyl spinor of $\Spin(32)$ with
dimension $2^{15}=32768$.

Now, consider the representations allowed in the
spectrum:

\smallskip
\itemitem{$\scriptstyle\bullet$}
only $1$ Weyl spinor of $\Spin(32)$;

\smallskip
\itemitem{$\scriptstyle\bullet$}
even numbers of $\lam$'s applied to ground state.

\smallskip\noindent
Thus, we obtain representations not of $\Spin(32)$,
but only of $\Spin(32)/Z_2$.

Physically, the spinor representation of
$\Spin(32)/Z_2$ is a very interesting object: it is a
massive string state, but it is {\it stable} (in the
$10$-dimensional theory) since there are no massless
states that are spinors, and since a spinor cannot
decay into only vector representations.

\bigskip\noindent
\Item{\bf C)} {\bf Free Fermion Realization of the
$E_8\times E_8$ Heterotic String}

In the free fermion construction of the
$\Spin(32)/Z_2$ theory, we have restricted
$\lam_-^\alpha$'s to have either all $\P$ or all $\A$
spin structure in keeping with $\SO(32)$ invariance.
We now investigate in what way this restriction may be
relaxed, and we shall find one other consistent string
theory: the $E_8\times E_8$ model.

We assume that $\lam^\alpha$'s fall into two
groups of sizes $k$ and $32-k$ respectively and
$0<k<32$, with all $\lam^\alpha$'s in each group carrying
the same spin structure, either $\A$ or $\P$.
The manifest symmetry of the action $S_I$ with these
spin structure assignments is then only 
$\SO(k)\times\SO(32-k)$, and the possible spin
structures are $\A\A$, $\A\P$, $\P\A$, and $\P\P$.
Here the first and second assignments refer to the
spin structures of the first group of $\lam_-^\alpha$'s
transforming under $\SO(k)$, and the second group of
$\lam_-^\alpha$'s transforming under $\SO(32-k)$,
respectively.
(The case of more than two groups of $\lam_-^\alpha$'s
with independent spin structures may be treated
analogously, but will ultimately
not yield any consistent string theories.

The first modification in the setup of \S{B} arises
from the fact that the $\atil$ constants of the
Virasoro conditions are now given by
$$
\left\{\eqalign{
&\atil_{\A\A}=1\cr
\noalign{\bigskip}
&\atil_{\P\P}=-1\cr}\right.\qquad\qquad
\left\{\eqalign{
&\atil_{\A\P}={{\textstyle 8}\over{\textstyle 24}}
+{{\textstyle k}\over{\textstyle 24}}-{{\textstyle
32-k}\over{\textstyle 24}}=-1+{{\textstyle k}\over
  {\textstyle 16}}\cr
\noalign{\bigskip}
&\atil_{\P\A}={{\textstyle 8}\over{\textstyle 24}}-
{{\textstyle k}\over{\textstyle 24}}+{{\textstyle 32-k}
\over{\textstyle 48}} =1-{{\textstyle
k}\over{\textstyle 16}}\,\,.\cr}
\right.
\eqno{(8.C.1)}
$$
To investigate which states are allowed, we examine
the $\mass^2$ conditions, by matching the $M^2$
eigenvalue in the left- and right-moving sectors
$$
\matrix{
M^2 &\!\!\!=2N_\R=2 N_{\NS}-1\hfill 
  &\qquad\qquad \hbox{(left movers)}\hfill\cr
\noalign{\bigskip}
 &\!\!\!=2\Ntil-2\atil \hfill 
  &\qquad\qquad \hbox{(right movers)}\hfill\cr}
\eqno{(8.C.2)}
$$
Here, $\Ntil$ denotes the eigenvalue of the number
operator introduced in (8.A.15).
Now, in view of the fact that $N_\R$ is integer $\Ge
0$ and that the $\GSO$ projection in the $\NS$
sector restricts to $2N_{\NS}-1$ integer, it follows
that $M^2$ is an even integer $\Ge 0$. 
By (8.C.2), $\Ntil-\atil$ must be integer as well.
Now, $\Ntil$ is
half-integer by construction; so if there are to be
any states in the $\A\P$ and $\P\A$ sectors, we must
have
$\atil_{\A\P}=-\atil_{\P\A}$ half-integer.

If $k\not=0\pmod{8}$, then we find from the above
reasoning that there are no physical states in the
$\A\P$ and $\P\A$ states.
Thus, the only remaining sectors are $\A\A$ and
$\P\P$.
But, then all $\lam_-^\alpha$'s always have the same spin
structure, either $\A$ or $\P$, and the theory really
has $\SO(32)$ symmetry, and reduces to the case
studied in \S{B}.

There remain 2 new, inequivalent cases:
$\SO(16)\times\SO(16)$ and $\SO(8)\times\SO(24)$.
The latter turns out to be inconsistent at the
$1$-loop level (as will be demonstrated in \S{IX}), 
and we shall not investigate it any
further here.

We now analyze the $\SO(16)\times\SO(16)$ model, and
concentrate on the representations of
$\SO(16)\times\SO(16)$, under which the massless gauge
fields (space-time vectors) transform.
These states belong to the multiplet with
$N_{\NS}=1/2$ in the left-moving sector, and to those
states with $\Ntil=\atil$ and {\it with zero $\xtil$
oscillator number}, in the right moving sector.
There are now 4 contributions, from the $\A\A$, $\A\P$,
$\P\A$ and $\P\P$ sectors.
Below, we list the allowed states and indicate their
respective representation of $\SO(16)\times\SO(16)$
(by listing the dimension of the representations)
under each set of states.
$$
\matrix{
\A\A\,:\,\, &\lam_{-1/2}^\alpha\lam_{-1/2}^\beta
\left.\vert0,\A\right>_\R\otimes\left.\vert0,\A\right>_\R\cr
\noalign{\bigskip}
&(120,1)\oplus (1,120)\oplus (16,16)\hfill\cr
\noalign{\bigskip}
\A\P\,:\,\, &\qquad\qquad\qquad\!\left.\vert0,\A\right>_\R
 \otimes\left.\vert0,\sigma,\P\right>_\R\cr
\noalign{\bigskip}
&(1,128)\oplus(1,128')\hfill\cr
\noalign{\bigskip}
\P\A\,:\,\,
&\qquad\qquad\qquad\,\left.\vert0,\sigma,\P\right>_\R\otimes
  \left.\vert0,\A\right>_\R\cr
\noalign{\bigskip}
&(128,1)\oplus(128',1)\hfill\cr
\noalign{\bigskip}
\P\P\,:\,\, &\hbox{no massless states}\hfill\cr}
\eqno{(8.C.3)}
$$
Here, we have decomposed the Dirac spinor of $\SO(16)$
(actually $\Spin(16)$)
into its two irreducible Weyl components denoted by $128$ and
$128'$.

Now, all these states are to
correspond to massless Yang-Mills
fields, and they can form a consistent quantum field theory
only if they correspond to the adjoint representation
of some compact Lie algebra.
(A theory of vector fields that does not arise as a
Yang-Mills theory will not produce sufficient gauge
invariance to eliminate the negative norm states
created by the vector field.)
But, there is no Lie algebra that contains all the
states listed in (8.C.3).
From the multiplet structure, it is clear that the Lie
algebra has rank $16$, with $\SO(16)\times\SO(16)$ as
a maximal subalgebra.
If all states were to occur, then the Lie algebra
should have dimension $1008$, but there is no such Lie
algebra.

Again, to obtain a consistent theory, we have to
project \`a la $\GSO$.
Of course, we can project out the $\A\P$ and $\P\A$
sectors, to recover the $\Spin(32)/Z_2$ theory.
But, there is another solution: if we project out the
$(16,16)$ representation, then we ought to obtain a
factorized group, as is evident from the representation
contents.
Projecting out the $(16,16)$ representation is the
only possible consistent projection that will not lead back
to $\Spin(32)/Z_2$.
This can be seen as follows.
The representation $(1,120)\oplus(120,1)$ in the
$\A\A$ sector is the adjoint representtion of
$\SO(16)\times\SO(16)$, and cannot be projected out if
the Yang-Mills multiplet has to have at least
$\SO(16)\times\SO(16)$ symmetry.
But then, if we were to retain the $(16,16)$
representation, the $\A\A$ sector by itself would
produce the gauge group $\SO(32)$.
Including any of the representations in the $\A\P$ and
$\P\A$ sectors would upset the Lie aglebra structure.
Thus, it remains only to project out the $(16,16)$.

Projecting out the $(16,16)$ representation may be
achieved by using the following construction.
We have naturally $2$ $\SO(16)\times\SO(16)$-invariant
fermion number operators.
$$
\matrix{
(-1)^{F_1} &\equiv
(-1)^{\,\,\sum\limits_{r>0}
  \lam_{-r}^{\alpha_1}\lam_r^{\alpha_1}}
  &\qquad \alpha_1=1,\ldots,16\hfill\cr
\noalign{\bigskip}
(-1)^{F_2} &\equiv(-1)^{\,\,\sum\limits_{r>0}
  \lam_{-r}^{\alpha_2}\lam_r^{\alpha_2}}
  &\qquad\null\hfill \alpha_2=17,\ldots,32\cr}
\eqno{(8.C.4)}
$$
The product $(-1)^{F_1}(-1)^{F_2}$ is just the
operator $(-1)^F$ of (8.B.7), in the $\A$ sector.
The states $(16,16)$ are now eliminated by imposing the
$\GSO$ projection
$$
(-1)^{F_1}=1\qquad\hbox{\undertext{and}}\qquad
(-1)^{F_2}=1\,\,.\eqno{(8.C.5)}
$$
This projection preserves all of the $\A\P$ and $\P\A$
sectors.

Having projected out $(16,16)$, we are now left with
the following representation content in each factor:
$$
120\oplus 128\oplus 128'\eqno{(8.C.6)}
$$
This content still does not close into a Lie algebra.
But if we retain only one of the two $\SO(16)$ Weyl
spinors, say $128$, then we obtain the $\SO(16)$
contents of the adjoint representation of the
exceptional Lie algebra $E_8$.
Projecting out $(128')$ is easily achieved by the
following $\GSO$ projection in the $\P$ sector.
We define
$$
\matrix{
(-1)^{F_1} &\equiv \lam_0^1\lam_0^2\ldots\lam_0^{16}
  &(-1)^{\,\,\sum\limits_{n>0}\lam_{-n}^{\alpha_1}
  \lam_n^{\alpha_1}} &\qquad \alpha_1=1,\ldots,16\hfill\cr
\noalign{\bigskip}
(-1)^{F_2} &\equiv \lam_0^{17}\ldots\lam_0^{32}
  &(-1)^{\,\,\sum\limits_{n>0}\lam_{-n}^{\alpha_2}
  \lam_n^{\alpha_2}} &\hfill\qquad \alpha_2=17,\ldots,32\,\,,\cr}
\eqno{(8.C.7)}
$$
and require in the $\P$ sector
$$
(-1)^{F_1}=1\qquad\qquad
(-1)^{F_2}=1\,\,.
\eqno{(8.C.8)}
$$
The Yang-Mills states that we are left with after this
projection now correspond to the Lie algebra
$E_8\times E_8$.

In the functional integral formulation, the above
$\GSO$ projections correspond to an {\it
independent} summation over all spin structures,
separately for the two groups of $16$ spinors
$\lam^{\alpha_1}$ and $\lam^{\alpha_2}$.
We shall show in the next chapter that this summation
leads to modular invariant amplitudes.
Insistence on modular invariance will ultimately be
what selects out the two gauge groups $\Spin(32)/Z_2$
and $E_8\times E_8$, as we shall start to see in the
next section, and prove in full in \S{IX}.

We leave it as an exercise to the reader to show that
the $\GSO$ projected states at the massless level are
those of the $\SO(32)$ heterotic string, but with the
gauge group $\Spin(32)/Z_2$ replaced by $E_8\times
E_8$.
Notice that the adjoint representations of both groups
have the same dimensions $496$, and that both groups
have the same rank $16$.

\bigskip\noindent
\Item{\bf D)} {\bf Bosonic Realizations of the
$\Spin(32)/Z_2$ and $E_8\times E_8$ Strings}

%\nobreak

For the $\Spin(32)/Z_2$ heterotic string, we exhibited
all the generators of the associated Kac Moody
algebra, in the form of fermion bilinear operators
$$
J^a(\zbar)
=t_{\alpha\beta}^a\lam_-^\alpha\lam_-^\beta(\zbar)\,\,.
\eqno{(8.D.1)}
$$
Indeed, all $\lam_-$ have the same spin structure, and
so the current $J^a(\zbar)$ is a single valued
$1$-form.
For the $E_8\times E_8$ construction, however, we can
only exhibit an $\SO(16)\times\SO(16)$ current algebra
this way, since products of $\lam_-$'s with opposite spin
structure will not produce single-valued currents.
Thus, while the adjoint representation of $\SO(16)$ is
represented this way, the spinor of $\SO(16)$ cannot be
represented by currents of the form (8.D.1).

A general construction of vertex operators for
simply laced algebras (all of whose roots have equal
length) allows us to represent all current generators.
This construction was developed by Lepowski and Wilson
and by Frenkel, Kac and Segal; we shall not reproduce
it here in detail.
The key ingredient is that scalar fields compactified
on a torus may produce enhanced symmetries at special
radii of the torus.
For a single scalar, this is familiar from
Witten's lectures.

The starting point is a bosonized realization of the
internal degrees of freedom $\lam_-^\alpha$ in terms
of right-moving scalars $x_R^I(\zbar)$,
$I=1,\ldots,16$.
These $x^I$ are assumed to be propagating on a
$16$-dimensional torus $\dbR^{16}/2\pi\Gamma$,
specified by a lattice $\Gamma$, with basis vectors
$e_i^I$, $i=1,\ldots,16$.
We define the metric of the lattice $\Gamma$ by
the matrix of inner products of basis vectors
$e_i{}^{I}$:
$$
g_{ij}=\sum\limits_{I=1}^{16} e_i^I e_j^I\,\,.
\eqno{(8.D.2)}
$$
By construction, this lattice has positive definite
metric.

To describe the $x_R^I(\zbar)$, we begin by describing
a full $x^I=x_L^I+x_R^I$ field propagating on
$\dbR^{16}/2\pi\Gamma$.
The solution for $x^I$ on a worldsheet of a cylinder is:
$$
x^I(z,\zbar)=q^I-ix_0^I\ln\,z-i\xtil_0^I\ln\,\zbar
+\hbox{ oscillators}\,\,.
\eqno{(8.D.3)}
$$
Now, $x^I$ is valued in $\dbR^{16}/2\pi\Gamma$, and
this allows $x^I$ to be multiple valued by shifts in
$2\pi\Gamma$ as $z\to e^{2\pi i}z$:
$$
x^I\left(e^{2\pi i}z,e^{-2\pi i}\zbar\right)=
x^I(z,\zbar)+2\pi\left(x_0^I-\xtil_0^I\right)
\eqno{(8.D.4)}
$$
with 
$$
x_0^I-\xtil_0^I=\sum\limits_{i=1}^{16}n_i
e_i^I\in\Gamma
\qquad\qquad n_i\in\dbZ\,\,.
\eqno{(8.D.5)}
$$
To see the effect of this multi-valuedness, it is
helpful to consider the topology of the cylinder
instead of the torus.
Thus we set $z=\exp(\tau+i\sigma)$ with
$\sigma\in[0,2\pi]$, so that
$$
x^I(\tau,\sigma)=q^I-i(x_0^I+\xtil_0^I)\tau+
(x_0^I-\xtil_0^I)\sigma+\hbox{\rm oscillators}\,\,.
\eqno{(8.D.6)}
$$
Clearly now, as $\sigma\to\sigma+2\pi$, and we wind
around the string once, the integers $n_i$ indicate
how many times the string wraps around direction $i$
of th torus $\dbR^{16}/2\pi\Gamma$.
Such winding configurations can occur because the
torus $\dbR^{16}/2\pi\Gamma$ is non-simply connected,
and these configurations are called {\it winding
modes}, with winding numbers $n_i$.

Now $x_0^I$ and $\xtil_0^I$ are momenta of the string,
and ${1\over 2}(x_0^I+\xtil_0^I)$ is the total momentum.
It must be valued in the lattice dual to $\Gamma$,
which we denote by $\Gamma^*$, and which is generated
by the basis of vetors $e_i^{*I}$, dual to the
$e_i^I$:
$$
\sum\limits_{I} e_i^{*I}e_j^I=\delta_{ij}\,\,.
\eqno{(8.D.7)}
$$
Thus, we must have
$$
x_0^I+\xtil_0^I=\sum\limits_{i=1}^{16}m_i
e_i^{*I}\in\Gamma^*\qquad\qquad
m_i\in\dbZ\,\,.
\eqno{(8.D.8)}
$$
The discreteness of the momenta results from the
compactness of the torus $\dbR^{16}/2\pi\Gamma$.
These modes are called {\it Kaluza-Klein modes}, and the
integers $m_i$ are the Kaluza-Klein momentum mode
numbers.
Now, the Virasoro constraints are just as in the
bosonic string, except for the fact that we have split
momenta and oscillators into $10$ dimensions of
space-time and $16$ internal bosons.
We reflect this splitting in the Virasoro
$$
\matrix{\displaystyle
L_0 &=\,\half\,p^2+\half\,x_0^Ix_0^I+N \hfill
  &\qquad N=\sum\limits_{n=1}^\infty(x_{-n}\cdot x_n+
  x_{-n}^Ix_n^I)\hfill\cr
\noalign{\bigskip}
\Ltil_0 &=\,\half\,p^2+\half\,\xtil_0^I\xtil_0^I+\Ntil\hfill
  &\qquad\Ntil=\sum\limits_{n=1}^\infty(\xtil_{-n}
  \cdot\xtil_n+\xtil_{-n}^I\xtil_n^I)\hfill\cr}
\eqno{(8.D.9)}
$$
The relevant Virasoro constraints are unmodified:
$L_0=\Ltil_0=1$, so that we have
$$
\eqalign{
&\half\,x_0^Ix_0^I-\half\,\xtil_0^I\xtil_0^I+N-\Ntil=0\cr
\noalign{\bigskip}
&M^2=\half\,x_0^Ix_0^I+\half\,\xtil_0^I\xtil_0^I+
N+\Ntil-2\,\,.\cr}
\eqno{(8.D.10)}
$$

Now, we replace left-movers of the bosonic string
by the left-movers of the $\RNS$, $\GSO$
projected string.
Thus, the Virasoro and mass constraints become:
$$
\eqalignno{
L_0=\half\,p^2+N &\quad\cases{
N_\R=\sum\limits_{n=1}^\infty(x_{-n}\cdot
x_n+nd_{-n}\cdot d_n) &$(\R)$\cr
\noalign{\medskip}
N_{\NS}=\sum\limits_{n=1}^\infty(x_{-n}\cdot
x_n)+\sum\limits_{r=1/2}^\infty rb_{-r}\cdot b_r
&$(\NS)$\cr}\cr
\Ltil_0=\half\,p^2+\half\,\xtil_0^I\xtil_0^I+\Ntil\hfill &\qquad
\Ntil=\sum\limits_{i=1}^\infty(\xtil_{-n}\cdot
  \xtil_n+\xtil_{-n}^I\cdot\xtil_n^I) &(8.D.11)\cr}
$$
Thus, we have
$$
\eqalign{
M^2 &=2N_\R=2N_{\NS}-1 \cr
 &=\xtil_0^I\xtil_0^I+2\Ntil-2\cr}
\eqno{(8.D.12)}
$$
Furthermore, $\xtil_0^I$ must now live on $\Gamma$ and
on $\Gamma^*$, and the length of any allowed
$\xtil_0^I\xtil_0^I$ must be {\it even}.
For general lattices $\Gamma$, $\Gamma$ and $\Gamma^*$
will not have any common points.
Thus, the above condition puts severe constraints on what
lattices are allowed for $\Gamma$.
We assume that $16$ independent directions for
$\xtil_0^I$ are allowed, which essentially implies
that $\Gamma$ must be {\it self-dual}.
(We cannot fully prove at this point that self-duality
is necessary; the latter condition will be firmly
established from the requirement of modular invariance
of the transition amplitudes at loop order $h\Ge 1$,
as we shall see in \S{IX}.
For the time being, we assume self-duality.)

There are then only two possible lattices:
$$
\matrix{
E_8\times E_8\hfill &\hbox{root lattice},\hfill 
  &g_{ij} &\hbox{Cartan matrix}\cr
\noalign{\bigskip}
\Spin(32)/Z_2\hfill &\hbox{root lattice},\hfill
  &g_{ij} &\hbox{Cartan matrix}\cr}
\eqno{(8.D.13)}
$$
The states of the heterotic strings are
reproduced as follows: for massless states, we have
$$
\vbox{
\eqalignno{
&\left(
\matrix{N_\R=0\hfill\cr
\noalign{\medskip}
N_{\NS}=1/2\hfill\cr}\right)_L\otimes
\left\{\left(\matrix{
\Ntil=1\hfill\cr
\noalign{\medskip}
\xtil_0^I\xtil_0^I=0\hfill\cr}\right)_R
\oplus\left(\matrix{\Ntil=0\hfill\cr
\noalign{\medskip}
\xtil_0^I\xtil_0^I=2\hfill\cr}\right)_R\right\}\cr
\noalign{\bigskip}
&\left(\matrix{
\Ntil=1\hfill\cr
\noalign{\medskip}
\xtil_0^I\xtil_0^I=0\hfill\cr}\right)
\matrix{&\quad \xtil_{-1}^\mu\hbox{ states} &\to
  &N=1\hbox{ supergravity multiplet}\hfill\cr
 &\quad \xtil_{-1}^I\hbox{ states} &\to
&16\hbox{ states of super-Yang-Mills multiplet}\hfill\cr}\cr
\noalign{\bigskip}
&\left(\matrix{
\Ntil=0\hfill\cr
\noalign{\medskip}
\xtil_0^I\xtil_0^I=2\hfill\cr}\right)
\matrix{&\quad 480\hbox{ states of }
  N=1 \hbox{ super-Yang-Mills multiplet}\hfill \cr
 &\quad \hbox{obtained as 
  \undertext{winding} states}\hfill\cr}\cr}}
$$
The total number of Yang-Mills states is $480+16=496$,
which coincides with the dimension of $E_8\times E_8$
and $\Spin(32)/Z_2$.

In Problem Set \#10, toroidal compactifications of the
heterotic strings, their relation by $T$-duality and
their moduli spaces are discussed.




\bye






