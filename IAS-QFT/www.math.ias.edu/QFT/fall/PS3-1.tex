%From: Pierre Deligne <deligne@math.ias.edu>
%Date: Thu, 31 Oct 1996 16:41:13 -0500
%Subject: Witten's Set Three, No. 1

%Below is the AMS-TeX file for the solution written
%by P. Deligne and D. Freed to Witten's Problems, Set Three,
%No. 1, to be put on the web page.

\input amstex
\documentstyle{amsppt}
\magnification=1200
\pagewidth{6.5 true in}
\pageheight{8.9 true in}
\loadeusm

\catcode`\@=11
\def\logo@{}
\catcode`\@=13

\NoRunningHeads

\font\boldtitlefont=cmb10 scaled\magstep1

\def\dspace{\lineskip=2pt\baselineskip=18pt\lineskiplimit=0pt}
\def\wedgeop{\operatornamewithlimits{\wedge}\limits}
\def\w{{\mathchoice{\,{\scriptstyle\wedge}\,}
  {{\scriptstyle\wedge}}
  {{\scriptscriptstyle\wedge}}{{\scriptscriptstyle\wedge}}}}
\def\Le{{\mathchoice{\,{\scriptstyle\le}\,}
  {\,{\scriptstyle\le}\,}
  {\,{\scriptscriptstyle\le}\,}{\,{\scriptscriptstyle\le}\,}}}
\def\Ge{{\mathchoice{\,{\scriptstyle\ge}\,}
  {\,{\scriptstyle\ge}\,}
  {\,{\scriptscriptstyle\ge}\,}{\,{\scriptscriptstyle\ge}\,}}}
\def\vrulesub#1{\hbox{\vrule height7pt depth5pt\,}_{#1}}
\def\mapright#1{\smash{\mathop{\,\longrightarrow\,}%
     \limits^{#1}}}
\def\plus{{\sssize +}}
\def\upvee{{\sssize\vee}}

\def\eps{{\varepsilon}}
\def\Lam{{\Lambda}}
\def\mynabla{{\nabla\!}}

\def\xtil{\widetilde{x}}

\def\Adot{\Dot{A}}
\def\Bdot{\Dot{B}}
\def\Xdot{\Dot{X}}
\def\xdot{\Dot{x}}
\def\psidot{\Dot{\psi}}

\def\dbR{{\Bbb R}}

\def\GL{\text{\rm GL}} \def\Hom{\text{\rm Hom}}
\def\aff{\text{\rm aff}} \def\red{\text{\rm red}}
\def\Map{\text{\rm Map}} \def\Lie{\text{\rm Lie}}
\def\Diff{\text{\rm Diff}} \def\Vol{\text{\rm Vol}}
\def\Tr{\text{\rm Tr}}


\def\scr#1{{\fam\eusmfam\relax#1}}

\def\scrA{{\scr A}}   \def\scrB{{\scr B}}
\def\scrC{{\scr C}}   \def\scrD{{\scr D}}
\def\scrE{{\scr E}}   \def\scrF{{\scr F}}
\def\scrG{{\scr G}}   \def\scrH{{\scr H}}
\def\scrI{{\scr I}}   \def\scrJ{{\scr J}}
\def\scrK{{\scr K}}   \def\scrL{{\scr L}}
\def\scrM{{\scr M}}   \def\scrN{{\scr N}}
\def\scrO{{\scr O}}   \def\scrP{{\scr P}}
\def\scrQ{{\scr Q}}   \def\scrR{{\scr R}}
\def\scrS{{\scr S}}   \def\scrT{{\scr T}}
\def\scrU{{\scr U}}   \def\scrV{{\scr V}}
\def\scrW{{\scr W}}   \def\scrX{{\scr X}}
\def\scrY{{\scr Y}}   \def\scrZ{{\scr Z}}


\document
\line{{\boldtitlefont Witten's Problems}, Set Three --- 
N$^{\text{o}}$. 1 
\hfill (solution written by P. Deligne \& D. Freed)}
\smallskip
\hbox to \hsize{\hrulefill}

\bigskip
\subhead
Preface
\endsubhead
\dspace
It should have been added to the assumptions on
$(W^{2,2},\Lambda_\plus,\Lambda_-)$ that the Frobenius
pairing
$$
[\quad,\quad]\colon\,
\Lambda_\plus\otimes\Lambda_-\to
T/(\Lambda_\plus\oplus\Lambda_-)
\tag1
$$
vanishes.
If this condition holds, there are local coordinates
$(u,v,\theta_\plus,\theta_-)$ in which $\Lambda_\plus$ and
$\Lambda_-$ are generated by $D_\plus$, $D_-$ as in the
given formuli, with the ``$i$'' omitted.
We hope to explain elsewhere why ``$i$'' was there.
A down to earth explanation is that the formuli describe
$\Lam_\plus$ in $\Lam_-$ in a complex coordinate system
$(u,v,\theta_\plus,\theta_-)$, with $u$, $v$,
$\iota^{1/2}\theta_\plus$ and $\iota^{1/2}\theta_-$
real.
In what follows, the ``$i$'' will be suppressed.



\demo{Proof}
If $D_\plus$ generates $\Lambda_\plus$, $D_\plus$ and $D_\plus^2$
generates an integrable distribution, as results from
the identity $[D_\plus,D_\plus^2]=0$.
The distributions $\left<D_\plus,D_\plus^2\right>$ and
$\left<D_-,D_-^2\right>$ are integrable and, by
assumption, transversal.
They correspond locally to a decomposition
$$
W^{2,2}=W_\plus^{1,1}\times W_-^{1,1}\,\,.
$$

The assumption that $[\Lambda_\plus,\Lambda_-]$ be zero
modulo $\left<D_\plus,D_\plus^2,D_-\right>$ amounts to saying
that, locally, $\Lam_-$ comes from a distribution on the
factor $W_-^{1,1}$.
The assumption (1) hence assume that $\Lam_\plus$ and
$\Lam_-$ come from distribution, still noted $\Lam_\plus$
and $\Lam_-$, on $W_\plus^{1,1}$ and $W_-^{1,1}$.
\enddemo

To find local coordinates in which $\Lam_\plus$ and $\Lam_-$
are as given, one can consider separately
$(W_\plus^{1,1},\Lam_\plus)$ and $(W_-^{1,1},\Lam_-)$.
Write $(W,\Lam)$ for $(W_\plus^{1,1},\Lam_\plus)$ or
$(W_-^{1,1},\Lam_-)$.
It remains to choose an odd vector field $D$ generating
$\Lam$ and to apply the

\proclaim{Lemma}
Consider the systems $(W,D)$: \ $W$ of dimension
$(1,1)$, $D$ an odd vector field such that $D$ and $D^2$
span the tangent bundle $T$.
They are all locally isomorphic.
\endproclaim

\demo{Proof}
The Lie algebra $F$ freely generated by an odd generator
$x$ is spanned by $x$ and $x^2$, as is shown by the
identity $[x,x^2]=0$.
Let $G$ be the corresponding simply connected group.
The Lie algebra of infinitesimal diffeomorphisms of $W$
is the Lie algebra of vector fields, with the opposite
of the usual bracket.
Let $\rho$ be the morphism $F\to\Lie\,\Diff(W)$ for
which $\rho(x)=D$.
it integrates to a local action of $G$ on $W$.
For $w$ a point of $W$, the map $g\mapsto gw$ is an
isomorphism from a neighborhood of $e$ in $G$ to one of
$w$ in $W$.
By construction, $D$ corresponds to the right invariant
vector field $x$ on $G$.
\enddemo

\proclaim{Remark} \rm
The formuli given, in local coordinates, for $D_\plus$
and $D_-$ show that $D_\plus$ and $D_-$ commute: \
the distributions $\Lam_\plus$ and $\Lam_-$ are locally
generted by commuting vector fields $D_\plus$ and $D_-$.
Other commuting vector fields $D_\plus$, $D_-$
generating $\Lam_\plus$ and $\Lam_-$ are of the form
$f_\plus D_\plus$, $f_- D_-$, with $D_- f_\plus=D_\plus
f_-=0$.
\endproclaim

\proclaim{Remark} \rm
The discussion up to now works as well in a relative
setting, over a basis $B$.
What comes next does not: \ as $S$ is of dimension
$(*,1)$, we have a canonical projection $W\to W_{\red}$: \ 
the quotient $\scrO_{\red}$ of $\scrO_W$ is an isomorphic
image of the subsheaf $\scrO^\plus$ of $\scrO$.
For $W^{2,2}=W^{1,1}\times W^{1,1}$, this gives a
canonical retraction
$$
p\colon\, W^{2,2}\to W_{\red}^{2,2}\,\,.
$$
By integration along the fibers of $p$, a density on
$W^{2,2}$ projects to a density on $W_{\red}^{2,2}$.
In local coordinates, it is given by
$$
p_*(f.du\,dv(d\theta_\plus d\theta_-)^{-1})=(D_-D_\plus
f)du\,dv
$$
where $D_-$ $D_\plus$ $f$ is to be restricted to
$W_{\red}^{2,2}$ (coordinates $u$, $v$), defined in
$W^{2,2}$ by the vanishing of $\theta_\plus$ and $\theta_-$.

Of course, once we have $p$, we keep having it after
change of basis: \ we have a morphism induced by $p$
from $W^{2,2}\times B$ to $W_{\red}^{2,2}\times B$.
The formula for $p_*$ remains valid.
\endproclaim

To give $\phi\colon\, W^{2,2}\to M$ amounts to giving
its restriction $X$ to $W_{\red}^{1,1}$, $\psi_\pm=D_pm
X$: \ odd sections of the pull back $X^*TM$ of the
tangent bundle of $M$, and $F=\mynabla_{D^\plus}D_-
X=-D_-\mynabla_{D^\plus}X$, an even section of this pull
back.
For the abuse of language used, see Set One,
N$^{\text{o}}$ 2.
Here $D_\plus$ and $D_-$ are commuting generators of
$C_\plus$ and $D_-$.
if $\Lam_\pm^0$ is the restriction of $\Lam_\pm$ to
$W_{\red}^{1,1}$, indepedently of a choice of $D_\pm$,
one can see $\psi_\pm$ and $F$ as even sections of
$(\Lam_\pm)^{\upvee}\otimes X^*TM$ and
$(\Lam_\plus\otimes\Lam_-)^{\upvee}\otimes X^*TM$.
Using $X$, $\psi_\pm$ and $F$ instead of $\phi$, and
$p_*\scrL$ instead of $\scrL$, is working ``in
components''.


\proclaim{Remark {\rm (to (b))}} \rm
If to a langrangian density $\scrL(\phi)$ is added a
density of the form $d\scrM(\phi)$, with $\scrM(\phi)$ a
codimension $1$ integral form attached locally to
$\phi$, the space of stationary points does not change:
\ we are to consider deformations with compact support
$\phi_u$ of $\phi$, and
$$
\int d\scrM(\phi_u)-d\scrM(\phi)=\int
d(\scrM(\phi_u)-\scrM(\phi))=0
$$
as $\scrM(\phi_u)-\scrM(\phi)$ has compact support.

The $1$-form attached to a space like hypersurface
$\Gamma$ changes, but the $2$-form on the space of
stationary points does not: \ the $1$-form changes by
$$
\delta\phi\longmapsto
\int_{\Gamma}\partial_u\scrM(\phi\plus\delta\phi)\,\,,
$$
the differential of
$\phi\mapsto\int_{\Gamma}\scrM(\phi)$.
\endproclaim

\bigskip
\subhead
Solution
\endsubhead
(a)\enspace
The bundle of volume forms is canonically isomorphic to
$\Lam_\plus^{-1}\otimes\Lam_-^{-1}$.
To see this, note that the Frobenius pairing is an
isomorphism
$$
\Tr\colon\,
\Lam_\plus\otimes\Lam_\plus
\oplus\Lam_-\otimes\Lam_-\longrightarrow
TW/\Lam_\plus\oplus\Lam_-\,\, ,
$$
and from the exact sequence
$$
0 @>>> \Lam_\plus\oplus\Lam_- @>>> TW @>>>
TW/\Lam_\plus\oplus\Lam_- @>>> 0\,\, 
$$
we conclude
$$
\align
\Vol(TW) &\simeq \Vol(\Lam_\otimes\otimes\Lam_-)
  \otimes\Vol(TW/\Lam_\plus\oplus\Lam_-)\\
&\simeq(\Lam_\plus\otimes\Lam_-)
  \otimes\Vol(\Lam_\plus^{\otimes 2}
     \oplus\Lam_-^{\otimes 2})\\
&\simeq (\Lam_\plus\otimes\Lam_-)\otimes(\Lam_\plus^{\otimes-2}
     \otimes \Lam_-^{\otimes-2})\\
&\simeq \Lam_\plus^{-1}\otimes\Lam_-^{-1}
\endalign
$$
The lagrangian density is
$$
\scrL=(D_\plus\Phi,D_-\Phi)D_\plus^{-1}\otimes D_-^{-1}\,\,.
$$

The same argument shows that for a $2$-form $B$,
$$
\scrL'_B=B(D_\plus\Phi,D_-\Phi)D_\plus^{-1}\otimes D_-^{-1}
$$
is well defined.

For any vector field $\xi$ we have
$\scrL_\xi=[d,\iota_\xi]=d\iota_\xi\plus i_\xi d$, relating the
Lie derivative and the contraction, so if $\lambda$ is
an even $1$-form, then\footnote"$^{\dsize *}$"{See
forthcoming 
Deligne's appendix to Bernstein's lectures for details
about the sign conventions.}
$$
\aligned
d\lambda(\xi_1,\xi_2)
=\pm\iota_{\xi_2}\iota_{\xi_1}d\lambda &=\pm\iota_{\xi_2}
  \left(\scrL_{\xi_1}\lambda-d\iota_{\xi_1}\lambda\right)\\
&=\left[\scrL_{\xi_1}\iota_{\xi_2}-\iota_{[\xi_1,\xi_2]}
 \lambda\right]\mp \scrL_{\xi_2}\iota_{\xi_1}\lambda\\
\endaligned
\tag1
$$
Here the sign $\pm$ is $-$ if both $\xi_1$, $\xi_2$ are odd
and $+$ otherwise.
We apply this to $\scrL'_B$ for $B=d\Lam$
$$
\aligned
\scrL'_{d\Lam} &=d\Lam(D_\plus\Phi,D_-\Phi)\Delta\\
&=-\iota_{D_-}\iota_{D_\plus}\Phi^*\Lam\cdot\Delta\\
&=\left\{\scrL_{D_\plus}\left(\iota_{D_-}\Phi^*\Lam\right)
  +\scrL_{D_-}\left(\iota_{D_\plus}\Phi^*\Lam\right)\right\}
  \Delta\\
&=\scrL_{D_\plus}\left(\iota_{D_-}\Phi^*\Lam\cdot\Delta\right)
   +\scrL_{D_-}\left(\iota_{D_\plus}\Phi^*\Lam\Delta\right)\\
&=-d\left(\iota_{D_-}\Phi^*\Lam\cdot
\iota_{D_\plus}\Delta+\iota_{D_\plus}\Phi^*\Lam\cdot
  \iota_{D_-}\Delta\right)\,\,.
\endaligned
\tag2
$$
In this computation $\Delta=D_\plus^{-1}\otimes D_-^{-1}$,
and we made use of the fact that $\scrL_{D_\pm}(\Delta)=0$.

\bigskip\noindent
(b)\enspace
Let $i\colon\,\dbR^2\hookrightarrow \dbR^{2,2}$ be
the inclusion.
One has $\iota^*(\theta^\pm)=0$.
If $p$ is the projection $(u,v,\theta_\plus,\theta_-)\to(u,v)$ 
of $\dbR^{2,2}$ to $\dbR^2$, and $\Delta$ the density
$du\,dv(d\theta_\plus d\theta_-)^{-1}$, we have
$p_*(f\Delta)=\iota^*(D_\plus D_-f)du\,dv$.
For $\scrL$ we have, omitting $du\,dv$ for convenience:
$$
\align
p_*(\scrL) &=i^*D_{\plus}D_-\scrL
=i^*D_{\plus}D_-(D_{\plus}\Phi,D_-\Phi)\\
&=i^*D_{\plus}\left\{(\mynabla_-D_{\plus}\Phi,D_-\Phi)-
   (D_{\plus}\Phi,\mynabla_-D_-\Phi)\right\}\\
&=i^*\{(\mynabla_{\plus}\mynabla_-
  D_{\plus}\Phi,D_-\Phi)+(\mynabla_-
  D_{\plus}\Phi,\mynabla_{\plus}D_-\Phi)-
  (\mynabla_{\plus}D_{\plus}\Phi,
  \mynabla_-D_-\Phi)+\\
&\qquad\qquad\qquad\qquad +(D_{\plus}\Phi,\mynabla_{\plus}
     \mynabla_-\Phi)\}\\
&=i^*\{(R(D_{\plus}\Phi,D_-\Phi)D_{\plus}
  \Phi,D_-\Phi)-(\mynabla_-
\mynabla_{\plus} D_{\plus} \Phi,D_-\Phi)+\\
&\qquad\qquad\qquad\qquad +(\mynabla_{\plus}
D_-\Phi,\mynabla_{\plus} 
D_-\Phi)-(\mynabla_{\plus} D_{\plus} \Phi,\mynabla_-D_-\Phi)+
(D_{\plus} \Phi,\mynabla_{\plus} 
  \mynabla_-D_-\Phi)\}\\
&=(R(\psi_{\plus},
  \psi_-)\psi_{\plus} ,\psi_-)+(i^*\mynabla_-\partial_v
\Phi,\psi_-)+(F,F)-(\partial_vX,\partial_uX)
   -(\psi_{\plus},i^*\mynabla_{\plus} \partial_u\Phi)\\
&=(R(\psi_{\plus},
  \psi_-)\psi_{\plus} ,\psi_-)+(\mynabla_v\psi_-\psi_-)
  +(F,F)-(\partial_v X,\partial_uX)-(\psi_{\plus} ,\partial_u
  \psi_{\plus} )
\endalign
$$
Here we modified the notation of the problem to remove the
factors of $\sqrt{-1}\,$:
$$
\align
D_{\plus}
&=\tfrac{\partial}{\partial\theta_\plus}-\theta_\plus
  \tfrac{\partial}{\partial v}\\
D_- &=\tfrac{\partial}{\partial\theta_-}-\theta_-
   \tfrac{\partial}{\partial u}\\
\intertext{and we expressed $\phi$ in components:}
X &=i^*\Phi\\
\psi_\plus &=i^*D_\plus\Phi\\
\psi_- &=i^*D_-\Phi\\
F &=i^*\mynabla_\plus D_-\Phi\,\,.
\endalign
$$
Also, $\mynabla_{\pm}$ denotes the covariant derivative
with respect to $D_{\pm}$.

We compute how $\scrL$ changes by an {\it even}
deformation of $\phi$:
$$
\align
\delta\scrL =\delta(D_\plus\Phi,D_-\Phi) &=
  (\mynabla_\delta D_\plus\Phi,D_-\Phi)+(D_\plus\Phi,
  \mynabla_\delta D_-\Phi)\\
&=(\mynabla_\plus\delta\Phi,D_-\Phi)+(D_\plus\Phi,
  \mynabla_-\delta\Phi)\\
&=D_\plus(\delta\Phi,D_-\Phi)-D_-(D_\plus\Phi,\delta\Phi)
     -2(\mynabla_\plus D_-\Phi,\delta\Phi)
\endalign
$$
The first two terms are exact, and the equation of
motion is
$$
\mynabla_\plus D_-\Phi=0\,\,.
\tag3
$$
We evaluate the components of (3) by applying $i^*$,
$i^*\mynabla_\plus$, $i^*\mynabla_-$, and
$i^*\mynabla_\plus\mynabla_-$.
The results are:
$$
\align
&i^*(\mynabla_\plus D_-\Phi)=F\,\,.\\
&i^*\mynabla_\plus(\mynabla_\plus D_-\Phi)=
  R(\psi_\plus,\psi_-)\psi_\plus+\mynabla_v\psi_-\\
&i^*\mynabla_-(\mynabla_\plus
D_-\Phi)\!=R(\psi_-,\psi_\plus)\psi_-+\mynabla_u\psi_\plus
\endalign
$$
We give details for the last of these computations:
$$
\align
i^*\mynabla_\plus\mynabla_-(\mynabla_\plus D_-\Phi)
&=i^*\mynabla_\plus\left\{R(D_-\Phi,D_\plus\Phi)D_-\Phi-
  \mynabla_\plus\mynabla_-D_-\Phi\right\}\\
&=i^*\mynabla_\plus\left\{-
  \tfrac12\,R(D_-\Phi,D_-\Phi)D_\plus\Phi
  -\mynabla_\plus\mynabla_-D_-\Phi\right\}\\
&=i^*\{(\mynabla_\plus
R)(D_-\Phi,D_\plus\Phi)D_\plus\Phi-R(\mynabla_\plus
D_-\Phi,D_-\Phi)D_\plus\Phi-\\
&\qquad\qquad 
  -\tfrac12\,R(D_-\Phi,D_-\Phi)\mynabla_\plus D_\plus
  \Phi+\mynabla_u\partial_v\Phi-\tfrac12\,R(D_\plus\Phi,
  D_\plus\Phi)\partial_v\Phi\}\\
&=(\mynabla_\plus R)(\psi_-,\psi_-)\psi_\plus-R(F,\psi_-)
  \psi_\plus-\tfrac12\,R(\psi_-,\psi_-)\partial_vX+
  \mynabla_u\partial_vX-\\
&\qquad\qquad
-\tfrac12\,R(\psi_\plus,\psi_\plus)
 \partial_vX\,\,.
\endalign
$$
In the second line we use the Bianchi identity and in
the third we use
$$
\mynabla_\plus\mynabla_\plus=-\mynabla_u+\tfrac12\,
   R(D_\plus,D_\plus)\,\,.
$$
Of course, we can set $F=0$ in the final expression
because of the first equation of motion.
Collating the results, we obtain the equations of motion
in components:
$$
\align
F &=0\\
\mynabla_v\psi_- &=-R(\psi_\plus,\psi_-)\psi_\plus\\
\mynabla_u\psi_\plus &=-R(\psi_-,\psi_\plus)\psi_-\\
\mynabla_u\partial_v X
&=\tfrac12\left\{R(\psi_-,\psi_-)\partial_vX+R(\psi_\plus,
  \psi_\plus)\partial_u X\right\}
  -(\mynabla_\plus R)(\psi_-,\psi_-)\psi_\plus
\endalign
$$

\bigskip\noindent
(d)\enspace
Let $j\colon\,\dbR^{2,1}\hookrightarrow\dbR^{1,1}$ so that $j^*$
sets $\theta_-=0$.
Then
$$
\align
j^*D_\scrL &=j^*D_-(D_\plus\Phi,D_-\Phi)\\
&=j^*\left\{(\mynabla_-D_\plus\Phi,D_-\Phi)-
  (D_\plus\Phi,\mynabla_- D_-\Phi)\right\}\\
&=j^*\left\{(\mynabla_\plus D_-\Phi,D_-\Phi)+(D_\plus
  \Phi,\partial_u\Phi)\right\}\\
&=(\mynabla_\plus\Lam,\Lam)+(D_\plus\Phi',\partial_u\Phi')
\endalign
$$
where
$$
\align
j^*\Phi &=\Phi'\\
j^*D_-\Phi &=\Lam\,\,.
\endalign
$$



\enddocument



