%From: "Eric D'Hoker" <dhoker@IAS.EDU>
%Date: Wed, 5 Feb 1997 16:07:38 -0500 (EST)

%%%%%%%%%%%%%%%%%%%%%%%%%%%%%%%%%%%%%%%%%%%%%%%%%%%%%%%%%%%%%%%%%%%%%%%%%%%
%%%%
%%%%    STRING THEORY : Problem Set 3, February 6, 1997
%%%%
%%%%%%%%%%%%%%%%%%%%%%%%%%%%%%%%%%%%%%%%%%%%%%%%%%%%%%%%%%%%%%%%%%%%%%%%%%%


\magnification=\magstep1
\overfullrule=0pt
\baselineskip=17pt

\def\12{{1 \over 2}}

\centerline{{\bf STRING THEORY}}
\centerline{ Problem Set \# 3}
\centerline{February 6, 1997}

\bigskip
\bigskip

Transition amplitudes of a single point particle in Quantum
Field Theory may be described by summation over the space of 
paths, in complete analogy with the summation over surfaces
that we prescribed for transition amplitudes of strings. In
this problem set, we propose to show that certain point particle amplitudes,
prescribed this way, precisely agree with those obtained from
corresponding Feynman diagrams.

Let $M={\bf R}^D $ be flat Euclidean space-time, and define 
the world-line action for a single point particle of electric 
charge $e$ in the 
presence of a fixed external U(1) gauge field $A$ by
$$
S[x,g,A] = \12 \int _0 ^1 d \tau \biggl \{
           g(\tau)^{-1} \dot x \cdot \dot x+ g(\tau ) m_0 ^2 \biggr \}
           +i e \int _0 ^1 d\tau \dot x \cdot A(x)
\eqno (1)
$$ 
We define amplitudes for transition between points $y,\ y' \in M$
by
$$
A(y,y') = \int _{Met([0,1])} Dg {1 \over {\cal N} (g)}
\int _{Map([0,1];M)} Dx e^{-S}
$$
The measures $Dg$ and $Dx$ are defined with respect to the 
Diff([0,1]) invariant $L^2$ norms on $\delta g$ and $\delta x$,
and ${\cal N}(g)$ is formally equal to the volume of 
Diff([0,1]).

\bigskip

\noindent
1) Derive the infinitesimal transformations under Diff([0,1])
(generated by a vector field $v$) of $g$ and $x$,
verify that $S$ is invariant and find the Diff([0,1])
invariant $L^2$ norms of $\delta g$ and $\delta x$.

\noindent
2) Derive an orthogonal decomposition of an arbitrary $\delta g$
into infinitesimal diffeomorphisms and a change in length
$L= \int _0 ^1 d \tau g( \tau)$.

\noindent
3) Use the above decomposition to factorize the measure $Dg$
into $Dv \ dL$, and compute the associated Jacobian.

\noindent
4) Show that $A(y,y')$ coincides with the Green function
for a scalar field of mass $m$ in the presence of an Abelian 
gauge field $A$. (The relation between $m_0$ and $m$ may
depend upon the regularization scheme used to calculate
the Jacobian in 3.)

\noindent
5) Generalize the above prescription and adapt it to the
calculation of one loop amplitudes for charged scalar particles
in the presence of an external U(1) gauge field. 

\noindent
6) Using 5), show that the one loop amplitudes obtained 
this way agree with those derived from Feynman diagrams,
in a perturbation expansion in powers of $A$. (You may
restrict attention to the first non-trivial order in $A$.)
Show that the length parameters ($L$, introduced in 2)) 
of the worldlines between
successive insertions of $A$ in this expansion are related to 
the Feynman parameters used to evaluate the corresponding 
Feynman diagrams 
in momentum space. 
          
\end
--6440_41c5-6f55_2edf-2d32_4c6d--

