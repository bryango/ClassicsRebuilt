
% DEFINITIONS  FOR ADS3 SECTION.

\def\ra{\rightarrow}
\def\quart{{1\over4}}
\def\p{\partial}
\def\ket#1{|#1\rangle}
\def\bra#1{\langle#1|}
\def\grad{\vec \nabla}
\def\bp{\bar \p}
\def\RN{Reissner-Nordstr\"om}
\def\apm{\alpha'}
\def\at{{\tilde \alpha}}
\def\s42{ 2^{-{1\over 4} } }
\def\shalf{{\scriptscriptstyle{1\over2}}}
\def\sign{{\rm sign}}
\def\csc{closed string channel}
\def\osc{open string channel}
\def\cL{{\cal L}}
\def\goto#1{\mathrel {\mathop {\longrightarrow} \limits_{#1}}}
\def\lr{\goto{r\to\infty}}
\def\exp{{\rm exp}}
\def\lb{\left\langle}
\def\rb{\right\rangle}
\def\ie{i\epsilon}
\def\ra{\rightarrow}
\def\propaga#1{\left({\theta_1(#1) \over \theta_4(#1)} \right) }
\def\propp{\left({\theta'_1(0) \over \theta_4(0)}\right) }
\def\g{\gamma}
\def\gb{\bar{\gamma}}
\def\a{\alpha}
\def\sa{r_0^2 {\rm sinh}^2\alpha }
\def\sg{r_0^2 {\rm sinh}^2\gamma }
\def\ss{r_0^2 {\rm sinh}^2\sigma }
\def\ca{r_0^2 {\rm cosh}^2\alpha }
\def\cg{r_0^2 {\rm cosh}^2\gamma }
\def\cs{r_0^2 {\rm cosh}^2\sigma }
\def\[{\left [}
\def\]{\right ]}
\def\({\left (}
\def\){\right )}
\def\b{\beta}




% \section{AdS$_3$ }
% \label{adsthree}


In this chapter we will study the relation between gravity theories
(string theories) on 
$AdS_3$ and two  dimensional conformal field theories. 
First we are going to describe some generalities which are valid for any
$AdS_3$ quantum gravity theory, and then we will discuss in more detail 
IIB string theory compactified on $AdS_3 \times S^3 \times M^4$ with 
$M^4 = K3$ or $T^4$. 

$AdS_3$ quantum gravity is conjectured to be dual to 
a two dimensional conformal field theory which can be thought of
as living on the boundary of $AdS_3$.
The boundary of $AdS_3$ (in global coordinates)
is a cylinder, so the conformal 
field theory is defined on this cylinder. We choose the cylinder to have 
radius one, which is the usual convention for conformal field theories.
 Of course, all circles are equivalent since
this is a conformal field theory, but we have to rescale energies
accordingly. If the spacetime theory or the conformal field theory contain
 fermions then they have anti-periodic boundary conditions 
 on the circle. The reason is that
the circle is contractible in $AdS_3$, 
 and close to the ``center'' of $AdS_3$ 
a translation by $2 \pi$ on the circle looks like a rotation
by $2 \pi$, and fermions get a minus sign. So, the dual conformal field
theory is in the NS-NS sector. Note that we 
will not sum over sectors as we
do in string theory, since in this case the conformal field theory 
describes string theory on
the given spacetime and all its finite energy excitations,
and we do not  have to 
second-quantize it. 

%We will first discuss some general properties that follow just from
%the fact that we have $AdS_3$ gravity and a conformal field theory.



\section{The Virasoro Algebra}

The  isometry group  of $AdS_3$ is $SL(2,\IR) \times SL(2,\IR)$, or
$SO(2,2)$. 
%In a conformal field theory in two dimensions the 
The conformal group in two dimensions is infinite.  
This seems to be, at first sight, a 
contradiction, since in our previous discussion we identified the
conformal group with the isometry group of $AdS$. 
However, out of the infinite set of generators 
only an $SL(2,\IR)\times SL(2,\IR)$ subgroup leaves the vacuum invariant.
The vacuum corresponds to empty $AdS_3$, and this
 subgroup corresponds to the group of isometries of $AdS_3$.
 The other generators map the vacuum into some
excited states. So, we expect to find that the other generators of the
conformal group map empty $AdS_3$ into $AdS_3$ with (for instance)
a graviton inside.
These other generators are associated to reparametrizations that
leave the asymptotic form of $AdS_3 $ invariant at infinity. 
This problem was analyzed in detail in \cite{Brown:1986nw}
and we will
just sketch the argument here. 
The metric on $AdS_3$ can be
written as
\eqn{metadst}{
ds^2 = R^2 (- \cosh^2  \rho d \tau ^2 + \sinh^2 \rho d\phi^2  + d \rho^2 ).
}
When $\rho$ is large (close to the boundary) this is approximately 
\eqn{metasym}{
ds^2 \sim R^2 \left[
  - e^{2 \rho }  d\tau^+ d \tau^-  + d\rho^2 \right],
}
where $\tau^\pm \equiv \tau \pm \phi$. 
An infinitesimal reparametrization
generated by a 
 general vector field $\xi^\alpha (\tau,\phi,\rho)$ changes the metric
by $ g_{\alpha \beta} \to  g_{\alpha \beta} + \nabla_\alpha \xi_\beta
+ \nabla_\beta \xi_\alpha $. 
If we want to preserve the asymptotic form of the metric \metasym, 
we require that \cite{Brown:1986nw}
\eqn{asymp}{\eqalign{
\xi^+ & = f(\tau^+) +
 {e^{-2\rho} \over 2 } g''(\tau^-) + O(e^{-4 \rho})~, \cr 
\xi^- & = g(\tau^-) +
 {e^{-2\rho} \over 2 } f''(\tau^+) +  O(e^{-4 \rho})~, \cr 
\xi^\rho & = - { f'(\tau^+)  \over 2 } - {  g'(\tau^-) \over 2 }  
+ O(e^{-2\rho})~,
}}
where $f(\tau^+)$ and $g(\tau^-)$ are arbitrary functions.
Expanding the functions $f = \sum L_n e^{n \tau^+ } $, $g = 
\sum {\bar L}_n e^{ n \tau^- } $, we recognize the
Virasoro generators $L_n, \bar L_{n}$. 
For the cases $n =0,\pm1$ one can find some isometries that
reduce to \asymp\ at infinity, are globally defined, and 
leave the metric invariant. These are the $SO(2,2)$ isometries
discussed above. For the other generators it is possible to 
find a globally defined vector field $\xi$, but it does not 
leave the metric invariant. 

It is possible to calculate the classical Poisson brackets 
among these generators, and one finds that 
this classical algebra has a central
charge which is equal to \cite{Brown:1986nw}
\eqn{brown}{
 c = { 3 R \over 2  G_N^{(3)} },
}
where $G_N^{(3)}$ is the three dimensional Newton constant. 
So, this should also be the central charge of the dual conformal
field theory, since \asymp\ implies that
these Virasoro generators are acting on the boundary
as the Virasoro generators of a 1+1 dimensional conformal field 
theory. 

A simple calculation of the central charge term \brown\ 
was given in \cite{Balasubramanian:1999re}.
Under a diffeomorphism of the form \asymp, 
the metric near the boundary changes to 
\eqn{chnmet}{
ds^2 \to  R^2 \left [ - e^{2 \rho} d \tau ^+d\tau^- + d \rho^2  +
 {1 \over 2} (\p_+^3 f) (d\tau^+)^2  +
 {1 \over 2} (\p_-^3 g) (d\tau^-)^2 \right].
}
The metric retains its asymptotic form, but we have kept track of the
subleading correction. This subleading correction changes the 
expectation value of the stress tensor. If we start with 
a zero stress tensor, we get 
\eqn{newexcp}{
\vev{T_{++}} \to { R \over 16 \pi G^{(3)}_N} \p_+^3 f
}
after the transformation.
Under a general conformal transformation, $ \tau^+ \to \tau^+ + f(\tau^+)$,
the stress tensor changes 
as 
\eqn{chstress}{
T_{++} \to T_{++} +  2 \p_+f T_{++} + f \p_+T_{++} + { c \over 24 \pi} 
\p_+^3 f.
}
So, comparing \chstress\ with \newexcp\ we can calculate the central
charge \brown .


It is also possible to show that if we have boundary 
conditions on the metric at infinity that  in the dual 
conformal field theory correspond to considering the theory on a 
curved geometry, then we get the right conformal anomaly 
\cite{Henningson:1998gx} (generalizing the discussion in section 
\ref{anomalies}). 

\section{The BTZ Black Hole}

Three dimensional gravity has no propagating degrees of freedom.
But, if we have a negative cosmological constant, we can have
black hole solutions. 
They are given by \cite{Banados:1992wn,Banados:1993gq}
\eqn{lorbtz}{
ds^2 = - { (r^2 - r^2_+)(r^2 - r^2_-) \over r^2} dt^2 
+{  R^2 r^2 \over  (r^2 - r^2_+)(r^2 - r^2_-)} dr^2 + 
r^2 ( d \phi + {r_+r_- \over r^2} dt)^2,
}
with $\phi \equiv \phi + 2 \pi $. 
We can combine the temperature $T$ and the angular momentum potential
$\Omega$ into 
\eqn{tleftr}{
{ 1 \over T_\pm} \equiv { 1 \over T} \pm { \Omega \over T}, 
}
and their relation to the parameters in \lorbtz\ is 
$ r_\pm = \pi R ( T_+ \pm T_-)$. 
The mass and angular momentum are
\eqn{massang}{
 8 G_N^{(3) } M = R + {(r_+^2 + r_-^2) \over R } ,~~~~~~~~~~J = 
{  r_- r_+   \over 4 G_N^{ (3) }  R   },
}
where we are measuring the mass relative to the 
$AdS_3$ space, which we define to have $M=0$ (the scale of the mass is
set by the radius of the circle in the dual CFT). 
This is not the usual convention, but it is much more
natural in this context since we are measuring energies with 
respect to the NS-NS vacuum.  
 Note that the mass of a black hole is always 
at least
\eqn{massmin}{
M_{min} = {R \over 8 G_N^{(3)}} = { c \over 12 }.
} 
The black hole with this minimum mass (sometimes called the zero mass
black hole) has a singularity at $r= r_+ = r_- =0$. 
All these black holes are locally the same as $AdS_3$ but they differ
by some global identifications \cite{Banados:1992wn,Banados:1993gq},
 i.e. they are quotients of 
$AdS_3$. 
In theories that have supersymmetry it can be checked that the 
zero mass black hole preserves some supersymmetries provided that
we make the fermions periodic as we go around the circle 
\cite{Coussaert:1994jp}, which 
is  something we have the freedom to do once the circle 
is not contractible in the gravity geometry.
These 
supersymmetries commute with the Hamiltonian conjugate to $t$. 
Furthermore, we will see below that if we consider the near horizon
geometry of branes wrapped
on a circle with periodic boundary conditions for the spinors, 
 we naturally obtain the BTZ black hole with 
mass $M_{min}$. This leads us to identify the $M = M_{min}$ 
BTZ black hole with the RR vacuum of the conformal field theory
\cite{Coussaert:1994jp}. 
The energy $M_{min}$ \massmin\  is precisely the energy difference
between the NS-NS vacuum and the RR vacuum. 
Of course, we could still have the $M=M_{min}$ BTZ black hole with
anti-periodic boundary conditions as
an excited state  in the 
NS-NS sector. 

Next, let us calculate the black hole entropy. The Bekenstein-Hawking
entropy formula gives 
\eqn{entgra}{
S = {{\rm Area} \over 4 G_N^{(3)}} =
 { 2 \pi r_+ \over 4 G_N^{(3)} }  =  { \pi^2  c \over 3} 
(T_+ + T_-),
} 
where we used \brown . 
We can also calculate this in the conformal field theory.
All we need is the central charge of the conformal field theory,
which we argued had to be \brown .
 Then, we can use the general formula \cite{Cardy:1986ie} for the 
growth of states in a unitary conformal field theory 
\cite{Strominger:1998eq,Maldacena:1998bw}, which gives
\eqn{entcft}{
S \sim 
{ \pi^2  c \over 3} 
(T_+ + T_-).
}
Thus, we see that the two results agree.
This result if valid for a general conformal field theory 
 as long as we are in the asymptotic high energy
regime (where energies are measured in units of the radius of the
circle), so in particular we need that  $ T \gg 1$. 
When is the result \entgra\ valid? In principle we would say that
it is valid as long as the area of the horizon is much bigger than
the Planck length, $ r_+ \gg G_N^{(3)}$. This gives 
$T \gg 1/c$, which is a much weaker bound on the 
temperature for large $c$.
%\footnote{Strictly speaking we  should   be considering
%either  
%a black hole in the RR sector, if we want a stable state, or 
% the black hole as some unstable, long lived 
%set of excitations.}.
So, we see that the corresponding
conformal field theory has to be quite special,
since the number of states should grow as determined by the asymptotics
\entcft\ for energies that are much smaller than one would expect 
for a generic  conformal field theory. 

\begin{figure}[htb]
\begin{center}
\epsfxsize=4.5in\leavevmode\epsfbox{rectang.eps}
\end{center}
\caption{
Calculation of the partition function at finite temperature 
through the Euclidean conformal field theory. Since the two directions
are equivalent we can choose the ``time'' direction as we wish.
The partition function is dual under $ \beta \to 4 \pi^2/\beta$.
 (a)~At low temperatures
$ \beta $ is large and only the vacuum propagates in the $\beta $
 direction. (b)~At high temperatures, small $\beta$, only the 
crossed channel vacuum propagates in the $\phi$ direction. 
(c)~When $\beta =2 \pi$ we have a sharp transition according to 
supergravity.
}
\label{rectang}
\end{figure}
 
A related  manifestation of this curious feature of the ``boundary'' 
conformal field theory  is the following. 
We could consider the canonical ensemble by going to Euclidean 
space and  making the Euclidean time coordinate periodic, $ \tau =
\tau + \beta$. We consider the case $\Omega =0$, the general
case is considered in \cite{Maldacena:1998bw}.
 The conformal field theory 
is then defined on a rectangular two-torus, and the free energy will 
be the partition function of the theory on this two-torus. 
Due to the thermal boundary condition in the NS sector, the 
two-torus ends up having   NS-NS boundary conditions on both circles.
In order to calculate the partition function in the dual gravitational 
theory 
 we should find a three-manifold that has the two-torus as its
 boundary (the correspondence tells us to sum over all
 such manifolds).
One possibility is to have the  original $AdS_3$ space but with time
identified, $ \tau = \tau + \beta$. 
%This would be describing a gas of particles in $AdS_3$ and 
The  value of the free energy is then given, to leading order,  
by the ground state energy of $AdS_3$.
This is the expected result 
for large $\beta$, where the torus is very elongated and only the 
vacuum propagates in the $\tau$ channel, see figure \ref{rectang}(a).
For high temperatures, only the vacuum propagates in the crossed 
channel (fig. \ref{rectang}(b)), and this corresponds to the BTZ black hole
in $AdS_3$. Note that the Euclidean BTZ  geometry is the same as $AdS_3$ but 
``on its side'', with $ \tau \leftrightarrow \phi$, so now
the $\tau$ circle is contractible.
 The transition between the two regimes occurs at $ \beta = 2 
\pi $, which corresponds to a square torus (fig. \ref{rectang}(c)).
This is a sharp transition when the gravity approximation is correct, 
i.e. when $ R/G_N^{(3)}\sim c  \gg 1 $.
 This sharp transition will not be present in the partition function of
a generic  conformal field theory, for example it is not present if 
we consider $c$ free bosons. When we discuss in more detail the conformal
field theories that correspond to string theory on $AdS_3$, 
we  will see that they have a  feature that makes it possible 
to explain  this transition. This sharp transition is the
two dimensional version of the large $N$  phase transition
discussed in section \ref{TPhaseT} \cite{Witten:1998zw}
(in this case $c$ plays the role of $N$). 


 


\section{Type IIB String Theory on $AdS_3 \times S^3 \times M^4$}

In this section we study IIB string theory on
 $AdS_3 \times S^3 \times M^4$ \cite{Maldacena:1998bw,deBoer:1998ip}. 
Throughout this section 
 $M^4 = K3$ or $ T^4$. 
In this case we can get some insight on the dual conformal 
field theory  by deriving this duality from D-branes, as we did
for the $AdS_5 \times S^5 $ case. 
We start with type IIB string theory on $M^4$. 
% This is a 
% six dimensional theory, which contains many string-like objects, each
% associated to a self-dual or anti-self dual three form field strength. 
We consider a set of  $Q_1$ D1 branes along a non-compact direction,
and $Q_5$ D5 branes wrapping $M^4$ and sharing the non-compact direction
with the D1 branes. All the branes are coincident in the transverse
non-compact directions. 
The unbroken Lorentz symmetry of this configuration is  $SO(1,1)\times
SO(4)$. $SO(1,1)$ corresponds to boosts along the string, and
$SO(4)$ is the group of rotations in the four non-compact directions
transverse to both branes.
This configuration also  preserves eight supersymmetries, actually
$\cn=(4,4)$
supersymmetry once we decompose them into left and right moving
spinors of $SO(1,1)$\footnote{
If $M^4 = K3$ we need that the sign of the D1 brane charge and the
sign of the D5 brane charge are the same, otherwise we break supersymmetry
(except for the single configuration with charges $(Q_5, Q_1) =
 ( \pm 1 , \mp 1)$).}.
It is possible to find the supergravity solution for this configuration
(see \cite{Youm:1997hw} for a review)
and then  take the near horizon limit as we did in section
\ref{correspondence}
\cite{Maldacena:1997re},
and we get the metric (in string frame)
\eqn{nearhorzn}{
{ d s^2 \over \alpha'} = 
{ U^2 \over g_6 \sqrt{Q_1Q_5} } ( -dt^2 + dx^2_1 ) + 
g_6 \sqrt{Q_1Q_5} { dU^2 \over U^2} + 
g_6 \sqrt{Q_1Q_5} d \Omega^2_3.
}
This is  $AdS_3 \times S^3$ with radius 
$R^2 = R_{AdS}^2 = R_{S^3}^2 = g_6 \sqrt{Q_1 Q_5} l_s^2$, where $g_6$ is
the six dimensional string coupling. The full ten dimensional 
geometry also includes an $M^4$ factor. In this case the volume  of 
the $M^4$  factor in the near-horizon geometry
is proportional to $ Q_1/ Q_5 $, and it is independent
of the volume of the original $M^4$ over which we wrapped the branes.
In the full D1-D5 geometry, which includes the asymptotically flat
region, the volume of $M^4$ varies, and it is equal to the above fixed value
in the near horizon region 
\cite{Andrianopoli:1996ve,Ferrara:1995ih,Ferrara:1996dd,Ferrara:1996um}. 



\subsection{The Conformal Field Theory}

The dual conformal field theory is the low energy field theory 
living on the D1-D5 system \cite{Maldacena:1996ky}.
One of the properties of this conformal field theory that we will need
is its central charge, so that we will be able to compare it with 
supergravity. We can calculate this central charge in a way that
is not too
dependent on the precise structure of the conformal field theory. 
The conformal field theory that we are interested in is the IR fixed
point of the field theory living on D1-D5 branes. The field theory
living on D1-D5 branes, before we go to the IR fixed point, is 
some $1+1$ dimensional field theory with $\cn=(4,4)$ supersymmetry. 
This amount of supersymmetry is equivalent to ${\cal N } = 2 $ in
four dimensions, so we can classify the multiplets in a similar fashion.
There is a vector multiplet and a hypermultiplet. In two dimensions
both multiplets have the same propagating degrees of freedom, four
scalars and four fermions, but they have different properties
under the $SU(2)_L \times SU(2)_R$ global R-symmetry. 
Under this group the scalars in the hypermultiplets are in the trivial
representation, while the scalars in the vector multiplet are in the
 ${\bf(2,2)}$. On the fermions these global symmetries act chirally. 
The left moving vector multiplet fermions 
are in the ${\bf (1,2)}$, and
the left moving hypermultiplet fermions are in the ${\bf (2,1)}$.  
The right moving fermions have similar properties with 
$SU(2)_L \leftrightarrow SU(2)_R$. 
The theory can have a Coulomb branch where the scalars in the 
vector multiplets have 
expectation values, and a Higgs branch where the scalars in the
hypermultiplets have expectation
values. 

From the spacetime origin of the supercharges it is clear that the
 $SU(2)_L \times SU(2)_R$ global R-symmetry is the same as the
SO(4) symmetry of spatial rotations in the 4-plane orthogonal to the
D1-D5 system \cite{Breckenridge:1996is,Vafa:1994tf,Vafa:1996bm}.
 The vector multiplets describe motion of the branes
in the transverse directions, this is consistent with their $SO(4)$ 
transformation properties. The vector multiplet ``expectation values''
should be zero if we want the branes to be on top of each other.
We have put quotation marks since expectation values do not  exist in a $1+1$
dimensional field theory. It is possible to show that if $Q_1$ and
$Q_5$ are coprime then,  
by turning on some of the $M^4$ moduli (more precisely some NS B-fields),
one can remove the Coulomb branch altogether, forcing the branes
to be at the same point in the transverse directions 
\cite{Dijkgraaf:1998gf,Seiberg:1999xz}. 

Since the fermions transform chirally under $SU(2)_L$, this theory 
has  a chiral anomaly. The chiral anomaly for 
$SU(2)_L$ is proportional to the number of left moving fermions 
minus the number of right moving fermions that transform under this 
symmetry. The 't Hooft anomaly matching conditions imply that 
this anomaly should be the same at high and low energies
\cite{'tHooft:1980xb}.
At high energies (high compared to the IR fixed point) 
 the anomaly is  $k_a = N_H - N_V$, the difference 
between the number of vector multiplets and hypermultiplets. 
Let us now calculate this, starting with the $T^4$ case. 
On a D1-D5 brane worldvolume there are massless excitations coming from 
(1,1) strings, (5,5) strings and (1,5) (and (5,1)) strings. 
The (1,1) or (5,5) strings come from a vector multiplet of an
 $\cn=(8,8)$ theory,
which gives rise to both a vector multiplet and a hypermultiplet of
 $\cn=(4,4)$ supersymmetry, so they do not contribute to the
anomaly.
 The massless modes of the (1,5) strings 
come only in hypermultiplets, and they contribute to the 
anomaly with $k_a = Q_1 Q_5$. 
For the K3 case the analysis is similar. The D5 branes are 
now wrapped on K3, so the (5,5) strings give rise only to a vector
multiplet. The difference from the $T^4$ case comes from the fact
that in the $T^4$ case the (5,5) hypermultiplet came from Wilson lines on
the torus, and on K3 we do not have one-cycles so we do not have
Wilson lines. On the fivebrane worldvolume there is (when it is
 wrapped on K3) an induced
one-brane charge equal to $Q_1^{ind} = - Q_5 $. The total D1 brane
charge is equal to the sum of the charges carried by 
explicit D1 branes and this negative induced charge,
$Q_1 = Q_1^{ind} + Q_1^{D1}$ \cite{Bershadsky:1996qy}.
Therefore, the number of D1 branes
is really $Q_1^{D1} = Q_1 + Q_5$, and the number of (1,5) strings
is $Q_1^{D1} Q_5 $. So, we conclude that the anomaly is
$k_a = Q_1^{D1} Q_5  - Q_5^2 = Q_1 Q_5 $, which in the end is the same
result as in the $T^4$ case.
Note that in order to calculate this anomaly we only need to know
the massless fields, since all massive fields live in larger
representations which are roughly like a vector multiplet plus a
 hypermultiplet, and
therefore they do not contribute to the anomaly.  
%In the IR limit the Higgs branch will decouple from the Coulomb 
%branch (if certain other conditions are met, we will make this
%more precise later).

When we are on the Higgs branch all the vectors become massive 
except for the center of mass multiplet, which contains fields
describing the overall motion of all the branes in the four 
transverse directions. This is just a free multiplet, containing
four scalar fields. 
 On the Higgs branch, at the  
 IR fixed point, the $SU(2)_L$ symmetry becomes a current 
algebra with an anomaly $k_{cft}$. 
The total anomaly should be the same, so that $k_a = k_{cft} -1$.
The last term comes from the center of mass $U(1)$ vector multiplet
(which is not included in $k_{cft}$). 
So, we conclude that $k_{cft} = Q_1Q_5 +1 $. 
Since the $U(1)$ vector multiplet is decoupled, we drop it in the rest
of the discussion and we talk only about the conformal field theory
of the hypermultiplets. 
The $\cn=(4,4)$ superconformal
symmetry relates the anomaly in the $SU(2)$ current algebra to the
central charge,  $ c = 6k_{cft} = 6 ( Q_1Q_5 +1) $. 
Using the value for
the $AdS_3$ radius $R = ( g_6^2 Q_1 Q_5)^{1/4} l_s$ and the three dimensional
Newton constant $G_N^{(3)} = g_6^2 l_s^4 / 4 R^3$, we
 can now check that \brown\ is satisfied to leading order for large $k$.
This also ensures, as we saw above, that the black hole entropy comes
out right. 
%In the $T^4$ case there is also a U(1) center
%of mass for motion on the four torus \cite{Maldacena:1999bp}
% and then it looks like we could 
%be  taking
%out another factor there,
%in the supergravity description they appear as singleton  fields
%living on the boundary of $AdS_3$ \cite{Fronsdal:1982gq,deBoer:1998ip}. 
%In any case they are subleading 
%corrections to the large $Q$ behaviour. In what follows we will 
%not discuss them explicitly.


\begin{figure}[htb]
\begin{center}
\epsfxsize=3in\leavevmode\epsfbox{instantons.eps}
\end{center}
\caption{
(a) The D1 branes become instantons on the D5 brane gauge theory.
(b)~The instanton moduli can oscillate in time and along $x_5$. 
}
\label{instantons}
\end{figure} 


Now we will try to describe this conformal field theory a bit 
more explicitly. 
We start with $ Q_5$ D5 branes, and we view the D1 branes as
instantons of the low-energy SYM theory
on the five-branes \cite{Douglas:1995bn}. 
These instantons live on $M^4$ and
are translationally invariant (actually also $SO(1,1)$ invariant) along
 time and the $x_5$ direction, where $x_5$ is the non compact direction 
along the 
$D5$ branes.  See figure \ref{instantons}(a).
This instanton configuration, with
instanton number $Q_1$, has  moduli, which are the parameters that 
parameterize a continuous family of solutions (classical instanton
configurations). All of these solutions
have the same energy.
Small fluctuations of this configuration (at low energies)
are described by fluctuations of the  instanton moduli. These moduli
can fluctuate in time as well as in the $x_5$ direction. See figure 
\ref{instantons}(b).
So, the low energy dynamics is given by a $1+1$ dimensional sigma model
whose target space is the instanton moduli space. 
Let us be slightly 
more explicit, and choose four coordinates $x^6,...,x^9$
parameterizing $M^4$. The instantons are described in the UV SYM
theory as $SU(Q_5)$ gauge fields
$A_{6,7,8,9}(\xi^a; x^6,...,x^9)$  with field strengths which 
satisfy $F = *_4 F$, where $*_4$ is the epsilon  symbol in $M^4$ and
 $\xi^a $ are the moduli  parameterizing
the family of instantons.
 The dimension of the instanton moduli space
for $Q_1$ instantons in $SU(Q_5)$   is
$4k$, where
\eqn{defofk}{
k \equiv Q_1Q_5 ~~~~{\rm for ~ }T^4, ~~~~~~~~  
k \equiv Q_1Q_5 +1 ~~~~{\rm for ~ }K3.
}
The leading 
behaviour for large $Q$ is the same. In the $T^4$ case we have four
additional moduli coming from the Wilson lines of the $U(1)$ factor
of $U(Q_5)$ \cite{Maldacena:1999bp}. 
It has been argued  in \cite{Vafa:1996bm,Witten:1997yu}
 that the instanton moduli space is 
a deformation of the symmetric product of $k$ copies of $M^4$,
 $Sym(M^4)^k \equiv (M^4)^k/S_k$. The deformation involves blowing up 
the fixed points of the orbifold, as well as modifying the
$B$-fields that live at the orbifold point. We will discuss this 
in more detail later.
The parameter that blows up the singularity can be identified
with one of the supergravity moduli of this solution.
For some particular value of these moduli (which are not to be
confused with the moduli of the instanton configuration)
the CFT will be precisely the
symmetric product, but at that point the gravity approximation will
not be valid, since we will see that the supergravity description
predicts fewer states at low conformal weights than the symmetric
product CFT. When we deform the symmetric product, some of the states 
can get large  corrections and have high energies (i.e. they
correspond to operators 
having high conformal weight). 
Other studies of this D1-D5 system include 
\cite{Costa:1998nq,Hassan:1997ai,Hassan:1997sf}

\subsection{Black Holes Revisited}

We remarked above that the BTZ black hole entropy can be calculated 
just from the  value of the central charge, and therefore the 
gravity result agrees with the conformal field theory result. 
Note that the calculation of the central charge that we did 
above in the CFT is valid for any value of the coupling (i.e. the
moduli), so the field theory 
calculation of the central charge and the entropy
is valid also in the black hole regime (where the gravity
approximation is valid).
This  should be contrasted to the  
 $AdS_5 \times S^5$ case,  where the field theory calculation of the
entropy was only done at weak coupling (in two dimensions the entropy
is determined by the central charge and cannot change as we vary moduli). 
In \cite{Behrndt:1998nt} corrections to the central charge 
in the gravity picture
were analyzed. 

We noticed  above that the gravity description predicted a sharp 
phase transition when the temperature was $ T = 1/(2\pi) $,  and
we remarked that the field theory had to have some special properties
to make this happen.  
We will now explain qualitatively this phase transition.
Our discussion 
will be qualitative because we will work at the orbifold point, and
this is not correct 
if we are in the supergravity regime.  We  will see that
the symmetric product has a feature that makes this sharp 
phase transition possible.


The orbifold theory can be interpreted  
in terms of a gas of strings \cite{Dijkgraaf:1996xw,Dijkgraaf:1997cv}.
These are strings that wind along $x_5$  and move on $M^4$. The total
winding number is $k$. The strings can be singly wound or 
multiply wound. 
In the R-R sector it does not cost any energy to multiply wind the 
strings. 
 If we have NS-NS boundary conditions, which are the appropriate
ones to describe $AdS_3$, 
it will cost some energy to multiply wind the strings. 
The energy cost in the orbifold CFT
is the same as twice the  conformal weight of the 
corresponding twist operator, which is $ h = \bar h = w/4  + O(1/w)$
for a configuration with winding number $w$. 
 If the strings are singly wound and we have a temperature
of order one (or $1/2\pi$), we will not have
many oscillation modes excited on these strings, and the 
entropy will be small. Note that the fact that we have many singly
wound strings
does not help, since we are supposed to symmetrize over all strings,
so most of the strings will be in similar states and they will not
contribute much to the entropy. 
So, the free energy of such a state is basically $F \sim 0  $. 
On the other hand, if we multiply wind all the strings, we raise the 
energy of the system but we also increase the entropy 
\cite{Maldacena:1996ds}, since now the energy gap of the system
will be much lower (the multiply wound strings behave effectively like
a field theory on a circle with a radius which is $w$ times bigger).  
If we multiply 
wind $w$ strings, with $w \gg 1 $, 
 we get an energy $E \sim w/2 + 2 \pi^2 w T^2 $, where
the last term comes from thermal excitations along the string.
The entropy is also  larger, $ S = 4 \pi^2 w T $. 
So, the free energy is $F = E-TS=  w/2 -  2 \pi^2 w T^2 $. Comparing this
to the free energy of the state with all strings singly wound,
we see that the latter  wins when
$T<1/(2\pi)$, and the multiply wound state wins when $T>1/(2 \pi) $.
 This explains the presence of the sharp phase
transition at $T = 1/(2 \pi)$ when we are at  the orbifold point. 

%If we go away from the 
%orbifold point, many states that had low conformal weights gain larger
%conformal weights, and this will reduce the free energy for low temperatures. 
%Since the high temperature behavior is given by doing a modular
%transformation, we see that going away from the orbifold point will just
%make the transition sharper. 
Note that the mass of the black hole at the transition point
is $M = M_{min} + k/2 $, which is (for large $k$) much bigger 
than the minimum mass for a BTZ black hole, like the situation in
other $AdS_{d>3}$. 
We could have black holes which are smaller than this, but they 
cannot be in thermal equilibrium with an external bath.
Of course they could be in equilibrium inside $AdS_3$ if we
do not couple $AdS_3$ to an external bath to keep the temperature
finite. In this case we are considering the microcanonical ensemble,
and there are more black hole solutions that we could be 
considering \cite{Banks:1998dd,Li:1999jy,Martinec:1999ja}. 

\begin{figure}[htb]
\begin{center}
\epsfxsize=4.5in\leavevmode\epsfbox{wind.eps}
\end{center}
\caption{Some configurations with winding number four.
(a)~Two singly wound strings and one doubly wound string.
(b)~A maximally multiply wound configuration.
}
\label{wind}
\end{figure} 



If we were considering the conformal field theory on a circle with
RR boundary conditions, the corresponding supergravity background 
would be the $M =M_{min}$ BTZ black hole. 
%Note that $M_{min} = c/12$ is the correct energy difference 
%between the RR and the NS-NS ground states. 
 This follows from the 
fact that we should have preserved supersymmetries that commute 
with the Hamiltonian (in $AdS_3$ the preserved supersymmetries
do not commute with the Hamiltonian generating evolution in
global time). In order to have these supersymmetries we need to
have RR boundary conditions on the circle. Notice that the RR vacuum
is not an excited state on the NS-NS vacuum, it is just in a different
sector of the conformal field theory, even though the $M= M_{min}$ 
BTZ black hole appears in both sectors. 

In the case with RR boundary conditions a black hole forms as soon 
as we raise the temperature (beyond $T \sim 1/k$). 
This seems at first sight
paradoxical, since the temperature could be much smaller than 
one, which would be the natural energy gap for a generic
conformal field theory on a circle. The reason that the 
energy gap is very small for this conformal field theory 
is due to the presence
of ``long'', multiply wound strings. 
In the RR sector all multiply wound strings
have the same energy. But, as we saw before, multiply wound strings
lead to higher entropy states so they are preferred. In fact, one
can estimate the energy gap of the system by saying that 
it will be of the order of the minimum energy excitation that can 
exist on a string multiply wound $k$-times, which is of the order of
$1/k$. This estimate of the energy gap agrees with 
a semiclassical estimate as follows. We can trust the thermodynamic
approximation for black holes as long as the specific heat is 
large enough \cite{Preskill:1991tb}.
 For any system we need a large specific heat, $C_e \equiv
{\partial E \over \partial T}$,  in 
order to trust the thermodynamic approximation.
% For example the
%energy change due to the emission of a particle with typical thermal
%energy $T$ will change the temperature of the system by an amount
%$\delta T = \delta E/c_e \sim T/c_e $ where $c_e$ is the specific 
%heat of the system $c_e = {\partial E \over \partial T}$. 
In this case $E \sim k T^2$, so the condition $C_e \gg1$ boils down to
$E\gg 1/k$ . So, this estimate of the energy gap agrees with the 
conformal field theory estimate. 
Note that in the RR supergravity vacuum (the $M=M_{min}$ black hole)
we could seemingly have arbitrarily low energy excitations as 
waves propagating on this space. The boundary condition on these waves
at the singularity should be such that one gets the above gap, but
in the gravity approximation $k =\infty$ and this gap is not seen. 
Note also that the $M= M_{min}$ black hole does not correspond to
a single state (as opposed to the $AdS_3$ vacuum), but to a large
number of states, of the order of $e^{ 2 \pi \sqrt{2 k} }$ for 
$T^4$ case and  $e^{ 2 \pi \sqrt{4 k} }$ for $K3$\footnote{ 
 An easy way
to calculate this number of BPS states 
is to consider this configuration as a system
of D1-D5 branes on $S^1\times M^4 $ and then do a U-duality 
transformation, transforming this into 
a system of fundamental string momentum and winding.}.

There are other black holes that preserve some supersymmetries,
which are extremal BTZ black holes with $M-M_{min} = J$
\cite{Coussaert:1994jp}. $J$ is the angular momentum in $AdS_3$,
identified with the
momentum along the $S^1$ in the CFT. 
Of course, these black holes will preserve supersymmetry only if the boundary
conditions on $S^1$ are periodic, i.e. only if we are considering
the RR sector of the theory. In the RR sector it becomes more 
natural to measure energies so that the RR vacuum has zero energy.
The extremal black holes correspond to states in the CFT in the 
RR sector with no left moving energy, $\bar L_0=0$, and some right
moving energy, $ L_0 =J>0$. The entropy of these states is
\eqn{entropycft}{
S= 2 \pi \sqrt{ k  J}.
}
This is the entropy as long as $k L_0$ is large, even for
$L_0 = 1$. The reason for this is again the presence of 
multiply wound strings, that ensure that the asymptotic 
formula for the number of states in a conformal field theory
is reached at very low values of $L_0$. In this argument it
is important that we are in the RR sector, and since we 
are counting BPS states we can deform the theory until we 
are at the symmetric product point, and then the argument
we gave in terms of multiply wound strings is  rigorous
\cite{Strominger:1996sh,Maldacena:1999bp}. 

It is possible to consider also black holes which carry angular momentum
on $S^3$. They are characterized by the eigenvalues $J_L, ~J_R$, of 
$J_L^3$ and $J_R^3$ of $SU(2)_L\times SU(2)_R $. 
These rotating black holes can be found by taking the near horizon
limit of rotating black strings  in six dimensions 
\cite{Cvetic:1998xh,Cvetic:1996xz}.
Their metric is locally $AdS_3\times S^3$ but with some
discrete identifications \cite{Cvetic:1998ja}.
Cosmic censorship implies that
their mass has a lower bound 
\eqn{masslb}{
E \equiv M-M_{min} \geq J_L^2/k + J_R^2/k.
}
We can also calculate the entropy for a general configuration 
carrying angular momenta $J_{L,R}$
 on $S^3$, linear momentum $J$ on $S^1$, and energy $E = M-M_{min}$ :
\eqn{entropyrot}{
S = 2 \pi \sqrt{k( E+J)/2 - 
J_L^2} +  2 \pi \sqrt{k( E-J)/2 - 
J_R^2}.
}
We can understand this formula in the following way 
\cite{Breckenridge:1996sn,Breckenridge:1996is}. 
If we bosonize the $U(1)$ currents, $J_{L} \sim {k \over 2} 
\partial \phi $,
and similarly for $J_R$, we can construct the operator
$e^{i J_L \phi}$ with conformal weight $J_L^2/k$. This explains
why the minimum mass is \masslb . This also explains \entropyrot,
since  only a portion of the energy equal to 
 $ L_0-J^2_L/k  = ( E+J)/2-J^2_L/k $ can be distributed freely among
the oscillators\footnote{ Other
black holes were studied in \cite{Kaloper:1998vw}.}. 




\subsection{Matching of Chiral-Chiral Primaries }

The CFT we are discussing here, and also its string theory dual, 
have moduli (parameters of the field theory). At some point in the
moduli space
the symmetric product description is valid, and at that point 
the gravity description is strongly coupled and cannot be trusted. 
As we move away from that point we can get to regions in moduli
space where we can trust the gravity description. The energies 
of most states will change when we change the moduli. 
There are, however, states 
that are protected, whose energies are not changed. 
These are chiral primary states \cite{Lerche:1989uy}. 
The superconformal algebra contains  terms of the form\footnote{
Our normalization for $J^3_0$ follows the standard $SU(2)$ practice
and differs by a factor of two from the U(1) current in 
\cite{Lerche:1989uy,Maldacena:1998bw,deBoer:1998ip,deBoer:1998us}.}
\eqn{suprc}{\eqalign{
 \{ Q^{++}_r ,Q^{--}_s \} &= 2 L_{r+s} + 2(r-s) J^3_{r+s} +
{ c\over 3 } \delta_{r+s} (r^2 -{1 \over 4} ), \cr
\{ Q^{+-}_r ,Q^{-+}_s \} &= 2 L_{r+s} +2 (r-s) J^3_{r+s} +
{ c\over 3 } \delta_{r+s} (r^2 -{1 \over 4} ),
 }}
where $Q^{\pm\pm}_r = (Q^{\mp\mp}_{-r})^\dagger $, and $r,s \in
\IZ + {1 \over 2} $. The generators that belong to the global
 supergroup (which leaves the vacuum and $AdS_3$ invariant)
have $r, s = \pm 1/2$.  The first superscript indicates 
the eigenvalues under the global $J_0^3$ generator of $SU(2)$, and
the second superscript corresponds to a global $SU(2)$
 exterior automorphism
of the algebra which is not associated to a symmetry in the theory. 
If we take a state $|h\rangle $
 which has $L_0 = J^3_0$, then we see from \suprc\ 
that $ Q^{+\pm}_{-1/2} |h\rangle $ 
has zero norm, so in a unitary field theory
it should be zero. Thus, these states are annihilated by $Q^{+ \pm}_{-1/2}$.
Moreover, if a state is annihilated by $Q^{+ \pm}_{-1/2}$ then 
$L_0 =  J^3_0 $. These states are called right chiral primaries, and
if  $\bar L_0 = {\bar J}_0^3$ it is a left chiral primary. 
  The possible values of $J_0^3 $ for 
chiral primaries are  bounded by  $J_0^3 \leq c/6 =k $. 
This can be seen by 
 computing  the norm of $ Q_{-3/2}^{+\pm} |h\rangle$.
 Note that $k$ is the
level of the $SU(2)$ current algebra. The values of $J_0^3$ for generic 
states are not bounded. The spins of $SU(2)$ current algebra
primary fields are bounded by $J_0^3 \leq k/2 $, which is {\it not}
 the same as the bound on chiral primaries. 

Let us now discuss the structure of the supermultiplets under the
$SU(1,1|2)$ subgroup of the ${\cal N} = 4 $ algebra 
\cite{Gunaydin:1986fe}. This is the subgroup
generated by the supercharges with $r,s = \pm 1/2 $ in \suprc , plus
the global $SU(2)$ generators $J_0^a$ and the $SL(2,\IR)$ subgroup of the
Virasoro algebra. The structure of these multiplets is the following. 
By acting with $Q_{1/2}^{\pm\pm}$ on a state we lower 
its energy, which is the $L_0$ eigenvalue. Energies are all positive
 in a unitary conformal field theory, since  $L_0$ eigenvalues
are related to scaling dimensions of fields which should be
positive. So, we conclude that at some point $Q_{1/2}^{\pm\pm}$ will
annihilate the state. Such a state is also annihilated by $L_1$ \suprc .
 We call such a state a primary, or highest
weight, state. Then, we can generate all other states by acting
with $Q_{-1/2}^{\pm\pm}$. See figure \ref{multiplet}.
This will give in general a set
of $1 +4 + 6 + 4 + 1$ states, where we organized the states according
to their level. On each of these states we can then act with
arbitrary powers of $L_{-1}$. 
However, we could also have a short representation where some of the
$Q_{-1/2}$ operators
annihilate the state. This will happen when $L_0 =\pm J_0^3$,
i.e. only when we have a chiral primary (or an antichiral primary). 
Since by $SU(2)$ symmetry each chiral primary comes with an antichiral
primary, we concentrate on chiral primaries.  
These short multiplets are of the form
\begin{equation}
\begin{array}{rcc}
{\rm states} & J_0^3  &  L_0 \\
|0\rangle & j  & j \\
Q^{-\pm}_{-1/2} |0\rangle & j-1/2 &  j+1/2 \\
Q^{-+}_{-1/2} Q^{--}_{-1/2} |0\rangle & j-1 & j+1. 
\end{array}
\end{equation}
The multiplet includes four states (which are $SL(2,\IR)$ primaries), 
except in the case that $j=1/2$ when
the last state is missing. 
We get a similar structure if we consider the right-moving
part of the supergroup. 

\begin{figure}[htb]
\begin{center}
\epsfxsize=2in\leavevmode\epsfbox{multiplet.eps}
\end{center}
\caption{ Structure of $SU(1,1|2)$ multiplets. 
We show the spectrum of possible $j$'s and conformal 
weights. We show only the $SL(2,\IR)$ primaries that appear
in each multiplet and their degeneracies. The minus sign
denotes opposite statistics. The full square is a long 
multiplet. The encircled states form a short multiplet. 
Four short multiplets can combine into a long multiplet.
}
\label{multiplet}
\end{figure} 


We will first consider states that are left and right moving chiral
primaries, with $L_0 = J^3_0 $ and $\bar L_0 = \bar J_0^3 $.
From now on we drop the indices on $J^3_0, \bar J_0^3$,
and denote the chiral primaries by $(j,\bar j)$. 
By acting with $Q^{-\pm}_{-1/2}$ and $\bar Q^{-\pm}_{-1/2}$
we generate the whole supermultiplet.
We will calculate the spectrum of chiral-chiral primaries both 
in string theory (in the gravity approximation)
and in the conformal field theory at the orbifold point. Since these states
lie in short representations we might expect that
they remain in short representations also
after we deform the theory away from the orbifold point. 
Actually this argument is not enough, since in principle
short multiplets could combine and become long multiplets.
In the K3 case we can give a better argument. We will see that
all chiral primaries  that appear are bosonic in nature, while we
see from figure \ref{multiplet} that we need some bosonic and 
some fermionic chiral primaries to make a long multiplet.
Therefore, all chiral primaries must remain for any value of the
moduli.
  
Let us start with the conformal field theory. Since these states
are protected by supersymmetry we can go to the orbifold point
 $Sym(M^4)^k$. The chiral primaries in this case can be understood 
as follows. In a theory with $\cn=(4,4)$ 
supersymmetry we can do calculations
in the RR sector and then translate them into results about the
NS-NS sector. This process is called ``spectral flow'', and it 
amounts to an automorphism of the ${\cal N} = 4$ algebra. 
Under spectral flow, the chiral primaries of the NS-NS sector (that 
we are interested in) are in one
to one correspondence with the ground states of the RR sector.
It is easier to compute the properties of the RR ground states of
the theory. Orbifold conformal field theories, like $Sym(M^4)^k$, can
be thought of as describing a gas of strings winding on a circle,
the circle where the CFT is defined, with total winding
number $k$ and moving on $M^4$. 
The ground state energies
 of a singly wound string and a multiply wound
string are the same if we are in the RR sector. 
Then, we can calculate a partition function over the RR ground states.
It is more convenient to relax the constraint on the  total winding 
number by introducing a chemical potential for the winding number,
and then we can recover the result with fixed winding number by extracting 
 the appropriate term in the partition function as in 
\cite{Dijkgraaf:1996xw}. 
Since our conformal field theory has fixed $k$ we will be implicitly 
assuming that we are extracting the appropriate term  from the 
partition function. 
The RR ground states for the strings moving on $M^4$ are the same
as the ground states of a quantum mechanical supersymmetric sigma 
model on $M^4$. It was shown by Witten \cite{Witten:1982df}
 that these are in 
one-to-one correspondence with the harmonic forms on $M^4$.
Let us denote by $h_{rs}$ the number of harmonic forms of holomorphic
degree $r$ and antiholomorphic degree $s$. States with
degree $r+s$ odd are fermionic, and states with $r+s$ even are bosonic.
In the case of $K3$ $h_{00} = h_{22} = h_{20}=h_{02}=1$ and 
$h_{11} = 20$. In the case of $T^4$ $h_{00} = h_{22} = h_{20}=h_{02}=1$,
$h_{01}= h_{10} = h_{12} = h_{21} = 2 $, and $h_{11} = 4$. 
A form with degrees $(r,s)$  gives rise to a state with 
angular momenta $ (j,\tilde j) = ( (r-1)/2,(s-1)/2)$.
The partition function in the RR sector  becomes \cite{Dijkgraaf:1996xw}
\eqn{partram}{
\sum_{k\geq 0}  p^kTr_{Sym(M^4)^k} 
[(-1)^{2J + 2 \bar J}  y^{J} {\bar y}^{ \bar J}] = 
{ 1 \over \prod_{n\geq 1}\prod_{r,s} ( 1 -p^n y^{(r-1)/2} 
{\bar y}^{(s-1)/2})^{
(-1)^{r+s} h_{rs} }},
}
where the trace is over the ground states of the RR sector.
Spectral flow boils down to the replacement 
$p\to p y^{1/2} {\bar y}^{1/2}$. Thus, we get the NS-NS partition function,
giving a prediction for the chiral primaries,
\eqn{chiralp}{
\sum_{k}  p^kTr_{Sym(M^4)^k} 
[(-1)^{ 2 J + 2 \bar J }  y^J {\bar y}^{\bar J}] =
{ 1 \over \prod_{n\geq 0}\prod_{r,s} 
( 1 -p^{n+1} y^{(n+r)/2} {\bar y}^{(n+s)/2})^{(-1)^{r+s} h_{rs} }},
}
where here the trace is over the chiral-chiral primaries in the NS-NS
sector.

Now, we should compare this with supergravity. 
In supergravity we start by calculating the spectrum 
of single particle chiral-chiral primaries. 
We then calculate the full spectrum by considering multiparticle states.
Each single particle state  contributes with a factor 
$(1- y^{j} {\bar y}^{\bar j})^{-d(j,\bar j)} $ to the partition function,
were $d(j,\bar j)$ is the total number of single particle states with 
these spins.
The supergravity spectrum was calculated in 
\cite{Maldacena:1998bw,Deger:1998nm,Larsen:1998xm,deBoer:1998ip}.
The number of single particle states is given by 
\eqn{grav}{
\sum_{j,\bar j }   d(j,\bar j ) y^j { \bar y}^{\bar j}  =
 \sum_{n,r,s \geq 0} h_{rs}  y^{ n + r \over 2}
{\bar y}^{n+ s \over 2 }  -1.
}
We  have excluded  the identity, which is not represented by any state
in supergravity.
So, the gravity partition function is given by 
\eqn{gravitypa}{
Tr_{Sugra}[(-1)^{ 2 J + 2 \bar J }  y^J {\bar y}^{\bar J}]_{\rm 
c-c~primaries} =
{ 1 \over \prod_{n\geq 0}\prod'_{r,s} 
( 1 - y^{(n+r)/2} {\bar y}^{(n+s)/2})^{(-1)^{r+s} h_{rs} }},
}
where $\prod'$ means that we are not including the term with $n=r=s=0$. 

Let us discuss some the particles appearing in \grav\ and \gravitypa\
 more explicitly.
Some of them are special because
 they carry only left moving quantum numbers or only right moving
quantum numbers. For example, we have the $(0,1)$ and $(1,0)$ states
that are related to the $SU(2)_L$ and $SU(2)_R $ gauge fields on $AdS_3$.
These $SU(2)$ symmetries come from the $SO(4)$ isometries of the 
3-sphere. 
These gauge fields have a Chern-Simons action 
\cite{Achucarro:1986vz,Achucarro:1989gm} and they 
give rise to $SU(2)$ current algebras on the boundary 
\cite{Witten:1989hf,Elitzur:1989nr}. 
The chiral primary in the current algebra is the operator
$J_{-1}^+$, which has the quantum numbers mentioned above. 
When we apply $Q^{--}_{-1/2} Q^{-+}_{-1/2}$ to this state we 
get the left moving stress tensor. Again, this should correspond to part 
of the physical modes of gravity on $AdS_3$. Pure gravity in three
dimensions is a theory with no local degrees of freedom. 
In fact, it is equivalent to an $SL(2,\IR)\times SL(2,\IR)$ Chern-Simons
theory 
\cite{Achucarro:1986vz,Howe:1996zm,%
Witten:1988hc,Witten:1989sx}.
 This gives rise to some physical degrees of freedom
living at the boundary. It was argued that we get a Liouville
theory at the boundary 
\cite{Coussaert:1995zp,Banados:1998pi,Banados:1998ta,%
Banados:1996tn,Martinec:1998st}, which includes a stress tensor
operator.
In the $T^4$ case we also have some other special particles
which correspond to fermion zero modes $(1/2,0)$ and $(0,1/2)$. 
These fermion zero modes are the supersymmetric partners of the $U(1)$ 
currents associated to isometries of $T^4$.
The six dimensional  theory corresponding to type IIB string theory on
$T^4$ has 16 vector fields transforming
in the spinor  of $SO(5,5)$. From the symmetric product we get only
8 currents ($4_L + 4_R$). The other eight are presumably related to
an extra copy of $T^4$ appearing in the CFT due to the Wilson
lines of the $U(1)$ in $U(Q_5) $ \cite{Maldacena:1999bp,Larsen:1999uk}.  

Besides these purely left-moving or purely right-moving modes,
which are not so easy to see in supergravity,
all other states arise as local bulk  excitations
of  supergravity fields on $AdS_3$ and are clearly present. 
Higher values of $j$ typically correspond to higher Kaluza-Klein 
modes of lower $j$ fields. More precisely, we have 
$n$ $(1/2,1/2) $ states where $n = h_{11} + 1$ 
\cite{Maldacena:1998bw,Deger:1998nm,deBoer:1998ip}. 
By applying $Q$'s, each of these states
gives rise to four $SU(2)$-neutral scalar fields, which
 have conformal weights
 $h = \bar h = 1$. Therefore, they correspond to 
  massless fields in spacetime
by (\ref{dimenmass}).
These are the   $ 4 n$ 
moduli of the supergravity compactification, which are identified with
the moduli of
the conformal field theory. In the conformal field theory 
$4 h_{11}$ of them correspond to deformations of each copy of $M^4$
in the symmetric product, while the extra four are associated 
to a blowup mode, the blowup mode of the $\IZ_2$ singularity that
arises when we exchange two copies of $M^4$. 
Next, we have $n+1$ fields  with 
quantum numbers $(1,1)$, $n $ of these are higher order
Kaluza Klein modes of the $n$ fields we had before, and the new one
corresponds to deformations of the $S^3$. 
Each of these states gives rise to 
$SU(2)$-neutral fields with positive mass, since we have to apply 
$Q$'s twice and we  get  $h =\bar h =2$. 
These are the $n$ fixed scalars of the supergravity background
plus one more field related to changing the size of the $S^3$. 
The fields with  $j,\bar j$'s  above these values are just higher
Kaluza Klein modes of the fields we have already mentioned 
explicitly. See \cite{Maldacena:1998bw,Deger:1998nm,deBoer:1998ip}
for a more systematic treatment and 
derivation of these results. 

Now, we want to compare the supergravity result with the gauge theory
results. 
In \chiralp\ there is an ``exclusion principle'' since the total
power of $p$ has to be $p^k$, thus limiting the total number of 
particles. In supergravity \gravitypa\ we do not have any indication 
of this exclusion principle. Even if we did not know 
about the conformal field theory, from the fact that
there is an ${\cal N } = 4$ superconformal spacetime symmetry
we  get  a bound on the angular momentum of the chiral primaries
$j \leq k $.
%\footnote{ This is NOT the same 
%as the bound on $SU(2)$ current
%algebra primaries which is $j \leq k/2$}.
However, this bound is less
restrictive than implied by  \chiralp . There are 
multi-particle states with $j< k$ that are excluded from 
\chiralp .  The bounds from \chiralp\  appear for very large angular 
momenta and, therefore, very large energies, where we would not
necessarily trust the gravity approximation.
In fact, the gravity result and the conformal field theory result
match precisely, as long as the conformal weight or spin of 
the chiral primaries  
is $j,{\bar j} \leq k/2 $. 
One can show that the gravity description exactly matches the 
$k \to \infty$ limit of \chiralp\ \cite{deBoer:1998ip}. 
 This limit is extracted from \chiralp\  by noticing that
there is a factor of $(1-p)$ in the denominator, which  is related to the
identity operator. So, we can extract the $k  \to \infty$ limit 
by multiplying \chiralp\ by $(1-p)$ and setting $p\to 1$. 
In principle, we could get precise agreement between the conformal field
theory calculation and the supergravity calculation if we incorporate the
exclusion principle by assigning a ``degree'' to each supergravity
field, as explained in \cite{deBoer:1998us}, and then considering only 
multiparticle states with degree smaller than $k$. 
% We do not regard this
%as very natural from the supergravity point of view since for example
%it involves selecting one of the $n$ (1/2,1/2) fields and assigning
%degree two to it while degree one to the rest.
One can further wonder whether there is something special that
happens at $j=k/2$, when the exclusion principle starts making a
 difference.  Since we are considering states with 
high conformal weight and angular momentum it is natural to wonder 
whether there are any black hole states that could appear. 
There are black holes which carry angular momentum on $S^3$. 
These black holes are characterized by the two angular momenta
$J_L$, $J_R$, of $SU(2)_L\times SU(2)_R$. 
The minimum black hole mass for given angular momenta was given 
in \masslb,
$M_{min}(J_L,J_R) = k/2 + J^2_L/k + J^2_R/k$,
where we used $c  =6k$ and \brown .
We see that these masses are always bigger than the mass 
of the chiral primary states with angular momenta
$(J_L,J_R)$, except when $J_L = J_R = k/2 $. So we see that 
something special is happening at $j=k/2$, since at this
point a black hole appears as a  chiral primary state. 
Connections between this exclusion principle and quantum groups
and non-commutative geometry were studied in 
\cite{Jevicki:1999rr,Chang:1999jm}.

\subsection{Calculation of the Elliptic Genus in Supergravity}

We could now consider states which are left moving chiral 
primaries and anything on the right moving side. 
These states are also in small representations, and 
one might be tempted to compute the spectrum of chiral 
primaries at the orbifold point and then try to match it 
to supergravity. However,
this is not the correct thing to do, and in fact 
the spectrum does not match \cite{Vafa:1998nt}. 
It is not correct because some chiral primary
states could pair up and become very massive 
 non-chiral primaries.
In the case of chiral-anything states, a useful tool to 
count the number of states, which gives a result that is independent
of the deformations of the theory, is the ``elliptic genus'',
which is the partition function 
\eqn{elliptic}{
Z_k = Tr_{RR} [ (-1)^{ 2j + 2 {\bar j} }q^{L_0} {\bar q}^{{\bar L}_0}  
y^{j} ].
}
This receives contributions only from the left  moving ground states,
$\bar L_0 = 0$. These states map into (chiral, anything) under
spectral flow, i.e. states that are chiral primaries on the left moving 
side but are unrestricted on the right moving side.

The number of states contributing to the elliptic genus
goes like $e^{ 2 \pi \sqrt{n k} }$  for  large powers $q^n$.
This raised some doubts that \elliptic\ would agree with supergravity.
The elliptic genus diverges when we take the limit $k\to \infty $.
The origin of this divergence is the contribution of the $(2,0)$ form,
which is a chiral primary on the left but it carries zero 
conformal weight on the right. So, we get a contribution of order $k$ 
from the fact that this state could be occupied $k$ times without 
changing the powers of $q$ or $y$. The function that has a smooth
limit in the $k\to \infty $ limit is then $Z_k^{NS}/k $. 
In the K3 case this function is 
\eqn{ellsu}{
\lim_{k\to \infty} { Z_k^{NS} \over k} =
{  \prod_{m\geq 1} (1-q^{m/2} y^{1/2})^2(1-q^{m/2}y^{-1/2} )^2 
(1-q^{m/2})^{20}
 \over
\prod_{m\geq 1} (1 - q^{m/2} y^{m/2})^{24} (1 -q^{m/2} y^{-m/2} )^{24}  }.
}
We can now compare this expression to the supergravity result. In the
supergravity result we explicitly exclude the contribution of the
$(2,0)$ form, since it is directly related to the factor of $k$ that
we extracted, but we keep the contribution of the $(0,2)$ form and the 
rest of the fields. The supergravity result then agrees precisely with 
\ellsu\ \cite{deBoer:1998us}.
 Both in the supergravity calculation and in the conformal field
theory calculation at the orbifold point there are many fields of the 
form (chiral,anything), but most of them cancel out to give \ellsu.
For example, we can see that the only supergravity single particle
states  that 
contribute for large powers of  $y^{>1/2}$
 are the (chiral, chiral) and (chiral, antichiral) states.
One can further incorporate the exclusion principle in supergravity
by assigning degrees to the various fields, and then one finds
that the elliptic genus agrees up to powers of $q^h$ with $h \leq (k+1)/4$
 \cite{deBoer:1998us}. 
Here again this is the point where a black hole starts contributing
to the elliptic genus. It is an extremal
 rotating black hole with angular 
momentum $J_L = k/2$ and $J_R=0$, which has $L_0 = k/4$ and 
$\bar L_0 = k/2$. 




\section{Other $AdS_3$ Compactifications}

We  start by discussing the compactifications discussed in
the last section more broadly, and then we will discuss other
$AdS_3$ compactifications.
In the previous section we started out with type IIB string theory
compactified on 
$M^4$ to six dimensions. 
 The theory has many charges carried by string like objects,
which come from branes wrapping on various cycles of $M^4$.
These  charges transform as vectors 
under the duality group of the theory $SO(5,n)$, where 
$n = 21,5 $ for the $K3$ and $T^4$ cases respectively. 
These $5+n$ strings correspond to the fundamental and D strings, 
the NS and D fivebranes wrapped on $M^4$,
 and to D3 branes wrapped on the 
 $n+1$  two-cycles of $M^4$. 
A general charge configuration is given by a vector $q^I$ transforming
under $SO(5,n)$. The radius of curvature  of the gravity 
solution is proportional to $q^2$, $R^4 \sim q^2$,
 where we use the $SO(5,n)$ metric. 
In the K3 case $q^2 >0$ for  supersymmetric configurations. 
The six dimensional space-time theory
has $5n$ massless scalar fields, which parameterize the coset manifold
$SO(5,n)/SO(5)\times SO(n)$ \cite{Romans:1986er}. 
When we choose a particular charge vector, with $q^2 >0$, we break the duality
group to $SO(4,n)$, and out of the original $5n$ massless scalars
$n$ becomes massive and have values determined by the charges (and the
other scalars) \cite{Andrianopoli:1998qg}. 
The remaining 
$4n$ scalars are massless and represent moduli of the supergravity 
compactification
and, therefore, moduli of the dual conformal
field theory. Note that the conformal field theory involves 
the instanton moduli space, but here the word ``moduli'' refers to the
parameters of the CFT, such as the shape of $T^4$, etc. 

If we start moving in this moduli space we sometimes find that the
gravity solution 
 is best described by doing duality transformations 
\cite{Dijkgraaf:1998gf,Seiberg:1999xz}. One
interesting region in moduli space is when the system is best described
in terms of a 
 system of NS fivebranes and 
fundamental strings. This is 
the S-dual version of the D1-D5 system that we
were considering above.
% This is in the same moduli space as the 
%D1-D5 system, so changing the massless moduli we can go from 
%a region where the system is best described by supergravity 
%on the D1-D5 background to a point in moduli space where it is described
%by the F1-NS5 background. 
In this  NS background the
radius of the $S^3 $ and of $AdS_3$ is $R^2 = Q_5 \alpha'$, and it is 
independent of $Q_1$. Actually, $Q_1$ only enters through the 
six dimensional string coupling, which in this case is a fixed scalar
$g_6^2 = Q_5/Q_1 $. The volume of $M^4$ is a free scalar in this case. 
%S-duality in
%ten dimensions translates into a an S-duality symmetry in six dimensions
%which exchanges the six dimensional string coupling with the volume
%of $M^4$. 
The advantage of this background is that one can 
solve string theory on it to all orders in $\alpha'$,
 since it is a WZW model, actually 
an $SL(2,\IR)\times SU(2)$ WZW model.
String propagation in  $SL(2,\IR)$ WZW models were studied in 
\cite{Balog:1989jb,Petropoulos:1990fc,Mohammedi:1990dp,%
Bars:1991rb,Hwang:1991aq,%
Henningson:1991ua,Henningson:1991jc,Hwang:1992an,Bars:1996mf,%
Bars:1995cn,Satoh:1997xe,Evans:1998wq,Evans:1998qu,% 
Giveon:1998ns,deBoer:1998pp,Kutasov:1999xu,Pesando:1999ex,% 
Andreev:1999nt,Andreev:1999uh,Ito:1998vd,Hosomichi:1999be}.
Thus,
in this case we can also consider states corresponding to massive
string modes, etc. We can also define the spacetime Virasoro generators
in the full string theory,
and check that they act on  string states as they should 
\cite{Giveon:1998ns,deBoer:1998pp,Kutasov:1999xu}\footnote{
Configurations with NS fluxes that lead to $AdS_{2d+1}$ spaces
where studied in \cite{deBoer:1999ie}. It has also been suggested 
\cite{O'Loughlin:1998qx} that  (2,1) strings can describe $AdS_3$ 
spaces.}.
In the string theory description the Virasoro symmetry appears
directly in the formalism as a spacetime symmetry. One can also 
study D-branes in these $AdS_3$ backgrounds \cite{Stanciu:1999nx}.
Conditions for spacetime supersymmetry for string theory on $SL(2,\IR)$
WZW backgrounds were studied in \cite{Giveon:1999jg,Berenstein:1999gj}. 
In the D1-D5 configuration it is much harder to solve string theory,
since RR backgrounds are involved. Classical actions for 
strings on these backgrounds were written in 
\cite{Pesando:1999wm,Rahmfeld:1998zn,Park:1999un}. 
 However,  a formulation of 
string theory on these backgrounds was proposed in 
\cite{Berkovits:1999im} (see also 
\cite{Bershadsy:1999hk,Berenstein:1999gj,Yu:1998qw}).
For some values of the moduli the CFT is singular. 
What this means is that we will have a continuum of states in the 
cylinder picture. In the picture with NS charges this happens, for
example,
when all RR B-fields on $M^4$
 are zero. This continuum of states
comes from fundamental strings stretching close to the boundary of 
$AdS_3$. These states have finite energy, even though
they are long, due to the interaction 
with the constant three form  field strength, $H =dB_{NS}$, on $AdS_3$
\cite{Maldacena:1999uz,Seiberg:1999xz}.


A simple variation of the previous theme is to quotient (orbifold) 
the three-sphere
by a $\IZ_N \subset SU(2)_L$. 
This preserves $\cn=(4,0)$ supersymmetry.
This quotient changes the central charge of the theory by a factor
of $N$ through \brown\ (since the volume of the $S^3$ is smaller by a
factor of $N$). It is also possible to obtain this 
geometry by considering 
 the near horizon behavior of a D1-D5 + KK monopole 
system, or equivalently a D1-D5 system near an $A_N$ singularity. 
It is possible to analyze the field theory by using the methods
in \cite{Douglas:1996sw}, and using the above anomaly argument one
can calculate the right moving central charge. The left moving 
central charge should be calculated by a more detailed argument. 
When 
we have NS 5 branes and fundamental strings on an $A_N$ singularity,
the worldsheet theory is solvable, and one can calculate the spectrum 
of massive string states, etc. \cite{Kutasov:1998zh}. One can also 
consider also both RR and NS fluxes simultaneously \cite{Duff:1998cr}.
 Other papers analyzing aspects of these
quotients or orbifolds are
 \cite{Sugawara:1999qp,Sugawara:1999qp,Behrndt:1998gr,Yamaguchi:1999gb,%
Balasubramanian:1998ee}.


A related configuration arises if we consider M-theory on $M^6$, 
where $M^6 = T^6, T^2 \times K3$ or $CY_3$, and we wrap  M5 branes
on a four-cycle in $M^6$ with non-vanishing 
 triple self-intersection number.
Then, we get a string in five dimensions, and the near horizon geometry
of the supergravity solution is $AdS_3 \times S^2 \times M^6_f $,
where the subscript on $M^6_f$ indicates that the vector moduli of $M^6$
are fixed scalars. In this case we get again an $\cn=(0,4)$
 theory, and the
$SU(2)_R$ symmetry is associated  to rotations of the sphere. 
It is possible to calculate the central charge by counting 
the number of moduli of the brane configuration. Some of the moduli
correspond to geometric deformations and some of them correspond
to $B$-fields on the fivebrane worldvolume 
\cite{Maldacena:1997de,Minasian:1999qn}. A supergravity analysis of this
compactification was done in \cite{deBoer:1998ip,Fujii:1998tc}.

Another 
 interesting case is string theory compactified on $AdS_3\times S^3 \times
S^3 \times S^1 $, which has a large ${\cal N}=4$ 
symmetry \cite{Boonstra:1998yu,Elitzur:1998mm,deBoer:1999rh}.
 This algebra is sometimes called ${\cal A}_\gamma$.
It includes an $SU(2)_k \times SU(2)_{k'} \times U(1) $ current algebra. 
The relative sizes of the 
levels of the two $SU(2)$ factors are related to the relative sizes
of the  radii of the 
spheres. This case seems to be conceptually simpler than the case
with an $M^4$, since all the spacetime dimensions are 
associated to a symmetry of the system\footnote{In the case of $T^4$
one can show that the $U(1)^4$ symmetries of the torus can 
be viewed as the $ k' \to \infty $ limit of the large $ {\cal N} = 4$ 
algebra \cite{Maldacena:1999bp}.}.
In  \cite{Boonstra:1998yu} 
 a geometry like this was obtained from branes, except that
the $S^1$ was 
replaced by $\IR$, and it is not clear which brane configuration 
gives the geometry with the $S^1$. This makes it more difficult
to guess the dual conformal field theory.
 In \cite{Elitzur:1998mm} a CFT dual was proposed for this system
in the case that $k=k'$. One starts with a theory with a free boson
and four free fermions, which has large ${\cal N} = 4$ symmetry. Let
us call this theory $CFT_3$. Then, we can consider the theory
based on the symmetric product $Sym(CFT_3)^k $. 
The space-time 
 theory has two moduli, which are the radius of the circle and the
value of the RR scalar. These translate into the  radius of the 
compact  
$U(1)$-boson in $CFT_3$ and a  blow up mode of the orbifold.
In \cite{deBoer:1999rh} a dual CFT was proposed for the general case 
($k \not = k'$). 


Another interesting example is 
the D1-D5 brane system in Type I string theory
\cite{Johnson:1998ms,Barbon:1998ln,Oz:1999it}.
The $\cN=(0,4)$ theory on the D1 brane worldvolume theory 
encodes in the Yukawa
couplings the ADHM data for the
construction 
of the moduli space of instantons \cite{Douglas:1996gf, Witten:1994sm}.
What distinguishes 
the Type I system from the Type IIB case is the $SO(32)$
gauge group in the open string sector.
When the D5 branes wrap a compact space $M^4$ with $M^4=T^4,K3$, 
the near horizon geometry
of the Type I supergravity solution
is $AdS_3\times S^3\times M^4$ \cite{Oz:1999it}. 
As in the previous examples, one is lead
to conjecture
a duality between Type I string theory on $AdS_3\times S^3\times M^4$ 
and the two-dimensional $(0,4)$ SCFT in the IR limit
of the D1 brane worldvolume theory.
 The supergroup of the Type I compactification
is $SU(1,1|2)\times SL(2,\IR)\times SU(2)$, 
and the Kaluza-Klein spectrum in the supergravity
can be analyzed
as in \cite{deBoer:1998ip}. 
The correspondence to the two-dimensional SCFT has not been much
explored
yet. 


The relation between $AdS_3$ compactifications and 
matrix theory 
\cite{Banks:1997vh} was addressed in \cite{Hosomichi:1999uj}.


%One can replace $AdS_5$ in the $AdS_5 \times S^5$ geometry by 
%some other negatively curved space, and one such possibility is
%$AdS_3\times H_2/\Gamma$ where $H_2$ is the two dimensional hyperbolic
%plane and $\Gamma$ is a discrete subgroup which quotients $H_2$ to 
%a compact Riemman surface \cite{Kiem:1998ga}. 
%This  breaks supersymmetry and contains dangerous tachyons,
%leading to instabilities.  

 
\section{Pure Gravity}

One might suspect that the simplest theory we could have on 
$AdS_3$ is pure Einstein gravity. 
In higher dimensions this is not possible since pure gravity  is not
renormalizable, so the only known sensible quantum gravity theory is
string theory, but in three dimensions gravity can be rewritten 
as a Chern-Simons theory \cite{Witten:1988hc,Witten:1989sx},
 and this theory is renormalizable.
Gravity in three dimensions has no dynamical degrees of freedom. 
We have seen, nevertheless,  that it has black hole solutions
when we consider gravity with a negative cosmological constant 
\cite{Banados:1992wn}
\lorbtz . So, it should at least describe the dynamics of these
black holes, black hole collisions, etc. 
It has been argued that this Chern-Simons theory 
reduces to a Liouville theory at the boundary 
\cite{Coussaert:1995zp,Banados:1998ta,Banados:1996tn,Oh:1998sv},
 with the right central
charge \brown . Naively, using the Cardy formula, 
this Liouville theory does not seem to 
give the same entropy as the black holes, but the Cardy formula
does not hold in this case (Liouville theory does not satisfy the
assumptions that go into the Cardy formula). 
Hopefully, these problems will be solved
once it is understood how to properly quantize Liouville theory. 
Since we have the right central charge it seems that
we should be able to calculate the BTZ black hole entropy 
\cite{Strominger:1998eq}, but Liouville theory is very peculiar
and the entropy seems smaller \cite{Carlip:1998qw}. 
Other papers studying $AdS$ pure gravity or BTZ black holes in
pure gravity include   
\cite{Banados:1998gg,Martinec:1998wm,Myung:1998fw,Behrndt:1998wg,%
Emparan:1998qp,Lee:1998qe,Navarro-Salas:1998ks,Navarro-Salas:1999sr,%
Nakatsu:1999wt,Chekhov:1999uk,Chandia:1998uf,Banados:1999ir,%
Brotz:1999xx,Park:1999nc,Banados:1998gz,Yoshida:1999sh,Cho:1999xj,%
Mano:1999xs,Banados:1998sm,Birmingham:1998pn,Hyun:1999vg}.

The Chern-Simons approach to gravity has also led to a proposal
for a black hole entropy counting in this pure gravity theory. 
In that approach the black hole entropy is supposed to come
from degrees of freedom in the Chern-Simons theory that become
dynamical when  a horizon is present  \cite{Carlip:1995qv}. 

One interesting question in three dimensional gravity is 
whether we should consider the Chern-Simons theory on a fixed 
topology or whether we should sum over topologies. Naively it
is the second possibility, however it could be that the sum 
over topologies is already included in the Chern-Simons
path integral over a fixed topology. 

In any case, three dimensional pure gravity is part of the 
full string theory compactifications, and it would be interesting
to understand it better. 

The situation is similar if one studies pure $AdS_3$ supergravities 
\cite{Achucarro:1989gm,Banados:1998pi,David:1999nr}.




\section{Greybody Factors}
\label{adsthree_greybody}

In this section we consider  an extremal or  near extremal black string
in six dimensions. We take the direction along the 
string to be  compact, with radius $R_5 \gg l_s $. We need to take
it to be compact since classically an infinite black string is unstable
\cite{Gregory:1994bj,Gregory:1993vy}\footnote{
It might seem that we can avoid the instability of  
\cite{Gregory:1994bj,Gregory:1993vy} by
going very near extremality. Note, however, that for an infinite
 string it is entropically favorable to create  a Schwarzschild black hole
threaded by an extremal string. }.  Here we assume that the 
temperature is small enough 
so that the configuration is classically
stable\footnote{ A general supergravity analysis of the  
various regimes  in  the D1-D5 system was given in 
\cite{Martinec:1999sa}.}. We take a configuration with D1 brane charge
$Q_1$ and D5 brane charge $Q_5$. The general solution with these
charges, and arbitrary energy and momentum along the string, has
the following six dimensional Einstein  metric\footnote{
Throughout this section we use the six dimensional Einstein metric,
related to the six dimensional string metric by $g_E = e^{-\phi_6}
g_{str}$, where $\phi_6$ is the six dimensional dilaton.} 
\cite{Cvetic:1996xz,Horowitz:1996ay} :
\eqn{metricsix}{\eqalign{
ds^2_{E} = &
 \( 1 + { \sa \over r^2}\)^{-1/2} \( 1 + { \sg \over r^2}\)^{-1/2}
\left[ - dt^2 +dx_5^2
\right. \cr
& \quad \quad +\left. {
r^2_0  \over r^2} (\cosh \sigma dt + \sinh\sigma dx_5)^2
 +\( 1 + {\sa \over r^2}\) ds^2_{M^4} \] \cr
 +& \( 1 + { \sa \over r^2}\)^{1/2}\( 1 + { \sg \over r^2}\)^{1/2} 
\left[
\(1-{r_0^2 \over r^2}\)^{-1} dr^2 + r^2 d \Omega_3^2 \right]~.
}}
We consider the case that the internal space $M^4 = T^4$. 
In general we will also have some scalars that are non-constant. 
These  become fixed scalars in the near-horizon $AdS_3$ limit.
In this case there are five fixed scalars, which are
three self-dual NS B-fields, a combination of the RR scalar and the
four-form on $T^4$, and finally the volume of $T^4$. If we take the
first four to zero at infinity they stay zero throughout the 
solution. Then, the physical volume of $T^4$ is
\eqn{voltffix}{
\nu (r) \equiv {{\rm Volume }\over (2\pi)^4 {\alpha'}^2} =
 v 
\(1+   { \sg \over r^2 }\)^{-1}\(1 + {\sa\over r^2 } \),
}
where $v= \nu(\infty)$ 
is the value of the dimensionless volume at infinity.
The solution \metricsix\
 is parameterized by the four  independent quantities
$\alpha,\g,\sigma,r_0$. There are two extra parameters 
which enter through the charge quantization conditions, which are
 the radius of the $x_5$ 
 dimension $R_5$ and the volume $ v$
of $T^4$. The three charges are
\eqn{charges}{
\eqalign{
   Q_1 &= { 1 \over 4\pi^2  \alpha' \sqrt{v}}\int \nu  *H'
   = { \sqrt{v} r_0^2  \over 2  \alpha'} \sinh 2 \a , \cr
 Q_5 &= {1\over 4\pi^2  \sqrt{v} \alpha'} \int H'  =  
{ r_0^2\over 2 \sqrt{v} \alpha'} \sinh 2 \g ,
\cr
  N &= {  R^2  r_0^2 \over 2  \alpha'^2} \sinh 2 \sigma ,
}}
where $*$ is the Hodge dual in the six dimensions $x^0,..,x^5$ and
$H'$ is the RR 3-form field.
The last charge $N$ is related to the  momentum around the $S^1$
by $P_5=  N/R_5$. All three charges are normalized to be integers.

The ADM energy of this solution is
\eqn{mss}{M=  {   R_5  r_0^2  \over 2  {\alpha'}^2}
(\cosh 2 \alpha + \cosh 2 \gamma + \cosh 2 \sigma  )~.
}
The Bekenstein-Hawking entropy is
\eqn{entropyfd}{
S = {A_{10}\over 4 G^{(10)}_N} = {A_6\over 4 G^{(6)}_N} =
{ 2 \pi  R_5  r_0^3 \over  {\alpha'}^2 } 
 \cosh \alpha \cosh \gamma \cosh \sigma,
}
where $A$ is the area of the horizon and
we have used the fact that in the six dimensional Einstein metric
$G_N^{(6),E} =  {\alpha'}^2 \pi^2/2$.
The Hawking temperature is
\eqn{thwk}{T= {1 \over 2 \pi  
r_0 \cosh\alpha \cosh \gamma \cosh \sigma } .}

The near extremal black string corresponds to the case that 
 $R_5$ is large and the total mass is just above the rest 
energy of the branes. By ``rest energy'' of the branes we 
mean the mass given by the BPS bound,
\eqn{energyab}{
E = M -  { Q_5 R_5 \sqrt{v} } -  { Q_1 R_5  \over \sqrt{v}}.
}
Note that this includes the mass due to the excitations carrying 
momentum along the circle.
In the limit that $ \alpha' \to 0 $ with $E , R_5 $ and $N$ fixed we
automatically go into the regime described by the conformal field
theory living on the D1-D5 system which is decoupled.
% flat six dimensional space. 
Instead, we are going to take here $\alpha'$ small but nonzero, so that
we keep some coupling of the CFT to the rest of the degrees of freedom.
The geometry is $AdS_3$ (locally) close to the horizon, but  far away
it is just the flat six dimensional space $\IR^{1,4}\times S^1$. 
In this limit we can approximate the six dimensional geometry by 
\eqn{sixdimmet}{
ds^2_{E} = f^{-1/2}\left[ -dt^2 + dx_5^2 +
 {r_0^2 \over r^2}( \cosh \sigma dt 
+ \sinh \sigma dx_5)^2\right]  + f^{1/2}(dr^2 + r^2 d\Omega_3^2),
}
where 
\eqn{defoff}{
f = \left( 1 + { r_1^2 \over r^2}\right) 
 \left( 1 + { r_5^2 \over r^2}\right)~,~~~~~~~
r_5^2 = \alpha'Q_5 \sqrt{v}~,~~~~
r_1^2 = \alpha' Q_1 /\sqrt{v}.
}


Let us consider a minimally coupled scalar field, $\phi$,  i.e. a 
scalar field that is {\it not} a fixed scalar. 
Let us send a quantum of that field to the black string, and
calculate the absorption cross section for low energies. 
This calculation was already discussed in section~\ref{gbFactorsBH}, but
        for the reader's convenience we resummarize the computations
        here.  The low-energy condition is
\eqn{lowener}{
 \omega \ll 1/r_5 , 1/r_1.
}
We will consider here just an s-wave configuration. 
We also set the momentum in the direction
of the string of the incoming
particle to zero, the general case can be found in 
\cite{Maldacena:1997ix,Gubser:1996xe}.
 Separation of variables,
$\phi = e^{-i \omega t} \chi(r)$, leads to the
radial equation
\eqn{RadialLaplace}{
   \left[ {h \over r^3} \partial_r h r^3 \partial_r + 
     \omega^2 f \right] \chi = 0 ~,~~~~~~~~~ h =  1 -
{ r_0^2 \over r^2}.
}
 Close to the horizon, a convenient radial variable is $z = h = 1 -
r_0^2/r^2$.  The matching procedure can be summarized as follows:
  \eqn{MatchSolnAgain}{\seqalign{\span\TT & \span\TR}{
   far region:  & 
    \eqalign{& \left[ {1 \over r^3} \partial_r r^3 \partial_r + 
      \omega^2 \right] \chi = 0,  
       \cr\noalign{\vskip-0.5\jot}
      & \quad \chi = A {J_1(\omega r) \over r^{3/2}}},
     \cr\noalign{\vskip2\jot}
   near region:  & 
    \eqalign{& \left[ z(1-z) \partial_z^2 + 
      \left( 1 - i {\omega \over 2\pi T_H} \right) (1-z) \partial_z + 
      {\omega^2 \over 16 \pi^2 T_L T_R} \right] 
       z^{i\omega \over 4\pi T_H} \chi = 0,
       \cr\noalign{\vskip-0.5\jot}
      & \quad \chi = z^{-{i \omega \over 4\pi T_H}}
       F\left( -i {\omega \over 4\pi T_L}, -i {\omega \over 4\pi T_R};
        1 - i {\omega \over 2\pi T_H}; z \right),  }
  }}
where $T_L,~T_R$ are defined in terms of the Hawking temperature
$T_H$ and the chemical potential, $\mu$, which is conjugate to momentum
on $S^1$ :
\eqn{tleftright}{
{ 1 \over T_{L,R} } \equiv { 1 \pm \mu \over T_H} ~~,~~~~~~~~
T_{L,R} = { r_0 e^{\pm \sigma} \over 2 \pi  r_1 r_5 }.
}
 After matching the near and far regions together and comparing the
infalling flux at infinity and at the horizon, one arrives at
  \eqn{absor}{
   \sigma_{\rm abs} = \pi^3 r_1^2 r_5^2 \omega
    {e^{\omega \over T_H} - 1 \over 
     \left( e^{\omega \over 2 T_L} - 1 \right) 
     \left( e^{\omega \over 2 T_R} - 1 \right)} \ .
  }
Notice that this has the right form to be interpreted as the 
creation of a pair of particles along the string. 

According to the $AdS_3/CFT_2$ correspondence, we can replace the 
near horizon region by the conformal field theory. The field $\phi$
couples to some operator ${\cal O}$ in the conformal field 
theory \cite{Maldacena:1997ih} :
\eqn{couplingop}{
S_{int} = \int dt dx_5  {\cal O}(t,x_5) \phi(t,x_5,\vec 0).
}
Then, the absorption cross section can be calculated by 
\eqn{abscr}{\eqalign{
\sigma & \sim {1 \over N_i } \sum_i  \sum_{f} \left|
\langle f| \int dt dx_5  {\cal O}(t,x_5)e^{i k_0 t + i k_5 x_5 }
|i \rangle \right|^2 \cr  & \sim 
{1 \over N_i }\sum_i \int e^{i k_0 t + i k_5 x_5 }
 \langle i |  {\cal O}(t,x_5)
 {\cal O}^\dagger(0,0)| i  \rangle \cr
  & \sim 
\int e^{i k_0 t + i k_5 x_5 }
 \langle   {\cal O}(t,x_5)
 {\cal O}^\dagger(0,0)  \rangle_\beta,
}}
where we have summed over final states in the CFT and averaged over
initial states. We will calculate the numerical coefficients later. 
The average over initial states is essentially 
an average over a thermal 
ensemble, since the number of states is very large so the
microcanonical ensemble is the same as a thermal ensemble. 
So, the final result is that we have to compute the two point function
of the corresponding operator over a thermal ensemble. 
This  essentially translates into computing the correlation function
on the Euclidean cylinder, and the result is 
proportional to \absor\ \cite{Das:1996wn,Das:1996jy,Maldacena:1997ih}.
This argument reproduces the functional dependence on $\omega$ of 
\absor . For other fields (non-minimally coupled) the 
functional dependence on $\omega$ is determined just 
in terms of the conformal weight of the associated operator. 



Let us emphasize that the matching procedure  \MatchSolnAgain\
is valid only in the low energy regime \lowener .
 In this regime the
typical gravitational size of the configuration, which is of order
$r_5$, is much smaller than the Compton wavelength of the particle. See
figure \ref{nature}. 
In fact, note that in the connecting region 
 $r \sim r_5 $
the 
function $\phi$ does not vary very much.
Let us see this more explicitly. We see from \RadialLaplace\
that we can approximate the equation by something like
$ \omega^2 r_5^2 \phi + \phi'' =0$. From \lowener\ we see that
the variation of 
$\phi$ is very small over this connecting region. 
Furthermore, since absorption will turn out to be small,
we can approximate the value of $\phi $ at the origin by the 
value it has in flat space.
So, we can directly match the values of $\phi$ at the origin 
for a wave propagating in flat space with 
the value of $\phi$ near the boundary of $AdS_3$. 
 
In order to match the numerical coefficient we
need to determine the numerical coefficient in the two-point
function of the operator ${\cal O}$. 
This can be done for minimally coupled scalars using a 
non-renormalization theorem,  as it was done
for the case of absorption of gravitons on a D3 brane. 
The argument is the following.
We first notice that the moduli space of minimally coupled scalars
in supergravity is  $SO(4,5)/SO(4)\times SO(5)$.
This is a homogeneous space
with some metric, so the gravity Lagrangian in spacetime will include
\eqn{lagrangr}{
S = { 1 \over 2 \kappa_6^2 } 
\int d^6x g_{ab}(\phi)\partial \phi^a \partial \phi^b.
}
The fields $\phi^a$ couple to  operators ${\cal O}_a$, and
we are interested in computing
\eqn{metrcft}{
\langle {\cal O}_a(x) {\cal O}_b(0) \rangle = { G_{ab} \over x^4 }.
}
The operators ${\cal O}_a$ are a basis of marginal deformations
of the CFT,
  and  $ G_{ab}$ is the metric
on the moduli space of the CFT. 
Since 
the conformal field theory has $\cn=(4,4)$ supersymmetry, this metric
is highly constrained. In fact, it was shown in  
\cite{Cecotti:1991kz} that it is  the homogeneous metric  on 
$SO(4,5)/SO(4)\times SO(5)$ (up to global identifications). 
Since the CFT moduli space is the same as the supergravity 
moduli space, the two metrics
 could differ only by an overall numerical
factor $G_{ab} = D g_{ab}$, where $D$ is a number. 
In order to compute this number we can go to a point in moduli 
space where the CFT is just the orbifold $Sym(T^4)^k$. 
This point corresponds to having a single D5 brane and $k=Q_5Q_1$ 
D1 branes. We can also choose the string coupling to be arbitrarily 
small. For example, we can choose the scalar $\phi$ to be an off-diagonal 
component of the metric on $T^4$.
The absorption cross section  calculation then reduces to the one  
done in \cite{Das:1996wn}, which we now review. 
We take the metric on the four-torus to be 
$g_{ij} = \delta_{ij} + h_{ij}$, where $h$ is a small perturbation,
and choose $\phi = h_{12}$. 
The bulk action for $\phi$ then reduces to 
\eqn{normac}{ { 1 \over 2 \kappa_6^2} \int d^6x {1\over 2}
 (\partial \phi)^2.
}
The coupling of $h$ to the fields on the D1 branes can be derived
by expanding the Born-Infeld action. The leading term is
  \eqn{donebi}{
   S =  { 1 \over 2 \pi g_s \alpha' } \int dt  dx_5 \,
     \left[ {1 \over 2} (\partial X^i)^2  +   
      h_{12}(\tau,\sigma,\vec{x}=0) \partial X^1 \partial  X^2 
    + {\rm fermions} \right].
}
To extract the cross-section we take $R_5 = \infty$, but the 
volume of the transverse space $V$ finite, 
and we use the usual 2-d S-matrix formulas:
  \eqn{FermiThermal}{\eqalign{
{1  \over \sqrt{2} \kappa_6}   \phi(t,\vec{x}) &=
  \sum_{\vec k} \int {d k_5 \over (2\pi) }  { 1 \over 
    \sqrt{ V  2k^0}} 
\left( a^{12}_k e^{i k \cdot x} + \hbox{h.c.} \right),  \cr
{ 1\over \sqrt{2 \pi g_s \alpha' } } X^i(t,x^5) &= 
\int { dk_5 \over 2 \pi}  {1 \over \sqrt{2   k^0}}
    \left( a_k^i e^{ik \cdot x} + \hbox{h.c.} \right),  \cr
   | \tilde{i} \rangle &= (a_k^{12})^\dagger |0\rangle, \qquad
   | \tilde{f} \rangle = (a_p^1)^\dagger (a_q^2)^\dagger |0\rangle,  \cr
   \langle \tilde{f} | V_{\rm int} | \tilde{i} \rangle &= 
    { \sqrt{2} \kappa_6 p\cdot q \over \sqrt{ V }}, \cr 
%   \rho_L(p^0) &= {1 \over e^{p^0/T_L} - 1} \qquad
%   \rho_R(q^0)  = {1 \over e^{q^0/T_R} - 1}  \cr
%   \Gamma(k^0) &= 2 \sum_{|\tilde{f}\rangle} 
%     (\rho_L(p^0)+1)(\rho_R(q^0)+1) 
%     \left| \langle \tilde{f} | V_{\rm int} | \tilde{i} \rangle \right|^2 
\Gamma(k^0) &= {2 Q_1Q_5 \over 2 k^0 2p^02q^0}\int
 {dp^5 \over 2 \pi }  
{ dq^5 \over 2 \pi } 
 \left| \langle \tilde{f} | V_{\rm int} | \tilde{i} \rangle \right|^2
2\pi\delta(p^5+q^5) 
     2\pi \delta(\omega - p^0 - q^0),  \cr
   \sigma_{\rm abs} &= V  \Gamma(\omega) 
%(1 - e^{-{\omega \over T_H}})
     = \pi^3 \alpha'^2 Q_1Q_5  \omega.
%    {e^{\omega \over T_H} - 1 \over 
%     \left( e^{\omega \over 2 T_L} - 1 \right) 
%     \left( e^{\omega \over 2 T_R} - 1 \right)} \ ,
 }}
% in perfect agreement with \SigmaSwave.\fixit{product of charges}

Since we have  put the four transverse dimensions
into a box of volume $V$, 
 the flux of the $h_{ij}$ gravitons on the brane is
${\cal F} = 1/V$. 
 To find the cross-section
we divide the net decay rate   by the flux.
% The states $|\tilde{i}\rangle$ and $|\tilde{f}\rangle$
%have unit norm.  
The unusual factors of $ \sqrt{2} \kappa_6$
 and $1/\sqrt{ 2 \pi g_s \alpha'}$ 
come from the coefficients of the kinetic
terms for $h_{12}$ and $X^i$ \normac \donebi . 
  The leading factor of $2$ in the equation for $\Gamma(k^0)$
in \FermiThermal\ is there because there are two distinguishable final
states that can come out of a given $h_{12}$ initial state: an $X^1$ boson
moving left and an $X^2$ boson moving right, or $X^1$ moving right and
$X^2$ moving left. The factor of $Q_1Q_5$ comes from the fact that we
have $Q_1Q_5$ D1 branes.  Note that the delta function constraints
plus the on shell conditions imply that $ p^0 =q^0=p^5=-q^5 =\omega/2$
and $p\cdot q = \omega^2/2$. 

The final answer in \FermiThermal\ agrees with the zero temperature
limit of \absor . As we remarked before, the thermal-looking
factors in \absor\ can be derived just by doing a calculation of
the two point function on the cylinder  \cite{Maldacena:1997ih}. 
Finally, we should remark that this calculation implies that
the metric on the moduli space of the CFT has an overall factor of 
$k=Q_1Q_5$
as compared with the metric that appears in the six dimensional gravity
action \lagrangr . This blends in perfectly with the expectations
from $AdS_3$/CFT${}_2$, since in the $AdS_3$ region, by the time we go 
down to three dimensions, we get factors of the volume of the $S^3$ 
and the radius of $AdS_3$ which produce the correct factor of $k$ 
in the gravity answer for the metric on the moduli space. 

Of course, this  absorption cross section calculation is also 
related to the time reversed process of Hawking emission. Indeed,
the Hawking radiation rates calculated in gravity and in the 
conformal field theory coincide. 

Many other  greybody factors  were calculated and compared with
the field theory predictions 
\cite{Dhar:1996vu,Gubser:1996xe,Gubser:1996zp,Callan:1997tv,%
Klebanov:1997gy,Maldacena:1997ih,Krasnitz:1997gn,Gubser:1997qr,%
Mathur:1997et,Klebanov:1997gt,Birmingham:1997rj,Hosomichi:1997if,%
David:1998ev,Kim:1998yw,Taylor-Robinson:1998tk,%
Cvetic:1997xv,Cvetic:1997uw,Cvetic:1997vp,Cvetic:1998ap,%
Lee:1998vg,Muller-Kirsten:1998mt,Ohta:1998xh,Keski-Vakkuri:1999nw,%
Lee:1998xz}.
In some of these references the ``effective string'' model is mentioned.
This effective string model is essentially the conformal field
theory at the orbifold point $Sym(T^4)^k$. Some of the gravity 
calculations did not agree with the effective string calculation.
Typically that was because either the energies considered were not
low enough, or because one needed to take into account the 
effect of the deformation in the CFT away from the symmetric product
point in the moduli space. 




\section{Black Holes in Five Dimensions}
\label{fivedbh}

If we Kaluza-Klein reduce, using \cite{Maharana:1993my,Sen:1994fa},
 the metric \metricsix\
 on the circle along the string,  we get a five 
dimensional charged black hole  solution : 
\eqn{solnfd}{ds_5^2 =  - \lambda^{-2/3} \(1-{r_0^2 \over r^2}\) dt^2 + 
\lambda^{1/3}
\[\(1-{r_0^2 \over r^2}\)^{-1} dr^2 + r^2 d \Omega_3^2 \right]~,}
where
\eqn{deff}{ \lambda 
= \(1+{\sa\over r^2} \)\(1+{\sg\over r^2} \)\(1+{\ss\over r^2} \)~.}
This is just the five-dimensional Schwarzschild metric, with the time
and space components rescaled by different powers of $\lambda$. 
The solution is
manifestly invariant under permutations of the three boost parameters, as
required by U-duality.
The event horizon is clearly at $r=r_0$.
The coordinates we have used present the solution
in a simple and symmetric form, but they do not always cover the entire
spacetime. When all three charges are nonzero, the surface $r=0$ is
a smooth inner horizon. This is analogous to the situation in four
dimensions with four charges 
\cite{Cvetic:1995mx,Cvetic:1995kv}.

The mass, entropy and temperature of this solution are the same
as those calculated above for the black string \mss \entropyfd \thwk .
It is interesting to take the extremal limit $r_0 \to 0$ with
$r_0e^\gamma,~r_0 e^\alpha,~r_0 e^\sigma $ finite and nonzero. 
This is an extremal black hole solution in five dimensions with
a non-singular horizon which has non-zero horizon area. 
The entropy becomes
\eqn{entroextr}{
S = 2 \pi \sqrt{ Q_1Q_5 N},
}
which is independent of all the continuous parameters in the theory, and
depends only on the charges \charges .
We can calculate this entropy as follows \cite{Strominger:1996sh}. 
These black hole states saturate the BPS bound, so they are
BPS states. Thus, we should find an ``index'', which is
a quantity that is invariant under deformations and counts the
number of BPS states. Such an index was computed in 
\cite{Strominger:1996sh} for the case where the internal space was
$M^4=K3$ and in \cite{Maldacena:1999bp} for $M^4=T^4$. 
These indices are also called  helicity supertrace formulas
 \cite{Kiritsis:1997gu}. Once we know that they do not
receive contributions from non-BPS quantities, we can change the
parameters of the theory and go to a point where we can do the
calculation, for example, we can take $R_5$ to be large and then go to 
the point where we have the $Sym(M^4)^k$ description. 

It is interesting that we can also consider near extremal black 
holes, in the approximation that the contribution to the mass
of two of the charges is much bigger than the third and much
bigger than the mass above extremality. This region in parameter
space is sometimes called the ``dilute gas'' regime. 
In the five dimensional context it is natural to take $R_5 \sim l_s$,
and at first sight we would not  expect the CFT description to
be valid. Nevertheless, it is ``experimentally'' observed
that the absorption cross section is still  \absor ,
since the calculation is exactly the same as the one we did above.
This suggests that the CFT description is also valid in this case.
A qualitative explanation of this fact was given in 
\cite{Maldacena:1996ds}, where it was observed that the the strings
could be multiply wound leading to a very low energy gap, much lower
than $1/R_5$, and of the right order of magnitude as expected for
a 5d black hole. 

Almost all that we said in this subsection can be extended to four 
dimensional black holes.









%\end{document}
