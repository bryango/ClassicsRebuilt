%Date: Thu, 03 Oct 1996 15:22:14 EDT
%From: Edward Witten <witten@sns.ias.edu>

\input harvmac

\def\R{{\bf R}}
(1) Let $M$ be a fixed Riemannian manifold, and 
let $X:\R\to M$ be a map from the $t$-line $\R$ (endoowed with
the metric $(dt)^2$) to $M$.

Let

\eqn\neff{L={1\over 2}\int dt g_{IJ}{dX^I\over dt}{dX^j\over dt}.}

(a) Describe the space $W$ of critical points of $L$.
(You should find that it is closely related to the space of
geodesics on $M$.)

(b) Show that given any choice of a point $t_0\in \R$,
$W$ has a natural identification with $T^*M$.  (Use the metric on $M$
to identify $TM$ and $T^*M$.)

(c) Compute the symplectic structure $\omega$ on $W$.  Show that
for any choice of $t_0$,  $W$ becomes identified (under the
identification in (b)) with $T^*M$ with its natural symplectic
structure.

(c) The vector field $d/dt$ on $\R$ induces a vector field
$v$ on $W$.  The ``Hamiltonian'' is a function $H$ on $W$ such
that 

\eqn\jury{dH=i_v\omega.}


This is usually expressed by saying that ``$H$ generates time
translations via Poisson brackets.''  Compute $H$.


\def\A{{\cal A}}
(2)  Consider $\R^{1,1}$ with even and odd coordinates $t$ and $\theta$
and an odd distribution $\A$
generated by the vector field
\eqn\burry{D=
{\partial\over\partial\theta} -\theta{\partial\over\partial t}.}
Thus, $D^2=-\partial/\partial t$.

As in the last exercise, let $M$ be a fixed Riemannian manifold.
Let $X$ be a map from $\R^{1,1}$ to $M$.
Consider the Lagrangian

\eqn\purly{L={1\over 2}\int_{\R^{1,1}}g_{IJ}{\partial X^I\over\partial t}
DX^j.}

(a) Make sense of  the definition of $L$ by interpreting the
integrand as a section of the Berezinian of the tangent bundle
of $\R^{1,1}$.

(b) Setting $X^I=x^I+\theta\psi^I$, write $L$ explicitly
as an ordinary integral, over the $t$-line $\R^{1,0}$, of
a function of $x^I$ and $\psi^I$.

(c) Describe the space $Y$ of critical points of $L$.
You should get a description of the following kind.
Describe the reduced space $Y_{red}$ as an ordinary manifold,
and the normal bundle to $Y_{red}$ in $Y$ as $\Pi V$, where
$V$ is a vector bundle over $Y_{red}$.  What are the ``initial data''
at a given point $t=t_0$ in $\R^{1,1}$ needed to determine
a critical point?  (This is the analog of problem 1(b) above.)

(d) Compute the symplectic structure on $Y$.  

(e) The vector fields
\eqn\uffpo{{\partial\over\partial\theta}+\theta{\partial\over\partial
t}~~{\rm and}~~{\partial\over \partial t} }
induce vector fields on $Y$.  Find the functions $Q$ and $H$ 
that generate these vector fields by  Poisson brackets.
Explain why $\{Q,Q\}=2H$.

\bye
