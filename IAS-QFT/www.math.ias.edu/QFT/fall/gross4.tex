%Date: Tue, 24 Feb 1998 18:33:26 -0500 (EST)
%From: Pavel Etingof <etingof@abel.math.harvard.edu>

\input amstex
\documentstyle{amsppt}
\magnification 1200
\NoRunningHeads
\NoBlackBoxes
\document

\def\bb{\overline{\beta}}
\def\bga{\overline{\gamma}}
\def\bg{\overline{g}}
\def\eps{\epsilon}
\def\tW{\tilde W}
\def\Aut{\text{Aut}}
\def\tr{{\text{tr}}}
\def\ell{{\text{ell}}}
\def\Ad{\text{Ad}}
\def\u{\bold u}
\def\m{\frak m}
\def\O{\Cal O}
\def\tA{\tilde A}
\def\qdet{\text{qdet}}
\def\k{\kappa}
\def\RR{\Bbb R}
\def\be{\bold e}
\def\bR{\overline{R}}
\def\tR{\tilde{\Cal R}}
\def\hY{\hat Y}
\def\tDY{\widetilde{DY}(\g)}
\def\R{\Bbb R}
\def\h1{\hat{\bold 1}}
\def\hV{\hat V}
\def\deg{\text{deg}}
\def\hz{\hat \z}
\def\hV{\hat V}
\def\Uz{U_h(\g_\z)}
\def\Uzi{U_h(\g_{\z,\infty})}
\def\Uhz{U_h(\g_{\hz_i})}
\def\Uhzi{U_h(\g_{\hz_i,\infty})}
\def\tUz{U_h(\tg_\z)}
\def\tUzi{U_h(\tg_{\z,\infty})}
\def\tUhz{U_h(\tg_{\hz_i})}
\def\tUhzi{U_h(\tg_{\hz_i,\infty})}
\def\hUz{U_h(\hg_\z)}
\def\hUzi{U_h(\hg_{\z,\infty})}
\def\Uoz{U_h(\g^0_\z)}
\def\Uozi{U_h(\g^0_{\z,\infty})}
\def\Uohz{U_h(\g^0_{\hz_i})}
\def\Uohzi{U_h(\g^0_{\hz_i,\infty})}
\def\tUoz{U_h(\tg^0_\z)}
\def\tUozi{U_h(\tg^0_{\z,\infty})}
\def\tUohz{U_h(\tg^0_{\hz_i})}
\def\tUohzi{U_h(\tg^0_{\hz_i,\infty})}
\def\hUoz{U_h(\hg^0_\z)}
\def\hUozi{U_h(\hg^0_{\z,\infty})}
\def\hg{\hat\g}
\def\tg{\tilde\g}
\def\Ind{\text{Ind}}
\def\pF{F^{\prime}}
\def\hR{\hat R}
\def\tF{\tilde F}
\def\tg{\tilde \g}
\def\tG{\tilde G}
\def\hF{\hat F}
\def\bG{\overline{G}}
\def\Spec{\text{Spec}}
\def\tlo{\hat\otimes}
\def\hgr{\hat Gr}
\def\tio{\tilde\otimes}
\def\ho{\hat\otimes}
\def\ad{\text{ad}}
\def\Hom{\text{Hom}}
\def\hh{\hat\h}
\def\a{\frak a}
\def\t{\hat t}
\def\Ua{U_q(\tilde\g)}
\def\U2{{\Ua}_2}
\def\g{\frak g}
\def\n{\frak n}
\def\hh{\frak h}
\def\sltwo{\frak s\frak l _2 }
\def\Z{\Bbb Z}
\def\C{\Bbb C}
\def\d{\partial}
\def\i{\text{i}}
\def\ghat{\hat\frak g}
\def\gtwisted{\hat{\frak g}_{\gamma}}
\def\gtilde{\tilde{\frak g}_{\gamma}}
\def\Tr{\text{\rm Tr}}
\def\l{\lambda}
\def\I{I_{\l,\nu,-g}(V)}
\def\z{\bold z}
\def\Id{\text{Id}}
\def\<{\langle}
\def\>{\rangle}
\def\o{\otimes}
\def\e{\varepsilon}
\def\RE{\text{Re}}
\def\Ug{U_q({\frak g})}
\def\Id{\text{Id}}
\def\End{\text{End}}
\def\gg{\tilde\g}
\def\b{\frak b}
\def\S{\Cal S}
\def\L{\Lambda}

\topmatter
\title Lecture 4: Renormalization groups (continued)
\endtitle
\author {\rm {\bf David Gross} }\endauthor
\endtopmatter

\centerline{Notes by P.Etingof and D.Kazhdan}
\vskip .1in

{\bf 2.1. Dynamical mass generation.}

In the previous three lectures we developed a general formalism 
of the method of renormalization group, derived the renormalization 
group equations, and considered its applications. In particular, we 
considered the notion of asymptotic freedom. In this lecture, we will 
actually consider an example of a good quantum theory which exists 
nonperturbatively and is asymptotically free. But before this let us finish 
the discussion of general properties of the renormalization group equation. 

Consider a theory given by a Lagrangian which classically 
 has a symmetry prohibiting masses (e.g. QED with a chiral fermion 
with no antichiral partner; such a theory has a chiral symmetry
which prohibits the mass of a fermion). Let us ask 
the question: can a physical mass 
be dynamically generated in the corresponding quantum theory? 
Of course, this can only happen if the prohibiting symmetry is broken 
quantum mechanically. 

Let $m_p$ be the physical mass of the particle in this
 theory. That is, $m_p$ is the pole of the 2-point function
in momentum space. 

Since the Lagrangian can have no mass term, there is no bare masses. 
Therefore, it is clear
from counting of dimensions that the physical mass 
of the particle is given by 
$$
m_p=\mu f(g),\tag 4.1
$$
where $f$ is a scalar function and $g$ is a dimensionless coupling. 

As we made clear in the previous lectures, the main property of the 
renormalization  group vector field on the space $P\times M$ 
(where $M$ is the set of momentum scales, and $P$ the space of parameters) 
is that any physical parameters -- actual (not bare!) masses of particles,
scattering amplitudes -- are CONSTANT along the trajectories of this vector 
field. 
Thus, the renormalization group equation says 
$$
(\mu\d_\mu+\beta\d_g)\mu f(g)=0,\tag 4.2
$$
 which yields
$$
m_p=C\mu e^{-\int_{g_0}^g\frac{dx}{\beta(x)}}.\tag 4.3
$$
Consider the behavior of this function for weak coupling: $g\to 0$.
As usual, this behavior is determined from the behaviour of the beta-function, 
which can be obtained from a 1-loop calculation.  
For example, assume that in our theory $\beta(g)\sim \pm g^3$ 
(e.g. QED,QCD). Then 
$$
m_p\sim \tilde C\mu e^{\mp 1/g^2}.\tag 4.4
$$

Thus, for an asymptotically free theory we have $m_p\sim \mu e^{-1/g^2}$.
This is a reasonable behavior, since $m_p\to 0$ as $g\to 0$, so the theory 
is massless in the UV limit, as it should be. 
Also, if $\Lambda$ is a cutoff and $g_\Lambda$ the coupling at $\Lambda$, 
we have $m_p\sim \Lambda e^{-1/g_\Lambda^2}$. In particular, 
$g_\Lambda^2\sim 1/\ln\Lambda$ as $\Lambda\to\infty$, which can 
also be seen from the renormalization group equation. 

Note that (4.4)
goes to $0$ as $g\to 0$ faster than any polynomial; so it is a nonperturbative 
correction; in particular, all perturbative 
(Feynman diagram) approximations to $m_p(g)$ vanish. 
This is not surprising, since the Feynman diagrams are written 
from the Lagrangian in which there is no room for mass corrections. 

Of course, it is possible that the proportionality coefficient 
in (4.4) vanishes, in which case $m_p=0$ even in 
the nonperturbative quantum theory. This can happen for example 
if the prohibiting symmetry is unbroken (e.g. supersymmetry). 

However, if the theory is not asymptotically free, but IR free, we get 
$m_p\sim \mu e^{1/g^2}$, which is absurd. Thus, 
in this case, mass generation 
is not possible. 
 
Finally, let us explain why theories which are not asymptotically 
free are usually not believed to exist as well defined nonperturbative
quantum theories. For this purpose consider 
the dependence of the bare coupling $g_0$ of the cutoff $\e$ (in the 
dimensional regularization -- minimal subtraction scheme); this 
dependence can be derived from the renormalization group equation.
Namely, we have (see Lecture 2)
$$
g_0(\e,g)=ge^{\int_0^g\frac{\beta(x)dx}{x(\e x/2-\beta(x))}},\tag 4.5
$$
where $\e=4-d$. 

Let $\beta(x)=\pm x^2+...$. Then formula (4.5) can be written in the form 
$$
g_0=g+\sum_{m=1}^\infty g^{m+1}P_m(1/\e),\tag 4.6
$$
where $P_m$ is a polynomial of degree $m$ with zero free term. 
This expression yields counterterms in all orders of perturbation theory. 

Let us see, however, whether (4.5) and (4.6) make sense nonperturbatively. 
If the sign in the beta-function is minus (asymptotic freedom), then the 
integral in (4.5) is well 
defined, at least for small enough $g$, since the denominator is positive. 
However, if the sign is plus (no asymptotic freedom), then 
the denominator vanishes for $x\sim \e$, so 
if $\e$ is small then the integral gets a singularity even for small $x$. 
This is usually regarded as a sign that the theory does not exist, i.e. that 
ist perturbation expansion does not correspond to any nonperturbative theory. 
In particular, the lack of asymptotic freedom for $\phi^4$ theory 
is one of the reason people believe this theory does not exist. 

{\bf Remark.} In spite of this, physicists often use 
Lgarangians which are not asymptotically free and even nonrenormalizable
for describing quantum field theories. In this case, it is usually meant 
that the the theory is considered with a permanent UV cutoff, which is not 
sent to infinity; thus the theory only makes sense as a 
``low energy effective theory''. See Witten's lecture II-1 for more details. 

{\bf 2.2. The Gross-Neveu model.} 

Now we are ready to consider a concrete field theory, on whose example 
we will be able to illustrate the method of renormalization group. 
This is a 2-dimensional, purely fermionic theory, which is called the 
$(\bar \psi\psi)^2$-model, or the Gross-Neveu model. 

This model exhibits the following nice properties:

1. It is described by a local Lagrangian and is both renormalizable and 
asymptotically free. 

2. It exists nonperturbatively; this is a nontrivial fact which can be proved 
by applying a nonperturbative version of the renormalization group method. 

3. It has dynamical symmetry breaking and dynamical mass generation 
(these things go together as the symmetry which is broken is the one 
prohibiting mass). 

4. This theory is in 2 dimensions, but we can formally continue its
large N expansion 
to $2+\e$ (using dimensional regularization techniques), and the obtained 
amplitudes will be regular at $\e=1$. Thus we obtain a 3-dimensional 
``theory'' 
which is non-renormalizable but exists.  

5. This theory is soluble (integrable). This means, its S-matrix can be 
computed exactly for a any fixed N and the Green's functions can be computed 
exactly in the large N limit (where N is the number 
of particles). Computation of the S-matrix leads to the theory of quantum 
groups (Yangians) -- see Witten's lecture II-3, where a similar model is 
studied. 

6. It is closely related to 2-dimensional nonlinear sigma-model 
to a homogeneous space with positive curvature (see Witten's lecture II-6). 
 
Let us now define the Gross-Neveu model. It is a theory of N massless 
Dirac fermions in two dimensions with a quartic interaction.
Namely, let $V$ be a complex N-dimensional vector space with a positive 
definite Hermitian form $(,):\bar V\times V\to \C$, 
and let $\psi$ be a Dirac spinor on $\R^{1,1}$ 
with coefficients in $V$. Consider the Lagrangian in Minkowski space
$$
L(\psi)=\int d^2x(i(\bar\psi,D\psi)+\frac{g^2}{2}(\bar\psi,\psi)^2).\tag 4.7
$$

Let us determine the symmetries of this Lagrangian.
Let $S_+,S_-$ be the spin bundles over $\R^{1,1}$ (these are real, 
1-dimensional representations of $Spin(1,1)$ with spins $1/2$ and $-1/2$.
Let $S=S_+\oplus S_-$. 
By definition, $\psi$ is a section of $\Gamma(S)\o V$, so 
$\psi=(\psi_+,\psi_-)$. In terms of the chiral components 
$(\psi_+,\psi_-)$, Lagrangian (4.7) can be written as 
$$
L(\psi)=\int d^2x(i(\bar\psi_+,D\psi_+)+i(\bar\psi_-,D\psi_-)+
g^2((\bar\psi_+,\psi_-)+(\bar\psi_-,\psi_+)).\tag 4.8
$$
We see that (4.8) has an obvious global $U(N)$ symmetry 
(unitary transformations of $V$), as well as a chiral symmetry 
$\psi_\pm\to \pm \psi_\pm$. This chiral symmetry is the one that prevents a 
mass term $m(\bar\psi,\psi)$. We will see that in quantum theory, 
the chiral symmetry will be broken and the mass will be dynamically generated. 

Recall that in 2 dimensions spinor fields have scaling dimension $1/2$. 
This implies that the theory defined by (4.7) is purely renormalizable
(i.e. classically conformally invariant), with a single renormalizable 
coupling $g$. 

{\bf Remark.} One can show 
that the $U(N)$ symmetry survives in quantum theory, and 
so no new couplings will appear in the process of renormalization. 

We will see below that the Gross-Neveu model is asymptotically free 
for $g^2>0$ and IR free for $g^2<0$. So we could hope to have a nice theory 
only for $g^2>0$. Note that unlike $\phi^4$-theory, the positivity of the 
coefficient at the quartic term in Minkowski signature is not a problem, 
since our theory is fermionic and positivity of $g^2$ does not create 
a divergence in the functional integral.  

To study the Gross-Neveu model, it will be useful to introduce 
an auxiliary real scalar field $\sigma$ (with values in $\R$), and consider 
the Lagrangian
$$
L_\sigma(\psi,\sigma)=\int d^2x(i(\bar\psi,D\psi)+g\sigma(\bar\psi,\psi)-\frac{1}{2}
\sigma^2).\tag 4.9
$$
It is clear that this Lagrangian is equivalent to (4.7): indeed, 
(4.7) is obtained from (4.9) by integrating out $\sigma$ 
(i.e. setting $\sigma$ to equal its stationary point, $\sigma=g(\bar\psi,\psi)$. 

{\bf 2.3. The large N limit.}

The Gross-Neveu model becomes especially simple in the limit $N\to\infty$. 
More precisely, in this limit the theory is exactly solvable. 
This means, we can compute the asymptotic expansion of 
the Green's functions in this theory in terms of the variable $1/N$. This is 
a kind of perturbation expansion which is useful also 
for many other theories. The advantage of this expansion versus
perturbation expansions with respect to other parameters (like $g$) is that 
$1/N$ is not changed by the renormalization group flow. 
 
We will solve the Gross-Neveu model in the large $N$ limit 
in two ways: 1) by explicitly summing the perturbation series; 
and 2) by using presentation (4.9).

{\bf Remark.} It is not claimed that there is a well defined quantum field 
theory at $N=\infty$; we just have a family of theories depending on $N$ whose 
Green's functions have a nice asymptotic expansion in $N$ as $N\to\infty$.

We start with summing the perturbation series. We note that the perturbation 
series for bosonic theories (with a cutoff) usually diverges, 
and one has to apply Borel summation to get a nonperturbative result. 
However, in purely fermionic theories (like the Gross-Neveu model) 
the perturbation theory may converge, defining an analytic 
(not only $C^\infty$) function of the perturbation parameter. 
To understand why it is so, it is best to look at two integrals 
$$
I_1(h)=\int_{\R^n} e^{-\frac{v^2+(v^2)^2}{h}}dv, 
I_2(h)=\int_{\R^{0,2n}}e^{-\frac{(\psi,A\psi)+(\psi,A\psi)^2}{h}}d\psi
$$
where $\R^{0,2n}$ 
is an odd vector space, and $A$ a skewsymmetric nondegenerate 
matrix. Clearly, the first integral gives a divergent perturbation series, 
while the second gives a polynomial. For infinite dimensional integrals, 
convergence is not obvious even for purely fermionic theories, but one can 
show that for the Gross-Neveu model the perturbation series indeed converges. 
(see the paper of Gawedzki and Kupiainen, CMP, v.102(1),p.1-30). 

Consider Lagrangian (4.7) for large $N$. It is clear that the first (kinetic)
term is of order $N$ and the second (quartic) term is of order $N^2$. 
Therefore, if set $g_N=\bg N^{-1/2}$, where $\bg$ is fixed, and consider 
the sequence of theories with N-dimensional $V$ and coupling constant $g_N$, 
then both terms in the Lagrangian scale in the same way for large $N$ 
(they are both of order $N$). Thus, the large N limit for our theory is 
similar to its quasiclassical limit (i.e. $1/N$ plays the role 
of the Planck constant). 

Let us consider the Feynman rules for our theory. For this we will choose 
an orthonormal basis $e_i$ in $V$ and write $\psi=\sum \psi_je_j$, where 
$\psi_j$ are complex valued spinors. The Feynman rules are as follows:
the fermion propagator (in momentum space of Minkowski signature) 
is $\frac{ip}{p^2}$
(as usual), and the 4-vertex corresponding to the term 
$(\bar\psi_j,\psi_{j'})(\bar\psi_k,\psi_{k'})$ carries the factor 
$ig^2\delta_{jj'}\delta_{kk'}$. This makes in natural to draw the 4-vertex as 
two separate (oriented) arcs:
$$
\gather
i\ \ \ j\\
\biggr)\biggl(\\
\bar i\ \ \ \bar j\endgather
$$
(bars over $i,j$ remember the orientation). 
In particular, every Feynman diagram is a union of lines and 
circles, like
$$
\biggr)OOO\biggl(  \tag 4.10
$$
For large $N$, each circle gives a factor of $N$ (because of summation 
over the index that lives on this circle) and every ``tangency'' 
of two circles (a 4-vertex) gives a factor of $g^2$. 
Thus the order of the contribution from such diagram is 
$N^K$, where $N$ is the number of circles minus the number of  
tangencies. 

Now consider the perturbation expansion of the 
1-particle irreducible 
2-point function $\Gamma_2$ for $\psi$. This means we should consider 
diagrams with one line and any number of circles. 
For convenience let us close up the line (this will only give us 
an extra factor of $N$). Then we get diagrams with circles only. 
For any such diagram $D$, consider the graph $\Gamma(D)$ 
in which vertices are the circles of $D$ and edges are tangencies of circles. 
We see that $\Gamma_2=\sum_DA_D$, 
where $A_D$ is the amplitude of $D$, and for large $N$ one has
$A_D\sim N^{-n}$, where $n$ is the number of cycles in $\Gamma(D)$.
(note that $\Gamma(D)$ is always connected). 

Thus, the leading term of the large $N$ expansion of our theory is given 
by ``tree'' diagrams, i.e. diagrams for which $\Gamma(D)$ (not $D$ itself!)
is a tree. 

Now, it is easy to see that the amplitude of the diagram 
$$
\biggr)O
$$
vanishes (it contains the factor 
$\int_{ |p|\le \Lambda} \frac{ip}{p^2}d^2p$
which is zero by symmetry arguments). Therefore, there is no nontrivial trees
which give a contibution of order $N^0$ to the 2-point function
(indeed, such a tree would have to have only one 1-valent vertex, which 
is impossible for a nontrivial tree); 
thus the propagator does not get a correction in perturbation 
theory in the  top order in $N$. 

Let us now consider the 4-point function
$\<\psi_j\bar\psi_k\bar\psi_l\psi_m\>$. Arguing similarly to the 2-point 
case, we see that all leading contributions (in $N$) come from diagrams 
$D$ for which $\Gamma(D)$ is a tree with only two 1-valent vertices. 
Any such tree must be a chain. Thus all diagrams of importance
in the leading order are the following diagrams, which we call 
$D_n$ and $D_n'$ (where $n$ is the number of bubbles):
$$
\gather
j\ \ \ \ \ \ \ \bar k\\
\biggr)OOO\biggl(\\
\bar l\ \ \ \ \ \  \ m\\
\text{and}\\
j\ \ \ \ \ \ \ \bar l\\
\biggr)OOO\biggl(\\
\bar k\ \ \ \ \ \  \ m
\endgather
$$

Thus, computing the leading term of the 4-point function reduces 
to summing a geometric series. Namely, we have 
$$
\Gamma_4^{jklm}(p_1,p_2,p_3,p_4)|_{N=\infty}=\delta_{jk}\delta_{lm}F(s)+
\delta_{jm}\delta_{kl}F(t),\tag 4.11
$$
where $$
s=(p_1+p_2)^2, t=(p_3+p_4)^2, \tag 4.12
$$ 
$i,k,l,m$ are the labels of the components of $\psi_i$ for which 
the 4-point function is computed, and $p_1,p_2,p_3,p_4$  
are the momenta corresponding to labels $i,j,k,l$. 
In this formula, the first summand comes from diagrams $D_n$ 
  and the second from $D_n'$. Thus, 
$$
F(s)=\sum_{n\ge 0}A_{D_n}.\tag 4.13
$$

The amplitude of $D_n$ is easily computed. Since the Feynman integrals
are IR convergent, it suffices to introduce an ultraviolet cutoff $\Lambda$. 
Then 
$$
A_{D_n}=(ig^2)^{n+1}A^n,\tag 4.14
$$
where 
$$
A=\frac{iN}{\pi}\ln(-\Lambda^2/s)\tag 4.15
$$
(remember that a fermionic loop contributes a minus sign). 
Thus, after introducing a counterterm 
$Cg^4\ln(\Lambda/\mu)(\bar\psi,\psi)^2$ in the Lagrangian, 
using some scale $\mu$, we will get by summing the geometric series:
$$
F(s)=\frac{ig^2}{1+\frac{g^2N}{\pi}\ln(-s/\mu^2)}.\tag 4.16
$$

Formula (4.16) implies that the coupling $g$ has the meaning 
of the strength of the interaction at the scale $\mu$: indeed, if 
we take $p_i$ such that $(p_i+p_j)^2=-\mu^2$ then $\Gamma_4^{jklm}=
ig^2(\delta_{jk}\delta_{lm}+\delta_{jm}\delta_{kl})$. 

Another consequence of formula (4.16) is that our theory is asymptotically 
free for large $N$. Indeed, as $s\to -\infty$, function (4.16) behaves as 
$1/(\ln(-s))$, i.e. goes to 0. Moreover, we can compute the $\beta$-function 
exactly: from the renormalization group equation and (4.16) we get:
$$
\beta(g)=-\frac{g^3N}{\pi}+O(1),\ N\to\infty.\tag 4.17 
$$
This is the exact $\beta$-function to leading order in $1/N$. 

Thus, we should expect that our theory is well defined in the UV limit. 

Let us now consider what happens to this theory 
in the infrared limit. 
Then we notice that 
formula (4.16) has an unfortunate property that the function 
$F(s)$ has a pole at $s=-\mu^2e^{-\pi/g^2N}$. Thus, the $\Gamma$-function 
has a pole outside of $0$ for ``Euclidean'' momenta. This cannot happen 
in a well defined quantum field theory: it would mean that the theory 
contains a ``tachyon'' -- particle of imaginary mass.  

So what is wrong with our theory? There could be two things: either 
the theory is ``sick'' (does not exist) or we have been doing 
perturbation theory around a ``wrong'' (unstable) vacuum. 

In fact, the situation (which can be expected if we believe the theory
exists)
is as follows. This theory classically has no mass gap, 
so the quasiclassical approximation is not enough to determine what it does 
quantum mechanically. What in fact happens is that the theory has two 
different realizations, which are permuted by the chiral symmetry
(see Witten's lecture II-1). In particular, the chiral symmetry is broken 
quantum mechanically. These two vacuum states are massive (i.e. mass 
is dynamically generated), and the expectation 
value of $(\bar\psi,\psi)$ at them is nonzero: we have two opposite values for 
two different vacua. However, these expectation values are exponentially 
small (in terms of the coupling), so they cannot be distinguished in 
perturbation theory. 

Thus, 
in order to find the quantum vacuum states, we have to use nonperturbative 
methods. We will use formula (4.9), which contains an auxiliary bosonic 
field $\sigma$.  

Consider the Euclidean path integral with Lagrangian (4.9)
(remember that we have to change the signs when going from 
Minkowski to Euclidean space, and remove the $i$ in front of the 
Dirac operator):
$$
\int D\psi D\sigma e^{-\int d^2x((\bar\psi,D\psi)-g\sigma(\bar\psi,\psi)+
\frac{1}{2}\sigma^2)}.\tag 4.18
$$ 
In this integral, $\psi$ has $4N$ real components, and integration is carried 
out over all of them. 
Remembering the asymptotics of the terms of the Lagrangian in terms of $N$, 
we see that it is convenient to replace $\sigma$ with $\sigma N^{1/2}$
and rewrite this path integral as
$$
\int D\psi D\sigma e^{-N\int d^2x(\frac{1}{N}(\bar\psi,D\psi)-\frac{g}{\sqrt{N}}
\sigma(\bar\psi,\psi)+
\frac{1}{2}\sigma^2)}\tag 4.19
$$ 

Now, since the path integral (4.19) is Gaussian in $\psi$, we can integrate
$\psi$ out. We get (up to an overall factor)
$$
\gather
\int D\sigma \text{det}^N(D-g\sqrt{N}\sigma)e^{-N\int d^2x \sigma^2/2}=\\
\int D\sigma e^{N\text{Tr}
\ln (iD-g\sqrt{N}\sigma)-N\int d^2x \sigma^2/2}.\tag 4.20
\endgather
$$ 
The large N limit of 
last integral can be treated by the stationary phase method. 
In particular, the expectation value of $\sigma$ 
for large N is such that it makes 
the functional 
$$
I(\sigma)=\text{Tr}\ln(D-g\sqrt{N}\sigma)-\int d^2x \sigma^2/2\tag 4.21
$$
stationary. 

We look for constant functions $\sigma$. 
Then the stationarity equation for (4.21) is
$$
\sigma=-\text{Tr}\frac{g\sqrt{N}}{D-g\sqrt{N}\sigma}=
-\int \frac{d^2p}{(2\pi)^2}\frac{g\sqrt{N}}{iP-g\sqrt{N}\sigma},\tag 4.22
$$
where $P$ is the operator of Clifford multiplication by $p$. 
Replacing $\sigma$ with $-i\rho$, we get  
$$
\rho=
Tr\int \frac{d^2p}{(2\pi)^2}\frac{g\sqrt{N}}{P+g\sqrt{N}\rho},\tag 4.23
$$

It can be shown that this equation has two opposite nonzero solutions
as well as the zero solution. 
To find the nonzero solutions, the right hand side, which is divergent, 
needs to be renormalized in the UV, using some scale $\mu$. The renormalized 
equation has the form 
$$
\rho=
\lim_{\Lambda\to\infty}
(\int_{|p|=\Lambda} 
\frac{d^2p}{(2\pi)^2}\frac{g\sqrt{N}}{P+g\sqrt{N}\rho}-\frac{g\sqrt{N}}
{2\pi}\ln(\Lambda/\mu)).\tag 4.24
$$
Simplifying (4.24), we get
$$
\gather
\rho=2\lim_{\Lambda\to\infty}
\frac{g^2N\rho}{4\pi}\ln[\frac{\mu^2}{g^2N\rho^2}
(1+g^2N\rho^2/\Lambda^2)]=\\
2\frac{g^2N\rho}{4\pi}\ln[\frac{\mu^2}{g^2N\rho^2},\tag 4.25
\endgather
$$
(the factor $2$ in the front is due to the fact that $P$ is a 2 by 2 matrix)
which implies that 
$$
\rho=\frac{\mu}{g\sqrt{N}}e^{-\pi/g^2N}.\tag 4.26
$$
Thus, the expected physical mass of the fermions is
$$
m_p=g\sqrt{N}\rho=\mu e^{-\pi/g^2N}.\tag 4.27
$$

Another
way to  find the explicit expression 
for the nonzero solutions of equation (4.23) is to 
compute the effective potential for $\rho$. The computation 
is analogous to the above computation of the 4-point function, and 
yields (up to scaling 
$$
V_{eff}(m)=\frac{m^2}{2}+\frac{g^2Nm^2}{4\pi}
(\ln\frac{m^2}{\mu^2}-1).\tag 4.28
$$
The critical points of this function are $0$ and $\pm m_p$, where 
$m_p$ is given by (4.27). However, the point $m=0$ is unphysical 
(=does not define a quantum vacuum): at this point $V_{eff}''(m)=-\infty$, 
so the point is a local maximum of the effective potential.  

Thus, we have shown that our theory exhibits dynamical mass generation. 
This is a special case of so called ``dimensional transmutation'' --
a phenomenon in which a bare dimensionless coupling ($g$) transforms 
dynamically into a dimensionful measurable quantity ($m$). 


\end


Now remember that as we tend $N\to\infty$, we decided to keep 
$g\sqrt{N}$ fixed.  
 
\end 

Note that our theory has only one parameter. This means, after fixing the 
scale of fields, 

 

Suppose that $m$ is the bare mass for  
this particle in this theory at a scale $\mu$. Then by dimensional analysis 
we have 
$$
m_p=mf(g,Z), z=\ln(m/mu).\tag 4.1
$$
where $f$ is a dimensionless function of two dimensionless variables 
(here we use scale-of-$\phi$- independent versions of $g,m$
as was explained in previous lectures; note that here we have slightly 
different notation). Then, using formula 
(2.5) of lecture 2,  we see that the equation $Wm_p=0$ ($m_p$ is constant 
along the renormalization group flow) can be expressed 
as
$$
(\mu\frac{\d}{\d\mu}+\gamma m\frac{\d}{\d m}+\beta
\frac{\d }{\d g})m_p=0.\tag 4.2
$$
(we use the scale-of-the-field independence of $m_p$ to drop the term 
with $\frac{\d}{\d a}$).
Substituting (4.1) into (4.2), we have
$$
-m\d_zf+\gamma m f+\gamma m \d_zf+\beta m\d_gf=0, \tag 4.3
$$ 
or
$$
(\frac{\d}{\d z}-\frac{\beta}{1-\gamma}\frac{\d}{\d g}-
\frac{\gamma_m}{1-\gamma})f(g,z)=0.\tag 4.4
$$
Let us set $\bb=\beta/(1-\gamma)$, $\bga=\gamma/(1-\gamma)$. 
Then, solving (4.4), we get 
$$
m_p=mf(g,z)=f(\bg(z),0)e^{\int_0^z\bga(\bg(t))dt},\tag 4.5
$$
where $\bg$ satisfies the differential equation 
$\frac{d\bg}{dz}=\beta(\bg)$ with the initial condition $\bg(0)=g$. 
(assuming that the renormalization prescription 
used is mass-independent). 

Now consider the question of dynamical generation of mass. The question is:
can a theory which is asymptotically free and massless in the ultraviolet 
limit generate a mass in the infrared? 

Let us try to answer this question. Since our theory is asymptotically 
free, it is not infrared free. Assume that it has an IR stable 
fixed point $g^*\ne 0$, and is driven to this fixed point in the infrared
by the renormalization group flow. In this case $\lim_{z\to -\infty}\bg(z)=
g^*$, and formula (4.5) yields the asymptotics
$$
m_p\sim mf(g^*,0)(m/\mu)^{\frac{\gamma(g^*)}{1-\gamma(g^*)}}, m\to 0\tag 4.6
$$
(for sufficiently nice function $\gamma$). In particular, if 
$g^*$ is so small that $\gamma(g^*)<1$ (remember that ($\gamma(0)=0$)
then $m_p\to 0$ as $m\to 0$, 
so the mass cannot be generated dynamically. However
for $\gamma(g^*)>1$ mass generation takes place. Thus,  
whether dynamical mass generation occurs or not
for and IR stable fixed point $g^*\ne 0$
-- cannot be decided on the basis of perturbation theory. 

However, 
if the original theory was not asymptotically free, then mass generation 
surely is not possible: in this case $g_*=0$, and 
$m_p\sim m(\ln m)^p$ for some $p$, so $m_p\to 0$ as $m\to 0$. 

