%Date: Thu, 30 Jan 1997 17:06:22 -0500
%From: Edward Witten <witten@sns.ias.edu>

\input harvmac
Term Two

Problem Set Two

(1) Consider the supersymmetric model of maps of  ${\bf R}^{2,2}$
to $S^N$.  We formulated the Lagrangian in the past for maps $X:{\bf 
R}^{2,2}\to M$ for $M$ an arbitrary Riemannian manifold:

\eqn\abo{L=\int d^2x d^2\theta\,\, g_{IJ}(X)D_+X^ID_-X^J.}

Consider the case that $M$ is a sphere $S^N$ with a round metric.

By imitating the steps that we carried out in class for the bosonic
model with the same target, determine the large $N$ behavior of this
model.

Suggestion: Include the Lagrange multiplier in a manifestly supersymmetric
way, in superspace.   But when it comes to doing the path integral,
first do the theta integrals to reduce to ordinary Lagrangians and
path integrals and proceed in a fashion that should by now be familiar.   (We 
haven't developed supergraph techniques for
superspace evaluation of Feynman diagrams, and such techniques really
don't give much simplification for most problems as simple as this one,
anyway.) 

 This explicit procedure should make it clear that two similar-looking
mechanisms that we've studied are in fact related to each other by
supersymmetry.

(2) Prove that the model considered in (1), though it has a mass gap,
has a conserved charge of the sort ``not allowed'' by the Coleman-Mandula
theorem, that is, it is not a linear combination of an element of the 
Poincar\'e Lie algebra and an operator commuting with Poincar\'e.

 \end




