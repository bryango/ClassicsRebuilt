%%%%%%%%%%%%%%%%%%%%%%%%%%%%%%%%%%%%%%%%%%%%%%%%%%%%%%%%%%%%%%%%%%%%%%%%%%
%%%%%%%%%%%%% THIS IS AN AMSLATEX FILE %%%%%%%%%%%%%%%%%%%%%%%%%%%%%%%%%%%
%%%%%%%%%%%%%%%%%%%%%%%%%%%%%%%%%%%%%%%%%%%%%%%%%%%%%%%%%%%%%%%%%%%%%%%%%%

%Style section
\documentstyle[amscd,amssymb,verbatim,12pt]{amsart}

%Declaration section
\theoremstyle{plain}
\newtheorem{Thm}[subsection]{Theorem}
\newtheorem{Cor}[subsection]{Corollary}
\newtheorem{Lem}[subsection]{Lemma}
\newtheorem{Prop}[subsection]{Proposition}
\newtheorem{Exc}{Exercise}
\newtheorem{Claim}[subsection]{Claim}

\theoremstyle{definition}
\newtheorem{Def}[subsection]{Definition}

\theoremstyle{remark}
\newtheorem{notation}{Notation}
\renewcommand{\thenotation}{}
\newtheorem{Rem}{Remark}

%\newtheorem{exs}{Examples}
%\renewcommand{\theexs}{}

\newtheorem{note}{Note}
\renewcommand{\thenote}{}


%Commandsection
\errorcontextlines=0
\numberwithin{equation}{section}
\renewcommand{\rm}{\normalshape}



   

%Labeling macros
\newif\ifShowLabels
\ShowLabelstrue
\newdimen\theight
\def\TeXref#1{%
	\leavevmode\vadjust{\setbox0=\hbox{{\tt
		\quad\quad  {\small \rm #1}}}%
	\theight=\ht0
	\advance\theight by \lineskip
	\kern -\theight \vbox to
	\theight{\rightline{\rlap{\box0}}%
	\vss}%
	}}%

\ShowLabelsfalse% comment this out if labels should be printed

%Section titles that can be referenced
\renewcommand{\sec}[2]{\section{#2}\label{S:#1}%
	\ifShowLabels \TeXref{{S:#1}} \fi}
\newcommand{\ssec}[2]{\subsection{#2}\label{SS:#1}%
	\ifShowLabels \TeXref{{SS:#1}} \fi}
\newcommand{\sssec}[2]{\subsubsection{#2}\label{SSS:#1}%
	\ifShowLabels \TeXref{{SSS:#1}} \fi}

%Referencing sections and declarations
\newcommand{\refs}[1]{\ref{S:#1}}
\newcommand{\refss}[1]{\ref{SS:#1}}
\newcommand{\refsss}[1]{\ref{SSS:#1}}
\newcommand{\refe}[1]{\eqref{E:#1}}

\newenvironment{theor}{\sl}{\relax}

\newcommand{\sth}[1]{\subsection{Theorem}\label{SS:#1} \begin{theor} }
\newcommand{\th}{\subsection*{Theorem} \begin{theor} }
\renewcommand{\eth}{\end{theor} }

\newcommand{\sclaim}[1]{\subsection{Claim}\label{SS:#1} \begin{theor} }
\newcommand{\claim}{\subsection*{Claim}\begin{theor} }
\newcommand{\eclaim}{\end{theor}}

\newcommand{\slem}[1]{\subsection{Lemma}\label{SS:#1}\begin{theor} }
\newcommand{\lem}{\subsection*{Lemma}\begin{theor} }
\newcommand{\elem}{\end{theor}}



\newcommand{\sprop}[1]{\subsection{Proposition}\label{SS:#1}\begin{theor} }
\newcommand{\prop}{\subsection*{Proposition}\begin{theor} }
\newcommand{\eprop}{\end{theor}}


\newcommand{\scor}[1]{\subsection{Corollary}\label{SS:#1} \begin{theor} }
\newcommand{\cor}{\subsection*{Corollary} \begin{theor} }
\newcommand{\ecor}{\end{theor}}


\newcommand{\sdefe}[1]{\subsection{Definition}\label{SS:#1} \begin{theor} }
\newcommand{\defe}{\subsection*{Definition}\begin{theor} }
\newcommand{\edefe}{\end{theor}}

\newcommand{\sconj}[1]{\subsection{Conjecture}\label{SS:#1}\begin{theor}  }
\newcommand{\conj}{\subsection*{Conjecture}\begin{theor}  }
\newcommand{\econj}{\end{theor}}

\newcommand{\fact}{\subsection*{Fact}\begin{theor} }
\newcommand{\efact}{\end{theor}}

\newcommand{\eq}[1]%
	{ \ifShowLabels \TeXref{E:#1} \fi 
	   \begin{equation} \label{E:#1} }
\newcommand{\eeq}{ \end{equation} }

\newcommand{\prf}{ \begin{pf} }
\newcommand{\epr}{ \end{pf} }


%----------------------------------------------





%------------------------------------------------------------
%------------------------------------------------------------ 
 
%Greek letters
\newcommand\alp{\alpha}		\newcommand\Alp{\Alpha}
\newcommand\bet{\beta}		
\newcommand\gam{\gamma}		\newcommand\Gam{\Gamma}
\newcommand\del{\delta}		\newcommand\Del{\Delta}
\newcommand\eps{\varepsilon}		
\newcommand\zet{\zeta}
\newcommand\tet{\theta}		\newcommand\Tet{\Theta}
\newcommand\iot{\iota}
\newcommand\kap{\kappa}
\newcommand\lam{\lambda}		\newcommand\Lam{\Lambda}
\newcommand\sig{\sigma}		\newcommand\Sig{\Sigma}
\newcommand\vphi{\varphi}
\newcommand\ome{\omega}		\newcommand\Ome{\Omega}

%Caligraphic roman letters
\newcommand\calA{{\cal{A}}}
\newcommand\calB{{\cal{B}}}
\newcommand\calC{{\cal{C}}}
\newcommand\calD{{\cal{D}}}
\newcommand\calE{{\cal{E}}}
\newcommand\calF{{\cal{F}}}
\newcommand\calG{{\cal{G}}}
\newcommand\calH{{\cal{H}}}
\newcommand\calI{{\cal{I}}}
\newcommand\calJ{{\cal{J}}}
\newcommand\calK{{\cal{K}}}
\newcommand\calL{{\cal{L}}}
\newcommand\calM{{\cal{M}}}
\newcommand\calN{{\cal{N}}}
\newcommand\calO{{\cal{O}}}
\newcommand\calP{{\cal{P}}}
\newcommand\calQ{{\cal{Q}}}
\newcommand\calR{{\cal{R}}}
\newcommand\calS{{\cal{S}}}
\newcommand\calT{{\cal{T}}}
\newcommand\calU{{\cal{U}}}
\newcommand\calV{{\cal{V}}}
\newcommand\calW{{\cal{W}}}
\newcommand\calX{{\cal{X}}}
\newcommand\calY{{\cal{Y}}}
\newcommand\calZ{{\cal{Z}}}



%Bold roman letters
\newcommand\bfa{{\bf a}}		\newcommand\bfA{{\bf A}}
\newcommand\bfb{{\bf b}}		\newcommand\bfB{{\bf B}}
\newcommand\bfc{{\bf c}}		\newcommand\bfC{{\bf C}}
\newcommand\bfd{{\bf d}}		\newcommand\bfD{{\bf D}}
\newcommand\bfe{{\bf e}}		\newcommand\bfE{{\bf E}}
\newcommand\bff{{\bf f}}		\newcommand\bfF{{\bf F}}
\newcommand\bfg{{\bf g}}		\newcommand\bfG{{\bf G}}
\newcommand\bfh{{\bf h}}		\newcommand\bfH{{\bf H}}
\newcommand\bfi{{\bf i}}		\newcommand\bfI{{\bf I}}
\newcommand\bfj{{\bf j}}		\newcommand\bfJ{{\bf J}}
\newcommand\bfk{{\bf k}}		\newcommand\bfK{{\bf K}}
\newcommand\bfl{{\bf l}}		\newcommand\bfL{{\bf L}}
\newcommand\bfm{{\bf m}}		\newcommand\bfM{{\bf M}}
\newcommand\bfn{{\bf n}}		\newcommand\bfN{{\bf N}}
\newcommand\bfo{{\bf o}}		\newcommand\bfO{{\bf O}}
\newcommand\bfp{{\bf p}}		\newcommand\bfP{{\bf P}}
\newcommand\bfq{{\bf q}}		\newcommand\bfQ{{\bf Q}}
\newcommand\bfr{{\bf r}}		\newcommand\bfR{{\bf R}}
\newcommand\bfs{{\bf s}}		\newcommand\bfS{{\bf S}}
\newcommand\bft{{\bf t}}		\newcommand\bfT{{\bf T}}
\newcommand\bfu{{\bf u}}		\newcommand\bfU{{\bf U}}
\newcommand\bfv{{\bf v}}		\newcommand\bfV{{\bf V}}
\newcommand\bfw{{\bf w}}		\newcommand\bfW{{\bf W}}
\newcommand\bfx{{\bf x}}		\newcommand\bfX{{\bf X}}
\newcommand\bfy{{\bf y}}		\newcommand\bfY{{\bf Y}}
\newcommand\bfz{{\bf z}}		\newcommand\bfZ{{\bf Z}}

%Capital roman double letters 
\newcommand\QQ{\Bbb{Q}}
\newcommand\WW{\Bbb{W}}
\newcommand\EE{\Bbb{E}}
\newcommand\RR{\Bbb{R}}
\newcommand\TT{\Bbb{T}}
\newcommand\YY{\Bbb{Y}}
\newcommand\UU{\Bbb{U}}
\newcommand\II{\Bbb{I}}
\newcommand\OO{\Bbb{O}}
\newcommand\PP{\Bbb{P}}
\renewcommand\AA{\Bbb{A}}
\newcommand\SS{\Bbb{S}}
\newcommand\DD{\Bbb{D}}
\newcommand\FF{\Bbb{F}}
\newcommand\GG{\Bbb{G}}
\newcommand\HH{\Bbb{H}}
\newcommand\JJ{\Bbb{J}}
\newcommand\KK{\Bbb{K}}
\newcommand\LL{\Bbb{L}}
\newcommand\ZZ{\Bbb{Z}}
\newcommand\XX{\Bbb{X}}
\newcommand\CC{\Bbb{C}}
\newcommand\VV{\Bbb{V}}
\newcommand\BB{\Bbb{B}}
\newcommand\NN{\Bbb{N}}
\newcommand\MM{\Bbb{M}}


  %Euler Fraktur letters
\newcommand\grA{{\frak{A}}}	\newcommand\gra{{\frak{a}}}
\newcommand\grB{{\frak{B}}}	\newcommand\grb{{\frak{b}}}
\newcommand\grC{{\frak{C}}}	\newcommand\grc{{\frak{c}}}
\newcommand\grD{{\frak{D}}}	\newcommand\grd{{\frak{d}}}
\newcommand\grE{{\frak{E}}}	\newcommand\gre{{\frak{e}}}
\newcommand\grF{{\frak{F}}}	\newcommand\grf{{\frak{f}}}
\newcommand\grG{{\frak{G}}}	\newcommand\grg{{\frak{g}}}
\newcommand\grH{{\frak{H}}}	\newcommand\grh{{\frak{h}}}
\newcommand\grI{{\frak{I}}}	\newcommand\gri{{\frak{i}}}
\newcommand\grJ{{\frak{J}}}	\newcommand\grj{{\frak{j}}}
\newcommand\grK{{\frak{K}}}	\newcommand\grk{{\frak{k}}}
\newcommand\grL{{\frak{L}}}	\newcommand\grl{{\frak{l}}}
\newcommand\grM{{\frak{M}}}	\newcommand\grm{{\frak{m}}}
\newcommand\grN{{\frak{N}}}	\newcommand\grn{{\frak{n}}}
\newcommand\grO{{\frak{O}}}	\newcommand\gro{{\frak{o}}}
\newcommand\grP{{\frak{P}}}	\newcommand\grp{{\frak{p}}}
\newcommand\grQ{{\frak{Q}}}	\newcommand\grq{{\frak{q}}}
\newcommand\grR{{\frak{R}}}	\newcommand\grr{{\frak{r}}}
\newcommand\grS{{\frak{S}}}	\newcommand\grs{{\frak{s}}}
\newcommand\grT{{\frak{T}}}	\newcommand\grt{{\frak{t}}}
\newcommand\grU{{\frak{U}}}	\newcommand\gru{{\frak{u}}}
\newcommand\grV{{\frak{V}}}	\newcommand\grv{{\frak{v}}}
\newcommand\grW{{\frak{W}}}	\newcommand\grw{{\frak{w}}}
\newcommand\grX{{\frak{X}}}	\newcommand\grx{{\frak{x}}}
\newcommand\grY{{\frak{Y}}}	\newcommand\gry{{\frak{y}}}
\newcommand\grZ{{\frak{Z}}}	\newcommand\grz{{\frak{z}}}



\newcommand\nek{,\ldots,}
\newcommand\sdp{\times \hskip -0.3em {\raise 0.3ex
\hbox{$\scriptscriptstyle |$}}} % semidirect product

%words in roman font

\newcommand\area{\operatorname{area}}
\newcommand\Aug{\operatorname{Aug}}
\newcommand\Aut{\operatorname{Aut}}
\newcommand\Char{\operatorname{Char}}
\newcommand\Cl{\operatorname{Cl}}
\newcommand\cf{{\rm \,cf\,}}
\newcommand\Cone{\operatorname{Cone}}
\newcommand\cont{\operatorname{cont}}
\newcommand\codim{\operatorname{codim}}
\newcommand\conv{\operatorname{conv}}
\newcommand\Conv{\operatorname{Conv}}
\newcommand\const{\operatorname{const}}
\newcommand\Const{\operatorname{Const}}
\newcommand\Det{\operatorname{Det}}
\newcommand\diag{\operatorname{diag}}
\newcommand\diam{\operatorname{diam}}
\newcommand\Diam{\operatorname{Diam}}
\newcommand\dist{\operatorname{dist}}
\newcommand\Dom{\operatorname{Dom}}
\newcommand\dom{\operatorname{dom}}
\newcommand\End{\operatorname{End\,}}
\newcommand\Ext{\operatorname{Ext}}
\newcommand\esssup{\operatorname{ess\ sup}}
\newcommand\Ran{{\rm Ran}}
\newcommand\RANK{\operatorname{rank}}
\newcommand\Geo{\operatorname{Geo}}
\newcommand\GL{\operatorname{GL}}
\newcommand\Gr{\operatorname{Gr}}
\newcommand\gl{\operatorname{gl}}
\newcommand\grad{\mathop {\rm grad}}
\newcommand\Hom{\operatorname{Hom}}
\newcommand\RHom{\operatorname{RHom}}
\newcommand\im{\operatorname {im}}
\newcommand\IM{\operatorname{Im}}
\newcommand\Ind{\operatorname{Ind}}
\newcommand\ind{\operatorname{ind}}
\newcommand\Inf{\operatorname{Inf}}
\newcommand\Int{\operatorname{Int}}
\newcommand\Min{\operatorname{Min}}
\newcommand\MOD{\operatorname{mod}}
\newcommand{\Mor}{\operatorname{Mor}}
\newcommand{\Ob}{\operatorname{Ob\,}}
\newcommand\ord{\operatorname{ord}}
\newcommand\Ka{\operatorname{Ka}}
\newcommand\Ker{\operatorname{Ker}}
\newcommand\PGL{{\rm PGL}}
\newcommand\PGSp{{\rm PGSp}}
\newcommand\Plt{\operatorname{Plt}}
\newcommand\proj{\operatorname{proj}}
\newcommand\Proj{\operatorname{Proj}}
\newcommand\res{\rm res}
%\newcommand\rank{\rm rank}
\newcommand\rk{\operatorname{rk}}
\newcommand\Range{\operatorname{Range}}
%\newcommand\RE{\operatorname{Re}} 
\newcommand\Res{\operatorname{Res}}
\newcommand\rot{\operatorname{rot}}
\newcommand\Max{\operatorname{Max}}
\newcommand\Maximum{\operatorname{Maximum}}
\newcommand\Minimum{\operatorname{Minimum}}
\newcommand\Minimize{\operatorname{Minimize}}
\newcommand\Prob{\operatorname{Prob}}
\newcommand\sech{\rm sech}
\newcommand\sgn{\operatorname{sgn}}
\newcommand{\sign}{\operatorname{sign}}
\newcommand\SL{{\rm SL}}
\newcommand\Sbm{\operatorname{Sbm}}
\newcommand\SO{{\rm SO}}
\newcommand\SPAN{\operatorname{span}}
\newcommand\spec{{\rm spec}}
\newcommand\supess{\operatorname{sup\ ess}}
\newcommand\supp{\operatorname{supp}}
\newcommand\Supp{\operatorname{Supp}}
\newcommand\Sup{\operatorname{Sup}}
\newcommand\Sym{\operatorname{Sym}}
\newcommand\tr{\operatorname{tr}}
\newcommand\Tr{\operatorname{Tr}}
\newcommand\Tor{\operatorname{Tor}}
\newcommand\Var{\operatorname{Var}}
\newcommand\Vol{\operatorname{Vol}}
%\newcommand\vol{\operatorname{vol}}

%overlined math alphabet
\newcommand\oa{{\overline{a}}}
\newcommand\oA{{\overline{A}}}
\newcommand\ob{{\overline{b}}}
\newcommand\oB{{\overline{B}}}
\newcommand\oc{{\overline{c}}}
\newcommand\oC{{\overline{C}}}
\newcommand\oD{{\overline{D}}}
\newcommand\od{{\overline{d}}}
\newcommand\oE{{\overline{E}}}
\renewcommand\oe{{\overline{e}}}
\newcommand\of{{\overline{f}}}
\newcommand\oF{{\overline{F}}}
\newcommand\og{{\overline{g}}}
\newcommand\oG{{\overline{G}}}
\newcommand\oh{{\overline{h}}}
\newcommand\oH{{\overline{H}}}
\newcommand\oI{{\overline{I}}}
\newcommand\oj{{\overline{j}}}
\newcommand\oJ{{\overline{J}}}
\newcommand\ok{{\overline{k}}}
\newcommand\oK{{\overline{K}}}
\newcommand\oL{{\overline{L}}}
\newcommand\om{{\overline{m}}}
\newcommand\oM{{\overline{M}}}
\newcommand\oN{{\overline{N}}}
\newcommand\oO{{\overline{O}}}
\newcommand\oo{{\overline{o}}}
\newcommand\op{{\overline{p}}}
\newcommand\oP{{\overline{P}}}
\newcommand\oq{{\overline{q}}}
\newcommand\oQ{{\overline{Q}}}
\newcommand\OR{{\overline{r}}}
\newcommand\oS{{\overline{S}}}
\newcommand\os{{\overline{s}}}
\newcommand\ot{{\overline{t}}}
\newcommand\oT{{\overline{T}}}
\newcommand\ou{{\overline{u}}}
\newcommand\oU{{\overline{U}}}
\newcommand\ov{{\overline{v}}}
\newcommand\oV{{\overline{V}}}
\newcommand\ow{{\overline{w}}}
\newcommand\oW{{\overline{W}}}
\newcommand\ox{{\overline{x}}}
\newcommand\oX{{\overline{X}}}
\newcommand\oy{{\overline{y}}}
\newcommand\oY{{\overline{Y}}}
\newcommand\oz{{\overline{z}}}
\newcommand\oZ{{\overline{Z}}}

%overlined Greek alphabet
\newcommand\oalp{{\overline{\alpha}}}
\newcommand\obet{{\overline{\bet}}}
\newcommand\odel{{\overline{\del}}}
\newcommand\oDel{{\overline{\Del}}}
\newcommand\ocup{{\overline{\cup}}}
\newcommand\ovarphi{{\overline{\varphi}}}
\newcommand\ochi{{\overline{\chi}}}
\newcommand\oeps{{\overline{\eps}}}
\newcommand\oeta{{\overline{\eta}}}
\newcommand\ogam{{\overline{\gam}}}
\newcommand\okap{{\overline{\kap}}}
\newcommand\olam{{\overline{\lambda}}}
\newcommand\oLam{{\overline{\Lambda}}} 
\newcommand\omu{{\overline{\mu}}}
\newcommand\onu{{\overline{\nu}}}
\newcommand\oOme{{\overline{\Ome}}}
\newcommand\ophi{\overline{\phi}}
\newcommand\oPhi{{\overline{\Phi}}}
\newcommand\opi{{\overline{\pi}}}
\newcommand\oPsi{{\overline{\Psi}}}
\newcommand\opsi{{\overline{\psi}}}
\newcommand\orho{{\overline{\rho}}}
\newcommand\osig{{\overline{\sig}}}
\newcommand\otau{{\overline{\tau}}}
\newcommand\otet{{\overline{\theta}}}
\newcommand\oxi{{\overline{\xi}}}
\newcommand\oome{\overline{\ome}}
\newcommand\opart{{\overline{\partial}}}


%underlined math alphabet
\newcommand\ua{{\underline{a}}}
\newcommand\ub{{\underline{b}}}
\newcommand\uc{{\underline{c}}}
\newcommand\uD{{\underline{D}}}
\newcommand\uk{{\underline{k}}}
\newcommand\ue{{\underline{e}}}
\newcommand\uj{{\underline{j}}}
\newcommand\ul{{\underline{l}}}
\newcommand\uL{{\underline{L}}}
\newcommand\uo{{\underline{o}}}
\newcommand\uO{{\underline{O}}}
\newcommand\uP{{\underline{P}}}
\newcommand\uQ{{\underline{Q}}}
\newcommand\um{{\underline{m}}}
\newcommand\uM{{\underline{M}}}
\newcommand\un{{\underline{n}}}
\newcommand\us{{\underline{s}}}
\newcommand\ut{{\underline{t}}}
\newcommand\uu{{\underline{u}}}
\newcommand\uv{{\underline{v}}}
\newcommand\uV{{\underline{V}}}
\newcommand\ux{{\underline{x}}}
\newcommand\uX{{\underline{X}}}
\newcommand\uy{{\underline{y}}}
\newcommand\uz{{\underline{z}}}

%underline Greek alphabet
\newcommand\ualp{{\underline{\alp}}}
\newcommand\ubet{{\underline{\bet}}}
\newcommand\uchi{{\underline{\chi}}}
\newcommand\udel{{\underline{\del}}}
\newcommand\uell{{\underline{\ell}}}
\newcommand\ueps{{\underline{\eps}}}
\newcommand\ueta{{\underline{\eta}}}
\newcommand\uGam{{\underline{\Gamma}}}
\newcommand\unu{{\underline{\nu}}}
\newcommand\uome{{\underline{\omega}}}
\newcommand\utet{{\underline{\tet}}}
\newcommand\ulam{{\underline{\lam}}}


%math alphabet with hat
\newcommand\hata{{\widehat{a}}}
\newcommand\hatA{{\widehat{A}}}
\newcommand\hatb{{\widehat{b}}}
\newcommand\hatc{{\widehat{c}}}
\newcommand\hatC{{\widehat{C}}}
\newcommand\hatB{{\widehat{B}}}
\newcommand\hatD{{\widehat{D}}}
\newcommand\hatd{{\hat{d}}}
\newcommand\hate{{\widehat{e}}}
\newcommand\hatE{{\widehat{E}}}
\newcommand\hatf{{\widehat{f}}}
\newcommand\hatF{{\widehat{F}}}
\newcommand\hatg{{\widehat{g}}}
\newcommand\hatG{{\widehat{G}}}
\newcommand\hath{{\widehat{h}}}
\newcommand\hatH{{\widehat{H}}}
\newcommand\hati{{\hat{i}}}
\newcommand\hatI{{\hat{I}}}
\newcommand\hatj{{\widehat{j}}}
\newcommand\hatJ{{\widehat{J}}}
\newcommand\hatk{{\widehat{k}}}
\newcommand\hatK{{\widehat{K}}}
\newcommand\hatL{{\widehat{L}}}
\newcommand\hatm{{\widehat{m}}}
\newcommand\hatM{{\widehat{M}}}
\newcommand\hatn{{\widehat{n}}}
\newcommand\hatN{{\widehat{N}}}
\newcommand\hatp{{\widehat{p}}}
\newcommand\hatP{{\widehat{P}}}
\newcommand\hatr{{\widehat{r}}}
\newcommand\hatR{{\widehat{R}}}
\newcommand\hatq{{\widehat{q}}}
\newcommand\hatQ{{\widehat{Q}}}
\newcommand\hatT{{\widehat{T}}}
\newcommand\hatu{{\widehat{u}}}
\newcommand\hatU{{\widehat{U}}}
\newcommand\hatV{{\widehat{V}}}
\newcommand\hatv{{\widehat{v}}}
\newcommand\hatw{{\widehat{w}}}
\newcommand\hatW{{\widehat{W}}}
\newcommand\hatx{{\widehat{x}}}
\newcommand\hatX{{\widehat{X}}}
\newcommand\haty{{\widehat{y}}}
\newcommand\hatY{{\widehat{Y}}}
\newcommand\hatZ{{\widehat{Z}}}
\newcommand\hatz{{\widehat{z}}}

%Greek alphabet with hat
\newcommand\hatalp{{\widehat{\alpha}}}
\newcommand\hatdel{{\widehat{\delta}}}
\newcommand\hatDel{{\widehat{\Delta}}}
\newcommand\hatbet{{\widehat{\beta}}}
\newcommand\hateps{{\hat{\eps}}}
\newcommand\hatgam{{\widehat{\gamma}}}
\newcommand\hatGam{{\widehat{\Gamma}}}
\newcommand\hatlam{{\widehat{\lambda}}}
\newcommand\hatmu{{\widehat{\mu}}}
\newcommand\hatnu{{\widehat{\nu}}}
\newcommand\hatOme{{\widehat{\Ome}}}
\newcommand\hatphi{{\widehat{\phi}}}
\newcommand\hatPhi{{\widehat{\Phi}}}
\newcommand\hatpi{{\widehat{\pi}}}
\newcommand\hatpsi{{\widehat{\psi}}}
\newcommand\hatPsi{{\widehat{\Psi}}}
\newcommand\hatrho{{\widehat{\rho}}}
\newcommand\hatsig{{\widehat{\sig}}}
\newcommand\hatSig{{\widehat{\Sig}}}
\newcommand\hattau{{\widehat{\tau}}}
\newcommand\hattet{{\widehat{\theta}}}
\newcommand\hatvarphi{{\widehat{\varphi}}}
\newcommand\hatZZ{{\widehat{\ZZ}}}


%roman with widetilde

\newcommand\tilA{{\widetilde{A}}}
\newcommand\tila{{\widetilde{a}}}
\newcommand\tilB{{\widetilde{B}}}
\newcommand\tilb{{\widetilde{b}}}
\newcommand\tilc{{\widetilde{c}}}
\newcommand\tilC{{\widetilde{C}}}
\newcommand\tild{{\widetilde{d}}}
\newcommand\tilD{{\widetilde{D}}}
\newcommand\tilE{{\widetilde{E}}}
\newcommand\tilf{{\widetilde{f}}}
\newcommand\tilF{{\widetilde{F}}}
\newcommand\tilg{{\widetilde{g}}}
\newcommand\tilG{{\widetilde{G}}}
\newcommand\tilh{{\widetilde{h}}}
\newcommand\tilk{{\widetilde{k}}}
\newcommand\tilK{{\widetilde{K}}}
\newcommand\tilj{{\widetilde{j}}}
\newcommand\tilm{{\widetilde{m}}}
\newcommand\tilM{{\widetilde{M}}}
\newcommand\tilH{{\widetilde{H}}}
\newcommand\tilL{{\widetilde{L}}}
\newcommand\tilN{{\widetilde{N}}}
\newcommand\tiln{{\widetilde{n}}}
\newcommand\tilO{{\widetilde{O}}}
\newcommand\tilP{{\widetilde{P}}}
\newcommand\tilp{{\widetilde{p}}}
\newcommand\tilq{{\widetilde{q}}}
\newcommand\tilQ{{\widetilde{Q}}}
\newcommand\tilR{{\widetilde{R}}}
\newcommand\tilr{{\widetilde{r}}}
\newcommand\tilS{{\widetilde{S}}}
\newcommand\tils{{\widetilde{s}}}
\newcommand\tilT{{\widetilde{T}}}
\newcommand\tilt{{\widetilde{t}}}
\newcommand\tilu{{\widetilde{u}}}
\newcommand\tilU{{\widetilde{U}}}
\newcommand\tilv{{\widetilde{v}}}
\newcommand\tilV{{\widetilde{V}}}
\newcommand\tilw{{\widetilde{w}}}
\newcommand\tilW{{\widetilde{W}}}
\newcommand\tilX{{\widetilde{X}}}
\newcommand\tilx{{\widetilde{x}}}
\newcommand\tily{{\widetilde{y}}}
\newcommand\tilY{{\widetilde{Y}}}
\newcommand\tilZ{{\widetilde{Z}}}
\newcommand\tilz{{\widetilde{z}}}

%Greek alphabet with widetilde
\newcommand\tilalp{{\widetilde{\alpha}}}
\newcommand\tilbet{{\widetilde{\beta}}}
\newcommand\tildel{{\widetilde{\delta}}}
\newcommand\tilDel{{\widetilde{\Delta}}}
\newcommand\tilchi{{\widetilde{\chi}}}
\newcommand\tileta{{\widetilde{\eta}}}
\newcommand\tilgam{{\widetilde{\gamma}}}
\newcommand\tilGam{{\widetilde{\Gamma}}}
\newcommand\tilome{{\widetilde{\ome}}}
\newcommand\tillam{{\widetilde{\lam}}}
\newcommand\tilmu{{\widetilde{\mu}}}
\newcommand\tilphi{{\widetilde{\phi}}}
\newcommand\tilpi{{\widetilde{\pi}}}
\newcommand\tilpsi{{\widetilde{\psi}}}
\renewcommand\tilome{{\widetilde{\ome}}}
\newcommand\tilOme{{\widetilde{\Ome}}}
\newcommand\tilPhi{{\widetilde{\Phi}}}
\newcommand\tilQQ{{\widetilde{\QQ}}}
\newcommand\tilrho{{\widetilde{\rho}}}
\newcommand\tilsig{{\widetilde{\sig}}}
\newcommand\tiltau{{\widetilde{\tau}}}
\newcommand\tiltet{{\widetilde{\theta}}}
\newcommand\tilvarphi{{\widetilde{\varphi}}}
\newcommand\tilxi{{\widetilde{\xi}}}


\newcommand\twolongrightarrow{\ \hbox{$\longrightarrow\hskip -17pt
\longrightarrow$}\ }
 
 
\renewcommand\+{\oplus }
\newcommand\x{\times }
\newcommand\ten{\otimes}
\newcommand\opd{\oplus\cdots\oplus}
\newcommand\oxd{\otimes\cdots\otimes}
\renewcommand\pd{+\cdots +}
\newcommand\dm{-\cdots -}
\newcommand\dx{\cdot\dots\cdot}

\renewcommand{\>}{\rangle}
\newcommand{\<}{\langle}

%\newcommand\gt{\text{{\bf $>$--}}}
%\newcommand\gf{{\bf >\text{--}<}}
%\newcommand\gc{\text{{\bf --$\bigcirc$--}}}

\newcommand\loc{\calL oc}
\newcommand\la{\langle}
\newcommand\ra{\rangle}






\renewcommand\d{\partial}
\newcommand\Lie{\text{Lie}}
\newcommand\boxx{\boxtimes}
\begin{document}
\title[]{
Solution of homeworks in string theory}

\author[]{by Alexander Braverman and Dennis Gaitsgory}

\maketitle


\sec{}{Solutions to problem sets 1 and 2}
%---------------------------------------------------------------------------
\ssec{not}{Remarks on notations}
In what follows $\Sigma$ will denote a compact Riemann surface of genus $h$,
usually with a fixed complex structure. We let $\Delta:\Ome^{0,0}(\Sig)\to
\Omega^{1,1}(\Sigma)$ denote the corresponding Laplace operator $\Delta=
\d{\overline\d}$. Let us now choose a metric $g$ on $\Sigma$ compatible 
with the complex structure. Then we denote by $R_g\in \Omega^{1,1}(\Sigma)$
the curvature of the corresponding Levi-Civita connection and by
$d\mu_g$ -- the corresponding measure ($(1,1)$-form) on $\Sig$. We set 
$\Ome_{\Sig}=\Ome^{1,0}(\Sig)$.

Assuming again that the metric $g$ as above is chosen we may define
an operator ${\Delta}_g^{-1}:\Omega^{1,1}(\Sig)\to \calO(\Sig)$.
By definition, $\Del^{-1}_g(w)$ (for any $(1,1)$-form $\ome$) 
is the unique function on 
$\Sig$, such that 

%------------------------------------------------------------------------
\eq{}
\int\limits_{\Sig}\Del^{-1}_g(w)d\mu_g=0\quad\text{and}\quad
\Del(\Del^{-1}_g(\ome))=\ome-{d\mu_g
\over \text{vol}(\Sigma)}\cdot\int\limits_{\Sig}\ome
\end{equation}
%-------------------------------------------------------------------------
\ssec{}{Problem set 1: Problem 1b}
%-------------------------------------------------------------------------
\sssec{proj}{Digression on projective connections}
Let $\Sig$ and $\Omega$ be as in \refss{not}.
Take $c\in \CC$. 
We want to associate to it a canonical
torsor $\calP_c$ over the line bundle $\Omega^{\ten 2}_{\Sig}$, 
called the torsor
of projective $c$-connections. In fact,
we will construct an $\calO$-module, which is an extension of
$\calO$ by $\Omega^{\ten 2}$ and $\calP_c$ will be the preimage of 
$c\in\CC\subset \calO$.
The basic property of $E$ is that it is natural. In particular, vector fields
on $\Sig$ act on $E$ by Lie derivatives.

First a simple observation.
%-----------------------------------------------------------------------
\claim For any $i,j,k\in \ZZ$ there exists a canonical isomorphism 
%--------------------------------------------------------------------------
\eq{}
\Omega^{\ten i}\boxtimes \Omega^{\ten j}(k\Delta)/
\Omega^{\ten i}\boxtimes \Omega^{\ten j}((k-1)\Delta)
\simeq
\Delta_*(\Omega^{\ten(i+j-k)})
\end{equation}
%-------------------------------------------------------------------------
Here $\Delta$ is the diagonal in $\Sig\x \Sig$.
\eclaim
%----------------------------------------------------------------------------

Consider now the sheaf (it's not a sheaf of $\calO$-modules) on
$\Sig\x \Sig$
$$
(\Omega\boxtimes\Omega(2\Delta)/\Omega\boxtimes\Omega(-\Delta))^{S_2}
$$

Here $S_2$ denotes the corresponding symmetric group, which acts naturally
on $\Omega\boxtimes\Omega(2\Delta)/\Omega\boxtimes\Omega(-\Delta)$.

Let $E$ denote the sheaf-theoretic direct image of this sheaf to $\Sig$
with respect to either the first or the second projection (the two 
direct images are canonically isomorphic, since we took $S_2$-invariants). 
It is easy to see that $E$ acquires a canonical structure of an 
$\calO_{\Sig}$-module (which, however, will not be used in the sequel).
We have canonical exact sequence
%--------------------------------------------------------------------------
\eq{}
0\to \Ome^{\ten 2}\to E\to \calO\to 0
\end{equation}
%-------------------------------------------------------------------------
We let $p$ denote the above morphism $E\to \calO$ and set $\calP_c=p^{-1}(c)$.

If $z$ is a local coordinate on an open subset of $\Sigma$, the torsor
$\calP_c$ acquires a canonical section $P^c_z$ given by 
$(c/2){dz_1\boxtimes dz_2\over (z_1-z_2)^2}
\in \Omega\boxtimes\Omega(2\Delta)$ (one has to divide $c$ by $2$ in this
formula, since the residue of ${dz_1\boxtimes dz_2\over (z_1-z_2)^2}$ on
the diagonal in the natural normalization is equal to $2$).

Let now  $z'=f(z)$ be another local coordinate on the same open subset.

\claim $P^c_z-P^c_{z'}=-{c\over 12}(dz)^{\ten 2}\{f(z),z\}$, where
$\{f(z),z\}$ is the Schwartzian derivative of $f$ with respect to $z$
$$
\{f(z),z\}={f'''f'-{3\over 2}(f'')^2\over (f')^2}
$$

\eclaim
%----------------------------------------------------------------------
\sssec{}{Solution of (b)}Now we are ready to solve part (b) of the problem.
We are working with free conformal field theory, whose classical action
is
\eq{action}
S[\phi,g]={1\over 4\pi}\int\limits_{\Sigma} {1\over 2}\d \phi{\overline \d}
\phi
\end{equation}
Here $\phi$ is a map from $\Sigma$ to $\RR^c$.
We want to make sense of $:(\d\phi(x))^2:$ as of a quantum field with 
values in projective $c$-connections (note that $c$ is precisely the
central charge of our theory). This means that for
any collection of points $y_1,...,y_n\in\Sigma$, such that
$y_i\neq y_j$ for $i\neq j$ and $y_i\neq x$, and a collection
of fields $\phi_1,...,\phi_n$, the correlation function
$\la :(\d\phi(x))^2: \phi_1(y_1)...\phi_n(y_n)\ra$ is a section
of the torsor $\calP_c$ in the variable $x$.

The normal ordering $:(\d\phi(x))^2:$ is defined as follows.
Let $z$ be a local complex coordinate on an open
subset $U\subset \Sigma$ and let $y_1,...,y_n,\phi_1,...\phi_n$
be as above. Assume also that $y_i\not\in U$ for any $i=1,...,n$. Then we
have a section
%---------------------------------------------------------------------------
\eq{}
\la \d\phi(x)\d\phi(x')\phi_1(y_1)...\phi_n(y_n)\ra\in 
\Gamma(U\x U\backslash \Delta(U),\Omega\boxtimes\Omega)
\end{equation}
%----------------------------------------------------------------------------
The correlation function $\la :(\d\phi(x))^2: \phi_1(y_1)...\phi_n(y_n)\ra$
is defined by
%----------------------------------------------------------------------------
$$
\begin{aligned}
\la :(\d\phi(x))^2: \phi_1(y_1)...\phi_n(y_n)\ra=&\\
\lim_{x'\to x}(\la \d\phi(x)\d\phi(x')\phi_1(y_1)...\phi_n(y_n)\ra
+c{dz\boxtimes dz\over (z(x_1)-z(x_2))^2}&
+P^c_z\la \phi_1(y_1)...\phi_n(y_n)\ra
\end{aligned}
$$
%---------------------------------------------------------------------------
%\end{equation}

It is now obvious that $\la :(\d\phi(x))^2: \phi_1(y_1)...\phi_n(y_n)\ra$
as a section of $\calP_c$ (when $y_1,...y_n$ are kept fixed)
does not depend on the choice of the coordinate $z$ due to the definition
of $P^c_z(x)$ and to the fact that 
$\la :(\d\phi(x)\d\phi(x')\phi_1(y_1)...\phi_n(y_n)\ra$ is independent of $z$ (as a section of 
$\Omega\boxtimes\Omega)(2\Delta)$.

Now, in order to solve the other items of this problem 
we will first explain the solution of problem 1 in problem set 2
(i.e. we will explain the invariant meaning of the operator product
expansion for the stress tensor).

%------------------------------------------------------------------------------
\ssec{}{Problem set 2: problem 1}In this problem we have to show that
the right hand side of the OPE of the holomorphic stress tensor
with itself is independent of the choice of coordinates. Let us
say what it means using the abvoe notations.

We are working again with conformal field theory described by \refe{action}.
Let $z$ be a local coordinate
on some open subset $U$ of $\Sigma$. We are given the following section of
$E$ on $U$ denoted by $T(x)$
$$
\la :(\d\phi(x)^2):\phi_1(y_1)...\phi_n(y_n)\ra
$$
where $\phi_1...\phi_n$, $y_1,...,y_n$ are as above. The right-hand side 
of the OPE is the following section of 
$\Omega^{\ten 2}\boxtimes\Ome^{\ten 2}(4\Delta)/
\Omega^{\ten 2}\boxtimes\Ome^{\ten 2}$
over $U\times U$:
\eq{key}
{(c/2) dz^{\ten 2}\boxtimes dz^{\ten 2}\over (z(x)-z(y))^4} +
{2dz^{\ten 2}\boxtimes(T(y)-P^c_z(y))\over (z(x)-z(y))^2}+
{\Lie_{\d_z}(T(y)-P^c_z(y))\over z(x)-z(y)}
\end{equation}

Our task is to show that this expression is, in fact, independent of
the choice of the coordinate $z$. We claim, that it has, in fact, 
the following invariant meaning. 

Let $T$ be any projective $c$-connection on $U$, i.e. a section
of $(\Ome\boxtimes\Ome(2\Delta)/\Ome\boxx\Ome(-\Del))^{S_2}$.
Then {\it a priori} $T^2=T\ten T$ is a well-defined section of 
$(\Ome^{\ten 2}\boxtimes\Ome^{\ten 2}(4\Delta)/
\Ome^{\ten 2}\boxx\Ome^{\ten 2}(\Del))^{S_2}$.
However, it is easy to see, that
%-------------------------------------------------------------------------- 
\eq{}
(\Ome^{\ten 2}\boxtimes\Ome^{\ten 2}(4\Delta)/
\Ome^{\ten 2}\boxx\Ome^{\ten 2}(\Del))^{S_2}=
(\Ome^{\ten 2}\boxtimes\Ome^{\ten 2}(4\Delta)/
\Ome^{\ten 2}\boxx\Ome^{\ten 2})^{S_2}
\end{equation}
%-------------------------------------------------------------------------
(on $U\x U$). Therefore, $T^2$ is a well defined section
of $(\Ome^{\ten 2}\boxtimes\Ome^{\ten 2}(4\Delta)/
\Ome^{\ten 2}\boxx\Ome^{\ten 2})^{S_2}$.

The invariance property of \refe{key} follows from the next 

%--------------------------------------------------------------------------
\claim For any choice of a coordinate $z$
\eq{}
T^2= 
{c\over 2}({(c/2) dz^{\ten 2}\boxtimes dz^{\ten 2}\over (z(x)-z(y))^4} +
{2dz^{\ten 2}\boxtimes(T(y)-P^c_z(y))\over (z(x)-z(y))^2}
{dz^{\ten 2}\boxx\Lie_{\d_z}(T(y)-P^c_z(y)\over z(x)-z(y)})
\end{equation}
\eclaim
%--------------------------------------------------------------------------

\prf
Let our projective connection $T$ be written in the local 
coordinate $z$ as 
\eq{}
T={{c\over 2} dz\boxx dz\over (z(x)-z(y))^2} + 
f(x,y)dz\boxx dz
\end{equation}
where $f(x,y)=f(y,x)$.
Then $T^2$ as a section of 
$\Ome^{\ten 2}\boxx\Ome^{\ten 2}(4\Del)/\Ome\boxx\Ome$ equals
$$
\begin{aligned}
T^2=({{c\over 2} dz\boxx dz\over (z(x)-z(y))^2} + 
f(x,y)dz\boxx dz)^2=&\\
{c^2\over 4} {dz^{\ten 2}\boxx dz^{\ten 2}\over (z(x)-z(y))^4}+
cf(x,y)&{dz^{\ten 2}\boxx dz^{\ten 2}\over (z(x)-z(y))^2}=\\
{c^2\over 4} {dz^{\ten 2}\boxx dz^{\ten 2}\over (z(x)-z(y))^4}+
cf(y,y){dz^{\ten 2}\boxx dz^{\ten 2}\over (z(x)-z(y))^2}+&
c{dz^{\ten 2}\boxx dz^{\ten 2}\over z(x)-z(y)}
({f(x,y)-f(y,y)\over z(x)-z(y)}=\\
{c\over 2}({c\over 2}{dz^{\ten 2}\boxx dz^{\ten 2}\over (z(x)-z(y))^4}+
{2dz^{\ten 2}\boxx (T(y)-P^c_z(y))\over (z(x)-z(y))^2}+&
{dz^{\ten 2}\boxx\Lie_{d_z}(T(y)-P^c_z(y)\over z(x)-z(y)})
\end{aligned}
$$
\epr
%----------------------------------------------------------------------------
\ssec{1a}{Problem 1a}

Here we are going to prove the OPE for the stress tensor in
an arbitrary conformal field theory of central charge $c$.
Let us remind that a conformal field theory assigns to a 
finite collection of labels $\phi_1,...,\phi_n$ a correlation
function $\la \phi(x_1)...\phi(x_n)\ra$, which is an element
in $\Gamma(\Sigma^n\backslash\text{Diag},\calL_1\boxx...\boxx\calL_n)$
where $\calL_i$ is a natural $\calO_{\Sigma}$-module, assigned to the field
$\phi_i$ (the word ``natural'' should be understood in the sense that vector
fields on $\Sigma$ act on the sections of $\calL_i$ by Lie
derivatives).

We will use also the following notation. Let $\calM$ be an $\calO$-module
on $\Sigma$ and $m$ its section. The expression
$\< (m\ten\phi(x))\phi_1(y_1)...\phi_n(y_n)\>$ is by definition a section of
$\calM\ten \calL\boxx\calL_1...\calL_n$, equal to the tensor product
of $m$ by $\<\phi(x)\phi_1(y_1)...\phi_n(y_n)\>$.

The holomorphic stress tensor is a field, 
denoted by $T$, with the corresponding
$\calO$-module being $E$ (cf. solution of problem 1b), with the property
that
%--------------------------------------------------------------------------- 
\eq{}
p_x(\la T(x)\phi(x_1)...\phi(x_n)\ra)=c\la \phi(x_1)...\phi(x_n)\ra
\end{equation}
%------------------------------------------------------------------------
(recall that $p:E\to \calO$ denotes 
the natural projection -- cf. \refss{not}).

The basic property of the stress tensor $T$ is its connection with
Noether's currents (cf. Dan Freed's lecture). Namely, let $U$ be an
open subset of $\Sigma$ with a coordinate $z$. The choice
of $z$ allows us to view all projective $c$-connections
on $U$ as quadratic differentials by means
of $P\to P-P^c_z$. Let now $v$ be a holomorphic vector field defined
in a neighbourhood of a point $x_0$ in $U$.

\claim 
\eq{}
\Res_{x\to x_0}\la (v\ten (T(x)-P_z^c(x)))\phi_0(x_0)\phi(y_1)...\phi(y_n)\ra=
\text{Lie}_{x_0}v(\la\phi_0(x_0)\phi(y_1)...\phi(y_n)\ra)
\end{equation}

where $\text{Lie}_{x_0}v(\la\phi_0(x_0)\phi(y_1)...\phi(y_n)\ra)$ 
denotes the Lie derivative of the correlation function 
$\la\phi_0(x_0)\phi(y_1)...\phi(y_n)\ra$
with respect to $v$ acting on the zero-th component.
Note that the right-hand side is {\it a priori} independent of the
choice of the coordinate $z$.
\eclaim

\

This formula can be used to derive the OPE between the stress
tensor and any other fields. Here we will use it in the case of
the OPE for $T(x)T(y)$.

Let $z$ be once again a coordinate on $U\subset \Sigma$.
We must prove the following
%--------------------------------------------------------------
\lem 
$$
\begin{aligned}
(T(x)-&P_z^c(x))T(x_0)-\\& ({(c/2) dz^{\ten 2}\boxtimes 
dz^{\ten 2}\over (z(x)-z(x_0))^4} +
{dz^{\ten 2}\boxtimes(T(x_0)-P^c_z(x_0))\over (z(x)-z(x_0))^2}+
{\Lie_{\d_z}(T(x_0)-P^c_z(x_0)\over z(x)-z(x_0)})
\end{aligned}
$$
is a regular section of 
$\Ome^{\ten 2}\boxx E$ (meaning that it becomes such, when
inserted into a correlation function with any other fields).
\elem

\prf
We assume that $z(x_0)=0$.
According to the above claim, all we have to check is the following:
\begin{itemize}
\item
\eq{zero}
\Lie_{\d _z}(T_{x_0})=\Lie_{\d _z}(T_{x_0}-P_z^c(x_0))
\end{equation}

This holds since $\Lie_{\d _z}(P_z^c(x_0))=0$ (as $\d _z$ belongs to 
the Lie algebra of $\text{PGL}(2)$ when $U$ is embedded into $\PP^1$
by means of $z$).
\item
\eq{one}
\Lie_{z\d _z}(T_{x_0})=2(T_{x_0}-P_z^c(x_0))
\end{equation}

This holds because $\Lie_{z\d _z}(P_z^c(x_0))=0$ for the same reason
as above, and for any quadratic differential $\ome$ one has
$\Lie_{z\d _z}(\ome)(0)=2\ome(0)$.
\item
\eq{three}
\Lie_{z^3\d _z}(T_{x_0})=(c/2)(dz)^{\ten 2}
\end{equation}

Indeed, $\Lie_{z^i\d _z}(T_{x_0}-P_z^c(x_0))=0$ for $i\geq 2$
(this holds for any quadratic differential) and
$\Lie_{z^3\d _z}(P_z^c(x_0))=(c/2)(dz)^{\ten 2}$
which is easy to check by a direct calculation.
\item
\eq{two}
\Lie_{z^i\d _z}(T_{x_0})=0\qquad \text{for $i=2$ or $i>3$}
\end{equation}

This holds since $\Lie_{z^i\d _z}(T_{x_0}-P_z^c(x_0))=0$
as explained above and also one has $\Lie_{z^i\d _z}(P_z^c(x_0))=0$,
since $z^2\d _z$ still belongs to the Lie
algebra of $\text{PGL}(2)$, and $\Lie_{z^i\d _z}(P)=0$
for $i>3$ for any projective $c$-connection $P$ 
(for $i>3$ $z^i\d _z$ is a trivial vector field on 
$\text{Spec}(\CC[z]/z^4)$). 
\end{itemize}
\epr

%--------------------------------------------------------------------------
\ssec{}{Problem set 1: problem 2}
\sssec{}{Formulation of the problem}
Let $\Sigma$ be Riemann surface of
genus $h$, endowed with a complex structure. 
We are dealing with the action 
\eq{}
S[\phi,g]={1\over 4\pi}\int\limits_{\Sigma} {1\over 2}\d \phi{\overline \d}
\phi+Q\phi R_g
\end{equation}   
where $\phi$ is a real-valued function on $\Sigma$ and $g$ and $R_g$ are as in
\refss{not}. The goal of the problem is to investigate the corresponding
quantum theory. Namely, we have to do the following

\begin{itemize}
\item to show that the quantum theory is conformal and to 
determine its central
charge (part (c))

\item to compute the OPE of $\d\phi(x)\d\phi(y)$ and to determine whether
$\d\phi$ is a primary field (parts (a) and (e))

\item to show that $\exp(\beta\phi(x))$ is a primary field
and to compute its conformal weight
\end{itemize}
%-----------------------------------------------------------------------
\sssec{}{Computation of the partition function and the central charge}
Let us first compute the partition function. From it will be able to read
off the central charge of the theory. 
We must compute the following functional integral:
\eq{statsumma}
Z^Q_g=\int D\phi 
e^{-S[\phi,g]}=\int D\phi 
\exp(-{1\over 4\pi}\int\limits_{\Sigma} 
{1\over 2}\d \phi{\overline \d}
\phi+Q\phi R_g)
\end{equation}
where the integral is taken over all functions $\phi:\Sigma\to \RR$
such that 
$$
\int\limits_{\Sig}\phi d\mu_g=0
$$
The action is a non-homogeneous quadratic function in $\phi$ and therefore
we may compute the functional integral by completing the square and we
obtain 
\eq{}
Z_g^Q=Z_g^0\exp({Q^2\over 8\pi}\int\limits_{\Sig}\Delta^{-1}_g(R_g)\cdot R_g)
\end{equation}

Let us now compute how the partition function changes when we replace $g$
by a new metric $g'$ of the form $e^{\sigma}g$, where $\sigma$ is a real-
valued function on $\Sigma$. We know that
\eq{}
Z_{g'}^0=Z_g^0\exp( {1\over 96\pi}(\int\limits_{\Sigma}(\sigma,\Del\sig)+
4\sig R_g))
\end{equation}
(cf. Gawedzki's second lecture).

Therefore, since $R_{g'}=R_g+{\Del(\sig)\over 2}$, 
\eq{}
\begin{aligned}
Z_{g'}^Q=Z_g^Q e^{ {1\over 96\pi}\int\limits_{\Sigma}((\sigma,\Del\sig)+
4\sig R_g)}
e^{{Q^2\over 8\pi}(\int\limits_{\Sig}\Delta^{-1}_g(R_{g'})\cdot R_{g'}-
\Delta^{-1}_g(R_g)\cdot R_g}=\\
Z_g^Q\exp( {1+3Q^2\over 96\pi}\int\limits_{\Sigma}((\sigma,\Del\sig)+
4\sig R_g))
\end{aligned}
\end{equation}

Therefore the central charge of the theory is equal to $1+3Q^2$ (note that 
we haven't defined our theory as a quantum conformal field theory yet, which 
is done below).
%-----------------------------------------------------------------------------
\sssec{}{Computation of the correlation functions}
We will now define the basic fields of our CFT and compute some of their
correlation functions.

For $Q=0$ we had a primary field $\d \phi$ which took values in
$\Ome^{1,0}(\Sig)$. 

For general $Q$ we will show that $\d\phi$ is a section of
the $\Ome^{1,0}(\Sig)$-torsor $\Ome^{1,0}_Q(\Sig)$ 
of affine $2\pi Q$-connections. This torsor
can be explicitly described as follows. 
For any metric $g$, compatible with the complex structure on $\Sigma$,
$\Ome^{1,0}_Q(\Sig)$ acquires a canonical trivialization (this corresponds to
the fact, that given a metric, one can trivialize the torsor of affine
connections by means of
the $(1,0)$-part of the corresponding Levi-Civita connection) and when the
metric $g$ is replaced by $g'=e^{\sig}g$, the trivialization is changed
by $2\pi Q\d\sig$. 
%--------------------------------------------------------------------------
\claim $\d\phi$ is 
a quantum field which takes values in $\Ome^{1,0}_Q(\Sig)$ 
\eclaim
%---------------------------------------------------------------------------

\prf We will sketch the computation of a one-point function of $\d\phi$.
So, let $g$ be a metric on $\Sigma$, $x\in \Sig$ -- a point of $\Sig$
and $\xi$ -- a holomorphic tangent vector to $\Sigma$ at $x$.
Then the expression  $\la \d\phi(x)\xi\ra^Q_g$ makes sense as a complex
number. 
%--------------------------------------------------------------------------
\eq{one-point}
\la \d\phi(x)\xi\ra^Q_g={1\over Z_g^Q}\int D\phi(\d\phi(x),\xi)\cdot\exp
({1\over 4\pi}\int\limits_{\Sigma} 
{1\over 2}\d \phi{\overline \d}
\phi+Q R_g)
\end{equation}
%-----------------------------------------------------------------------------
Arguing as above (by completing the square) we obtain that 
\refe{one-point} is equal to
\eq{}
\Lie_{\xi}(\Delta^{-1}_g(R_g))
\end{equation}

Therefore, when $g$ is replaced by $g'=e^{\sig}g$,  
$\la \d\phi(x)\xi\ra^Q_{g'}=\la \d\phi(x)\xi\ra^Q_g+2\pi\d\sig$, as claimed.
\epr

The above claim implies, in particular, that the field $\d\phi(x)$ is
not primary, since primary fields were defined as fields whose correlation
functions are sections of $\Ome^{p,q}$ (for some $p, q$) and
$\Ome^{1,0}_Q$ is definitely not of this form for $Q\neq 0$.
This solves (e).
%----------------------------------------------------------------------
\sssec{}{Solution of (a)}In order to establish the OPE of $\d\phi(x)$ and
$\d\phi(y)$ we will make the computation only for the 2-point function.

\claim
$\la \d\phi(x)\d\phi(y)\ra^Q_g-\la \d\phi(x)\d\phi(y)\ra^0$ is regular
when $x\to y $.
\eclaim

\prf
In fact, it is easy to compute in the same way as above, that
\eq{}
\la \d\phi(x)\d\phi(y)\ra^Q_g=\la \d\phi(x)\d\phi(y)\ra^0_g+
\la\d\phi(x)\xi\ra_g^Q\otimes \la\d\phi(x)\xi\ra_g^Q
\end{equation}

which proves the claim
\epr
Therefore, the OPE of $\d\phi(x)$ and
$\d\phi(y)$ is the same as for $Q=0$.
%----------------------------------------------------------------------------
\sssec{}{Solution of (d)}
We claim that classical field $\exp(\beta\phi(x))$ can be quantized to
a primary field of conformal weight 
$(Q\beta+{\beta^2\over 2},Q\beta+{\beta^2\over 2})$.
We will calculate the two-point function
\eq{}
\la e^{\beta\phi(x)}e^{-\beta\phi(y)}\ra^Q_g=
\la e^{\beta\phi(x)}e^{-\beta\phi(y)}\ra^0_g
\cdot \exp(Q\int\limits_{\Sig}(\Delta^{-1}_gR_g)\cdot (\delta(x)-\delta(y)))
\end{equation}

When we make a Weyl transformation of the metric $g\to g'=e^{\sig}g$, we
see that 
$$
\begin{aligned}
\la &e^{\beta\phi(x)}e^{-\beta\phi(y)}\ra^Q_{g'}= \\
&\la e^{\beta\phi(x))}e^{-\beta\phi(y))}\ra^Q_g
\cdot \exp({\beta^2\over 2}(\sig(x)-\sig(y)))\cdot
\exp(Q\beta(\sig(x)-\sig(y))
\end{aligned}
$$

This shows that $\exp(\beta \phi(x))$ is a field whose correlators
take values in $(\Ome^{1,1})^{({\beta^2\over 2}+Q\beta)}$.






\end{document}






















