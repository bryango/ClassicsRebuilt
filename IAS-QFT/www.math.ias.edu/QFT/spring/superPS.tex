%Date: Thu, 27 Feb 1997 10:33:36 -0500
%From: Edward Witten <witten@sns.ias.edu>

%have fun....



\input harvmac

Superhomework, Term 2

The purpose of the following is to describe facts that we will
need in order to study supersymmetric quantum field theories.

In doing so, we will mainly be interested in flat models (for the
supermanifolds on which the quantum field theory is formulated), and
in any event our supermanifolds will always be ``split'' (because
we will not include supergravity) which generally means that
one won't lose much that is essential by considering the flat model.

(Even without doing quantum field theory, one of the important things
one learns from these exercises is that the interesting relativistic
field theories 
 have supersymmetric generalizations, with some restriction
on the dimension of spacetime.)
 
We make one small change in notation.  A manifold with $p$ even
and $q$ odd coordinates will be said to be of dimension $(p|q)$.
This frees up the comma for other purposes; for instance, if the
reduced space has a metric of signature $(n,m)$ (with $n+m=p$) we 
might call this a manifold of dimension $(n,m|q)$.

I haven't tried to explain everything below. You will have to consult
Wess and Bagger, or other references, especially for the four-dimensional
gauge theory.  However, if you understand these problems, you are probably
familiar with the classical facts that are most important background
for beginning to discuss supersymmetric quantum theories.

\def\R{{\bf R}}
(1) The first part is review from last fall.  We consider a $(3|2)$
supermanifold $\R^{(3|2)}$ 
(or one can dimensionally reduce to $\R^{(n|2)}$ with $n<3$
by imposing invariances under some translations).  We call 
the fermionic coordinates $\theta^A$, $A=1,2$ and the bosonic
coordinates $y^{AB}$ (symmetric in $A$ and $B$).  
The supersymmetry
generators are
\eqn\joggo{Q_A={\partial\over\partial \theta^A}+i\theta^B
{\partial\over \partial y^{AB}}}
and commute with 
\eqn\joggo{D_A={\partial\over\partial \theta^A}-i\theta^B
{\partial\over \partial y^{AB}}}
which is used in writing Lagrangians.

The symbol $\epsilon^{AB}$ ($=-\epsilon^{BA}$) will denote a 
translation-invariant volume form
on what Bernstein called the odd distribution -- the
odd subspace of the tangent bundle generated by the $D$'s.

The super-Poincar\'e group is generated by the $Q_A$, their anticommutators
$\partial/\partial y^{AB}$, and an $SU(2)\cong SO(3)$ for which the
$\theta^A$ and the $y^{AB}$ are in the two- and three- dimensional
representations, respectively.

A superfield is just a function $\Phi$ on $\R^{(3|2)}$.  Let $X$
be a fixed Riemannian manifold with metric $g$.  Let $\Phi$ be a map
from $\R^{(3|2)}$ to $X$.  Picking local coordinates on $X$, we describe
$\Phi$ via functions $\Phi^I$ and write simply $g_{IJ}$ for
the pullback via $\Phi$.

(a) Then as we hopefully
remember from last fall
we can make the supersymmetric Lagrangian
\eqn\aba{L_0=\int d^3y\,d^2\theta \,\epsilon^{AB}g_{IJ}D_A\Phi^ID_B\Phi^J.}

Refresh your recollections; show that this is equivalent to an 
ordinary sigma model with target space $X$ plus fermions and describe
the fermions.

Determine the conserved supercurrent.

(b) Now let $h$ be a smooth function on $X$ and let $L=L_0+L_1$ where
\eqn\baba{L_1=\int d^3y\,d^2\theta\,\Phi^*(h).}

Identify the reduced model.  In particular, what is the potential
energy and what is the mass term for the fermions?  

Suppose that $X$ is compact and that $h$ has only isolated critical
points.  Can you
state a lower bound for the number of classical minima of the energy?
(After dimensional reduction to $\R^{(1|2)}$ this observation was of
course the starting point for my article on Morse theory in 1982.  Even if $h$
does not have isolated critical points, there is a sense in which the 
same lower bound on the number of vacua holds.)

(c) 
If $X$ is noncompact it may be the case that there is no critical
point of $h$ and moreover that the set with $|dh|<c$ is compact for
any constant $c$.  In this case the energy is bounded strictly above
zero.  This is a theory in which supersymmetry is spontaneously
broken.  As you can read in Wess and Bagger, unbroken supersymmetry
means that the energy of the vacuum is zero.

Consider an arbitrary local minimum of the classical energy function
at which the energy is not zero.  Show that there is a massless
fermion -- called the Goldstone fermion.  What you have just checked
is a special case of the analog of Goldstone's theorem for supersymmetry.

(2) Now we want to do gauge theory, but again in $\R^{(3|2)}$ to begin
with.  

First we will do pure gauge theory.  $G$ is a compact gauge group and
$A$ is a connection on a $G$ bundle $P$ 
over $\R^{(3|2)}$.  The curvature $F$ 
is defined in the usual way.  One can freely write down all kinds of
supersymmetric actions by integrating over $\R^{(3|2)}$ an appropriate
polynomial in $F$ and its derivatives.  However, such constructions
(like the generic classical field theory) are ``wrong,'' that is the
bosonic part does not have positive energy, for instance.  We will
verify the analogous statement later in a case that may be a bit
easier.

\def\D{{\cal D}}

To get a good theory, we consider connections such that $F(D_A,D_B)=0$.
(Bernstein expressed this somewhat differently, I think.)
An important fact is that this condition on the connection can be
solved locally, without imposing any differential equations (but reducing
the number of independent functions in the description of the connection).

Let $\D_A$ be the gauge-covariant extension of $D_A$ and
$\D_{AB}$ the gauge-covariant exension of $\partial/\partial y^{AB}$.

The basic curvature invariant is then
\eqn\udago{\lambda_A=\epsilon^{BC}\left[\D_A,\D_{BC}\right].}
Show that the whole curvature can be expressed in terms of $\lambda$
and its covariant derivatives.

The basic gauge theory Lagrangian is then
\eqn\gago{L_2=\int d^3y\,d^2\theta\,\, \epsilon^{AB}
\Tr\,\lambda_A\lambda_B.}
Here $\Tr$ should be understood simply as a gauge-invariant quadratic
form on the Lie algebra.

(a) Show that this Lagrangian has for its bosonic reduction
ordinary Yang-Mills theory.  Identify what the fermions are. (Hint:
you might use the relation between $d\theta^A$ and $\D_A$.) 

You should see that this model is equivalent to one of the simplest
models in three dimensions of Yang-Mills plus fermions; we might
have studied it as a simple example of such without noticing it
was supersymmetric.

What is the conserved supercurrent?

(b) Suppose that $G$ acts on the manifold $X$ considered in (1) above
and reinterpret $\Phi$ as a section of $X\times_GP$.  $L_0$ and
$L_1$ are defined
by the same formulas as in problem (1) above.

Show that, in terms of ordinary fields, $L_0$ and $L_1$ are now
the obvious gauge-covariant extensions of what you had before.


(3) Now we move on to four bose dimensions.
We cannot work in $\R^{(4|2)}$ because although $SO(3,1)$ has
two-dimensional spinor representations, they are not real; we would
not be able to get real Lagrangians in Minkowski space.

So we introduce spinors of both kinds and work on $\R^{(4|4)}$.
The following will be written in Euclidean signature without imposing
any reality condition.  We write $\theta^A,\,A=1,2$ and 
$\bar\theta^{\dot A},\,\dot A=1,2$ for
the two kinds of spinors\foot{$\bar \theta^A$ and $\theta^{\dot A}$
will never appear (except as a result of a possible misprint - in
that case remove bars from undotted $\theta$'s and add them to dotted
ones).  In Lorentz signature if one imposes
a reality condition, then $\bar\theta^{\dot A}$ is
the complex conjugate of $\theta^A$.  In Euclidean signature there
is no reality condition and no relation between these two. Similar
statements hold for other objects introduced below such as the $W$'s.}
 and $x^{A\dot A}$ for the bose coordinates.
Under the action of $SO(4)\cong SU(2)\times SU(2)$, the undotted
indices $A,B,C$ transform in the two-dimensional representation of the
first $SU(2)$ and the dotted indices $\dot A, \dot B,\dot C$ transform
in the two-dimensional representation of the third.

The supersymmetry generators are
\eqn\upu{\eqalign{Q_A & ={\partial\over\partial\theta^A} +i
\bar\theta^{\dot A}{\partial\over\partial x^{A\dot A}} \cr
\bar Q_{\dot A} & ={\partial\over\partial\bar\theta^{\dot A}} +i
\theta^{ A}{\partial\over\partial x^{A\dot A}} \cr}}
and commute with
\eqn\pupu{\eqalign{D_A & ={\partial\over\partial\theta^A} -i
\bar\theta^{\dot A}{\partial\over\partial x^{A\dot A}} \cr
\bar D_{\dot A} & ={\partial\over\partial\bar\theta^{\dot A}} -i
\theta^{ A}{\partial\over\partial x^{A\dot A}} \cr}}
which are used in writing Lagrangians.

Now, the most obvious kind of field, imitating what we did in (1)
above, would appear to be a field $\Phi$ which is simply a function
on $\R^{(4|4)}$.  This is called an ordinary superfield.

However, {\it it is impossible to write a sensible
Lagrangian for such a field}, where again sensible means that the bosonic
part of the energy is positive.  


(a) Write some supersymmetric Lagrangians for an ordinary superfield
and show that (if you specialize to Lorentz signature)
the bosonic part of the energy is not positive.


\def\bar{\overline}
Instead, we use a {\it chiral superfield} $\Phi$, which is a function
on $\R^{(4|4)}$ such that 
\eqn\juffo{\bar D_{\dot A}\Phi=0.}
Likewise $\bar\Phi$ is called an antichiral superfield if
\eqn\toffo{D_A\bar\Phi=0.}

If $X$ is a complex manifold, a chiral map from $\R^{(4|4)}$ to $X$
is a map such that local holomorphic functions on $X$ pull back to
chiral superfields, and antiholomorphic functions pull back to
antichiral superfields.

\def\L{{\cal L}}
\def\RR{{\bf R}^{(4|4)}}
If $K$ is a real analytic function on $X$, and $\Phi:\R^{(4|4)}
\to X$ is a chiral map, then one defines a Lagrangian
\eqn\iago{\L_0=\int d^4x\,D^4\theta \Phi^*(K).}

(I will use $\L$ for four-dimensional Lagrangians, and $L$ for somewhat
similar three-dimensional ones.)

(b) Show that this is invariant under $K\to K+f+\bar f$ where
$f$ is a holomorphic function on $X$.  

(c) Show that $\L_0$ is equivalent to an ordinary sigma model
Lagrangian on $\R^4$, with target $X$ -- the metric on $X$ being
a Kahler metric with $K$ as the Kahler potential -- plus fermion
terms.  Describe the fermion terms.  

Notice that $\L_0$, after doing all (or as you should be able to see,
just half) of the $\theta$ integrals, makes sense globally for maps
to any Kahler manifold $X$ -- even though the description here
required the choice, and in particular the existence, of a Kahler
potential, which of course exists locally but may not exist globally.  
Problem: find a good and economical global construction of $\L_0$.
(To some extent this can be done using gauge theory as developed below.)

(d) Show that this theory has a $U(1)$ symmetry (classically - it may
be anomalous) coming from a symmetry of $\R^{(4|4)}$ that acts
only on the odd coordinates and acts by an outer automorphism of the
super-Poincar\'e algebra.

A symmetry that acts by outer automorphisms of super-Poincar\'e and
in superspace acts only on the odd coordinates is called
an R symmetry, a term one will often
encounter.

(e) Show that if one imposes invariance under one translation in
$\R^4$ (by a vector that is not a null vector -- if you are
doing Lorentz signature it should be a spatial translation, here
and in other cases below), 
$\L_0$ reduces to $L_0$ as considered in (1) above -- for the
special case that the target is actually Kahler.

It is therefore true, given also results from problem sets 1-3 in
term one, that if one imposes invariance under three translations
in $\R^4$, to reduce to $\R^{(1|4)}$, one gets the Lagrangian
whose quantization gives Hodge theory of differential forms on $X$.

Deduce that there is an action of $SU(2)\times U(1)$  on differential forms
on a Kahler manifold, commuting with 
the Hamiltonian, where the $U(1)$ was seen in (d) and the $SU(2)$
comes from spatial rotations of $\R^4$.

(f) The following is something that we will need to know about
if we really want to do Donaldson theory.

Suppose that $X$ is hyper-Kahler.  Then the Lagrangian $\L_0$ has
twice as much supersymmetry as for a Kahler $X$.  



It can be shown comparatively easily as follows:

After writing down ``in components'' the sigma model action,
show that the $U(1)$ found in (d) actually extends to $SU(2)$
in case $X$ is hyper-Kahler.  (The $SU(2)$ only acts on fermions.
This should be obvious - there is no universal action of $SU(2)$ on
a hyper-Kahler manifold.)  

This $SU(2)$ does {\it not} act by outer automorphisms on the
super-Poincar\'e algebra of $\R^{(4|4)}$.   Can you describe what
algebra is generated by the $SU(2)$ together with the automorphisms
of $\RR$?  Hint: it can be naturally realized geometrically as
a supergroup of automorphisms of an $\R^{(4|8)}$.  It is called
the $N=2$ supersymmetry algebra in four dimensions, another term
that is likely to be encountered frequently.

Unfortunately, there  is no reasonable way known to make 
a superspace construction starting from $\R^{(4|8)}$.  That is
why I asked you to construct this $N=2$ theory by looking at it as
an $N=1$ theory that just happens to have extra supersymmetry.
The above computation is thus analogous to the way that mathematicians
usually find the extra symmetries of Hodge theory on a Kahler manifold
``by hand,'' not by starting with a geometrical construction
of maps from $\R^{(1|4)}$.

(g) Deduce from the above that there is an action of $SU(2)\times SU(2)$
on the differential forms on a hyper-Kahler manifold, commuting
with the Hamiltonian.  Actually, this extends to an action of $Sp(2)=
Spin(5)$.
That is because there is a supersymmetric model in six dimensions
in which the data consist of maps of $\R^6$ to a hyper-Kahler manifold
plus fermions.  This theory is one of the limiting cases permitted
by Nahm's theorem (that is, six is the maximum dimension for sigma
models according to Nahm's theorem).  It does not have a known superspace
construction.

Write down the Lagrangian for this model by using the fact that
it must reduce in four dimensions to a Lagrangian found above.
(For extra credit do the following not very hard exercise: Verify
supersymmetry of this Lagrangian.) 

(4) Going back to the case that $X$ is only Kahler,
let $W$ be a holomorphic function on $X$ -- called  the
superpotential.

Let
\eqn\ummu{\L_1=\int d^4x \,d^2\theta\Phi^*(W) +
\int d^4x\,d^2\bar\theta \Phi^*(\bar W).}

(a) Show that this is supersymmetric.

(b) Reduce to an ordinary Lagrangian and compute the potential
energy function on $X$ and fermion mass terms.

(c) Upon imposing invariance under translation of one coordinate, show
that $\L_1$ becomes equivalent to $L_1$ as constructed in 1(b) above
with $h={\rm Re}(W)$.

(d) Under what conditions on $X$ and $W$
does the R symmetry -- the outer automorphism
of super-Poincar\'e that comes from a symmetry of $\RR$, as found
above, lift to a symmetry of this theory? (That is, a classical
symmetry - quantum mechanically it may be anomalous.)

(5) Now we want to do gauge theory.  Once again,
we introduce a compact connected Lie group $G$ and $G$ bundle
$P$ over $\RR$.  $A$ is a connection on $P$ and $F$ is the
corresponding curvature.  

One can write all kinds of supersymmetric Lagrangians for
such a general connection, but not of much interest -- the usual
classical pathology is non-positive energy for the bosons.

For the good construction, one restricts $A$ by requiring
the vanishing of some components of the curvature,
in fact $F(D_A,D_B)=F(D_A,\bar D_{\dot A})=F(\bar D_{\dot A},\bar 
D_{\dot B})$. (Or more succinctly, $F$ vanishes when evaluated
on any two fermionic $D$'s)

(a) Show that these conditions can be solved locally, without
differential equations.  

(b) With these conditions, the basic curvature invariants are
\eqn\orpo{W_A=\epsilon^{\dot B\dot C}[\bar\D_{\dot B},\D_{A\dot C}],}
and
\eqn\orpo{\bar W_{\dot A}=
\epsilon^{ B C}[\D_{ B},\D_{ C\dot A}].}
where as before $\D$ is the gauge-covariant extension of $D$.  

Show that $W_A$ is a chiral superfield (with values in one of the
two spin bundles\foot{In this flat model the spin bundle has
a flat connection. A chiral superfield with values in a flat
bundle is defined in the obvious way.} and $\bar W_{\dot A}$ is
antichiral, and that all components of the curvature can be expressed
in terms of the $W$'s and their covariant derivatives
by using the Bianchi identity.

(c) The basic Lagrangian is 
\eqn\urmu{\L_2=\tau\int d^4x \,d^2\theta \epsilon^{AB}
\Tr W_AW_B+\bar \tau\int d^4x\,d^2\bar\theta
\epsilon^{\dot A\dot B}\Tr \bar W_{\dot A}\bar W_{\dot B}.}
$\tau$ is a complex conjugate and (to get a real Lagrangian
in Lorentz signature) $\bar\tau$ is its complex conjugate.

Reduce this to an ordinary Lagrangian.  It is equivalent to a simple
special case of four dimensional gauge theory coupled to fermions.
Which special case?

In order to do this, you may have made contact with the formalism
in Wess and Bagger, chapter 6, where a field that they call $V$ is
introduced.  It will probably be necessary to make contact with that
in order to solve an important part of a later exercise.

(d) Show that this theory at the classical level possesses the R symmetry
that was discussed in the previous exercise.

\def\C{{\bf C}}
Now the one quantum question in these exercises.  Using your
knowledge of determinant line bundles, do you believe that
this R symmetry is valid quantum mechanically?

If you claim that
there is an anomaly breaking the $U(1)$ R symmetry to a finite
subgroup of $U(1)$, can you identify the subgroup in question?

(e) If you impose invariance under one of the spatial translations,
this theory does {\it not} reduce to $L_2$ as formulated in the three
dimensional case in problem 2 above.  However, it does reduce
to one of the problems of gauge theory plus matter formulated
in problem 2 (that is, a Lagrangian of the form $L_0+L_2$).
Can you identify which special case?



(6) Now we want to consider a Kahler manifold $X$ with a $G$ action.
To begin with, we suppose that $X={\bf C}^n$ for some $n$ with
a linear action of $G$ on $X$, leaving fixed the origin.  We pick
a $G$-invariant Kahler potential $K$ on $X$.  

We consider a theory in which the data are a connection $A$ on a $G$
bundle $P$
(obeying the curvature constraints discussed in the last problem)
and a chiral  section $\Phi$ of $X\times_GP$.

(a) Explain what it means for $\Phi$ to be chiral.

(b)  Convince yourself that the Lagrangian
\eqn\ombo{\L_0=\int d^4x d^4\theta \,\Phi^*(K) }
makes sense in this situation and is, in fact, equivalent to
the obvious gauge covariant extension of $\L_0$ as defined
above.

(c) Consider the Lagrangian $\L_0+\L_2$ (with $\L_2$ as defined
above).  Determine the moduli space of classical states of minimum
energy.  You should meet some notions of ``symplectic quotient''
(related to geometric invariant theory) and you should find
that the moment map you get vanishes at the origin in ${\bf C}^n$.

In fact, it is possible (in case $G$ has $U(1)$ factors -- if
$G$ is semi-simple the following generalization is vacuous)
to add an additional supersymmetric interaction that adds
a constant to the moment map.  

In the notation of Wess and Bagger, for $G=U(1)$, the additional
interaction is
\eqn\ambo{\L_3=r\int d^4x\,d^4\theta \,\, V.}
Here $r$ is a real constant.  ($\L_3$ is commonly called
the Fayet-Iliopoulos term or $D$ term and is introduced by Wess and Bagger
starting on p. 52.)
Show that this is supersymmetric (a trivial consequence of what
in my edition of Wess and Bagger is eqn. 6.4).  I do not know
in four dimensions a more intrinsic way to write \ambo\ -- my
apologies for using a formula that depends on Wess-Zumino gauge
(which leads to the definition of $V$).  After reduction to two dimensions
there is a nice formula that we will see later.

Show that replacing
$\L_0+\L_2$ by $\L_0+\L_2+\L_3$  has
the effect of shifting the moment map by a constant $r$.  
The space of classical zeroes of the energy (or supersymmetric vacua)
 is the symplectic quotient with the
new moment map.  Of course, if $G$ has several $U(1)$ factors,
we add a term like \ambo\ for each such factor, so we can get a general
moment map.  

(e) The global generalization of the above is as follows.
Let $X$ be any Kahler manifold with $G$ action (and Kahler
metric invariant under $G$) and such that there is a moment
map for the $G$ action; pick a moment map.  For each such choice,
an appropriate global version of $\L_0+\L_3$ exists.  
(There is no canonical $\L_0$ just as in ordinary geometry there is no
canonical choice of a moment map. There is the canonical family
of $\L_0+\L_3$'s with the variable constant $r$ in the definition
of $\L_3$.) The
verification of this is in a paper by Bagger and Witten 
(which will be available from Paula Bozzay or Val Nowak).   
I do not know a natural global construction in superspace.
(The above is not even a universal local construction since in
identifying $X$ at least locally with $\C^n$ with linear action of $G$
we assumed that $G$ has a fixed point.  The supersymmetric theory
exists for any $G$ action on $X$, even if there is no fixed point,
as long as there is a moment map.) 

(f) Now, let $W$ be a $G$-invariant holomorphic function on $X$
and consider the superpotential term $\L_1$.  Convince yourself
that $\L_1$  makes sense in this gauge theory setting and
describe the moduli space of classical vacua in the theory
with Lagrangian $\L_0+\L_1+\L_2+\L_3$ (the inclusion of the $\L_3$
term with a constant $r$ in $\L_3$ 
just means, of course, that you are to work with a general
moment map).

(g) (We will need this when we get to Donaldson theory.)
Consider the case that $G$ is $n$-dimensional and $X={\bf C}^n$
with $G$ acting on $X$ via the adjoint representation.
Consider the Lagrangian $\L_0+\L_2$.  By writing it out in terms
of ordinary fields, show that the $U(1)$ R symmetry extends
to to an $SU(2)$ symmetry which does {\it not} act by outer
automorphisms of super-Poincar\'e, so that super-Poincar\'e is
extended to a super-algebra with twice as many odd generators.
(In fact, in this case, unlike the sigma model case, there
is a rather nice realization via gauge theory on $\R^{(4|8)}$.)

More generally, if $X={\bf C}^n\times Y$, where (i) $G$ acts on
${\bf C}^n$ via the adjoint representation; (ii) $Y$ is hyper-Kahler;
(iii) $G$ acts on $Y$ preserving the metric and with a hyper-Kahler
moment map, then (for every choice of all this data including
the moment map) there is a gauge-invariant sigma model with
the data being the connection $A$ and a section of $X\times_GP$.
One again needs this generalization for Donaldson theory; however,
I won't explain it here.  (The moduli space of classical supersymmetric vacua
is the hyper-Kahler quotient of $Y$ by $G$, which therefore
must naturally be a hyper-Kahler manifold;  that is how
the hyper-Kahler quotient was discovered.)

(7) The supersymmetric theories that I will lecture about in
two dimensions are actually all obtained by dimensional
reduction from $\RR$, that is, by imposing invariance under two
translations to get a theory on $\R^{(2|4)}$.  However,
some properties that appear upon reduction to $\R^{(2|4)}$ deserve
special discussion.

First of all, the rotation of the two coordinates that are removed
in going from four to two dimensions acts as an outer automorphism
of the two-dimensional super-Poincar\'e.  Since there already
was such an outer automorphism in four dimensions, there are two
of them in two dimensions, making up a group $U(1)\times U(1)$ of
R symmetries.  (It is not the case that all the theories we
can write admit the full $U(1)\times U(1)$ even classically, since
in four dimensions we had the possibility of breaking the R symmetry
classically with the choice of $W$.)

The bosonic coordinates in two dimensions are, say, $x^+$ and $x^-$,
of weights $1$ and $-1$ under Lorentz boosts.  The fermionic coordinates
consist of two
of weight $1/2$ and two
of weight $-1/2$. 


(a) Show -- just using the 
geometrical properties of $\RR$ -- that of the R symmetry group
 $U(1)\times U(1)$  in two dimensions, one $U(1)$
acts only on the weight one half odd coordinates and the other
$U(1)$ acts only on the weight minus one half odd coordinates.

I label the weight one half odd coordinates as $\theta^+$, $\bar\theta^+$
where one of the two $U(1)$'s in the R symmetry group
 acts by $\theta^+\to e^{i\alpha}\theta^+$,
$\bar\theta^+\to e^{-i\alpha}\bar\theta^+$. Likewise I label the
weight minus one half odd coordinates as $\theta^-$, $\bar\theta^-$
with analogous action of the second $U(1)$.
 (As you know well, in Lorentz signature the barred objects
are complex conjugates of the unbarred ones, but in Euclidean signature
as always there is no reality condition.)

For the supersymmetries I write
\eqn\uttu{\eqalign{Q_+ & ={\partial\over\partial\theta^+}+i
\bar\theta^+{\partial\over\partial x^+}\cr
\bar Q_+ & = {\partial\over\partial\bar\theta^+}+i\theta^+{\partial
\over\partial x^+}.\cr}}
Thus $Q_+^2=\bar Q_+^2=0$ and $\{Q_+,\bar Q_+\}$ is a multiple
of $\partial/\partial x^+$.  $Q_-$ and $\bar Q_-$ are defined in the
same way with $+\to -$.  

The supersymmetries commute with vector fields $D_+$, $\bar D_+$,
$D_-$, $\bar D_-$ that are defined by the same formulas with $i\to -i$.
(Or if you've suppressed the $i$'s, then reverse the sign of
$\partial/\partial x^{\pm}$.)  

Mirror symmetry is the automorphism of $\R^{(2|4)}$ that
exchanges $\theta^-$ and $\bar \theta^-$ while doing nothing
to other odd or even coordinates.  Thus it reverses one of the
R symmetries and commutes with the other.  More precisely, two
field theories defined in $\R^{(2|4)}$ are said to be mirrors
if the stated automorphism of $\R^{(2|4)}$ lifts to an isomorphism
between the given theories.  

(b) A chiral superfield is a function $\Phi$ on superspace
such that $\bar D_+\Phi=\bar D_-\Phi=0$. An antichiral superfield
$\bar\Phi$ obeys $D_\pm \bar\Phi=0$.

A twisted chiral superfield is a function $\Sigma$ on superspace
such that $\bar D_+\Sigma=D_-\Sigma=0$.
A twisted antichiral superfield $\bar \Sigma$ obeys $D_+\bar\Sigma
=\bar D_-\bar\Sigma=0$.  

The sigma model of a chiral superfield is
our friend
\eqn\impo{\L_0'=\int d^2x \,d^4\theta \, K(\Phi,\bar \Phi)}
and the sigma model of a twisted chiral superfield is defined
exactly analogously.  One can also straightforwardly define
sigma models containing both chiral and twisted chiral superfields,
but we will not study them.

Likewise, if $W$ is a holomorphic function of a chiral superfield,
we have
\eqn\ujimpo{\L_1'=\int d^2x d^2\theta \Phi^*(W) +\int d^2x d^2\bar \theta
\Phi^*(\bar W),}
as in four dimensions.

(c) Some novelties occur if one considers gauge theory.
We want to consider gauge theory with chiral superfields.
For this, we consider a connection $A$ on $G$ bundle $P$ such
that the curvature $F$ obeys $F(u,v)=0$ where $u,v$ are either
both linear combinations of $D_+$ and $D_-$
or both linear combinations of $\bar D_+ $ and $\bar D_-$.
(If we wanted to use twisted chiral superfields, then we would
pair $D_+$ and $\bar D_-$ in the constraint on the curvature.)

For a connection of this kind, the basic curvature invariants are
$\Sigma=F(\bar D_+,D_-)$ (or equivalently $\Sigma=\{\bar \D_+,\D_-\}$
where the $\D$'s are gauge-covariant extensions of the $D$'s)
and $\bar \Sigma=F(D_+,\bar D_-)$.

Show that all components of the curvature can be expressed in terms
of $\Sigma$, $\bar \Sigma$, and their covariant derivatives.  

Show that $\Sigma$ is a twisted chiral superfield and that $\bar\Sigma$
is a twisted antichiral superfield.

(d) The basic Lagrangian for the gauge multiplet (that is, the 
``supermultiplet'' of fields that includes the connection) is
\eqn\pollo{\L_2'={1\over 4e^2}\int d^2x d\theta^+d\bar\theta^-\Tr\Sigma^2
+{1\over 4e^2}\int d^2x d\theta^-d\bar \theta^+\Tr\bar\Sigma^2.}
Here $\Tr $ is to be understood as any invariant
quadratic form on the Lie
algebra.


Either by reducing this to an ordinary Lagrangian or in any other
way of your choosing, show that this is  equivalent to the reduction
to two dimensions of our four dimensional Lagrangian $\L_2$.

In particular, it describes gauge fields, fermions, {\it and scalars},
in contrast to the three and four-dimensional gauge actions
$L_2$ and $\L_2$ constructed earlier, which describe gauge fields
and fermions only.

(e) This will be very important in applications.  If $G=U(1)$,
 one has the additional possible term
\eqn\opollo{\L_3'=r\int d^2x d\theta^+d\bar\theta^- \Sigma
+r\int d^2x d\theta^-d\bar \theta^+\bar\Sigma}
with $r$ a constant.
More generally, for any $G$ one has such a term for any $U(1)$ factor
in $G$.    In other words, $\Sigma$ can be replaced by any gauge-invariant
linear function of the $\Sigma$"s.  (One can more generally use
any gauge invariant holomorphic function of the $\Sigma$"s, obviously.)

$\L_3'$ arises by dimensional reduction from the four-dimensional
interaction $\L_3$, but as you can see, in two dimensions a much
nicer and more intrinsic description is possible.  

This also
enables one to give a nicer global description of the sigma
model of maps to a Kahler manifold in two dimensions, at least
if the Kahler class is integral.  

The supersymmetric Lagrangians we will study in two dimensions
will be special cases of $\L_0'+\L_1'+\L_2'+\L_3'$.


\end















