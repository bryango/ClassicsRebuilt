%Date: Thu, 17 Oct 1996 16:18:20 EDT
%From: Edward Witten <witten@sns.ias.edu>


\input harvmac
\def\R{{\bf R}}\def\Z{{\bf Z}}
{\it ``Problem Set Three''}

(1) Here we will study the theory related to ``differential
forms on the loop space of a manifold.''

We let $W^{2,2}$ be a supermanifold of the indicated dimension
with one-dimensional odd distributions $\Lambda_+$, $\Lambda_-$
generated by vector fields $D_+$, $D_-$ such that $D_+$, $D_-$,
$D_+^2$, $D_-^2$ generate the tangent bundle of $W^{2,2}$.
The flat model is $\R^{2,2}$ 
with even coordinates $u,v$, and odd coordinates $\theta_-$, $\theta_+$,
with
\eqn\igg{\eqalign{ D_+ & = {\partial\over\partial\theta_+}-i\theta_+{\partial
                 \over \partial v} \cr
                   D_- & = {\partial\over\partial \theta_-}-i\theta_-
                         {\partial\over\partial u}. \cr}}

$\R^{2,2}$ has an infinite dimensional supergroup of automorphisms,
roughly $G\times G$, where $G$ is the symmetry group of the supermanifold
$\R^{1,1}$ (with even and odd coordinates $t,\theta$ and
distribution generated by $\partial_\theta + i\theta\partial_t$). 
For instance, generators are
\eqn\bigg{\eqalign{Q_+ & = {\partial\over\partial\theta_+}+i\theta_+{\partial
\over\partial v} \cr
Q_- & = {\partial\over\partial\theta_-}+i\theta_-{\partial\over\partial u}.\cr
   }}

Let $M$ be a fixed Riemannian manifold with metric $g$.  Let $\Phi:W^{2,2}
\to M$ be a map, and let
\eqn\really{L=\int_{W^{2,2}} g_{IJ}D_+\Phi^I D_-\Phi^J.}

(a) 
By now it should be routine to interpret the integrand as a section
of the sheaf of volume forms and thereby show that the integral makes sense
and that $L$ is invariant under all of the symmetries of $W^{2,2}$.

Show that this has the following generalization: if $B$ is a two-form
on $M$, we can add another term
\eqn\elly{L'=\int_{W^{2,2}} B_{IJ} D_+\Phi^I D_-\Phi^J}
which still preserves the symmetries of $W^{2,2}$.  (Of course we can
write a lot of other terms, some of which will be discussed later, if
we pick a volume form on $W^{2,2}$.)  Adding $L'$ will give a theory
that involves not ordinary differential forms on the path or loop space
of $M$, but differential forms with values in a certain line bundle.
Show that \elly\ is unchanged (up to a total derivative) if one replaces
$B$ by $B'=B+d\Lambda$.  

(b) To understand more concretely what this theory describes,
compute in the flat $\R^{2,2}$ and set
\eqn\ugg{\eqalign {X & = \Phi|_{\theta_\pm = 0}\cr
        \psi_+& = D_+\Phi|_{\theta_\pm= 0}\cr
        \psi_- & = D_-\Phi|_{\theta_\pm = 0} \cr
          F    & = -iD_+D_-\Phi|_{\theta_\pm = 0}.\cr}}  
(So $X$ is a map $X:W_{red}\to M$, while
$\psi_\pm$ and $F$ take values in the obvious
line bundles on $W_{red}$ tensored with $X^*(TM)$, and the definition
of $F$ uses the connection on $X^*(TM)$ as well as the explicit $D_\pm$.)

Calculate $L=\int du\,dv \,d\theta_+\,d\theta_- S$ (with $S$ as in \really\
above -- omit the generalization using the two-form) 
by writing it as
$L=\int du\,dv \,D_+D_-S|_{\theta_\pm = 0}$.  

\def\L{{\cal L}}  Don't be too careless in differentiating the metric --
note the appearance of the Riemann tensor!

You should see that $\psi_+$ are right-moving fermions, with values
in $X^*(TM)$, and $\psi_-$ are left-movers, again with values in $X^*(TM)$.
So as in last week's problem set, quantization of $\psi_+$ and $\psi_-$
will each give spin bundles 
on the loop or path space $\L M$ of $M$.  Call these spin bundles
$S_L(\L M)$ and $S_R(\L M)$.
$S_L$ and $S_R$ are not isomorphic, unlike the finite-dimensional
case, because the positive energy condition gives opposite
polarizations for left and right movers.


Nonetheless, the tensor product $S_L\otimes S_R$ would in finite
dimensions be isomorphic to the de Rham complex $\Omega^*(\L M)$,
so this theory gives a version of Hodge - de Rham  theory of $\L M$.

Show that
this theory is $\Z_2\times \Z_2$ graded, where one $\Z_2$ acts
by $\theta_+\to -\theta_+$ and the other by $\theta_-\to -\theta_-$.
The two $\Z_2$'s would correspond for a finite-dimensional manifold
to the Hodge $*$ and the operator that acts on $p$-forms by $(-1)^p$.
The $\Z$-valued grading of differential forms is lost in this (good!)
version of Hodge theory of loop space.

(c) Let $u=t-x$, $v=t+x$, and consider only $\Phi$'s that are invariant
under translations of $x$.  The theory thus reduces to a theory on
$\R^{1,2}$ with coordinates $t,\theta_+, \theta_-$, 
and
\eqn\umpo{L=\int dt d\theta_+ d\theta_- g_{IJ} D_+\Phi^I D_-\Phi^J.}
$Q_+$ and $Q_-$ in this theory correspond to $d+*d*$ and $i(d-*d*)$
acting on differential forms.


Note that in reducing from $\R^{2,2}$ to $\R^{1,2}$ we have lost
the infinite-dimensional symmetry algebra of $\R^{2,2}$.  Show
however, that one new symmetry appeared in the reduction --
an $SO(2)$ that rotates $\theta_+,$ $\theta_-$.  Show that this corresponds
in the quantum theory to the $\Z$-grading of differential forms.
(A $\Z_2$ -subgroup of $SO(2)$, which acts by sign change of both
$\theta_+$ and $\theta_-$, is present in the two-dimensional version,
as seen in (b).)

(d) Now we want to compare this theory to a theory on $\R^{2,1}$ which
is obtained from $\R^{2,2}$ by setting $\theta_-=0$.  

We set $\Phi'=\Phi|_{\theta_-=0}$, and $\Lambda_-=D_-\Phi|_{\theta_-=0}$.
Write $\int du \,dv\,d\theta_+\,d\theta_- \,\, S=\int du\,dv\,d\theta_+
\,D_-S|_{\theta_-=0}$ to convert the Lagrangian on $\R^{2,2}$ to
an equivalent Lagrangian on $\R^{2,1}$.

Compare with the theory of ``spinors on $\L M$ with values in a vector
bundle'' that was in problem 5 of problem set 2, and show that
-- as befits ``differential forms on $\L M$'' --  the theory discussed
here is a special case of ``spinors on $\L M$ with values in a vector
bundle.''  

(e) So far we have seen that the theories of Dirac and Hodge-de Rham
operators on a manifold $M$ as customarily studied are the one-dimensional
versions of two-dimensional theories.  (The physics that goes with this
is of course string theory.)  For Dirac, that is as far as we can go --
the chain that leads in this way to the Dirac operator on a manifold
begins in two dimensions.  But I want to point out that the
Hodge-de Rham theory can come by reduction of a three-dimensional
theory (generally less important in physics, so far at least, as it is
less well-behaved quantum mechanically).

Here we start with a supermanifold $W^{3,2}$ of the indicated dimension,
with a two-dimensional odd distribution $A$ which is generic in the
sense that $A$ and $\{A,A\}$ generate the tangent bundle of $W^{3,2}$.
A sharp difference from the two-dimensional case is that we need a
volume form $\epsilon$ on $A$.  Letting $\Phi:W^{3,2}\to M$ be a map
as before to a fixed Riemannian manifold $M$, and $D_1,$ $D_2$ a basis
of $A$ such that $\epsilon(D_1,D_2)=1$, the Lagrangian
is
\eqn\oppo{ L=\int g_{IJ} D_1\Phi^I \,D_2\Phi^J.}
Show that this is well-defined, and invariant under all symmetries
that preserve the distribution and volume form.

The flat model is $\R^{3,2}$ with  odd coordinates $\theta^A$, $A=1,2$
and even coordinates $y^{AB}$ (with $y^{AB}=y^{BA}$ so there are three
$y$'s) and
\eqn\oggo{D_A={\partial\over\partial\theta^A}-i\theta^B{\partial\over
\partial y^{AB}}.}
Given a real structure and volume form on the space of the $\theta^A$'s
the group that can act linearly on the $\theta^A$'s is $SL(2,\R)$.
This is the double cover of $SO(2,1)$, which acts on the $y$'s,
preserving a metric of signature $+--$.
The model lacks the infinite-dimensional symmetry group of the
two-dimensional case, but has the following finite-dimensional group
of symmetries.
Let
\eqn\poggo{Q_A={\partial\over\partial\theta^A}+i\theta^B{\partial\over
\partial y^{AB}}.}
Show that the $Q_A$ preserve the distribution, and that together
with the $\{Q_A,Q_B\}$ and the $SL(2,\R)$, they generate
three-dimensional super-Poincar\'e.
Together with some orientation and time-orientation reversing
symmetries, this is the full symmetry group of $\R^{3,2}$ with the
volume form.

Show that, if we impose invariance under translation of one of the
$y$'s, the model reduces to the two-dimensional model introduced
above.

If one imposes invariance under translations of {\it two} of the $y$'s
-- in  fact, pick coordinates in which the metric is $dt^2-dx_1^2-dx_2^2$
and impose invariance under translation of $x_1$, $x_2$ --
then we get the theory on $\R^{1,2}$ that is related 
upon quantization to ordinary differential forms on $M$ 
with the $d$ and $d^*$ operators.  This theory therefore
has an $SO(2)$ symmetry of rotation of the $x_1 - x_2$ plane
-- it is the $SO(2)$ noted above that rotates $\theta_+$ and $\theta_-$
in the original presentation; the generator acts on $p$-forms by
multiplication by $p$ (or $p-\half \dim M$).

(2) On differential forms on $M$ there acts the $SO(2)$ just noted,
commuting with the Laplacian.
It comes by dimensional reduction from a $1+2=3$ dimensional theory.
Differential forms on a generic $M$ cannot come from reduction of
a four-dimensional theory, as the $SO(2) $ - which is better called
$Spin(2)$ - would then have to extend to a $Spin(3)$.

Likewise, the Dirac equation on a general $M$ cannot come by reduction
from a theory above two dimensions.  If it came from a three-dimensional
theory, we would need to find a $Spin(2)$ action on the spin bundle,
commuting with the Laplacian and acting non-trivially on the Dirac operator.
So $d=2$ is the ``starting point'' in this sense for the Dirac equation.

But suppose $M$ is a Kahler manifold.  Then the $Spin(2)$ action on 
differential forms {\it does} extend to a $Spin(3)=SU(2)$ action
(sometimes called $SL(2)$, but $SU(2)$ is the group that acts on
the {\it Hilbert space} of differential forms preserving the structure).
$d$ and $d^*$ can be decomposed into four operators
$\partial,\,\bar\partial,\,\partial^*,\,\bar\partial^*$.  These
are (complex) linear combinations of four self-adjoint operators that
transform as the four dimensional real spinor representation of $Spin(3)$.
This suggests that differential forms on a Kahler manifold $M$ can
come from a theory in $1+3=4$ dimensions.

Likewise if $M$ is hyper-Kahler then the $Spin(3)$ extends further
to a $Spin(5)$ with $d$ and $d^*$ generating eight self-adjoint
operators that transform as the eight dimensional real spinor of
$Spin(5)$.  This suggests that differential forms on $M$ can come
from a theory in $1+5=6$ dimensions.

Recall that, according to Nahm's theorem, six is the maximum dimension
for supersymmetric sigma models.  The supersymmetric theory in six
dimensions of maps to a hyper-Kahler manifold (plus odd variables)
does exist -- but like all the limiting cases allowed by Nahm's theorem
(the others being gauge theory in ten dimensions and gravity in eleven
dimensions) there is no known construction via supermanifolds.

For the four-dimensional case, however, with $M$ a Kahler manifold,
a superspace construction does exist.  For the time being,
we will only describe it locally.  (It takes a few more steps
to describe a more global version.)

We consider a (4,4) supermanifold $W$ with
the structure described in Bernstein's lectures -- two complex
conjugate integrable distributions $A_+$ and $A_-$ of dimension $(0,2)$
such that $\{A_+,A_-\}$ generates $TM/(A_+\oplus A_-)$,  $A_+$ and $A_-$
are endowed with volume forms.
A chiral superfield is a field $\phi$ that is a function on
$W/A_-$.  If $M$ is a complex manifold, a chiral map $\Phi:W\to M$ is
one such that, for $f$ a local holomorphic function on $M$, 
$\Phi^*(f)$ is a chiral function on $W$.  In that case, $\Phi^*(\bar f)$ is
antichiral, that is, it is a function on $W/A_+$.

Let $K$ be a function on $W$ and consider the Lagrangian
\eqn\uggu{L=\int_Wd^4yd^4\theta\,  \Phi^*(K).}
$d^4y\,d^4\theta$ is the section of the Berezinian coming from the
volume forms on $A_\pm$.  

(a) Show that $L$ is invariant under $K\to K+f+\bar f$, where $f$ is
a local holomorphic function on $W$.

(b) Recall that any $K$ determines the two-form 
$\omega=-i\bar\partial\partial K$
which for a suitable class of $K$'s is the Kahler class of a Kahler
metric $g$ on $M$.  Of course, $\omega$ is invariant under the
transformation considered in (a).

Show that $L$ can be written just in terms of $\omega$ (or equivalently
$g$) and that the ``reduced'' part of $L$ is the harmonic map action,
specialized to maps to a Kahler manifold:
\eqn\huggu{L=\int_{W_{red}} d^4y g_{i\bar j}(d X^i,dX^{\bar j})}
where $  X^i$ are local holomorphic coordinates on $M$.

(c) Specialize now to the flat model with $W=\R^{4,4}$.
$A_+$ is generated by
\eqn\ppp{D_A={\partial\over\partial\theta^A}-
i\sigma^{m}_{A\dot A}\bar\theta^{\dot A}
{\partial\over\partial y^m}}
where $\theta^A$ are odd coordinates transforming as ``positive''
spinors of $Spin(1,3)=SL(2,{\bf C})$, $\bar \theta^{\dot A}$ are
their complex conjugates, and $y^m$ are even coordinates.
$\sigma$ is the isomorphism $S_+\otimes S_-= V$.
$A_-$ is generated by 
\eqn\ppp{D_{\dot A}={\partial\over\partial\theta^{\dot A}}-
i\sigma^{m}_{A\dot A}\bar\theta^{ A}
{\partial\over\partial y^m}.}

Impose invariance under a space-like translation of $W$, say
a translation of $y^3$ (the metric being $(dy^0)^2-\sum_{i=1}^3 (dy^i)^2$ 
to get a model on $\R^{3,4}$.  Show that this is equivalent to
the $\R^{3,2}$ 
$\sigma$ model described in problem 1(e)  (but specialized to the
case that $M$ is Kahler).  Here is an attempt  to explain the strategy:
under $Spin(1,2)$, the two spinor representations of $Spin(1,3)$
become isomorphic.  So after the reduction, $A_+$ and $A_-$ are
naturally isomorphic.  One can in an $SO(1,2)$-invariant way
take the ``real'' combination $\chi^A=\theta^A+\bar\theta^A$ of
the  $\theta$'s and $\bar\theta$'s. The $\R^{3,2} $ we want
has even coordinates $y^i$,          $i=0,1,2$, and odd coordinates
$\chi^A$.  By ``integrating over the fibers'' of a map from $\R^{3,4}$
to $\R^{3,2}$, one reduces \huggu\ (specialized to $y^3$-independent maps)
to the $\R^{3,2}$ Lagrangian of problem 1(e).

\end

