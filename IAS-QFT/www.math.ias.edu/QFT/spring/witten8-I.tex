%Date: Mon, 19 May 1997 17:00:31 -0400 (EDT)
%From: Pavel Etingof <etingof@abel.math.harvard.edu>

\input amstex
\documentstyle{amsppt}
\magnification 1200
\NoRunningHeads
\NoBlackBoxes
\document

\def\opsi{\overrightarrow{\psi}}
\def\h{\frak h}
\def\tW{\tilde W}
\def\Aut{\text{Aut}}
\def\tr{{\text{tr}}}
\def\ell{{\text{ell}}}
\def\Ad{\text{Ad}}
\def\u{\bold u}
\def\m{\frak m}
\def\O{\Cal O}
\def\tA{\tilde A}
\def\qdet{\text{qdet}}
\def\k{\kappa}
\def\RR{\Bbb R}
\def\be{\bold e}
\def\bR{\overline{R}}
\def\tR{\tilde{\Cal R}}
\def\hY{\hat Y}
\def\tDY{\widetilde{DY}(\g)}
\def\R{\Bbb R}
\def\h1{\hat{\bold 1}}
\def\hV{\hat V}
\def\deg{\text{deg}}
\def\hz{\hat \z}
\def\hV{\hat V}
\def\Uz{U_h(\g_\z)}
\def\Uzi{U_h(\g_{\z,\infty})}
\def\Uhz{U_h(\g_{\hz_i})}
\def\Uhzi{U_h(\g_{\hz_i,\infty})}
\def\tUz{U_h(\tg_\z)}
\def\tUzi{U_h(\tg_{\z,\infty})}
\def\tUhz{U_h(\tg_{\hz_i})}
\def\tUhzi{U_h(\tg_{\hz_i,\infty})}
\def\hUz{U_h(\hg_\z)}
\def\hUzi{U_h(\hg_{\z,\infty})}
\def\Uoz{U_h(\g^0_\z)}
\def\Uozi{U_h(\g^0_{\z,\infty})}
\def\Uohz{U_h(\g^0_{\hz_i})}
\def\Uohzi{U_h(\g^0_{\hz_i,\infty})}
\def\tUoz{U_h(\tg^0_\z)}
\def\tUozi{U_h(\tg^0_{\z,\infty})}
\def\tUohz{U_h(\tg^0_{\hz_i})}
\def\tUohzi{U_h(\tg^0_{\hz_i,\infty})}
\def\hUoz{U_h(\hg^0_\z)}
\def\hUozi{U_h(\hg^0_{\z,\infty})}
\def\hg{\hat\g}
\def\tg{\tilde\g}
\def\Ind{\text{Ind}}
\def\pF{F^{\prime}}
\def\hR{\hat R}
\def\tF{\tilde F}
\def\tg{\tilde \g}
\def\tG{\tilde G}
\def\hF{\hat F}
\def\bg{\overline{\g}}
\def\bG{\overline{G}}
\def\Spec{\text{Spec}}
\def\tlo{\hat\otimes}
\def\hgr{\hat Gr}
\def\tio{\tilde\otimes}
\def\ho{\hat\otimes}
\def\ad{\text{ad}}
\def\Hom{\text{Hom}}
\def\hh{\hat\h}
\def\a{\frak a}
\def\t{\hat t}
\def\Ua{U_q(\tilde\g)}
\def\U2{{\Ua}_2}
\def\g{\frak g}
\def\n{\frak n}
\def\hh{\frak h}
\def\sltwo{\frak s\frak l _2 }
\def\Z{\Bbb Z}
\def\C{\Bbb C}
\def\d{\partial}
\def\i{\text{i}}
\def\ghat{\hat\frak g}
\def\gtwisted{\hat{\frak g}_{\gamma}}
\def\gtilde{\tilde{\frak g}_{\gamma}}
\def\Tr{\text{\rm Tr}}
\def\l{\lambda}
\def\I{I_{\l,\nu,-g}(V)}
\def\z{\bold z}
\def\Id{\text{Id}}
\def\<{\langle}
\def\>{\rangle}
\def\o{\otimes}
\def\e{\varepsilon}
\def\RE{\text{Re}}
\def\Ug{U_q({\frak g})}
\def\Id{\text{Id}}
\def\End{\text{End}}
\def\gg{\tilde\g}
\def\b{\frak b}
\def\S{\Cal S}
\def\L{\Lambda}
\def\tl{\tilde\lambda}

\topmatter
\title Lecture II-8, Part I: Solitons 
\endtitle
\author {\rm {\bf Edward Witten} }\endauthor
\endtopmatter

\centerline{Notes by Pavel Etingof and David Kazhdan}

\vskip .1in

{\bf 8.1. What is a soliton?}

In classical 
mathematical physics, by a soliton one usually means a ``traveling wave'' 
solution of a nonlinear PDE $u_t=F(u,u_x,...)$, i.e. a solution 
of the form $u(x,t)=f(x-vt)$.   
Solitons play a very important role in the theory of integrable systems, 
where any solution can be approximated by a superposition of solitons, 
moving at different velocities. As a result, the theory of integrable systems 
is sometimes called soliton theory. 

Today we will be interested in solitons arising in field theory 
(as traveling wave solutions of the classical field equations) and primarily 
in the role they play in quantization of field theories. This is 
a different point of view from the one in soliton theory. 
In particular, no claim is made about nonlinear superposition of solitons, 
and the models we consider will not, in general, be exactly integrable. 

We will consider solitons for Poincare invariant field theories on
Minkowski space. By a soliton for a particular field theory we will mean 
a traveling wave solution of the field equations (i.e. a solution which 
depends on $x-vt$), which is localized in space and has finite energy.
By Poincare invariance, we can always assume that $v=0$, i.e. that the 
solution is time-independent. We will be mostly interested in solitons which 
provide the global minimum for the energy in the corresponding homotopy class. 

{\bf Remark.} It is important to distinguish solitons from instantons. 
Instantons are localized in Euclidean spacetime 
(i.e. only exist for an instant) and have finite action,
while solitons are localized in space (of Minkowski spacetime), exist 
eternally, and have finite energy. 

{\bf 8.2. Solitons and components of the space of classical solutions}

Classically, existence of solitons is related with existence of different 
components of the space of classical solutions of finite energy.
In a connected component which does not contain a zero energy solution, 
the minimum of energy is often attained at a soliton. As an example of this, 
you may recall the situation discussed in Lecture II-1: a 2-dimensional 
scalar field theory with the potential $U=(\phi^2-a^2)^2$.
In this case, the space of t-independent classical solutions
of finite energy has 4 components: $X_{++},X_{-+},X_{+-},X_{--}$, 
where $X_{+-}=\{\phi:\phi(-\infty)=a,\phi(\infty)=-a\}$ etc.
On two of these components, $X_{+-}$ and $X_{-+}$, the energy 
is strictly positive, and its minimum is attained at 2 solitons, 
$\phi=f(x)$ and $\phi=-f(x)$, where $f(x)$ is the solution of the 
Newton equation $f''=U'(f)$ for the potential $-U$, with boundary  
values $a$ at $-\infty$ and $-a$ at $\infty$
(such a solution is defined uniquely up to translations).  

Notice that solitons are not invariant under the Poincare group. But 
one often encounters
 solitons which have rotational symmetry in space, around some
``center of mass''. In a scalar field theory, this would mean that 
the group of symmetry of such a soliton is $P_s=SO(d-1)\times \R$. 
Since the Poincare group is $P=SO(d-1,1)\ltimes \R^{d-1,1}$, the 
P-orbit $\O$ of the soliton in the space of solutions, i.e. the quotient
$P/P_s$, is a $d-1$-dimensional vector bundle over the upper part 
of a 2-sheeted hyperboloid. It is easy to check the following.

1. This bundle is naturally isomorphic to the cotangent bundle. 

2. The restriction of the symplectic structure on the space of solutions 
$\O$ is nondegenerate, and thus defines a symplectic structure on $\O$. 

3 (normalization of symplectic form) Let $m$ be the mass 
(i.e. the energy in the center of mass frame) of the 
soliton. Then there exists a P-equivariant symplectic diffeomorphism 
$\O\to T^*\O_m^+$, where $\O_m^+$ is the 
upper part of the hyperboloid $x^2=-m^2$ in $\R^{d-1,1}$.
    
{\bf 8.3. Solitons and quantization}

Since classically solitons correspond to components in the space of 
solutions, quantum mechanically they should correspond to 
direct summands in the Hilbert space. As an example, consider a scalar 
field theory with the space $X$ of classical solutions of finite energy,
in which there is a component $X_s$ containing a soliton $s\in X_s$. 
Then we have a symplectic embedding $T^*\O_m^+\to X_s$, where 
$m$ is the mass of $s$. Let us assume that the minimum of energy 
at the image of this map is nondegenerate, in the sense that there is no other 
solutions of energy $m$, and the second derivative of the energy in a 
direction  transversal to the image is positive. 

In this case, in the weak coupling region we should expect that 

1. The component $\Cal H_s$ of the Hilbert space $\Cal H$ of the quantum 
theory has no vacuum (because classically there is no P-invariant solution).

2. The Hamiltonian $H$ on $\Cal H_s$ satisfies the inequality $H\ge m'$,
where $m'$ is some positive mass parameter, such that $m'\to m$ in the weak 
coupling limit (in general, we should expect $m$ to get quantum corrections).
If the theory has a mass gap, we should expect that     
the states with $H=m$ form a space which is a quantization of $T^*\O_m^+$, 
i.e. the irreducible representation of $P$ of the form $L^2(\O_m^+)$. 
We should also expect that the spectrum near $m$ is discrete since the second 
derivative is positive. 

Note that if the action (or energy) functional of the theory is multiplied 
by a constant $C$, the mass of the soliton is also multiplied by $C$, while
masses of usual particles do not change. This means that in the classical 
approximation ($C\to\infty$), solitons are much heavier    
than usual particles. Therefore, a soliton cannot be seen in perturbation 
theory: the contribution to the correlation 
functions of an intermediate state 
containing a soliton is exponentially small 
(compared to the coupling constant) in the weak coupling limit.

{\bf 8.4. Solitons in theories with fermions}

In theories with fermions, the orbit of a soliton under $P$ 
is often not the whole 
space of lowest energy states in the corresponding connected component 
of the space of solutions. 
For example, if the model is supersymmetric, 
it is clear apriori that the orbit is not the whole space
of lowest energy solutions: the space of lowest energy states 
is the orbit of the superPoincare and not just the Poincare group. 

Let us consider an example of such a situation. Consider 
the theory in 2 dimensions with a scalar and a pair of fermions:
$$
\Cal L=\frac{1}{2\l}\int d^2x(|d\phi|^2+(\phi^2-a^2)^2)+
i\int d^2x (\psi_+\d_-\psi_++\psi_-\d_+\psi_--g\phi\psi_+\psi_-).\tag 8.1
$$

{\bf Remark 1.} For a suitable value of $g$ this model is supersymmetric.

{\bf Remark 2.} This model has a chiral symmetry $\phi\to -\phi$, 
$\psi_\pm\to \pm \psi_\pm$, which prohibits a mass term $m\psi_+\psi_-$ 
for the fermions.  

Consider the soliton $\phi(x,t)=f(x)$ for the bosonic part of the theory.
Fermionic extensions of this solution are functions of finite energy 
$\psi(x)=\left(\matrix\psi_+(x)\\ \psi_-(x)\endmatrix\right)$ satisfying 
the Euler-Lagrange equations
$$
\left[\d_x\psi+\left(\matrix 0&-gf(x)\\ -gf(x)&0\endmatrix\right)\right]
\psi=0.\tag 8.2
$$
Thus, 
$$
\psi=\left(\matrix \e\\ \e\endmatrix\right)e^{g\int_0^xf(y)dy}+
\left(\matrix \e'\\ -\e'\endmatrix\right)e^{-g\int_0^xf(y)dy},\tag 8.3
$$
where 
$\e,\e'$ are odd variables. 
It is easy to see that only the solutions with $\e'=0$ 
are in $L^2$, so the space of solutions of finite energy is 
1-dimensional.

Thus, each pair of fermions $\psi_+,\psi_-$ interacting with $\phi$ via
(8.1) creates a fermionic degree of freedom in the space of configurations
of minimal energy in the connected component of $f(x)$. Namely, if the number
of such pairs is $n$ then the space of minimal energy configurations is 
not $T^*\O_m^+$, but the supermanifold $T^*\O_m^+\times \R^{0|n}$. 

Therefore, if $n$ is even, in quantum theory the space of 
lowest energy states in the corresponding component of the Hilbert space 
is $L^2(\O_m^+)\o S$, where $S=S_++S_-$ is the spin representation of
$Spin(n)$. 

{\bf 8.5. Solitons in 2+1 and 3+1 dimensions.}

In spite of the difference between instantons and solitons, 
there is a connection between them.  
Namely, often an instanton in a Euclidean field theory in $d-1$ dimensions
gives rise to a soliton in the Minkowski version of the same theory in $d$ 
dimensions. Consider for example the 2+1-dimensional $U(1)$ gauge theory 
with a complex scalar:
$$
\Cal L=\int d^3x(\frac{F_A^2}{e^2}+\frac{1}{\l}(|d_A\phi|^2+(|\phi|^2-a^2)^2)).
\tag 8.4
$$
In Lecture II-6 we saw that the 2-dimensional version of this theory 
has instantons with Chern classes $1$ and $-1$. 
In the 2+1-dimensional theory, these instantons become solitons, and 
govern the lowest energy modes in the corresponding components 
of the Hilbert space as described above. 

Now consider nonabelian gauge theory in 3+1 dimensions. 
Namely, consider an $SO(3)$ gauge theory with a boson in the 3-dimensional 
representation, and the Lagrangian 
$$
\Cal L=\int d^4x(\frac{F_A^2}{4e^2}+
\frac{1}{2\l}(|d_A\phi|^2+(|\phi|^2-a^2)^2)).
\tag 8.5
$$   
We considered in Lecture II-7 the 3-dimensional version of this theory.
For convenience, we identified the spacetime $\R^3$ 
with the Lie algebra of the gauge group and with the space of values of 
$\phi$ (the bracket in the Lie algebra is the cross-product). 
This allows to write the scalar field $\phi$ and infinitesimal gauge
transformations as vector fields on $\R^3$.

We found that in 3 dimensions this theory   
has an instanton in which $\phi$ is
of the form $\phi=\frac{x}{r}f(r)$, $r=|x|$ (as we have explained, 
we identify the spacetime and the space of values of $\phi$). 
In 4 dimensions this instanton will become a soliton. 
Such solitons are called magnetic monopoles. 

In fact, we have not one soliton, 
but infinitely many, since the center of the soliton can be any point in 
$\R^3$. So the space of time-independent solitons is at least $\R^3$. 
In fact, this space is not $\R^3$ but $\R^3\times S^1$. 
The reason is that there are some gauge symmetries compatible with 
spherical symmetry, which allow to produce new solitons out of old ones. 
Let us see how it happens. 

First of all, recall that
any field configuration $(A,\phi)$ of finite energy 
defines an integer 
topological invariant -- ``the first Chern class at infinity''
$c_1$. Indeed, in order for the energy to be finite, we must have 
$|\phi|=a$ at infinity, so $\phi$ defines a map from a
sphere at infinity in $\R^3$ to the sphere of radius $a$, and $c_1$ is
the degree of this map. Another definition of $c_1$:  
the section $\phi$ defines 
at infinity a splitting of our 3-dimensional vector bundle into a direct sum 
$\phi\oplus\phi^\perp$ of a 1-dimensional and a 2-dimensional vector bundle. 
Thus, $\phi$ defines a reduction of the structure group to $SO(2)=U(1)$ at 
infinity. The number $c_1$ is the first Chern class of this bundle restricted
to the infinite 2-sphere in $\R^3$. 

For example, the soliton configuration discussed above has $c_1=1$. 

In physical language, this topological phenomenon means that
at spacial infinity the $SO(3)$ gauge symmetry is broken to $U(1)$.  
However, the remaining group $U(1)$ of transformations at infinity acts
nontrivially on the space of classical solutions.
In particular, it produces new solitons. 
To understand this action, let us consider the solitons $(A,\phi)$ 
with $c_1=1$, discussed above.  
Let us represent the infinitesimal 
operator of the group $U(1)$ by a spherically symmetric gauge symmetry: 
$\e=\frac{x}{r}g(r)$, where $g$ is some function. 
If the function $g(r)$ satisfies $g(r)\sim c_0r$ at $r\to 0$ and 
$g(+\infty)=c$, then this formula defines a smooth gauge transformation 
which is ``constant'' at infinity (with respect to the reduction 
of structure group defined by the soliton). 
Then the action of $\e$ is
$$
A\to A-d_A\e, \phi\to \phi.\tag 8.6
$$
It is clear that $d_A\e$ cannot be identically zero, since 
the soliton connection $A$ is not flat. 
On the other hand, if $c\in 2\pi \bold Z$, the connection 
$A-d_A\e$ is equivalent to $A$ by the gauge transformation 
$e^{i\e}$ which vanishes at infinity, so the solutions 
$(A,\phi)$ and $(A-d_A\e,\phi)$ are the same in this case. 

This shows that the space of 
time-dependent solitons with Chern class 1 at infinity 
is at least $\R^3\times S^1$
(with no canonical zero on either $\R^3$ or $S^1$). 
One can show that in fact it is exactly 
that. We will denote the circle coordinate on this space by $\alpha$.

By using the Poincare group transformations, we can 
generate solutions which are time-dependent and propagate 
at a constant speed. We can also perform a 
time dependent gauge transformation, which will make  
the $\alpha$-coordinate time dependent, i.e. $\alpha=\alpha_0+st$, $s\in \R$. 
As a result, the space of 
time dependent solitons is the product $T^*\O\times T^*S^1$.

{\bf 8.6. The 3+1-dimensional theory with the $\theta$-angle}

Consider the 3+1-dimensional theory of the previous section with the 
$\theta$-angle, i.e. let us add to the (Minkowski) Lagrangian a term
$$
-\frac{\theta}{16\pi^2}\int \text{Tr}(F\wedge F).\tag 8.7
$$

To give this term a topological interpretation, let us compactify the time
and consider the theory on the spacetime $\R^3\times S^1$. 
In this case, for any field configuration of finite energy, 
besides the first Chern class $c_1$ at infinity 
(which is also called the monopole number, or the hedgehog number),
we can define another integer topological invariant -- 
the second Chern class $c_2$, which is given by the integral 
$\frac{1}{8\pi}\int Tr(F\wedge F)$. Thus, 
classically, term (8.7) just counts the second Chern class 
of the bundle. Therefore,  
quantum mechanically, it weights the contribution 
from bundles with $c_2=k$ to the path integral with $e^{ik\theta}$.

It is clear that this theory has the same time-independent 
solitons as the theory 
without $\theta$, since the added term is topological. 
However, since time has been compactified, the time dependent solitons 
which were discussed in the previous section have to 
satisfy the equality $\alpha=\alpha_0+\frac{n}{T}t$, where $n$ is an integer,
which we will call the winding number,  
and $T$ is the circumference of the compactified time axis. In other words, the
parameter $s$ defined in the previous section has 
to have the form $s=n/T$. 

\proclaim{Claim} If the monopele number of a soliton configuration 
is $k$ and the winding number is $n$ then $c_2=kn$. 
\endproclaim

The proof is by a direct calculation. 

Now 
consider 3 operators in our theory:

1. $\frac{\d}{\d\alpha}$. (This operator is 
the generator of the Lie algebra of the unbroken 
$U(1)$). 

2. The electric charge 
$$
Q_{el}=\frac{1}{ae^2}\int_{\R^3} Tr(d_A\phi\wedge *F_A).\tag 8.8
$$

3. The magnetic charge
$$
Q_{mag}=\frac{1}{4\pi a}\int_{\R^3} Tr(d_A\phi\wedge F_A).\tag 8.9
$$

In perturbation theory, $Q_{mag}=0$ (as nonzero $c_2$ requires big action), 
and $Q_{el}=-i\frac{\d}{\d\alpha}$. Thus, in perturbation theory 
$Q_{el}$ has integer eigenvalues. 

Let us see, however, what happens nonperturbatively, i.e. 
when we take into account field configurations with nonzero $c_2$. 

We compute in the gauge where $A_0=0$. 
It is easy to compute that 
classically we have $\frac{\d}{\d\alpha}=\int Tr(d_A\phi\frac{\delta}
{\delta A})$. Therefore,   
the operator $-i\frac{\d}{\d\alpha}$ is the charge for 
the current $J=Tr(d_A\phi\wedge \pi_A)$, where 
$\pi_A$ is the conjugate (momentum) variable for $A$.

Now compute $\pi_A$:
$$
\pi_A=\frac{\delta \Cal L}{\delta A_t}=\frac{A_t\wedge dt}{e^2}-
\frac{\theta}{8\pi^2}F_A.\tag 8.10
$$    
Thus, we get
$$
-i\frac{\d}{\d \alpha}=Q_{el}-\frac{\theta}{2\pi}Q_{mag}.\tag 8.11
$$ 

Since the spectrum of $-i\frac{\d}{\d\alpha}$ is integer, we 
get
$Q_{el}=\frac{\theta}{2\pi}Q_{mag}\text{ mod }\Bbb Z$. 
But $Q_{mag}$ is also an integer, since it is a topological invariant 
classically. Thus, $Q_{el}$ need not be an integer beyond perturbation 
theory. In other words, there exist states (of 
very high energy $\sim 1/\hbar$) on which $Q_{el}$ is not an integer. 
So the electric charge is discretized, but in presence of magnetic monopoles 
we do not expect an integral electric charge. 

It is not hard to show that operators $Q_{el},Q_{mag}$ commute.
(Classically, it is obvious, as $Q_{mag}$ is a locally constant function). 
Thus, the joint spectrum of them is a lattice in $\R^2$. If we consider 
the family of theories parametrized by $\theta\in S^1$, the monodromy 
transformation of this lattice around the circle is 
given by the matrix $\left(\matrix 1&1\\ 0&1\endmatrix\right)$ 
in the basis $Q_{mag},Q_{el}$. 

In the second part of this lecture we will explain
(in the free theory) how to 
extend this monodromy representation to an action of $SL_2(\Bbb Z)$. 
I.e. how to construct a family of theories parametrized by a complex
parameter $\tau=\frac{\theta}{2\pi}+ir$ modulo modular transformations, 
such that the monodromy representation is the standard action of 
$SL_2(\Bbb Z)$ on a 2-dimensional lattice. 



\end


