%From: John W Morgan <jmorgan@math.ias.edu>
%Date: Thu, 13 Feb 1997 15:25:20 -0500
%Subject: Witten Lecture notes

\documentstyle[10pt]{article}


%These are the macros which are in common with all of the
% sections in the paper mmr
% Each section, for now, should begin with \documentstyle[11pt,cd]{article}
% and then have \input{mmrmacros} followed by \begin{document}
% The only exception is that the \Label macro is slightly different
% in each file and should be put in separately.
%New CD macros
\newcommand{\cdrl}{\cd\rightleftarrows}
\newcommand{\cdlr}{\cd\leftrightarrows}
\newcommand{\cdr}{\cd\rightarrow}
\newcommand{\cdl}{\cd\leftarrow}
\newcommand{\cdu}{\cd\uparrow}
\newcommand{\cdd}{\cd\downarrow}
\newcommand{\cdud}{\cd\updownarrows}
\newcommand{\cddu}{\cd\downuparrows}
% (S) Proofs.
% (S-1) Head is automatically supplied by \proof.

\def\proof{\vspace{2ex}\noindent{\bf Proof.} }
\def\tproof#1{\vspace{2ex}\noindent{\bf Proof of Theorem #1.} }
\def\pproof#1{\vspace{2ex}\noindent{\bf Proof of Proposition #1.} }
\def\lproof#1{\vspace{2ex}\noindent{\bf Proof of Lemma #1.} }
\def\cproof#1{\vspace{2ex}\noindent{\bf Proof of Corollary #1.} }
\def\clproof#1{\vspace{2ex}\noindent{\bf Proof of Claim #1.} }
% End of Proof Symbol at the end of an equation must precede $$.

\def\endproof{\relax\ifmmode\expandafter\endproofmath\else
  \unskip\nobreak\hfil\penalty50\hskip.75em\hbox{}\nobreak\hfil\bull
  {\parfillskip=0pt \finalhyphendemerits=0 \bigbreak}\fi}
\def\endproofmath$${\eqno\bull$$\bigbreak}
\def\bull{\vbox{\hrule\hbox{\vrule\kern3pt\vbox{\kern6pt}\kern3pt\vrule}\hrule}}
\addtolength{\textwidth}{1in}                  % Margin-setting commands
\addtolength{\oddsidemargin}{-.5in}
\addtolength{\evensidemargin}{.5in}
\addtolength{\textheight}{.5in}
\addtolength{\topmargin}{-.3in}
\addtolength{\marginparwidth}{-.32in}
\renewcommand{\baselinestretch}{1.6}
\def\hu#1#2#3{\hbox{$H^{#1}(#2;{\bf #3})$}}          % #1-Cohomology of #2
\def\hl#1#2#3{\hbox{$H_{#1}(#2;{\bf #3})$}}          % #1-Homology of #2
\def\md#1{\ifmmode{\cal M}_\delta(#1)\else  % moduli space, delta decay of #1
{${\cal M}_\delta(#1)$}\fi}
\def\mb#1{\ifmmode{\cal M}_\delta^0(#1)\else  %moduli space, based, delta
					      %decay of #1
{${\cal M}_\delta^0(#1)$}\fi}
\def\mdc#1#2{\ifmmode{\cal M}_{\delta,#1}(#2)\else    %moduli space, delta
						      %decay, chern class #1
						      %of #2
{${\cal M}_{\delta,#1}(#2)$}\fi}
\def\mbc#1#2{\ifmmode{\cal M}_{\delta,#1}^0(#2)\else   %as before, based
{${\cal M}_{\delta,#1}^0(#2)$}\fi}
\def\mm{\ifmmode{\cal M}\else {${\cal M}$}\fi}
\def\ad{{\rm ad}}
\def\msigma{\ifmmode{\cal M}^\sigma\else {${\cal M}^\sigma$}\fi}
\def\cancel#1#2{\ooalign{$\hfil#1\mkern1mu/\hfil$\crcr$#1#2$}}
\def\dirac{\mathpalette\cancel\partial}
\newtheorem{thm}{Theorem}
\newtheorem{theorem}{Theorem}[subsection]
\newtheorem{proposition}[theorem]{Proposition}
\newtheorem{lemma}[theorem]{Lemma}
\newtheorem{claim}[theorem]{Claim}
\newtheorem{example}[theorem]{Example}
\newtheorem{corollary}[theorem]{Corollary}
\newtheorem{D}[theorem]{Definition}
\newenvironment{defn}{\begin{D} \rm }{\end{D}}
\newtheorem{addendum}[theorem]{Addendum}
\newtheorem{R}[theorem]{Remark}
\newenvironment{remark}{\begin{R}\rm }{\end{R}}
\newcommand{\note}[1]{\marginpar{\scriptsize #1 }} 
\newenvironment{comments}{\smallskip\noindent{\bf Comments:}\begin{enumerate}}{\end{enumerate}\smallskip}

\renewcommand{\thesection}{\Roman{section}}
\def\eqlabel#1{\addtocounter{theorem}{1}
\write1{\string\newlabel{#1}{{\thetheorem}{\thepage}}}
\leqno(\rm\thetheorem)}
\def\cS{{\cal S}}
\def\ov{\overline}












\title{The Dirac Index on Manifolds and Loop Spaces: Part I}
\author{Edward Witten\thanks{\tt{Lecture notes by John Morgan}}}
\date{December 10, 1997}

\begin{document}
\maketitle

\section{Introduction}

One mathematical approach
to the proof of the Hirzebruch index theorem 
is through the  heat equation.  One considers, for each $t>0$, the
operator $e^{-t\Delta}$ acting on the differential forms on a closed,
oriented riemannian manifold $Y$. A direct spectral argument shows
that the trace of this operator (or more precisely the difference of
its trace on the self-dual forms and on the antiself-dual forms) is
independent of $t$ and limits as $t\mapsto \infty$ to the index of the
manifold. On the other hand, the limit as $t\mapsto 0$ of this
operator can be computed in terms of local invariants, curvature, of
the manifold.  One can even explicitly evaluate these local
expressions, and in this manner recover  the result that the signature
of the manifold is given by $L$-polynomials of Hirzebruch.  In a
related context, that of equivariant signatures when there is a group
action, one sees a localization taking place in the Atiyah-Hirzebruch
formula for the equivariant signature in terms of local data
(characteristic classes) at the fixed points.
These localizations are reminiscent of localizations in quantum field
theory, e.g., perturbation expansions around the classical solutions
of Lagrange's equations, and raise the question of whether there are
quantum field theory arguments establishing these and similar
results. 

The answer to this question is yes, and 
it is our purpose here to show how to establish results  of a similar
nature (for Dirac operators on a spin manifold) by
doing a perturbation expansion of an 
appropriate path integral or Hamiltonian eigenstates.  The explanation
of the quantum field theories that do indeed lead 
to index theorems for the Dirac operator on a
spin manifold (an its equivariant analogue) are the first subject of
these lectures. Once these classical cases are been covered, we turn
to more sophisticated applications of the same techniques.  This
involves the index of Dirac operators on the loop spaces of spin
manifolds. While the results on loop spaces were first discovered by
the path integral approach that we take here,  eventually,
mathematical proofs were also found for these results [Bott-Taubes].


Before we get into the details of the computations, one general
comment is in order. So far the perturbative expansions we have
encountered have been around an isolated, non-degenerate critical
point.
That is to say the space of classical solutions (the space
of minima of the Lagrangian) is a single point and the Hessian 
(the quadratic term in the Lagrangian) is positive definite at that point.  
The expansions that we encounter here will be of a more general
type. The set of classical solutions  will be a compact, finite dimensional 
smooth manifold (or super-manifold) and the
quadratic term in the Lagrangian restricted to the
normal bundle to the critical set is positive definite.
The same techniques will work for oscillatory integrals as long as the
Hessian  in the normal directions is non-degenerate (the Chern-Simons
functional in three dimensions gives examples of this type).
  Of course, one
could meet situations where the critical set is singular and
the Hessian has null-space which is not of constant rank.  Such
situations require techniques that are not discussed here.

In the situation where the critical set is a super-manifold $M$ and at
each point of the critical set the Hessian has null space equal to the
tangent space of the critical set, the basic approach is to apply the
usual perturbative expansion using Feynman diagrams and regularization
in the normal direction where there is an isolated, non-degenerate
critical point to  
obtain a  result, and then integrate this result over the critical
set. Of course, it may well happen that the individual expansions over
the normal spaces have no intrinsic meaning and depend on choices
but we expect the resulting
integral over $M$ to be well-defined. We will also be computing only
the constant term in the expansions.  These of course will depend only on
the quadratic term in the Lagrangian and will be written in terms of
the determinant of the propagator for the Bosons and
Pfaffian of the
propagator of the fermions integrated over the normal directions.



\section{The Dirac Operator on a Spin Manifold}


In this section we shall give a path integral derivation of the index
theorem for the Dirac operator on a spin manifold. Fix for this
section 
$Y$ a compact, oriented, $2n$-dimensional spin manifold.

\subsection{The Lagranian Formulation}

Let ${\bf R}^{1,1}$ denote the super Euclidean space with one even
variable and one odd variable; i.e., $C^\infty({\bf
R}^{1,1})=C^\infty({\bf R})\otimes \wedge^*({\bf R})$.  We denote by
$t,\theta\in C^\infty({\bf R}^{1,1})$ the natural even and odd
variables, respectively.
We consider a quantum theory of maps
$$X\colon {\bf R}^{1,1}\to Y,$$
where $Y$ is a compact spin manifold of dimension $2n$.
The Lagrangian we take is
$${\cal L}=-\int_{{\bf R}^{1,1}}dtd\theta^{-1}
\frac{1}{2}g_{I,J}(X)\frac{dX^I}{dt}DX^J=-\int_{{\bf
R}^{1,1}}\frac{1}{2}\langle \dot X,DX\rangle ,$$ 
where as usual $g_{I,J}$ denotes the metric tensor on $Y$ and $D$ is
the vector field 
$$D=\frac{\partial}{\partial\theta}-\theta\frac{\partial}{\partial
t}$$
on ${\bf R}^{1,1}$.

Let us recall some of the material from Problem 2 in Problem
Set One. 
We can view $X$ as a map $x$ from
${\bf R}^1$ to $Y$ together with a section $\psi$ of the odd vector
bundle $\Pi(x^*TY)$ over ${\bf R}^1$. 
We write $X=x(t)+\theta \psi(t)$ with this understanding as to the
meanings of $x$ and $\psi$.
Then we can re-write the
Lagrangian as
\begin{equation}\label{rewrite}
{\cal L}=
-\int_{{\bf R}^{1,1}}\frac{1}{2}\langle \dot
x(t)+\theta\nabla_t(\psi(t)) ,\psi(t)-\theta  
\dot x(t)\rangle dtd\theta^{-1}.
\end{equation}
If one prefers an expression in local coordinates,
doing the $\theta$-integration yields:
$${\cal L}=\frac{1}{2}\int_{{\bf R}^1}\langle \dot x(t) ,
\dot x(t) \rangle -\langle \nabla_t\psi(t),\psi(t)\rangle dt.$$


The classical solutions to this Lagrangian form a
supermanifold of maps $X\colon {\bf R}^{1,1}\to Y$ satisfying
$$\nabla_tDX=0.$$
We can also write this equation as a pair of equations:
$$\nabla_t\psi=0$$
$$F(\psi,\dot x)(\psi)=\nabla_t(\dot x),$$
where $F$ is the curvature of $Y$.
Fixing a time $t_0$ and evaluating at this time identifies this
supermanifold with space of ``initial conditions'' at time $t=t_0$.
The initial conditions are simply a pair -- a tangent vector $\dot x_0$
at some point $x_0$ of $Y$ and an element $\psi_0$ in a copy of
$TY_{x_0}$ considered as an odd vector bundle. That
is to say, the space of initial conditions is identified  with the odd
vector bundle 
$p^* \Pi (TY)$ over $TY$,
where $p\colon TY\to Y$ is the natural projection mapping.
The symplectic structure on this space of classical solutions is given
by (see Problem 2 of Problem Set One)
$$\omega =
\omega_{T^*Y}+\frac{1}{2}(\nabla\psi,\nabla\psi)
-\frac{1}{2}\left(F(\psi),\psi\right),$$
where the first term is the canonical symplectic form on the cotangent
bundle which is identified with the tangent bundle using the
riemannian metric on $Y$.


\subsection{The Quantization}

We wish to apply canonical quantization to this super-manifold of
classical solutions.  Were the space of classical solutions simply
$TY$ with the usual symplectic structure, then canonical quantization
would yield the Hilbert space $L^2(Y)$ of complex-valued $L^2$-functions on
$Y$. (More precisely, we would get the space of $L^2$ half-densities.)
The functions on $Y$ act by multiplication and sections of $TY$
(i.e., vector fields) act by 
differentiation. But in our case where the space of classical
solutions is a super-manifold, we must mix this canonical quantization
of $TY$ with a quantization of the vector spaces which are the fibers
of the odd bundle $p^*(\Pi TY)\to TY$.

Let us consider a real, even dimensional super vector space $V$ which is
purely odd  with respect to the ${\bf Z}/2{\bf Z}$-grading.
Suppose further that $V$ has  a positive
definite inner product and an orientation and a spin structure.
We quantize this space by taking as the Hilbert superspace
the complex spin bundle $S(V)$, with the ${\bf Z}/2{\bf Z}$-grading given by
the usual decomposition of $S(V)$ into $S^+(V)\oplus S^-(V)$.  The
linear forms on $V$ are quantized to 
operators  on this space which act by Clifford multiplication:
$$V^*\otimes S(V)\to S(V).$$ 

Now we must mix together the two constructions.
We have the odd vector bundle $p^*(\Pi TY)$ over the symplectic
manifold $TY$. The Hilbert space is the 
space of $L^2(S(Y))$, the space of $L^2$-sections of the spin bundle
 $S(Y)$, again tensored by the half-densities on $Y$.
The functions on $Y$ act as usual by multiplication; the linear forms
in $TY$ act by covariant differentiation, and the linear forms on $\Pi
TY$ act by Clifford multiplication.
This space of sections of $S(Y)$ is  a super Hilbert space -- the
${\bf Z}/2{\bf Z}$-grading 
is induced with the usual decomposition of spinors in
even dimensions $S(Y)=S^+(Y)\oplus S^-(Y)$.

The Hamiltonian operator for this quantization is given by quantizing
the Hamiltonian function $h$ 
associated with the flow $\partial/\partial t$ on the theory:
$$dh= -i_{\partial/\partial t}\omega.$$
In this case it is easy to see (cf, Problem 2 in Problem Set 1)
that 
$$h=\frac{1}{2} |\dot x|^2.$$
In quantizing this function
there is in general an ambiguity in the resulting second order
operator (adding zero and first order operators), but
one representative for it is the Laplacian $\nabla^*\circ\nabla$ on
spinor fields.
In our situation the vector field 
$$Q=\frac{\partial}{\partial \theta}+\theta\frac{\partial}{\partial
t}$$ on ${\bf R}^{1,1}$ (whose square is $\partial/\partial t$)
yields a vector field acting on the theory.
The Hamiltonian function for this vector field, when quantized,
yields a naturally defined first-order operator whose square is the
Laplacian $\nabla^*\circ\nabla$ on spinor fields, up to lower order
terms. 
This first-order operator is in fact the Dirac operator $\dirac$ on
the spinor bundle. 
Thus, it is fairly natural to use the square of the Dirac operator on
the spin bundle as the Hamiltonian operator.  (Of course, all the various
operators we could use will have the same symbol and hence give the
same answers in the computations we do below.)

For any $\beta>0$, let us denote by $[0,\beta]^{1,1}$ the pre-image in
${\bf R}^{1,1}$ under the natural projection mapping of the interval
$[0,\beta]\subset {\bf R}$. Quantize as above.
The Hamiltonian operator associated with infinitessimal
time evolution  is given by 
$$H= \dirac^2$$
acting on the $L^2$-sections of spin bundle  $S(Y)$ thought of as a
super vector bundle over $Y$.
Let  $S^{1,1}$ denote the supercircle of length $\beta$
obtained by dividing ${\bf R}^{1,1}$ by a translation of distance
$\beta$.
The partition function for the resulting supersymmetric sigma model of
maps of $S^{1,1}$ into $Y$
is  the supertrace of the operator of time evolution from $t=0$ to
$t=\beta$; that is to say with 
\begin{equation}\label{strform}
{\rm tr_s}rm exp(-\beta H)={\rm tr_s}\left({\rm exp}(-\beta
\dirac^2)\right) 
\end{equation}
acting on the $L^2$-sections of the spin bundle $S(Y)$.
Of course, by the usual spectral analysis (as  alluded to int the
introduction in the case of the heat equation proof of the HirzeCruch
signature theorem), we see that the supertrace
of this operator is independent of 
$\beta>0$ and as $\beta\mapsto \infty$ it
approaches 
$${\rm dim}\left({\rm Ker}\ \dirac^2|\Gamma(S^+(Y))\right)-{\rm
dim}\left({\rm Ker}\ \dirac^2|\Gamma(S^-(Y))\right)$$
which is the index of
$$\dirac^+\colon \Gamma(S^+(Y))\to \Gamma(S^-(Y)).$$
Our strategy is to compute a perturbative expansion for this partition
function near $\beta=0$.  In fact, as we mentioned above, we will only
compute the constant term in this expansion.


\subsection{The Path Integral}

Now we wish to compute a power series expansion near $\beta=0$ for
the supertrace in Equation~\ref{strform}.
Of course, as was explaned in Fadde'ev's lectures, the basic
connection between the Lagrangian and Hamiltonian 
formulations is given by the Feynman-Kac formula.  For the current
application, the formula must be worked out in the context of spinor
fields instead of scalar fields.  This was done in detail in [Bismut].
In any event, the perturbative power series expansion that we shall
compute by path integrals over Feynman diagrams is in fact a
power series that is asymptotic
to the above supertrace. 
We shall ignore all terms which are positive powers of $\beta$, thus
computing the constant term in this power series.  Since we know that
the series is asymptotic to a constant function, this will suffice to
compute the constant term in the series.

Recall that we are considering the space of maps of the supercircle
$S^{1,1}$ of length $\beta$ into $ Y$.
The super Lagrangian  is
$${\cal L}(X)=-\int_{{\bf S}^{1,1}}dtd\theta^{-1}
\frac{1}{2}g_{I,J}(X)\frac{dX^I}{dt}DX^J,$$ 
The supermanifold of classical solutions is the
supermanifold  of maps of
$S^{1,1}\to Y$,
which are critical points for this Lagrangian.
The underlying geometric manifold of this super-manifold is the space
of closed geodesics in $Y$. The supermanifold $M$ of critical points that
minimize the Lagrangian is identified with  $\Pi TY\to Y$, an odd
vector bundle over the space of 
constants maps from $S^1$ to $Y$.
We shall do our perturbation expansion around $M$.
Other critical points give contributions which are exponentially small
in $\beta$. 
(In fact, one can prove directly
that the non-constant maps contribute zero to the final answer.)


By Equation~\ref{rewrite} we can re-write this path integral
as
$$\int_{x\in {\rm Maps}( S^1,Y);\psi\in \Gamma(x^*\Pi TY)}{\cal D}x{\cal
D}\psi\ {\rm exp}\left(-\int_0^\beta 
dt\left[\frac{g_{I,J}}{2}\left(\frac{dx^I}{dt}\frac{dx^J}{dt}-\psi^I
\frac{D\psi^J}{Dt}\right)\right]\right).$$

Rescale by setting $t'=\beta t$ and
$\psi'=\frac{1}{\sqrt{\beta}}\psi$.
Doing this and then rewriting the equation dropping the primes from
the notation yields
\begin{equation}\label{eqno}
\int_{x\in{\rm Maps}(S^1,Y);\psi\in \Pi x^*TY}{\cal D}x{\cal
D}\psi\ {\rm exp}\left(-\frac{1}{\beta}\int_0^1 
dt\left[\frac{g_{I,J}}{2}\left(\frac{dx^I}{dt}\frac{dx^J}{dt}
-\psi^I\frac{D\psi^J}{Dt}\right)\right]\right).
\end{equation}
This change of variables on the circle introduces a change in the
measure of integration by a formally infinite constant. The
regularization procedure will determine a finite value for this
constant which depends only on the dimension of $Y$.  In fact it is of
the form $C^{2n}$ for some universal constant. (Recall that the
dimension of $Y$ is $2n$.) We shall not try to
explicitly evaluate this constant  
nor others like it which arise later in the argument, though it is
possible to do so.  Thus, our final answer must be taken up to a
multiplicative constant of the form $C^{{\rm dim} Y}$.


As we have already remarked a minimum for the Lagrangian is a pair
$(x_0,\psi_0)$  where $x_0\in Y$ is regarded as the constant map of
$S^1$ to $Y$ and 
$\psi_0(t)$ is a constant section of $\Pi T(Y)_{x_0}$.
We expand to first order around this point by taking $x=x_0+a$
where $a(t)\in TY_{x_0}$ satisfies
\begin{equation}\label{ave1}
\int_0^1a(t)=0
\end{equation}
 and
$\psi=\psi_0+\eta$ where $\eta(t)\in \Pi TY|_{x_0}$ satisfies
\begin{equation}\label{ave2}
\int_0^1\eta dt=0.
\end{equation}
 These variations describe the normal bundle to
M at $(x_0,\psi_0)$.
To project this parametrization of the normal bundle  to a submanifold
in the space of maps transverse to $M$ we consider
$$x(s,t)={\rm exp}_{x_0}(s a(t))$$
as a map of $S^1\times (-\epsilon,\epsilon)$ into $Y$ and 
$$\psi(s,t)=P_{(s,t)}\left(\psi_0+\eta(t)\right)$$
where $P_{s,t}$ denotes parallel translation in $\Pi TY$ from $x_0$
along the path $x(s',t)$ as $s'$ runs from $0$ to $s$.
Thus, $\psi(s,t)\in \Pi TY_{x(s,t)}$, so that
$x(s,t)+\theta\psi(s,t)$ is a map from $S^{1,1}$ to $Y$. 

Since we are at a critical point, the first-order variation of the
Lagranian with respect to $a$ and $\eta$ is zero.
We are interested in the quadratic term, which is all that is needed
to compute the constant term in the $\beta$-expansion of
the path integral in Equation~\ref{eqno}.
To compute the quadratic term with respect to $a$ and $\eta$ 
in the Lagrangian we take the second 
derivative at a minimum $(x_0,\psi_0)$.
We find that the quadratic term
is
$$I=-\frac{1}{2\beta}\int_0^1dt\left[|\dot a|^2+\langle
\eta,\dot\eta\rangle-\frac{1}{2}\psi_0^I\psi_0^JR_{IJKL}a^K\dot
a^L\right].$$
As noted above the higher order terms in $a$ and $\eta$ in the action
will contribute to positive powers of 
$\beta$ in the asymptotic expansion.
In this expression the term $\dot \eta$ is the ordinary derivative of
$\eta$ with respect to $t$, which makes sense because for all $t$, the
image $\eta(t)$ is
contained in the vector space $\Pi TY_{x_0}$.
The term $R_{IJKL}$ is the curvature of $Y$.
Notice that in this integral there is no mixing in the quadratic terms
between the $a$'s and the $\eta$'s. 
Thus, $I$ is a sum  $I_1+I_2$
where
$$I_1(x,\psi)=-\frac{1}{2\beta}\int_0^1|\dot a|^2-\frac{1}{2} \langle
a,\psi_0^I\psi_0^J R_{I,J}\dot a\rangle $$
and
$$I_2=-\frac{1}{2\beta}\int_0^1\langle \eta,\dot \eta\rangle.$$
Hence,  the path integral
is a product 
$$\int_a{\cal D}a\ {\rm exp}\left(-I_1\right)\cdot
\int_\eta{\cal D}\eta\  
{\rm exp}\left(-I_2 \right).$$
Also notice that the $\eta$ integral
$$\int_\eta{\cal D}\eta{\rm exp}\left(-\frac{1}{2\beta}\langle
\eta,\dot\eta\rangle\right) $$
 is independent of $x_0,a$ and
$\psi$, since this integral depends only on $TY_{x_0}$ with its inner
product and the isomorphism class of this finite dimensional positive
definite inner product space depends only on the dimension of $Y$.
This path integral is purely quadratic and is identified
with the Pfaffian of $\frac{d}{dt}$ acting on the Hilbert space
of maps $\eta$. 
This is a constant which depends only on the dimension.
In fact it is of the form $C^{2n}$ for some universal constant $C$.
Notice also that in the regularization that we are performing, these
purely quadratic integrals are independent of $\beta$.

Thus, ignoring such multiplicative constants, to compute the constant term
of the expansion in $\beta$, we must compute
$$\int_a{\cal D}a\ {\exp}\left(-\frac{1}{2}\int_0^1|\dot
a|^2-\frac{1}{2}\langle a,\psi_0^I\psi_0^JR_{I,J}\dot a\rangle\right)$$
By integrating by parts from $0$ to $1$ we can re-write the integrand
of the functional integral as the exponential of
$$-\frac{1}{2}\int_0^1\langle D_B(a),a\rangle$$
where the propagator $D_B$ depends on $(x_0,\psi_0)$ and is given by
$$D_B(x_0,\psi_0)=-\frac{d^2}{dt^2}-\frac{1}{2}
\psi_0^I\psi_0^JR_{I,J}\frac{d}{dt}.$$ 
Thus, modulo $\beta$,  the path
integral over the normal  space to 
the super manifold $M$ of minima at the point $(x_0,\psi_0)$
of $I_a$ gives
$$\frac{1}{\sqrt{{\rm det}_a\ {D_B(x_0,\psi_0)}}}$$
where the determinant is computed over the space of $a$ for fixed
$(x_0,\psi_0)$ and is regularized using the $\zeta$-function
regularization. 
The final formula for the path integral is then
\begin{equation}\label{formula}
\int{\cal D}x{\cal D}\psi\ {\rm exp}(-I_a-I_\eta)\equiv
C^{2n}\int_{\Pi TY}
\frac{1}{\sqrt{{\rm det}\ D_B(x,\psi)}} \pmod\beta.
\end{equation}
Here $C$ is a universal constant (explicitly calculable given enough
patience). 

It is convenient to realize that this integral over the  supermanifold
$\Pi TY$ can be re-interpreted 
as an ordinary integral of a top dimensional form
over the underlying manifold $Y$.
To see this note  that, as explaned in Berstein's lectures,  the
$C^\infty$-functions on $\Pi TY$ are simply the 
$C^\infty$-sections of the bundle $\wedge^*(T^*Y)$ over $Y$ with the
usual ${\bf Z}/2{\bf Z}$-grading; that is to
say the $C^\infty$-functions on $\Pi TY$ are exactly the
$C^\infty$-differential forms on $Y$ with the ${\bf Z}/2{\bf
Z}$-grading being the degree modulo two.
Let $f$ be such a $C^\infty$-function and let $\omega$ be the
corresponding differential form on $Y$.
We have  an orientation on $Y$.
For a function $f\in C^\infty(\Pi TY)$ the integral
$$\int_{\Pi TY}f$$
is simply the integral over the fundamental cycle of $Y$ of the 
the corresponding form $\omega$.
The point is ${\rm Ber}(TY)$ and ${\rm Ber}(T^*Y)$ are naturally dual
line bundles so that their product is canonically trivial.



Thus, we have the path
integral expression for the index of the Dirac operator: 
$${\rm index}(\dirac^+)=\left(C'\right)^n\int_Y
\frac{1}{\sqrt{{\rm det}\ D_B(x,\psi)}},$$
where the determinant is computed using the $\zeta$-function
regularization, where $C'$ is a universal constant, and, as discussed
above, the $\psi$-dependence 
turns the integrand into a differential form (with compnents of
various degrees).  


\subsection{Evaluation of the Determinant}

Our next task is to compute this determinant as a differential form.
Let us fix first a element $g\in so(TY_{x_0})$ and consider 
the operator 
$$D_g=-\frac{d^2}{dt^2}-\frac{1}{2}g\frac{d}{dt}$$
acting on the space of maps $a\colon S^1\to TY_{x_0}$ which satisfy
the constraint given in Equationi~\ref{ave1}.
We can decompose $TY_{x_0}$ into two-dimensional spaces invariant
under $g$ such that on the $i^{th}$-space $g$ is given by the matrix 
$$\pmatrix{0 & x_i \cr -x_i & 0}$$
with respect to an orthonormal basis.
The determinant of $D_g$ will be the product over the blocks in this
decomposition  of the determinants associated to the
individual two-dimensional pieces associated with each block.
This allows us to reduce to the two-dimensional case.  (This step is
similar in spirit to the reduction step to riemann surfaces in some
other proofs of the same result.)
Fix a two-dimensional subspace $V$ of $TY_{x_0}$ on which $g$ has the form
$$g=\pmatrix{0 & x \cr -x & 0}.$$
We use the metric and  orientation of $TY_{x_0}$ to induce a complex
stucture on $V$. With respect to this structure $g$ is simply
multiplication by $-ix$.
We can decompose the complexification of the space of maps of $S^1$
into $V$ into  
two-dimensional spaces of the form  
${\rm exp}(2\pi ikt)a$ for $a\in V$.
Notice that by Condition~\ref{ave1}, this decomposition is over $k\not= 0$.
It is easy to see that the eigenvalues of $D_g$ on the two-dimensional
subspace indexed by $k$ are
$$(2\pi k)^2\pm (2\pi k)ix/2.$$
Thus,
$$\sqrt{{\rm det}D_g}=\prod_{\pm}\prod_{k\not= 0}\sqrt{\left((2\pi
k)^2\pm i(2\pi k)x/2\right)}.$$ 
Grouping together the four terms for each pair $k, -k$ we get
$$\sqrt{{\rm det}D_g}=\prod_{k=1}^\infty(2\pi k)^4\left(1+\left(\frac{x}{4\pi
k}\right)^2\right).$$ 
As in Problem 2 of Problem Set Five, the $\zeta$-function
regularization sets the 
infinite factor
$$\prod_{k=1}^\infty(2\pi k)^4$$
equal to  $C^2$ for some universal constant $C$.
Thus, the  result is that
$$\sqrt{{\rm det} D_g}=C^2\prod_{k=1}^\infty\left(1+\left(\frac{x}{4\pi 
k}\right)^2\right)=C^2\left(\frac{x/2}{{\rm sinh}(x/2)}\right)^{-1}.$$ 
This computation was done under the assumption that $TY$ was
two-dimensional.
In the general case when $TY$ is a sum of two-dimensional spaces of
the above form we find
$$\sqrt{{\rm det D_g}}=C^{2n}\prod_{i=1}^n\left(\frac{x_i/2}{{\rm
sinh}(x_i/2)}\right)^{-1} .$$
This power series can be re-written using the elementary symmetric
functions of the $x_i$. When we do so by the definition of $\hat A$ we
obtain 
$$({\rm det}D_g)^{-1/2}=C^{2n}\hat A(g).$$

Now let us return to ${\rm det}D_B^{-1/2}$ as a function on $\Pi TY$,
or as we have seen as a differential form on $Y$. Consider  the
function $\sum_{I,J}R_{I,J}(x_0)\psi_0^I\psi_0^J$ on $\Pi TY$ with values in
${\rm ad}(TY)$. As indicated above this function is identified with a
two-form on $Y$ with values in ${\rm ad}(TY)$. Not suprisingly 
under this identification the sum becomes the curvature two-form $R(Y)$
of $Y$.
Thus, 
$${\rm det}(D_B)^{-1/2}=C^{2n}\hat A(R(Y)).$$



We have now established that 
$${\rm index}\ (\dirac^+)=C^{2n}\int_Y
\hat A(R(Y))$$
where $C$ is a  universal constant.
Of course, the usual formula of Atiyah-Singer is
$${\rm index}\ \dirac^+)=\int_Y\hat A(R(Y)).$$



In this derivation we have been careless with universal constants
depending only on dimension
(infinite but regularized in some manner).
To evaluate the constant $C$ one can either follow through the
regularization procedure or simply evaluate it on one example where
the index is non-trivial.

\section{The Case of a Circle Action}

In this section $Y$ is a compact $2n$-dimensional spin manifold with
an isometry 
$g\colon Y\to Y$. We suppose that this automorphism lifts to
the bundle of spinors. We define the $g$-index of
the Dirac operator to be
$${\rm Tr}_{{\cal H}_+}g-{\rm Tr}_{{\cal H}_-}g$$
where ${\cal H}_{\pm}$ are the plus and minus harmonic spinors.
Equivalently, we can compute this  as 
$${\rm tr_s}\left(g\ {\rm exp}(-\beta H)\right),$$
where $H$ is the square of the Dirac operator.
As before, there is a path integral computation of a power series
expansion in $\beta$ 
aysmptotic to this supertrace at $\beta=0$.
The constant term in this series computes
this $g$-index.
In fact, this procedure works for any isometry $g$ which lifts to the
spin bundle, (see the first problem in the final exam for the Fall
Semester).  But here we shall suppose that $g$ lies in a connected,
compact 
group of isometries, which is equivalent to assuming that it lies in a
circle of symmetries.

So we consider an action $S^1\times Y\to Y$ by isometries. At the
expense of doubling the circle, this action
automatically lifts to the bundle of spinors.
We take a Killing vector field $V$ for this action (the vector field
generating the action), and denote by ${\cal L}_V$ the Lie derivative
of this vector field on the spinors.  We consider 
$$F(\theta)={\rm tr_s}\left({\rm exp}(i\theta{\cal L}_V){\rm exp}(-\beta
H)\right)=\sum_{n\in {\bf Z}}{\rm exp}(in\theta){\rm ind}_n\dirac^+,$$  
where ${\rm ind}_n\dirac^+$ is the index of the Dirac operator from
subspace ${\cal H}_n^+$ of plus spinors on which the character of
$S^1$ is the 
$n^{th}$-power to the corresponding subspace ${\cal H}_n^-$ of the
minus spinors. 
In what follows we denote ${\rm ind}_n\dirac^+$ by $a_n$.

Our goal is to prove the theorem of Atiyah-Hirzebruch [AH];

\begin{theorem}
If $S^1\times Y\to Y$ is a non-trivial isometric circle action on a
spin manifold 
$Y$ then ${\rm index}_n(\dirac)^+=a_n=0$ for all $n$.
\end{theorem}

\subsection{Proof of Atiyah-Hirzebruch Theorem using the Hamiltonian
Formalism}

Instead of computing this via path integrals as we did in the case of
trivial action above, we introduce a Hamiltonian approach.  We perturb
the Dirac 
operator to an operator depending on a real parameter $t$ defined by
$$\dirac_t(\psi )=\dirac(\psi)+tV\cdot \psi$$
on spinors. (Here, $V\cdot\psi$ is Clifford
multiplication.)
It is a formally self-adjoint operator. 
The Hamiltonian is 
$$H_t=(\dirac_t)^2=(\dirac)^2+t^2|V|^2+t\left(2\nabla_V+{\rm
div}(V)+dV^*\cdot \psi\right),$$
where $V^*$ is the one-form dual under the metric to $V$ and the last
term is Clifford multiplication by $dV^*$.
Once again this is a formally self-adjoint operator.
Since the circle action perserves the metric, ${\rm div}(V)=0$.
Thus, we have
$$H_t=(\dirac_t)^2=(\dirac)^2+t^2|V|^2+t\left(2\nabla_V+dV^*\cdot
\psi \right).$$
We also have a formula for the Lie derivative ${\cal L}_V$ as follows 
$${\cal L}_V=\nabla_V+\frac{1}{4}dV^*\cdot(\ ),$$
so that we can re-write the Hamiltonian as
$$H_t= (\dirac_t)^2=(\dirac)^2+t^2|V|^2+t\left(2{\cal
L}_V+\frac{1}{2} dV^*\cdot(\ )\right).$$

Since ${\cal L}_V$ generates a circle action on the spin bundle, its
eigenvalues are integers.
We denote by ${\cal H}_n$ the eigenspace for ${\cal L}_V$ of
eigenvalue $n$ in the Hilbert space of all
$L^2$-sections of the spin bundle. 
The vector field $V$ generates a symmetry of $H$ and in fact of $H_t$
for every $t\in {\bf R}$.  Hence, ${\cal
L}_V$ commutes with $H_t$. The index $a_n$ we are trying compute is simply
the index of the restriction of the operator defined by  $H$ (or
$H_t$)  from the invariant space  
${\cal H}^+_n$ to the invariant subspace ${\cal H}^-_n$; or put 
another way, using self-adjointness of $H_t$ it is the dimension 
of the kernel of $H_t$ on ${\cal H}^+_n$ minus the
dimension of the kernel of $H_t$ on ${\cal H}^-_n$.
We shall prove the theorem by showing that if $tn\ge 0$ and $|t|>>1$ then
$H_t$ is positive on
${\cal H}_n$ and in particular has no kernel on this subspace.
It follows immediately that $a_n=0$.

For large $|t|$ the spectrum of $H_t$ restricted to the subspace
${\cal H}_n$ can be calculated in an asymptotic expansion in powers of
$1/|t|$ by expanding near the minima of the potential.
These minima are the zeros of $V$.
We shall work first in the case when the $S^1$-action has isolated
fixed points.  We consider the more general case briefly at the end of
this section.
Let us consider an isolated fixed point $\alpha\in Y$.  We decompose the
linearization of the $S^1$-action at this fixed point into a sum of
two-planes: 
$$TY_\alpha=\sum_jW_j$$
where the restriction of  $V$  on $W_j$ is given in orthonormal
coordinates $(x_{2j-1},x_{2j})$ by
$$k_j\left(x_{2j-1}\frac{\partial}{\partial
x_{2j}}-x_{2j}\frac{\partial}{\partial x_{2j-1}}\right), k_j\in{\bf Z}^+.$$ 
We use the orientation induced on $W_j$ by these coordinates.
The conditition that the $S^1$-action lifts to the spin bundle implies
that 
\begin{equation}\label{even}
\sum_jk_j\equiv 0\pmod 2.
\end{equation}


To leading approximation in $1/|t|$ the restriction of $H_t$ near
$\alpha$ is given by 
$$H_t^{\rm
loc}(\alpha)=-\sum_{j=1}^{2n}\frac{\partial ^2}{\partial x_j^2}+
\sum_{j=1}^nt^2k_j^2(|x_{2j-1}|^2+|x_{2j}|^2)-t\sum_{j=1}^nk_j\pmatrix{ 1
& 0 \cr 0 & -1}+2t{\cal L}_V. $$
Ignoring the last term and splitting along the two-dimensional
subspaces for the monent we consider the operator 
\begin{equation}\label{local}
H^{{\rm
loc},j}_t(\alpha)=-\frac{\partial ^2}{\partial
x_{2j-1}^2}-\frac{\partial ^2}{\partial x_{2j}^2}+ 
t^2k_j^2(|x_{2j-1}|^2+|x_{2j}|^2) -tk_j\pmatrix{1 & 0 \cr 0 & -1}, \ \
\ 1\le j\le n.  
\end{equation}
With these definitions we have
$$H_t^{\rm loc}(\alpha)=\sum_{j=1}^nH_t^{{\rm loc},j}(\alpha)+2t{\cal
L}_V.$$ 

Let us consider the spectrum of the harmonic oscillator which is the
sum of all but the last term in the expression for   $H_t^{{\rm
loc},j}(\alpha)$:
$$-\frac{\partial ^2}{\partial x_{2j-1}^2}-\frac{\partial ^2}{\partial
x_{2j}^2}+t^2k_j^2(|x_{2j-1}|^2+|x_{2j}|^2).$$ 
Its ground state $\Omega$ has eigenvalue $2|t||k_j|$, and its spectrum is
$$\{(4\ell+2)|t||k_j\}\,\ \ell=0,1,2,\cdots$$
with each eigenvalue being of multiplicity two.

The last term in Expansion~\ref{local} commutes with the harmonic
ocsillator and has eigenvalues $\pm 
tk_j$. Thus, it splits each of the two-dimensional eigenspaces into
two one-dimensional spaces with distinct eigenvalues.
Thus, the lowest eigenvalue for $H_t^{{\rm loc},j}(\alpha)$ is $|t||k_j|$, and
its spectrum is 
$$\{(2r+1)|t||k_j|\},\ r=0,1,2,\cdots.$$
Though we do not need it now, it will be important later to understand
how ${\cal L}_V$ acts on these eigenstates.
A direct computation in these local coordinates shows that
the eigenstate of $H_t^{{\rm loc},j}(\alpha)$ with eigenvalue
$(2r+1)k_j|t|$ is  an eigenstate for  ${\cal L}_V$ with eigenvalue 
$(r+1/2)k_j$.
Also notice that if $n>0$ then ${\rm Ker}H_t\cap {\cal H}_n$ consists
of plus spinors  and if $n<0$ it consists of minus spinors.

Summing up over the two-planes and using the definition of ${\cal
H}_n$, we see that the lowest eigenvalue for
$H_t^{\rm loc}(\alpha)$ on ${\cal H}_n$ is
$$\sum_j|t||k_j|+2tn .$$  
As $|t|\mapsto \infty$, any element of norm one in the kernel of
the global Hamiltonian operator must localize around the fixed
points. In fact it must converge at each fixed point an element in the
kernel of  $H_t^{\rm loc}(\alpha)$.
But we have already seen that for $tn\ge 0$ the local operators are
positive and have no kernel.
It follows that  if $t>>0$ and $tn\ge 0$ there is no kernel for the
restriction of $H_t$  to ${\cal H}_n$. 


This proves the Atiyah-Hirzebruch theorem stated above at least in the
case when the  circle action has isolated fixed points.
The case of non-isolated fixed points is briefly discussed at the end
of this section.

Notice that in proving this result for all $n$ we had to use
both $t>0$ and $t<0$. 
 When we get to 
the loop space setting in the next lecture, we will not be able to
take $t=0$ and hence we can not pass from $t>0$ to $t<0$.  The
consequence of this is that we will only establish that half the
$S^1$-indices  vanish.

\subsection{A local integral for the index of the Dirac operator}

Even though we have established the Atiyah-Hirzebruch theorem that the
$a_n$ vanish, it is still interesting to give a different computation
of the $a_n$ in terms of local data around the fixed points of the
circle action.
This computation is particularly relevant when we get to the case of
twisted Dirac operators where the equivariant indices do not all vanish.

The function $F(\theta)=\sum_na_ne^{in\theta}$ is computed as
supertrace of the circle 
action  on the kernel of the operator $H_t$ for any $t$.
The convergence of the operators $H_t$ to the sum of the local operators
$H_t^{\rm loc}$ for large $|t|$ implies that in fact we can do the
computation by using the kernels of the local operators at the fixed points.
Let us fix $t<0$ and compute the kernel of $H_t^{\rm
loc}(\alpha)$. Note that the
argument in the last subsection did not show that this kernel
vanished, only that the restriction 
of $H_t$ to the ${\cal H}_n$ with $tn\ge 0$ had no kernel.


As we have already seen, the spectrum of the operator $H_t^{{\rm
loc},j}(\alpha)$  
is $\{(2r+1)k_j||t|\},\ r=0,1,\cdots$, with the eigenspaces of
dimension one, generated by $\lambda_r$, say.
Furthermore,  the vector field $V$ is the infinitessimal generator of
an $S^1$-action on the 
state  $/lambda_r$ is given by 
$$e^{(r+1/2)k_j\theta}.$$
Thus, $V$ is the infinitessimal generator of an $S^1$-action whose character
on the sum of the states given by 
\begin{equation}\label{thetaform}
e^{k_ji\theta/2}+e^{3k_ji\theta/2}+\cdots
=\frac{e^{k_ji\theta/2}}{1-e^{-i\theta k_j}}= \frac{i}{2}\frac{1}{{\rm
sin}(k_j\theta/2)}. 
\end{equation}
Nex we consider the sum of all possible tensor products over $j$ of 
eigenspaces of the $H_t^{{\rm loc},j}$:
The  vector field $V$ is the infinitessimal generator of an $S^1$-action on
this sum of tensor products whose character is given by 
\begin{equation}\label{product}
F_\alpha(\theta)=\prod_{j=1}^n\frac{i}{2}\frac{1}{{\rm sin}(k_ji\theta/2)}.
\end{equation}
Notice also that since $t<0$, the computation above of the action of
${\cal L}_V$ on the eigenstates of the $H_t^{{\rm loc},j}$ immediately
implies taht  all these tensor product states are in the kernel of our
local operator $H_t^{\rm loc}(\alpha)$ and are plus spinors.
It is clear that this accounts for the entire kernel of the local
operator. 
Notice that even though the $k_j$ are not necessarily even, all the
exponents that appear in the above product expansion are of the form
$$\left((2r_1+1)k_1+\cdots+(2r_d+1)k_d\right)/theta/2,$$
and by the spin condition (Equation~\ref{even}) are integral multiples
of $\theta$.  Hence, $F_/alpha(\theta)$ is a character of an action of
the original $S^1$, not some covering of it.
It is the character of the action of the circle on  the kernel of
$H_t^{\rm loc}(\alpha)$. Thus, $F_\alpha(\theta)$ 
is the local contribution at $\alpha$ of the 
$S^1$-equivariant index.
 
Summing up over  the various isolated fixed
points  $\alpha$ gives
\begin{eqnarray*}
F(\theta) = {\rm tr_s}\left(e^{i\theta{\cal L}_V}e^{-\beta
H}\right)
& = & \left(\frac{i}{2}\right)^n\sum_{\alpha}
(-1)^{\sigma_\alpha}\prod_{j=1}^n\frac{1}{{\rm
sin}(k_{\alpha,j}\theta/2)}. 
\end{eqnarray*}
 where $\alpha$ ranges over the fixed points of the circle action
(still assumed to be isolated), and $\sigma_\alpha$ compares the
orientation on $TY_\alpha$ coming 
from the ambient orientation of $Y$ with the orientation induced from
the direct sum of the orientations on the various two-planes $V_{\alpha,j}$.
(The $k_{\alpha,j},\ j=1,\ldots n$ are the rotation numbers of the
circle action at the fixed point $\alpha$.)

Of course, in this case we have shown that this sum is in fact zero.
This  is a non-trivial statement about the nature of the $S^1$-action
at the fixed points.
(Recall that we are assuming in the discussion so far that the
$S^1$-action has only isolated fixed points.)




\subsection{Dirac Operators with Values in  a Bundle}


If we pass from the case considered above of the pure Dirac operator
to the case of a Dirac operator twisted by a bundle with a connection,
then the vanishing theorem of Atiyah-Hirzebruch does not generalize.
Nevertheless, the second argument computing the equivariant index in
terms of local data at the fixed points does generalize. It produces
the Atiyah-Bott formula [AB].

Let $\xi\to Y$ be a vector bundle over a compact, even dimensional
spin manifold. We consider the Dirac operator twisted by this bundle.
The local analysis around the fixed points $\alpha$
involves the two-plane bundles
$W_{\alpha, j}$ at $\alpha$ as well as the restriction  $\xi|_{\alpha}$. 
The formula is (assuming that the $S^1$-action has only isolated fixed
points) 
$$F_\xi(\theta)={\rm ind}_{S^1}(\dirac\otimes
\xi)=\sum_{\alpha}(-1)^{\sigma_\alpha} \prod_j
\prod_{j=1}^n\frac{1}{{\rm
sin}(k_{\alpha,j}\theta/2)}{\rm Tr}_{\xi|\alpha}\left(e^{i\theta{\cal
L}_V}\right).$$ 
If $\xi$ is a trivial bundle then we are back to the case of the pure
Dirac operator and we know that the $S^1$-index $F_\xi(\theta)$ is
identically
$0$. There is an important generalization of this result for a certain
class of spin bundles $\xi\to Y$.
Let us consider the case when 
$\xi= S(Y)$, the spin bundle of $Y$.  While  it is no longer true that 
$F_\xi(\theta)=0$, it is true is that
$F_\xi(\theta)$ is a constant function.  The reason is that
$S(Y)\otimes S(Y)$
can be identified with the differential forms and hence the kernel
spaces are identified with the harmonic forms. Of course, ${\cal L}_V$
acts trivially on these spaces. 
One could ask a more general question here:  Let $P_Y\to Y$ be the
principal ${\rm Spin}(2n)$-bundle of $Y$ and let ${\cal R}$ be (finite
dimensional) representation of ${\rm Spin}(2n)$.  We define
$$\xi=P \times_{{\rm Spin}(2n)} {\cal R}$$
and consider the character-valued index function $F_\xi(\theta)$ as
above.
The question is:
For which $W$ constructed in this manner is the function $F_\xi(\theta)$
equal to a constant? We shall see in the next lecture that there is an
infinite series of such representations ${\cal R}$ with this
property. By constructing enough examples Landweber and Stong [LS]
showed that the infinite series that we construct includes all the
representations with this property.

\subsection{The Case of Non-isolated Fixed Points}

Let us finish this section with a brief discussion of the case when
the $S^1$-action does not have isolated fixed points.  Suppose that
the fixed point set is $\coprod_rM_r$, with the $M_r$ being
connected. Again we consider the 
deformation
$$\dirac_t(\ )=\dirac(\ )+tV\cdot(\ ).$$
Exactly as before, taking $|t|\mapsto \infty$ localizes the low
eigenstates of $H_t$ around the fixed 
components $M_r$ of the circle action. The local Hamiltonian
$H_t^{{\rm loc},M_r}$ at the 
fixed component $M_r$ has harmonic ocsillators  in the normal
direction, analogous to the one encountered when the fixed points were
isolated, and is the square of the Dirac operator along $M_r$.  
The decomposition of $TY_\alpha$ into two-dimensional spaces  
$V_{\alpha,j}$, which occurred in the
case of isolated fixed points,  is replaced in this case by a
decomposition of the normal bundle $V_r$ to $M_r$ as a  direct sum of
sub-bundles $V_{r,j}\to M_r$ of dimension $2d_j$, say, on which the 
circle action is given by the character which is raising to the
$k_j^{th}$-power. We orient the $V_{r,j}$ so that $k_j>0$, hence inducing
an orientation on $V_r$ and consequently on $M_r$.  (If $M_r$ is a
point, then this orientation differs from the natural one on the point
by the sign $(-1)^{\sigma_\alpha}$ encountered before.)
The argument given above generalizes easily to this case to show that
for $tn\ge 0$ there is no kernel for $H^{{\rm loc},M_r}_t$ on the
subspace ${\cal 
H}_n$. 
Thus, we see in this case as well, that $H_t$ has no kernel on ${\cal H}_n$
for $|t|>>1$ and $tn\ge 0$, establishing the Atiyah-Hirzebruch theorem
in the case of general circle actions.


One can also generalize the second argument to produce a formula in
terms of fixed point data for the equivariant index.
Once again it suffices to consider $|t|>>1$ and to find the kernel of
$H^{{\rm loc},M_r}_t$.
Let us suppose first, for the purposes of illustration, that the component
$M_r$ is of codimension two and is a spin submanifold.
Let $k$ be the character of the circle action in the normal direction.
Of course, our spin hypotheses imply that $k\equiv 0\pmod 2$.
For $t<0$ we find, exactly as before,  one state in the kernel of the
normal operator restricted to ${\cal H}_n$ for $n=(2s+1)k/2\
,s=0,1,2,\ldots$. 
Each of these states varies with the point of $M_r$
and hence forms a complex  bundle over $M_r$.
We then have  line bundles $L_1,L_3,\ldots$ with
$L_{2s+1}$ having $S^1$-character $(2s+1)k/2$.
 Let us relate the $L_{2s+1}$ to the normal bundle $V_r$ (which is a
complex line bundle).
Since the eigenstate $\varphi_s$ with character $(2s+1)k/2$ is
homogeneous of degree $2s+1$ in the spin bundle associated to $V_r$,
it follows that $L_{2s+1}=
\sqrt{V_r}^{\otimes (2s+1)}$, the $2s+1$
tensor power of the square root of the normal bundle of $M_r$. 
(The fact that $M_r$ is spin implies that $V_r$ has a square root and
even picks out one, so that these bundles are
well-defined.) 
Thus,  the spaces of vacua in the normal direction fit together to
form  a sum of 
$S^1$-equivariant complex line bundles.
The local contribution around this component of the fixed point set is
given by a sum like the one given by Equation~\ref{thetaform} with
$ik\theta/2$ replaced by $(x+ik\theta)/2$ where $x=c_1(V_r)$.
Of course, along the fixed point set $M_r$, the operator $H_t^{{\rm
loc},M_r}$ is the square of the Dirac operator.
The usual index theorem for the Dirac operator then computes the local
contribution at this component $M_r$ to the equivariant index to be
$$\int_{M_r}\hat A(M_r)\frac{1}{2}\frac{1}{{\rm
sinh}((x+ik\theta)/2)}.$$ 


If $M_r$ is codimension-two but not spin, then two things breakdown in
the above argument:  There are no spin bundles on $M_r$ and hence  no
Dirac operator on $M_r$,  and  we cannot form the square root line
bundle to $V_r$.  Nevertheless, the description of the kernel states
in the normal direction is still valid locally.  What happens is that
the obstruction to defining the spin bundles and Dirac operator
globally on $M_r$ is exactly cancelled by the obstruction to forming
the square root line bundle to $V_r$.  Thus, the local descriptions
fit together to define an operator which is locally described as a
Dirac operator with values in a line bundle but is globally twisted.
The index theorem still 
applies in this twisted situation and gives the same formula.
(Though of course, $x/2$ is only a rational cohomology class.)

More generally, when the normal bundle $V_r$ to $M_r$ splits as a sum
of complex line bundles invariant under the $S^1$-action, the space of
vacua is a sum of terms which are tensor products of the above type,
and the normal contribution is a  product 
of terms of the above type:
$$\prod_j\frac{e^{(x_j+ik_j\theta)/2}}{1-e^{x_j+ik_j\theta}}=\prod_j
\frac{1}{2{\rm sinh}((x_j+ik_j\theta)/2)},$$
where $x_j$ is the first Chern class of the $j^{th}$-line bundle and 
$k_j$ is the weight of the $S^1$-action on this line bundle.
In the more general case when $V_r$
does not split as a sum of complex line bundles invariant under the
$S^1$-action case we get
the expression for normal contribution is obtained using the splitting
principal. 
Suppose that the normal bundle $V_r$  to $M_r$ splits as $\oplus_j
V_{r,j}$ as described before.
Recall that $d_j={\rm dim}_{\bf C}(V_{r,j})$.
The resulting $S^1$-equivariant Chern character is
$$\hat A_\theta(V_r)=2^{-{\rm dim}(V_r)/2}\prod_{j}\prod_{\ell=1}^{d_j}
\frac{1}{{\rm sinh}((x_{\ell,j}+ik_j\theta)/2) }$$
with the $x_{\ell,j}$ being the formal roots whose elementary
symmetric functions are the Chern classes of $V_{r,j}$.
(Of course, as always, this expression must be re-written in terms of
the elementary symmetric functions in order to find the expression in
terms of the Chern classes of $V_{r,j}$.)
This describes the  bundle of
vacua over $M_r$ for the operator $H_t^{{\rm loc},M_r}$. 
Of course, so far we have been describing this operator  only in the
normal directions to $M_r$.  Along $M_r$ it is the square of the usual
Dirac operator.  Thus, the  index of $H_t^{{\rm loc}, M_r}$ is simply
that of the Dirac operator with values in the bundle of vacua
in the normal direction. (As before, if $M_r$ is not a spin
submanifold, then neither the spin bundles along $M_r$ nor the spin
bundle in the normal direction can be independently defined, but the
index formula works exactly as if they could.
The usual index theorem now evaluates the $S^1$-index of this operator  to
be
$$F_{M_r}(\theta)=\int_{M_r}\hat A(M_r)\hat A_{\theta}(V_r).$$

Adding up the above
expression for the local $S^1$-indices
over all the components of the fixed point set
yields the following  formula for the $S^1$-index of $H$:
\begin{equation}\label{genform}
F(\theta)=\sum_r\int_{M_r}\hat A(M_r)\hat A_{\theta}(V_r).
\end{equation}

This formula together with the  Atityah-Hirzebruch theorem that the
equivariant signature vanishes can then be viewed as giving
non-trivial information about the nature of the fixed points
submanifolds of the $S^1$-action.


More generally, if we are considering a Dirac operator coupled to a
connection on an $S^1$-vector bundle $\xi\to Y$ then we replace ${\cal
V}_r$ in Equation~\ref{genform} by 
$$\left(\oplus_j(V_{r,j}\right)\otimes\left(\oplus_be^{i\theta b}
\xi_{r,b}\right)$$  
where
$$\xi|M_r=\oplus_b\xi_{r,b}$$
and the character of the circle on $\xi_{r,b}$ is the $b^{th}$-power. 
With this substitution for ${\cal V}_r$, the same equation gives the
index of the coupled Dirac operator as a function on $S^1$.


\section{$\sigma$-models in $1+1$ dimensions}

Let us study in a rough qualitative way the case of maps of a
two-dimensional supermanifold into  a compact manifold $Y$ of
dimension $2n$.  We shall consider several contexts, but always with an
action given by the same formula.
First, let us
consider the super-manifold 
$${\rm Maps}({\bf R}^{2,2},Y),$$
with the action given by 
$${\cal L}=\int_{{\bf
R}^{2,2}}dudv(d\theta_+)^{-1}(d\theta_-)^{-1}g_{I,J}D_+X^ID_-X^J$$
where the even coordinates on ${\bf R}^{2,2}$ are $(u,v)$, the odd
coordinates are $\theta_+,\theta_-$, and
$$D_+=\frac{\partial}{\partial\theta_+}-\theta_+\frac{\partial}{\partial
u}$$
$$D_-=\frac{\partial}{\partial\theta_-}-\theta_-\frac{\partial}{\partial
v}$$
are vector fields on ${\bf R}^{2,2}$.
Notice that  a map of $X\colon {\bf R}^{2,2}\to Y$ 
can be written as
$$x+\psi_+\theta_++\psi_-\theta_-+F\theta_+\theta_-$$
wher $x$ is  a map
$x\colon {\bf R}^2\to Y$, $\psi_+,\psi_-$ are sections of the
pullback $x^*\Pi TY$ of the  tangent bundle of $Y$, considered as an
odd bundle and a section $F$ of $x^*TY$ considered as an even bundle.
For any superriemann surface $\Sigma$ (supermanifold of type
$(2,2)$) we have this action, but invariantly the $\psi_\pm$ are
sections  of
$S^\pm(\Sigma)\otimes x^*TY$. 
With this identification,
we can re-write the action as
$${\cal L}=\int_{{\bf
R}^2}dudv\left[g_{I,J}\partial_uX^I\partial_vX^J+i\psi_+\partial_v
\psi_++i\psi_-\partial_u\psi_-+
R_{I,J,K,L}\psi_+^I\psi_+^J\psi_-^K\psi_-^L+g_{I,J}F^IF^J\right].$$ 
Since the field $F$ enters the Lagrangian purely quadratically and
only algebraically (no derivatives of $F$ appear), it is easy to
integrate out the $F$ dependence.  When we do this gaussian
integration in $F$ we are left with the constraint $F=0$ and the same
action (without the $|F|^2$-term to integrate over the space  of
fields $x,\psi_\pm$. 

We shall study to some degree this Lagranian formulation of this
supersymmetric $\sigma$-model, but we shall
also need a Hamiltonian formulation.
In the Hamiltonian formulation  we study 
a quantum theory of maps of the super cylinder ${\bf R}^{1}/S\times
{\bf R}^{1,2}$ into $Y$ where $S$ is translation by a fixed amount.
Equivalently, we consider the space of maps of  ${\bf R}^{1,2}$ into
the loop space ${\cal L}Y$ of $Y$.
Letting $t$ be the geometric variable in ${\bf R}^{1,2}$ and letting
$s$ be the loop variable in ${\cal L}Y$, we identify $t=u+v$ and
$s=u-v$. Under this identification
the action above becomes a flow equation in this context: it is not
the usual analogue of the geodesic flow equation for ${\bf R}^{1,2}$
in ${\cal L}Y$ 
since there are extra potential terms. Nevertheless, the space of
classical solutions is given by the initial conditions at $t=0$, where
as before $F=0$ is a constraint.  That
is to say, the space of classical solutions is identified with the odd
vector bundle $\Pi T{\cal L}Y\oplus \Pi T{\cal L}Y$ over $T{\cal L}Y$
which is two
copies of the tangent bundle to the loop space.
Quantizing this bundle yields  the super-Hilbert space of
sections of ${\rm Sym}^*\Pi T({\cal L}Y)$ which should be viewed as
the differential forms on ${\cal L}Y$.

There is one important feature we wish to discuss.  There is a
${\bf Z}/2{\bf Z}\times {\bf Z}/2{\bf Z}$-symmetry of the Lagrangian
given by  
$$\sigma\cdot\pmatrix{ \psi_+ \cr \psi_-}\mapsto \pmatrix{-\psi_+ \cr
\psi_-}$$
and
$$\tau\cdot \pmatrix{\psi_+ \cr \psi_-} \mapsto \pmatrix{\psi_+ \cr
-\psi_-}.$$
We wish to discuss whether these symmetries lift to the quantum field
theory. 





Before addressing this problem directly, 
let us consider the analogous question for the space of maps of ${\bf
R}^{1,1}$ into a finite, even dimensional  spin manifold $Z$.
Quantizing the odd vector bundle $\Pi TZ$ over an even dimensional
spin manifold $Z$ yields the Hilbert space of $L^2$-sections of the
spin bundle $S(Z)$, and the field $\psi$ (analogous to say $\psi_+$
above) acts by Clifford
multiplication.  The analogue of the symmetry $\sigma$ above is then
implemented  
on this Hilbert space by the involution $\epsilon$ whose $\pm
1$-eigenspaces are 
the sections of $S^\pm(Z)$.  By this we mean 
$${\rm Cl}(-\psi_+)=\epsilon\circ {\rm Cl}(\psi_+)\circ \epsilon,$$
where ${\rm Cl}(\chi)$ denotes Clifford multiplication by $\chi$.
Now in the case of maps of ${\bf R}^{1,2}\to Z$ one can implement both
of the symmetries $\sigma$ and $\tau$ using the two copies of the spin
bundle associated to the fields $\psi_\pm$. 
Thus, it is clear that in this finite dimensional analogue the
symmetries $\sigma$ and $\tau$ lift to the quantum level when $Z$ is
an even dimensional spin manifold.

Let consider the case of mapping ${\bf R}^{1,2}$ into an finite, even
dimensional manifold $Z$, no longer assumed to be spin.
Even though the manifold is not spin, as we indicated above, we can
quantize the direct sum of the two copies 
of $\Pi TZ$ resulting in the differential forms on $Z$ with their
mod $2$ grading. (Of course, if $Z$ is spin we got the space of
$L^2$-sections of the tensor product of $S(Z)\otimes S(Z)$, but this
is naturally identified with the space of $L^2$ differential forms on $Z$.)
With this quantization, it turns out that, provided that $Z$ is
oriented, it is still the case that  the symmetries $\sigma$ and
$\tau$ lift to the realization.  They are given by 
$$(-1)^{\rm deg}=\sigma\tau$$
and 
$$*=\sigma,$$
where $*$ is the Hodge $*$-operator multiplied by a factor of $i^{\rm
deg}$ in order to make it an involution.
Thus, in this finite dimensional analogue, the ${\bf Z}/2{\bf Z}\times
{\bf Z}/2{\bf Z}$-symmetry group lifts to the quantum theory of maps
of ${\bf R}^{1,2}$ into $Z$ as long as $Z$ is orientable.

We claim that the answer in the infinite dimensional case is the same,
with a fairly natural definition of the notion of ${\cal L}Y$ being
orientable. 
We shall examine this question via path integrals considering the
theory of all maps of super riemann surfaces $\Sigma$ (supermanifolds
of type $(2,2)$) into $Y$.
Since $\sigma$ and $\tau$ are symmetries of the Lagrangian, the
question of whether they lift is
the same as asking whether the symmetries are symmetries of the
formal measure of integration ${\cal D}X{\cal D}\psi_+{\cal D}\psi_-$.
It turns out that just as in the finite dimensional case  $\sigma\tau$
always lifts to a symmetry of the quantum theory.
Let us consider $\sigma$. 
First, we claim that $\sigma$ is a symmetry of the measure of integration
if and only if ${\rm dim}\ {\rm Ker}{\cal D}_+\equiv 0\pmod 2$
for all
maps $x\colon \Sigma\to Y$ for any riemann surface $\Sigma$.
The reason is that  $\sigma$ clearly preserves the ${\cal D}X{\cal
D}_-$ and $\sigma$ will be a symmetry of ${\cal D}_+$ provided that
the space of sections of the pullback bundle $S^+(\Sigma)\otimes x^*TY$
is even dimensional.
The idea is that $\sigma$ is acting by $-1$ on these fields and thus
is preserving the orthogonal form. The    issue is whether formally
$\sigma$ preserves the orientation.  (In finite dimensions $-1$ is a
symmetry of the integration measure if and only if it is
orientation-preserving which is equivalent to the statement that the
space is even-dimenional.)
Of course, in our context the space is  infinite dimensiona, so in
order to determine its dimension we have to regularize 
it. The way this is done is the following.  We have a 
skew-symmetric pairing on this bundle of sections given by 
$$\langle \chi_1,\chi_2\rangle=\int_\Sigma \langle
\chi_1,{\cal D_+}(\chi_2)\rangle.$$
Here ${\cal D}_+$ is the dirac operator on $S^+(\Sigma)$ coupled with
covariant differentiation on the pullback
$x^*TY$. 
This pairing has a finite dimensional null space which is exactly the
kernel of the Dirac operator ${\cal D}_+$.
Given a skew-pairing $A$ on a finite dimensional
vector space $V$ we know that ${\rm dim}(V)\equiv {\rm dim}({\rm
Ker}(A))\pmod 2$.
In the infinite dimensional theory that we are studying here, this is
true for the pairing given above.  Thus, we see that for the dimension
of the space of sections of $S^+(\Sigma)\otimes x^*(TY)$  modulo two
we use the dimension of the kernel of ${\cal D}_+$.
Since $TY$  real this coupled Dirac operator is skew-adjoint.
Its spectrum is purely imaginary and symmetric about zero.
Thus, its algebraic spectral flow is zero, and the dimension of its
kernel modulo two is a topological invariant (unchanged under
deformation of the operators through ellipic skew-adjoint operators).
In fact, there is a mod two index theorem for the dimension of the
kernel.  Since $TY$ is even 
dimensional, the mod two index theorem identifies the dimension of the
kernel with $w_2(x^*TY)$.
Thus, we see that $\sigma$ will lift to an involution on the quantum
field theory for all maps of all superriemann surfaces $\Sigma$ into
$Y$ if and only if $w_2(TY)=0$. 

At least in the case when $\pi_1(Y)=\{1\}$,
it is natural to view the condition that $w_2(Y)=0$  as
the statement that the loop space ${\cal L}(Y)$ is orientable.  
(Of course, in analogy with finite dimensions, we do not expect to
implement the Hodge star operator unless the manifold in question is
orientable. Or put another way, we are attemping to use this theory of
maps of ${\bf R}^{1,2}$ into ${\cal L}Y$ to compute equivariant
signatures of the tautological circle action on ${\cal L}Y$.  In order
for these signatures to be defined, we must assume that ${\cal L}Y$ is
orientable.) 
Let us explain how $w_2(Y)=0$ should be viewed as equivalent to the
orientability of ${\cal L}Y$.
The tangent space of ${\cal L}(Y)$ at a map $x\colon S^1\to Y$ is the
space of sections of $x^*(TY)$. This vector bundle has a natural skew
pairing on it given by 
$$\omega_x(\alpha,\beta)=\langle
\alpha,\beta\rangle=\int_0^1(\alpha,\nabla_\theta\beta)d\theta$$
where the inner product on the right-hand-side is induced from the
metric on $Y$. It is clear that the radical of this pairing is the
space of covariantly constant sections of $x^*(TY)$.  In particular,
the radical of this pairing is always finite-dimensional.
As before with the coupled Dirac operator, it is easy to see that the
spectrum of the skew-adjoint operator associated to  this 
pairing is discrete with all eigenvalues of finite multiplicity.
Furthermore,
the eigenvalues are purely imaginary and symmetric with repect to the
origin. 
Given a closed path $x_t$ in ${\cal L}(Y)$, then a local orientation
of ${\cal L}(Y)$ at the initial point $x_0$ will be preserved around
the loop if and only if the spectral flow of the family $T_{x_t}$ from
negative to positive is even.  (By skew symmetry the full spectral
flow is algebraically zero: here we are looking at half of it -- the
half from negative to positive.)
This spectral flow modulo two is easily identified with $w_2(Y)$
evaluated on map of the torus 
$S^1\times S^1$ into $Y$ given by $(t,s)\mapsto x_t(s)$.  Thus,
provided that $Y$ is simply connected, we have that $w_2(Y)=0$ if and
only if ${\cal L}Y$ is orientable if and only if the kernel of ${\cal
D}_+$ is even dimensional for every map of a superriemann surface into
$Y$. 





If $w_2(Y)\not= 0$ this doesn't mean that we can not perform path
integrals for maps $X$ with $X^*w_2(Y)\not=0$. It simply means that
the results of such path integrals will be odd under $\sigma$, as the
measure is. To get something non-zero  we need  an
operator which is a product of  local operators
${\cal O}_1(x_1)\cdots {\cal O}_n(x_n), \ x_i\in \Sigma$, which is
$\sigma\tau$ 
invariant but $\sigma$ anti-invariant. Then we can calculate the
expectation value
$$\langle {\cal O}_1(x_1)\cdots{\cal O}_n(x_n)\rangle.$$
The symmetry $\sigma$ tells us that we get zero contribution at the
minima (constant maps) and 
the non-zero contributions come only from maps $x\colon \Sigma\to Y$ with
the property that $x^*(w_2(Y))\not= 0$.
Fix a homotopy class of maps with this property.
Of course, since the maps must be homotopically non-trivial, we see
that
$$\int_\Sigma|dx|^2$$
is bounded away from zero. Furthermore, the  maps minimizing this
integral are exactly the harmonic maps in the given homotopy class.
That is to say we need to find minimal area surfaces $S\subset Y$
and then take holomorphic maps
$\Sigma \to S\subset Y$.
We consider $\Sigma={\bf C}P^1$, or better delete a point and consider
$\Sigma ={\bf R}^2$.  Thus we are considering maps
$X\colon {\bf R}^2\to Y$ which converge at infinity to a constant map.
The expectation value
$$\langle {\cal O}_1(x_1)\cdots{\cal O}_n(x_n)\rangle$$
will be of the order ${\rm
exp}(-tM)$ where $M$ is the energy of the harmonic map.
(Notice that we have not normalized this expectation value by dividing
by the partition function $Z$.  If we were to do that we must take the
partition function which is the sum over all maps $x\colon \Sigma\to
Y$ not just those that pullback $w_2$ non-trivially.  The leading
order term in this partition function will come from the constant
maps.)
The critical points of the Lagrangian will not be isolated:
  Even with the
boundary condition at infinity of ${\bf R}^2$ there are four obvious
parameters for 
such maps: rotation, translation and scaling of ${\bf R}^2$. 
There may well be others in addition to these universal ones.
Thus, we
will be computing an expression of the form
$$\int_{\cal M}({\rm perturbation\ expansion)}$$
where ${\cal M}$ is the instanton moduli space, i.e., the space of
harmonic maps in the given homotopy class.
This time the propagator is more complicated and the linearization
around the instanton moduli space gives only an asymptotic
expansion.


\end{document}


An element in the kernel of $H^{\rm loc}_t(\alpha)$ will a tensor
product over $j$ of eigenstates of $H_t^{{\rm loc},j}(\alpha)$.>>>>>>>>>>>>>>>
also lies in ${\cal H}_n$ with $2nt$ being minus the sum of the eigenvalues
Consider the formal sum
$$e^{k_ji\theta/2}+e^{3k_ji\theta/2}+\cdots=\frac{e^{k_ji\theta/2}}
{1-e^{-i\theta k_j}}. $$





