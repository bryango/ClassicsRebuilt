%From: Lisa C Jeffrey <jeffrey@math.ias.edu>
%Date: Tue, 22 Oct 1996 09:45:23 -0400

\documentstyle[12pt]{article}
\input amssym.def
\input amssym.tex
\newcommand{\Tr}{\,{\rm Tr}\,}
\newcommand{\tr}{{\rm tr}\,}

%L. Jeffrey preamble, 5 April 1996

\newcommand{\nc}{\newcommand}

\nc{\isq}{{i}}
\newcommand{\colvec}[2]{\left  ( \begin{array}{cc} #1  \\
     #2  \end{array} \right ) }

%%%%%\newcommand{\Tr}{\,{\rm Tr}\,}
\newcommand{\End}{\,{\rm End}\,}
\newcommand{\Hom}{\,{\rm Hom}\,}

\newcommand{\Ker}{ \,{\rm Ker} \,}

\newcommand{\bla}{\phantom{bbbbb}}
\newcommand{\onebl}{\phantom{a} }
\newcommand{\eqdef}{\;\: {\stackrel{ {\rm def} }{=} } \;\:}
\newcommand{\sign}{\: {\rm sign}\: }
\newcommand{\sgn}{ \:{\rm sgn}\:}
\newcommand{\half}{ {\frac{1}{2} } }
\newcommand{\vol}{ \,{\rm vol}\, }


% define abbreviations for most common commands
%

\newcommand{\beq}{\begin{equation}}
\newcommand{\eeq}{\end{equation}}
\newcommand{\beqst}{\begin{equation*}}
\newcommand{\eeqst}{\end{equation*}}
\newcommand{\barr}{\begin{array}}
\newcommand{\earr}{\end{array}}
\newcommand{\beqar}{\begin{eqnarray}}
\newcommand{\eeqar}{\end{eqnarray}}
\newtheorem{theorem}{Theorem}[section]
%\newtheorem{conjecture}{Conjecture}
\newtheorem{corollary}[theorem]{Corollary}
%\newtheorem{problem}{Problem}
\newtheorem{lemma}[theorem]{Lemma}
\newtheorem{prop}[theorem]{Proposition}
\newtheorem{definition}[theorem]{Definition}
\newtheorem{remit}[theorem]{Remark}
\newtheorem{conjecture}[theorem]{Conjecture}

\newtheorem{example}[theorem]{Example}

\newcommand{\matr}[4]{\left \lbrack \begin{array}{cc} #1 & #2 \\
     #3 & #4 \end{array} \right \rbrack}



\newenvironment{rem}{\begin{remit}\rm}{\end{remit}}




% black board bold face
%note \AA is already defined!
\newcommand{\aff}{{ \Bbb A }}
\newcommand{\RR}{{{\bf  R }}}
\newcommand{\CC}{{{\bf  C }}}
\nc{\FF}{ {\Bbb F} } 
\newcommand{\ZZ}{{{\bf   Z }}}
\newcommand{\PP}{ {\Bbb P } }
\newcommand{\QQ}{{\Bbb Q }}
\newcommand{\UU}{{\Bbb U }}








%***************************




%


%


%Replace greek letters by their roman equivalents with \
%Slightly nonstandard:  theta is \t, tau is \ta, no omicron
\def\a{\alpha}
\def\b{\beta}
\def\g{\gamma}
\def\d{\delta}
\def\e{\epsilon}
\def\z{\zeta}
\def\h{\eta}
\def\t{\theta}
%\def\i{\iota}
\def\k{\kappa}
\def\l{\lambda}
\def\m{\mu}
\def\n{\nu}
\def\x{\xi}
\def\p{\pi}
\def\r{\rho}
\def\s{\sigma}
\def\ta{\tau}
\def\u{\upsilon}
\def\ph{\phi}
\def\c{\chi}
\def\ps{\psi}
\def\o{\omega}

\def\G{\Gamma}
\def\D{\Delta}
\def\T{\Theta}
\def\L{\Lambda}
\def\X{\Xi}
\def\P{\Pi}
\def\S{\Sigma}
\def\U{\Upsilon}
\def\Ph{\Phi}
\def\Ps{\Psi}
\def\O{\Omega}




% calligraphic letters
\newcommand{\calA}{{\mbox{$\cal A$}}}
\newcommand{\calB}{{\mbox{$\cal B$}}}
\newcommand{\calC}{{\mbox{$\cal C$}}}
\newcommand{\calD}{{\mbox{$\cal D$}}}
\newcommand{\calE}{{\mbox{$\cal E$}}}
\newcommand{\calF}{{\mbox{$\cal F$}}}
\newcommand{\calG}{{\mbox{$\cal G$}}}
\newcommand{\calH}{{\mbox{$\cal H$}}}
\newcommand{\calI}{{\mbox{$\cal I$}}}
\newcommand{\calJ}{{\mbox{$\cal J$}}}
\newcommand{\calK}{{\mbox{$\cal K$}}}
\newcommand{\calL}{{\mbox{$\cal L$}}}
\newcommand{\calM}{{\mbox{$\cal M$}}}
\newcommand{\calN}{{\mbox{$\cal N$}}}
\newcommand{\calO}{{\mbox{$\cal O$}}}
\newcommand{\calP}{{\mbox{$\cal P$}}}
\newcommand{\calQ}{{\mbox{$\cal Q$}}}
\newcommand{\calR}{{\mbox{$\cal R$}}}
\newcommand{\calS}{{\mbox{$\cal S$}}}
\newcommand{\calT}{{\mbox{$\cal T$}}}
\newcommand{\calU}{{\mbox{$\cal U$}}}
\newcommand{\calV}{{\mbox{$\cal V$}}}
\newcommand{\calW}{{\mbox{$\cal W$}}}
\newcommand{\calX}{{\mbox{$\cal X$}}}
\newcommand{\calY}{{\mbox{$\cal Y$}}}
\newcommand{\calZ}{{\mbox{$\cal Z$}}}

%% To load script letters:

\font\teneusm=eusm10  \font\seveneusm=eusm7 
\font\fiveeusm=eusm5 
\newfam\eusmfam 
\textfont\eusmfam=\teneusm 
\scriptfont\eusmfam=\seveneusm 
\scriptscriptfont\eusmfam=\fiveeusm 
\def\Scr#1{{\fam\eusmfam\relax#1}}

% Script letters
\newcommand{\ScrA}{{\Scr A}} \newcommand{\ScrB}{{\Scr B}}
\newcommand{\ScrC}{{\Scr C}} \newcommand{\ScrD}{{\Scr D}}
\newcommand{\ScrE}{{\Scr E}} \newcommand{\ScrF}{{\Scr F}}
\newcommand{\ScrG}{{\Scr G}} \newcommand{\ScrH}{{\Scr H}}
\newcommand{\ScrI}{{\Scr I}} \newcommand{\ScrJ}{{\Scr J}}
\newcommand{\ScrK}{{\Scr K}} \newcommand{\ScrL}{{\Scr L}}
\newcommand{\ScrM}{{\Scr M}} \newcommand{\ScrN}{{\Scr N}}
\newcommand{\ScrO}{{\Scr O}} \newcommand{\ScrP}{{\Scr P}}
\newcommand{\ScrQ}{{\Scr Q}} \newcommand{\ScrR}{{\Scr R}}
\newcommand{\ScrS}{{\Scr S}} \newcommand{\ScrT}{{\Scr T}}
\newcommand{\ScrU}{{\Scr U}} \newcommand{\ScrV}{{\Scr V}}
\newcommand{\ScrW}{{\Scr W}} \newcommand{\ScrX}{{\Scr X}}
\newcommand{\ScrY}{{\Scr Y}} \newcommand{\ScrZ}{{\Scr Z}}

%German (Faktur) letters

\newcommand{\grA}{{\frak A}}



\def\eps{\varepsilon}

\setlength{\textwidth}{6.5in}
\setlength{\textheight}{9.1in}
\setlength{\evensidemargin}{0in}
\setlength{\oddsidemargin}{0in}
\setlength{\topmargin}{-.75in}
\setlength{\parskip}{0.3\baselineskip}

%\renewcommand{\theequation}{\thesection.\arabic{equation}}
%\newcommand{\renorm}{{ \setcounter{equation}{0} }}

\newcommand{\Le}{{{\mathchoice{\,{\scriptstyle\le}\,}
  {\,{\scriptstyle\le}\,}
  {\,{\scriptscriptstyle\le}\,}{\,{\scriptscriptstyle\le}\,}}}}
\newcommand{\Ge}{{{\mathchoice{\,{\scriptstyle\ge}\,}
  {\,{\scriptstyle\ge}\,}
  {\,{\scriptscriptstyle\ge}\,}{\,{\scriptscriptstyle\ge}\,}}}}


%\renewcommand{\baselinestretch}{1.5}


%more newcommands
\nc{\bra}{  < }
\nc{\ket}{ > }
%\nc{\isq}{ { \sqrt{-1} } }
%\nc{\isq}{{ i }}
%\nc{\hbar}{{ h}}
\nc{\triang}{ { \bigtriangleup} }

\nc{\astar}{{ a^*}}
\nc{\hata}{ { \hat{a} }}
\nc{\hatastar}{{ \hat{\astar} }}
\nc{\normfac}{{\frac{1}{\sqrt{2 \omega} } }}

\begin{document}

\title{Lecture 2:\\
 The Harmonic Oscillator and Free Fields}
\author{Ludwig Faddeev}
\date{17  October 1996}

\maketitle
%\renorm

The treatment given here closely follows Sections 2.2 and 2.3 of 
\cite{FS}.

We revisit the functional integral formalism in 
holomorphic quantization used to describe the harmonic oscillator. 
We make use of operators $\hata$ and $\hatastar$ satisfying
$[\hata, \hatastar] = 1$. States are represented by 
analytic functions $\phi(\astar)$ of one complex variable $\astar$; operators
are represented by integral kernels $A(\astar'', a')$ depending on two 
{\em independent} complex variables $\hatastar''$ and $a'$. 
The operator whose integral kernel is $A$ may equivalently be represented
by its {\em normal symbol}
\beq K (\astar'', a') = e^{ - \astar'' a'} A(\astar'', a'). \eeq
We compose operators by writing the integral kernel of the 
composition of the operators with integral kernels $A$ and $B$ as
\beq 
AB (\astar'', a') = 
\int_{\RR^2} A(\astar'', a) e^{ - \astar a} B(\astar, a') 
\frac{d\astar da}{2 \pi i }. \eeq
The operator corresponding to the normal symbol is written as 
\beq {\cal K} = K(\hatastar, \hata) \eeq
where the ordering of the variables has been changed 
in such a way that all starred variables appear to the left of all 
unstarred variables. 

The evolution operator $U_\triang = e^{- \isq H \triang}$ was written 
$$U_\triang (\astar'', a' ) = 
\exp (-\astar'' a' - \isq h(\astar'', a') \triang). $$

\nc{\dphi}{\tilde{\phi}} 

We pause to point out that we have been regarding  $\phi(\astar) $
(which represents a state) 
as an analytic {\em function} of one complex variable. In geometric 
quantization it is actually  a {\em section} of a line 
bundle $L $ over the phase space $\RR^2$. We denote  the section of the 
line bundle as
\beq \dphi (\astar, a) = e^{- \half \astar a} \phi (\astar); \eeq
the object  $\dphi$ (the ``dressed'' analytic function) 
is a section of a line bundle $L$ over the phase space $\RR^2$.
The dressed kernel $\tilde{A} (\astar'', a') $ is a section of the 
pullback of two copies of $L$ over $\RR^2 \times \RR^2$: 
\beq \tilde{A} (\astar'', a') = 
e^{ - \half \astar'' a'' } A(\astar'', a') e^{ - \half \astar' a'}. \eeq
Dressed kernels are composed by integrating over $\RR^2$. 
One may also recover the ordinary kernel from the dressed kernel by 
integration:
\beq A(\astar'', a') = 
\int e^{ \astar'' a - \half \astar a} \tilde{A} (\astar, a | b^*, b) 
e^{ - \half b^* b} e^{ b^* a'} \frac{d\astar da}{2 \pi \isq} 
\frac{d b^* db} {2 \pi \isq}. \eeq
The time evolution operator becomes
\beq \label{e:timeev}
U(\astar'', a') = \int e^{ \astar'' a_N - \half a_N^* a_N} 
e^{ a_N^* a_{N-1} - a_{N-1}^* a_N + \dots - a_1^* a_1 + a_1^* a_0} \times \eeq
$$ e^{ - i [h(a_N^*, a_{N-1}) + \dots h(a_1^*, a_0)]\triang } 
\prod_{k = 0}^N  e^{ - \half a_0^* a_0 + a_0^* a'}\frac{da_k^* da_k}{2 \pi i}
$$
$$
 = \int \exp \Bigl  \{ \astar'' a(t'') - \half \astar (t'') a(t'') 
+ i \int_{t'}^{t''}\left [
 \frac{\dot{a}^* a - \astar \dot{a} }{2 i } - h(\astar, a) 
\right  \} dt $$
\beq \label{explint}
 - \half \astar(t') a(t') + \astar (t') a' \Bigr  \} 
\prod_{t' \le t \le t''} \frac{d \astar (t) da(t) } {2 \pi i }. \eeq
This formula is better than (41) in Lecture I.
All integrations here are done over a real plane $R^2$,
in which $a^*(+)$ and $a(+)$ are complex coordinates.
If one takes the explicit Hamiltonian $h (\astar, a) = \omega \astar a, $
this integral is Gaussian: we may compute it by ``completing the square'', 
or equivalently observing that the integrand is an inhomogeneous 
quadratic and so the integral may be obtained by replacing the value 
of the integrand by its value at the (complex) critical point of the
integrand. Explicitly, if $A$ is a matrix with determinant $1$ we have
\beq \int \exp \{ \half (Az, z) + (b,z) \} dz= e^{ - \half (A^{-1} b, b) }
. \eeq
The critical points of the integrand (\ref{explint}) are the 
solutions of the equation of motion 
$$\dot{a}^* - i \omega \astar = 0, ~~ \astar (t'') = \astar'' $$
$$ \dot{a} + i \omega a = 0, ~~ a(t') = a'. $$
We obtain as in Lecture 1 
\beq \label{uodef} U_0 (\astar'', a' ; t'' - t') = 
\exp (\astar'' a' e^{i \omega (t'' - t')} ), \eeq
which implies 
$$U_0 (t'') \phi(\astar) = \phi(e^{ - i \omega t'} \astar). $$






Now let us consider a more complicated example: we perturb the 
Hamiltonian by adding a term involving an  external force $\eta (t)$ (which
should be thought of as a real valued function),  which depends on the 
time $t$. If we introduce 
$$q = \frac{1}{\sqrt{2 \omega} } (\astar + a), $$
the new hamiltonian takes the form 
$$H = \omega \astar a + \eta(t) q(t) $$
and is explicitly time dependent.
The equations of motion for $a$ and $\astar$ acquire
corrections:
\beq \dot{a}^* - i \omega \astar - i \eta(t) = 0, \eeq
\beq \dot{a}^* + i \omega \astar + i \eta(t) = 0. \eeq
The time evolution operator now depends on $t''$ and $t'$ and not only 
on their difference: we obtain
$$ U(\astar, a; t'', t') = 
\exp  \Bigl \{  \astar a e^{ - i \omega (t'' - t') } 
- i \astar e^{ - i \omega t''} \int_{t'}^{t''} 
e^{ i \omega s} \frac{\eta(s)}{\sqrt{2\omega} } ds 
- i a e^{  i \omega t'} \int_{t'}^{t''} 
e^{- i \omega s} \frac{\eta(s)}{\sqrt{2\omega} } ds  $$
\beq \label{udef} - \int_{t = t'}^{t''} \frac{\eta(t) }{\sqrt{2\omega} }
\int_{s = t}^{t''} e^{ - i \omega t } e^{i \omega s }
\frac{\eta(s) }{\sqrt{2\omega} }   dtds \Bigr \}, \eeq
(where from now on the primes on the 
variables $a$ and $\astar$ will be suppressed).
We note that the expression for $U$ contains both terms linear
in $\eta$ and terms quadratic in $\eta$. 

Suppose $\eta = 0 $ for $t$ outside some interval 
$[\tau_1, \tau_2]$ 
(in other words the external force is turned on at time $\tau_1$ 
and turned off at time $\tau_2$). We would like to study how the motion
of the system deviates from its motion  in the absence  of an external force:
this is described by the normal 
symbol for the {\em transition operator} $T$, defined as follows.
\beq 
T(\astar, a) = \lim_{t'' \to \infty, ~t' \to - \infty}
U_0 (t'')^{-1} U(t'', t') U_0 (t'). \eeq
Recalling the expression for $U_0$ given in  (\ref{uodef}), the
transition operator is obtained from the expression (\ref{udef}) for 
$U$ by removing the initial term 
$\astar a e^{ -i \omega (t'' - t')} $ from the argument of the exponential,
and also removing the factors $e^{-i \omega t''} $ and 
$e^{i \omega t'}$ from the second and third terms in the argument of the 
exponential in (13).
Notice that the integration variable $s$ in (\ref{udef}) is subject to 
the constraint  $s < t$; this enables us to rewrite the formula for
the normal symbol of  $T$ 
in the form 
\beq \label{tdef}
T(\astar, a)_{\rm symb} = 
\exp \left \{ - i \int_{- \infty}^\infty q(t) \eta(t) dt 
+ \frac{i}{2} \int_{- \infty}^\infty \int_{- \infty}^\infty 
\frac{e^{-i \omega |t-s| } } {2i\omega} \eta(t) \eta(s) dt ds \right \}. \eeq
To obtain this formula we have symmetrized over $t$ and $s$. 

Here, 
\beq \label{e:16}
q(t) = \normfac (\astar e^{i \omega t} + a e^{-i \omega t} ) \eeq
is a solution of the free equation of motion 
$$\ddot{q} + \omega^2 q = 0,  $$
or $$D q = 0 . $$
The operator $D$ is defined by 
$$D = \frac{d^2}{dt^2} + \omega^2. $$
The kernel 
$$ G(t,s)= G(t-s) = \frac{e^{-i \omega |t-s|} }{2 i \omega} $$
is 
the {\em Green's function}  for the operator $D$, or the fundamental 
solution to the equation 
\beq  (\frac{d^2}{dt^2} + \omega^2) G(t,s) = \delta (t-s).\eeq
Since $G$ depends only on the difference $t-s$, we shall from now
on regard $G$ as a function of one variable $t$.
We may describe it in terms of its  Fourier transform as
\beq G(t) =\lim_{\eps \to 0+}  \frac{1}{2 \pi} \int_{- \infty}^\infty 
\frac{e^{ikt} }{- k^2 + \omega^2 - i \epsilon} dk, \eeq
which is often denoted as 
$$ G(t) =  \frac{1}{2 \pi} \int_{- \infty}^\infty 
\frac{e^{ikt} }{- k^2 + \omega^2 - i 0} dk. $$
The integral over $k$ should be completed to a contour integral 
and `$- i 0$' tells us how we have moved the poles in the integrand
(which were on the real axis). The concrete choice of Green's function
corresponds to the domain of definition $\calO$ 
of the operator $D$: $\calO$ consists of functions $f(t)$ which 
are asymptotic  to  $a e^{i \omega t}$  as
$t \to - \infty$ and to  $b e^{-i \omega t}$ as
$t \to \infty$ (where $a$ and $b$ are constants). 

We now pass from quantum mechanics to quantum field theory, replacing
$\RR$ by Minkowski  space $V$ of dimension $d + 1$
with signature $(+,-,\dots, -)$ (in other words $d$ space 
dimensions and one time dimension; sometimes this will 
be denoted as $V = V' \oplus \RR$, where we have
denoted $d$-dimensional space by $V'$.) 
We shall usually index the coordinates of $V$ by $\mu = 0, \dots, d$ while
the coordinates of space $V'$  are indexed by $j = 1, \dots, d$. 
Thus a point $x$ in $V$ is denoted $x = (x^0, \vec{x})$; its
coordinates are $x^\mu = (x^0, x^1, \dots, x^d)$. 

Let us list  the formulas 
from quantum mechanics which we shall generalize:
\beq \label{eq:1} h = \half p^2 + \half \omega^2 q^2 ~= \omega \astar a \eeq
\beq  \label{eq:2} \ddot{q} + \omega^2 q = 0. \eeq
\beq \label{eq:2a} [\hata, \hatastar] = 1 \eeq
\beq \label{eq:3} \phi(\astar) = \frac{1}{\sqrt{n!}} (\astar)^n , 
~ H \phi = n \omega \phi \eeq
\beq \label{eq:4} q = \frac{\astar + a}{\sqrt{2 \omega} } \eeq
\beq \label{eq:5} T (\astar, a) = \exp \left \{ 
- i \int q(t) \eta(t) dt + \frac{i}{2} \int G(t-s) \eta(t) \eta(s) dt ds 
\right \}. \eeq




The analogue of the coordinate $q(t) $  is a 
scalar field (denoted by $\varphi$) which depends on $x \in V$.
The field $\varphi$ satisfies the classical equation of motion, which 
is the {\em Klein-Gordon equation} 
\beq \label{kg} \square \varphi + m^2 \varphi = 0 , \eeq
where we have introduced 
$$ \square = \partial_\mu \partial^\mu = \partial_0^2 - \bigtriangleup. $$
To see the analogy with (\ref{eq:2}) we introduce the 
operator $\Omega^2 = - \bigtriangleup + m^2$, so that 
(\ref{kg}) becomes
$$ \ddot{\varphi} + \Omega^2 \varphi = 0 . $$

\nc{\vac}{{ \sl v}}
\nc{\basvec}[1]{\Phi_{#1} (\astar) } 
\nc{\basvecn}{\basvec{n} } 
\nc{\kn}{(\vec{k})_{n}} 

A free field $\varphi$ should be thought of as a collection of 
oscillators with different frequencies, which are 
eigenvalues of the operator $\Omega$:
$\Omega^2$ can be diagonalized by Fourier transforms and we get
for the eigenfrequencies
\beq \omega^2 = \vec{k}^2 + m^2, \eeq
or $$\omega(\vec{k}) = \sqrt{\vec{k}^2 + m^2}. $$
The space coordinate (or equivalently its Fourier  transform, the 
momentum coordinate $\vec{k}$) labels the oscillators. 
Quantization now is evident.
In the
holomorphic quantization
the basis elements, denoted by 
$\basvecn$, are given by 
\beq \basvecn = \prod_{j = 1}^n \astar (\vec{k}_1) \dots \astar (\vec{k}_n); \eeq
at times we shall use the notation $\kn$ to denote
the $n$-tuple of {\em unordered} points  $(\vec{k}_1, \dots, \vec{k}_n)$ in
$(V')^*$.
The Hilbert space corresponding to $n$ particles is the 
space of {\em symmetric} functions of $n$ momentum variables.
Proper states are given by 
\beq \int f(\kn) \astar (\vec{k}_1) \dots \astar(\vec{k}_n) d\vec{k}_1 
\dots d\vec{k}_n. \eeq
The Hamiltonian becomes
\beq H = \int \omega (\vec{k})
 \hatastar (\vec{k}) \hata(\vec{k}) d\vec{k}. \eeq
The state $\Phi(\kn) $ is a generalized eigenstate for the Hamiltonian 
\beq H \Phi(\kn) =  \left (
\sum_{j = 1}^n \omega (\vec{k}_j)\right )  \phi(\kn). \eeq
and the  momentum operator $\vec{P}$ :
\beq \vec{P} \Phi(\kn) = 
\left ( \sum_{j = 1}^n \vec{k}_j \right )  \Phi(\kn). \eeq
Thus we obtain a description of the spectrum in terms of particles.
In a one particle subspace, 
the components 
of the momentum 
operator form a complete set of commuting operators, and the 
Hamiltonian is a function $\omega (\vec{k})$ of them. 

In field theory one requires multiparticle states in order 
to obtain  scattering, and one must also allow the possibility
of creating and annihilating particles. A  Hilbert space accommodating
all possible numbers of particles and on which the creation and 
annihilation operators may act is the {\em Fock space}
$\calH_{\rm Fock} $, which should be thought of as a linear combination
of all the $n$-particle states, or as the result of tensoring together 
a collection of Hilbert spaces 
representing  one particle systems  and symmetrizing over the action
of the permutation group. 

If we introduce the variable $\pi(\vec{x} )$ which is canonically conjugate
to the field operator $\varphi(\vec{x})$, the free field Hamiltonian becomes
(compare with (\ref{eq:1})
\beq \label{hamfock}
H_0 = \half \int_{V'} \left ( \pi^2(\vec{x} ) 
+ (\nabla \varphi, \nabla \varphi) + m^2 \varphi^2 (\vec{x} ) 
\right ) d\vec{x}. \eeq
The initial data analogous to (\ref{eq:4}) is 
\beq \label{eq:4bis} 
\phi(\vec{x}) = \frac{1}{(2 \pi)^{d/2} }
\int \left ( \astar (\vec{k} ) 
e^{ - i (\vec{k}, \vec{x} ) }  + 
 a (\vec{k} ) 
e^{  i (\vec{k}, \vec{x} ) }  \right ) \frac{ d\vec{k}}{\sqrt{2 \omega} }. \eeq
We may  perturb the Hamiltonian $H_0$ by adding a term:
$$H = H_0 + 
\int\eta(\vec{x}, t) \varphi(\vec{x}, t) d\vec{x}. $$
The normal symbol of the corresponding
transition operator is then (by analogy with (\ref{eq:5}))
\beq \label{transopd} T_\eta(\astar(\vec{k}), a(\vec{k}) )
= \exp \left \{ - i \int_{V} \eta(x) \varphi(x) dx + 
\frac{i}{2} \int_{V \times V} 
G(x - y) \eta(x) \eta(y) dx dy \right \}, \eeq
where the Green's function satisfies 
\beq (\square + m^2) G(x,y)  = \delta^{d + 1} (x - y); \eeq
it is given in terms of its Fourier transform by 
\beq G(x) = \frac{1}{(2 \pi)^{d+1} } \int_{V^*} 
\frac{e^{ - i (k,x) } }{k^2 - m^2 + i 0 } dk. \eeq

The field operator $\varphi$ generalizing (\ref{e:16}) is 
\beq \varphi(\vec{x}, t) = 
\frac{1}{(2 \pi)^{d/2} } 
\int \left \{  \astar (\vec{k} ) e^{ - i (\vec{k}, \vec{x} )} e^{i \omega t} 
+ a (\vec{k} ) e^{  i (\vec{k}, \vec{x} )} e^{-i \omega t}  \right \} 
\frac{ d\vec{k}}{\sqrt{2 \omega} }. \eeq

\nc{\newv}{\calV}

Let us now consider what happens when we perturb the Hamiltonian by 
adding, not a term corresponding to an external force, but
a term corresponding to an interaction between particles. 
We suppose for example that 
\beq H = H_0 + \lambda \int_{V'} \newv(\varphi(\vec{x} ) )
d\vec{x}, \eeq 
where $\newv (\varphi)$ is a polynomial in $\varphi$ in which 
all terms are higher than quadratic; we could for example take
\beq \newv(\varphi) = \frac{\varphi^3}{3!}. \eeq
We shall expand in terms of powers of the {\em coupling constant}
$\lambda$, which is assumed  small. 
The term $\newv$ is independent of the time; it should 
be regarded as analogous to a potential. 

We shall compute the {\em scattering operator} or {\em S-matrix}
\beq \calS' = 
\lim_{t'' \to \infty, ~ t' \to - \infty} 
e^{i H_0 t''} e^{ - i H (t'' - t') } e^{- i H_0 t'}. \eeq
Repeating the previous  formalism we obtain
a  functional integral for the time evolution operator where the 
integrand contains  a factor
\beq \exp \left \{ 
- i \lambda \int \calV (\varphi(x) ) dx \right \}. \eeq
This factor is contained in  the integrand for the time evolution operator 
$U$ in the path integral which is the 
generalization to $d+1 $ dimensions of 
the integral given in (\ref{explint}).
%(\ref{udef}) and ultimately to the formula (\ref{tdef}) for the time 
%evolution operator. 
We now use the identity 
\beq \exp \left \{ -i \lambda \int \calV(\varphi(x) ) dx \right \} 
= \exp \left \{ -i \lambda \int \calV (\frac{1}{i} (\frac{\delta}{\delta 
\eta(x) } ) dx \right \} 
\exp \left \{ i \int \varphi(x) \eta(x) dx \right \}|_{\eta = 0 } \eeq
to enable us to use our formula (\ref{transopd}) 
for the transition operator to extract a formula for the $S$-matrix.

We replace integration of functions involving $\calV(\varphi(x))$ by
differentiation with respect to $\eta$ of the formula (\ref{tdef}) for the 
transition operator. The purpose of this device is to replace
integration (which involves the fields $\varphi$ in a nonlinear manner)
by differentiation with respect to the parameter $\eta(x)$, which
pairs linearly with $\varphi(x)$
The formula (34) for 
 $T_\eta$ may be thought of as a {\em generating functional}
for the S-matrix when one perturbs the Hamiltonian by adding
a potential term involving terms of higher than quadratic order in 
$\varphi$. 

We obtain
\beq \label{smat}
\calS'(\astar, a) = 
\exp \left \{ - i \lambda \int \calV (\frac{\delta}{\delta \eta(x)}) dx 
\right \} 
 \exp \left \{ i\int_V \eta(x) \varphi(x) dx + 
\frac{i}{2} \int_{V \times V} G(x-y) \eta(x) \eta (y) dx dy 
\right \} . \eeq
We are applying $\delta/\delta \eta(x) $ to the exponential of an 
expression which contains one term linear in $\eta$ and another
term quadratic in $\eta$. One application 
of $\delta/\delta \eta(x)$ 
produces terms of the form $\varphi(x)$ or of the form 
$\int_V G(x-y) \eta(y) dy. $ 
Differentiating twice with respect to $\eta$ produces 
factors of $G(x-y)$. 
We may represent the combinatorics of the resulting expression for the 
S-matrix diagrammatically: this is the formalism of {\em 
Feynman diagrams}.  To obtain the 
term of order $\lambda^N$ in the formula for the S-matrix one should 
proceed as follows.
 We restrict for the moment to the potential $\calV(\varphi) 
= \varphi^3/3!$  The factor $\calV(\varphi)$ is represented by a 
vertex with three edges emanating from it ({\em trivalent} vertex) , which
represent the three derivatives $\delta/\delta \eta(x)$ applied to the 
expression (\ref{transopd}). Each vertex 
carries a factor of the coupling constant $\lambda$, and the $j$-th
vertex is 
labelled by a coordinate $x_j \in V$. 
The combination $(\delta/\delta \eta)^2$ yields a 
term $G(x_j - x_k)$; diagrammatically this is 
represented by a line connecting the $j$-th and 
$k$-th vertices. (Here, $j$ need not be distinct from $k$, so 
the line could start and end at the same vertex.) 

Diagrammatically the term of order $\lambda^N$ in the 
asymptotic expansion of $\calS'$ is  obtained by the 
following procedure:

\begin{enumerate}
\item Draw all diagrams with $N$ trivalent vertices. Label the $j$-th 
vertex by the coordinate $x_j \in V$. 
\item A line connecting the $j$-th and $k$-th vertices ({\em internal} line)
carries a factor $G(x_j - x_k)$ (the {\em propagator}).
\item A line emanating from the $j$-th vertex but not connected to any
other vertex carries a factor $\varphi(x_j)$. 
\item The expression thus obtained must be 
multiplied by $(-i \lambda)^N/N!(3!)^N $ and integrated over 
the $x_1, \dots, x_N$. 
\end{enumerate}

We note that many of the diagrams thus obtained will differ from each other
only by permutations of the integration variables 
$x_1, \dots, x_N$ or of the edges leaving any given 
vertex; these permutations are compensated by the 
factor $1/(3!)^N N!$ One obtains the same answer by summing only over
one representative of each equivalence class under such diagrams and 
omitting the factor $1/(3!)^N N!$, provided one divides
by the automorphism group of the diagram (the group of relabellings
of the variables $x_1, \dots, x_N$ and permutations
of the lines leaving each vertex which leaves the diagram invariant). 

Of course if the potential $\lambda \varphi^3/3!$ were replaced by 
a different order polynomial in $\varphi$, for example 
$\lambda \varphi^4/4!$, one
would obtain an analogous formula involving vertices with 
a larger number of edges (in this case four)  emanating from them.

We note that one must include contributions from disconnected diagrams, 
and also from diagrams where an edge starts and ends at the same vertex.




















\begin{thebibliography}{99}
\bibitem{FS} L.D. Faddeev, A.A.  Slavnov, {\em Gauge Fields: An Introduction 
to Quantum Theory}, Addison-Wesley (Frontiers in Physics vol. 83), (second 
edition), 1991. 
\end{thebibliography}


\end{document}


