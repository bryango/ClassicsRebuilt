%From: Krzysztof M Gawedzki <bongo@math.ias.edu>
%Date: Fri, 25 Oct 1996 18:58:14 -0400

%Here is the version with corrections.


%%%%%%%%%%%%%%%%%%%%%%%%%%%%%%%%%%%%%%%%%%%%%%%%%%%%%%%%%%%%%%%%%%%
%%%%%%%%%%%%%%%%%%%%%%%%%%%%%%%%%%%%%%%%%%%%%%%%%%%%%%%%%%%%%%%%%%%
%%%%%%                                                     %%%%%%%%
%%%%%%             Conformal Field Theory                  %%%%%%%%
%%%%%%                                                     %%%%%%%%
%%%%%%                  Problem set 1                      %%%%%%%%
%%%%%%                                                     %%%%%%%%
%%%%%%                Krzysztof Gawedzki                   %%%%%%%%
%%%%%%                                                     %%%%%%%%
%%%%%%                   (latex file)                      %%%%%%%%
%%%%%%                                                     %%%%%%%%
%%%%%%%%%%%%%%%%%%%%%%%%%%%%%%%%%%%%%%%%%%%%%%%%%%%%%%%%%%%%%%%%%%%
%%%%%%%%%%%%%%%%%%%%%%%%%%%%%%%%%%%%%%%%%%%%%%%%%%%%%%%%%%%%%%%%%%%
\documentstyle[12pt]{article}
\setlength{\textwidth}{17.3cm}
\setlength{\textheight}{21.82cm}
\hoffset -20mm
\topmargin= -1cm
\raggedbottom
\raggedbottom
\newcommand{\be}{\begin{eqnarray}}
\newcommand{\en}{\end{eqnarray}\vs 0.5 cm}
\newcommand{\non}{\nonumber}
\newcommand{\no}{\noindent}
\newcommand{\vs}{\vskip}
\newcommand{\hs}{\hspace}
\newcommand{\e}{\'{e}}
\newcommand{\ef}{\`{e}}
\newcommand{\dd}{d\hspace{-0.16cm}{}^-}
\newcommand{\hh}{h\hspace{-0.23cm}{}^-}
\newcommand{\p}{\partial}
\newcommand{\pp}{-\hspace{-0.28truecm}-\hspace{-0.67truecm}o}
\newcommand{\un}{\underline}
\newcommand{\Nt}{{{\bf t}}}
\newcommand{\Nn}{{{\bf n}}}
\newcommand{\NR}{{{\bf R}}}
\newcommand{\NA}{{{\bf A}}}
\newcommand{\NP}{{{\bf P}}}
\newcommand{\NC}{{{\bf C}}}
\newcommand{\NT}{{{\bf T}}}
\newcommand{\NZ}{{{\bf Z}}}
\newcommand{\NQ}{{{\bf Q}}}
\newcommand{\NE}{{{\bf E}}}
\newcommand{\NH}{{{\bf H}}}
\newcommand{\NN}{{{\bf N}}}
\newcommand{\Nx}{{{\bf x}}}
\newcommand{\Nk}{{{\bf k}}}
\newcommand{\Nm}{{{\bf m}}}
\newcommand{\qq}{\begin{eqnarray}}
\newcommand{\de}{\bar\partial}
\newcommand{\da}{\partial}
\newcommand{\ee}{{\rm e}}
\newcommand{\Ker}{{\rm Ker}}
\newcommand{\qqq}{\end{eqnarray}}
\newcommand{\llambda}{\mbox{\boldmath $\lambda$}}
\newcommand{\aalpha}{\mbox{\boldmath $\alpha$}}
\newcommand{\xx}{\mbox{\boldmath $x$}}
\newcommand{\xxi}{\mbox{\boldmath $\xi$}}
\newcommand{\kk}{\mbox{\boldmath $k$}}
\newcommand{\tr}{\hbox{tr}}
\newcommand{\ch}{\hbox{ch}}
\newcommand{\ad}{\hbox{ad}}
\newcommand{\Lie}{\hbox{Lie}}
\newcommand{\w}{{\rm w}}
\newcommand{\R}{{\bf R}}
\newcommand{\CA}{{\cal A}}
\newcommand{\CB}{{\cal B}}
\newcommand{\CC}{{\cal C}}
\newcommand{\CD}{{\cal D}}
\newcommand{\CE}{{\cal E}}
\newcommand{\CF}{{\cal F}}
\newcommand{\CG}{{\cal G}}
\newcommand{\CH}{{\cal H}}
\newcommand{\CI}{{\cal I}}
\newcommand{\CJ}{{\cal J}}
\newcommand{\CK}{{\cal K}}
\newcommand{\CL}{{\cal L}}
\newcommand{\CM}{{\cal M}}
\newcommand{\CN}{{\cal N}}
\newcommand{\CO}{{\cal O}}
\newcommand{\CP}{{\cal P}}
\newcommand{\CQ}{{\cal Q}}
\newcommand{\CR}{{\cal R}}
\newcommand{\CS}{{\cal S}}
\newcommand{\CT}{{\cal T}}
\newcommand{\CU}{{\cal U}}
\newcommand{\CV}{{\cal V}}
\newcommand{\CX}{{\cal X}}
\newcommand{\CY}{{\cal Y}}
\newcommand{\CZ}{{\cal Z}}
\newcommand{\s}{\hspace{0.05cm}}
\newcommand{\m}{\hspace{0.025cm}}
\newcommand{\La}{\Lambda}
\newcommand{\la}{\lambda}
\newcommand{\hf}{{_1\over^2}}
\newcommand{\hslash}{{h\hspace{-0.23cm}^-}}
\newcommand{\Di}{{\slash\hs{-0.21cm}\partial}}
\pagestyle{plain}
\renewcommand{\baselinestretch}{0.3}
\begin{document}


\ 
\hspace*{12.8cm}Krzysztof Gaw\c{e}dzki


\vskip 0.8cm

\noindent{\large{\bf Problem Set for Lecture 1.\ \ 
Simple functional integrals}}
\vskip 1.3cm


\no PROBLEM 1. \ {\un{Feynman-Kac formula for harmonic oscillator}}
\vskip 0.4cm

\no The Feynman Kac formula in the one-dimensional
case ($d=0,\ \Sigma=[0,L]_{periodic}$) asserts for the 2-point 
function that, for $x_1\leq x_2\s$,
\qq
\left(\int\limits_{_{\CC_{per}([0,L])}}\hs{-0.2cm}
\phi(x_1)\s\m\phi(x_2)\s\s d\mu_G(\phi)\s=\right)\s
G(x_1,x_2)\ =\ {\tr\s\s\ee^{-x_1H}\s\varphi\s\ee^{\m(x_1-x_2)H}\s
\varphi\s\ee^{\m(x_2-L)H}\over\tr\s\s\ee^{-LH}}\s.
\label{1}
\qqq
In the normalizations from the lecture,
\s$G={_{2\pi}\over^\beta}\s(-{d^2\over dx^2}+m^2)^{-1}\s$ 
with \s${d^2\over dx^2}\s$ with periodic b.c. on $[0,L]$
and \s$H\s$ is the Hamiltonian of a harmonic oscillator,
\qq
H\s=\s-{_\pi\over^\beta}\s{_{d^2}\over^{d\varphi^2}}\s+\s{_
{\beta m^2}\over^{4\pi^2}}\s\varphi^2\s-\s{_m\over^2}\s=\s
m\s a^*a
\non
\qqq
where 
\qq
a^*\s=\s-\sqrt{{_\pi\over^{\beta m}}}\s{_{d}\over^{d\varphi}}
\s+\s\sqrt{{_{\beta m}\over^{4\pi}}}\s\varphi\s,\quad\ \ 
a=\s\sqrt{{_\pi\over^{\beta m}}}\s{_{d}\over^{d\varphi}}
\s+\s\sqrt{{_{\beta m}\over^{4\pi}}}\s\varphi\s,\quad\ \ [a,a^*]=1\s,
\non
\qqq
are operators in \s$L^2(\NR,d\varphi)$. Recall that the spectrum
of $H$ is \s$m\{0,1,\dots\}\s$.
\vskip 0.2cm

\no (a). Use the Fourier transform to write the left hand side.
Show that its limit $G_\infty(x_1,x_2)$ when $L\to\infty$ exists. 
\vskip 0.2cm

\no (b). Prove that for $x_1,\dots,x_n>0$ and complex numbers
$\lambda_1,\dots,\lambda_n$
\vskip 0.2cm
\qq
\sum\limits_{k,l=1}^n\bar\lambda_k\s\lambda_l \s\s G_\infty(-x_k,x_l)
\ \geq\ 0\s.
\non
\qqq

\no (c). What is the limit of the right hand side of Eq.\s\s(\ref{1})
when $L\to\infty$?
\vskip 0.2cm

\no (d). Show that both sides of Eq.\s\s(\ref{1}) coincide at $L=\infty$. 
Prove (b) using this result.
\vskip 0.2cm

\no (e). Prove the relation (\ref{1}) for finite $L$.
\vskip 1cm


\no PROBLEM 2. \ {\un{Zeta-regularized determinants}} (for calculators)
\vskip 0.4cm

\no (a). \ Using the identity \s$\la^{-s}=\Gamma(s)^{-1}\int_0^\infty
t^{s-1}\s\ee^{-\la\m t}\s dt\s$ show that \s$\zeta(0)=-{_1\over^2}$
where $\zeta$ is the Riemann zeta function \s$\zeta(s)=
\sum\limits_{n=1}^\infty n^{-s}\s$ (for \s${\rm Re}\m s>1$,
analytically continued elsewhere).
\vskip 0.2cm

\no (b). \ For $\tau=\tau_1+i\tau_2$ with $\tau_i$ real, $\tau_2>0$
show using the identity from (a) and the Poisson resummation
that for ${\rm Re}\m s$ sufficiently large
\qq
\sum\limits_{n=-\infty}^\infty|\tau+n|^{-2s}\ =\ 
{\sqrt{_\pi}\over^{\Gamma(s)}}\left(\sum\limits_{n\not=0}
\ee^{\m2\pi in\tau_1}\int_0^\infty t^{s-3/2}\s\m\ee^{-\tau_2^2 t
-\pi^2n^2/t}\s\s dt\ +\ \tau_2^{-2s+1}\s\Gamma(s-1/2)\right)\s.
\non
\qqq
Note that the right hand side is analytic in $s$ around $s=0$.
Using (easy) relations \s$\Gamma(s)^{-1}=s+\CO(s^2)\s,$\ \s$
\Gamma(-\m{_1\over^2})=-2\sqrt{\pi}\s$ and
\s$\int_0^\infty t^{-3/2}\s\ee^{-x(t+t^{-1})}\s dt\s=\s\sqrt{\pi}\s 
x^{-1/2}\s\ee^{-2x}\s$ obtain:
\qq
{_d\over^{ds}}\bigg|_{_{s=0}}\sum\limits_n|\tau+n|^{-2s}\ =\ -\s
\ln{|1-q|^2}\s-\s 2\pi\tau_2
\non
\qqq
with the standard notation \s$q\equiv\ee^{\m 2\pi i\tau}\s$.
\vskip 0.1cm

\no (c). \ By taking $\tau\to 0$ in the last formula show that
 \s$\zeta'(0)=-\m{_1\over^2}\s\ln(2\pi)\s$. 
\vskip 0.2cm

\no (d). \ Prove that for the periodic b.c. operator 
\s${{d^2}\over{dx^2}}\s$ on $[0,L]$ the zeta-regularized
determinant 
\qq
{\rm det}'(-\m{_\beta\over^{2\pi}}\s{_{d^2}\over^{dx^2}})\s=\s2\pi L^2\s.
\non
\qqq
\vskip 0.2cm

\no (c). \ Show that the spectrum of the Laplacian $\Delta_\tau$
on the torus $\NC/(\NZ+\tau\NZ)$ in the metric $|dz|^2$ is given 
by \s$\la_{m,n}=-({2\pi\over\tau_2})^2\s|\tau m+n|^2\s$ for $n,m\in\NZ$.
\vskip 0.2cm

\no (d). \ Proceeding as in (b) decompose
\qq
\sum\limits_{(m,n)\not=(0,0)}|\tau m+n|^{-2s}\s=\s
\sum\limits_{m\not=0,\ n}|\tau m+n|^{-2s}\s+\s 2\m\zeta(2s)\hs{3cm}\cr
=\s{_{\sqrt{\pi}}\over^{\Gamma(s)}}\left(\sum\limits_{m,n\not=0}
\ee^{\m 2\pi imn\tau_1}\int_0^\infty t^{s-3/2}\s
\m\ee^{-m^2\tau_2^2 t-\pi^2 n^2/t}\s\m dt\s+\s\sum\limits_{m\not=0}
m^{-2s+1}\s\tau_2^{-2s+1}\s\Gamma(s-1/2)\right)\s+\s 2\m\zeta(2s)
\non
\qqq
and show that (after analytic continuation)
\qq
\zeta_{-\Delta'_\tau}(s)\s\equiv\s
\sum\limits_{(m,n)\not=(0,0)}(-\la_{m,n})^{-s}\s=\s
-1\s-\s2s\s\ln{|\prod\limits_{m=1}^\infty(1-q^m)|^2}
\s-2s\s\ln{\tau_2}\s+\s{_1\over^3}\m\pi\s\tau_2\m s\s.
\non
\qqq
Infer that 
\qq
\zeta_{-\Delta'_\tau}(0)\s=\s-1\s,\quad\quad\zeta'_{-\Delta'_\tau}(0)=
-2\s\ln{|\prod\limits_{m=1}^\infty(1-q^m)|^2}
\s-2\s\ln{\tau_2}\s+\s{_1\over^3}\m\pi\s\tau_2\s,&\label{2}
\non
\qqq
and that
\qq
{\det}'(-\Delta_\tau)\s=\s\tau_2^2\s|\eta(\tau)|^4\label{3}
\non
\qqq
where the Dedekind eta function \s$\eta(\tau)=q^{\m 1/24}
\m\prod\limits_{n=1}^\infty(1-q^n)\s$.
\vskip 1cm


\no PROBLEM 3. \ {\un{Rationality of the free field with values
in $S^1$}}
\vskip 0.4cm

\no The toroidal partition function
of the free field compactified on the circle
of radius squared $r^2$ is 
\qq
Z_{r^2}(\tau)\s=\s\sum\limits_{n,m\in\NZ}\ee^{\m\pi i\tau\m(mr+n/r)^2/2\s
-\s\pi i\bar\tau\m(mr-n/r)^2/2}\s\m|\eta(\tau)|^{-2}\s.
\label{tpf}
\qqq
Show that for rational $r^2$ it is a finite sesqui-linear combination
of functions analytic in $\tau$.
\vskip 1cm


\no PROBLEM 4. {\un{Toroidal compactification to a complex torus}}
\vskip 0.4cm

\no Let $(\Sigma,\gamma)$ be a Riemann surface.
Let $R=R_1+i R_2$ and $T_1+ i T_2$ be two numbers
in the upper half plane. Consider on the torus
\s$(\NR/2\pi\NZ)^2\s$ the complex structure given
by the coordinate \s$\psi=\phi^1+T\phi^2\s$, the metric
\s$g={R_2\over T_2}|d\psi|^2\s$ and the 2-form \s$\omega=
i{R_1\over T_2}d\psi\wedge d\bar\psi\s$. \s Show that the partition 
functions $Z_{R,T}$ of the free field on $\Sigma$ with values in 
$(\NR/2\pi\NZ)^2$ with action \s$S(\phi)={1\over 4\pi}
(\|d\phi\|_{L^2}^2+\m i\int_\Sigma\phi^*\omega)\s$
satisfy the identities:
\qq
Z_{R,T}\s=\s Z_{R+1,T}\s=\s Z_{R, T+1}\s
=\s Z_{R,-T^{-1}}\s=\s Z_{-R^{-1},-T^{-1}}\s.
\non
\qqq
additionally to the mirror symmetry \s$Z_{R,T}=Z_{T,R}\s$ shown
in the lecture.
\vskip 1cm

\no PROBLEM 9. \ {\un{Conformal anomaly}}
\vskip 0.4cm

\no Using the short time expansion of the heat kernel of the Laplacian
$-\Delta$ on a Riemann surface \s$(\Sigma,\gamma)\s$
\qq
\ee^{\m t\Delta}(x,x)\s=\s{_1\over^{4\pi t}}
\s+\s{_1\over^{12\pi}}\s r(x)\s+\s\CO(t)
\non
\qqq
where \s$r\s$ stands for the scalar curvature, prove that
under the infinitesimal local rescaling of the metric
\s$\gamma\mapsto\ee^\sigma\gamma\s$
\qq
{_{\delta}\over^{\delta\sigma(x)}}\bigg|_{_{\sigma=0}}
\ln{\left({_{{\det}'(-\Delta)}\over^{{\rm vol}_\Sigma}}\right)}
\ =\ -\m{{_1\over^{12\pi}}}\s\m r(x)\s.
\label{coa}
\qqq
Show that this infinitesimal relation is equivalent
to the global one
\qq
{_{{\det}'(-\Delta)}\over^{{\rm vol}_\Sigma}}\bigg|_{_{\ee^\sigma\gamma}}
\s =\ 
\ee^{-{1\over 48\pi}\s(\m\Vert d\sigma\Vert^2_{L^2}
\s+\s4\int_\Sigma\sigma\m r\s dv\m)}\ 
{_{{\det}'(-\Delta)}\over^{{\rm vol}_\Sigma}}\bigg|_{_{\gamma}}\s.
\non
\qqq
\vskip 1cm

\no PROBLEM 10. \ {\un{Massless Fermions on Riemann surface}}
\vskip 0.4cm

\no Let \s$(\Sigma,\gamma)\s$ be a Riemann surface.
Spin structure on \s$\Sigma\s$ may be identified with 
the square root \s$L\s$ of the canonical 
bundle $K={T^*}^{1,0}(\Sigma)\s$. \s A Dirac spinor
\s$\Psi=(\psi,\tilde\psi)\s$ is an element of 
\s$\Gamma(L\oplus\bar L)\s$ where $\bar L$ is the bundle
complex conjugate to $L$. The conjugate spinor is 
\s$\bar\Psi=(\tilde\chi,\chi)\in\Gamma(\bar L\oplus L)\s$
and in the euclidean Dirac theory it should be treated as an 
independent field
($\chi=\psi,\ \tilde\chi=\tilde\psi\s$ for Majorana fermions).
Denote by \s$\bar\da_L\s$ the $\bar\da$ operator of $L$
and by \s$\da_{\bar L}\s$ its complex conjugate
which may be naturally identified with \s$\de_L^{\s*}\s$.
The action is a function on the odd vector space
\s$\Pi(\Gamma(L\oplus\bar L)\oplus\Gamma(\bar L\oplus L))\s$:
\qq
S(\Psi,\bar\Psi)\s=\s{_1\over^\pi}\int_{_\Sigma}(\chi\bar\da_{\bar L}
\psi\s+\s\tilde\chi\da_L\tilde\psi)\s
\non
\qqq
(note that the integrand is naturally a 2-form).
Partition functions of the Dirac fermions are given by
the formal Berezin integral
\qq
Z_L\s=\s\int\ee^{-S(\Psi,\bar\Psi)}\s D\bar\Psi\s D\Psi
\s=\s\det(\da_{\bar L})\s\det(\bar\da_L)\s=\s\det(\bar\da_L^{\s*}
\bar\da_L)\s.
\non
\qqq
The last determinant may be 
zeta-regularized giving a precise sense to the partition function 
\s$Z_L\s$ of the Dirac field on $\Sigma$.
\vskip 0.2cm

On the elliptic curve \s$\NC/(\NZ+\tau\NZ)\s$ with $\tau$ 
in the upper half-plane,
the canonical bundle \s$K\s$ may be trivialized by the section \s$dz\s$
and spin structures correspond to the choice of periodic or
anti-periodic boundary conditions under 
\s$z\to z+1\s$ and \s$z\to z+\tau\s$:
\qq
L\s=\s pp,\ pa,\ ap,\ aa\s.
\non
\qqq
\vskip 0.1cm

\no (a). \ Show that the eigenvalues of \s$\bar\da_L^{\s*}
\bar\da_L\s$ are
\qq
\lambda_{m,n}\s=\s({_\pi\over^{\tau_2}})^2\m|\tau m+n|^2
\non
\qqq
with
\qq
\hbox to 5cm{\hspace{2cm}$m\in\NZ\s,$\hfill}
\hbox to 2cm{$n\in\NZ$\hfill}
\quad\quad{\rm for}\quad L=pp\s,\cr
\hbox to 5cm{\hspace{2cm}$m\in\NZ\s,$\hfill}
\hbox to 2cm{$n\in\NZ+{_1\over^2}$\hfill}
\quad\quad{\rm for}\quad L=pa\s,\cr
\hbox to 5cm{\hspace{2cm}$m\in\NZ+{_1\over^2}\s,$\hfill}
\hbox to 2cm{$n\in\NZ$\hfill}
\quad\quad{\rm for}\quad L=ap\s,\cr
\hbox to 5cm{\hspace{2cm}$m\in\NZ+{_1\over^2}\s,$\hfill}
\hbox to 2cm{$n\in\NZ+{_1\over^2}$\hfill}
\quad\quad{\rm for}\quad L=aa\s.
\non
\qqq
\vskip 0.1cm

\no (b). \ Infer that 
\qq
Z_{pp}(\tau)\s=\s 0\s.
\non
\qqq
\vskip 0.1cm

\no (c). \ Show that
\qq
\zeta_{\bar\da_{pa}^{\s*}\de_{pa}}(s)\s=\s2^{2s}\s(\m
\zeta_{-\Delta'_{2\tau}}(s)
\s-\s\zeta_{-\Delta'_\tau}(s)\m)\s.
\non
\qqq
Infer from Eq.\s\s(\ref{2}) that
\qq
Z_{pa}(\tau)\s=\s4\m|q^{\m 1/24}\prod\limits_{n=1}^\infty
(1+q^n)|^4\s.
\non
\qqq
In the Hilbert space picture 
\qq
Z_{pa}(\tau)\s=\s\tr_{\CH_R\otimes\tilde\CH_R}
\s\s q^{L_0-1/24}\bar q^{\tilde L_0-1/24}\s.
\qqq
The ``Ramond sector'' Hilbert space 
is \s$\CH_R\otimes\tilde\CH_R\s$ with
\qq
\CH_R\s=\s\NC^2\otimes\left(
\wedge(\mathop{\oplus}\limits_{n=1}^\infty\NC)
\right)^{\otimes 2}
\qqq
and \s$\tilde\CH_R\s$ is another copy of \s$\CH_R\s$.
\s$L_0$ acts in the first copy. It has eigenvalue 
\s${1\over 8}\s$ on \s$\NC^2\s$ (the ``Ramond ground states'')
and the occupied $n^{\s\rm th}$ mode in the fermionic Fock space
adds \s$n\s$ to it.
\vskip 0.1cm

\no The periodic partition function is
\qq
Z_{pp}(\tau)\s=\s\tr_{\CH_R\otimes\tilde\CH_R}
\s\s(-1)^{F+\tilde F}\s\s q^{L_0-1/24}\bar q^{\tilde L_0-1/24}
\ \equiv\ {\rm str}_{\CH_R\otimes\tilde\CH_R}
\s\s\s\s q^{L_0-1/24}\bar q^{\tilde L_0-1/24}
\qqq
where \s$(1,0),(0,1)\in\NC^2\s$ correspond to the eigenvalues
\s$+1,-1\s$ of \s$(-1)^F\s$
and each occupied fermionic Fock space mode
adds \s$1\s$ to \s$F\s$. \s$Z_{pp}(\tau)\s$ vanishes since
modes with odd and even Fermi numbers are paired.
\vskip 0.2cm

\no (d). \ Show that
\qq
\zeta_{\bar\da_{ap}^{\s*}\de_{ap}}(s)\s=\s2^{4s}\m
\zeta_{-\Delta'_{\tau/2}}(s)
\s-\s2^{2s}\m\zeta_{-\Delta'_\tau}(s)\ \ \quad
{\rm and}\quad\ \ 
Z_{ap}(\tau)\s=\s|q^{-1/48}\prod\limits_{n=0}^\infty
(1-q^{n+1/2})|^4\s.
\non
\qqq
The Hilbert space interpretation is
\qq
Z_{ap}(\tau)\s={\rm str}_{\CH_{NS}\otimes\tilde\CH_{NS}}
\s\s\s\s q^{L_0-1/24}\bar q^{\tilde L_0-1/24}
\qqq
where the ``Neveu-Schwarz sector'' Hilbert space is
\qq
\CH_{NS}\s=\s\left(
\wedge(\mathop{\oplus}\limits_{n=0}^\infty\NC)
\right)^{\otimes 2}\s.
\qqq
The ``Neveu-Schwarz ground state'' has eigenvalue zero of $L_0$
and the \s$n^{\s\rm th}\s$ occupied zero mode contributes 
\s$(n+{1\over2})\s$ to it. The fermion number of the NS-ground state 
vanishes and each occupied fermionic mode adds \s$1\s$ to it.
\vskip 0.2cm

\no (e). \ Show that
\qq
\zeta_{\bar\da_{aa}^{\s*}\de_{aa}}(s)\s=\s2^{2s}\m
\zeta_{\bar\da_{pa}^{\s*}\de_{pa}}(s)\bigg|_{_{\tau/2}}
\s-\s\m\zeta_{\bar\da_{pa}^{\s*}\de_{pa}}(s)\bigg|_{_{\tau}}\ \ 
\quad{\rm and}\quad\ \ 
Z_{aa}(\tau)\s=\s|q^{-1/48}\prod\limits_{n=0}^\infty
(1+q^{n+1/2})|^4\s
\non
\qqq
and that
\qq
Z_{aa}(\tau)\s={\rm tr}_{\CH_{NS}\otimes\tilde\CH_{NS}}
\s\s\s\s q^{L_0-1/24}\bar q^{\tilde L_0-1/24}\s.
\qqq
\vskip 0.1cm

\no (f). \ Prove the modular properties:
\qq
Z_{pa}(\tau+1)\s=\s Z_{pa}(\tau)\s,\quad\quad Z_{ap}(\tau+1)\s=\s 
Z_{aa}(\tau)\s,\quad\quad Z_{aa}(\tau+1)\s=\s Z_{ap}(\tau)\s,\cr
Z_{pa}(-1/\tau)\s=\s Z_{ap}(\tau)\s,\quad\quad Z_{ap}(-1/\tau)\s=\s 
Z_{pa}(\tau)\s,\quad\quad Z_{aa}(-1/\tau)\s=\s Z_{aa}(\tau)\s.
\non
\qqq
\vskip 0.9cm

\no PROBLEM 10. \ {\un{Bosonization}}
\vskip 0.4cm

\no 
The spin structure is called even (odd)
if the dimension of the kernel of $\de_L$ is even (odd).
Denote by $\sigma(L)\s$ the parity of $L$.
The bosonization formula asserts that
\qq
\sum\limits_L(-1)^{\,\sigma(L)}\s\s Z_L\s=\s C^{h-1}\s Z_{1/2}
\label{4}
\qqq
where on the right hand side we have the partition function
of the bosonic free field with values in the circle of radius
squared $1\over 2$, $C$ is a constant and \s$h$ the genus of 
the Riemann surface $\Sigma$. These equalities extend to correlations.
For example, the fermionic fields \s$(\psi\tilde\psi)(x)\s$
correspond to bosonic fields \s$:\ee^{\m i\phi(x)}:\s$
and \s$(\tilde\chi\chi)(x)\s$ to \s$:\ee^{-i\phi(x)}:\s$.
What are their conformal weights?
Prove identity (\ref{4}) for \s$\Sigma=\NC/(\NZ+\tau\NZ)\s$
using the expression (\ref{tpf}) for \s$Z_{1/2}(\tau)\s$ and 
the classical product expressions for the theta functions
\qq
\vartheta(z|\tau)\equiv\sum\limits_{n\in\NZ}\ee^{\m\pi i\tau n^2\s+\s
2\pi in z}\s=\s\prod\limits_{n=1}^\infty(1-q^n)\s(1+\ee^{\m 2\pi i z}
\s q^{\m n-1/2})\s(1+\ee^{-2\pi i z}\m q^{\m n-1/2})\s.
\non
\qqq
What is the Hilbert space interpretation of the left hand side
of Eq.\s\s(\ref{4}) on the elliptic curve?


\end{document}







