%From: Pierre Deligne <deligne@math.ias.edu>
%Date: Wed, 23 Oct 1996 16:44:48 -0400
%Subject: Set Two, No. 4

\input amstex
\documentstyle{amsppt}
\magnification=1200
\pagewidth{6.5 true in}
\pageheight{8.9 true in}
\loadeusm

\catcode`\@=11
\def\logo@{}
\catcode`\@=13

\NoRunningHeads

\font\boldtitlefont=cmb10 scaled\magstep1

\def\dspace{\lineskip=2pt\baselineskip=18pt\lineskiplimit=0pt}
\def\upvee{{\sssize \vee}}
\def\w{{\mathchoice{\,{\scriptstyle\wedge}\,}
  {{\scriptstyle\wedge}}
  {{\scriptscriptstyle\wedge}}{{\scriptscriptstyle\wedge}}}}
\def\Le{{\mathchoice{\,{\scriptstyle\le}\,}
  {\,{\scriptstyle\le}\,}
  {\,{\scriptscriptstyle\le}\,}{\,{\scriptscriptstyle\le}\,}}}
\def\Ge{{\mathchoice{\,{\scriptstyle\ge}\,}
  {\,{\scriptstyle\ge}\,}
  {\,{\scriptscriptstyle\ge}\,}{\,{\scriptscriptstyle\ge}\,}}}
\def\vrulesub#1{\hbox{\,\vrule height7pt depth5pt\,}_{#1}}
\def\mapright#1{\smash{\mathop{\,\longrightarrow\,}%
     \limits^{#1}}}
\def\arrowsim{\smash{\mathop{\,\longrightarrow\,}%
   \limits^{\lower1.5pt\hbox{$\scriptstyle\sim$}}}}
\def\plus{{\sssize +}}

\def\scrLbar{\overline{\scrL}}

\def\Adot{\Dot{A}}
\def\Bdot{\Dot{B}}
\def\Xdot{\Dot{X}}
\def\xdot{\Dot{x}}
\def\phidot{\Dot{\phi}}

\def\Ber{\text{\rm Ber}} \def\pr{\text{\rm pr}}
\def\red{\text{\rm red}}


\def\scr#1{{\fam\eusmfam\relax#1}}

\def\scrA{{\scr A}}   \def\scrB{{\scr B}}
\def\scrC{{\scr C}}   \def\scrD{{\scr D}}
\def\scrE{{\scr E}}   \def\scrF{{\scr F}}
\def\scrG{{\scr G}}   \def\scrH{{\scr H}}
\def\scrI{{\scr I}}   \def\scrJ{{\scr J}}
\def\scrK{{\scr K}}   \def\scrL{{\scr L}}
\def\scrM{{\scr M}}   \def\scrN{{\scr N}}
\def\scrO{{\scr O}}   \def\scrP{{\scr P}}
\def\scrQ{{\scr Q}}   \def\scrR{{\scr R}}
\def\scrS{{\scr S}}   \def\scrT{{\scr T}}
\def\scrU{{\scr U}}   \def\scrV{{\scr V}}
\def\scrW{{\scr W}}   \def\scrX{{\scr X}}
\def\scrY{{\scr Y}}   \def\scrZ{{\scr Z}}

\NoBlackBoxes
\document
\line{{\boldtitlefont Witten's Problems}, Set Two --- 
N$^{\text{o}}$. 4
\hfill (solution written by P. Deligne)}
\smallskip
\hbox to \hsize{\hrulefill}

\bigskip
\dspace
\noindent
(a)\enspace
Let $P$ be the subbundle of the tangent bundle $T$ of
$W$ generated by $D$, and $P_2$ be generated by $D$ and
$D^2:= \frac12[D,D]$.
We have
$$
B_-\oplus P_2\arrowsim T
$$
and by the Frobenius pairing, $P_2/P\simeq P\otimes P$.

\noindent
The filtration $T\supset P_2\supset P$ of $T$, with
successive quotients $B_-$, $P_2/P$ and $P$, gives a
canonical isomorphism
$$
\align
\Ber(T) &\simeq \Ber(B_-)\otimes\Ber(P_2/P)\otimes
  \Ber(P)\\
&\simeq B_-\otimes P_2/P\otimes P^{\upvee}.
\endalign
$$
As $P_2/P\simeq P^{\otimes 2}$, this gives
$$
\Ber(T)^{\upvee}=B_-^{\upvee}
$$
and the lagrangian density
$$
\left<dX\vrulesub{B^-}, dX\vrulesub{P^{\upvee}}\right>
$$
is canonically a section of the bundle of volume
elements.

\bigskip\noindent
(b)\enspace
Not all $W$ are locally isomorphic to the flat model
given.
Indeed, $P_2$ and $B_-$ are integrable distributions,
giving locally a decomposition of $W$ as
$$
W^{1,1}\times W^1.
$$
The odd distribution $P$ is in the tangent space of the
fibers of the projection $\pr_2$ to $W^1$.
All fibers are identified with $W^{1,1}$ by the
projection to $W^{1,1}$ and the given flat model
corresponds to the case where the induced distributions
on the fibers $\pr_2^{-1}(t)\arrowsim W^{1,1}$ are all
equal.

Let us compute the Euler-Lagrange equations for the
extremals for the ``flat'' $W$ given.
Because $D^2=\partial_{\plus}$, the relevant trivialization of
$\Ber(T)$ is the one corresponding to the basis
$(\partial_-,\partial_{\plus},D)$ or, what amounts to the
same, corresponding to the basis $(\partial_u,\partial_v,
\partial_\theta)$ (triangular change of basis).
The lagrangian density is
$$
\left<\partial_-X,DX\right>\,.\,\text{(volume form
$[u,v,\theta]$ given by the coordinates)}.
$$
If we let $X$ depend on an additional even parameter
$a$, the variation $\frac{d}{da}$ of the lagrangian
density is given as follows at $a=0$:
let $X$ be what $X$ is at $a=0$, and $\delta X$ be
$\delta_aX$: a section of $X^*T_M$.
Defining derivatives of vectors using the Levi-Civita
connection, using that this connection preserves
$\left<\,\,,\,\,\right>$ and that it is torsion free
(cf. exercise 1 of set 1), we get at $a=0$ that 
$$
\alignat2
\tfrac{d}{da}\,\scrL &=\left(\left<\partial_a \partial_-
X, DX\right> +\left<\partial_- X,\partial_a
DX\right>\right) &\quad &[u,v,\theta]\\
&=\left(\left<\partial_-\partial_a X,DX\right>+
  \left<\partial_- X, D\partial_a X\right>\right)
  &&[u,v,\theta]\\
&=\left(\left<\partial_-\delta X, DX\right>+
  \left<\partial_- X, D\delta X\right>\right)
   &&[u,v,\theta]\\
&=\left(\partial_-\left<\delta X,DX\right>+
  D\left<\partial_- X, \delta X\right>\right)
   &&[u,v,\theta]\\
&\qquad-\left( \left<\delta X,\partial_- DX\right>-
  \left< D\partial_- X,\delta X\right>\right)
  &&[u,v,\theta]
\endalignat
$$
where the first term is
$$
d\left(\left<\delta X,DX\right>\partial_-\bot
[u,v,\theta]+\left<\partial_- X, \delta X\right>
D\bot [u,v,\theta]\right),
$$
because the Lie derivative $\scrL_D ([u,v,\theta])$
vanishes, and the second is
$$
\delta\scrL=-2\left<\delta X,\partial_- DX\right>\quad
[u,v,\theta].
$$

This gives the equation of motion
$$
\partial_- DX=0.\tag1
$$

If one works over a purely even parameter space $B$, a
map $X\colon\, W\to M$ factors as
$$
W\to W_{\red}\mapright{x} M,
$$
the first map being $(u,v,\theta)\to(u,v)$.
Indeed, the even functions on $W\times B$ are just the
functions on $W_{\red}\times B$, as $W\times B$ is of
dimension $(*,1)$.
The equation (1) amounts to
$$
\partial_-\partial_{\plus}x=0\leqno{\hbox{$(1)_{\red}$}}
$$
for $x$: the reduced space $Y_{\red}$ of the space $Y$
of extremals is the space of ``harmonic'' maps from
$W_{\red}$ (viewed as a quotient of $W$) to $M$.
Beware that ``harmonicity'' is of an hyperbolic
character, as we are in a Minkowsky space setting.

Over a general parameter space $B$, a map $X\colon\,W\to
M$ (i.e. a map from $W\times B$ to $M$) can be described
by

\medskip\noindent
(a)\enspace
its restriction $x$ to the subvariety $\theta=0$ of $W$
(i.e. its restriction to $W_{\red}\times B$ (not:
$(W\times B)_{\red}$);

\smallskip\noindent
(b)\enspace
its derivative $\psi=\partial_\theta X$, restricted to
$\theta=0$.
It is an odd section of $x^*T_M$.

\smallskip
If we use local coordinates on $M$, this means writing
$$
X^i=x^i+\theta\psi^i,
$$
with $x^i$ and $\psi^i$ depending only on $u$ and $v$.

Let $\scrLbar$ be obtained from the lagrangian density
by integrating along the fibers of
$(u,v,\theta)\to(u,v)$.
It is
$$
\scrLbar=\left(-\left<\partial_-
x,\partial_{\plus}x\right>+\left<\nabla_-\psi,\psi\right>
\right)du\,dv\,\,.
$$

We can view $\scrLbar$ as a lagrangian density for the
space of fields $(x,\psi)$ on the $(u,v)$-space
$W_{\red}$: $x$ a map to $M$, $\psi$ an odd section of
$x^*T_M$.
For general reasons, the lagrangian density $\scrLbar$
give rise to the same space of extremals as $\scrL$
does, and to the same $2$-form on the space of
extremals.

As in Set 1, N$^{\text{o}}$ 2, the Euler Lagrange
equations for $(x,\psi)$ are of the form

\medskip\noindent
(1)\enspace
$\nabla_-\partial_{\plus}x=$ quadratic in $\psi$
(involving the curvature),

\smallskip\noindent
(2)\enspace
$\nabla_-\psi=0$.

\medskip
If the parameter space $B$ is purely even, the odd section 
$\psi$ of $x^*T_M$ (on $W_{\red}\times B$) is zero, and
one finds again that $Y_{\red}$ is a space of ``harmonic''
maps.

Over a general $B$, if $I$ is the ideal defining
$B_{\red}$ and if one works modulo $I^2$, (1) again
reduces to $\nabla_-\partial_{\plus}x=0$.
This gives the normal bundle of $Y_{\red}$ in $Y$:
at $x$ in $Y_{\red}$, it is $\Pi V$, for $V$ the space
of sections $s$ of $x^*TM$ such that $\nabla_- s=0$.

Because of the quadratic terms in (1), I don't know any
natural way to map $Y$ to $Y_{\red}$, to view $Y$ as a
bundle over $Y_{\red}$.
Taking $(x,\psi)\mapsto x=(x,0)$ will not do.

However, the quadratic terms in (1) don't spoil the kind
of Cauchy data for which (1) (2) is a well posed system
of equations.
With the notations of (d): $u=t+x$, $v=t-x$, a Cauchy
data consist in $x$, $\xdot$ and $\psi$ at $t=0$.
In terms of $X\colon\,W\to M$, this is $X$ and its
derivatives $\Xdot$ and  $\partial_\theta X$ at
$t=\theta=0$.
The projection $(x,\xdot,\psi)\mapsto (x,\xdot)$:
Cauchy data for $Y_{\red}\to$ Cauchy data for $Y$, gives
a projection $Y\to Y_{\red}$.
It is, however, not ``natural'' in the sense that it is
not invariant by time evolution: it depends on having
chosen the line $t=0$.

Suppose the parameter space $B$ is of dimension $(*,b)$.
Let $I$ be the ideal defining $B_{\red}$ in $B$.
To solve (1) (2) for a given Cauchy data, one can
proceed as follows:

\medskip\noindent
(1)\enspace
Solve (1) (2) over $B_{\red}$.
After pull back to $B_{\red}$, we have $\psi=0$ by
parity reason and one has to solve for an ``harmonic''
map: $\nabla_-\partial_{\plus}x_0=0$.

\smallskip\noindent
(2)\enspace
One can view $x_0$ as being over $B/I^2$: modulo $I^2$,
the even part of $C^\infty(W\times B)$ is
$C^\infty(W\times B_{\red})$.
One solves (1) (2) modulo $I^2$, using $x_0$ given by
(1) and solving (2).
This amounts to solving an equation $\nabla_-\psi=0$ on
$W_{\red}\times B_{\red}$, for $\psi$ a section of
$x_0^*T_M\otimes \pr_2^*(I/I^2)$.
We now have $(x_1,\psi_1)$ modulo $I^2$.

\smallskip\noindent
(3)\enspace
If we already have $(x_r,\psi_r)$ which is a solution
$\mod\,I^{r+1}$, i.e. a $r$-jet of solution along
$B_{\red}\subset B$, we extend it in any way to
$(x^0,\psi^0)$ modulo $I^{r+2}$.
Then correct $(x^0,\psi^0)$ into a solution
$(x_{r+1},\psi_{r+1})$, by a correction trivial modulo
$I^{r+1}$.
At each step, this amounts to solving on $W_{\red}\times
B_{\red}$ an equation
$$
\align
\nabla_-\partial_{\plus}x &=\text{second member,\qquad or}\\
\nabla_-\psi &=\text{second member},
\endalign
$$
with $x$ or $\psi$ in $x_0^* T_M\otimes\pr_2^*I^{r+1}/
I^{r+2}$.

This process terminates when $r=b$.

Assuming $M$ complete, and some control at spatial
infinity, we can now identify the space $Y$ of extremals
with the space of Cauchy data.







\enddocument



