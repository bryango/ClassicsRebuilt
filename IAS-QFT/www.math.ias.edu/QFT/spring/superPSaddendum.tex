%Date: Tue, 08 Apr 1997 11:17:58 -0400
%From: Edward Witten <witten@IAS.EDU>

\input harvmac


{\it Addendum To Superhomework}

\def\Z{{\bf Z}}
The discussion in and after the last lecture suggested that the following
exercises on representations of super-Poincar\'e in four dimensions
would be helpful.

Recall that unitary irreducible representations of super-Poincar\'e are
constructed from certain equivariant vector bundles supported on a cone or
hyperboloid in momentum space.
The equivariant vector bundles in question are classified by representations
of the subgroup $H_p$ of super-Poincar\'e that leaves fixed a chosen vector 
$p$ 
on the cone or hyperboloid.  Moreover, one can replace $H_p$ by its maximal
reductive subgroup, whose bosonic part is $Spin(2)$ if we are on the cone
and $Spin(3)$ if we are on the hyperboloid.  

The $Spin(2)$ Lie algebra has one generator which we call $J$; its eigenvalues
are in $\Z/2$.  The $Spin(3)$ Lie algebra has one representation $R_n$ of
every positive dimension $n$.

The $N=1$ super-Poincar\'e Lie algebra in four dimensions is (with notation
as in the superhomework)
\eqn\ozo{\{Q_A,Q_{\dot A}\}=P_{A\dot A}}
with other anticommutators vanishing and Poincar\'e acting on the odd
generators $Q$ in the usual fashion.  

Upon fixing a vector on the cone or hyperboloid, the $P_{A\dot A}$ become
numbers, say $p_{A\dot A}$.  Unitary representations of \ozo\ are analyzed in
a fashion we should know from last fall: $Q$'s 
that are in the kernel of the quadratic form in \ozo\ are represented by
zero, and the remaining $Q$'s generate a Clifford algebra.

(1) Show that if $p$ is lightlike, the quadratic form in \ozo\ has a
two-dimensional null space, and that quantization of the Clifford algebra
gives a two-dimensional representation  with $J$ eigenvalues
$j,j+1/2$, for arbitrary $j\in\Z/2$.  We will call the $0,1/2$ and $-1/2,0$
representations
the chiral representations $C_+$ and $C_-$ and the $1/2,1$ and $-1,-1/2$
representations the vector representations $V_+$ and $V_-$.

Show that quantization of a chiral superfield gives a sum of $C_-$ and $C_+$ 
for the one particle states, and quantization of a vector superfield gives
similarly a sum of $V_+$ and $V_-$.

(2) Show that if $p$ is timelike, the quadratic form is nondegenerate.
Its irreducible representation  is four-dimensional
and is (as a $Spin(3)$ respresection)  $C'=R_1\oplus R_1\oplus R_2$.
An irreducible  representation of $H_p$ is then $C'\otimes R$ where $R$ is 
any irreducible representationa of $Spin(3)$.

We call $C'$ the massive chiral representation and we call $V'=C'\otimes R_2$
the massive vector representation.

(3) Show that when a null vector $p$ is perturbed to be timelike,
$C_+\oplus C_-$ can be deformed to $C'$.  Now do a bit of field theory:
show that the representation theory fact just stated  corresponds to the
fact that it is possible to add a bare mass for a massless chiral superfield.

(4) The analogous statement for the vector representation is a bit more
complex (as we should expect from our study of the Higgs mechanism in the
bosonic case).  Show that the representation $V_+\oplus V_-\oplus C_+\oplus 
C_-$
can be perturbed to $V'$.

Now show how this can be realized in field theory.\foot{I write the following
in a slightly naive way; the model presented here is actually anomalous.
One can add additional fields to cancel the anomaly without affecting
the occurrence of the super-Higgs mechanism.  (The supersymmetry breaking
that occurs in one case in the text may be destroyed.)}
Consider a theory with a vector superfield  and a chiral superfield
$\Phi$ of charge 1.  Take the Lagrangian to be the kinetic energy
for both superfields plus the ``Fayet-Iliopoulos $D$-term'' 
\eqn\bozo{r\int d^4x D}
(related to the constant term in the moment map) discussed in the 
superhomework.

Show that (a) for $r=0$, we get the representation $V_+\oplus V_-\oplus
C_+\oplus C_-$ at tree level, (b) for $r\not=0$ one gets at tree
level depending on the sign of $r$ either supersymmetry
breaking, or a supersymmetric vacuum with mass gap and representation $V'$.
The occurrence of the latter is called the super-Higgs effect.

(5) Now we consider the $N=2$ supersymmetry algebra.  In the absence of
a central term, the algebra reads
\eqn\jozo{\{Q_A^I,Q_{\dot A J}\} = \delta^I_JP_{A\dot A},\,\,\,\,\,I,J=1,2}
with other anticommutators vanishing.

Classify the unitary representations as follows:

(a) In the massless case, the ``helicities'' or $J$ eigenvalues in
an irreducible representation are $j,j+1/2,j+1/2,$ and $j+1$ for any 
$j\in {\bf Z}/2$.

(b) In the superhomework you met two kinds of $N=2$ fields.
Show that the massless
hypermultiplet gives on quantization a representation $H$ which
is the sum of two copies of the $j=-1/2$ representation
(helicities from $-1/2$ to $+1/2$; there are two copies
because of the quaternionic structure of the hypermultiplet)
while the vector multiplet gives a representation $V$
that is the sum of a representation with $j=-1$
(helicities from $-1$ to 0) and $j=0$ (helicities from 0 to 1).  We call the
$j=-1/2$ representation $H_0$.

(c) In the massive case, the basic representation $H'$, obtained
by quantizing the nondegenerate Clifford algebra in \jozo, has helicities
ranging from $-1$ to $1$ with multiplicities $1,4,6,4,1$.  
\foot{In the massive case, by helicity we mean the weight under the generator
of a maximal torus of $Spin(3)$.} 
Any irreducible
unitary representation of the little group $H_p$ 
is $H'\otimes R$ with $R$ an irreducible representation of $Spin(3)$.


 The above facts might seem to imply, for instance, that a bare
mass for the hypermultiplet is impossible (why?) but it is in fact possible
to add a supersymmetric bare mass term.  In the presence of the bare
mass, the $N=2$ super-Poincar\'e is deformed to the most general possibility
allowed by the Haag-Sohnius-Lopuszanski theorem (the subject of Bernstein's
first lecture).  
In the process, the $R$ symmetry group of outer automorphisms, isomorphic
to $U(2)$ (and discussed in the superhomework), is reduced to $SU(2)$.

Concretely, the deformation of the algebra is as follows.
\jozo\ is unchanged, but instead of the other anticommutators
vanishing one has
\eqn\lozo{\{Q_{A}^I,Q_B^J\}=\epsilon^{IJ}\epsilon_{AB} Y,}
where the $\epsilon$'s are the antisymmetric tensors invariant
under the action of Poincar\'e and of the $SU(2)$ group of $R$ symmetries.
$Y$ is a ``central charge,'' that is an operator that commutes with
the whole super-Poincar\'e, including its adjoint $\overline Y$ which
appears in the remaining anticommutator:
\eqn\olozo{\{Q_{\dot A I},Q_{\dot B J}\}=\epsilon_{AB}\epsilon_{IJ}\overline 
Y.}

(6) Consider a representation of mass $M$ and $Y=y$ ($y$ is a complex number;
as $Y$ is central, in an irreducible representation it can be set to a 
constant).
We have already analyzed the representations for $y=0$.  Show
that more generally the representations with $M>|y|$ are similar (for
instance, deformable to)  the
massive representations with $y=0$ and the representations with $M=|y|$ are
likewise similar to the representations with $M=y=0$.  

(7) There are two important cases of this in practice.  One is the 
hypermultiplet,
for which it is in fact possible to add a bare mass.  Recall from the
superhomework that a single hypermultiplet has for its bosonic fields
a map to the flat hyper-Kahler manifold ${\bf H}={\bf R}^4$.  ${\bf H}$ has
a group $SU(2)$ of symmetries that preserve the hyper-Kahler structure
(and so commute with super-Poincar\'e).  Show that it is possible to add a bare
mass, in such a way that $Y$ is equal to an arbitrary element of the $SU(2)$
Lie algebra.  (This is easy if you have the right frame of mind, but perhaps
not otherwise.  In any event it is less immediately necessary than the
results stated in the next exercise, 
which are essential for doing Donaldson theory.) 

(8) The other important case is the vector multiplet. First we have
to discuss the Higgs mechanism for $N=2$.   We recall that the
bosonic fields of a vector multiplet for gauge group $G$ are a connection $A$
and a complex scalar $\phi$ in the adjoint representation.  There may also
be charged scalars that transform in hypermultiplets; this means that
one has for bosonic fields not only the vector multiplet but also
a map from Minkowski space to a hyper-Kahler manifold $X$ (on which $G$ acts).

There are two basic cases of the Higgs mechanism for $N=2$.  We describe
these in a language that is tied to weak coupling (so we can speak informally
of the expectation value of a charged field):

(i) The scalar fields that have vacuum expectation values are all in the
vector multiplet; this means that the vacuum is determined by a point in $X$
that is $G$-invariant.

(ii) The vacuum corresponds to a non-trivial orbit in the action of $G$ on $X$,
for instance a free orbit.

In this context:

(a) Consider case (i), which is really essential for Donaldson theory.
 You may as well assume there are no hypermultiplets.
Then because of the term $\Tr [\phi,\bar \phi]^2$ term in the potential
(recall the superhomework) the unbroken subgroup of $G$ has the same rank
as $G$ and is in fact, for generic $\phi$,  a maximal torus.  By going from
$\phi=0$ to $\phi\not=0$, most of the representations get mass, but the
representations do not change in size; the massive representations we get
are obviously representations with $M=|Y|$, as those are the massive
representations to which the massless vector representation can be deformed.
The relevant $Y$ is computed in the Olive-Witten paper that was distributed
previously.


(b) In this same situation, magnetic monopoles, which arise from solutions
of the BPS equation $F=*D\phi$, are likewise in ``small'' representations of
supersymmetry.  This is shown as follows:
(1) The classical solution is invariant under half of the $Q$'s.
(2) The other half of the $Q$'s give fermion zero modes, four of them
to be precise. There are no other fermion zero modes.
 (3) Quantization of four fermion zero modes gives a four-dimensional
Clifford algebra with a four-dimensional irreducible representation, which
is what we called above  representation $H_0$.
The monopole {\it of fixed electric charge}\foot{We have  seen in one of the 
lectures that monopoles have an ``arbitrary'' electric charge. We recall that 
the 
electric charge of the monopole is obtained by quantizing a bosonic zero 
mode.} 
transforms in the representation $H_0$.  The antimonopole (obtained by 
quantizing the
solution of $F=-*D\phi$, and with opposite electric charge) gives another
copy of $H_0$.  Together they make the hypermultiplet representation $H$;
this will be important in Donaldson theory.

(c) Now show that a Higgs effect of type (ii) gives a representation of $M>
|Y|$.
To show that this is so generically it is enough to consider a special case,
and I would recommend working through in detail the following example.
Take $G=U(1)$, with a linear action on the quaternions ${\bf H}$ (preserving
the hyper-Kahler structure) and let $X={\bf H}^2$.  A Higgs mechanism is 
possible,
since the hyper-Kahler quotient of $X$ by $U(1)$ is non-trivial and in fact
is four-dimensional.  (What is it?)
For a generic classical vacuum, show that the spectrum
at tree level is as follows: a massless hypermultiplet and a copy of the basic
massive $Y=0$ representation $H'$.  We recall that
the latter has helicities running from $-1 $ to 1
with multiplicities $1,4,6,4,1$.  (As a $Spin(3)$ representation it is the sum 
of
$R_3$, four copies of $R_2$, and five copies of $R_1$.)
\end








