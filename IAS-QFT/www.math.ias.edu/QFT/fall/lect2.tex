 

%% This is an AMS-TeX file.
%% The command to compile it is: amstex <file>
%%
\input amstex
\documentstyle{amsppt}
\loadeusm
\magnification=1200
\pagewidth{6.5 true in}
\pageheight{8.9 true in}

\catcode`\@=11
\def\logo@{}
\catcode`\@=13

\NoRunningHeads

\font\boldtitlefont=cmb10 scaled\magstep1
\font\bigboldtitlefont=cmb10 scaled\magstep2

\def\dspace{\lineskip=2pt\baselineskip=18pt\lineskiplimit=0pt}

%\def\oplusop{\operatornamewithlimits{\oplus}\limits}
%\def\Piop{\operatornamewithlimits{\Pi}\limits}
\def\prodop{\operatornamewithlimits{\prod}\limits}
%\def\w{{\mathchoice{\,{\scriptstyle\wedge}\,}
%  {{\scriptstyle\wedge}}
%  {{\scriptscriptstyle\wedge}}{{\scriptscriptstyle\wedge}}}}
\def\Le{{\mathchoice{\,{\scriptstyle\le}\,}
  {\,{\scriptstyle\le}\,}
  {\,{\scriptscriptstyle\le}\,}{\,{\scriptscriptstyle\le}\,}}}
\def\Ge{{\mathchoice{\,{\scriptstyle\ge}\,}
  {\,{\scriptstyle\ge}\,}
  {\,{\scriptscriptstyle\ge}\,}{\,{\scriptscriptstyle\ge}\,}}}
\def\vrulesub#1{\hbox{\,\vrule height7pt depth5pt\,}_{#1}}
%\def\rightsubsetarrow#1{\subset\kern-6.50pt\lower2.85pt
%     \hbox to #1pt{\rightarrowfill}}
%\def\mapright#1{\smash{\mathop{\,\longrightarrow\,}\limits^{#1}}}
%\def\arrowsim{\smash{\mathop{\to}\limits^{\lower1.5pt
%  \hbox{$\scriptstyle\sim$}}}}

\def\eps{{\varepsilon}}
\def\plus{{\sssize +}}
\def\Mbar{\overline{M}}
\def\Qbar{\overline{Q}}
\def\Ebar{\overline{E}}
\def\fbar{\overline{f}}
\def\gbar{\overline{g}}
\def\Sbar{\overline{S}}
\def\Vbar{\overline{V}}
\def\scrRbar{\overline{\scrR}}
\def\scrSbar{\overline{\scrS}}
\def\grSbar{\overline{\grS}}
\def\Vspace{V_{\space}} \def\QFT{\text{\rm QFT}}
\def\Stil{\widetilde{S}}
\def\scrEtil{\widetilde{\scrE}}
\def\scrStil{\widetilde{\scrS}}
\def\scrTtil{\widetilde{\scrT}}
\def\rhotil{\tilde{\rho}}
\def\grStil{\widetilde{\grS}}

\def\uhat{\widehat{u}}
\def\Phat{\widehat{P}}
\def\Ghat{\widehat{G}}

%\def\Stab{\text{\rm Stab}}
\def\Hom{\text{\rm Hom}}
\def\Map{\text{\rm Map}}
%\def\Vbarspace{\overline {V}_{space}}
%\def\Vspace{ V_{space}}
\def\SO{\text{\rm SO}} 
\def\Spin{\text{\rm Spin}}
%\def\Res{\text{\rm Res}}
%\def\Aut{\text{\rm Aut}}
%\def\Out{\text{\rm Aut}}
\def\End{\text{\rm End}}
\def\supp{\text{\rm supp}}
\def\spin{\text{\rm spin}}
%\def\SU{\text{\rm SU}} \def\Lie{\text{\rm Lie}}
%\def\SL{\text{\rm SL}} \def\Sym{\text{\rm Sym}}

\def\dbC{{\Bbb C}} 
\def\dbR{{\Bbb R}}
%\def\dbZ{{\Bbb Z}} 

\def\gr#1{{\fam\eufmfam\relax#1}}

%Euler Fraktur letters (German)
\def\grA{{\gr A}}	\def\gra{{\gr a}}
\def\grB{{\gr B}}	\def\grb{{\gr b}}
\def\grC{{\gr C}}	\def\grc{{\gr c}}
\def\grD{{\gr D}}	\def\grd{{\gr d}}
\def\grE{{\gr E}}	\def\gre{{\gr e}}
\def\grF{{\gr F}}	\def\grf{{\gr f}}
\def\grG{{\gr G}}	\def\grg{{\gr g}}
\def\grH{{\gr H}}	\def\grh{{\gr h}}
\def\grI{{\gr I}}	\def\gri{{\gr i}}
\def\grJ{{\gr J}}	\def\grj{{\gr j}}
\def\grK{{\gr K}}	\def\grk{{\gr k}}
\def\grL{{\gr L}}	\def\grl{{\gr l}}
\def\grM{{\gr M}}	\def\grm{{\gr m}}
\def\grN{{\gr N}}	\def\grn{{\gr n}}
\def\grO{{\gr O}}	\def\gro{{\gr o}}
\def\grP{{\gr P}}	\def\grp{{\gr p}}
\def\grQ{{\gr Q}}	\def\grq{{\gr q}}
\def\grR{{\gr R}}	\def\grr{{\gr r}}
\def\grS{{\gr S}}	\def\grs{{\gr s}}
\def\grT{{\gr T}}	\def\grt{{\gr t}}
\def\grU{{\gr U}}	\def\gru{{\gr u}}
\def\grV{{\gr V}}	\def\grv{{\gr v}}
\def\grW{{\gr W}}	\def\grw{{\gr w}}
\def\grX{{\gr X}}	\def\grx{{\gr x}}
\def\grY{{\gr Y}}	\def\gry{{\gr y}}
\def\grZ{{\gr Z}}	\def\grz{{\gr z}}

\def\so{{\frak {so}}}

\def\v{\underline v}

\def\scr#1{{\fam\eusmfam\relax#1}}

\def\scrA{{\scr A}}   \def\scrB{{\scr B}}
\def\scrC{{\scr C}}   \def\scrD{{\scr D}}
\def\scrE{{\scr E}}   \def\scrF{{\scr F}}
\def\scrG{{\scr G}}   \def\scrH{{\scr H}}
\def\scrI{{\scr I}}   \def\scrJ{{\scr J}}
\def\scrK{{\scr K}}   \def\scrL{{\scr L}}
\def\scrM{{\scr M}}   \def\scrN{{\scr N}}
\def\scrO{{\scr O}}   \def\scrP{{\scr P}}
\def\scrQ{{\scr Q}}   \def\scrR{{\scr R}}
\def\scrS{{\scr S}}   \def\scrT{{\scr T}}
\def\scrU{{\scr U}}   \def\scrV{{\scr V}}
\def\scrW{{\scr W}}   \def\scrX{{\scr X}}
\def\scrY{{\scr Y}}   \def\scrZ{{\scr Z}}

\def\gr#1{{\fam\eufmfam\relax#1}}

%Euler Fraktur letters (German)
\def\grg{{\frak g}}
\def\grp{{\frak p}}


\def\spinact{s}
\def\spinel{\tau}

\NoBlackBoxes
\document

\centerline{\bigboldtitlefont Lecture 2}


\centerline{\boldtitlefont Eucledian formulation of Wightman QFT}

\bigskip
\centerline{David Kazhdan}


\dspace
\bigskip\bigskip



\bigskip


\subhead{2.1} {Analytic continuation of Wightman functions}\endsubhead
In lecture 1 we were working with Minkowski space-time $V$. However it is
often desirable to be
able to work with Euclidean (Riemannian rather than pseudo-Riemannian)
space-time as well. In this lecture we explain what data over the Euclidean
space-time $E= \dbR^d$ corresponds to a QFT in Wightman 
axioms.

More precisely, let $V_\dbC=V\otimes \dbC$, $V_n^\dbC=V_n\otimes \dbC$, and
consider the Euclidean affine space $E=i\dbR\times \dbR^{d-1}\subset V_\dbC$. 

Going between QFT data on Minkowski space-time $V=\dbR\times (\dbR)^{d-1}$
and the one on $E=i\dbR\times (\dbR)^{d-1}$ is often called {\it Wick
rotation}.

The plan is to obtain  Euclidean counterparts of Wightman functions
by analytically continuing $\scrW_n$ to an open domain in $V_n^\dbC$ and
then restricting the obtained analytic functions to $E$. 
 

We need some notations.
Let $G_\dbC$ be complexification of $G$.
We set: 
$\scrT=V-iV_+\subset V_{\dbC}$ (the {\it backward tube});
 $\scrT_n=(\scrT)^n\subset
V_n^\dbC$ (the {\it extended tube});
 $\scrTtil_n=G_\dbC(\scrTtil_n)\subset V_n^\dbC$ (the {\it permuted 
extended tube}) ;
$J_n=\scrTtil_n\cap V_n$.
Recall that we identified $V_n=\Vbar_{n+1}$, so we have an
action of the symmetric group $\Sigma_{n+1}$ on $V_n$.

\proclaim{Proposition} 

\roster
\runinitem"{\rm a)}"
Assume $d>2$. Then for any $\sigma \in \Sigma_{n+1}$ the subset
$J_n\cap \sigma (J_n)\subset V_n$ is open and nonempty.

\smallskip
\item"{\rm b)}"
If $d>2$ then for any $\sigma \in \Sigma_{n+1}$ the subset $\scrTtil_n\cap
\sigma (\scrTtil_n)$ is connected. 

\smallskip
\item"{\rm c)}"
Put $E=i\dbR\times \dbR^{d-1}\subset V_\dbC$;
$E_n=E\times\cdots\times E\subset V_n^\dbC$.
Also let $E_n^0\subset E_n$ be the complement to all
diagonals, and:
$\Ebar_n=E_n/\Delta(E)$; $\Ebar_n^0=E_n^0/\Delta(E)$.
Then $\operatornamewithlimits{\cup}\limits_{\sigma\in\Sigma_{n+1}}
\sigma(\scrTtil_n)\supset\overline{E_{n+1}^0}$.
\endroster
\endproclaim

\demo{Proof}
We start with the following result.
\enddemo

\proclaim{Lemma} (Jost's theorem)
$J_n=\{(v_1,\dotsc,v_n)\in V_n\vert
\forall\,(\lambda_1,\dotsc,\lambda_n)\in\dbR_+^n-\{0\}$:
$(\lambda_1 v_1+\cdots+\lambda_n v_n)\in V_{space}\}$.
\endproclaim

\demo{Proof of the Lemma}
Assume that $(v_1,\dotsc,v_n)\in J_n$.
Then for some $g\in G_\dbC$:\break
 $g(v_1), g(v_2),\dotsc,
g(v_n)\in\scrT$, hence $g(\lambda_1 v_1+\cdots+\lambda_n
v_n)\in\scrT$ for $(\vec {\lambda} )\in\dbR_+^n-\{0\}$.
Let us write $\sum\lambda_k v_k=\xi-i\eta$, where $\eta\in
V_+$.
Then $(\sum\lambda_k v_k)^2=(\xi)^2-(\eta)^2-2i(\xi,\eta)$.
The L.H.S. is real, hence $(\xi,\eta)=0$, and $(\xi)^2\Le
0$, i.e. $(\sum \lambda_k v_k)^2<0$.

Let us prove the converse statement. So assume that the 
(convex) cone generated by $v_1,\dotsc,v_n$ does not intersect 
$\Vbar_+\cup \Vbar _-$. Then  this cone can be 
separated from $V_\pm$ by a hyperplane tangent to  $\Vbar_\pm$.
This means that there exist $\alpha \in \Vbar_+ $ 
and $\beta \in \Vbar _-$
such that: $(\alpha)^2=(\beta)^2=0,\, 
(\alpha , v_k)<0,\,(\beta,v_k)<0$
for $k=1,\dotsc,n$
and $(\alpha, w)>0$ for $w\in V_+$; $(\beta,w)>0 $ for $w\in V_-$. 
Then $(\alpha, \beta)\not = 0$, and it is easy to  show that there exists 
 $g\in G_\dbC$ such that 
$g(\alpha)=i\alpha$, $g(\beta)=-i\beta$
and $g$ is identity on the subspace orthogonal to
 $\langle \alpha, \beta \rangle$. This $g$ satisfies the condition
$g(v_k)\in \scrT$ for $k=1,\dotsc,n$.
\enddemo

\demo{Proof of the Proposition}
a)\enspace
It is obvious that this subset is open.
So we must show it is nonempty, i.e. we are to find a point 
$\underline v=(v_1,\dotsc,v_n)$ such that $(v_1,\dotsc,v_n)$
and $(v_{\sigma(1)},\dotsc,v_{\sigma(n)})$
satisfy the conditions of the last lemma.
We can choose two linearly independent  vectors
$\ell_1,\ell_2\in\dbR^{d-1}\subset V$ (since $d>2$) and put
$\underline v=(\ell_1,\dotsc,\ell_1)+
\sigma^{-1}(\ell_2,\dotsc,\ell_2)$. Then it is 
immediate to check the conditions. 



\smallskip
b)\enspace 
It is enough to show that $\sigma (\scrTtil_n)\cap \scrT_n$ is connected
for any $\sigma$. 


We have a standard

\proclaim{Lemma} Generic element of $G_\dbC$ can be written as $g= g'\cdot 
g_\dbR$, where $g_\dbR \in G$ and $g'$ preserves an orthogonal
decomposition of $V_\dbC$
 into sum of subspace of dimension not greater than 2  defined
over $\dbR$. 
Here by generic we mean an element belonging to an open (in fact
real Zariski open) dense subset.
\endproclaim

%%%%%%%%%%\demo{Proof of the Lemma} For an element $g\in \SO(V_\dbC)$ its
%%%%%%%%%%class in $ \SO(V_\dbC)/ \SO(V)$ is determined by the subspace
%%%%%%%%%%$g(V)\subset V_\dbC$. For generic $g$ this subspace is a graph of
%%%%%%%%%%a  map $L:V\to iV\cong V$. Since imaginary part of
%%%%%%%%%%$(\quad,\quad)|_{g(V)}$ vanishes  we see that $L\in \so (V)$. For
%%%%%%%%%%generic $g$ the element $L\in \so (V)$ is generic, hence is a
%%%%%%%%%%direct sum of standard 2-dimensional blocks, and possibly a
%%%%%%%%%%1-dimensional block. The condition that the real part of
%%%%%%%%%%$(\quad,\quad)|_{g(V)}$ has signature $(1,d-1)$ implies that the
%%%%%%%%%%imaginary eigenvalues of $L$ have absolute value less than 1.
%%%%%%%%%%Consider now  the element $g'\in G_\dbC$ which acts on $V_\dbC$
%%%%%%%%%%with the same  eigenspaces as $L$, and eigenvalues given by
%%%%%%%%%%$\lambda'= \frac{1+i\lambda}{\sqrt{ 1+ \lambda^2 }}$ where
%%%%%%%%%%$\lambda$ is the corresponding eigenvalue of $L$. Then
%%%%%%%%%%$g'(V)=g(V)$, i.e. $g^{-1}g'$ projects to a real element of
%%%%%%%%%%$\SO(V)$. After possibly changing the sign of the action of $g'$
%%%%%%%%%%on one of the 2-dimensional blocks we will  achieve that
%%%%%%%%%%$g^{-1}g'\in G$ is  itself real.  By construction $g'$ preserves
%%%%%%%%%%complex eigenspaces of $L$, so the proof is finished.
%%%%%%%%%%\enddemo

Let us now finish the proof of b). 
Fix $\sigma\in \Sigma_{n+1}$ and  consider  $S=\{( g\in G_\dbC,\,
\v\in V_n^{\dbC})\  | \  g(\sigma(\v))\in \scrT_n\}$.
Take a point $(g,\v)\in S$.

 Moving $(g,\v)$ slightly inside $S$ we can
achieve that
 $g$ is generic as in the last lemma. We can now take $g'$, $g_\dbR$
as in the Lemma. Obviously
 $(g', g_\dbR^{-1}(\v))$ and $(g,\v)$ lie in one connected component of
$S$. Set $\v'= g_\dbR^{-1}(\v)$.

Let $V_0\subset V$ be the unique block in the decomposition corresponding
to $g'$ on which $(\quad,\quad)$ is not negative, and let $\pi:V_n^\dbC\to
(V_0^\dbC)^n$ be the orthogonal projection. One sees immediately that for
$\v\in \scrT_n$ the segment connecting $\v$ with $\pi(\v)$ lies in
$\scrT_n$. Also $\pi$ commutes with $g'$ and $\sigma$; it follows that
 this segment also lies in $\sigma^{-1}( {g'}^{-1}(\scrT_n))$,
hence $(g',\v')$ and
$(g',\pi(\v'))$ are connected  inside $S$. 


If $\dim(V_0)=1$ then $g'|_{V_0}=\pm Id$. So in that case we have
$\pi(\v')\in \scrT_n$, $\pm \sigma (\pi(\v'))\in \scrT_n$, which is only
possible if $\sigma=1$ or $\sigma=w_0$ is the ``long'' permutation
($w_0(i)=n+2-i$). But in these two cases $\sigma
(\scrTtil_n)\supset \scrT_n $, so there is nothing to prove anyway. 

Suppose now that $\dim(V_0)=2$. Set $G^0=\Spin(V_0)$,
 $G^0_\dbC=\Spin(V_0\otimes\dbC)$, $\scrT_n^0=V_0+iV_0^+$.
  Then $G^0_\dbC/G^0$ is the circle $S^1$, and
 one can show that for $\sigma \ne 1, \, w_0$  the set  $\{g\in
 G^0_\dbC/G^0\ |\ g(\sigma(\scrT_n^0))\cap \scrT_n^0 \ne \emptyset
 \}$ coincides with $S^1-\{1,-1\}$.
 For given $g$ the set $g(\sigma(\scrT_n^0)) \cap \scrT_n^0$ is convex and
 hence connected; thus the set $S_{V_0}:= \{( g\in G_\dbC^0,\,
\v\in \scrT_n^0\ |\  g(\sigma(\v))\in \scrT_n\}$ has 2 connected components.
It is easily seen that these 2 components are permuted by conjugation by an
element of $G$, hence they lie in the same connected component of $S$. But
we have  shown that any point in $S$ is connected to a point  in
$(Stab_{V_0}\times V_0) \cap S$, and hence to a point in $S_{V_0} \subset
S$. (Here we use that the action of $g'$ on $V_0$ has determinant
1, because otherwise it has  eigenvalues 1, $-1$, which does not happen for
generic  $g$.)

Since $G$  transitively permutes
 2-dimensional subspaces of $V$ of signature $(1,1)$ we see that $S$ is
connected, which implies the statement.

\smallskip
c)\enspace
For $v\in V_\dbC$ let us write $v=(v^0,\vec{v}\,)$,
$v_0\in\dbC,\,\vec{v}\in\dbC^{d-1}$.
It is clear that for $(v_1,\dotsc,v_{n+1})\in E_{n+1}^0$
there exists $g\in\SO(n+1)$, such that $(gv_i)^0\not=(gv_j)^0$
for $i\not=j$.
Then we can permute $(v_1,\dotsc,v_{n+1})$ so as to get:
$(v_i)^0>(v_j)^0$ for $i>j$.
But this implies that the corresponding vector in $\Vbar_{n+1}$
lies in $\scrTtil_n$.

\enddemo

Now we are ready to prove

\proclaim{Theorem} Let $d>2$.
Then $W_n$ extends to an analytic function on
$\operatornamewithlimits{\cup}\limits_{\sigma\in\Sigma_{n+1}}
\sigma(\scrTtil_n)$.
These functions are $\Sigma_{n+1}$ and $G_\dbC$-equivariant.
(The $\Sigma_{n+1}$-equivariance is understood as usually in the
``super'' sense.)
\endproclaim

\demo{Proof}
Since $\scrF(W_n)$ is supported in  $V_+$ it follows
that $W_n$ extends to an analytic function on $\scrT$
(which will be denoted by the same letter).
Since $W_n$ is $G$-equivariant, this analytic function is
$G$-equivariant.

To ensure that $W_n$  extends to a single-valued analytic 
function on $\scrTtil$, which  then is automatically $G_\dbC$-equivariant,
 it is enough to check that $W_n(x)=W_n(g(x))$
for $g\in G_\dbC$, $x\in \scrT _n$ and $g(x)\in \scrT_n$.

If $g^{-1}\scrT_n \cap \scrT_n \not = \emptyset$ then also
$ g^{-1}\scrT \cap \scrT \not = \emptyset$. Let us take
$v \in  g^{-1}\scrT \cap \scrT$. It is easy to see that 
for $v\in \scrT$ the set $\{h\in G_\dbC | h(v)\in \scrT \}$
is connected. Hence there exists a neighborhood $U\subset \scrT_n\cap
g^{-1}\scrT_n$
of the point $(v,\dotsc,v)$ such that $g^{-1}$ lies in the connected
component of 1 of  $\{h\in G_\dbC | h(U)\subset \scrT \}$.
From $G$-invariance of $W_n$ it follows that for fixed
$x\in V_n^\dbC$ the function 
$W_n(g(x))$ defined on some open subset of $G_\dbC$ is locally
constant. Hence  $W_n(x)=W_n(g(x))$ for $x\in U$.
Since $U$ is  open in the connected set $ g^{-1}\scrT_n \cap \scrT_n$
we can conclude that  $W_n(x)=W_n(g(x))$ for all
 $x\in  g^{-1}\scrT_n \cap \scrT_n$, so analyticity of $W_n$ in  $\scrTtil$
is proved.




Now by Proposition a)  for any $\sigma \in \Sigma_{n+1}$
the set $\sigma^{-1}(\scrTtil)\cap\scrTtil$ contains
an open non-empty subset of $V_n$.
Moreover,
 from the Jost's theorem it follows that
$$
\sigma^{-1}(\scrTtil)\cap\scrTtil\cap V_n\subset (V_{space})^n.
$$
But $W_n\vrulesub{(V_{space})^n}$ is $\Sigma_{n+1}$
(super)equivariant.
Hence $W_n$ and $\sigma^*(W_n)$ agree on a neighborhood of the set of real
points in
$\sigma^{-1}(\scrTtil)\cap\scrTtil$, and hence
on the whole of $\sigma^{-1}(\scrTtil)\cap\scrTtil$ by Proposition b). 
So
the collection of functions $\sigma^* (W_n)$ for all $\sigma \in \Sigma_n$
 give a well-defined analytic 
function on $\operatornamewithlimits{\cup}\limits_{\sigma\in\Sigma_{n+1}}
\sigma(\scrTtil_n)$, which is obviously $\Sigma_{n+1}$-(super)equivariant. 
The theorem is proven.
\enddemo

We denote the restriction of these functions  to $\overline{E_n^0}$
by  $\overline {\grS}_n$. We also put ${\grS}_n=\pi^*(\overline {\grS}_n)$
where $\pi :E_n^0 \to  \overline{E_n^0}$ is the projection. 

\definition{Definition} The functions ${\grS}_n$
 are called {\it Schwinger functions}.
\enddefinition

From the theorem it follows that $\grS_n$ is a real-analytic
$\SO(n)\ltimes E$ and $\Sigma_n$ equivariant function.


\remark{Remark}
The theorem (and its proof) relies only on weak Wightman's
axioms. Hence Schwinger functions can be constructed from a
set of $\scrW _n \in \scrStil_n '$ which satisfy weak Wightman
axioms. 
\endremark

\remark{Remark} The assumption that $d>2$ is not essential in the Theorem.
A  method which allows to treat also the case $d=2$ and  furhter extend
the analyticity domain (the so called ``edge of the wedge Theorem'')
 is explained e.g. in [Streater-Wightman], \S 3.3. 
\endremark

\remark{Remark}
It was explained that QFT has two formulations: Minkowski
and Euclidean.
The experience says that it is more convenient to work
technically with Euclidean picture, though the basic
notions are naturally formulated in Minkowski one.
\endremark

\remark{Remark}
The analytic Schwinger functions on $\Sigma_{n+1}(\scrTtil_n)$  seem to
be the correct object of consideration.
So it is natural to ask whether they can be further
extended to some larger domain.

The domain $\Sigma_{n+1}(\scrTtil_n)$ is not
holomorphically convex.

It may be worthwhile to describe its convex hull and
look at the extension of Schwinger functions to it.

\endremark

\subhead{2.2} {Euclidean formulation of Wightman QFT}\endsubhead
Recall that decomposition $E=(i\dbR)\times \dbR ^{d-1}$ is fixed. 
Let $\theta$ denote reflection at the hyperplane $\{0\}\times \dbR^{d-1}$;
put: $E^+= (i\dbR_{>0})\times \dbR^{d-1}$; $E_n^+=E^+\times
E^+\times\cdots\times E^+\subset E_n$. Let $\scrE _n$ denote the space of
Schwartz sections of $\scrR _n^\dbC|_{E_n}$, and $\scrE_n^0\subset
\scrE_n$ be the subspace of sections with support in $E_n^0$.


\proclaim{Claim}

\roster
\runinitem"a)" 
 Schwinger functions satisfy the following properties:

\smallskip
\itemitem"{i)}" $\grS_n$ is a real-analytic $\Phat$-invariant 
and $\Sigma_n$
(super) invariant real valued function on $E_n^0$ where
$\Phat=\Ghat\ltimes E$,  $\,\Ghat=\Spin(E)$;

\noindent
 If moreover 
$\grS_n$ come from a strong (as opposed to weak) QFT then
they satisfy:

\smallskip
\itemitem"{ii)}"
$\grS_n$ is of moderate growth; i.e. $\grS_n$ defines a tempered 
distribution on $\scrE_n^0$.

\smallskip
\itemitem"{iii)}"
Reflection positivity.

\noindent
Put: $\scrE_n^+=\{F\in \scrE_n^0\vert \supp\,F\subset E_n^+\}$.
Also let $\scrE^+=\oplus \scrE_n^+$,
$\scrEtil=\oplus\scrE_n^0$, and $\grStil=\Pi \grS_n\in
(\scrEtil)'$.
Then for $F\in \scrE^+$ we have: $\grStil(\theta(F)
\boxtimes F)\Ge 0$.

\smallskip
\itemitem"{iv)}" 
Let $\overline {\scrE} _n$ be the space of Schwartz sections
of $\overline {\scrR_n^\dbC}|_{\overline E _n}$, and let 
$\overline{\scrE_n ^+}\subset  \overline {\scrE} _n$ be the subspace
of sections with support in $E^+_{n-1}\subset 
\overline E_n \cong E_{n-1}$.
For $r\in \dbR^+$ and a differential 
operator $D$ consider the seminorm on 
$\overline {\scrE_n ^+} $ defined by: 
$$
|f|_{r,D} = \sup \limits 
_{x\in (\dbR_{>0}\times \dbR^{d-1})^n} |(1+|x|^2)^{r/2}D
\scrF(f)(x)|
$$
(here for $x\in V,\, x=(x_0, \vec x)$ 
we put $|x|^2= x_0^2+|\vec x^2|$).
Then $\overline \grS _n$ defines a 
distribution on $\overline {\scrE_n ^+}$
which is continuous with respect to the seminorms $|\quad|_{r,D}$.

\smallskip
\item"b)"  The construction of \S {2.1} provides a bijection
between Wightman QFT's and collections of functions $\grS_n$
satisfying the conditions i)--iv) above.
\endroster
\endproclaim

\demo{Proof} 
See [Osterwalder-Schrader(1975)].
\enddemo

It should be mentioned that this theorem is 
not really useful, because
the property iv) of the list is impossible to check in practice. 

A more useful list is obtained by replacing the property iv)
by the following axiom:

\remark{Axiom $\operatorname {iv}')$} 
(Linear growth condition) There exist
$\alpha, \beta, r \in \dbR _{>0} $ such that 
$$
|\grS _n (f)| < \alpha (n!)^\beta  |f|_{nr}
$$
for all $f\in \scrE_n^0$ 
where the seminorm $|f|_{nr}$ is given by 
$|f|_{nr}=\sup\limits_{x\in E_n^0,D} 
|(1-x^2)^{r/2}Df(x)|$ and $D$ runs over the  
set of differential monomials of degree less then $nr$. 
\endremark

This property is more tractable, but it is more restrictive than iv).

Some remarks are in order. 
For Schwinger functions constructed from a (strong) Wightman
QFT the reflection positivity property is equivalent to the
positivity property (5) of Wightman axioms.
However, to formulate it we do not have to assume anything
about the behavior of $\grS_n$ on diagonals.
(Note that if $F_1,F_2\in\scrE^+$ then $\theta(F_1)\boxtimes
F_2$ automatically vanishes on diagonals, so
$(F_1)\boxtimes F_2\in\scrEtil)$.
So it can be formulated for Schwinger functions of weak QFT.
It is not clear how to restate this property directly in
Minkowski picture  (which is
more intuitive from the physical point of view) for weak Wightman QFT.

\subhead{2.3} {Schwinger functions and measures on the map-spaces}\endsubhead
In fact, the following picture should stand behind the
formalism of Schwinger functions.
Assume we have a (quasi) measure $\nu$ on $\scrE'$ (where
$\scrE$ is the space of Schwartz sections of
$\scrR^\dbC\vrulesub{E}$), such that if
$F_1,\dotsc,F_n\in\scrE$ and
$\supp(F_i)\cap\supp(F_j)=\emptyset$ for $i\not=j$ then the
function $\alpha\to \alpha(F_1)\ldots\alpha(F_n)$ is
measurable.
Then we can define a distribution in $(\scrE_n^0)'$ for any
$n$ by requirement that its value on
$F_1\boxtimes\ldots\boxtimes F_n$ equals
$\int\limits_{\scrE'}\alpha(F_1)\ldots\alpha(F_n)$.
This distribution should coincide with $\grS_n$; thus $\grS_n$ describes 
$n$-th moment of  $\nu$.
If the measure $\nu$ exists, it 
is uniquely determined by its moments $\grS_n$.
However it does not necessarily exist for given
$\grS_n$ coming from a Wightman QFT. 


So working with $\grS_n$ is a way to talk about the
properties of a measure when the measure itself is not
defined.



The formulation of QFT involving a measure on $\scrE '$ has another
serious drawback. 
In general one is interested in field theories where
the fields are not  sections of a vector bundle, but rather  
maps from a manifold to another manifold (no linear structure on the
target space). Then we can not even say on what space the measure 
$\nu$ is defined. Indeed, in the linear case the measure lives
on the space of generalized sections of a vector bundle. 
But there is no
notion of a generalized map from a manifold to another manifold. The
formalism of Schwinger functions also does not work in this setting:
to define moments of a measure one needs linear structure on the underlying
space.

The difficulty can be overcome if $d=1$. 
In this case for a linear QFT
the measure is actually defined and is concentrated 
on the subspace of
continuous  sections. 
So one gets a reasonable theory considering
the space of continuous maps from $V=\dbR$ to $M$ and 
introducing a measure on this space. 
This is illustrated in the next two examples. 

\example{Example 1}
(Wightman QFT for $d=1$)\enspace
In this case QFT is equivalent to usual quantum mechanics.
Precisely, let us put: $\scrH=L^2(\dbR)$, let
$A\in\dbC[u,\partial_u]$ be a Hamiltonian.
Let $A$ be of the form $A=-\frac{d^2}{du^2}+r(u)$, where
$r(u)\to+\infty$ for $u\to \pm\infty$.
Then $A$ has the smallest eigenvalue, and the corresponding
eigenspace is one dimensional.
Assume that the smallest eigenvalue of $A$ is $0$.
Let $\Omega$ be the corresponding eigenfunction.
Also put: $\scrD=\scrS(R)$, then $\Omega\in \scrD\subset\scrH$.
Finally, let $\uhat\in\End(\scrD)$ be the operator of
multiplication by the function $f(u)=u$.
We define the action of $P=\dbR$ on $\scrH$ by $t\to
e^{iAt}$ and put:
$$
\phi(f)=\int f(t)e^{iAt}\uhat e^{-iAt}dt.
$$
Let us describe the corresponding measure on $\scrS'(\dbR)$.
We start with Gaussian measure $\nu_0$ with covariance
$(-(\frac{d}{dt})^2+m^2)^{-1}$, where $m^2>0$ is such that $r(u)-m^2u^2>0$
for $|u|\gg 0$. 
(So for $f\in\scrS$ we have
$\scrF(\nu_0)(f)=e^{((-(d/dt)^2+m^2)^{-1} f,f)}$.)
We look for $\nu$ of the form $\nu=a\nu_0$ where $a$ is a
function. It is enough to define $a$ on a set of measure 1.
We define $a$ by the formula: 
$$a(\alpha)= e^{ \int\limits _{-\infty}^{\infty}( r(\alpha
(t))-m^2\alpha(t)^2)dt}$$
Using Sobolev theorem one can show that $\nu _0$ is essentially
concentrated on the set of such continuous functions $\alpha$ 
that the above integral converges. Hence the measure $\nu = a\nu_0$
is correctly defined. 

 The fact that the measure $\nu$ and the field operators defined above give
the same set of Schwinger functions follows from the so-called Feynman-Kac
formula, see Gawedzki's lecture 1, \S 1.3.
\endexample

\example{Example 2} (Abstract QFT for $d=1$)
\enspace
Let $M$ be an arbitrary compact Riemannian manifold, $\scrH=L^2(M)$,
$\Delta$ is the Laplacian, $A=-\Delta+r(m)$. 
In this case one still can describe a measure $\nu$ on
the space of continuous maps $\Map(\dbR,M)$ (one can get rid of the
compactness assumption by introducing requirements on asymptotic behavior
of $r$; Example 1 is a particular case of this situation).



The measure can be characterized as follows.
Let $t_1,\dotsc,t_k$ be points in $\dbR$, and
$U_1,\dotsc,U_k\subset M$ be open subsets.
Put $X_{U_1,\dotsc,U_k}^{t_1,\dotsc,t_k}=\{f\in\Hom(\dbR,M)\vert
f(t_i)\in U_i$ for $i=1,\dotsc,k\}$.
Let $K_t(m_1,m_2)$, $m_1,m_2\in M$, $t\in\dbR_{>0}$ be the
fundamental solution of the differential equation
$\frac{df}{dt}=Af$.
Then $\nu$ is the unique measure, such that:
$$
\nu(X_{U_1,\dotsc,U_k}^{t_1,\dotsc,t_k})=
\int\limits_{U_1\times\ldots\times U_k}
\prodop_{i=1}^{k-1}K_{t_{i+1}- t_i}(m_i,m_{i+1})
dm_1\ldots dm_k.
$$

But  there is no easy way to translate it to a
Wightman-type picture.
% unless an embedding of  $M$ into $\dbR^N$ for some $N$ is chosen.
\endexample


We finish by a technical variation on the definition of Wightman functions
which will be used later.

\subhead{2.4} {Time-ordering}\endsubhead
Let $\theta_0\in \SO(1,d-1)$ be reflection at the hyperplane
$\dbR^{d-1}\subset V$ (change of time direction). Then $\theta_0$
can be lifted to an element $\theta\in G_\dbC$, but not to an element of
$G$. Hence the analytic continuations of $\scrW_n$
 we constructed in section 1 are
$\theta$-invariant, though Wightman functions themselves 
 are not. 

We will now present a variant of the definition of $\scrW_n$
which does not have this drawback.
 
Namely, the time-ordered Wightman function $\scrW_n^T$ is given
by: $\scrW_n^T(v)=\lim\limits_{\eps\to \plus 0}$ $\grS_n(v+i\eps
v)$.
For $pr(v)\in J_{n-1}$  we have:
$\scrW_n^T(v)=\scrW_n(v)$, where $pr$ is the projection $V_n\to \Vbar_n$.
The function $\scrW_n^T$ is $\theta$ and $\Sigma_{n}$-invariant (unlike
$W_n$).

Like a usual Wightman function, $\scrW_n^T$ is lifted from a unique 
 function $W_n^T$ on $\Vbar_n$.

Let us describe $W_2^T$ for the free scalar $\QFT$ of mass
$m$; we will denote it by $\Delta_F(\,\,,m^2)$, where $F$
stands for Feynman.
It can be expressed through functions
$\Delta_\plus(\,\,,m^2)=\scrF(\delta_{\sigma_m^\plus})$, and
$\Delta_-(\,\,,m^2)=\scrF(\delta _{\sigma_m^-})$. 

\proclaim{Claim} 
We have:

 a)\enspace
$\Delta_F(v,m^2)\!=\!\Delta_\pm(v,m^2)$ for
$v\in\Vbar\!_\pm$;
$\Delta_F(v,m^2)\!=\!\Delta_+(v,m^2)\!=\!\Delta_-(v,m^2)$ for
$v\in\Vspace$;

\medskip
b)\enspace
$(-\square+m^2)(\Delta_F(\,\,,m^2))=2\pi i\,\delta_0$,
i.e. 
$W_2^T$ is the fundamental solution of
$\left(-\square+m^2\right)$;

\smallskip
c)\enspace 
$\scrF(\Delta_F(\,\,,m^2))=\frac{1}{2\pi i (p^2+m^2+i0)}\,\,
{\overset {\text {def}}\to=}\lim\limits_{\eps\to \plus 0}
\frac{1}{2\pi i(p^2+m^2+i\eps)}$
(this is a well-defined distribution).
\endproclaim

\noindent
[Just to remind how such things look, we recall that
$\frac{1}{x+i0}$ and $\frac{1}{x-i0}$ are well-defined
distributions on $\dbR$ and we have:
$\frac{1}{2\pi i}
\bigl(\frac{1}{x-i0}-\frac{1}{x+i0}\bigr)=\delta_0$
and $\frac12\bigl(\frac{1}{x+i 0}+\frac{1}{x-i 0}\bigr)
=\fracwithdelims()1x_{\text {p.v.}}$, where
$\bigl<\fracwithdelims()1x_{\text {p.v.}},f\bigr>\,\,{\overset
 {\text {def}}\to=}
\lim\limits_{\eps\to +0}
\int\limits_{-\infty}^{-\eps}\frac1x\,f(x)dx
+\int\limits^{\infty}_{\eps}\frac1x\,f(x)dx$.]


\enddocument



