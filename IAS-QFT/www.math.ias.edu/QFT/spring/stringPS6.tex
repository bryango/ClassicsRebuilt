%From: "Eric D'Hoker" <dhoker@sns.ias.edu>
%Date: Wed, 5 Mar 1997 15:13:17 -0500 (EST)

%%%%%%%%%%%%%%%%%%%%%%%%%%%%%%%%%%%%%%%%%%%%%%%%%%%%%%%%%%%%%%%%%%%%%%%%%%%
%%%%
%%%%    STRING THEORY : Problem Set 6, March 6, 1997
%%%%
%%%%%%%%%%%%%%%%%%%%%%%%%%%%%%%%%%%%%%%%%%%%%%%%%%%%%%%%%%%%%%%%%%%%%%%%%%%


\magnification=\magstep1
\overfullrule=0pt
\baselineskip=17pt
\def\det{{\rm det}}
\def\Det{{\rm Det}}
\def\tr{{\rm tr}}
\def\Tr{{\rm Tr}}
\def\12{{1 \over 2}}
\def\ker{{\rm Ker}}
\def\O{{\cal O}}

\centerline{{\bf STRING THEORY}}
\centerline{ Problem Set \# 6}
\centerline{March 6, 1997}

\bigskip
\bigskip

In this problem set, we investigate a number of properties 
of the combined matter (for flat space-time of dimension $D$) 
and ghost system.
Recall that the worldsheet action in locally flat complex 
coordinates $z$ and $\bar z$ is given by
$$
S= { 1 \over 2 \pi} \int _\Sigma d^2z
\bigl ( \12 \partial x \cdot \bar \partial x 
        + b \bar \partial c + \bar b \partial \bar c \bigr ).
$$
The matter stress tensor is denoted by $T^{(x)} (z)$ and 
the quantum ghost stress tensor is given by
$$
T^{(bc)} (z) = \lim _{w \to z} \bigl ( - \partial b (z) c(w)
-2 b(z) \partial c (w) + {1 \over (z-w)^2 } \bigr ).
$$
The classical BRST current is given by
$$
j ^B (z) = c(z) \bigl ( T^{(x)} (z) + \12 T^{(gh)} (z) \bigr ) + 
{3 \over 2} \partial ^2 c(z).
$$

\noindent
{\it Problem 1}

\medskip

a) Give a suitable definition of the quantum BRST current,
by subtracting OPE singularities.

b) Compute the OPE of $j^B$ with $b,~c$ and $x$, and deduce the 
(anti-) commutators of the BRST operator
$$
Q^B = \oint {dz \over 2 \pi i} j^B_z
$$
with these fields.

c) Show that the resulting transformation laws on the fields
$b,~c$ and $x$ give rise to a classical symmetry of $S$ for
any dimension $D$ of space-time, and that $j^B$ is related 
to the Noether current of this symmetry.

d) Show that $Q^B$ is nilpotent as a quantum mechanical 
operator if and only if $D=26$, using your favorite method.
(Hint : It is easy to show that when $D\not=26$, $Q^B$ 
cannot be nilpotent, by using the relation between the full
stress tensor $T= T^{(x)} + T^{(bc)}$ and the anti-commutator
of $Q^B$ with $b$.)

\bigskip

\noindent
{\it Problem 2}

\medskip

Henceforth, set $D=26$, and consider correlation functions 
of vertex operators $V_i$, including ghosts :
$$
\langle V_1 \cdots V_N \rangle 
= \int Dx \int D(b\bar b) D(c\bar c) V_1 \cdots V_N \prod _{j=1} ^\nu
|(\mu _j , b)|^2 e^{-S}
$$
For simplicity, we restrict to surfaces $\Sigma$ of genus $h >1$;
$\mu _j$ is a basis of Beltrami differentials, associated 
with moduli coordinates $m_j$, and $\nu = 3h-3$.

Assume that $V_i$ are physical vertex operators, so that $V_i$
commutes with the BRST operator $Q^B$. Now assume that one vertex 
operator, say $V_N$, is BRST exact, i.e. 
$$
V_N = \int _\Sigma d^2 w \oint _C {dz \over 2\pi i} j^B(z) W_N (w)
$$
for {\it some} operator $W_N$, which is not necessarily a
physical vertex operator. The curve $C$ is a simple contour
around the point $w$. By deforming the contour $C$, show that 
$\langle V_1 \cdots V_N \rangle$ is a total derivative in the
moduli $m_j$. 
 
\end
--1268_5c2-251c_6a2-50b7_231--

