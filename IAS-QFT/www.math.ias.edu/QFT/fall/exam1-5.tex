\input amstex
\magnification=1200
\documentstyle {amsppt}
\pagewidth{6.5 true in}
\pageheight{8.9 true in}
\nologo

\noindent
{\bf Fall Term Exam, N$^{\text{0}}$. 5}\qquad\qquad\qquad
\qquad\qquad\qquad\qquad\qquad(solution by D. Freed)
\smallskip
\hbox to \hsize {\hrulefill}
\bigskip
\noindent
{\bf Problem:}
\medskip
QCD is an $SU(3)$ gauge theory with fermions (quarks) which
describes the nuclear forces.  For this problem you only
need to know a few facts.  For instance, the theory
contains a ``current'' operator, which is simply a Lorentz
vector $J^i$ (that is a local operator of spin one) that is
conserved, $d_jJ^i=0$ (or $d*J=0)$, has zero anomalous
dimension (we'll learn why later), and has the property
that its two point function needs an additive $c$-number
renormalization.
\medskip
The renormalization two point function of $J$ is therefore
the limit as a cutoff $\Lambda$ goes to infinity of
$$
\int d^4x\,\,e^{ik\cdot x}\langle
J^i(x)J^j(0)\rangle-A(\Lambda\slash\mu,g)(k^ik^j-\delta^{ij}
k\cdot k).\tag1
$$
Here $g$ is the (dimensionless) coupling constant,
$\Lambda$ the cutoff, $\mu$ a mass that must be introduced
in renormalization, and $A$ is a funciton of the indicated
arguments which must be ``subtracted'' to make the two
point function finite.  Also, the theory is asymptotically
free and for present purposes you can ignore quark bare
masses and assume that there are no parameters in the
theory except $g,\mu$, and $\Lambda$ (and of course
$\Lambda$ is eventually eliminated by taking it to
infinity).  There is no multiplicative renormalization of
$J$ in (8) because the anomalous dimension vanishes.
Because $J$ is conserved, the expression in (8) is of the
form
$$
(k^ik^j-\delta^{ij}k^2)B(k^2,g,\mu)\tag2
$$
where $\Lambda$ is omitted as we want to consider the limit
$\Lambda\rightarrow\infty$.
\medskip
I suppose it also helps to know that $J$ is built in terms
of quarks as $J=\overline q\Gamma^iq$, so that the one
loop approximation to $\langle JJ\rangle$ is given by a one
loop diagram with a fermion loop that we computed in
homework.  (For those who are getting a paper and not
internet copy of this, the loop is drawn below at the end
of problem 7.)  But all you really need to know is that in
the free field theory, that is if $g=0,\,A=a$
ln$(\Lambda\slash\mu)$, where $a$ is a constant,
independent of $g$.
\roster
\item"{(a)}"  Write down the renormalization group equation
obeyed by $B$.
\smallskip
\item"{(b)}"  Use this equation, together with the fact
that QCD is asymptotically free, to determine the large
$k^2$ behavior of $B$.
\endroster
\medskip
This large $k^2$ behavior, after including some additional
theoretical wrinkles to put it in a form that can be
compared with experiment, is one of the classic tests of
QCD.
\bigskip
\noindent
{\bf Solution:}
\medskip
First, a reminder about the renormalization group equation.
A perturbative theory depends on some coupling constants
$g$ and masses $m$.  (For simplicity we assume there is
just one of each.)  Renormalization introduces a scale
$\mu$.  The renormalization group operator is 
$$
\Cal
D=\mu\,\frac{\partial}{\partial\mu}+\beta\,\,\frac{\partial}
{\partial g}+\gamma_m\,m\,\frac{\partial}{\partial m},
$$
where $\beta$ and $\gamma_m$ (the {\it anomalous} dimension
of mass) are in general functions of $g$ and
$\frac{m}{\mu}$.  They describe how the parameters of the
theory flow when $\mu$ is varied.  Their precise form
depends on the renormalization procedure; for dimensional
regularization/minimal subtraction, they are independent of
$\frac{m}{\mu}$.  Now any operator $\Cal O$ in the theory
has an anomalous dimension $\gamma_{\Cal O}$ (which is also a
function of $g$ and $\frac{m}{\mu}$).
Since the operators are renormalized they also depend on
$\mu$, and the meaning of the anomalous dimension is that
they satisfy the renormalization gorup equation
$$
\Cal D\Cal O=\gamma_{\Cal O}\Cal O.
$$
In general the operators mix under renormalization:  the
renormalization of a given operator $\Cal O_0$ involve
operators of equal and lower dimension.  Thus $\gamma_{\Cal
O}$ is a triangular matrix acting on operators of dimension
$\leq k$ for some fixed $k$.  Of course, the mixing only
occurs among operators with the same symmetries.  In the
renormalization group equation $\Cal D\Cal O=\gamma_{\Cal
O}\Cal O$, then, $\Cal O$ is a vector of operators and
$\gamma_{\lower 2pt\hbox{$\scriptstyle 0$}}$ a matrix (of functions of $g$ and
$\frac{m}{\mu}$).  Similar remarks apply to equation (1)
below, though we write it as if there is no mixing.  Notice
that this holds for the elementary operators (i.e., the
fields which appear in the Lagrangian) as well as for
composite operators.  What defines $\beta,\,\,\gamma_m$,
and $\gamma_{\Cal O}$ is that correlation functions are
preserved under the renormalization group flow.  That is,
for operators $\Cal O_1,\ldots,\Cal O_n$ and {\it
distinct} points $x_1,\ldots,x_n$, we have
$$
\Cal D\langle\Cal O_1(x_1)\cdots\Cal
O_n(x_n)\rangle=(\gamma_{\Cal O_1}+\cdots+\gamma_{\Cal
O_n})\,\,\langle\Cal O_1(x_1)\cdots\Cal
O_n(x_n)\rangle.\tag1
$$
The correlation function blows-up as some points $x$, come
together, and if we consider coinciding points this
necessitates further renormalization.  Such
renormalizations will introduce additional terms in (1).
\medskip
(a)\quad The function $B$ is defined by
$$
\align
\mathop{\lim}\limits_{\Lambda\rightarrow\infty}\biggl\{\,\,\int
\limits_{|x|\geq\frac{1}{\Lambda}}d^4x\,\,e^{-ik\cdot
x}\,\langle
J^i(x)J^j(0)\rangle&-A\,\left(^\Lambda\slash_\mu,g_{^n\slash
\mu}\right)\,\,(k^ik^j-\delta^{ij}k^2)\biggr\}\\
&=B(k^2,g,\mu)\,\,(k^ik^j-\delta^{ij}k^2),\tag2
\endalign
$$
where $g_{\Lambda\slash\mu}$ is the effective coupling
constant with initial value $g_1$.  Since there are no
masses the renormalization group operator is
$$
\Cal
D=\mu\frac{\partial}{\partial\mu}+\beta(g)\frac{\partial}
{\partial g}.
$$
We are given $\gamma_{\lower 2pt\hbox{$\scriptstyle J$}}=0$, 
so differentiating both sides of
(2) using (1), we obtain
$$
\Cal
DB=-\mathop{\lim}\limits_{\Lambda\rightarrow\infty}\Cal
DA\tag3  
$$
as the renormalization group equation of $B$.  There is no
operator mixing since $J$ is the lowest dimensional vector
operator.
\medskip
Now asymptotic freedom means
$\mathop{\lim}\limits_{\Lambda\rightarrow\infty}g_\Lambda=0$
and so to leading order
$$
\mathop{\lim}\limits_{\Lambda\rightarrow\infty}\Cal
DA=\mathop{\lim}\limits_{\Lambda\rightarrow\infty}\Cal
DA(^\Lambda\slash_\mu,0)=\mathop{\lim}\limits_{\Lambda
\rightarrow\infty}\mu\frac{\partial}{\partial\mu}\,\,(a\,\ell
n^\Lambda\slash_\mu)=-a.
$$
Hence (3) becomes
$$
\Cal DB=a.
$$
\medskip
(b)\quad  It follows from (2) that the naive engineering
dimension of $B$ is zero.  Thus simple scaling combined
with (4) gives
$$
\align
B(e^{2t}k^2_0,\,g_1,\mu)&=B(k^2_0,\,g_1,\,e^{-t}\mu)\\
&=B(k^2_0,\,g_{e^t},\,\mu)-at.\tag5
\endalign
$$
Now take $k^2_0=1$ and set $k^2=e^{2t}$.  Then taking
$t\rightarrow\infty$ we learn
$$
B(k^2,g_1,\mu)\sim-\frac{a}{2}\,\ell
n\,^{k^2}\slash_{\mu^2}\qquad\text{as}\qquad
k\longrightarrow\infty.
$$
Observe that this is the answer for the free theory.  In
other words, in the absence of anomalous dimensions the
high momentum behavior of a quantity in an asyptotically
free theory is the same as the high momentum behavior in
the free theory.
\bye
