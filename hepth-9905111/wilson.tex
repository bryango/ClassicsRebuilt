
\section{Wilson Loops}
\label{wilsonloops}


In this section we consider Wilson loop operators in the
gauge theory. 
The Wilson loop operator 
\eqn{wilsonlo}{
W({\cal C}) = \tr \left[ P \exp \left(i \oint_{\cal C} A  \right)
\right] } depends on a loop ${\cal
C}$ embedded in four dimensional space, and it involves the
path-ordered integral of the gauge connection along the contour. The
trace is taken over some representation of the gauge group; we will
discuss here only the case of the fundamental representation (see
\cite{Gross:1998gk} for a discussion of other representations).  From
the expectation value of the Wilson loop operator $\langle W({\cal C})
\rangle $ we can calculate the quark-antiquark potential. For this
purpose we consider a rectangular loop with sides of length $T$ and
$L$ in Euclidean space.  Then, viewing $T$ as the time direction, it
is clear that for large $T$ the expectation value will behave as
$e^{-TE}$ where $E$ is the lowest possible energy of the
quark-anti-quark configuration. Thus, we have
\eqn{rectangle}{
\langle W \rangle  \sim e^{ -T V(L)} ~,
}
where $V(L)$ is the quark anti-quark potential. For large $N$ and 
large $g_{YM}^2 N$, the AdS/CFT
correspondence maps the computation of $\langle W \rangle$ in the CFT
into a problem of finding a minimum surface in $AdS$ 
\cite{Maldacena:1998im,Rey:1998ik}. 

\subsection{Wilson Loops and Minimum Surfaces}
 
In QCD, we expect the Wilson loop to be related to the string running
from the quark to the antiquark.
%In a similar vein we expect that in this case 
%we will have a string,  a superstring
We expect this string to be analogous to the string in our configuration,
which is a superstring
which lives in ten dimensions, and which can stretch between two 
%So the string will go between two 
points on the boundary of $AdS$.  In order to motivate this
prescription let us consider the following situation. We start with
the gauge group $U(N+1)$, and we break it to $U(N)\times U(1)$ by
giving an expectation value to one of the scalars. This corresponds,
as discussed in section \ref{correspondence}, to having a D3 brane
sitting at some radial position $U$ in $AdS$, and
at a point on $S^5$.  The off-diagonal states, transforming in the
${\bf N}$ of $U(N)$, get a mass proportional to $U$, $m 
%= r / 2 \pi \alpha' 
= U / 2 \pi$.  So, from the point of view of the $U(N)$ gauge
theory, we can view these states as massive quarks, which act as a source
for the various $U(N) $ fields. Since they are charged they will act as a
source for the vector fields. In order to get a non-dynamical source
(an ``external quark'' with no fluctuations of its own, which will
correspond precisely to the Wilson loop operator) we need to take $m \to
\infty$, which means $U$ should also go to infinity. Thus, the string
should end on the boundary of AdS space.

These stretched strings will also act as a source for the scalar
fields. The coupling to the scalar fields can be seen qualitatively by
viewing the quarks as strings stretching between the $N$ branes and
the single separated brane. These strings will pull the $N$ branes and
will cause a deformation of the branes, which is described by the
scalar fields.  A more formal argument for this coupling is that these
states are BPS, and the coupling to the scalar (Higgs) fields is
determined by supersymmetry.  Finally, one can see this coupling
explicitly by writing the full $U(N+1)$ Lagrangian, putting in the
Higgs expectation value and calculating the equation of motion for the
massive fields
\cite{Maldacena:1998im}.
The precise definition of the Wilson loop operator corresponding to
the superstring will actually include also the field theory fermions,
which will imply some particular boundary conditions for the
worldsheet fermions at the boundary of $AdS$. However, this will not
affect the leading order computations we describe here.

So, the final conclusion is that the stretched strings
couple to the operator
\eqn{genwil}{
W({\cal C}) = \tr \left[ P \exp\left( \oint ( i A_\mu \dot x^\mu 
+ \theta^I \phi^I \sqrt{
\dot x^2 } ) d\tau  \right) \right], } 
where $x^\mu(\tau)$ is any parametrization of the loop and
$\theta^I$ ($I=1,\cdots,6$) is a unit vector in $\IR^6$ (the point on
$S^5$ where the string is sitting). This is the expression when
the signature of $\IR^4$ is Euclidean. In the Minkowski signature
case, the phase factor associated to the trajectory of the quark
has an extra factor ``$i$'' in front 
of $\theta^I$ \footnote{The difference
in the factor of $i$ between the Euclidean and the Minkowski cases
can be traced to the analytic continuation of 
$\sqrt{\dot x^2}$. A detailed derivation of (\ref{genwil})
can be found in \cite{Drukker:1999zq}.}. 

Generalizing the prescription of section \ref{correlators} for computing
correlation functions, the discussion above implies that in order to
compute the expectation value of the operator \genwil\ in $\cn=4$ SYM
we should consider the string theory partition function on
$AdS_5\times S^5$, with the condition that we have a string worldsheet
ending on the loop ${\cal C}$, as in figure
\ref{wi2} \cite{Rey:1998ik,Maldacena:1998im}.
In the supergravity regime, when $g_s N$ is large, the leading
contribution to this partition function will come from the area of the
string worldsheet. This area is measured with the $AdS$ metric, and it
is generally not the same as the area enclosed by the loop ${\cal
C}$ in four dimensions.


\begin{figure}[htb]
\begin{center}
\epsfxsize=.4in\leavevmode\epsfbox{wilone.eps}
\end{center}
\caption{
The Wilson loop operator creates a string worldsheet ending 
on the corresponding loop on the boundary of $AdS$.
}
\label{wi2}
\end{figure} 

The area as defined above is divergent. The divergence arises
from the fact that the string worldsheet is going all the way 
to the boundary of $AdS$. If we evaluate the area up to some 
radial distance $U=r$, we see that for large $r$ it diverges as
$ r  | {\cal C} |$, where $|{\cal C}|$ is the length
of the loop in the field theory \cite{Maldacena:1998im,Rey:1998ik}.
On the other hand, the perturbative computation in the field theory
shows that $\langle W \rangle$, for $W$ given by (\ref{genwil}), is
finite, as it should be since a divergence in the Wilson loop would
have implied a mass renormalization of the BPS particle.  The apparent 
discrepancy between the divergence of the area of the minimum surface 
in $AdS$ and the finiteness of the field theory computation can be 
reconciled by noting 
that the appropriate action for the string worldsheet 
is not the area itself but its Legendre transform with respect to
the string coordinates corresponding to $\theta^I$ 
and the radial coordinate $u$ \cite{Drukker:1999zq}. 
This is because these string coordinates obey
the Neumann boundary conditions rather than the Dirichlet conditions. 
When the loop is smooth, the Legendre
transformation simply subtracts the 
divergent term $r |{\cal C}|$, leaving the resulting action
finite. 

%This divergence can be interpreted as a 
%UV divergence in the field theory, and should be subtracted to 
%define the renormalized operator 
%\cite{Rey:1998ik,Maldacena:1998im,Susskind:1998dq,Peet:1998wn}. 
%We can also interpret the divergence as the contribution from
%interpreted as the contribution coming from 
%the mass of the (infinitely) massive 
%state that is propagating along ${\cal C}$. 

As an example let us consider a circular Wilson loop. Take ${\cal C}$
to be a circle of radius $a$ on the boundary, and let us work in the
\Poincare coordinates (defined in section \ref{adsgeom}).  We could find the
surface that minimizes the area by solving the Euler-Lagrange
equations. However, in this case it is easier to use conformal
invariance. Note that there is a conformal transformation in the field
theory that maps a line to a circle. In the case of the line, the
minimum area surface is clearly a plane that intersects the boundary
and goes all the way to the horizon (which is just a point on the
boundary in the Euclidean case). Using the conformal transformation to
map the line to a circle we obtain the minimal surface we want. It is,
using the coordinates (\ref{poinc}) for $AdS_5$,
\eqn{wilcir}{
\vec x = \sqrt{a^2 -z^2}  (\vec e_1 \cos\theta + \vec e_2 \sin \theta ),} 
where $\vec e_1$, $\vec e_2$ are two orthonormal vectors in four
dimensions (which define the orientation of the circle) and $ 0 \leq z
\leq a $.  We can calculate the area of this surface in $AdS$, and we
get a contribution to the action
\eqn{areacir}{
S \sim { 1 \over 2 \pi \alpha'} {\cal A } = { R^2 \over 2 \pi \alpha'}
\int d\theta \int_\epsilon^a { dz a \over z^2 } = {R^2 \over \alpha'}
( {a \over \epsilon} - 1 ),
}
where we have regularized the area by putting a an IR cutoff at 
$z=\epsilon$ in 
$AdS$, which is equivalent to a UV  cutoff in the field theory
\cite{Susskind:1998dq}.
Subtracting the divergent term we get
\eqn{reswil} {
\langle W \rangle 
\sim e^{ - S} \sim e^{ R^2/\alpha'} = e^{ \sqrt{ 4 \pi g_s N} }.
}
This is independent of $a$ as required by conformal invariance. 

We could similarly consider a ``magnetic'' Wilson loop, which is also
called a 't Hooft loop \cite{'tHooft:1979uj}. This case is related by
electric-magnetic duality to the previous case. Since we identify the
electric-magnetic duality with the $SL(2,\IZ)$ duality of type IIB
string theory, we should consider in this case a D-string worldsheet
instead of a fundamental string worldsheet. We get the same result as
in \reswil\ but with $ g_s \to 1/g_s$.

Using (\ref{rectangle}) it is possible to compute the quark-antiquark
potential in the supergravity approximation
\cite{Rey:1998ik,Maldacena:1998im}. In this case we consider a
configuration which is invariant under (Euclidean) time translations.
We take both particles to have the same scalar charge, which means
that the two ends of the string are at the same point in $S^5$ (one
could consider also the more general case
with a string ending at different points on $S^5$ \cite{Maldacena:1998im}). 
We put the quark at $x = -L/2$ and the
anti-quark at $x = L/2$. Here ``quark'' means an infinitely massive
W-boson connecting the $N$ branes with one brane which is (infinitely)
far away.  The classical action for a string worldsheet is
\eqn{string_action}{
S = { 1 \over 2 \pi \alpha'} \int d\tau d\sigma \sqrt{ \det (G_{MN} 
\partial_\alpha X^M \partial_\beta  X^N) },
}
where $G_{MN}$ is the Euclidean $AdS_5\times S^5$ metric.
 Note that
the factors of $\alpha'$ cancel out in (\ref{string_action}), as they should.
Since we are interested in a  static configuration we take
$\tau =t, ~ \sigma = x$, and then the action  becomes
\eqn{act}{
S = { T R^2 \over 2 \pi} \int_{-L/2}^{L/2} dx { \sqrt{ (\partial_x { z})^2+
1 } \over z^2}.
}
We need  to solve the Euler-Lagrange equations for this action.
Since the action does not depend on $x$ explicitly 
 the solution satisfies
\eqn{const}{
{1  \over  z^2 { \sqrt{ (\partial_x { z})^2+
1 } }}  = {\rm constant }.
 }
Defining $z_0$ to be the maximum value of $z(x)$, which by symmetry 
occurs at $x=0$, we find that the solution is\footnote{
All  integrals in this section
can be calculated in terms of elliptic or Beta  functions.}
\eqn{sol}{
 x = { z_0  } \int_{z/z_0}^1 
{ dy y^2 \over \sqrt{1- y^4 } },  
}
where $z_0$ is determined by the condition
\eqn{uzero}{
{ L \over 2 } = z_0
\int_0^1
{ dy y^2 \over  \sqrt{1- y^{4}  } }  =  
z_0 { \sqrt{2} \pi^{3/2} \over  \Gamma( 1/4)^2  }.
}
The qualitative form of the solution is shown in figure \ref{wi1}(b).
Notice that the  string quickly approaches $x =L/2$ 
for small $z$ (close to the boundary),
\eqn{approach}{
{ L\over 2} -x \sim { z^3  }~,~~~~~ z \to 0 ~ .
}
Now we  compute the total energy of the configuration.
We just plug in the solution \sol\ in \act , subtract the infinity
as explained above (which can be interpreted as the energy of two
separated massive quarks, as in figure \ref{wi1}(a)), and we find
\eqn{energy}{
E = V(L) = -  { 4 \pi^2 ( 2 g^2_{YM} N)^{1/2}  \over 
\Gamma({1 \over 4})^4 L}.
}
We see that the energy goes as $1/L$, a fact  which is determined by
 conformal invariance. 
Note that 
the energy is proportional to $(g_{YM}^2 N)^{1/2}$, 
as opposed to $g_{YM}^2 N$ which is the
perturbative result. 
This  indicates some screening of the charges at strong coupling. 
The above calculation makes sense for all distances $L$
when $g_s N$ is large, independently of the value of $g_s$.
Some subleading 
 corrections coming from quantum fluctuations of the worldsheet
were calculated in \cite{Forste:1999qn,Naik:1999bs,Greensite:1999wf}.


\begin{figure}[htb]
\begin{center}
\epsfxsize=4.0in\leavevmode\epsfbox{inout.eps}
\end{center}
\caption{ (a) Initial  configuration corresponding to two massive quarks
before we turn on their coupling to the $U(N)$ gauge theory.
(b) Configuration after we consider the 
coupling to the $U(N)$ gauge theory.
This configuration minimizes the action. The quark-antiquark energy
is given by the difference 
of the total length of the strings in (a) and (b).
}
\label{wi1}
\end{figure} 


In a similar fashion we could compute the potential between two
magnetic monopoles in terms of a D-string worldsheet, 
and the result will be the
same as \energy\ but with $ g_{YM} \to 4\pi/g_{YM}$.  One can also
calculate the interaction between a magnetic monopole and a quark. In
this case the fundamental string (ending on the quark) will attach to
the D-string (ending on the monopole), and they will connect to form a
$(1,1)$ string which will go into the horizon. The resulting potential
is a complicated function of $g_{YM}$
\cite{Minahan:1998xb}, but in the limit that $g_{YM}$ is small (but
still with $g_{YM}^2 N$ large) we get that the monopole-quark
potential is just $ 1/4$ of the quark-quark potential. This can be
understood from the fact that when $g$ is small the D-string is very
rigid and the fundamental string will end almost perpendicularly on
the D-string. Therefore, the solution for the fundamental string will
be half of the solution we had above, leading to a factor of $1/4$ in the
potential. Calculations of Wilson loops in the Higgs phase were done
in \cite{Minahan:1998xq}. 


Another interesting case one can study analytically is a surface 
near a cusp on $\IR^4$. In this case, the perturbative computation
in the gauge theory shows a logarithmic divergence with a coefficient
depending on the angle at the cusp. The area of the minimum surface
also contains a logarithmic divergence depending on 
the angle \cite{Drukker:1999zq}.  Other aspects of the gravity 
calculation of Wilson loops were discussed in
 \cite{Kogan:1998ti,Kogan:1999rw,Nojiri:1999gf,Zarembo:1999bu,%
Alvarez:1998dx}.

\subsection{Other Branes Ending on the Boundary}

We could also consider other branes that are ending at the boundary
\cite{Graham:1999pm}.  The simplest example would be a zero-brane
(i.e. a particle) of mass $m$. In Euclidean space a zero-brane
describes a one dimensional trajectory in anti-de-Sitter space which
ends at two points on the boundary.  Therefore, it is associated with
the insertion of two local operators at the two points where the
trajectory ends. In the supergravity approximation the zero-brane
follows a geodesic.  Geodesics in the hyperbolic plane (Euclidean AdS)
are semicircles. If we compute the action we get
\eqn{actizero}{
S = m \int ds = -2m R \int_\epsilon^a { a dz \over z \sqrt{a^2 - z^2 }},
}
where we took the distance between the two points at the boundary
 to be $L = 2a$
and  regulated the result. We find a logarithmic divergence when
$\epsilon \to 0$, proportional to $ \log(\epsilon/a) $. 
If we subtract the logarithmic divergence we get a residual dependence
on $a$. Naively we might have thought that (as in the previous 
subsection) the answer had
to be independent of $a$ due to conformal invariance. 
In fact, the dependence on $a$ is very important, since it leads to 
a result of the form
\eqn{res-zero}{
e^{-S} \sim e^{- 2 m R \log a } \sim { 1 \over a^{2 m R} },
}
which is precisely the result we expect for the two-point function of an
operator of dimension $\Delta = m R$. This is precisely the 
large $mR$ limit of the formula (\ref{dimenmass}), so we reproduce in
the supergravity limit the 2-point function described in section 
\ref{correlators}. 
In general, this sort of logarithmic 
divergence arises when the brane worldvolume is 
odd dimensional \cite{Graham:1999pm},
 and it implies that the expectation
value of the corresponding operator depends on the overall scale. 
In particular one could consider the ``Wilson surfaces'' that 
arise in the six dimensional $\cn=(2,0)$ theory which will be discussed
in section \ref{m5branes}. In that 
case one has to consider a two-brane, with a three
dimensional worldvolume,  ending on a two dimensional
surface on the boundary of $AdS_7$. Again, one gets a logarithmic term,
which is proportional to the rigid string action of the two 
dimensional surface living on the string in the $\cn=(2,0)$ field theory 
\cite{Berenstein:1998ij,Graham:1999pm}.

One can also compute correlation functions involving more than one
Wilson loop. To leading order in $N$ this will be just the product of
the expectation values of each Wilson loop. On general grounds one
expects that the subleading corrections are given by surfaces that end
on more than one loop. One limiting case is when the surfaces look
similar to the zeroth order surfaces but with additional thin tubes
connecting them. These thin tubes are nothing else than massless
particles being exchanged between the two string worldsheets
\cite{Gross:1998gk,Berenstein:1998ij}. We will discuss this further in
section \ref{adsqcd}.







% \end{document}






