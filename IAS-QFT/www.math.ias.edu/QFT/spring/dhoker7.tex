%% This is a plain TeX file
%%
\magnification=1200
\hsize=6.5 true in
\vsize=8.7 true in

\input amssym.def
\input amssym.tex

\font\boldtitlefont=cmb10 scaled\magstep2
\font\smallboldtitle=cmb10 scaled \magstep1
\font\sans=cmss10 scaled\magstep1

\footline={\hfil {\tenrm IV.\folio}\hfil}

\def\eps{{\varepsilon}}
\def\Eps{{\epsilon}}
\def\kap{{\kappa}}
\def\lam{{\lambda}}
\def\Lam{{\Lambda}}
\def\mynabla{{\nabla\!}}

\def\nspace{\lineskip=1pt\baselineskip=12pt%
     \lineskiplimit=0pt}
\def\dspace{\lineskip=2pt\baselineskip=18pt%
     \lineskiplimit=0pt}

\def\half{{\scriptstyle 1\over\scriptstyle 2}}
\def\w{{\mathchoice{\,{\scriptstyle\wedge}\,}
  {{\scriptstyle\wedge}}
  {{\scriptscriptstyle\wedge}}{{\scriptscriptstyle\wedge}}}}
\def\Le{{\mathchoice{\,{\scriptstyle\le}\,}
{\,{\scriptstyle\le}\,}
{\,{\scriptscriptstyle\le}\,}{\,{\scriptscriptstyle\le}\,}}}
\def\Ge{{\mathchoice{\,{\scriptstyle\ge}\,}
{\,{\scriptstyle\ge}\,}
{\,{\scriptscriptstyle\ge}\,}{\,{\scriptscriptstyle\ge}\,}}}
\def\plus{{\hbox{$\scriptscriptstyle +$}}}
\def\Item#1{\par%
     \smallskip\hang\indent\llap{\hbox to\parindent
     {#1\hfill\enspace}}\ignorespaces}

\def\im{{\rm Im}}  \def\Open{{\rm open}}
\def\Diff{{\rm Diff}}  \def\Closed{{\rm closed}}
\def\Map{{\rm Map}}  \def\spurious{{\rm spurious}}
\def\Met{{\rm Met}} \def\phys{{\rm phys}}
\def\diag{{\rm diag}}  \def\Vir{{\rm Vir}}
\def\spin{{\rm spin}} \def\Res{{\rm Res}}
\def\Null{{\rm null}} \def\mass{{\rm mass}}
\def\SO{{\rm SO}} \def\Tr{{\rm Tr\,}}
\def\SU{{\rm SU}} \def\tr{{\rm tr}}
\def\Weyl{{\rm Weyl}} \def\Ker{{\rm Ker}}
\def\Range{{\rm Range}} \def\SL{{\rm SL}}
\def\Det{{\rm Det}} \def\re{{\rm Re}}
\def\dist{{\rm dist}} \def\PSL{{\rm PSL}}
\def\Vol{{\rm Vol}} \def\ghosts{{\rm ghosts}}
\def\Fock{{\rm Fock}} \def\BRST{{\rm BRST}}
\def\inv{{\rm inv}}

\def\fbar{\bar{f}}  \def\mubar{\bar{\mu}}
\def\barh{\bar{h}}  \def\gammabar{\bar{\gamma}}
\def\kbar{\bar{k}}  \def\lambar{\bar{\lambda}}
\def\mbar{\bar{m}}  \def\phibar{\bar{\phi}}
\def\wbar{\bar{w}}  \def\etabar{\bar{\eta}}
\def\vbar{\bar{v}}  \def\partialbar{\bar{\partial}}
\def\xbar{\bar{x}}  \def\cbar{\bar{c}}
\def\zbar{\bar{z}}  \def\bbar{\bar{b}}
\def\Abar{\bar{A}}
\def\Kbar{\bar{K}}
\def\Pbar{\bar{P}}
\def\Sbar{\bar{S}}
\def\Tbar{\bar{T}}

\def\scrFbc{{\scrF^{(bc)}}}
\def\scrFbcbar{{\scrF^{(\bbar\cbar)}}}

\def\ghat{\hat{g}}
\def\muhat{\hat{\mu}}

\def\htil{\tilde{h}}
\def\xtil{\tilde{x}}
\def\Dtil{\widetilde{D}}
\def\Ltil{\tilde{L}}
\def\Ntil{\widetilde{N}}
\def\Ttil{\widetilde{T}}
\def\scrFtil{\widetilde{\scrF}}
\def\epstil{\tilde{\eps}}
\def\psitil{\tilde{\psi}}

\def\dbR{{\Bbb R}}
\def\dbZ{{\Bbb Z}}

%These two files (in this order!!) are necessary
%in order to use AMS Fonts 2.0 with Plain TeX

\input amssym.def
\input amssym.tex

%Capital roman double letters(Blackboard bold)
\def\db#1{{\fam\msbfam\relax#1}}

\def\dbA{{\db A}} \def\dbB{{\db B}}
\def\dbC{{\db C}} \def\dbD{{\db D}}
\def\dbE{{\db E}} \def\dbF{{\db F}}
\def\dbG{{\db G}} \def\dbH{{\db H}}
\def\dbI{{\db I}} \def\dbJ{{\db J}}
\def\dbK{{\db K}} \def\dbL{{\db L}}
\def\dbM{{\db M}} \def\dbN{{\db N}}
\def\dbO{{\db O}} \def\dbP{{\db P}}
\def\dbQ{{\db Q}} \def\dbR{{\db R}}
\def\dbS{{\db S}} \def\dbT{{\db T}}
\def\dbU{{\db U}} \def\dbV{{\db V}}
\def\dbW{{\db W}} \def\dbX{{\db X}}
\def\dbY{{\db Y}} \def\dbZ{{\db Z}}

\font\teneusm=eusm10  \font\seveneusm=eusm7 
\font\fiveeusm=eusm5 
\newfam\eusmfam 
\textfont\eusmfam=\teneusm 
\scriptfont\eusmfam=\seveneusm 
\scriptscriptfont\eusmfam=\fiveeufm 
\def\scr#1{{\fam\eusmfam\relax#1}}


%Upper-case Script Letters:

\def\scrA{{\scr A}}   \def\scrB{{\scr B}}
\def\scrC{{\scr C}}   \def\scrD{{\scr D}}
\def\scrE{{\scr E}}   \def\scrF{{\scr F}}
\def\scrG{{\scr G}}   \def\scrH{{\scr H}}
\def\scrI{{\scr I}}   \def\scrJ{{\scr J}}
\def\scrK{{\scr K}}   \def\scrL{{\scr L}}
\def\scrM{{\scr M}}   \def\scrN{{\scr N}}
\def\scrO{{\scr O}}   \def\scrP{{\scr P}}
\def\scrQ{{\scr Q}}   \def\scrR{{\scr R}}
\def\scrS{{\scr S}}   \def\scrT{{\scr T}}
\def\scrU{{\scr U}}   \def\scrV{{\scr V}}
\def\scrW{{\scr W}}   \def\scrX{{\scr X}}
\def\scrY{{\scr Y}}   \def\scrZ{{\scr Z}}

\def\gr#1{{\fam\eufmfam\relax#1}}

%Euler Fraktur letters (German)
\def\grA{{\gr A}}	\def\gra{{\gr a}}
\def\grB{{\gr B}}	\def\grb{{\gr b}}
\def\grC{{\gr C}}	\def\grc{{\gr c}}
\def\grD{{\gr D}}	\def\grd{{\gr d}}
\def\grE{{\gr E}}	\def\gre{{\gr e}}
\def\grF{{\gr F}}	\def\grf{{\gr f}}
\def\grG{{\gr G}}	\def\grg{{\gr g}}
\def\grH{{\gr H}}	\def\grh{{\gr h}}
\def\grI{{\gr I}}	\def\gri{{\gr i}}
\def\grJ{{\gr J}}	\def\grj{{\gr j}}
\def\grK{{\gr K}}	\def\grk{{\gr k}}
\def\grL{{\gr L}}	\def\grl{{\gr l}}
\def\grM{{\gr M}}	\def\grm{{\gr m}}
\def\grN{{\gr N}}	\def\grn{{\gr n}}
\def\grO{{\gr O}}	\def\gro{{\gr o}}
\def\grP{{\gr P}}	\def\grp{{\gr p}}
\def\grQ{{\gr Q}}	\def\grq{{\gr q}}
\def\grR{{\gr R}}	\def\grr{{\gr r}}
\def\grS{{\gr S}}	\def\grs{{\gr s}}
\def\grT{{\gr T}}	\def\grt{{\gr t}}
\def\grU{{\gr U}}	\def\gru{{\gr u}}
\def\grV{{\gr V}}	\def\grv{{\gr v}}
\def\grW{{\gr W}}	\def\grw{{\gr w}}
\def\grX{{\gr X}}	\def\grx{{\gr x}}
\def\grY{{\gr Y}}	\def\gry{{\gr y}}
\def\grZ{{\gr Z}}	\def\grz{{\gr z}}

\overfullrule=0pt

\parindent=18pt
\line{\sans --- DRAFT ---\hfill{\rm IASSNS-HEP-97/72}}

\bigskip\bigskip
\centerline{\boldtitlefont Lecture 7}
\medskip
\centerline{\smallboldtitle IV. Faddeev--Popov Ghosts ---
BRST Quantization}

\medskip
\centerline{Eric D'Hoker}

\frenchspacing

\dspace
\bigskip
It is standard practice to recast the Faddeev-Popov
determinant, which arose from fixing $\Diff(\Sigma)$
gauge invariance, in terms of ghost fields.
(See also Faddeev's lectures.)
One introduces Grassmann-valued {\it ghost fields}
$(c,\cbar)=(c^z(dz)^{-1}$, $c^{\zbar}(d\zbar)^{-1})$
and {\it anti-ghost} fields $(b,\bbar)=(b_{zz}(dz)^2,
b_{\zbar\zbar}(d\zbar)^2)$ transforming as the gauge
transformation vector fields $(v,\vbar)$ and as the
metric deformations 
$(\delta g_{zz},\delta g_{\zbar\zbar})$, respectively.
It is convenient to study such $b$-$c$ systems for general
weight first, since we shall make use of such systems as
well: $c\in K^{1-n}; b\in K^n$.
For now $n$ is an integer, but later $n$ will also be
allowed to be a half-integer.

\bigskip\noindent
\Item{\bf A.} {\bf Determinants and $b$-$c$ systems}

\smallskip
We therefore generalize by considering a system of a pair of
Grassmann-valued fields $(c,\cbar)=(c(dz)^{1-n}$,
$\cbar(d\zbar)^{1-n})$ and $(b,\bbar)=(b(dz)^n,
\bbar(d\zbar)^n)$, so that the above case is $n=2$.
In this notation, $b$ and $c$ are independent whereas
$\bbar$, $\cbar$ are their complex conjugates.
The action is first order in derivatives:
$$
S_n(b,c)={1\over 2\pi}\int_\Sigma
d\mu_g\left(
b\nabla_{(1-n)}^zc+\bbar
\nabla_{(n-1)}^{\zbar}\cbar\right)\,\,.\eqno{(4.1)}
$$
We have the following {\it Theorem}:
$$
\eqalign{
Z_{(n)}^-(g)
\vert &\det\,\phi_j(z_k)\vert^2\,\,\vert\det\,\psi_a
  (w_b)\vert^2\cr
&=\int D(b\bbar)D(c\cbar)e^{-S_n(b,c)}
\prod\limits_{j=1}^{\kern.3 cm N_{(n)}^-}b(z_j)\bbar(z_j)
\prod\limits_{a=1}^{\kern.5 cm N_{(n-1)}^{-}}
c(w_a)\cbar(w_a)\,\,,\cr}
\eqno{(4.2)}
$$
where the measures $D(b\bbar)$ and $D(c\cbar)$ are
obtained from the $L^2$ norms on $K^n$, $\Kbar^n$,
$K^{(1-n)}$ and $\Kbar^{(1-n)}$; 
$\phi_j\in\Ker\,\nabla_{(n)}^z$,
$\psi_a\in\Ker\,\nabla_{(1-n)}^z$ and $N_{(n)}^{\pm}=\dim\,
\Ker\,\Delta_{(n)}^{\pm}$.
Note that for $n\Ge 2$ there are no $\psi_\alpha$'s
whereas for $h=0$ there are no $\phi_j$'s.

We begin the proof by recalling a finite-dimensional Grassmann
integral, over variables $b_i$, $c_i$, $i=1,\ldots,N$,
such that $\{b_i,b_j\}=\{c_i,c_j\}=\{b_i,c_j\}=0$.
We use the standard Grassmann integration rules:
$$
\int dc_i\,c_j=\delta_{ij}\qquad\qquad
\int db_i\,b_j=\delta_{ij}\eqno{(4.3)}
$$
and $\int dc_i\,1=\int db_i\,1=0$.
Then, for any matrix $M_{ij}$, we have
$$
\prod\limits_{i=1}^N\int db_idc_i\,e^{b_i
M_{ij}c_j}=\det\,M\,\,.
\eqno{(4.4)}
$$

To prove the Theorem, we begin by expanding $b$ and $c$ in
an orthonormal basis of eigenfunctions of $\Delta_{(n)}^-$
and $\Delta_{(1-n)}^-$ as follows.
Let $\Psi_\alpha$ be an orthonormal basis of eigenfunctions
of $\Delta_{(1-n)}^-$ corresponding 
to the non-zero eigenvalues
$\lam_\alpha^2$, and let $\psi_a$ be an orthonormal basis of
zero modes of $\Delta_{(1-n)}^-$.
We have
$$
\cases{
\Delta_{(1-n)}^-\Psi_\alpha=\lam_\alpha^2\Psi_\alpha &\cr
\noalign{\medskip}
\Delta_{(1-n)}^-\psi_a=0\,\,,\cr}
\eqno{(4.5)}
$$
and we define $\Phi_\alpha$ by
$$
\cases{\sqrt{2\,\,}\nabla_{(1-n)}^z\Psi_\alpha=\lam_\alpha
  \Phi_\alpha &\cr
\noalign{\medskip}
-\sqrt{2\,}\nabla_z^{(1-n)}\Phi_\alpha=\lam_\alpha
   \Psi_\alpha\,\,,\cr}\eqno{(4.6)}
$$
so that $\Phi_\alpha$ provides an orthonormal basis of
eigenfunctions of $\Delta_{(-n)}^{\plus}$ for non-zero
eigenvalues.
Let $\phi_j$ be an orthonormal basis for 
$\Ker\,\nabla_z^{(-n)}$.
As a result, $(g_{z\zbar})^n\Phi_\alpha^*$ together with
the zero modes $(g_{z\zbar})^n\phi_j^*$ provides an
orthonormal basis of eigenfunctions for $\Delta_{(n)}^-$.
Thus, we have the decompositions
$$
\eqalign{
c &=\sum\limits_{\alpha}c_\alpha\Psi_\alpha+\sum\limits_{a}
  c_a\psi_a\cr
b &=\sum\limits_{\alpha}b_\alpha(g_{z\zbar})^n
\Phi_\alpha^*
+\sum\limits_{j}b_j(g_{z\zbar})^n\phi_j^*\,\,.\cr}\eqno{(4.7)}
$$
The $L^2$ norm then induces the following measures:
$$
\cases{
D(b\bbar)=\prod\limits_{\alpha}db_\alpha\,d\bbar_\alpha
  \prod\limits_{j}d\,b_j\,\,d\,\bbar_j\cr
\noalign{\medskip}
D(c\cbar)=\prod\limits_{\alpha}dc_\alpha\, d\cbar_\alpha
  \prod\limits_{a}dc_a\,d\cbar_a\cr} \eqno{(4.8)}
$$
and it is straightforward to evaluate the action:
$$
\eqalign{
S_n(b,c) &={1\over 2\pi}\sum\limits_{\alpha,\beta}b_\alpha
 c_\beta\int\nolimits_{\Sigma}d\mu_g (g_{z\zbar})^n
  \Phi_\alpha^*\nabla_{(1-n)}^z\Psi_\beta +c.c.\cr
&={1\over 2\pi\sqrt{2\,\,}}\sum\limits_{\alpha,\beta}
  b_\alpha c_\beta\lam_\beta\int\nolimits_{\Sigma}
  d\mu_g (g_{z\zbar})^n\Phi_\alpha^*\Phi_\beta+c.c\cr
&={1\over 2\pi\sqrt{2\,\,}}\sum\limits_{\alpha}(b_\alpha
  c_\alpha\lam_\alpha+\bbar_\alpha \cbar_\alpha
  \lam_\alpha)\,\,.\cr}\eqno{(4.9)}
$$
Notice that the zero mode coefficients $b_j$, $c_a$ do not
enter the action.
Modulo a constant $\left({1\over
2\pi\sqrt{2\,}}\right)^{\zeta(0)}$ coming from the
$\zeta$-function regularization,
$$
\int\prod\limits_{\alpha}db_\alpha d\bbar_\alpha
  dc_\alpha d\cbar_\alpha\,e^{-S_n(b,c)}=
\prod\limits_{\alpha}\lam_\alpha^2=\Det'\Delta_{(1-n)}^-\,\,.
\eqno{(4.10)}
$$
The remaining finite product of factors can only involve
the zero modes, since the number of factors of $b$ and $c$
precisely equals the number of zero-mode integrations.
$$
\eqalign{
\int\prod\limits_{a}dc_ad\cbar_a &\prod\limits_{j}
  db_jd\bbar_j\prod\limits_{j=1}^{N_{(n)}^-}
  b(z_j)\bbar(z_j)\prod\limits_{a=1}^{N_{(1-n)}^-}
  c(w_a)\cbar(w_a)\cr
&=\vert\det\,\phi_j(z_k)\vert^2\,\,
  \vert\det\,\psi_a(w_b)\vert^2\,\,.\cr}
\eqno{(4.11)}
$$
Recall that our bases $\phi_j$, $\psi_a$ had been chosen
orthonormal.
In an arbitrary basis, the result would be
$$
{\vert\det\,\phi_j(z_k)\vert^2
  \over \det(\phi_j,\phi_k)}\,\,
{\vert\det(\psi_a(w_b))\vert^2\over \det(\psi_a,
  \psi_b)}\,\,.\eqno{(4.12)}
$$
Putting all together, we obtain the result in the Theorem
by using the definition of the determinant combination
$Z_{(n)}^-(g)$.

Notice that the functional integral without $b$ or $c$
insertions,
$$
\int D(b\bbar)D(c\cbar)e^{-S_n(b,c)}\,\,,
$$
vanishes because of the existence of zero modes which do
not enter the action at all.

\vfill\eject

\noindent
\Item{\bf B.} {\bf Ghost representation of the
Faddeev-Popov determinant}

\smallskip
For bosonic string theory, $n=2$ is the suitable weight,
and we have the following general form for the amplitudes
$$
A=\sum\limits_{h=0}^\infty e^{-\chi\Phi_0}
  \int\nolimits_{\scrM_h}dm_j d\mbar_j\int
  D(b\bbar)D(c\cbar)e^{-S_2(b,c)}\prod\limits_{j}
  \vert(\mu_j,b)\vert^2
\int\limits_{\Map(\Sigma;M)}Dx\,\scrO\,e^{-S_m(x)}
\eqno{(4.13)}
$$
where
$$
\scrO=\left\{
\matrix{
\bullet\quad &\prod\limits_{a=N-2}^N c(z_a)\cbar(z_a)V_1\ldots
V_{N-3}W(z_{N-2})W(z_{N-1})W(z_N)\hfill 
  &(N\Ge 3;\,h=0)\hfill\cr
\noalign{\medskip}
\bullet\quad &0 \hfill   &(N=0,1,2;\,h=0)\hfill\cr
\noalign{\medskip}
\bullet\quad &c(z_N)\cbar(z_N)V_1\ldots V_{N-1}W(z_N)\hfill
  &(N\Ge 1;\,h=1)\hfill\cr
\noalign{\medskip}
\bullet\quad &V_1\ldots V_N\,\,,\hfill    &(h\Ge 2)\hfill\cr}
\right.
\eqno{(4.14)}
$$
and $S_2(b,c)$ is the action for the ghost fields in the
case $n=2$.
This representation is valid for any matter action $S_m$,
and any set of vertex operators $V_i$.
The fields $c$ and $b$ are called the Faddeev-Popov ghosts
and anti-ghosts.

\bigskip\noindent
\Item{\bf C.} {\bf Conformal Field Theory of the $b$-$c$
system}

\smallskip
{}From the Weyl transformation law of the determinant
combinations $Z_{(n)}^-(g)$, and the Weyl invariance of
$\det\phi_j(z_k)$ and $\det\psi_a(w_b)$, we know that
the $b$-$c$ system is a conformal field theory with central
charge
$$
-2c_{(n)}^-=-2(6n^2-6n+1)\,\,.
$$
In particular, we confirm that for the ghost system of the
bosonic string (i.e. $n=2$) the central charge is
$c_{\ghosts}=-26$, cancelling
the central charge of the matter part.
We now study this conformal field theory for any
$n\in\half\dbZ$, assuming that a spin structure has been
chosen on $\Sigma$.
In particular, for $n={1\over 2}$, we have a Dirac
fermion $\psitil_{\plus}=b$, $\psi_{\plus}=c$; 
$\psitil_-=\bbar$, $\psi_-=\cbar$ (in which case $b$, $c$
are in the same space of Dirac spinors), or 
two Majorana spinors:
$(\psi_{\plus},\psitil_{\plus})$ and $(\psi_-,\psitil_-)=
(\psi_{\plus},\psitil_{\plus})^*$.
(See Witten's lectures.)

{}From the classical action, we obtain the stress tensor by
varying with respect to $\delta g^{zz}$ and using
$$
\delta\nabla_{(1-n)}^z={1\over 2} \delta
g^{zz}\nabla_z^{(1-n)}+
{1-n\over 2}\nabla_z (\delta g^{zz})\,\,.\eqno{(4.15)}
$$
We find for the classical stress tensor:
$$
\eqalign{
T^{(bc)}(z)=T_{zz}^{(bc)} &=(1-n)\nabla_z bc-nb\nabla_z c\cr
&=(1-n)(\partial_z b-n\Gamma
b)c-nb(\partial_zc-(1-n)\Gamma c)\cr
&=(1-n)(\partial_z b)c-nb\partial_zc\,\,.\cr}\eqno{(4.16)}
$$

Using Weyl rescaling, we may study the system on a surface
with locally flat metric $\ghat=2\vert dz\vert^2$.
At the same time, covariant derivatives become just
Cauchy-Riemann operators:
$\nabla_{(n)}^z$ becomes $\partialbar_{(n)}$ and
$\nabla_z^{(n)}$ becomes $\partial_{(n)}$ ($\partialbar$
and $\partial$ for short).
The action is then
$$
S_n(b,c)={1\over 2\pi}\int\nolimits_{\Sigma}d^2\,z
(b\partialbar c+\bbar\partial\cbar)\,\,.\eqno{(4.17)}
$$
The field equations are $\partialbar c=\partialbar b=0$,
$\partial\cbar=\partial\bbar=0$.
The OPE of $b$ and $c$ is given by the Green function of
$\partialbar$:
$$
b(z)c(w)\sim{1\over z-w}\sim c(z)b(w)\,\,.\eqno{(4.18)}
$$
The quantum stress tensor may now be defined by
subtracting the OPE singularities from the classical
stress tensor.
The holomorphic stress tensor is defined by
$$
T^{(bc)}(z)=\lim\limits_{w\to z}\left\{
(1-n)\partial b(z)c(w)-nb(z)\partial c(w)+
{1\over (z-w)^2}\right\}\,\,.
\eqno{(4.19)}
$$
One immediately verifies
$$
\matrix{
{\displaystyle T^{(bc)}(z)T^{(bc)}(w)\hfill} 
&{\displaystyle \sim\hfill} &{\displaystyle -{(6n^2-6n+1)\over
 (z-w)^4}} &\displaystyle{+\,\,{2\over 
  (z-w)^2}T^{(bc)}(w)+{1\over z-w}
  \partial_w T_{(w)}^{(bc)}\hfill}\cr
\noalign{\medskip}
{\displaystyle T^{(bc)}(z)c(w)\hfill} 
&{\displaystyle \sim} &{\displaystyle {1-n\over (z-w)^2}c(w)} 
&{\displaystyle +\,\,{1\over z-w} \partial_w c(w)}\hfill\cr
\noalign{\medskip}
{\displaystyle T^{(bc)}(z)b(w)\hfill} 
&{\displaystyle \sim\hfill} 
&{\displaystyle {n\over (z-w)^2}b(w)}
&{\displaystyle +\,\,{1\over (z-w)}
  \partial_w b(w)\,\,.}\hfill\cr}
\eqno{(4.20)}
$$
(There is no singularity in the expansion of $b(z)b(w)$
or $c(z)c(w)$.)
The classical action $S_n$ is invariant under
$U(1)_{bc}\times U(1)_{\bbar\cbar}$ phase rotations:
$$
\cases{
c\to e^{i\theta}c &\cr
\noalign{\medskip}
b\to e^{-i\theta} b &\cr}\qquad\qquad
\cases{
\cbar\to e^{-i\theta'}\cbar &\cr
\noalign{\medskip}
\bbar\to e^{i\theta'}\bbar &\cr}
\eqno{(4.21)}
$$
Only the diagonal subgroup with $\theta=\theta'$ 
leaves the measure $D(b\bbar)D(c\cbar)$ invariant, and
results in a quantum symmetry.

The current associated with $U(1)_{bc}$ is given by
$$
j_z^{(bc)}= \,\,:cb:\,\, =\lim\limits_{w\to z}\left\{c(w)
  b(z)-{1\over w-z}\right\}\,\,,\eqno{(4.22)}
$$
and we find the OPE's:
$$
j_z^{(bc)}c(w)\sim-{1\over z-w}\,c(w)\,\,;
\qquad\qquad j_z^{(bc)}b(w)\sim{1\over z-w}\,b(w)\,\,.
\eqno{(4.23)}
$$
The current $j_z^{(bc)}$ has an anomaly, as can be
established from the fact that $j_z^{(bc)}$ does not
transform as a tensor (for $n\not={1\over 2}$):
$$
T^{(c)}(z)j_w^{(bc)}\sim{1-2n\over (z-w)^3}+
{1\over (z-w)^2}j_w^{(bc)}+{1\over z-w}
\partial_w j_w^{(bc)}\,\,.\eqno{(4.24)}
$$
Thus, under analytic diffeomorphisms $z\to z'=f(z)$, we
have
$$
j'{^{(bc)}}(z')dz'=j^{(bc)}(z)dz+\left(n-{1\over 2}\right)d
(\ln\,f'(z))\eqno{(4.25)}
$$
Note that there is no anomaly for the case of spinors
when $n=1/2$.
Also, on a worldsheet with general metric $g$, $j^{(bc)}$ is
not conserved; instead
$\nabla^zj_z^{(bc)}=$
$\left(\half-n\right)R_g$.

\bigskip\noindent
\Item{\bf D.} {\bf Bosonization of the $b$-$c$ system}

\smallskip
In Witten's lectures, a Bose-Fermi correspondence was
established between a $2$-dimen-\break
sional Dirac fermion and a
real scalar boson field.
The Dirac fermion corresponds here to the case $n=\half$.
We will generalize the Bose-Fermi correspondence to the
case of any $n\in\half\dbZ$.
We begin by noticing that the $b$-$c$ stress tensor can be
expressed in terms of the $U(1)_{bc}$ current only:
$$
T^{(bc)}(z)={1\over 2} j^{(bc)}(z)^2+\left(n-{1\over 2}\right)
\partial_z j^{(bc)}(z)\,\,,\eqno{(4.26)}
$$
where the product of operators is always renormalized \`a
la OPE.
Using
$$
j^{(bc)}(z)j^{(bc)}(w)\sim{1\over (z-w)^2}\,\,,\eqno{(4.27)}
$$
it is straightforward to verify that $T^{(bc)}$ satisfies
all identities listed above.

We now introduce a scalar field $\phi$, with action (for
general metric on $\Sigma$)
$$
S_Q(x)={1\over 4\pi}\int\nolimits_{\Sigma}d\mu_g\left\{
{1\over 2} g^{mn}\partial_m\phi \partial_n\phi+
  QR_g\phi\right\}\,\,.\eqno{(4.28)}
$$
This system was studied extensively in Problem Set \#1.
Let us consider it only on locally flat metrics.
Its central charge is $c=1+3Q^2$, and there is a current
$\partial_z\phi$.
Identifying the two systems will require that they have
equal central charge:
$$
1+3Q^2=-12n^2+12n-2\qquad
\hbox{or}\qquad Q=i(2n-1)\,\,.\eqno{(4.29)}
$$
(The second solution $Q=-i(2n-1)$ is equivalent upon
changing the sign of $\phi$.)
The identification of the currents gives
$$
\eqalign{
j^{(bc)}(z)&=i\partial_z\phi\cr
T^{(bc)}(z)=-{1\over 2} (\partial_z &\phi)^2+{1\over 2}\,
Q\partial_z^2\phi\,\,.\cr}
\eqno{(4.30)}
$$
The fields $b$, $c$ and $\bbar$, $\cbar$ may be identified
in terms of the {\it chiral boson fields}
$\phi_+(z)$,$\phi_-(\zbar)$, with
$\phi(z,\zbar)=\phi_+(z)+\phi_-(\zbar)$ by
$$
\cases{
c(z)=e^{i\phi_+(z)} &\cr
\noalign{\medskip}
b(z)=e^{-i\phi_+(z)} &\cr}\qquad\qquad
\cases{
\cbar(\zbar)=e^{+i\phi_-(\zbar)} &\cr
\noalign{\medskip}
\bbar(\zbar)=e^{-i\phi_-(\zbar)}\,\,. &\cr}
\eqno{(4.31)}
$$
The $U(1)_{bc}\times U(1)_{\bbar\cbar}$ transformations
now simply shift $\phi_+$ and $\phi_-$ by $\theta$ and
$-\theta'$, respectively.

It is now easy to see that the number 
$(N_b, N_c)$ of $b$ and $c$ operator insertions yielding non-zero
correlation functions is constrained by the
Riemann-Roch-Atiyah-Singer theorem:
$$
N_b-N_c-{i\over 2}Q\chi=0\qquad\hbox{or}\qquad
N_b-N_c=\left({1\over 2}-n\right)\chi\,\,,
$$
where $\chi$ is the Euler number of $\Sigma$.

\bigskip\noindent
\Item{\bf E.} {\bf The $b$-$c$  Fock Space}

\smallskip
The full Fock space is obtained as a tensor product of
the $bc$ Fock space $\scrF^{(bc)}$ and the $\bbar\cbar$
Fock space $\scrF^{(\bbar\cbar)}$:
$$
\Fock(b,c,\bbar,\cbar)=\scrF^{(bc)}\otimes
\scrF^{(\bbar\cbar)}\,\,.
\eqno{(4.32)}
$$
We proceed to construct $\scrFbc$, and $\scrFbcbar$ is
constructed analogously.
To do so, we consider the system on an annulus $\Sigma$,
centered at $z=0$, with flat metric $g=2\vert dz\vert^2$,
and we expand the holomorphic fields $b(z)$ and $c(z)$ in
a Laurent series:
$$
\eqalign{
c(z) &=\sum\limits_{r}c_r\,z^{-r-(1-n)}\cr
b(z) &=\sum\limits_{r}b_r\,z^{-r-n}\cr}\qquad\qquad\qquad
r\in\dbZ\,\,.
\eqno{(4.33)}
$$
The shifts in the powers of $z$ reflect the conformal
weights of the fields.
The OPE of $b$ and $c$ yields
$$
\{b_r,b_s\}=\{c_r,c_s\}=0\qquad
\{b_r,c_s\}=\delta_{r+s,0}\,\,.
\eqno{(4.34)}
$$
The stress tensor (renormalized as in IV.7) yields the
Virasoro generators
$$
\eqalign{
L_m^{(bc)} &=\sum\limits_{r\in\dbZ}(mn-r)b_r
c_{m-r}\qquad
m\not=0\cr
L_0^{(bc)} &=\sum\limits_{r=1}^\infty
r(b_{-r}c_r+c_{-r}b_r)
  -{1\over 2}\,n(n-1)\,\,.\cr}
\eqno{(4.35)}
$$
The  $U(1)_{bc}$ current gives the $bc$ number operator:
$$
U=\oint {dz\over 2\pi i}j_z^{(bc)}=
\sum\limits_{r=1}^\infty (c_{-r}b_r-b_{-r}c_r)+
{1\over 2}(c_0b_0-b_0c_0)+n-{1\over 2}\,\,.
\eqno{(4.36)}
$$
(The constant shifts in $L_0^{(bc)}$ and $U_0$ require
great care, and are fundamental in the sequel.)

The polarization here is again inherited from Minkowski
worldsheet signature:
$$
b_r^{\dagger}=b_{-r}\qquad
c_r^{\dagger}=c_{-r}\,;\qquad\qquad\qquad
r\in\dbZ\,\,.
\eqno{(4.37)}
$$
The ground state of the $b$-$c$ system is doubly degenerate,
since $b_0$ and $c_0$ commute with $L_0$ (in fact they do
not enter into the expession for $L_0$), and since $b_0$,
$c_0$ generate an algebra $\{b_0,c_0\}=1$, $b_0^2=c_0^2=0$
of which the ground state is a two-dimensional
representation.
We define the basis vectors by $\left.\mid\uparrow\right>$
and $\left.\mid\downarrow\right>$, with
$$
b_r,c_r\left.\mid\uparrow\right>\quad
\left.\mid\downarrow\right>=0\qquad\qquad\qquad
r>0\,\,.
\eqno{(4.38)}
$$
We choose the states so that
$$
\eqalign{
c_0\left.\mid\uparrow\right> &=0\cr
b_0\left.\mid\downarrow\right> &=0\,\,.\cr}
\eqno{(4.39)}
$$
It follows that
$$
\matrix{
b_0\left.\mid\uparrow\right>
&=\left.\mid\downarrow\right> &\qquad\qquad
U\left.\mid\uparrow\right>
&=n\left.\mid\uparrow\right>\hfill\cr
\noalign{\medskip}
c_0\left.\mid\downarrow\right>
&=\left.\mid\uparrow\right> &\qquad\qquad
  U\left.\mid\downarrow\right>
&=(n-1)\left.\mid\downarrow\right>\,\,.\cr}
\eqno{(4.40)}
$$
It is natural to choose the following inner products
$$
\left<\downarrow\mid\uparrow\right>
=\left<\uparrow\mid\downarrow\right>=1
\eqno{(4.41)}
$$
(Notice that $\left<\uparrow\mid\uparrow\right>=
\left<\downarrow\mid\downarrow\right>=0$ by their
properties under $b_0$ and $c_0$.)

The $b$-$c$ Fock space is naturally obtained by applying
the creation operators $\{b_{-r},c_{-r}\}_{r\in \dbN}$ to
either one of the ground states
$\left.\mid\uparrow\right>$ or
$\left.\mid\downarrow\right>$.
The inner product thus inherited on $\scrFbc$ is not
positive definite.

\bigskip\noindent
\Item{\bf F.} {\bf BRST Quantization}

\smallskip
We now return to string theory, where the relevant
$b$-$c$ system is the one with $n=2$, in the notation
used above.
The combined matter-ghost action is (in a locally flat
$g$)
$$
S={1\over 2\pi}\int d^2z \left({1\over 2} \partial x\cdot
\partialbar x+b\partialbar c+\bbar\partial\cbar\right)\,\,.
\eqno{(4.42)}
$$
The combined stress tensor (we denote the $x$ stress
tensor by $T^{(x)}$)
$$
T(z)=T^{(x)}(z)+T^{(bc)}(z)
\eqno{(4.43)}
$$
has zero central charge in $D=26$ and is thus a primary
field of weight $(2,0)$.
The associated Virasoro generators are
$$
\left\{
\eqalign{
L_m &=\sum\limits_{r\in\dbZ}(2m-r)b_rc_{m-r}+{1\over 2}
  \sum\limits_{r\in\dbZ}x_r\cdot x_{m-r}\quad
  (m\not=0)\cr
L_0 &={1\over 2} p^2-1+\sum\limits_{r=1}^\infty
 r(x_{-r} x_r+b_{-r}c_r+c_{-r}b_r)\cr}
\right.
\eqno{(4.44)}
$$
The second term $(-1)$ in the second equation arises from
the ghost stress tensor on p. IV. 8 for $n=2$.
The Virasoro algebra of $L_m$ reduces to the Witt algebra
(i.e. with zero central charge)
$$
[L_\ell,L_m]=(\ell-m)L_{\ell+m}
\eqno{(4.45)}
$$
Recall the ghost number
$$
U=1+c_0b_0+\sum\limits_{r=1}^\infty
(c_{-r}b_r-b_{-r}c_r)
\eqno{(4.46)}
$$
and recall that the round states
$\left.\mid\uparrow\right>$ and
$\left.\mid\downarrow\right>$ have ghost number 2 and 1,
respectively.

The full Fock space of states of the closed bosonic
string, including the ghost states, is
$$
\scrF_{\Closed}=\left(\,\,\bigoplus\limits_{k\in\dbR^{26}}
\scrF_k\otimes\scrFtil_k\right)\otimes
\scrFbc\otimes\scrF^{(\bbar\cbar)}
\eqno{(4.47)}
$$
We now wish to identify, inside $\scrF$, the subspace of
physical states.
We shall concentrate on $\scrF_k\otimes\scrFbc$, 
the treatment of $\scrFtil_k\otimes\scrF^{(\bbar\cbar)}$ is
analogous.

Since the Virasoro algebra of $L_\ell$ generators holds
with vanishing central charge, it is natural to impose
$$
L_\ell\left.\mid \phi\right>=0\qquad\qquad
\forall\,\, \ell\in\dbZ\,\,.
$$
However, since the Fock space now also contains ghost
states, this condition is insufficient.
One would like to supplement it with a condition of ``no
ghosts''.
But from the treatment above, we see that ghost number
changes under conformal mappings, so the no ghost
condition is not invariant.
The proper framework for imposing physical state
conditions, when ghosts are present, is BRST
quantization.
It permits to recast the physical state conditions in
terms of cohomology.

We begin by discussing a finite-dimensional case, with
finite-dimensional Lie group $G$ and associated Lie
algebra $\scrG$.
Let $\Lam$ be a representation of $\scrG$, with
representation matrices $\Lam_i$, $i=1,\ldots,\dim\,G$,
and let $f_{ij}^k$ be the structure constants of $\scrG$:
$$
\left[\Lam_i,\Lam_j\right]=f_{ij}^k\Lam_k
\eqno{(4.48)}
$$
One introduces anti-ghosts $b_i$ in the adjoint
representation of $\scrG$ and ghosts $c^i$ in the dual to
the adjoint representation of $\scrG$.
(For compact $G$, these are identified, but in general,
it may not be possible to do so.)
These satisfy
$$
\{b_i,c^j\}=\delta_j^j\qquad\qquad\qquad
\{c^i,c^j\}=\{b_i,b_j\}=0\,\,.
\eqno{(4.49)}
$$
The number operator is the analogue of the ghost number
$$
U=c^ib_i\qquad\qquad\qquad [U,c^i]=c^i\,;\,\,\,
[U,b_i]=-b_i
\eqno{(4.50)}
$$
and is invariant under $\scrG$. 
The Fock space $\scrF$ is determined by choosing a
polarization.
The natural polarization here is as follows
$$
b_i\left.\mid 0\right>=0\qquad\qquad\qquad
\scrF=\{c^i\}\left.\mid 0\right>\,\,,
\eqno{(4.51)}
$$
and $\scrF$ contains $2^{\dim\,G}$ states.

This model may be realized concretely by taking $c^i$,
$i=1,\ldots,\dim\,G$ to be a basis of left-invariant
$1$-forms on $G$, and $b_i$ to be the {\it contraction}
with the form $c^i$: \ $b_i=\partial/\partial c^i$.
The operator
$U$ gives the degree of the form and the ground state
$\left.\mid 0\right>$ corresponds to forms of degree $0$,
i.e. constants.
The Fock space $\scrF$ is the vector space of all
left-invariant forms on $G$: \ $\Omega^{\inv}(G)$.

In BRST theory, one introduces the {\it BRST operator}
(still for our finite-dimensional group $G$):
$$
Q\equiv c^i\Lam_i-{1\over 2}\,f_{ij}^k c^ic^jb_k\,\,.
\eqno{(4.52)}
$$
This operator acts on $\scrF\otimes\Lam$, where $\Lam$ is
the representation vector space of $\Lam$.
This operator is {\it nilpotent}.
To see this, first show that $Q^2=0$ when $\Lam_i=0$; 
the remaining terms with $\Lam$ present are
$$
Q^2={1\over 2}c^ic^j
\left([\Lam_i,\Lam_j]-f_{ij}^k\Lam_k\right)\,\,,
\eqno{(4.53)}
$$
which vanishes in view of the structure relations on
$\Lam$.

In terms of left-invariant forms on the group $G$, the
operator $Q$ has a natural interpretation as well.
This is understood by examining how a differential acts
on a left-invariant differential form in
$\Omega^{(p)}(G)$: \ if $\omega\in\Omega^{(p)}(G)$
then $\omega=\omega_{i_1}\ldots i_p\,\, c^{i_1}\w\ldots\w
c^{i_p},$
with constant $\omega_{i_1\ldots i_p}$.
The differential on $c^i$ is given by the Cartan
structure equations
$$
dc^i+{1\over 2}\,f_{jk}^ic^j\w c^k=0
\eqno{(4.54)}
$$
and thus its action on $\omega$ may be recast in a purely
algebraic form
$$
d\omega=-{1\over 2}\,f_{jk}^i c^jc^kb_i\,\,.
\eqno{(4.55)}
$$
Furthermore, when $\omega$ transforms under the
representation $\Lam$ of $\scrG$, then $Q$ is nothing but
the covariant derivative acting on $\scrF\otimes\Lam$,
and its square vanishes in view of the flatness of the
connection $c^i$.

We now consider states in the Fock space $\scrF_{(m)}$ of
degree $m=0,1,\ldots,\,\,\dim\,G$.
The BRST operator $Q$ has degree $1$, and maps
$Q\colon\,\scrF_{(m)}\to\scrF_{(m+1)}$ for $m\Le
\dim\,G-1$, and $Q\colon\,\scrF_{(\dim\,G)}\to 0$.
A state $\left.\vert\psi\right>$ is {\it BRST invariant}
if
$$
\eqalignno{
Q\left.\vert\psi\right>=0 &\kern5 true cm
  \hbox{\rm (i.e. $Q$-closed)}\cr
\noalign{\hbox{and {\it BRST trivial} if for some
$\left.\vert x\right>\in\scrF_{(m-1)}$, we have}}
\left.\vert\psi\right>=Q\left.\vert\chi\right>
  &\kern5 true cm \hbox{\rm (i.e. $Q$-exact)}\cr}
$$

BRST cohomology is defined by the following equivalence
relation: \ 
$\left.\vert\psi\right>\sim\left.\vert\psi'\right>$
provided there is some $\left.\vert\chi\right>$ such that
$$
\left.\vert\psi'\right>=\left.\vert\psi\right>+Q
\left.\vert\chi\right>\,\,.
$$
BRST cohomology classes of degree $m$ are denoted by
$H^{(m)}(Q)$.
The BRST cohomology of degree $0$ is easily
computed.
First, $\left.\vert\psi\right>\in\scrF_{(0)}$ means
that $b_j\left.\vert\psi\right>=0$ for all $j$, so that 
the condition of BRST invariance becomes
$$
Q\left.\vert\psi\right>
=c^i\Lam_i\left.\vert\psi\right>=0\qquad
\Rightarrow \qquad
\Lam_i\left.\vert\psi\right>=0
$$
Here, we have used the fact that, since
 $\{b_i,c^j\}=\delta_i^j$, we have $c^i\left.\vert\psi\right>
\not=0$ for all $i$, and these vectors are linearly
independent.
Thus, all BRST invariant states of ghost number $0$ are
$\scrG$-invariant.
Clearly no states in $\scrF_{(0)}$ are BRST trivial.
Vice versa, all $\scrG$-invariant states of ghost number
$0$ are BRST invariant.

In string theory, the BRST formalism is implemented as
follows.
The Lie aglebra $\scrG$ is the infinite-dimensional
Virasoro algebra, with structure constants
$$
[L_i,L_j]=(i-j)L_{i+j}+\hbox{ \rm central};\quad
f_{ij}^k=(i-j)\delta_{i+j}^k\,\,.
\eqno{(4.56)}
$$
We do not specify the central term, since we shall not
infer its implications from the finite-dimensional
model.
The representation $\Lam$ of $\scrG$ is given by the
matter fields $x$, with generators $L_i^{(x)}$, and
central charge $26$.
The ghost and anti-ghost modes are
$$
c^\ell=c_{-\ell}\qquad\hbox{and}\qquad
b_\ell\qquad\qquad
\ell\in\dbZ\,\,.
\eqno{(4.57)}
$$
The polarization for the string theory BRST is different
from that for the finite-dimensional model
$$
b_\ell, c_\ell \left.\vert\downarrow\right>\,,\quad
\left.\vert\uparrow\right>=0\qquad
\ell>0\,\,.
\eqno{(4.58)}
$$
The BRST operator is defined by the structure constants
$f_{ij}^k$, but the precise ordering prescription, and
the effects of the central terms cannot be obtained from
the finite-dimensional model.
It will turn out (see next paragraph) that the correct
prescription is given by
$$
Q\equiv\sum\limits_{r\in\dbZ}(L_r^{(x)}-\delta_{r,0})c_{-r}-
{1\over 2}\sum\limits_{r,s\in\dbZ}(r-s)\,\,:
c_{-r}c_{-s}b_{r+s}:
\eqno{(4.59)}
$$

The correct prescription may be derived most efficiently
from the fact that the BRST charge derives from a
conserved (in fact analytic) BRST current, which arises
as the Noether current of BRST symmetry of the action
$S$.
The transformation rules are (they may be derived from
the commutation of the operator $\lam Q$, for some
constant Grassmann parameter $\lam$, with the fields
$x^\mu$, $b$, $c$):
$$
\left\{
\eqalign{
\delta x^\mu &=\lam c\partial x^\mu\cr
\delta c &= \lam c\partial c\cr
\delta b &=2\lam(T^{(x)}+T^{(bc)})\,\,.\cr}
\right.
\eqno{(4.60)}
$$
The resulting BRST current is
$$
\eqalign{
j^{\BRST}(z) &=-{1\over 2}\,c\partial x\cdot\partial x+
  bc\partial c+{3\over 2}\,\partial^2 c\cr
&=cT^{(x)}+{1\over 2}\,cT^{(bc)}
+{3\over 2}\,\partial^2 c\,\,.\cr}
\eqno{(4.61)}
$$
The last term on the right hand side ensures that
$j^{\BRST}$ is a $(1,0)$ form as a quantum operator.
The quantum (renormalized) current is defined as usual by
OPE singularity subtractions.
The BRST charge is then
$$
Q=\oint\,{dz\over 2\pi i}\,j^{\BRST}(z)\,\,,
\eqno{(4.62)}
$$
and agrees with the expression of (4.59).

Using the OPE algebra for the $b$-$c$ system, derived in
\S{C}, we find that for $D=26$ we have
$Q^2=0$. 
Conversely, if $Q^2=0$, then the Virasoro algebra of
$L_m=L_m^{(x)}+L_m^{gh)}$ must be satisfied with zero central
charge.
This is easily established from the fact that
$$
L_m=\{Q,b_m\}\qquad\Longrightarrow\qquad
[L_m,Q]=0
\eqno{(4.63)}
$$
and hence
$$
[L_m,L_n]=\{Q,[L_m,b_n]\}=(m-n)\{Q,b_{m+n}\}=
(m-n)L_{m+n}\,\,.
$$

The BRST invariance condition $Q\left.\vert\psi\right>=0$
now reproduces the physical state conditions
$L_m\left.\vert\psi\right>=0$, $m\in\dbZ$, in the
following way.
First, we define a map $i$ from $\scrF_k^{\phys}$ into
$\scrF_k^{\phys}\otimes\scrF^{(bc)}$. which maps
physical states in $\scrF_k^{\phys}$ into states of
ghost number $1$ as follows:
$$
i\colon\,\scrF_k^{\phys}\to\scrF_k\otimes
\scrF^{(bc)},\,\,
\left.\vert\varepsilon,k\right>\to \left.
\vert\varepsilon,k,\downarrow\right>=
\left.\vert\varepsilon,k\right>\otimes
\left.\vert\downarrow\right>
\eqno{(4.64)}
$$
Since $b_m\left.\vert\downarrow\right>=0$ for all
$m\in\dbN$ and $c_n\left.\vert\downarrow\right>=0$ for
all $n\in\dbN^{\plus}$, it follows that if
$\left.\vert\epsilon,k\right>$ is physical, then
$Q\left.\vert\varepsilon,k,\downarrow\right>=0$.
Thus, $i$ is really a map into $\Ker\,Q$
$$
i\colon\,\scrF_k^{\phys}\to\Ker\,Q\,\,.\eqno{(4.65)}
$$
Next, let us consider a state
$\left.\vert\psi\right>\in\scrF_k\otimes\scrF^{(bc)}$,
of the form
$$
\left.\vert\psi\right>=\left.\vert\epsilon,k\right>
\otimes\left.\vert\downarrow\right>\eqno{(4.66)}
$$
with ghost number $1$, and belonging to $\Ker\,Q$:
$Q\left.\vert\psi\right>=0$.
Then, in view of $b_m\left.\vert\downarrow\right>=0$
for $m\in\dbN$ and $c_n\left\vert\downarrow\right>=0$
for $n\in\dbN^{\plus}$, we have
$$
\eqalign{
Q\left.\vert\psi\right> &=\sum\limits_{r\in\dbZ}
(L_r^{(x)}-\delta_{r,0})c_{-r}
\left.\vert\eps,k\right>\otimes\left.\vert\downarrow\right>\cr
&=\left\{(L_0^{(x)}-1)c_0+\sum\limits_{r=1}^\infty
  L_r^{(x)}c_{-r}\right\}\left.\vert\eps,k\right>
  \otimes\left.\vert\downarrow\right>\,\,,\cr}
\eqno{(4.67)}
$$
The condition 
$Q\left.\vert\psi\right>=0$ implies $(L_0^{(x)}-1)
\left.\vert\eps,k\right>=0$ and
$L_r^{(x)}\left.\vert\eps,k\right>=0$ for 
$r\in\dbN^{\plus}$, which are the correct physical state
conditions on $\scrF_k$.
(The ground state $\left.\vert\uparrow\right>$ would miss
the conditions
$(L_0^{(x)}-1)\left.\vert\eps,k\right>=0$.)

Which of these states are BRST trivial?
Again, from the definition of $Q$ and the fact that
$b_m\left.\vert\psi\right>=c_m\left.\vert\psi\right>=0$
for $m\in\dbN^{\plus}$, it follows that if
$\left.\vert\psi\right>=Q\left.\vert\lam\right>$, 
with $\left.\vert\lam\right>$ of ghost number $0$
then
$$
\left.\vert\psi\right>=\sum\limits_{r=1}^\infty
L_{-r}^{(x)}\left.\vert\lam_r\right>
\eqno{(4.68)}
$$
for some states $\left.\vert\lam_r\right>$.
But then $\left.\vert\lam\right>$ is spurious.

In fact, 
one can show that any state $\left.\vert\psi\right>$ in
$\Ker\,Q$, of ghost number $1$, is of the form
$$
\left.\vert\psi\right>=\left.\vert\varepsilon,k\right>
\otimes\left.\vert\downarrow\right>+Q\left.\vert
x\right>
$$
where $\left.\vert\varepsilon,k\right>\in\scrF_k^{\phys}$
and $\left.\vert x\right>$ is some state of ghost
number $0$.
Thus, the cohomology of $Q$ of ghost number $1$, i.e.
space of states of ghost number 1
which are  BRST invariant, modulo BRST trivial states,
is isomorphic to the 
physical state space $\scrF^{\plus}$ defined previously:
$$
\scrF^{\plus}=\scrF^{\phys}/\scrF^{\spurious}\sim
H^{(1)}(Q)
\eqno{(4.69)}
$$


\bye






