%Date: Wed, 09 Oct 1996 09:08:13 EDT
%From: Edward Witten <witten@sns.ias.edu>

\def\R{{\bf R}}
\input harvmac
\bigskip\noindent
{\it ``Problem Set Two''}

(1) Consider the free electromagnetic field on $\R^{n}$ (with
a metric of signature $(+--\dots -)$).  Consider the free electromagnetic
field -- a connection $A$ with curvature (or field strength $F=dA$)
and Lagrangian

\eqn\umbo{I=-{1\over 4e^2}\int_{\R^n} F\wedge *F}
regarded as a functional of $A$.  ($e$ is a constant, related
to the charge of the electron; it has been written here in its usual
place but in this classical exercise you can set it to 1 by rescaling $A$.) 

(a) Derive the classical field equations and verify that they are
second order differential equations for $A$.

(b) Compute the closed two-form $\omega_0$ on the space $M_0$ of solutions
of the classical field equations.

(c) Let $G$ be the group of gauge transformations, which act by
$A\to A+d\phi$, where $\phi$ is a function on $\R^n$.

Let $M=M_0/G$ be the space of solutions of the equations modulo
gauge transformations.

Show that $\omega_0$ is the pullback to $M_0$ of a closed two-form
$\omega$ on $M$, and show that $\omega$ is non-degenerate.

(2) In this problem, we will consider the free electromagnetic field
in three dimensions, say on $\R^3$ with flat metric of signature $(+--)$.  
The interest in three dimensions is that it is possible to add an extra
term to the Lagrangian, which becomes

\eqn\longerex{I=-{1\over 4e^2}\int F\wedge *F +\mu\int A\wedge F.}

(If you prefer, set $e$ to 1 
and $\mu$ to $\pm 1$  by rescaling of $A$ and dilation
of $\R^3$.)
Note that the second term is invariant under $A\to A+d\phi$ where
$\phi$ has compact support.  

(a) Write the equations of motion in Fourier space and show that the
space of solutions mod gauge transformations
is supported not on a cone $p^2=0$ but on a certain
hyperboloid $H$ of the form $p^2={\rm constant}$.

(b) The space of solutions of the equations mod gauge 
transformations is a line bundle over $H$ to which the action of
$SO(2,1)$ lifts.  Describe this line bundle.  What happens if you change
the sign of $\mu$?

(3) In the next few exercises, we develop facts and supersymmetric
models that are important in string theory.

(a) Consider in two dimensions $\R^2$ with coordinates
$t,x$ and metric $dt^2-dx^2$ the Lagrangian of a free massless boson
$\phi$

\eqn\tumbo{L=-{1\over 2}\int_{\R^2} d\phi \wedge *d\phi.}

Show that the general solution of the equations of motion is
of the form $\phi(t,x)=\phi_R(t-x)+\phi_L(t+x)$ where
the decomposition is unique up to adding constants to $\phi_R$ and
$\phi_L$.  We call them right- and left-moving waves.
Show that the decomposition in right and left- moving waves
is compatible with the symplectic structure.
(This means that the symplectic structure paired with one left-moving
variation of $\phi$ and one right-moving one gives zero.)

(b) Consider the Dirac equation in two dimensions

\eqn\itto{ \Gamma\cdot \partial \psi = 0}
where $\psi$ is a section of the spin bundle $S$  of $\R^2$.
Show that, under the decomposition $S=S_+\oplus S_-$, $\psi$
decomposes as a sum of a right-moving wave with values in $S_+$
and a left-moving wave with values in $S_-$.

(4) Now     we get closer to physics.
First we construct a supersymmetric model in two dimensions in which
all odd variables are right-moving.

We start with a $(2,1)$ supermanifold $W$ with a one-dimensional
even distribution
$B_-$ generated by a vector field $\partial_-$ and
a one-dimensional
 odd distribution generated by a vector field $D$; we
require that $TX$ is generated by $\partial_-$, $D$, and $\partial_+=-D^2$.

A ``flat'' model is $\R^{2,1}$ with even coordinates $u,v$, odd
coordinate $\theta$, and

\eqn\hope{\eqalign{ \partial_- & = {\partial\over\partial u} \cr
                        D      & = {\partial\over\partial\theta} 
                                          -{\theta}{\partial\over\partial v}\cr
          \partial_+ & = {\partial\over\partial v}.\cr}}

Notice that in this problem (unlike some previous exercises) one
is given only the distributions, and not specific vector fields generating
them.  

Let $M$ be a Riemannian manifold with metric $g$ and $X:W\to M$ a map.
The Lagrangian we want
\eqn\oggy{L_X=\int_W g_{IJ}\partial_-X^I DX^J.}

(a) Show that the Lagrangian is well-defined (and invariant under all
automorphisms of $W$ that preserve the distributions) by interpreting the
integrand as a section of the Berezinian.  

(b) By performing the theta integral, reduce this to an ordinary
Lagrangian.  Show that (i) the reduction of the Lagrangian (throwing
away odd variables) is the
Lagrangian for a harmonic map to $M$, (ii) the equations of motion are
such that the odd variables are right-moving, that is functions of $v$ and
not $u$.  

(c) Thus the phase space (=space of classical solutions) $Y$ is a bundle
over $Y_{red}$, where $Y_{red}$ is the space of harmonic maps and
the normal bundle to $Y_{red}$ in $Y$ is $\Pi V$, where $V $ is some
vector bundle over $Y_{red}$.  From (b) you should know that $Y_{red}$
is the cotangent bundle of $Hom(\R,M) $, where we identify a classical
solution with the initial conditions on the spacelike hypersurface
$u=-v$.  Show that $V$ is the tangent bundle of $Hom(\R,M)$.  Thus,
a quantization of this system gives (formally) a spin bundle of $Hom(\R,M)$,
together with a Dirac operator.  To be precise, the Dirac operator
is the one that arises upon quantization of the Hamiltonian function
associated with the motion on the phase space that
comes from the vector field on $W$ 
\eqn\murmo{{\partial\over\partial\theta}+\theta{\partial\over\partial v}.}

(d) Consider the flat model with $W=\R^{2,1}$ as above.
Let $u=t+x$, $v=t-x$,  Restrict the Lagrangian to maps $X$ that are 
invariant under translations in $x$.  The quotient of $\R^{2,1}$ by
such translations is $\R^{1,1}$.  Show that for these translation-invariant
maps, $L$ reduces to
\eqn\boggy{L'=\int dt\,d\theta g_{IJ} {dX^I\over dt}DX^J.}
Recall from a previous problem set that quantization of $L'$ gives
the spin bundle and Dirac operator of $M$.  Our previous result
is thus essentially a
consequence of the present more general one upon specializing
to the translation-invariant maps.

(5)  Now we will consider the Dirac operator on $Hom(\R,M)$ with
values in a certain kind of vector bundle.

Let $Q$ be a finite dimensional rank $n$ real vector bundle over $M$ with
structure group $ SO(n)$, and a choice of metric.

Let $B_-^{1/2}$ be a square root of the even distribution $B_-$ 
of the previous exercise.  Let $\Lambda$ be an odd section of
$B_-\otimes X^*(Q)$.    The Lagrangian for $\Lambda$ is
\eqn\ompo{ L_\Lambda =\int_W (\Lambda,D\Lambda).}
For the coupled system of the map $X$ and $\Lambda$, one 
simply adds the two Lagrangians to get $L=L_X+L_\Lambda$.

(a) Show that this is well-defined (and invariant under all of the
diffeomorphisms of $ W$ that preserve the distributions).

(b) Writing $\Lambda=\lambda+\theta G$,    perform the $\theta$ integral,
to show that $G$ is a non-progagating ``auxiliary field'' (whose
field equations have a unique  solution in terms of the other variables)
while the odd variables $\lambda$ are ``right-moving.''
Keeping $X$    fixed, show that
the space of critical points of $L_\Lambda$
is a vector bundle $P=\Gamma(\R,X^*(Q))$ over $Hom(\R,M)$.
Thus, quantization of this system will     give a spin bundle derived
from $P$, and the operator that in the quantum theory is derived
from \murmo\ is a Dirac operator on $Hom(\R,M)$ with values in $P$.

(c) Show that in the special case that $Q=TM$ is the tangent bundle to $M$,
$P$ is the tangent bundle of $Hom(\R,M)$, so quantization of
$L_X+L_\Lambda$ will give for the quantum Hilbert space a
spin bundle $S$ of $Hom(\R,M)$ with values in another spin bundle
$S'$.  (Unlike the finite-dimensional case, $S$ and $ S'$ are not
isomorphic as they are constructed using different polarizations: one set
of odd variables is left-moving and one is right-moving.)

In finite dimensions, $S\otimes S'$ is isomorphic to the   de Rham complex.
(An explanation of that is in exercise 6(b) below.)  Thus, this
theory in the special case $V=TM$ is formally  related to
some sort of differential forms on $Hom(\R,M)$.  One might suspect
that this special case has additional symmetry, if one
takes the metric on $V$ to be the Riemannian metric.  That is true and will
be pursued later.



(6) This is really more an explanation than an exercise.

(a) Let $M$ be an $n$-dimensional differentiable manifold (no metric).
The rank $2n$ 
bundle $W=TM\oplus T^*M$ has a natural quadratic form (in which $TM$
and $T^*M$ are isotropic and paired with each other) which can be interpreted
as a symplectic form on $\Pi W$. 

Quantize $\Pi W$ to get
 the associated bundle $S(W)$ of spinors of $W$.  Identify
it with the de Rham complex $\Omega^*(M)$.

Therefore, $TM\oplus T^*M$ must act on $\Omega^*(M)$.  Show
that a section of $TM\oplus T^*M$ of the form $v\oplus \alpha$
(with $v$ and $\alpha$ a vector field and one-form respectively)
acts on a differential form $\theta$ by $\theta\to i_v\theta +\alpha\wedge
\theta$.  Deduce the classical formula $(i_v\alpha+\alpha i_v)=\alpha(v)$.

(b) If a metric is introduced on $M$, then $\Pi TM$ (or equivalently
$\Pi T^*M$) can be quantized to get the spin bundle $S(M)$.  
(For present purposes ignore possible global obstructions to this.)   
Suppose for simplicity that $M$ is even-dimensional.
Explain from the present point of view the standard relation
between the de Rham complex and the spin bundle: $\Omega^*(M)\cong
S(M)\otimes S(M)$ (over ${\bf C}$).  

(c)  
Consider a supermanifold $N$.  To keep things simple,
take $N=\R^{0,n}$, with odd coordinates $\theta_i$, $i=1,\dots,n$.

$X=\Pi TN\oplus \Pi T^*N$ is a $2n$-dimensional vector space
with a natural symplectic form, in which $\Pi TN$ and $\Pi T^*N$ are
isotropic and paired with each other.

As in (a) above, one can try to quantize $X$ to get ``the de Rham complex
of  $N$.'' The difference from (a) is such a quantization will
be infinite dimensional.  Moreover, the only natural ($Sp(2n,\R)$-invariant)
quantization of such a symplectic vector space gives a Hilbert space
of certain ${\bf L}^2$ functions.

However, an algebraic quantization can be obtained if one picks a 
polarization, that is a maximal isotropic subspace $Y$ of $X$,
and asks that there should be a vector annihilated by $Y$.  The
quantization of $X$ -- that is, the ``differential forms on $N$'' --
is then the space of polynomials on $X/Y$.
So the Clifford algebra of $X$ (and in particular the subalgebra
$\Lambda^* \Pi T^*N$) acts on $S^*(X/Y)$ for any $Y$.  (But there
is no way to identify $S^*(X/Y)$ for different $Y$.)

Show that the case $Y=\Pi TN$ corresponds to what Berstein described
as the ``differential forms of finite order'' on $N$ while
the case $Y=\Pi T^*N$ corresponds to the ``differential forms of
finite codimension.''









\end

