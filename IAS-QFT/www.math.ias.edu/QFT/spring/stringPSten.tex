%From: "Eric D'Hoker" <dhoker@IAS.EDU>
%Date: Wed, 9 Apr 1997 15:40:30 -0400 (EDT)

%The last problem Set !!!!

%%%%%%%%%%%%%%%%%%%%%%%%%%%%%%%%%%%%%%%%%%%%%%%%%%%%%%%%%%%%%%%%%%%%%%%%%%%
%%%%
%%%%    STRING THEORY : Problem Set 10, April 10, 1997
%%%%
%%%%%%%%%%%%%%%%%%%%%%%%%%%%%%%%%%%%%%%%%%%%%%%%%%%%%%%%%%%%%%%%%%%%%%%%%%%


\magnification=\magstep1
\overfullrule=0pt
\baselineskip=17pt
\def\det{{\rm det}}
\def\Det{{\rm Det}}
\def\tr{{\rm tr}}
\def\Tr{{\rm Tr}}
\def\12{{1 \over 2}}
\def\ker{{\rm Ker}}
\def\O{{\cal O}}
\def\G{{\cal G}}

\centerline{{\bf STRING THEORY}}
\centerline{ Problem Set \# 10}
\centerline{April 10, 1997}

\bigskip
\bigskip

In the following problems, we shall derive certain properties, under
so-called T-duality, of closed oriented string theories, compactified
in one dimension on a circle of radius $R$. 
 
\medskip
 
\noindent
{\it Problem 1}

\medskip

Consider the closed oriented bosonic string on the space-time 
manifold $M= {\bf R}^{25} \times S^1$,
with Minkowskian metric $\eta = {\rm diag } (-+ \cdots +)$ and
with $S^1$ of radius $R >0$ (lying in a space direction of $\eta$). 

a) Obtain the mode expansion of the most general solution to the free theory
on a worldsheet with the topology of an annulus. (Make sure to include the 
effects of the fact that the space-time manifold $M$ is not simply connected.)

b)  Show that the states of the theory for radius $R$ are in one
to one correspondence with 
the states for the theory with radius $R'=b/R$ for some ($R$-independent)
constant $b>0$. This relation is usually called T-duality.

c) Show that transition amplitudes for the theory with radius $R$
and $R'$ -- as defined in b) -- are similarly related. One says that
the bosonic closed oriented string is self-dual under T-duality. 


\bigskip

\noindent
{\it Problem 2}

\medskip

Consider now the Type IIA and Type IIB closed oriented superstring
theories on the space-time manifold $M= {\bf R}^{9} \times S^1$,
with Minkowskian metric $\eta = {\rm diag } (-+ \cdots +)$ and
with $S^1$ of radius $R >0$ (lying in a space direction of $\eta$).

a) Begin by taking $R=\infty$ (the original superstring theories)
and show that the massive spectra of the Type IIA and Type IIB
theories are identical. 

b) Now, for finite $R$, obtain the mode expansion of the most general
solution to the free theories on a worldsheet with the toplogy of 
an annulus. 

c) Show that the states which are annihilated by all the oscillator 
annihilation operators of level $>0$ (these are the pure momentum and 
winding states) for the Type IIA string at radius $R$ are in one to one 
correspondence  with states for the Type IIB string at radius 
$\tilde R= c/R$ for some sonstant $c$. (You will have to explain
how the relative chirality of left and right movers can be changed
upon $R\to \tilde R$.)


d) Generalize the argument of c) to show that the full spectra of the
Type IIA superstring at radius $R$ and of the Type IIB superstring
at radius $\tilde R$ are identical. One says that the Type IIA and
Type IIB superstrings are mapped into one another under T-duality.

\bigskip

\noindent
{\it Problem 3}

\medskip

Finally, consider the heterotic strings with gauge groups 
$E_8 \times E_8$ and $Spin (32) /Z_2$, 
on the space-time manifold $M= {\bf R}^{9} \times S^1$,
with Minkowskian metric $\eta = {\rm diag } (-+ \cdots +)$ and
with $S^1$ of radius $R >0$ (lying in a space direction of $\eta$).

The circle compactification now allows for a non-trivial holonomy
of the gauge field along $S^1$ -- in physics language : a non-zero
value for the Wilson loop around $S^1$. We assume that this
holonomy element is such that it breaks both $E_8 \times E_8$ and 
$Spin (32) /Z_2$ down to $SO(16) \times SO(16)$. 

Show that the corresponding $SO(16) \times SO(16)$ theories
are again related to one another upon $R \to \tilde R$, where
$\tilde R$ was defined in Problem 2, c. One says that the
two heterotic theories are mapped into one another under
T-duality.

\end
--4f3e_36e6-356b_52ea-39e1_354f--

