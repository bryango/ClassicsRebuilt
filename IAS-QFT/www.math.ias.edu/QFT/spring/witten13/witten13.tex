\documentstyle{article}

\input epsf

%These are the macros which are in common with all of the
% sections in the paper mmr
% Each section, for now, should begin with \documentstyle[11pt,cd]{article}
% and then have \input{mmrmacros} followed by \begin{document}
% The only exception is that the \Label macro is slightly different
% in each file and should be put in separately.
%New CD macros
\newcommand{\cdrl}{\cd\rightleftarrows}
\newcommand{\cdlr}{\cd\leftrightarrows}
\newcommand{\cdr}{\cd\rightarrow}
\newcommand{\cdl}{\cd\leftarrow}
\newcommand{\cdu}{\cd\uparrow}
\newcommand{\cdd}{\cd\downarrow}
\newcommand{\cdud}{\cd\updownarrows}
\newcommand{\cddu}{\cd\downuparrows}
% (S) Proofs.
% (S-1) Head is automatically supplied by \proof.

\def\proof{\vspace{2ex}\noindent{\bf Proof.} }
\def\tproof#1{\vspace{2ex}\noindent{\bf Proof of Theorem #1.} }
\def\pproof#1{\vspace{2ex}\noindent{\bf Proof of Proposition #1.} }
\def\lproof#1{\vspace{2ex}\noindent{\bf Proof of Lemma #1.} }
\def\cproof#1{\vspace{2ex}\noindent{\bf Proof of Corollary #1.} }
\def\clproof#1{\vspace{2ex}\noindent{\bf Proof of Claim #1.} }
% End of Proof Symbol at the end of an equation must precede $$.

\def\endproof{\relax\ifmmode\expandafter\endproofmath\else
  \unskip\nobreak\hfil\penalty50\hskip.75em\hbox{}\nobreak\hfil\bull
  {\parfillskip=0pt \finalhyphendemerits=0 \bigbreak}\fi}
\def\endproofmath$${\eqno\bull$$\bigbreak}
\def\bull{\vbox{\hrule\hbox{\vrule\kern3pt\vbox{\kern6pt}\kern3pt\vrule}\hrule}
}
\addtolength{\textwidth}{1in}                  % Margin-setting commands
\addtolength{\oddsidemargin}{-.5in}
\addtolength{\evensidemargin}{.5in}
\addtolength{\textheight}{.5in}
\addtolength{\topmargin}{-.3in}
\addtolength{\marginparwidth}{-.32in}
\renewcommand{\baselinestretch}{1.6}
\def\hu#1#2#3{\hbox{$H^{#1}(#2;{\bf #3})$}}          % #1-Cohomology of #2
\def\hl#1#2#3{\hbox{$H_{#1}(#2;{\bf #3})$}}          % #1-Homology of #2
\def\md#1{\ifmmode{\cal M}_\delta(#1)\else  % moduli space, delta decay of #1
{${\cal M}_\delta(#1)$}\fi}
\def\mb#1{\ifmmode{\cal M}_\delta^0(#1)\else  %moduli space, based, delta
                                              %decay of #1
{${\cal M}_\delta^0(#1)$}\fi}
\def\mdc#1#2{\ifmmode{\cal M}_{\delta,#1}(#2)\else    %moduli space, delta
                                                      %decay, chern class #1
                                                      %of #2
{${\cal M}_{\delta,#1}(#2)$}\fi}
\def\mbc#1#2{\ifmmode{\cal M}_{\delta,#1}^0(#2)\else   %as before, based
{${\cal M}_{\delta,#1}^0(#2)$}\fi}
\def\mm{\ifmmode{\cal M}\else {${\cal M}$}\fi}
\def\ad{{\rm ad}}
\def\msigma{\ifmmode{\cal M}^\sigma\else {${\cal M}^\sigma$}\fi}
\def\cancel#1#2{\ooalign{$\hfil#1\mkern1mu/\hfil$\crcr$#1#2$}}
\def\dirac{\mathpalette\cancel\partial}
\newtheorem{thm}{Theorem}
\newtheorem{theorem}{Theorem}[subsection]
\newtheorem{proposition}[theorem]{Proposition}
\newtheorem{lemma}[theorem]{Lemma}
\newtheorem{claim}[theorem]{Claim}
\newtheorem{example}[theorem]{Example}
\newtheorem{corollary}[theorem]{Corollary}
\newtheorem{D}[theorem]{Definition}
\newenvironment{defn}{\begin{D} \rm }{\end{D}}
\newtheorem{addendum}[theorem]{Addendum}
\newtheorem{R}[theorem]{Remark}
\newenvironment{remark}{\begin{R}\rm }{\end{R}}
\newcommand{\note}[1]{\marginpar{\scriptsize #1 }} 
\newenvironment{comments}{\smallskip\noindent{\bf Comments:}\begin{enumerate}}{
\end{enumerate}\smallskip}

\renewcommand{\thesection}{\Roman{section}}
\def\eqlabel#1{\addtocounter{theorem}{1}
\write1{\string\newlabel{#1}{{\thetheorem}{\thepage}}}
\leqno(\rm\thetheorem)}
\def\cS{{\cal S}}
\def\ov{\overline}












\title{Lecture II-13:  $N=2$ SUSY theories in Dimension Two, Part II:
Chiral Rings and Twisted Theories} 

\author{Edward Witten\thanks{\tt{Lecture notes by John Morgan}}}
\date{April 16, 1997}

\begin{document}
\maketitle

This lecture is again concerned with $N=2$ supersymmetric theories in
dimension $2$.
Using one of the  supersymmetries $Q$, we shall define the
chiral ring of local operators and the $Q$-cohomology of states for
theories defined in flat Minkowski or Euclidean space.
We then discuss how to twist theories so that the supersymmetry $Q$
has global meaning at least over  a large class of 
$2|4$ supermanifolds. Once $Q$ has been globalized in a theory over
$\Sigma$ we 
have the notions of the ring of chiral local operators for that
theory.
We  can define what are called descendants of $Q$-invariant operators,
and we show that the integrals of these descendants over cycles in the
manifolds lead to correlation functions which have
topological meaning.
To compute these functions one does integrals over the $Q$-invariant
fields.

We then consider in detail two examples of twisted theories with a
global supersymmetry $Q$. The first is pure $N=2$ gauge theory in
dimension $2$ with a simple Lie group $G$. Here, we see that the
ring of chiral functions is identified with the invariant polynomials
on the Lie algebra.  Classically, the space of
$Q$-invariant states is the usual space of flat connections (plus
an extra parallel section of the adjoint bundle in the case of
reducible connections). The   correlation functions of the $Q$-invariant
operators are
identified with the  integrals over the moduli space of flat connections
 of the usual classes
 derived from the invariant
polynomials in the Lie algebra. (There are extra complications from 
reducible connections.)

The second example is $N=2$ supersymmetric
two-dimensional $\sigma$-models  into a K\"ahler manifold.
It turns out that there is one way of twisting that leads to a chiral
ring which is the topological cohomology of the K\"ahler manifold.
The space of $Q$-invariant states is the space of holomorphic maps
from the riemann surface into $X$. In this twisted theory the
path integrals which commute the topological correlation  functions
are the integrals in quantum cohomology.
When the K\"ahler manifold is Calabi-Yau, there is a second,
inequivalent way of twisting which leads to a different theory.  In this
theory the chiral ring is $H^*(X,\wedge^*T^{1,0}(X))$ and the
$Q$-invariant fields are constant maps from the riemann surface to
$X$.
Thus, the path integrals which compute the topological correlation
functions are integrals of these objects over $X$ --  these integrals
are related to variation of Hodge structures on $X$.

\section{$R$-symmetry revisited}
Most of the $N=2$ supersymmetric examples we shall consider will have
(at least classically) both left- and right-moving $R$-symmetries --
the $R$-symmetries denoted by $J_L$ and $J_R$ in the last lecture.
It is possible to carry out today's  discussion with only one of
these $R$-symmetries, but we must have at least one.
But we will suppose that the group of $R$-symmetries is $U(1)\times U(1)$
so that we have both $J_R$ and $J_L$.  The supermanifold on
which the theory is defined in ${\bf R}^{2|4}$. We shall also
eventually allow a certain class of split $2|4$
super-riemann surfaces with a Euclidean metric (and
$2|4$-supermanifolds with a flat Minkowski metric). The space of odd 
directions is a copy of the two-dimensional representation for $J_L$
and the two dimensional representation for $J_R$.
We expect a symmetry between $J_L$ and $J_R$ obtained by reversing the
orientation of the worldsheet.

One operation we can attempt to do to a symmetry of a Lagrangian is to
gauge it. 
If the symmetry group to be gauged is $H$, then the new Lagrangian
will involve gauge fields  with gauge group $ H$ times the  gauge
group (if any) of the original theory.
Intuitively, the process of gauging  involves replacing fields with
values in a constant space by fields with values in an $H$-bundle and
adding a new field --  the $H$-connection. In order to carry out this
program the original Lagrangian must be a local expression in the
fields and the symmetry must be also be local in the sense that it
maps local fields to local fields at the same point.
With these reasonable assumptions on the Lagrangian and its symmetry,
it is always possible to form a new, gauged, Lagrangian.
We replace ordinary derivatives in the Lagrangian by covariant
derivatives. 
Quite often there is a natural way to carry out this process, but in
any case different choices for gauging the symmetry lead to
Lagrangians which differ by gauge invariant terms up to a total
derivative. 
There is however the issue of whether the gauged theory makes sense as
a quantum theory.  If the original theory has this property then the
necessary and sufficient condition that the gauged theory does as well
is that the symmetry being gauged be anomaly-free. 

In our case it is never possible to gauge the entire
$U(1)\times U(1)$ group of $R$-symmetries to produce a theory that
makes sense as a quantum theory.
In fact the only 
combinations of $J_L$ and $J_R$ that can be gauged in this way are
$J_L\pm J_R$.
These are referred to as the vector and axial currents  
$$J_V=J_L+J_R$$
$$J_A=J_L-J_R.$$
If one of these is gauged, then the other develops anomalies
and stops being a symmetry of the quantum theory.
To explain this, we need some general observations about anomalies.
The statement that a theory has a symmetry, generated by a current
$J_L$, means that correlation funcions obey the identity
$$d_x\langle J_L(x){\cal O}(y_1)\dots\rangle =0  ~~(*)$$
as long as $x$ is distinct from the $y_i$.  
Here $d_x$ is the exterior derivative in the $x$ variable.
If one extends the correlation functions (as distributions) across
the diagonal, this identity will pick up extra terms supported
on the diagonal.  A very important special case is the two point function
of the current itself (or, in $2k$ spacetime dimensions, the $k+1$ point
function).   For this the general structure is 
$$d_x\langle J_L(x)J_L(y)\rangle ={k\over 2 \pi }d_y\delta(x-y)~~~ (**)$$
with some constant $k$, which may or may not be zero.  The condition
that the global symmetry generated by $J_L$ can be gauged is precisely
that $k=0$.  To see this, let us try to gauge the symmetry by adding
to the theory a $U(1)$ gauge field $A$.  The Lagrangian obtains an extra
term $\int A\wedge J_L$.  (There may be more terms, higher order in $A$,
if $J_L$ itself contains derivatives of charged fields.)
Now, let us try to determine whether the theory with this coupling
actually has the symmetry generated by $J_L$, that is whether
the identity of equation $(*)$ holds for this theory.
For this we must look, in the original theory, at
$$d_x\langle J_L(x){\cal O}(y_1)\dots\exp(\int A\wedge J_L)\rangle $$
The exponential is included to express correlation functions of the
new theory in terms of the old theory.  If we expand this in powers
of $A$, the term linear in $A$ will contribute, upon using $(**)$:
$$\langle {\cal O}(y_1)\dots \cdot kF(x)\rangle $$
with $F$ the curvature.
This can be expressed as an operator relation
$$ dJ_L={k\over 2\pi}F,$$
asserting that after gauging $J_L$ no longer generates a symmetry.

More generally, if there are several currents $J^1,\dots, J^s$, generating
a Lie group $G$ of dimension $s$, with Lie algebra ${\cal G}$,
then the two point functions obey
$$d_x\langle J^a(x)J^b(y)\rangle ={k^{ab}\over 2\pi}d_y\delta(x-y)$$
with $k^{ab}$ some invariant quadratic form on ${\cal G}$.
The quadratic form $k$ has no positivity (since for instance
left and right moving
massless fermions make contributions of opposite signs) but does
obey an integrality condition.
If we gauge a subgroup $H$ of $G$, with Lie algebra ${\cal H}$,
then by a reasoning along the lines of the above, the quantum theory
only makes sense if the quadratic form $k$ vanishes when restricted to
${\cal H}$.  If this is so, then the ``global symmetries'' are generated
by currents in ${\cal G}$, not in ${\cal H}$, that are orthogonal
to ${\cal H}$ with respect to $k$.

In our problem, with the two currents $J_L,J_R$, the structure of $k$
is typically that $k^{LL}=-k^{RR}\not= 0$, $k^{LR}=0$.  This quadratic
form has two null vectors, $J_L\pm J_R$, so either of those two
linear combinations of the $R$-symmetries can be gauged.  If one
is gauged, the second becomes ``anomalous'' -- it ceases to be conserved --
since the two null vectors are, of course, not orthogonal to each other.
``Luckily,'' these options agree with the choices that we will need
below for other reasons.

What we are really interested in is not just gauging but
twisting theories by
$R$-symmetries.
This process requires first gauging the $R$-symmetry and then twisting
it using a homomorphism from the orthogonal group determined by the
metric  to the symmetry group. It turns out that there will be two
equivalent ways of going about this process.
But before we embark on twisting, we must understand the local
story.
That is the subject of the next section.


\section{$Q$-cohomology of operators}
We consider ${\bf R}^{2|4}$ with its standard Euclidean or Minkowski
metric. 
Suppose that our $N=2$ supersymmetry theory in dimension two has the
following supersymmetries: 
\begin{center}
\begin{tabular}{c|c}
 characters of $U(1)_L$ &  characters of $U(1)_R$ \\ \hline
$Q_-$ & $Q_+$ \\ 
$\overline Q_-$ & $\overline Q_+$. \\
\end{tabular}
\end{center}
These supersymmetries obey the relations
$$Q_+^2=\overline Q_+^2=0$$
$$\{Q_+,\overline Q_+\}=P_+$$
with symmetric relations with $+$ replaced by $-$.
Also,
$$\{Q_+,Q_-\}=\{\overline Q_+,Q_-\}=\{Q_+,\overline Q_-\}=\{\overline
Q_+,\overline Q_-\}=0.$$
We will consider at the same time four closely related possibilities.  We
let $Q$ be one of the following:
$$Q_++Q_-, \ \ 
Q_++\overline Q_-,\ \ 
\overline Q_++Q_-,\ \ 
\overline Q_++\overline Q_-.$$
(The first two possibilities are inequivalent and the last two are
hermitian conjugate to the first two and would give equivalent theories.)
To fix notation let us choose the generators so that $Q=Q_++Q_-$.
Letting $\mu=\pm$ we see that $P_\mu$ can be written as a commutator
\begin{equation}\label{T}
P_\mu=\{Q,T_\mu\}
\end{equation}
where $T_\mu$ is a linear combination of $Q_\pm,\overline Q_\pm$.
We are interested in the cohomology of $Q$ acting as a differential
(by superbracketing) on
the space of local operators (and more generally on products of such
operators). Thus, a local operator $\phi(x)$ will 
define a $Q$-cohomology class if $\phi(x)$ is $Q$-closed, i.e., if
$\{Q,\phi(x)\}=0$. 
A simple computation using Equation~\ref{T} shows that for a $Q$-closed
operator $\phi(x)$ we have
$$d_\mu\phi=\{Q,[T_\mu,\phi]\}.$$
(This derivative is interpreted to be the covariant derivative if the
super-riemann surface is not flat.)
This equation means that the
operator $d\phi$ is $Q$-exact, and hence that the class of
$d\phi$ in  the space of $Q$-cohomology classes of operators is
trivial. 
This means that as we vary the point $x$ in the riemann surface the
$Q$-cohomology class of the local operator $\phi(x)$ is unchanged.
It then follows that in  $Q$-cohomology
such $Q$-closed local operators commute -- simply move the points to
be space-like separated.
That is to say, the space $Q$-cohomology of operators is a ${\bf
Z}/2{\bf Z}$-graded commutative ring.
We can always represent these operators by $0$-form
operators. This is an easy consequence of the independence of the
$Q$-cohomology class on the metric and diffeomorphism
invariance of the $Q$-cohomology class. These facts will be
established later.

Let us recall a construction that we have seen before\note{Cross
reference?}. Given a $Q$-invariant $0$-form local operator
$\phi=\phi^{(0)}$ we define 
a $1$-form local operator by 
$$\phi_\mu^{(1)}=[T_\mu,\phi^{(0)}]$$
and
$$\phi^{(1)}=\phi_\mu^{(1)}dx^\mu.$$
The same computation as we did in the previous paragraph shows that
$$d\phi^{(0)}=\{Q,\phi^{(1)}\}.$$
This means that if $\Sigma_1$ is a one-cycle in the super-riemann
surface then
$$\int_{\Sigma_1}\phi^{(1)}$$
is $Q$-invariant.
Let $\phi^{(2)}=[T,\phi^{(1)}]$. Then
$$d\phi^{(1)}=\{Q,\phi^{(2)}\}.$$
So in this way, beginning with a $Q$-invariant $0$-form  local operator
$\phi^{(0)}$, 
we can build a tower of $k$-form local operators
$\phi^{(k)}$, called the descendants of $\phi^{(0)}$, and
hence produce more general $Q$-invariant operators of the form
$$\int_{\Sigma_k}\phi^{(k)}.$$
Our computations show that
for any $Q$-invariant local operator $\phi^{(0)}$ the operator
$\phi^{(0)}+\phi^{(1)}+\phi^{(2)}$ is closed for the total differential   
$d-Q$. 


\section{Twisting the theory to give it global meaning}

So far we have been working on $R^{2|4}$ with fixed coordinates. If we
rotate space, then $Q=Q_++Q_-$ is not invariant. Our goal is to modify
things until $Q$ is invariant under rotations and hence can be defined
on more general Euclidean $2|4$-manifolds. Let us denote by $SO(2)_E$ the  
Euclidean rotation group. The entire group of symmetries at our
disposal for the Euclidean theory is $SO(2)_E\times U(1)_L\times
U(1)_R$.
This group has the obvious projection onto $SO(2)_E$ with a splitting
back into the first factor. We wish to find another splitting of this
projection map, i.e., a group $SO(2)_E'$ inside the product projecting
isomorphically onto $SO(2)_E$ but which has the extra property that it
commutes with $Q$. It is easy to see that there exists a unique such
splitting and it is given on the Lie algebra level by
$$K'=K\pm\frac{1}{2}J_L\pm \frac{1}{2}J_R$$
where $K$ is the generator for $SO(2)_E$ and the signs depend on which
of the four possibilities we have chosen for $Q$. In the case that we
are concentrating on when $Q=Q_++Q_-$ we have
$$K'=K+\frac{1}{2}J_L-\frac{1}{2} J_R.$$
The fact that $1/2$ appears in these formulas means that the twisting
actually involves the spin form of $SO(2)_E$.

Notice that in order to make such twisting in dimension $n$ there must
be non-trivial homomorphisms from $Spin(n)$ to the $R$-symmetry group.
In dimension two it is possible to do this when the $R$-symmetry group
is $U(1)$, e.g., for $N=1$ supersymmetry. But for $n=4$ it is
necessary to have $N=2$ supersymmetry so that the $R$-symmetry group
is $SU(2)\times U(1)$  which has a non-trivial homomorphism from $Spin(4)$.

With this preliminary discussion in place, we are now ready to
globalize our theory of $Q$-cohomology classes of local operators.
Suppose that we have a $2|4$ super-riemann surface locally modeled on
${\bf R}^{2|4}$.
The hypothesis we have in mind for the super-riemann surface is that
it be a split $2|4$-manifold complex with the odd directions being
$$\left(\Pi K^{1/2}\oplus \Pi \overline K^{1/2}\right)\oplus \left(\Pi
\overline K^{1/2}\oplus \Pi K^{1/2}\right).$$
Here $\theta^+,\overline\theta^+$ are sections of the first two line
bundles and $\theta^-,\overline\theta^-$ are sections of the last two.
The process of twisting changes the odd part of the  underlying
supermanifold of the theory. 
We twist the spinor bundles containing $\theta^\pm$ by the
square root of the canonical bundle to the power $\pm 1/2$.
The result of
twisting replaces the above line bundles by
$$\left({\Pi\cal O}\oplus \Pi K\right)\oplus \left(\Pi{\cal O}\oplus
\Pi\overline K\right).$$
and makes the supersymmetry
$$Q=\frac{\partial}{\partial \theta^+}+\frac{\partial}{\partial
\theta^-}$$
which is a global symmetry. Locally, this symmetry agrees with
the symmetry $Q$ on ${\bf R}^{2|4}$ that we have been discussing.

Once we have globalized the supersymmetry $Q$, all the previous
discussion globalizes. In particular, for a $Q$-closed local operator
${\cal O}(p)$ the $Q$-cohomology class of
${\cal O}(p)$ is independent of the point $p$. Thus, the $Q$-cohomology
classes of local operators on this global super-riemann surface
generate a ${\bf Z}/2{\bf Z}$-graded commutative ring    
called the {\sl chiral ring} of the theory. 



\subsection{Certain Correlation Functions and Their Basic Properties}


We will consider path integrals of the following types of path
integrals:
$$Z_{({\cal O}_\alpha)}=\int{\cal D}\Phi e^{-{\cal
L}}\prod_{\alpha=1}^n\int_{\Sigma_\alpha}{\cal O}_\alpha^{(k_\alpha)} $$
where for each $\alpha$ the operator ${\cal O}_\alpha$ is a $Q$-closed
local operator, $k_\alpha=0,1,2$ and $\Sigma_\alpha$ 
is a $k_\alpha$-cycle in the super-riemann surface $\Sigma$.

Here are the basic properties of $Z_{({\cal O}_\alpha)}$:

\noindent
1). It is independent of the metric on the super-riemann surface.

The reason is that as you vary the metric by $\delta g$ you get an
insertion of $\int_\Sigma\delta g\cdot T$, where $T$ is the
stress-energy tensor.
But as we have seen before\note{cross ref?} $T$ is $Q$-exact in the sense
that there is a field $S_{\mu\nu}$ with  $T_{\mu\nu}=\{Q,S_{\mu\nu}\}$.  We
then see that  this insertion produces a term of the form $\{Q,\cdot\}$.
The path integral of such a term is zero since since the fields are
$Q$-invariant. (Notice also that it is necessary to assume that $Q$
preserves the measure in the path integral.  This will be so
if the original theory was supersymmetric and the current $\frac{1}{2} J_L-
\frac{1}{2} J_R$ was gaugeable.   An important
 corollary of the fact that the correlation
function is metric-independent is that the zero-form operators
${\cal O}_\alpha^{(0)}$ must be scalar operators, of spin zero;
objects of higher spin would not be invariant under local rotations,
and their insertion in a correlation could not be metric-independent.)





\bigskip
\noindent
2). It is independent of most of the parameters in ${\cal L}$.

First, the correlation function is independent of any term
$\int d^2yd^4\theta(\cdots)$. The reason is that to evaluate such a
term we differentiate by all four of the $Q$'s.
For the same reason, only one of the terms
$$\int d^2yd\theta^+d\theta^-W+{\rm c.\ c.}$$
and
$$\int d^2yd\overline\theta^+d\theta^-\tilde W+{\rm c. c.}$$
can be non-trivial depending on which combination of $Q_+,Q_-,\overline
Q_+,\overline Q_-$ we are using.
In fact the correlation function is independent of one of these terms
and varies holomorphically with the other.

\bigskip
\noindent
3). We can calculate $Z_{({\cal O}_\alpha)}$ by using a fixed point
theorem applied to the action of $Q$. We consider $Q$ as acting
(infinitessimally) on the
function space of all the fields in the theory.
If this action has no fixed points, then the value of the path
integral would be zero. In general, we can reduce the computation to
an integral over the fixed point set of $Q$.  These of course are the
field configurations which are invariant under the supersymmetry $Q$.

\subsection{$Q$-cohomology of States}

Now we pass from the local operators, path integrals and correlation
functions world to the Hamiltonian framework.
Our riemann surface is now a cylinder $S^1\times {\bf R}$. We
trivialize all the relevant spin bundles over the $S^1$. We have the
Hilbert space ${\cal H}$ of states of the quantum theory over the
circle and the Hamiltonian operator $H$ on ${\cal H}$ associated with
time translation. Since $Q$ is acting on the theory it produces an
operator, which we also call $Q$, on ${\cal H}$ commuting with
$H$. Also, from the commutation 
relations in the superalgebra we see that
$$H=\{Q,\tilde Q\}$$
where $\tilde Q$ is the adjoint of $Q$. Of course, we still have the
relation $Q^2=0$, so that we can 
take the cohomology of the action of $Q$ on ${\cal H}$:
$$H^*(Q)={\rm ker}\,Q/{\rm Im}\,Q.$$
If our theory is sufficiently well-behaved so that $H$ has a discrete
spectrum, then $H^*(Q)$ is identified with $\{\psi|H\psi=0\}$.
(This is not the case for the gauge theory example we discuss later
today but it is true for the sigma models.)
In general $H^*(Q)$ is bigraded by the action of the $U(1)\times U(1)$
$R$-symmetry group.

\subsection{Comparison of $Q$-cohomology of local operators and
$Q$-cohomology of states} 

We have seen two types of $Q$-cohomology -- the first involving local
operators and yielding the chiral ring and the second involving states
on the circle. 
Let us construct a map between these $Q$-cohomologies. 
We consider a hemisphere

\centerline{\quad}
\centerline{\epsfxsize=1.5in\epsfbox{hemi.eps}}
\centerline{\quad}


Let ${\cal O}$ be a local operator at $p$. Doing the path
integral for the hemisphere over the space of fields with a given
boundary value produces a number, and hence a
function  of the  state on the boundary. This 
function is an element  in the Hilbert space of
the boundary.  This then gives a map  from  local
operators at $p$ to ${\cal H}$. One sees easily that it commutes with
the action of $Q$ and hence induces a map from the $Q$-cohomology of
local operators to the $Q$-cohomology of states on the circle:
$$\psi\colon ({\rm chiral\ ring})\to H^*(Q).$$

In examples we shall see that if the spectrum of $Q$ on ${\cal H}$ (at
least near zero) is discrete then $\psi$ is an isomorphism.

If $J_L$ and $J_R$ are both $R$-symmetries classically, then the
chiral ring and ${\rm ker}\,H$ are both bigraded by these
symmetries. Because of the anomaly, the map $\psi$ shifts the
bigrading
$$\psi\colon H^{p,q}(Q)\to {\rm ker}\,H^{p\pm k,q\pm k},$$
where $k$ is a measure of the anomaly.
Notice that the bigrading need not be by integers, only by rational
numbers whose denominators divide the order of the covering of
$U(1)\times U(1)$ that actually acts.


\section{A Gauge Theory Example}

The first example we take is that of gauge theory of a simple Lie
group $G$. We begin with $N=1$ supersymmetric gauge theory in four
dimensions and dimensionally reduce to an $N=2$ supersymmetric theory
in dimension two. As we saw in the last lecture, the basic field
strength is the field 
$$\Sigma=\{\overline{\cal D}_+,{\cal D}_-\},$$
which is a twisted chiral superfield with values in the adjoint bundle
of the principal bundle.
We use the expansion
$$\Sigma=\sigma+\cdots+\overline\theta^+\theta^-(F+iD).$$
The bosonic fields are the connection $A_\mu$ and  $\sigma,\overline
\sigma$.
Remember that the connection on four-space is
$$A^{(4)}=\sum_{\mu=0}^1A_\mu dy^\mu
+\sigma(dy^2-idy^3)+\overline\sigma(dy^2+idy^3).$$    
The two $R$-symmetries are:
$$J_A=J_L-J_R;\ \ J_V=J_L+J_R.$$
The first is induced by rotation in the $(y^2,y^3)$-plane and the second
comes from the chiral $U(1)$ $R$-symmetry in four dimensions.
(This symmetry is anomalous in four  dimensions but not in two
dimensions -- see the superhomework.)
We now twist by $J_V$, so that $SO(2)_E'$ is generated by $K'=K\pm
\frac{1}{2}J_V$.  Then $A_\mu$ remains a connection one-form and
$\sigma,\overline\sigma$ become $0$-forms. In general, this twist
eliminates spin bundles and turns  sections of the spin bundles with
differential forms. Then four spinor fields
$\lambda_\pm,\overline\lambda_\pm$ get twisted into
a $0$-form
$\eta=\psi_{(0)}$, a one form $\psi=\psi_{(1)}$ and a two-form
$\chi=\psi_{(2)}$. 
(This new theory is of course a topological theory existing on a
compact surface.  It is not a physical theory in the sense that there
is a Poincar\'e group action on the theory.)


In this example $J_A$ is twice the rotation of $(y^2,y^3)$-space,
normalized so that $[J_A,Q]=Q$.
Let us give the $J_A$ degrees of the various propagating fields in the
Lagrangian, cf. Formulae 7.15 of the superhomework. (The reference to
propagating fields, means that we have 
in mind using the equations of motion to set the auxiliary field
$F+iD$ equal to some expression in the other fields ($F$ is set
equal to the curvature of $A$ and $D$ is set equal to zero since there
are no matter fields.)
\begin{center}
\begin{tabular}{|l|c|c|c|c|c|}\hline
$J_A$-degree & $-2$ & $-1$ & $0$ & $1$ & $2$ \\ \hline
field & $\overline\sigma$ & $\psi_{(0)},\psi_{(2)}$ & $A_\mu$ &
$\psi_{(1)}$ & $\sigma$\\ \hline
\end{tabular}
\end{center}
Actually, the $J_A$-degree $-1$ and $+1$  terms could be reversed, but
after appropriately normalizing things the degrees will be as
indicated.

Recall from Lecture II-10\footnote{In that lecture we had a real field
$\phi$ instead of the complex field $\sigma$ here, but that changes
nothing.}  that
the super symmetry $Q$ lifts to a supersymmetry $\delta$ of the space
of super connections. It is not the case that $\delta^2=0$, but only
that $\delta^2$ is a gauge transformation. In fact,
$\delta^2=[\sigma,\cdot]$. In terms of the
coordinate fields  it is given by:
$$\delta  A=\psi_{(1)}$$
$$\delta\psi_{(1)}=-D_A\sigma$$
$$\delta\sigma=0.$$
The symbol $D_A$ of course refers to covariant derivative with respect
to the connection $A$.

In Lecture II-10  using the analogous symmetry led us to the
equivariant cohomology of the 
moduli space of connections, via the standard model of forms.  We let
${\cal G}$ be the Lie algebra of the gauge group $G$
and $M$ the space on which this group acts. Then we have the complex
$$\left(\Omega^*(M)\otimes Func({\cal G})\right)^G$$
with differential
$$D=d_M+i_{V(\phi)}.$$
Here, we use another closely related model which is not as well known
but should be.
We replace the Lie algebra ${\cal G}$ by its complexification ${\cal
G}_{\bf C}$ and we form the complex
$$\left(\Omega^*(M)\otimes \Omega^{0,*}({\cal G}_{\bf C})\right)^G.$$
Here complex linear function $\phi$ on ${\cal G}_{\bf C}$ have degree
$2$, the functions $\overline\phi$ have degree $-2$ and the
$(0,1)$-form $d\overline\phi$ has degree $-1$. The differential in
this complex is
$$\tilde D=d_M+i_{\tilde V(\phi)}+\overline\partial_{{\cal G}_{\bf
C}}$$
where $\tilde V$ is a vector field on $M\times {\cal G}_{\bf C}$
giving the action. To see that this complex also computes the
equivariant cohomology we view $\tilde D$ as a sum of  $(d_M+i_{\tilde
V(\phi)})$ and $\overline\partial_{{\cal G}_{\bf C}}$. We apply the
usual spectral sequence for the double complex, doing
$\overline\partial_{{\cal G}_{\bf C}}$ first.
Since this cohomology vanishes except in degree zero, where it
produces the polynomial functions on ${\cal G}_{\bf C}$, we see that
one recovers the same answer as for the first complex.
(Notice that this second approach leads to a Hodge theory
version of equivariant cohomology, using the operator $\tilde D^*\tilde D$
as a ``Laplacian''; it would be interesting to understand
 the resulting ${\bf L}^2$  version of equivariant cohomology.)


Let us complete the table describing the action of the supersymmetry
$\delta$ on the space of propagating fields.
The above model of equivariant cohomology translates into the
following formulae for the supersymmetry:
$$\delta  A=\psi_{(1)}$$
$$\delta\psi_{(1)}=-D_A\sigma$$
$$\delta\sigma=0$$
$$\delta\overline\sigma=\psi_{(0)}$$
$$\delta\psi_{(0)}=[\sigma,\overline\sigma]$$
$$\delta\psi_{(2)}=F_A.$$
It follows from the last equation that the fixed points of $\delta$
will be field configurations where the connection is flat,  the
section $\sigma$ is covariantly constant and commutes with
$\overline\sigma$. 
Next we see that the anomaly of $J_A$ is ${\rm dim}{G}\cdot
(-\chi(\Sigma))=(2g-2)\cdot {\rm dim}(G)$. The reason for this is that
the term that contributes to the anomaly in the measure of integration
for the path integral is ${\cal D}\psi_{(0)}{\cal D}\psi_{(1)}{\cal
D}\psi_{(2)}$. The symmetry $J_A$ of the Lagrangian acts by $-1$ on
$\psi_{(0)}$ and $\psi_{(2)}$ and by $+1$ on $\psi_{(1)}$. Thus, the
index computation for the anomaly becomes the dimension of the adjoint
representation times $-\chi(\Sigma)$. 
By invariance the correlation function
$$\langle
\prod_\alpha\int_{\Sigma_\alpha}{\cal O}_\alpha^{(k_\alpha)}\rangle=0$$  
unless $\sum_\alpha n_\alpha=(2g-2){\rm dim}(G)$ where $n_\alpha$ is 
the $J_A$-degree of ${\cal O}_\alpha^{(k_\alpha)}$. 
Let us consider a local operator which is of the form $P(\sigma)$ for
some gauge invariant polynomial on the Lie algebra of $G$.
The ring structure on  these operators is just the usual ring
structure on invariant polynomials.
It is easy to see that if $P$ is of degree  $r$, then the operator
${\cal O}_P^{(0)}$ has $J_A$-degree $2r$, and the descendants ${\cal
O}_P^{(k)}$ have $J_A$-degree equal to $2r-k$. Notice that these
degrees agree with the usual dimension of the analogous classes in
Donaldson theory.) 

Now let us calculate $\langle\prod_\alpha\int_{\Sigma_\alpha}{\cal
O}_\alpha^{(k_\alpha)}\rangle $ via  fixed point theory.
Recall that the Lagrangian is
$${\cal L}=\frac{1}{4e^2}\int d^2yd^4\theta{\rm
Tr}\overline\Sigma\Sigma.$$ 
We will recover a theoretical formula for this correlation function.
(We actually calculated the answer in Lecture
II-10.) 
In coordinates we have
\begin{equation}\label{lag}
{\cal L}=\frac{1}{4e^2}\int d^2y\left({\rm Tr}F^2+{\rm Tr}D\sigma
D\overline\sigma +{\rm
Tr}([\sigma,\overline\sigma]^2)+\cdots\right).
\end{equation}
By the fixed point theorem we need only calculate at points where the
curvature is zero and $D_A\sigma=[\sigma,\overline\sigma]=0$. 
There are two cases.
If $A$ is irreducible and $D_A\sigma=0$ then $\sigma=0$. These
solutions give  smooth points in the moduli space of classical
solutions to the equations of motion and this part of the
moduli space of classical $\delta$-invariant solutions is simply the
space of of 
irreducible representations of
$\pi_1(\Sigma)\to G$. The other possibility is that $A$ is reducible
and $\sigma\not=0$. These points produce singular points of the moduli
space of classical solutions.  Recall that $\sigma$
takes values in the 
complexified adjoint bundle.  Since $D_A\sigma=0$, $\sigma$ is
covariantly constant, and it suffices to study its structure over one
point of $\Sigma$. Since $\sigma$ and
$\overline\sigma$ commute, the $\sigma$ can be diagonalized
(conjugated into the Cartan subalgebra of the complexification of the
Lie algebra of $G$) by an element of $G$.

Now let us examine how perturbation theory will compute these
topological correlation functions. We will restrict to the smooth case
where $\sigma=0$. Of course, we are taking the  coupling
constant  $e$ in the Lagrangian, Equation~\ref{lag}  to be small. We
are reduced to 
computing an integral over the moduli space, taken in the supersense,
of classical solutions. Let us consider
integrating out the odd variables at a point $[\rho]$ of the ordinary
moduli space of flat connections.  The fermion
space is the space of zero modes of the $\psi_{(k)}$.  These are the
spaces of harmonic forms and hence are identified with the spaces 
$H^k(\Sigma,{\rm ad}(\rho))$. Since we are
assuming that $\rho$ is irreducible and that the moduli space is
smooth at $[\rho]$ we have that $H^0(\Sigma,{\rm
ad}(\rho))=H^2(\Sigma,{\rm ad}(\rho))=0$ and $H^1(\Sigma,{\rm
ad}(\rho))$ is the tangent space to the moduli space at $[\rho]$.
So in fact the space of odd directions at $[\rho]$ is identified with
$\Pi T{\cal M}_{[\rho]}$.  Thus, at the open subset of irreducible
connections, the supermanifold of classical 
solutions is $\Pi T{\cal M}$, the parity reversed 
tangent bundle over the moduli space of flat connection. 
As we have seen, the Berezinian of such a supermanifold is naturally
trivialized so that one integrates functions over this
supermanifold. We have also seen that  a function on this
supermanifold is the same thing as 
a differential form on ${\cal M}$ and integration of a function  over $\Pi
T{\cal M}$ is the same thing as the integral of the corresponding
differential form over ${\cal M}$. (This requires an orientation of
${\cal M}$ which is determined by an orientation of $\Sigma$.)

Of course, we wish to compute correlation functions involving the
operators $\int_{\Sigma_\alpha}{\cal O}_\alpha^{(k_\alpha)}$.
These operators determine functions on $\Pi T{\cal M}$ and the
correlation function of a product of them becomes the integral of the
product of the functions.
In this manner gauge invariant polynomials in $\sigma$ are mapped to
functions on $\Pi T{\cal M}$ and hence to differential forms on ${\cal
M}$. 
A crucial step in the reduction from polynomials to differential forms
is the observation
that with fermions present, it is no longer the case that
$\sigma=0$. Rather the equations of motion give
$$\Box \sigma=-[\psi_{(1)},\psi_{(1)}].$$
This is important because the right hand side is directly interpreted
as a two-form on moduli space (as the zero modes of $\psi_{(1)}$
represent one-forms on the moduli space), so the formula enables
us to express the function $\sigma$, which is formally of degree two
in the quantum field theory formalism, in terms of an ordinary two-form
on the moduli space.  One evaluates the correlation functions by 
integrating the differential forms obtained in this way over the moduli 
space.  (Of course, these computations have to be augmented by
computations at the components given by reducible connections and a
non-zero but parallel field $\sigma$.)


In this gauge theory example, because of the presence of reducible
connections, the Hamiltonian has a continuous
spectrum beginning at zero energy.  As a result
 the map from $Q$-cohomology of operators
to $Q$-cohomology of states is not very useful.


\section{A $\sigma$-model example}

Let us consider the $N=2$ supersymmetric $\sigma$-model 
$$\Phi\colon \Sigma\to X$$
where $X$ is a K\"ahler manifold and where $\Sigma$ is a split $2|4$
manifold as in the previous gauge theory example.
With this assumption, the fermions of the $\sigma$-model lie in
$$K_\Sigma ^{\pm 1/2}\otimes T^\pm X$$
where $K_\Sigma$ is the canonical bundle of $\Sigma$,
$T^+X=T^{1,0}X$ and $T^-X=T^{0,1}X$ are the holomorphic and
anti-holomorphic tangent bundle of $X$.
We denote these fermions by $\chi^{\pm,\pm}$.
The first sign refers to whether the fermion is left- or right-moving
(i.e., in whether it is a section of $K_\Sigma^+$ or $K_\Sigma^-$) and
the second sign refers to the type of the section 
(holomorphic or anti-holomorphic) on $X$.
Once again we twist the theory for it to exist globally on $\Sigma$
replacing $K$ by
$$K'=K\pm \frac{1}{2}J_L\pm \frac{1}{2} J_R.$$
Depending on the choice of signs in this twisting we get two possible
cases called the $A$-model (using $-\frac{1}{2}J_L+\frac{1}{2}J_R$) or
the $B$-model (using $\frac{1}{2}J_L+\frac{1}{2}J_R$).

\subsection{The $A$-model}

The zero-forms are $\chi^{++}\in K^{1/2}\otimes T^{1,0}X$ and
$\chi^{--}\in K^{-1/2}\otimes T^{0,1}$. The ring of $0$-form
operators is
$$F(\Phi,\chi^{++},\chi^{--})=F(\Phi,\partial \phi,\overline\partial
\phi).$$
This is simply the ring of functions on $\Pi TX$ which  is identified
with the ring of differential forms on the K\"ahler manifold.
Furthermore, with this identification of a differential form $F$ with
a corresponding operator ${\cal O}_F$, we have
$$Q{\cal O}_F={\cal O}_{dF},$$
so that 
$Q$ becomes the ordinary differential.
The chiral ring is then $H^*(X)$, the usual topological cohomology of
$X$, and the bigrading induced by $J_L$ and $J_R$ is the usual Hodge
bigrading.
In the space of maps the $Q$-fixed points are the holomorphic curves
$$\phi\colon \Sigma\to X.$$
Thus, to compute correlation functions in this model we must do integrals
over the spaces of holomorphic curves in $X$. This will lead, as we
shall see in the next lecture, to quantum cohomology.

\subsection{The $B$-model}
It is possible to construct this model only  in the case $X$ is
Calabi-Yau, for 
this is the only case when the symmetry ($J_L+J_R$) by which we wish  to
twist  is not anomalous.\footnote{In other words, if $c_1(X)$ is nonzero,
then $J_L+J_R$ is simply not a symmetry of the quantum theory, even
before gauging.  The discussion of the obstruction to gauging in the
first part of this lecture assumed that $J_L$ and $J_R$ both generated
symmetries prior to gauging, and thus does not apply to this case.}
We fix a non-zero holomorphic $n$-form $\omega$ on $X$.
In this case the zero-forms are
$\chi^{+-}\in K^{1/2}\otimes T^{1,0}X$ and $\chi^{--}\in
K^{-1/2}\otimes T^{0,1}X$. The ring of $0$-form operators is
$$F(\Phi,\chi^{+-},\chi^{--})$$
which is the ring 
$$\Omega^{0,*}(X)\otimes
\Omega^{0,*}(X)=\Omega^{0,*}\left(X,\wedge^*T^{1,0}(X)\right).$$
Under this identification we have
$$Q{\cal O}_F={\cal O}_{\overline\partial F},$$
so that the chiral ring is
$$\oplus_{q,p}H^q(X;\wedge^pT^{1,0}X).$$

The $Q$-fixed points in the space of fields are the constant maps
$\Sigma\to X$. Thus, the computations of correlation functions 
in this model will become integrals over $X$ 
$$\langle \prod_{i=1}^s{\cal
O}_{F_i}\rangle=\int_X\left(\wedge_{i=1}^sF_i\right)\cdot
\omega^{\otimes 2}.$$ 
These integrals lie squarely in classical mathematics; they have to do
with variation of Hodge structures under deformation of complex
structure.  
\end{document}







