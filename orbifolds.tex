\section{Other AdS$_5$ Backgrounds}
\label{other_backgrounds}

Up to now we have limited our discussion to the $AdS_5\times S^5$
background of type IIB string theory; in section \ref{deformations}
we will describe backgrounds which are
related to it by deformations. However, 
it is clear from the
description of the correspondence in sections \ref{correspondence} and
\ref{correlators} that a similar
correspondence may be defined for any theory of quantum gravity whose
metric includes an $AdS_5$ factor; the generalization of
equation (\ref{genera}) relates such a theory
to a four dimensional conformal field theory. The background does not
necessarily have to be of the form $AdS_5\times X$; it is enough that
it has an $SO(4,2)$ isometry symmetry, and more general possibilities
in which the curvature of $AdS_5$ depends on the position in $X$ are
also possible \cite{vanNieuwenhuizen:1985ri}. It is necessary,
however, for the $AdS$ theory to be a theory of quantum gravity, since
any conformal theory has an energy-momentum tensor operator that is
mapped by the correspondence to the graviton on $AdS_5$\footnote{If we
have a topological field theory on the boundary the bulk theory does
not have to be gravitational, as in \cite{Gopakumar:1998ki}.}. Thus,
we would like to discuss compactifications of string theory or M
theory, which are believed to be consistent theories of quantum
gravity, on backgrounds involving $AdS_5$. For simplicity we will only
discuss here backgrounds which are direct products of the form
$AdS_5\times X$.

Given such a background of string/M theory, it is not apriori clear
what is the conformal field theory to which it corresponds. A special
class of backgrounds are those which arise as near-horizon limits of
branes, like the $AdS_5\times S^5$ background. In this case one can
sometimes analyze the low-energy field theory on the branes by
standard methods before taking the near-horizon limit, and after the
limit this becomes the dual conformal field theory. The most
well-studied case is the case of D3-branes in type IIB string
theory. When the D3-branes are at a generic point in space-time the
near-horizon limit gives the $AdS_5\times S^5$ background discussed
extensively above. However, if the transverse space to the D3-branes
is singular, the near-horizon limit and the corresponding field theory
can be different. The simplest case is the case of a D3-brane on an
orbifold \cite{Kachru:1998ys} or orientifold \cite{Witten:1998xy}
singularity, which can be analyzed by perturbative string theory
methods. These cases will be discussed in sections \ref{orbifolds} and
\ref{orientifolds}. Another interesting case is the conifold
singularity \cite{Klebanov:1998hh} and its generalizations, which will
be discussed in section \ref{conifolds}. In this case a direct
analysis of the field theory is not possible, but various indirect
arguments can be used to determine what it is in many cases.

Not much is known about more general cases of near-horizon limits of
D3-branes, which on the string theory side were analyzed in
\cite{Figueroa-O'Farrill:1998nb,Acharya:1998db,Morrison:1998cs,
Ray:1999qj,Figueroa-O'Farrill:1999va}, and
even less is known about backgrounds which are not describable as
near-horizon limits of branes (several $AdS_5$ backgrounds were discussed
in \cite{Duff:1998us}). An example of the latter is the
$AdS_5\times \CP^3$ background of M theory \cite{Pope:1989xj}, which
involves a 4-form flux on the 4-cycle in $\CP^3$. Using the methods
described in the previous sections we can compute various properties
of such compactifications in the large $N$ limit, such as the mass
spectrum and the central charge of the corresponding field theories
(for the $AdS_5\times \CP^3$ compactification one finds a central
charge proportional to $N^3$, where $N$ is the 4-form flux). However,
it is not known how to construct an alternative description of the
conformal field theory in most of these cases, except for the cases
which are related by deformations to the better-understood orbifold
and conifold compactifications.

Some of the $AdS_5\times X$ backgrounds of string/M theory preserve
some number of supersymmetries, but most of them (such as the
$AdS_5\times \CP^3$ background) do not. In supersymmetric cases,
supersymmetry guarantees the stability of the corresponding solutions.
In the non-supersymmetric cases various instabilities may arise for
finite $N$ (see, for instance, \cite{Berkooz:1999qp,Berkooz:1999ji}) which may
destroy the conformal ($SO(4,2)$) invariance, but the correspondence
is still conjectured to be valid when all quantum corrections are taken
into account (or in the infinite $N$ limit for which the supergravity
approximation is valid). One type of instability
%, which we already discussed
%in section \ref{deformations}, 
occurs when the spectrum includes a
tachyonic field whose mass is below the Breitenlohner-Freedman
stability bound. Such a field is expected to condense just like a
tachyon in flat space, and generally it is not known what this
condensation leads to. If the classical supergravity spectrum includes
a field which saturates the stability bound, an analysis of the
quantum corrections is necessary to determine whether they raise the
mass squared of the field (leading to a stable solution) or lower it
(leading to an unstable solution). Apriori one would not expect to
have a field which exactly saturates the bound (corresponding to an
operator in the field theory whose dimension is exactly $\Delta=2$) in
a non-supersymmetric theory, but this often happens in orbifold
theories for reasons that will be discussed below. Another possible
instability arises when there is a massless field in the background,
corresponding to a marginal operator in the field theory. Such a field
(the dilaton) exists in all classical type IIB compactifications, and
naively corresponds to an exactly marginal deformation of the theory
even in the non-supersymmetric cases. However, for finite $N$ one
would expect quantum corrections to generate a potential for such a
field (if it is neutral under the gauge symmetries), which could drive
its expectation value away from the range of values where the
supergravity approximation is valid. Again, an analysis of the quantum
corrections is necessary in such a case to determine if the theory has
a stable vacuum (which may or may not be describable in supergravity),
corresponding to a fixed point of the corresponding field theory, or
if the potential leads to a runaway behavior with no stable
vacuum. Another possible source of instabilities is related to the
possibility of forming brane-anti-brane pairs in the vacuum (or,
equivalently, the emission of branes which destabilize the vacuum)
\cite{Brown:1988kg,Dowker:1996sg,Maldacena:1999uz,Seiberg:1999xz}; 
one would expect such an instability to arise, for example,
in cases where we look at the near-horizon limit of $N$ 3-branes which
have a repulsive force between them. For all these reasons, the study
of non-supersymmetric backgrounds usually requires an understanding of
the quantum corrections, which are not yet well-understood either in
M theory or in type IIB compactifications with RR backgrounds. Thus,
we will focus here on supersymmetric backgrounds, for which the
supergravity approximation is generally valid. In the
non-supersymmetric cases the correspondence is still expected to be
valid, and in the extreme large $N$ limit it can also be studied using
supergravity, but getting finite $N$ information usually requires
going beyond the SUGRA approximation. It would be very interesting to
understand better the quantum corrections in order to study
non-supersymmetric theories at finite $N$ using the AdS/CFT
correspondence.

\subsection{Orbifolds of $AdS_5\times S^5$}
\label{orbifolds}

The low-energy field theory corresponding to D3-branes at orbifold
singularities may be derived by string theory methods
\cite{Douglas:1996sw,Douglas:1997de}. 
Generally the gauge group is of the form $\prod_i U(a_i N)$, and there
are various bifundamental (and sometimes also adjoint) matter
fields\footnote{In general one can choose to have the orbifold group
act on the Chan-Paton indices in various ways. We will discuss here
only the case where the group acts as $N$ copies of the regular
representation of the orbifold group $\Gamma$, which is the only case
which leads to conformal theories. Other representations involve also
5-branes wrapped around 2-cycles, so they do not arise in the naive
near-horizon limit of D3-branes. The $AdS_5$ description of this was
given in \cite{Gubser:1998fp}.}. We are interested in the near-horizon
limit of D3-branes sitting at the origin of $\IR^4 \times
\IR^6/\Gamma$ for some finite group $\Gamma$ which is a discrete
subgroup of the $SO(6)
\simeq SU(4)_R$ rotation symmetry \cite{Kachru:1998ys}. 
If $\Gamma \subset SU(3) \subset
SU(4)_R$ the theory on the D3-branes has $\cn=1$ supersymmetry, and if
$\Gamma \subset SU(2) \subset SU(4)_R$ it has $\cn=2$
supersymmetry. The near-horizon limit of such a configuration is of
the form $AdS_5 \times S^5/\Gamma$ (since the orbifold commutes with
taking the near-horizon limit), and corresponds (at least for large
$N$) to a conformal theory with the appropriate amount of
supersymmetry. Note that on neither side of the correspondence is the
orbifolding just a projection on the $\Gamma$-invariant states of the
original theory -- on the string theory side we need to add also
twisted sectors, while on the field theory side the gauge group is
generally much larger (though the field theory 
can be viewed as a projection of the
gauge theory corresponding to ${\rm dim}(\Gamma)\cdot N$ D-branes).

We will start with a general analysis of the orbifold, and then discuss
specific examples with different amounts of supersymmetry\footnote{
We will not discuss here orbifolds that act non-trivially
on the AdS space, as in
\cite{Gao:1999er}.}. The action
of $\Gamma$ on the $S^5$ is the same as its action on the angular
coordinates of $\IR^6$. If the original action of $\Gamma$ had only
the origin as its fixed point, the space $S^5/\Gamma$ is smooth. On
the other hand, if the original action had a space of fixed points,
some fixed points remain, and the space $S^5/\Gamma$ includes orbifold
singularities. In this case the space is not geometrically smooth, and
the supergravity approximation is not valid (though of course in
string theory it is a standard orbifold compactification which is 
generically not
singular). The spectrum of string theory on $AdS_5\times S^5/\Gamma$
includes states from untwisted and twisted sectors of the
orbifold. The untwisted states are just the $\Gamma$-projection of the
original states of $AdS_5\times S^5$, and they include in particular
the $\Gamma$-invariant supergravity states. These states have (in the
classical supergravity limit) the same masses as in the original
$AdS_5\times S^5$ background \cite{Oz:1998of}, 
corresponding to integer dimensions in
the field theory, which is why we often find in orbifolds operators of
dimension 2 or 4 which can destabilize non-supersymmetric
backgrounds. If the orbifold group has fixed points on the $S^5$,
there are also light twisted sector states that are localized near
these fixed points, which need to be added to the supergravity fields
for a proper description of the low-energy dynamics. On the other
hand, if the orbifold has no fixed points, all twisted sector states
are heavy,\footnote{ Note that this happens even when in the original
description there were massless twisted sector states localized at the
origin.} since they involve strings stretching between identified
points on the $S^5$. In this case the twisted sector states decouple
from the low-energy theory in space-time (for large $g_s N$). There is a global
$\Gamma$ symmetry in the corresponding field theory, under which the
untwisted sector states are neutral while the twisted sector states
are charged.

In the 't Hooft limit of $N \to \infty$ with $g_s N$ finite, all the
solutions of the form $AdS_5\times S^5/\Gamma$ have a fixed line
corresponding to the dilaton, indicating that the beta function of the
corresponding field theories vanishes in this limit
\cite{Kachru:1998ys}. In fact, one can prove 
\cite{Lawrence:1998ja,Bershadsky:1998mb,Bershadsky:1998cb} (see also
\cite{Schmaltz:1999bg,Erlich:1998gb}) that in this
limit, which corresponds to keeping only the planar diagrams in the
field theory, all the correlation functions of the untwisted sector
operators in the orbifold theories are the same (up to multiplication by
some power of ${\rm dim}(\Gamma)$) as in the $\cn=4$ SYM theory
corresponding to $AdS_5\times S^5$ \footnote{There is no similar relation for
the twisted sector operators.}. This is the analog of the usual
string theory statement that at tree-level the interactions of
untwisted sector states are exactly inherited from those of the
original theory before the orbifolding. For example, the central
charge of the field theory (appearing in the 2-point function of the
energy-momentum tensor) is (in this limit) just ${\rm dim}(\Gamma)$
times the central charge of the corresponding $\cn=4$ theory. This may
easily be seen also on the string theory side, where the central
charge may be shown \cite{Gubser:1999vd} to be inversely proportional
to the volume of the compact space (and ${\rm Vol}(S^5/\Gamma) = {\rm
Vol}(S^5)/{\rm dim}(\Gamma)$).

The vanishing of the beta function in the 't Hooft limit follows from
this general result (as predicted by the AdS/CFT correspondence). This
applies both to orbifolds which preserve supersymmetry and to those
which do not, and leads to many examples of supersymmetric and
non-supersymmetric theories which have fixed lines in the large $N$
limit. At subleading orders in $1/N$, the correlation functions
differ between the orbifold theory and the $\cn=4$ theory, and in
principle a non-zero beta function may arise. In supersymmetric
orbifolds supersymmetry prevents this\footnote{At least, it prevents a
potential for the dilaton, so there is still some fixed line in the
field theory, though it can be shifted from the $\cn=4$ fixed line
when $1/N$ corrections are taken into account.}, but in
non-supersymmetric theories generically there will no longer be a
fixed line for finite $N$. The dilaton potential is then related to
the appearance of a non-zero beta function in the field theory, and
the minima of this potential are related to the zeros of the field
theory beta function for finite $N$.

As a first example we can analyze the case \cite{Kachru:1998ys} of
D3-branes on an $\IR^4/\IZ_k$ orbifold singularity, which preserves
$\cn=2$ supersymmetry. Before taking the near-horizon limit, the
low-energy field theory (at the free orbifold point in the string
theory moduli space) is a $U(N)^k$ gauge theory with bifundamental
hypermultiplets in the $\bf{(N,{\bar N},1,\cdots,1)+(1,N,{\bar
N},1,\cdots,1)+\cdots+ (1,\cdots,1,N,{\bar N})+({\bar
N},1,\cdots,1,N)}$ representation. The bare gauge couplings $\tau_i$
of all the $U(N)$ theories are equal to the string coupling
$\tau_{IIB}$ at this point in the moduli space. In the near-horizon
(low-energy) limit this field theory becomes the $SU(N)^k$ field
theory with the same matter content, since the off-diagonal $U(1)$
factors are IR-free\footnote{This does not contradict our previous
statements about the beta functions since the $U(1)$ factors are
subleading in the $1/N$ expansion, and the operators corresponding
to the off-diagonal $U(1)$'s come from twisted sectors.} 
(and the diagonal $U(1)$ factor is
decoupled here and in all other examples in this section so we will
ignore it).  This theory is dual to type IIB string theory on
$AdS_5\times S^5/\IZ_k$, where the $\IZ_k$ action leaves fixed an
$S^1$ inside the $S^5$.

This field theory is known (see, for instance, \cite{Witten:1997so})
to be a finite field theory for any value of the $k$ gauge couplings
$\tau_i$, corresponding to a $k$-complex-dimensional surface of
conformal field theories. Thus, we should see $k$ complex parameters
in the string theory background which we can change without destroying
the $AdS_5$ component of this background. One such parameter is
obviously the dilaton, and the other $(k-1)$ may be identified
\cite{Kachru:1998ys} with the values of the NS-NS and R-R 2-form
$B$-fields on the $(k-1)$ 2-cycles which vanish at the $\IZ_k$
orbifold singularity (these are part of the blow-up modes for the
singularities; the other blow-up modes turn on fields which change the
$AdS_5$ background, and correspond to non-marginal deformations of the
field theory).

The low-energy spectrum has contributions both from the untwisted and
from the twisted sectors. The untwisted sector states are just the
$\IZ_k$ projection of the original $AdS_5\times S^5$ states. The
twisted sector states are the same (for large $N$ and at low energies)
as those which appear in flat space at an $\IR^4/\IZ_k$ singularity,
except that here they live on the fixed locus of the $\IZ_k$ action
which is of the form $AdS_5\times S^1$. At the orbifold point the
massless twisted sector states are $(k-1)$ tensor multiplets (these
tensor multiplets include scalars corresponding to the 2-form
$B$-fields described above). Upon dimensional reduction on the $S^1$
these give rise to $(k-1)$ $U(1)$ gauge fields on $AdS_5$, which
correspond to the $U(1)$ global symmetries of the field theory (which
were the off-diagonal gauge $U(1)$'s before taking the near-horizon
limit, and become global symmetries after this limit); see, e.g.
\cite{Hanany:1998it}. The orbifold
point corresponds to having all the $B$-fields maximally turned on
\cite{Aspinwall:1995zi}. 
The spectrum of fields on $AdS_5$ in this background was successfully
compared \cite{Gukov:1998kk} to the spectrum of chiral operators in
the field theory. If we move in the string theory moduli space to a
point where the $B$-fields on some 2-cycles are turned off, the
D3-branes wrapped around these 2-cycles become tensionless, and the
low-energy theory becomes a non-trivial $\cn=(2,0)$ six dimensional
SCFT (see
\cite{Seiberg:1997ax} and references therein). The
low-energy spectrum on $AdS_5$ then includes the dimensional reduction
of this conformal theory on a circle. In particular, when all the
$B$-fields are turned off, we get the $A_{k-1}$ $(2,0)$ theory, which
gives rise to $SU(k)$ gauge fields at low-energies upon
compactification on a circle. Thus, the AdS/CFT correspondence
predicts an enhanced global $SU(k)$ symmetry at a particular point in
the parameter space of the corresponding field theory. Presumably,
this point is in a very strongly coupled regime (the string coupling
$\tau_{IIB} \propto
\sum_i \tau_i$ may be chosen to be weak, but individual $\tau_i$'s can
still be strongly coupled) which cannot be accessed directly in the
field theory. The field theory in this case has a large group of
duality symmetries \cite{Witten:1997so}, which includes (but is not
limited to) the $SL(2,\IZ)$ subgroup which acts on the couplings as
$\tau \to (a \tau + b) / (c \tau + d)$ at the point where they are all
equal. In the type IIB background the $SL(2,\IZ)$ subgroup of this
duality group is manifest, but it is not clear how to see the rest of
this group.

Our second example corresponds to D3-branes at an $\IR^6/\IZ_3$
orbifold point, where, if we write $\IR^6$ as $\IC^3$ with complex
coordinates $z_j$ ($j=1,2,3$), the $\IZ_3$ acts as $z_j \to e^{2\pi
i/3} z_j$. In this case the only fixed point of the $\IZ_3$ action is
the origin, so in the near-horizon limit we get \cite{Kachru:1998ys}
$AdS_5\times S^5/\IZ_3$ where the compact space is smooth. Thus, the
low-energy spectrum in this case (for large $g_s N$) includes only the
$\IZ_3$ projection of the original supergravity spectrum, and all
twisted sector states are heavy in this limit.

The corresponding field theory may be derived by the methods of
\cite{Douglas:1996sw,Douglas:1997de}. 
It is an $SU(N)^3$ gauge theory, with chiral multiplets $U_j$
($j=1,2,3$) in the $\bf{(N,{\bar N},1)}$ representation, $V_j$
($j=1,2,3$) in the $\bf{(1,N,{\bar N})}$ representation, and $W_j$
($j=1,2,3$) in the $\bf{({\bar N},1,N)}$ representation, and a
classical superpotential of the form $W = g\epsilon^{ijk} U_i V_j
W_k$. In the classical theory all three gauge couplings and the
superpotential coupling $g$ are equal (and equal to the string
coupling). In the quantum theory one can prove that in the space of
these four parameters there is a one dimensional line of
superconformal fixed points. The parameter which parameterizes this fixed
line (which passes through weak coupling in the gauge theory) may be
identified with the dilaton in the $AdS_5\times S^5/\IZ_3$
background. Unlike the previous case, here there are no indications of
additional marginal deformations, and no massless twisted sector
states on $AdS_5$ which they could correspond to.

As in the previous case, one can try to compare the spectrum of fields
on $AdS_5$ with the spectrum of chiral operators in the field
theory. In this case, as in all cases with less than $\cn=4$
supersymmetry, not all the supergravity fields on $AdS_5$ are in
chiral multiplets, since the $\cn=4$ chiral multiplets split into
chiral, anti-chiral and non-chiral multiplets when decomposed into
$\cn=2$ (or $\cn=1$) representations\footnote{Note that this means
that unlike the $AdS_5\times S^5$ case, in cases with less SUSY there
are always non-chiral operators which have a finite dimension in the
large $N, g_{YM}^2 N$ limit.} (in general there can also be various sizes of
chiral multiplets). However, one can still compare those of
the fields which are in chiral multiplets (and have the appropriate
relations between their AdS mass / field theory dimension and their
R-charges). The untwisted states may easily be matched since they are
a projection of the original states both in space-time and in the
field theory (if we think of the field theory as a projection of the
$\cn=4$ $SU(3N)$ gauge theory). Looking at the twisted sectors we seem
to encounter a paradox \cite{Morrison:1998cs}. On the string theory
side all the twisted sector states are heavy (they correspond to
strings stretched across the $S^5$, so they would correspond to
operators with $\Delta
\simeq mR \simeq R^2 / l_s^2 \simeq (g_s N)^{1/2}$). On the field
theory side we can identify the twisted sector fields with operators
which are charged under the global $\IZ_3$ symmetry which rotates the
three gauge groups, and naively there exist chiral operators which are
charged under this symmetry and remain of finite dimension in the
large $N,g_{YM}^2 N$ limit. However, a careful analysis shows that all
of these operators are in fact descendants, so their dimensions are
not protected. For example, the operator $\sum_{j=1}^3 e^{2\pi ij/3}
\tr((W_{\alpha}^{(j)})^2)$, where $W_{\alpha}^{(j)}$ is the field
strength multiplet of the $j$'th $SU(N)$ group, seems to be a chiral
superfield charged under the $\IZ_3$ symmetry. However, using linear
combinations of the Konishi anomaly equations
\cite{Konishi:1984hf,Konishi:1985tu} for the three gauge groups, one
can show that this operator (and all other ``twisted sector''
operators) is in fact a descendant, so there is no paradox. The
AdS/CFT correspondence predicts that in the large $N$, $g_{YM}^2 N$
limit the dimension of all these $\IZ_3$-charged operators scales as
$(g_{YM}^2 N)^{1/2}$, which is larger than the scaling $\Delta \sim
(g_{YM}^2 N)^{1/4}$ for the non-chiral operators in the $\cn=4$ SYM
theory in the same limit. It would be interesting to verify this
behavior in the field theory. Baryon-like operators also exist in
these theories \cite{Gukov:1998kn}, which are similar to those
which will be discussed in section \ref{conifolds}.

There are various other supersymmetric orbifold backgrounds which
behave similarly to the examples we have described in detail
here. There are also many non-supersymmetric examples 
\cite{Frampton:1998en,Frampton:1999ti} but, as
described above, their fate for finite $N$ is not clear, and we will
not discuss them in detail here.

\subsection{Orientifolds of $AdS_5\times S^5$}
\label{orientifolds}

The discussion of the near-horizon limits of D3-branes on orientifolds
is mostly similar to the discussion of orbifolds, except for the
absence of twisted sector states (which do not exist for
orientifolds). We will focus here on two examples which illustrate
some of the general properties of these backgrounds. Additional
examples were discussed in \cite{Kakushadze:1998tr,Kakushadze:1998tz,
Kakushadze:1999hb,Kakushadze:1998yq,
Ahn:1998tv,
Park:1998zh,Gukov:1998kt,Gremm:1999jj}.

Our first example is the near-horizon limit of D3-branes on an
orientifold 3-plane. The orientifold breaks the same supersymmetries
as the 3-branes do, so in the near horizon limit we have the full 32
supercharges corresponding to a $d=4, \cn=4$ SCFT. In flat space there
are (see \cite{Giveon:1998sr} and references therein) two types of
orientifold planes which lead to different projections on D-brane
states. One type of orientifold plane leads to a low-energy $SO(2N)$
$\cn=4$ gauge theory for $N$ D-branes on the orientifold, while the
other leads to a $USp(2N)$ $\cn=4$ gauge theory. In the first case we
can also have an additional ``half D3-brane'' stuck on the
orientifold, leading to an $SO(2N+1)$ $\cn=4$ gauge theory. In the
near-horizon limits of branes on the orientifold we should be able to
find string theory backgrounds which are dual to all of these gauge
theories.

The near-horizon limit of these brane configurations is type IIB
string theory on $AdS_5\times S^5/\IZ_2 \equiv AdS_5\times \RP^5$, where
the $\IZ_2$ acts by identifying opposite points on the $S^5$, so there
are no fixed points and the space $\RP^5$ is smooth. The manifestation
of the orientifolding in the near-horizon limit is that when a string
goes around a non-contractible cycle in $\RP^5$ (connecting opposite
points of the $S^5$) its orientation is reversed. In all the cases
discussed above the string theory perturbation expansion had only
closed orientable surfaces, so it was a power series in $g_s^2$ (or in
$1/N^2$ in the 't~Hooft limit); but in this background we can also
have non-orientable closed surfaces which include crosscaps, and the
perturbation expansion includes also odd powers of $g_s$ (or of $1/N$
in the 't~Hooft limit). In fact, it has long been known
\cite{Cicuta:1982fu} that in the 't~Hooft limit the $SO(N)$ and
$USp(N)$ gauge theories give rise to Feynman diagrams that involve
also non-orientable surfaces (as opposed to the $SU(N)$ case which
gives only orientable surfaces), so it is not surprising that such
diagrams arise in the string theory which is dual to these
theories. While in the cases described above the leading correction in
string perturbation theory was of order $g_s^2$ (or $1/N^2$ in the 't
Hooft limit), in the $AdS_5\times \RP^5$ background (and in general in
orientifold backgrounds) the leading correction comes from $\RP^2$
worldsheets and is of order $g_s$ (or $1/N$ in the 't Hooft
limit). Such a correction appears, for instance, in the computation of
the central charge (the 2-point function of the energy-momentum tensor) of
these theories, which is proportional to the dimension of the
corresponding gauge group.

Our discussion so far has not distinguished between the different
configurations corresponding to $SO(2N)$, $SO(2N+1)$ and $USp(2N)$
groups (the only obvious parameter in the orientifold background is
the 5-form flux $N$). In the Feynman diagram expansion it is
well-known \cite{Mkrtchian:1981bb,Cvitanovic:1982bq} that the $SO(2N)$
and $USp(2N)$ theories are related by a transformation taking $N$ to
$(-N)$, which inverts the sign of all diagrams with an odd number of
crosscaps in the 't Hooft limit. Thus, we should look for a similar
effect in string theory on $AdS_5\times \RP^5$. It turns out
\cite{Witten:1998xy} that this is implemented by a ``discrete
torsion'' on $\RP^5$, corresponding to turning on a $B_{NS-NS}$ 2-form
in the non-trivial cohomology class of $H^3(\RP^5, {\tilde \IZ}) =
\IZ_2$. The effect of turning on this ``discrete torsion'' is exactly
to invert the sign of all string diagrams with an odd number of
crosscaps. It is also possible to turn on a similar ``discrete
torsion'' for the RR 2-form $B$-field, so there is a total of four
different possible string theories on $AdS_5\times \RP^5$. It turns
out that the theory with no $B$-fields is equivalent to the $SO(2N)$
$\cn=4$ gauge theory, which is self-dual under the S-duality group
$SL(2,\IZ)$. The theory with only a non-zero $B_{RR}$ field is
equivalent to the $SO(2N+1)$ gauge theory, while the theories with
non-zero $B_{NS-NS}$ fields are equivalent to the $USp(2N)$ gauge
theory \cite{Witten:1998xy}, and this is consistent with the action of
S-duality on these groups and on the 2-form $B$-fields (which are a
doublet of $SL(2,\IZ)$).

An interesting test of this correspondence is the matching of chiral
primary fields. In the supergravity limit the fields on $AdS_5\times
\RP^5$ are just the $\IZ_2$ projection of the fields on $AdS_5\times
S^5$, including the multiplets with $n=2,4,6,\cdots$ (in the notation
of section \ref{tests}). This matches with almost all the chiral
superfields in the corresponding gauge theories, which are described
as traces of products of the fundamental fields as in section
\ref{tests}, but with the trace of a product of an odd number of
fields vanishing in these theories from symmetry arguments. However,
in the $SO(2N)$ gauge theories (and not in any of the others) there is
an additional gauge invariant chiral superfield, called the Pfaffian,
whose lowest component is of the form $\epsilon^{a_1 a_2 \cdots
a_{2N}} \phi^{I_1}_{a_1 a_2} \phi^{I_2}_{a_3 a_4} \cdots
\phi^{I_N}_{a_{2N-1} a_{2N}}$, where $a_i$ are $SO(2N)$ indices and
the $I_j$ are (symmetric traceless) indices in the $\bf 6$ of
$SU(4)_R$. The supersymmetry algebra guarantees that the dimension of
this operator is $\Delta=N$, and it is independent of the other
gauge-invariant chiral superfields. This operator may be identified
with the field on $AdS_5$ corresponding to a D3-brane wrapped around a
3-cycle in $\RP^5$, corresponding to the homology class $H_3(\RP^5,
\IZ) = \IZ_2$. This wrapping is only possible when no $B$-fields are
turned on \cite{Witten:1998xy}, consistent with such an operator
existing for $SO(2N)$ but not for $SO(2N+1)$ or $USp(2N)$. While it is
not known how to compute the mass of this state directly, the
superconformal algebra guarantees that it has the right mass to
correspond to an operator with $\Delta=N$; the naive approximation to
the mass, since the volume of the 3-cycle in $\RP^5$ is $\pi^2 R^3$, 
is $m R \simeq R
\cdot \pi^2 R^3 / (2\pi)^3 g_s l_s^4 = R^4 / 8\pi l_p^4 \simeq N$
(since in the orientifold case $R^4 \simeq 4\pi (2N) l_p^4$ instead of
equation (\ref{dthree})), which leads to the correct dimension for
large $N$. The existence of this operator (which decouples in the
large $N$ limit) is an important test of the finite $N$
correspondence. Anomaly matching in this background was discussed in
\cite{Blau:1999vz}.

Another interesting background is the near-horizon limit of D3-branes
on an orientifold 7-plane, with 4 D7-branes coincident on the
orientifold plane to ensure \cite{Sen:1996vd,Banks:1996nj} 
that the dilaton is constant and
the low-energy theory is conformal (this is the same as D3-branes in
F-theory \cite{Vafa:1996xn} 
at a $D_4$-type singularity). The field theory we get
in the near-horizon limit in this case is
\cite{Aharony:1997en,Douglas:1997js} an $\cn=2$ SQCD
theory with $USp(2N)$ gauge group, a hypermultiplet in the
anti-symmetric representation and four hypermultiplets in the
fundamental representation. In this case the orientifold action has
fixed points on the $S^5$, so the near-horizon limit is
\cite{Fayyazuddin:1998fb,Aharony:1998xz} type IIB string theory on
$AdS_5\times S^5/\IZ_2$ where the $\IZ_2$ action has fixed points on
an $S^3$ inside the $S^5$. Thus, this background includes an
orientifold plane with the topology of $S^3\times AdS_5$, and the
D7-branes stretched along the orientifold plane also remain as part of
the background, so that the low-energy theory includes both the
supergravity modes in the bulk and the $SO(8)$ gauge theory on the
D7-branes (which corresponds to an $SO(8)$ global symmetry in the
corresponding field theories)\footnote{Similar backgrounds were
discussed in \cite{Kehagias:1998gn}.}. The string perturbation expansion
in this case has two sources of corrections of order $g_s$, the
crosscap diagram and the open string disc diagram with strings ending
on the D7-brane, leading to two types of corrections of order $1/N$ in
the 't~Hooft limit. Again, the spectrum of operators in the field
theory may be matched \cite{Aharony:1998xz} with the spectrum of
fields coming from the dimensional reduction of the supergravity
theory in the bulk and of the 7-brane theory wrapped on the $S^3$. The
anomalies may also be matched to the field theory, including $1/N$
corrections to the leading large $N$ result \cite{Aharony:1999rz}
which arise from disc and crosscap diagrams.

By studying other backgrounds of D3-branes with 7-branes (with or
without orientifolds) one can obtain 
non-conformal theories which
exhibit a logarithmic running of the coupling constant
\cite{Aharony:1998xz,deMelloKoch:1999hn}. For instance, by separating the
D7-branes away from the orientifold plane, corresponding to giving a
mass to the hypermultiplets in the fundamental representation, one
finds string theory solutions in which the dilaton varies in a similar
way to the variation of the coupling constant in the field theory, and
this behavior persists also in the near-horizon limit (which is quite
complicated in this case, and becomes singular close to the branes,
corresponding to the low-energy limit of the field theories which is
in this case a free Abelian Coulomb phase). This agreement with the
perturbative expectation, even though we are (necessarily) in a regime
of large $\lambda = g_{YM}^2 N$, is due to special properties of
$\cn=2$ gauge theories, which prevent many quantities from being
renormalized beyond one-loop.


