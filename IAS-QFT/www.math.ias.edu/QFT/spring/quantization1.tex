%From: Roman Bezrukavnikov <roma@math.ias.edu>
%Date: Fri, 4 Apr 1997 15:36:55 -0500
%Subject: Kazhdan's lecture 1 of the 2nd term



%% This is an AMS-TeX file.
%% The command to compile it is: amstex <file>
%%
\input amstex
\documentstyle{amsppt}
\loadeusm
\magnification=1200
\pagewidth{6.5 true in}
\pageheight{8.9 true in}

\catcode`\@=11
\def\logo@{}
\catcode`\@=13

\NoRunningHeads

\font\boldtitlefont=cmb10 scaled\magstep1
\font\bigboldtitlefont=cmb10 scaled\magstep2

\def\dspace{\lineskip=2pt\baselineskip=18pt\lineskiplimit=0pt}

%\def\plus{{\sssize +}}
%\def\oplusop{\operatornamewithlimits{\oplus}\limits}
\def\otimesop{\operatornamewithlimits{\otimes}\limits}
%\def\Piop{\operatornamewithlimits{\Pi}\limits}
%\def\w{{\mathchoice{\,{\scriptstyle\wedge}\,}
%  {{\scriptstyle\wedge}}
%  {{\scriptscriptstyle\wedge}}{{\scriptscriptstyle\wedge}}}}
\def\Le{{\mathchoice{\,{\scriptstyle\le}\,}
  {\,{\scriptstyle\le}\,}
  {\,{\scriptscriptstyle\le}\,}{\,{\scriptscriptstyle\le}\,}}}
\def\Ge{{\mathchoice{\,{\scriptstyle\ge}\,}
  {\,{\scriptstyle\ge}\,}
  {\,{\scriptscriptstyle\ge}\,}{\,{\scriptscriptstyle\ge}\,}}}
%\def\vrulesub#1{\hbox{\,\vrule height7pt depth5pt\,}_{#1}}
%\def\rightsubsetarrow#1{\subset\kern-6.50pt\lower2.85pt
%     \hbox to #1pt{\rightarrowfill}}
%\def\mapright#1{\smash{\mathop{\,\longrightarrow\,}\limits^{#1}}}
%\def\arrowsim{\smash{\mathop{\to}\limits^{\lower1.5pt
%  \hbox{$\scriptstyle\sim$}}}}
\def\[[{[\![}
\def\]]{]\!]}

\def\eps{{\varepsilon}}
\def\kap{{\varkappa}}

\def\ch{{\check{\;}}}

\def\Atil{\tilde A}   \def\Btil{\tilde B}
\def\Avect{\vec{A}}
\def\Adot{\Dot{A}}
\def\Edot{\Dot{E}}
\def\gammatil{\tilde \gamma}

\def\supp{\text{\rm supp}}
\def\Maps{\text{\rm Maps}}
\def\Sol{\text{\rm Sol}}
\def\End{\text{\rm End}}
\def\Lie{\text{\rm Lie}}
\def\rank{\text{\rm rank}}

\def\dbC{{\Bbb C}} 
\def\dbR{{\Bbb R}}
%\def\dbZ{{\Bbb Z}} 

\def\gr#1{{\fam\eufmfam\relax#1}}

%Euler Fraktur letters (German)
\def\grA{{\gr A}}	\def\gra{{\gr a}}
\def\grB{{\gr B}}	\def\grb{{\gr b}}
\def\grC{{\gr C}}	\def\grc{{\gr c}}
\def\grD{{\gr D}}	\def\grd{{\gr d}}
\def\grE{{\gr E}}	\def\gre{{\gr e}}
\def\grF{{\gr F}}	\def\grf{{\gr f}}
\def\grG{{\gr G}}	\def\grg{{\gr g}}
\def\grH{{\gr H}}	\def\grh{{\gr h}}
\def\grI{{\gr I}}	\def\gri{{\gr i}}
\def\grJ{{\gr J}}	\def\grj{{\gr j}}
\def\grK{{\gr K}}	\def\grk{{\gr k}}
\def\grL{{\gr L}}	\def\grl{{\gr l}}
\def\grM{{\gr M}}	\def\grm{{\gr m}}
\def\grN{{\gr N}}	\def\grn{{\gr n}}
\def\grO{{\gr O}}	\def\gro{{\gr o}}
\def\grP{{\gr P}}	\def\grp{{\gr p}}
\def\grQ{{\gr Q}}	\def\grq{{\gr q}}
\def\grR{{\gr R}}	\def\grr{{\gr r}}
\def\grS{{\gr S}}	\def\grs{{\gr s}}
\def\grT{{\gr T}}	\def\grt{{\gr t}}
\def\grU{{\gr U}}	\def\gru{{\gr u}}
\def\grV{{\gr V}}	\def\grv{{\gr v}}
\def\grW{{\gr W}}	\def\grw{{\gr w}}
\def\grX{{\gr X}}	\def\grx{{\gr x}}
\def\grY{{\gr Y}}	\def\gry{{\gr y}}
\def\grZ{{\gr Z}}	\def\grz{{\gr z}}

\def\g{\grg}
\def\scr#1{{\fam\eusmfam\relax#1}}

\def\scrA{{\scr A}}   \def\scrB{{\scr B}}
\def\scrC{{\scr C}}   \def\scrD{{\scr D}}
\def\scrE{{\scr E}}   \def\scrF{{\scr F}}
\def\scrG{{\scr G}}   \def\scrH{{\scr H}}
\def\scrI{{\scr I}}   \def\scrJ{{\scr J}}
\def\scrK{{\scr K}}   \def\scrL{{\scr L}}
\def\scrM{{\scr M}}   \def\scrN{{\scr N}}
\def\scrO{{\scr O}}   \def\scrP{{\scr P}}
\def\scrQ{{\scr Q}}   \def\scrR{{\scr R}}
\def\scrS{{\scr S}}   \def\scrT{{\scr T}}
\def\scrU{{\scr U}}   \def\scrV{{\scr V}}
\def\scrW{{\scr W}}   \def\scrX{{\scr X}}
\def\scrY{{\scr Y}}   \def\scrZ{{\scr Z}}
\def\hatP{{\widehat{P}}}
\def\hatQ{{\widehat{Q}}}
\def\Qhat{{\widehat{Q}}}
\def\ten{\otimes}
\def\hatd{{\hat{d}}}

\def\scrTbar{{\overline{\scr T}}}
\def\scrGbar{{\overline{\scr G}}}
\def\Vbar{{\overline{V}}}

\def\iso{\widetilde{\to}}

%???????????
\def\sf{\bf}
\def\ksi{\xi}

\NoBlackBoxes
\document


\centerline{\boldtitlefont Lectures on quantization of gauge theories.}

\medskip
\centerline{\bigboldtitlefont Lecture 1.}

\bigskip
\centerline{David Kazhdan}

In gauge theory the space of field is the space of connections (with
additional data) considered modulo gauge transformation. Geometry of this
space is extremely nontrivial. The idea of BRST approach to construction of
quantum gauge theory is rougly as follows.
One first  quantizes a larger but simpler linear space (this linear space
contains the  space of all connections),
and then takes invariants of the gauge group in the resulting Hilbert space.
In functional integral approach to quantization one also replaces 
the integral over the space of connections  modulo gauge transformation
by a certain integral over this huge linear space.
 Both procedures are not quite
straightforward: the first one leads to the notion of semiinfinite
cohomology; the second one involves equivariant cohomology of the 
gauge group. 

\medskip 
   
We start with a simple example which does not belong to the gauge theory
but is in certain aspects analogous to some gauge invariant problems.

{\bf 1.Theory of free vector boson.} A field is a map $A:V\to V$.
We can and will consider $A$ as a 1-form on $V$. The Lagrangian is then 
given by
the formula: $$\scrL(A)=\frac12 dA \wedge *dA - \frac{m^2}{2} A\wedge *A$$
The second term is not invariant under the gauge transformations $A\to
A+df$
thus it is not a gauge theory if $m\not = 0$.

Let us write $A$ in components, $A= \sum\limits _{i=0}^{d-1}
 A_i dx_i$. 
 The classical equations of motion are:
$$\square A_i + \frac{\partial}{\partial x_i} *d*A=m^2 A_i$$

As usual it is convenient to describe the space of solutions in Fourier
coordinates. Let $\tilde A$ be the Fourier transform of $A$. Then the
equations of motion read as:
$$(p^2+m^2)\Atil(p)=(\Atil(p),p)p$$

For $m\not = 0$ this is equivalent to saying that $\supp(\Atil)\subset \scrO_m
=\{p\,|\,p^2+m^2=0\}$ and $(\Atil(p),p)=0$.
 If $m=0$ then the space of solutions is
invariant under the gauge tranformations $\Atil \to \Atil +f(p)p$, and the
space of solutions is generated by $\Atil$ such that $\supp \Atil \subset 
\scrO_0$, $(\Atil(p),p)=0$  under the gauge transformations.
Thus solutions modulo gauge are identified with the space of sections
of a bundle on $\scrO _0$ whose fiber is $p^\perp/\dbR p$ (see the 
lecture 3 from the fall semester). 

The closed 2-form on the space of solutions is given by 
$$\langle A,B \rangle = \int \limits _{\scrO_m^+} (\Atil(p)\Btil(-p)-
\Atil(-p)\Btil(p))d\mu$$ where $\mu$ is the invariant measure on $\scrO
_m^+$. This form is non-degenerate if $m\not = 0$; it is pull-back of a
non-degenerate 2-form on the space of solutions modulo gauge
transformations if $m=0$. 


We next  recall the decomposition $V= \dbR\times\dbR^{d-1}$, and
view $\Maps(V,V)$ as the space of paths on the manifold
$\Maps(\dbR^{d-1},V)$. 

The value of $\scrL(A)$ at a point $x\in \{t\}\times \dbR^{d-1}$ depends
only on $A_{space}=A|_{\{t\}\times \dbR^{d-1}}$ and $\Adot_{space}=
\Adot|_{\{t\}\times \dbR^{d-1}}$
(where $\Adot=\frac{\partial A}{\partial x_0}$). 

Let $\scrL_{space}$ be a functional on the tangent  bundle
$T(\Maps(\dbR^{d-1},V))$ with values in functions on  $\dbR ^{d-1}$
such that $\scrL(A)|_{\{t\}\times \dbR^{d-1}}=
\scrL_{space}(A _{space},\Adot_{space})$.


We put $L=\int \limits _{\dbR^{d-1}}\scrL_{space}$.

Then $L$ is quadratic-linear
 in fibers of the tangent bundle,
but the corresponding quadratic form is degenerate: the Lagrangian does not
depend on $\Adot_0$.

\medskip

To exploit the analogy with classical mechanics further let us 
briefly recall the corresponding   classical formalism.

\medskip


{\sf 1.1. Digression in classical mechanics.}
 Assume that $M$ is a manifold, and $\scrT$ is a quotient of the tangent
bundle to $M$; let $p:TM\to \scrT$ stand for the projection.
Let $L$ be a function on $\scrT$ which is assumed to be nondegenerate
quadratic-linear in fibers.    We consider  the Euler-Lagrange 
equation with Lagrangian $p^*(L)$.

As usual Lagrangian defines a map $P:TM\to T^*M$. The only difference with
the standard case is that $P$ is no longer an isomorphism; its image is the
subbundle $\scrT{\check{\;}} =Ker(p)^\perp\subset T^*M$.

One can define the Hamiltonian function $H$ on  $\scrT{\check{\;}}$
as the Legendre transform of the function $L$ on $\scrT$.

For a path $\gamma :\dbR \to M$ let $T\gamma :\dbR \to TM$ be its canonical
lifting to the tangent bundle.

 A map $\gammatil :\dbR \to \scrT\ch$ will be called a classical trajectory
if $\gammatil = P\circ T\gamma $ where $\gamma$ is a solution of the
Euler-Lagrange equations. 

Note that the symplectic form restricted to
$\scrT\ch$ is degenerate; its kernel is an integrable distribution
of dimension $\dim(M)-\rank(\scrT)$ which we denote by  $\scrK$.
 Let us define the submanifold  $X=\{x\in
\scrT\ch\, :\, d_x H|_{\scrK _x}=0\}$. 

We have a correctly defined section of
$T( \scrT\ch)/\scrK|_X$ which we denote by $\ksi_H$. (It is characterized
by the standard formula $\omega (\ksi _H,v)=\left<dH,v \right>$
for all $v\in T( \scrT\ch)$.)

Now it is not hard to see that a path  $ \gammatil: \dbR \to  \scrT\ch$
is a classical trajectory iff $\gammatil (t)\in X$ for all $t$ and
$\partial \gammatil / \partial t= \ksi_H \; \mod \, \scrK$.

Assume for simplicity that $X$ is smooth and $\dim (TX_x\cap \scrK_x)$
is constant. One can check  that $\ksi_H$ lies actually in the
subbundle $TX/(TX\cap \scrK) \subset T( \scrT\ch)/\scrK|_X$.

 To avoid irrelevant  notational complications 
let us  suppose also that the foliation is split, i.e.
 we have a decomposition $X=Y\times Z$ such that 
$\{y\}\times Z$ for $y\in Y$ is a leaf of the foliation. The section
 $\ksi _H$ defines  a vector
field on $Y$, and a general classical trajectory has the form $\gammatil = 
\gammatil _1 \times \gammatil _2$ where $\gammatil _1$ is a trajectory of
the vector field (defined by ) $\ksi _H$  and $\gammatil _2:\dbR \to Z$ is
arbitrary.

Thus the space of solutions of the Euler-Lagrange equation is identified
with $Y\times \Maps (\dbR, Z)$.

\medskip

{\sf 1.2.} Let us return to our situation.
So  $M=\Maps(\dbR^{d-1},V)$; the cotangent bundle $T^*(M)$
is identified with $M\times \Gamma(\Omega^1(V)|_{\dbR^{d-1}})$.

The kernel of $p$ is generated by $\Adot_0 \in T(\Maps(\dbR^{d-1},V))$;
thus $\scrT\ch \cong M\times \Omega^1 (\dbR^{d-1})$.

 The Hamiltonian $h$ is a functional on $T^*(M)$ with values in functions on
$\dbR^{d-1}$. It  is defined by
 $$h(A_{space},E)= E_k\Adot_k -\scrL_{space}(A_{space},\Adot_{space})$$
where $E_k=\frac{\partial \scrL}{\partial \Adot_k}=
\Adot_k-\frac{\partial A_0}{\partial x_k}$.

Then we put $H =\int\limits _{\dbR^{d-1}}h$.

We have 
$$h(A_{space},E)= \frac12(E_k \cdot (\Adot_k+\frac{\partial A_0}{\partial x_k})
-(d\Avect )^2 - m^2 (A)^2)= \frac12 ((E)^2-(d\Avect )^2-m^2
(\Avect)^2-A_0^2) +E\wedge *d A_0$$   


where $E=\sum\limits_{k=1}^{d-1}E_kdx_k$ and  $\Avect
=\sum\limits_{k=1}^{d-1}A_kdx_k$ are viewed as 1-forms on $\dbR^{d-1}$.



The
distribution $\scrK$ is generated by $A_0\in TM \subset T( M\times \Omega^1
(\dbR^{d-1}))$. Thus $X$ is given by the equation $\frac{\partial H}
{\partial A_0}=0$, i.e. $\frac{\delta h}
{\delta A_0}=0$ where $\delta$ is the variational derivative.
Integrating by parts the last summand in the formula for $h$ we  get:

$$-\frac{\delta h}{\delta A_0}=d*E+m^2A_0$$

So $X$ is a 
subvariety in $\Maps(\dbR^{d-1},V)\times \Omega^1(\dbR^{d-1})$ defined by
the equation
 $$ d*E +m^2A_0=0$$

If $m\not=0$ then $X$ is symplectic; the space of classical solutions
is identified with $X$. Since $A_0$ is determined from the equation
we see that  $X \cong T^*(\Omega^1(\dbR^{d-1}))$ in this case.
 

If $m=0$ then $X$ is no longer symplectic; we get a constraint on 
$(\Avect , E)$ but $A_0$  can be arbitrary. In terms of 1.1
we have $X=Y\times Z$
where   $Y\subset  T^*(\Omega^1(\dbR^{d-1}))$ is defined by $ d*E=0$,
and $Z=\Omega^0(\dbR^{d-1})$. The space of classical solutions is 
$Y\times \Maps (\dbR, Z)$.


We finish the discussion of the free vector QFT with the following remark.

Suppose we want to add a ``small'' term to the above Lagrangian and study  
the resulting theory perturbatively. The corresponding propagator in
momentum space is an $\End(V)$ valued function on $V$ given by 
$G(p)=(D_0(p)+(p^2+m^2) Id)^{-1}$ where $D_0(p)v=-(p,v)p$. We see that 
$G(p)$ does not tend to 0 when $p\to \infty$ (for example $G(p)p=m^{-2}p$).
 Thus the methods of
perturbation theory do not work: all Feynman integrals diverge, and we have
no way to renormalize them. This is an indication of a well-known fact
that a non-free non-gauge vector QFT does not exist.


 
This also illustrates one of the difficulties in quantizing gauge theory.
Let us consider  a $U(1)$-gauge theory
as an example. Thus fields are pairs 
$(A,\phi)$ where $A$ is a connection on the (trivial) $U(1)$-bundle,
 and $\phi$ is a
section of the corresponding complex line bundle. The Lagrangian is:
$$\scrL(A,\phi)=F_A\wedge *F_A+D_A\phi \wedge *D_A \phi +W(\phi)$$
where $W(\phi)=(|\phi|^2-a^2)^2$. For $L=\int \scrL$ to make sense we need
the condition $|\phi|^2 \to a^2$ for $v \to \infty$. 


 Let us choose a gauge (i.e. a
trivialization of the $U(1)$-bundle) in such a way that $\phi$ is real and 
$\phi \to a$ on infinity. We can write $\phi=a+\sigma$ where $\sigma$ is a
real-valued rapidly decreasing function on $V$. We have:
$$\scrL(A,\sigma)=dA\wedge *dA+a^2A\wedge *A +d\sigma\wedge *d\sigma +4a^2
\sigma\cdot *\sigma +\cdots$$
where $\cdots$ stand for terms of order $\geq 3$ in $A$ and $\sigma$.

Thus the sum of quadratic terms is  Lagrangian for the sum of 2
noninteracting free massive vector bosons; a naive approach to the quantization
problem 
 would be to study gauge theory as a perturbation of that free QFT. As was
explained this method does not have a chance to work.    

{\bf 2. Pure gauge theory.} Let $G$ be a compact Lie group, 
$\scrT \to V$ be a principle $G$-bundle on $V$. Let $\g_\scrT$
be the adjoint vector bundle of $\scrT$. 
We fix a positive
invariant scalar product on the Lie algebra of $G$, and will not
distinguish between the Lie algebra and its dual. 
 
 The Lagrangian of the theory  is $\scrL(A)=\frac12 F_A\wedge *F_A$.

The classical field equation is then $d_A*F_A=0$. 

Let us rewrite the theory in Hamiltonian form.
 Thus we fix a decomposition $V=\dbR \times
\Vbar$. Let $\scrT_{space}(t)$ be the restriction of $\scrT$ to 
$\{t\}\times \Vbar$, and $A_{space}(t)=A|_{\{t\}\times \Vbar}$
be the connection on  $\scrT_{space}(t)$. Introduce notations:
$B=B(t)=F_A|_{\{t\}\times \Vbar}=F_{A_{space}}\in \Omega^2(\{t\}\times
\Vbar, \g_\scrTbar) $ (the magnetic field), and
$E=i_{\partial_t}(F_A)|_{\{t\}\times \Vbar} \in \Omega ^1(\{t\}\times
\Vbar, \g_\scrTbar ) $ (the electric field). 

Let $\kap$ be the projection $\kap:V\to \Vbar$. Let us fix an isomorphism
$\scrT\cong \kap ^*(\scrTbar)$ for a $G$-bundle $\scrTbar$ on $\Vbar$.
 Then $\scrT|_{\dbR\times \{x\}}$ is trivialized for all $x\in \Vbar$.
So we can write any connection on $\scrT$ as $A=A_{space}(t)+A_0dx_0$ where
$A_0 \in \Omega^0(\g _\scrT)$;
thus a connection on $\scrT$ is the same as a path in $M$, where $M$ is
the space of pairs:  a connection on $\scrTbar$, 
and a section of the adjoint bundle of $\scrTbar$.

We have: $$E(t)=d_{A_{space}(t)}A_0 +\Adot_{space}(t)$$
$$\scrL(A)=\frac12 (F_A)^2=\frac12 ((B)^2+(E)^2)$$

Exactly as above we see that Lagrangian is nondegenerate in $\Adot_{space}$,
but does not depend on $\Adot_0$. Thus we can define the Hamiltonian
functional $h$ on $M \times \Omega^1(\Vbar, \g_\scrT )$ by:
$$\scrL (A)=E\wedge *\Adot_{space} -h(A_{space},E)$$
(Notice that $E=\frac{\partial \scrL}{\partial \Adot_{space}}$).
We have:
$$ h(A_{space},E)=\frac12 (E^2-B^2)-E\wedge*d_{A_{space}}A_0
$$


In terms of 1.1 we have: $X$ is the subspace 
in  $M \times \Omega^1(\Vbar, \g_\scrT )$ given by
$$ \frac{\delta h}{\delta A_0}=0$$
where $\delta$ is the variational derivative. In the formula for $h$ we can
integrate by parts to get:
 $$\frac{\delta h}{\delta A_0}=d_{A_{space}}(*E)$$
So we see that $Y\subset T^*(Conn)$ where $Conn$ stands for the space of
connections on $\scrTbar$, and $Y$  is defined by 
$$d_{A_{space}}(*E)=0$$
we also have  $Z=\Omega^1(\g_\scrTbar)$. So the space of classical solutions is
$Y\times \Maps (\dbR, Z)$. 

Now we notice that any connection is gauge-eqiuvalent to a one with
$A_0=0$, i.e. a one which is constant on every line $\dbR\times\{x\}$
in the above trivialization. 
(Such a  connection  is called a connection with temporal
gauge).



 A connection with a temporal gauge is the same as a map 
from $\dbR$ to $Conn$.

Let us  restrict  our Lagrangian to the space
of connections with temporal
gauge, which is identified with $\Maps (\dbR,  Conn)$.
 Then we obtain a functional on the tangent bundle $T(Conn)$ which is readily
seen to be  quadratic nondegenerate in  fibers. Hence the space $Y$ of 
critical points of Lagrangian on the space of
temporaly gauged connections is the (co)tangent bundle
 to $Conn$, which is identified with the space of pairs $(A_{space},E)$ where
$A_{space}$ is a connection on $\scrTbar$ and $E\in \Omega ^1(\scrTbar)$.

 Obviously the space $X$ of temporaly gauged
connections which are classical solutions of the original problem
(i.e. critical points of $\scrL$ as a function on the space of all
connections) is a subspace of $Y$ (we have to impose the condition that
variation in $A_0$ vanishes).  Moreover we have shown that
 $X$ is a subspace of $Y$ defined by the equation $d_A*E=0$.


\remark{Remark} A naive analogy with the finite dimensional case
may be misleading, as is illustrated by the following example.

 Let $U$ be a finite dimensional manifold with an action of a Lie group
$G$; let $Z\subset U$ be a submanifold such that $G\cdot Z=U$. Assume that
$L$ is a $G$-invariant function on $U$. Then a point $y\in Z$ is a critical
point of $L|_Z$ iff it is a critical point of $L$ as a function on $U$. 

However if we take $U$ to be the space of  connections,
take $Z$ to be the space of connections with temporal gauge, and
$L=\int\limits_V \scrL$ to be the action then this statement is no longer
true, as we saw few lines above. 

This has the following reason. 

A classical field (a connection in our case) 
is a solution of the classical equations of motion iff  it is a
stationary point of the action with respect to  variations  with
{\it compact support}.  

Suppose that $A$ is a connection with temporal gauge which is stationary
with respect to such variations in the class of connections with temporal
gauge. Consider a compactly supported variation in $A_0$-direction. We can
apply to it an infinitesimal gauge transformation so that the resulting
variation will stay in the class of connections with temporal gauge.
However we can not guarantee that this new variation has compact support,
so the action can actually vary infinitesimaly. 

\endremark


\proclaim{Proposition} Let $Y=\{(A_{space}, E)\}\cong T^*(Conn)$
and $X\subset Y$ be as above.  Then the map $m:Y\to
\Omega^{d-1}(\g_\scrTbar)$ given by $(A_{space},E)\to d_{A_{space}}*E$
is the moment map  for the action of the gauge group $\scrGbar= Aut (\scrTbar)$
on $Y$. Thus $X$ is the 0-set of the moment map, and the space of extremals
of $\scrL$ modulo gauge transformations is  symplectic reduction of $Y$
with respect to the action of the gauge group.

\endproclaim

\demo{Proof} It remains only to check that $m$ is the moment map. 

Let us take $x\in \Omega^0(\g _\scrTbar)=\Lie (\scrGbar)$ and a constant
vector field $w=(A',E')$ on $Y$ for $A',E' \in  \Omega^1(\g _\scrTbar)$.
Let $j$ denote the action of  $\Lie (\scrGbar)$ on $Y$.
Then we have to check the equality:
$$d \left< x, d_A*E \right>(w)= \omega (j(x),w)$$
where first  $d$ in the left-hand side is  external derivative of a function
 on $Y$, and $\omega$ is the standard symplectic form on $Y$. 

Now $j$  is given by $j(x)|_{(A,E)} = (d_A x, [x,E])$. 
So the equality reads as:
$$ \left< x, d_A *E' +[A', *E]\right>=\left<d_A x,* E'\right> +\left< [x,E],
* A' \right>$$
which is true by integration by parts. 

\end


