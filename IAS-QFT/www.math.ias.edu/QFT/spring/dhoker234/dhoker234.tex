%% This is a plain TeX file
%%
\magnification=1200
\hsize=6.5 true in
\vsize=8.7 true in
\input epsf.tex

\font\dotless=cmr10 %for the roman i or j to be
                    %used with accents on top.
                    %(\dotless\char'020=i)
                    %(\dotless\char'021=j)
\font\itdotless=cmti10
\def\itumi{{\"{\itdotless\char'020}}}
\def\itumj{{\"{\itdotless\char'021}}}
\def\umi{{\"{\dotless\char'020}}}
\def\umj{{\"{\dotless\char'021}}}
\font\smaller=cmr5
\font\boldtitlefont=cmb10 scaled\magstep2
\font\smallboldtitle=cmb10 scaled \magstep1
\font\ninerm=cmr9
\font\sans=cmss10 scaled\magstep1

\footline={\hfil {\tenrm II.\folio}\hfil}

\def\eps{{\varepsilon}}
\def\Eps{{\epsilon}}
\def\kap{{\kappa}}
\def\lam{{\lambda}}
\def\Lam{{\Lambda}}

\def\undertext#1{$\underline{\vphantom{y}\hbox{#1}}$}
\def\nspace{\lineskip=1pt\baselineskip=12pt%
     \lineskiplimit=0pt}
\def\dspace{\lineskip=2pt\baselineskip=18pt%
     \lineskiplimit=0pt}

\def\w{{\mathchoice{\,{\scriptstyle\wedge}\,}
  {{\scriptstyle\wedge}}
  {{\scriptscriptstyle\wedge}}{{\scriptscriptstyle\wedge}}}}
\def\Le{{\mathchoice{\,{\scriptstyle\le}\,}
{\,{\scriptstyle\le}\,}
{\,{\scriptscriptstyle\le}\,}{\,{\scriptscriptstyle\le}\,}}}
\def\Ge{{\mathchoice{\,{\scriptstyle\ge}\,}
{\,{\scriptstyle\ge}\,}
{\,{\scriptscriptstyle\ge}\,}{\,{\scriptscriptstyle\ge}\,}}}
\def\plus{{\hbox{$\scriptscriptstyle +$}}}
\def\xdot{\dot{x}}
\def\Item#1{\par%
     \smallskip\hang\indent\llap{\hbox to\parindent
     {#1\hfill\enspace}}\ignorespaces}
\def\Condition#1{\item{#1}}
\def\Firstcondition#1{\hangindent\parindent{#1}\enspace
     \ignorespaces}
\def\Proclaim#1{\medbreak
  \medskip\noindent{\bf#1\enspace}\it\ignorespaces}
  %the way to use this is:
  %"\Proclaim{Theorem 1.1.}" for instance.
\def\finishproclaim{\par\rm
     \ifdim\lastskip<\medskipamount\removelastskip
     \penalty55\medskip\fi}

\def\im{{\rm Im}}      \def\Open{{\rm open}}
\def\Diff{{\rm Diff}}  \def\Closed{{\rm closed}}
\def\Map{{\rm Diff}}   \def\spurious{{\rm spurious}}
\def\Met{{\rm Met}}    \def\phys{{\rm phys}}
\def\diag{{\rm diag}}  \def\Vir{{\rm Vir}}
\def\spin{{\rm spin}}  \def\Res{{\rm Res}}
\def\Null{{\rm null}}  \def\mass{{\rm mass}}
\def\SO{{\rm SO}}      \def\Tr{\hbox{\rm Tr}}
\def\SU{{\rm SU}}      \def\tr{{\rm tr}}
\def\Sp{{\rm Sp}}

\def\fbar{\bar{f}}  \def\mubar{\bar{\mu}}
\def\barh{\bar{h}}  \def\gammabar{\bar{\gamma}}
\def\kbar{\bar{k}}  \def\lambar{\bar{\lambda}}
\def\wbar{\bar{w}}  \def\Lambar{\bar{\Lambda}}
\def\zbar{\bar{z}}
\def\Abar{\bar{A}}
\def\Kbar{\bar{K}}
\def\Sbar{\bar{S}}
\def\Tbar{\bar{T}}

\def\htil{\tilde{h}}
\def\xtil{\tilde{x}}
\def\Ltil{\tilde{L}}
\def\Ntil{\widetilde{N}}
\def\Ttil{\widetilde{T}}
\def\scrFtil{\widetilde{\scrF}}
\def\epstil{\tilde{\eps}}

\def\dbR{{\Bbb R}}
\def\dbZ{{\Bbb Z}}

%These two files (in this order!!) are necessary
%in order to use AMS Fonts 2.0 with Plain TeX

\input amssym.def
\input amssym.tex

%Capital roman double letters(Blackboard bold)
\def\db#1{{\fam\msbfam\relax#1}}

\def\dbA{{\db A}} \def\dbB{{\db B}}
\def\dbC{{\db C}} \def\dbD{{\db D}}
\def\dbE{{\db E}} \def\dbF{{\db F}}
\def\dbG{{\db G}} \def\dbH{{\db H}}
\def\dbI{{\db I}} \def\dbJ{{\db J}}
\def\dbK{{\db K}} \def\dbL{{\db L}}
\def\dbM{{\db M}} \def\dbN{{\db N}}
\def\dbO{{\db O}} \def\dbP{{\db P}}
\def\dbQ{{\db Q}} \def\dbR{{\db R}}
\def\dbS{{\db S}} \def\dbT{{\db T}}
\def\dbU{{\db U}} \def\dbV{{\db V}}
\def\dbW{{\db W}} \def\dbX{{\db X}}
\def\dbY{{\db Y}} \def\dbZ{{\db Z}}

\font\teneusm=eusm10  \font\seveneusm=eusm7 
\font\fiveeusm=eusm5 
\newfam\eusmfam 
\textfont\eusmfam=\teneusm 
\scriptfont\eusmfam=\seveneusm 
\scriptscriptfont\eusmfam=\fiveeufm 
\def\scr#1{{\fam\eusmfam\relax#1}}


%Upper-case Script Letters:

\def\scrA{{\scr A}}   \def\scrB{{\scr B}}
\def\scrC{{\scr C}}   \def\scrD{{\scr D}}
\def\scrE{{\scr E}}   \def\scrF{{\scr F}}
\def\scrG{{\scr G}}   \def\scrH{{\scr H}}
\def\scrI{{\scr I}}   \def\scrJ{{\scr J}}
\def\scrK{{\scr K}}   \def\scrL{{\scr L}}
\def\scrM{{\scr M}}   \def\scrN{{\scr N}}
\def\scrO{{\scr O}}   \def\scrP{{\scr P}}
\def\scrQ{{\scr Q}}   \def\scrR{{\scr R}}
\def\scrS{{\scr S}}   \def\scrT{{\scr T}}
\def\scrU{{\scr U}}   \def\scrV{{\scr V}}
\def\scrW{{\scr W}}   \def\scrX{{\scr X}}
\def\scrY{{\scr Y}}   \def\scrZ{{\scr Z}}

\parindent=18pt
\line{\sans --- DRAFT ---\hfill{\rm IASSNS-HEP-97/72}}

\bigskip\bigskip
\centerline{\boldtitlefont Lectures 2, 3, 4}
\medskip
\centerline{\smallboldtitle II. Spectrum of Free Bosonic
Strings}

\medskip
\centerline{Eric D'Hoker}

\frenchspacing

\dspace
\bigskip

We begin by constructing the Hilbert space of physical
states and the spectrum of a single free bosonic
closed oriented string, whose worldsheet has the
topology of a cylinder (or equivalently an annulus)
and of a single free open oriented string, whose
worldsheet has the topology of half a cylinder (or
equivalently half an annulus).

A free string should not couple to any other strings,
or to any non-trivial background fields in the
space-time manifold $M$.
Thus we take $M=\dbR^D$, with constant $G_{\mu\nu}$
and $B_{\mu\nu}=\Psi=T=0$ to start.
We wish to examine the properties of physical strings
in physical space-time, so it is essential that the
space-time metric be Minkowskian, with
$G_{\mu\nu}=\eta_{\mu\nu}=\diag(- ++\cdots+)$ and
$\mu,\nu=0,1,\ldots,D-1$.
In quantum field theory, the problem of constructing
the Hilbert space of states and the spectrum of a
single particle amounts to finding the $1$-particle
unitary representations of the Poincar\'e group --- a
problem for which the Minkowski nature of space-time
is again essential.

A single string has an infinite number of excitation
states, each of which corresponds to a representation
of the Poincar\'e group.
In this lecture, we wish to identify which
representations of the Poincar\'e group occur (i.e.
find the spectrum) and determine the conditions under
which all these representations are unitary.
We will find that conformal invariance plays an
essential r\^{o}le.


Before we discuss the details of this fundamental
analysis of string theory, let us outline the main
steps of our strategy.

\medskip\noindent
1)\enspace
In the first lecture, we postulated a prescription for
the transition amplitudes of $N$ external string
states --- represented by vertex operators $V_i$,
$i=1,\ldots,N$ --- in a general Riemannian space-time
$M$ with metric $G$ (as well as possibly other fields
$B_{\mu\nu}$, $\Phi$, $T$)
$$
A=\sum\limits_{h=0}^\infty\int\nolimits_{\Met(\Sigma)}
Dg{1\over\scrN(g)}\int\nolimits_{\Map(\Sigma,M)}
Dx\,V_1\ldots V_N\,e^{-S[x,g]}
\eqno{(2.1)}
$$

\medskip\noindent
2)\enspace
To investigate the spectrum of the string in Minkowski
$\dbR_M^D$ we shall start with the above amplitudes in
Euclidean $\dbR_E^D$ with
$G_{\mu\nu}=\delta_{\mu\nu}$, $B_{\mu\nu}=\Phi=T=0$
and action
$$
S[x,g]={1\over 8\pi}\int\nolimits_{\Sigma}
d\mu_g g^{mn}\partial_m x\cdot\partial_n x\,\,.
\eqno{(2.2)}
$$
(For flat Euclidean or Minkowskian metric, we have
$x\in\dbR_E^D$ or $\dbR_M^D$ respectively
 and we use the notation
$(v,v')=v\cdot v'$.)
The worldsheet metric $g$ is always kept Riemannian.
The string tension $\kap$ introduced in \S{I} has been
set to $\kap=1/8\pi$.
In the physics literature one often finds the string
tension expressed in terms of $\alpha'$ (the Regge
slope parameter) by $\kap={1\over 4\pi \alpha'}$; so in
our conventions $\alpha'=2$.

\medskip\noindent
3)\enspace
At fixed $g$, the quantum field theory defined by
$S[x,g]$ provides an example of a {\it conformal
field theory} (see also the lectures by Gawedzki).
All we shall need of the conformal field theory to
solve for the free string spectrum is contained in the
operator product expansions.

\medskip\noindent
4)\enspace
Once we have derived the OPE's for  Euclidean
$\dbR_E^D$, we shall continue the time component of
the map $x$, let $x^0\to ix^0$ and obtain
well-defined OPE's for the operators in Minkowskian
$\dbR_M^D$.
In particular, we obtain two Virasoro algebras
$\Vir_L\otimes\Vir_R$ for the closed string, and a
single Virasoro algebra $\Vir_L$ for the open string.

\medskip\noindent
5)\enspace
{}From the Minkowskian OPE's and associated algebra, we
construct Fock spaces $\scrF_{\Open}$ and
$\scrF_{\Closed}$ for open and closed strings
respectively along the same lines
that Gawedzki constructed a Hilbert space.
Due to the Minkowski signature of the space-time
metric, however,
the Fock space we obtain does {\it not} carry
a positive definite inner product.
$\Vir_L\otimes\Vir_R$ acts on $\scrF_{\Closed}$ 
and $\Vir_L$ acts on $\scrF_{\Open}$.

\medskip\noindent
6)\enspace
The remaining integration over $g$ --- which is {\it
not} present in conformal field theory, but
characteristic to string theory --- effectively
selects out of $\scrF$ (either $\scrF_{\Closed}$ or
$\scrF_{\Open}$) a subspace $\scrF^{\plus}$ on
which we have a positive definite inner product, under
certain conditions which we shall determine.
In particular, this is how the critical dimension
$D=26$ arises.

(The occurrence of negative norm states in a standard
Fock space construction is not unique to string
theory.
It will take place for gauge theories and for gravity
as well, or more generally in any theory with
particles of $\spin\Ge 1$.)

\bigskip\noindent
A) \ {\bf Basics of Conformal Field Theory}

\smallskip\noindent
$\bullet$ \ An oriented surface $\Sigma$, with a
(fixed) Riemannian
metric $g$ automatically carries a complex
structure, and we choose local complex coordinates $z$
and $\zbar$ in which $g$ is conformally flat, and
given by
$$
g=g_{z\zbar}(dz\otimes d\zbar+d\zbar\otimes dz)=
2g_{z\zbar}\vert dz\vert^2
\eqno{(2.3)}
$$
This metric is left invariant under analytic or {\it
conformal} diffeomorphisms $z\to z'=f(z)$,
with $f$ complex analytic, combined with Weyl
transformations on $g$.
A quantum theory invariant under this transformation
is a conformal field theory (abbreviated CFT).

\medskip\noindent
$\bullet$ \ Fundamental operators in CFT 
are conformal tensors or
{\it primary fields} $\phi$ of conformal weight $(h,\barh)$
satisfying the transformation law
$$
\phi=\phi(z,\zbar)(dz)^h(d\zbar)^{\barh}=
\phi'(z',\zbar')(dz')^h(d\zbar)^{\barh}\,\,.
\eqno{(2.4)}
$$
For integer weights $h$ and $\barh$, these are
operator-valued sections of $K^h\otimes\Kbar^{\barh}$,
where $K$ is the holomorphic cotangent (or canonical)
bundle of $\Sigma$.
In general, $h$ and $\barh$ don't have to be integers, but
their difference is the spin of the field $\phi$ and must
satisfy $h-\barh=\spin\in{1\over 2}\dbZ$.

\medskip\noindent
$\bullet$ \ In the conformal 
field theories we shall deal with
in these lectures, the matrix elements of products of
primary fields (and other fields) may be expressed
in terms of functional integrals:
$$
\left<\phi_1(\xi_1)\ldots\phi_N(\xi_N)\right>_g=
\int Dx\,\,\phi_1(\xi_1)\ldots\phi_N(\xi_N)
e^{-S[x,g]}
\eqno{(2.5)}
$$
where it is understood that $\phi_i$ can be expressed
in terms of the canonical fields $x$.
Notice that here we do not normalize correlation functions
 by dividing out by the partition function
$\left<1\right>_g$.

\medskip\noindent
$\bullet$ \ The stress tensor is defined as the variation
of the action $S$ with respect to the inverse metric
$g^{-1}=g^{mn}\partial_m\otimes\partial_n$, 
where the metric $g$ is viewed as a background source field.
$$
\delta_g S=-{1\over 4\pi}\int\nolimits_{\Sigma}
d\mu_g T_{mn}\delta g^{mn}
\eqno{(2.6)}
$$
Assuming that the variation $g$ has support away from
the operator positions $\xi_1,\ldots,\xi_N$, the
variation of correlation functions is given by
$$
\delta_g\left<\phi_1(\xi_1)\ldots\phi_N(\xi_N)\right>=
{1\over 4\pi}\int\nolimits_{\Sigma}d\mu_g\delta g^{mn}
\left<T_{mn}\phi_1(\xi_1)\ldots\phi_N(\xi_N)\right>
\eqno{(2.7)}
$$
The latter equation may be used as a definition of
$T_{mn}$, even in conformal field theories where no
Lagrangian formulation is available.

\medskip\noindent
$\bullet$ \ Applying Noether's theorem to
conformal symmetry yields the following
{\it conserved currents}:
$$
j_z=v^zT_{zz}\qquad\qquad
j_{\zbar}=v^{\zbar}T_{\zbar\zbar}\eqno{(2.8)}
$$
where $v^z$ (resp.  $v^{\zbar}$) is any (locally)
holomorphic (resp. anti-holomorphic) vector field.
Conformal symmetry requires $\partial_{\zbar}j_z=0$,
$\partial_zj_{\zbar}=0$, so that we must also have
$$
\partial_{\zbar}T_{zz}=0\qquad\qquad
\partial_z T_{\zbar\zbar}=0
\eqno{(2.9)}
$$
Thus, $T(z)=T_{zz}$ and $\Ttil(\zbar)=T_{\zbar\zbar}$
are complex analytic and anti-analytic respectively.
In correlation functions, however, a product of $T(z)$ and
another field $\phi(w)$ 
may produce singularities, as $z\to w$, so the above
analyticity condition is to be understood as valid
when $T$ is away from other operators.

\medskip\noindent
$\bullet$ \ It was shown in  Gawedzki's lectures how
to extract these OPE singularities from conformal Ward
identities for locally flat metrics $g$; for the case
of general metric $g$, the derivation is the subject
of Problem Set \#2.
One finds
$$
\eqalign{
T(z)\phi(w,\wbar) &\sim\left({h\over (z-w)^2}+{1\over
z-w} {\partial\over \partial w}\right)\phi(w,\wbar)\cr
T(z)T(w) &\sim{c/2\over (z-w)^4}+{2\over(z-w)^2}
T_{ww}+{1\over z-w}\partial_w T_{ww}\cr}
\eqno{(2.10)}
$$
where $c$ is the {\it central charge} of the CFT and
$h$ is the holomorphic conformal weight of $\phi$.
(Analogous relations hold for $\Ttil$.)
As always, these relations are to be understood as
valid when inserted into any correlation function.
The $\sim$ sign used in 
(2.10) stands for the fact that we have
retained only the singular part on the r.h.s.
When $c=0$, $T(z)$ is a conformal tensor of weight
$(2,0)$, but for $c\not=0$, $T$ transforms as a
projective connection instead, with
$$
\eqalignno{T'(z')(dz')^2 &=T(z)(dz)^2-{c\over 12}
  \{z';z\}(dz)^2 &(2.11)\cr
\noalign{\hbox{where the Schwarzian derivative
$\{~;~\}$ is
defined by}}
\{f;z\} &\equiv f'''/f'-(3/2)(f''/f')^2,\qquad
  f'=\partial f/\partial z.\cr}
$$

\medskip\noindent
$\bullet$ \ Analytic currents $j_\alpha=j_{\alpha z}dz$, 
$\alpha$ labeling different species of
currents, give rise to time-independent or conserved
charges:
$$
Q_\alpha(C)={1\over 2\pi i}\oint\nolimits_{C}
j_\alpha\qquad\qquad\qquad\lower35pt
\hbox{\vbox{\epsfxsize=1.0in\epsfbox{fig1.eps}}}
$$
Here $Q_\alpha(C)$ is the charge enclosed in the
region bounded by $C$ and equals $Q_\alpha(C')$
provided no sources for $Q_\alpha$ occur between $C$
and $C'$.
The action of charges on fields $\phi$ may be
evaluated using only the residues in the OPE of
$j_\alpha$ with $\phi$:
$$
Q_\alpha(C)\phi(w,\wbar)={1\over 2\pi
i}\oint\nolimits_{C}
j_\alpha(z)\phi(w,\wbar)=\mathop{\Res}\limits_{z\to w}
j_\alpha(z)\phi(w,\wbar)\eqno{(2.12)}
$$
provided no other sources for $j_\alpha$ lie within
$C$.
The commutator between two charges is recovered as
follows:
$$
\eqalign{
Q_\alpha(&C_1)Q_\beta(C_2)-Q_\alpha(C_3)Q_\beta(C_2)\cr
&=[Q_\alpha,Q_\beta](C_2)\cr
&={1\over 2\pi i}\oint\nolimits_{C_2}
\mathop{\Res}\limits_{z\to z_2}j_\alpha(z)
j_\beta(z_2)\cr}
\qquad\qquad\lower30pt\hbox{\vbox{\epsfysize=1.0in\epsfbox{fig2.eps}}}
\eqno{(2.13)}
$$

\medskip\noindent
$\bullet$ \ The conserved charges associated with
conformal transformations are obtained from vector fields
$v_{(m)}^z=z^{m+1}$ and $v_{(m)}^{\zbar}=\zbar^{m+1}$:
$$
\eqalign{
L_m &\equiv \oint {dz\over 2\pi i}z^{m+1}T(z)\cr
\Ltil_m &\equiv\oint{d\zbar\over 2\pi i}\zbar^{m+1}
  \Ttil(\zbar).\cr}
\eqno{(2.14)}
$$
{}From the OPE of two $T$'s (and two $\Ttil$'s),
we recover two Virasoro
algebras
$$
\eqalign{
[L_m,L_n] &=(m-n)L_{m+n}+{c\over
12}m(m^2-1)\delta_{m+n,0}\cr
[\Ltil_m,\Ltil_n] &=(m-n)\Ltil_{m+n}+{c\over 12}
  m(m^2-1)\delta_{m+n,0}.\cr}
\eqno{(2.15)}
$$
{}From the OPE of $T$ (and $\Ttil$)
with a primary field $\phi$ of conformal weight
$(h,\barh)$, we obtain
$$
\eqalign{
[L_m,\phi(w,\wbar)] &=w^{m+1}\partial_w \phi(w,\wbar)+
 h(m+1)w^m\phi(w,\wbar)\cr
[\Ltil_m,\phi(w,\wbar)] &=\wbar^{m+1}\partial_{\wbar}
  \phi(w,\wbar)+\barh(m+1)\wbar^m\phi(w,\wbar)\cr}
\eqno{(2.16)}
$$

\medskip\noindent
$\bullet$ \ An important example of a CFT is the free
scalar field $x\colon\,\Sigma\to\dbR$ (or
$\dbR/\lam\dbZ$ as discussed in Witten's lectures), 
with action
$$
S[x]={1\over 4\pi}\int\nolimits_{\Sigma} d^2z\,\partial_z
x\partial_{\zbar}x\eqno{(2.17)}
$$
The OPE is derived from the Green function for $x$:
$$
x(z,\zbar)x(w,\wbar)\sim -\ln\,\vert z-w\vert^2\eqno{(2.18)}
$$
It is customary to define normal ordering by
$$
:\,x(z,\zbar)x(w,\wbar):\equiv 
x(z,\zbar)x(w,\wbar)+\ln\,\vert z-w\vert^2
\eqno{(2.19)}
$$
The normal ordered product admits a smooth limit as
$w\to z$.
The renormalized (also called the quantum) 
stress tensor may now
be defined by
$$
T(z)=-{1\over 2}:\,(\partial_z x)^2:\equiv -{1\over 2}
\lim\limits_{w\to z}\left(\partial_z x\partial_w
x+{1\over (z-w)^2}\right)\,\,.\eqno{(2.20)}
$$
Elementary primary fields in this CFT are listed
below, with their corresponding conformal weight.
$$
\eqalign{
&\partial_z x\cr
&\partial_{\zbar}x\cr
:\,&\exp\,ikx:\cr}\qquad\qquad
\eqalign{
&(1,0)\cr
&(0,1)\cr
&\left({1\over 2}\,k^2,{1\over 2}\,k^2\right)\cr}
\eqno{(2.21)}
$$
The central charge is $c=1$.
Notice that while the derivative of $x$ and the
exponential are primary fields, $x$ itself is not a
well-defined quantum field
(cf. Witten's lecture).

\vfill\eject

\noindent
B) \ {\bf The free closed bosonic string conformal field
theory}

\smallskip
The worldsheet topology for the free closed bosonic
string is that of the cylinder or equivalently the
annulus.
$$
\vbox{\epsfxsize=5.0in\epsfbox{fig3.eps}}
$$
The metric $g$ determines {\it global} complex
coordinates $z$, $\zbar$ in which $g=2g_{z\zbar}\vert
dz\vert^2$.

Recall from 2) that our starting point is the flat
Euclidean $\dbR_E^D$ worldsheet action
$$
S={1\over 4\pi}\int\nolimits_{\Sigma}d^2z\,
\partial_z x\cdot\partial_{\zbar}x
\eqno{(2.22)}
$$
which for fixed metric $g$ defines a CFT of $D$ scalar
fields $x^\mu$, which are the coordinates of the map
$x\colon\,\Sigma\to\dbR_E^D$.
The quantum stress tensor is
$$
T(z)=-{1\over 2}: \partial_z x\cdot\partial_z x:
\eqno{(2.23)}
$$
and the central charge is $c=D$.

Following point 3), we obtain the OPE's and associated
algebraic structure.
To do this, we take advantage of the fact that
$\partial_z x(z)$ and $T(z)$ are holomorphic, and thus
admit a Laurent expansion inside the annulus:
$$
\eqalignno{
\partial_z x(z) &=-i\sum\limits_{m\in\dbZ}x_m z^{-m-1}
\qquad\qquad T(z)=\sum\limits_{m\in\dbZ}L_m
z^{-m-2}\cr
\partial_{\zbar}x(\zbar) &=-i\sum\limits_{m\in\dbZ}\xtil_m
  \zbar^{-m-1}\qquad\qquad
\Ttil(\zbar)=\sum\limits_{m\in\dbZ}\Ltil_m
\zbar^{-m-2} &(2.24)\cr}
$$
{}From the definition of $T\equiv -{1\over 2}\colon\,
\partial_z x\cdot\partial_z x\colon$, we have
$$
\eqalign{
L_m &={1\over 2}\sum\limits_{n} x_{m-n}\cdot
x_n\qquad\qquad m\not=0\cr
L_0 &={1\over 2}x_0^2+\sum\limits_{n=1}^\infty
  x_{-n}\cdot x_n\cr}
\eqno{(2.25)}
$$
{}From the OPE of $\partial_z x$ with
itself and with $T(z)$, we get
$$
[x_m^\mu,x_n^\nu] =m\delta_{m+n,0}\delta^{\mu\nu}
\qquad\qquad [L_m,x_n] =-n x_{m+n}\,\,,\eqno{(2.26a)}
$$
also
$$
[\xtil_m^\mu,\xtil_n^\nu]
=m\delta_{m+n,0}\delta^{\mu\nu}
\qquad\qquad [\Ltil_m,\xtil_n] =-n\xtil_{m+n}\,\,,
\eqno{(2.26b)}
$$
while tilded and untilded operators mutually commute.
$L_m$ and $\Ltil_m$ satisfy Virasoro algebras with
central charge $c=D$.

Following point 4), we find that the above OPE's and
their algebra contents can safely be continued to
$x^0\to ix^0$ where $x^0$ is the time component of the
map $x\colon\,\Sigma\to\dbR_M^D$.
It is clear that this means
$$
\left.
\eqalign{
&x_m^0\to ix_m^0\cr
&\xtil_m^0\to i\xtil_m^0\cr}
\right\}\qquad\hbox{all $m\in\dbZ$}.
\eqno{(2.27)}
$$
The stress tensor is now
$$
T(z)=-{1\over 2}\colon\,\partial_z x^\mu\partial_z
x^\nu\colon\,\eta_{\mu\nu}\equiv -{1\over 2}\colon\,
\partial_z x\cdot\partial_z x\colon
\eqno{(2.28)}
$$
and the algebraic relations are
$$
\eqalign{
[x_m^\mu,x_n^\nu] &=m\delta_{m+n,0}\eta^{\mu\nu}
\quad [L_m,x_n^\mu] =-n x_{m+n}^\mu\cr
L_m &={1\over 2}\sum\limits_{n}\colon\,x_{m-n}^\mu
x_n^\nu\colon\,\eta_{\mu\nu}\cr}
\eqno{(2.29)}
$$
and analogously for $\xtil$ and $\Ltil$.
The central charge is unchanged, and we still have
$\Vir_L\otimes\Vir_R$.

The field $x$ itself is not a conformal tensor, but
may be obtained by integrating $\partial_z x$ and
$\partial_{\zbar}x$:
$$
x(z,\zbar)=x_L(z)+x_R(\zbar)\eqno{(2.30)}
$$
with
$$
\eqalign{
x_L(z) &=q_L-i x_0\ln\,z+i\sum\limits_{m\not=0}x_m
  {z^{-m}\over m}\cr
x_R(\zbar)
&=q_R-i\xtil_0\ln\,\zbar+i\sum\limits_{m\not=0}
  \xtil_m{\zbar^{-m}\over m}.\cr}
\eqno{(2.31)}
$$
Since $x:\Sigma\to \dbR_M^D$, $x$ must be a
single-valued map as a function of $z$.
This will require that $x_0=\xtil_0=p$, and $p$ is to
be identified with the {\it momentum} operator of the string,
while $q=q_L+q_R$ is the overall position operator of the
string.

It will turn out to be useful to record the explicit
form of the operators $L_0$ and $\Ltil_0$, expressed
in terms of $p$.
$$
\eqalign{
L_0 &={1\over 2}p^2+\sum\limits_{n=1}^\infty
x_{-n}\cdot x_n\cr
\Ltil_0 &={1\over
2}p^2+\sum\limits_{n=1}^\infty\xtil_{-n}\cdot\xtil_n\cr}
\eqno{(2.32)}
$$
For non-simply connected space-time manifolds $M$,
the map $x$ need not be single-valued, since the
string can wrap around non-trivial $1$-cycles in $M$.
As a result, $x_0$ need not be equal to $\xtil_0$,
but the difference $x_0-\xtil_0$ belongs 
to a lattice, dual to
$H^1(M,\dbZ)$.

\bigskip\noindent
C) \ {\bf The free open bosonic string conformal field
theory}

\smallskip
The worldsheet of a free open string has the topology
of a strip or a half annulus.
$$
\vbox{\epsfxsize=4.5in\epsfbox{fig4.eps}}
$$
Fundamental open strings should have freely moving
boundary points.
Thus Neumann conditions $\partial_{\im\,z}x=0$ at the
boundary $B$ are appropriate.
(Dirichlet b.c. would fix the end points of the
string on some submanifold $\scrD$ of $M$.
Such submanifolds are $D$-branes, or depending upon
their space dimension $p\colon\,p$-branes.
In certain string theories, $D$-branes become
dynamical (extended) objects, perhaps to be discussed
later on.)

It is standard to double $\Sigma$ by reflecting
$\Sigma$ onto the lower half plane into $\Sigma^*$
and then considering the union
$\Sigma'=\Sigma\cup\Sigma^*$ which is the full annulus.
Functions on $\Sigma$ with Neumann boundary
conditions on $B$ are
then even functions under $z\to\zbar$.
Thus, the mode expansion of the open string is
obtained from this restriction applied to
$x(z,\zbar)$
$$
x(z,\zbar)=x_L(z)+x_L(\zbar)\eqno{(2.33)}
$$
and only involves the oscillators $x_m$ but
there are no oscillators $\xtil_m$. 
Similarly, we only have Virasoro generators $L_n$ but
there are no $\Ltil_n$.
As a result, the energy-momentum tensor satisfies
$$
T(z)=\Ttil(\zbar)=0\qquad\hbox{on}\qquad
\im(z)=0.\eqno{(2.34)}
$$
This condition guarantees conservation of $T$ across
$B$, and results from the fact that $\Diff(\Sigma)$
leaves $B$ invariant.
We have the mode expansion
$$
x(z,\zbar)=x+2ip\ln\,\vert z\vert^2-i
\sum\limits_{m\not=0}x_m\left({z^{-m}\over m}+
{\zbar^{-m}\over m}\right)\eqno{(2.35)}
$$
Notice that $x_0=2p$, in contrast with the closed
string.
It is useful to also record the Virasoro generators
$$
\eqalign{
L_0 &=2p^2+\sum\limits_{n=1}^\infty x_{-n}\cdot x_n\cr
L_m &=\sum\limits_{n\in\dbZ} x_{m-n}\cdot
x_n\qquad\qquad m\not=0\cr}\eqno{(2.36)}
$$
Thus the open string only has a single Virasoro
algebra $\Vir_L$. 

\bigskip\noindent
D) \ {\bf Fock Space, Negative Norm States}

\smallskip
We now follow point 5) and construct the Fock space of
the conformal field theory of $x$, continued to
Minkowski space-time.
(This parallels Gawedzki's construction).

To do so, we must choose a polarization and split the
operators $\{x_m^\mu\}$ and $\{\xtil_m^\mu\}$ into a
group of raising, lowering and central generators.
This choice must be consistent with Poincar\'e
invariance.
(Also, here we must use a property inherited from the
Minkowski nature of the worldsheet, namely that $x_m$ and
$\xtil_m$ oscillators are independent, except at
$m=0$.
We shall not expand upon this issue now, but come back to
this part later on.)
The correct choice is $(x_0^\mu)^\dagger =x_0^\mu$,
while $x_m^\mu$ and $\xtil_m^\mu$ are raising
(resp. lowering) for $m<0$ (resp. $m>0$).
Also
$$
\eqalignno{
(x_m^\mu)^\dagger=x_{-m}^\mu\qquad\qquad
&(\xtil_m^\mu)^\dagger=\xtil_{-m}^\mu &(2.37)\cr
\noalign{\hbox{and this implies}}
L_m^\dagger=L_{-m}\qquad\qquad  
   &\Ltil_m^\dagger=\Ltil_{-m}. &(2.38)\cr}
$$
The Fock space $\scrF_k$ is constructed from $x_m^\mu$
oscillators only.
The ground state $\vert\left.0,k\right>_L$ 
(also called the highest weight state) is labeled by
the momentum $k$ (the eigenvalue of the momentum
operator $p$)
$$
\eqalignno{
&\cases{
p\vert \left.0,k\right>_L=k \vert \left. 0,k\right>_L &\cr
&\cr
x_m^\mu\vert\left.0,k\right>_L=0 &$m>0$.\cr}&(2.39)\cr
\noalign{\medskip\hbox{We may choose its
normalization as follows:}}
&_L\!\left<0,k\vert 0,k'\right>_L=\delta(k-k')\,\,.
&(2.40)\cr}
$$
Arbitrary vectors in $\scrF_k$ are obtained by
applying $x_{-m}^\mu$'s for $m>0$ to the ground state
$$
\vert\left.\eps,k\right>_L
=\eps_{\mu_1\ldots\mu_n}(k;m_1,\ldots,m_n)
x_{-m_1}^{\mu_1},\ldots
x_{-m_n}^{\mu_n}\vert \left.0,k\right>_L
\eqno{(2.41)}
$$
for all possible Lorentz {\it polarization tensors}
$\eps_{\mu_1\ldots\mu_n}$ $(k;m_1,\ldots,m_n)$; $n\in\dbN$,
with all possible $m_i\in\dbN$.

\medskip\noindent
$\bullet$ \ The open string Fock space is 
a sum over all momenta $k$ of $\scrF_k$:
$$
\scrF_{\Open}=\bigoplus\limits_{k\in\dbR_M^D}\scrF_k
$$

\medskip\noindent
$\bullet$ \ 
The closed string Fock space is a sum over all
momenta --- which must be the same for left and right
movers, in view of the constraint $x_0=\xtil_0$ ---
of the tensor product of $\scrF_k$ and $\scrFtil_k$,
where $\scrFtil_k$ is built from $\xtil_m^\mu$:
$$
\scrF_{\Closed}
=\bigoplus\limits_{k\in\dbR_M^D}\scrF_k\otimes\scrFtil_k.
$$

\medskip\noindent
$\bullet$ \ Using the norm on $\vert \left. 0,k\right>_L$
introduced in (2.40), $\scrF_k$ naturally inherits a
quadratic form.
However, $\scrF_k$ automatically contains {\it
negative norm states}.
To see this, consider states $\vert\left.\eps,k\right>_L=
\eps_\mu x_{-m}^\mu\left.\vert 0,k\right>_L$ and with
$m>0$ evaluate their norm:
$$
{}_L\left<\eps,k\vert\eps,k'\right>_L
=\left<0,k\right.\vert
\eps\cdot x_{-m}^\dagger \eps\cdot x_{-m}\left.\vert
0,k'\right>=m\eps^2\delta(k-k')\eqno{(2.42)}
$$
For $\eps^2>0$, the norm of the state
$\vert\left.\eps,k\right>_L$ is positive.
But, for $\eps^2<0$, which is allowed with Minkowski metric,
the norm of $\vert\left.\eps,k\right>_L$ is negative.
Such states are called {\it ghosts} and their presence
violates quantum mechanics: they cannot be present in a
physically sensible theory.

\bigskip\noindent
E) \ {\bf Integration over $\Met(\Sigma)$ --- Virasoro
Constraints}

\smallskip
We now deal with point 6): \ the integration over
$\Met(\Sigma)$, and the elimination of ghosts from the
spectrum.
We shall perform this integration with great care in 
forthcoming lectures, but we can already here extract its key
effects.

The metric $g$ enters in the action $S$ as a
non-dynamical field, i.e. without derivatives.
The r\^{o}le of the metric $g$ here is completely
analogous to that of the time-like component $A_0$ of the
gauge field in Yang-Mills theory (see also Faddeev's lecture).
The effect of such non-dynamical fields is to supply a
constraint on the dynamics of the remaining degrees of
freedom, and to reduce their phase space.
(You may loosely view the effect in the functional
integral as generating a delta function $\int
dA_0e^{iA_0x}=\delta(x)$.)
The constraints also generate a group action on the phase
space, and the reduced phase space may be viewed as the
quotient by this group action.

Let us analyze the effect of the constraints at the {\it
classical level} first.
The variational equations are
$$
\cases{
\partial_{\zbar}\partial_z x=0 &by varying $x$\cr
&\cr
T(z)=\Tbar(\zbar)=0 &by varying $g$\cr}
\eqno{(2.43)}
$$
The second equation is a constraint on $x$ since
$T(z)=-{1\over 2}\partial_z x\cdot\partial_z x$
(similarly for $\Ttil(\zbar)$).
Since $\partial_{\zbar}T(z)=0$, the
first equation shows that 
$T$ will vanish throughout $\Sigma$ as
long as $T=0$ has been imposed on the boundary.
The action of the constraints at the classical level
is something we have
already analyzed, namely
the Virasoro algebra but with zero central charge.
Geometrically, the constraint reflects the fact that
vibrations of $x(\Sigma)$ tangent to
the surface $x(\Sigma)$ are eliminated,
leaving only transverse direction:
$$
\raise3.5cm\hbox{$\Bigl.\eqalign{
&(D-2) \hbox{ transverse}\cr
\noalign{\vskip-10pt}
&\hbox{directions}\cr}\Bigr\}$}\quad
\vbox{\epsfxsize=1.5in\epsfbox{fig5.eps}}
\kern-10pt
\raise1.5cm\hbox{$\left.\phantom{\vrule
height15pt}\right\}
\eqalign{
&2\hbox{ longitudinal directions}\cr
\noalign{\vskip-10pt}
&\hbox{eliminated by $\Tbar=T=0$}\cr}$}
$$

\medskip
We now analyze the effects of the constraints at the
{\it quantum level}.
There are basically three ways of dealing with the
constraints:

\medskip\noindent
1) \ Eliminate the longitudinal degrees of freedom at the
classical level by solving for them in terms of the
transverse degrees of freedom, and then quantize the
transverse degrees of freedom.
This procedure goes under the name of {\it light-cone
gauge} quantization. 
It cannot be achieved in a manifestly Lorentz invariant
parametrization.
We shall discuss it very briefly later on in this
lecture.

\medskip\noindent
2) \ Ultimately, in lecture VII, we shall reformulate
this constraint problem in terms of the powerful BRST
quantization method.
This is a more complex scheme based on introducing extra,
unphysical degrees of freedom, called Faddeev-Popov
ghosts, and enforcing BRST (Becchi-Rouet-Stora-Tyutin)
invariance.
This method is manifestly Lorentz invariant.
The physical Hilbert space will in fact arise as a
cohomology group of a semi-infinite cohomology complex,
as developed by Feigin and later on by Freeman and Olive
in physics and Frenkel, Garland and Zuckerman in
mathematics.

\medskip\noindent
3) \ Before launching into BRST, however, we shall
analyze the quantum effects of the constraints in a more
direct fashion, which will also yield immediate results
on the spectrum.
In this third method, we impose the constraint as an
invariance condition and use it to select an invariant
subspace out of the full Fock space $\scrF$
(either $\scrF_{\Open}$ for open strings or
$\scrF_{\Closed}$ for closed strings).
In Yang-Mills theory, the corresponding constraint was
just Gauss' law, enforcing gauge invariance on physical
states.
In string theory, the constraints are the vanishing of
the stress tensor, enforcing $\Diff(\Sigma)$ invariance
on physical states.
Due to the central extension in $\Vir$, the reduced Fock
space $\scrF^{\plus}=\scrF/\Vir$ will have a positive
definite norm 
only when certain extra conditions are satisfied, which
we now determine.

\bigskip\noindent
F) \ {\bf Physical Spectrum, No-Ghost Theorem}

\smallskip\noindent
We use the Virasoro constraints to select an
invariant subspace out of the full Fock space $\scrF$.
It is tempting to try and impose $L_n\vert\left.\phi\right>
=\Ltil_n\vert\left.\phi\right>=0$ for all $n\in\dbZ$.
However, unless $c=0$, these conditions imply that
$\vert\left.\phi\right>=0$.
Instead, we impose invariance under half of $\Vir$.
The correct physical state conditions are therefore
$$
\matrix{
\vert\left.\phi\right>_L &\kern-5pt \in\scrF_k^{\phys}
  &\quad {\rm if} &\quad (L_m-a\delta_{m,0})\vert
  \left.\phi\right>_L=0 &\quad m\in\dbN\cr
\noalign{\smallskip}
\vert\left.\phi\right>_R &\kern-5pt\in\scrFtil_k^{\phys}
  &\quad {\rm if} &\quad (\Ltil_m-a\delta_{m,0})
  \vert\left.\phi\right>_R=0 &\quad m\in\dbN.\cr}
\eqno{(2.44)}
$$
and the full physical Fock spaces are obtained by
$$
\eqalign{
\scrF_{\Open}^{\phys} &=\mathop{\oplus}\limits_{k}
     \scrF_k^{\phys}\,\,,\cr
\scrF_{\Closed}^{\phys} &=\mathop{\oplus}\limits_{k}
\scrF_k^{\phys}\otimes\scrFtil_k^{\phys}\,\,.\cr}
$$

The constant $a$ is undetermined at this
point, and is allowed for since $L_0$ and $\Ltil_0$ were
naturally defined only up to an additive constant in view
of the ordering choice we made in their definition.
For closed strings,
the value of $a$ is taken the same for $L_0$ and
$\Ltil_0$ so that $L_0-\Ltil_0={\partial\over\partial
\sigma}$ enforces rotation invariance on the annulus. 
Notice that states
$\vert\left.\phi\right>,\,\vert\left.\psi\right>\in
\scrF^{\phys}$ automatically yield zero matrix 
elements of Virasoro generators
$L_m-a\delta_{m,0}$ for all $m\in\dbZ$
(similarly for $\Ltil_m-a\delta_{m,0}$ in the case of the
closed string).
Hence, we have now realized the physical state condition
$\left<\phi\vert T\vert\psi\right>=\bigl<\phi
\vert\Ttil\vert\psi\bigr>=0$!

States of the form
$\vert\left.\phi\right>=L_{-m}\vert\left.\psi\right>$,
for $m>0$, are called {\it spurious}, and we define
$$
\scrF_k^{\,\spurious} =\left\{\vert\left.\phi\right>=
\bigoplus\limits_{m=1}^\infty L_{-m}\vert\left.
\chi_m\right>\right\}
$$
It is easy to see that $\scrF_k^{\spurious}$ is
generated by states of the form
$L_{-1}\vert\left.\chi_1\right>\oplus
L_{-2}\vert\left.\chi_2\right>$, using the
commutation relations of the Virasoro algebra which
imply that $[L_{-1},L_{-n}]$ is a non-zero multiple
of $L_{-1-n}$ for $n\Ge 2$.
Notice that spurious states
are orthogonal to all physical states.
States that are both physical and spurious
are {\it null},
$$
\scrF_k^{\,\spurious}\cap
\scrF_k^{\phys}\subset \scrF_k^{\Null}\,\,,
$$
where $\scrF_k^{\Null}$ stands for the space of null
states of momentum $k$.

The special Virasoro conditions
$(L_0-a)\vert\left.\phi\right>
=(\Ltil_0-a)\vert\left.\phi\right>=0$ determine the value
of $k^2$ for a given level of operators, and thus
determines the mass of the state.
To see this, introduce the number (also called
level) operators
$$
N\equiv\sum\limits_{n=1}^\infty x_{-n}\cdot x_n
\qquad\qquad\Ntil\equiv\sum\limits_{n-1}^\infty
\xtil_{-n}\cdot\xtil_n\,\,.
\eqno{(2.45)}
$$
The operators
$N$ and $\Ntil$ count the number of operators $x_{-n}$ and
$\xtil_{-n}$, $n\Ge 1$ with weight $n$, 
applied to the ground state
$\vert\left.0,k\right>$.
The special Virasoro conditions thus become
$$
\matrix{
{\Closed}\hfill &\quad (k^2+M^2)\vert\left.\phi\right>=0
  &\quad M^2=2N-2a &\quad \hbox{and $N=\Ntil\vert
  \left.\phi\right>=0$}\cr
\noalign{\smallskip}
{\Open}\hfill &\quad (k^2+M^2)\vert\left.\phi\right>=0
  &\quad M^2={1\over 2}N-{1\over 2}a\,\,. &\cr}
\eqno{(2.46)}
$$
Thus, the $\mass$ squared of various string states are spread
by $2\times$ integers for closed strings
and ${1\over 2}\times$ integers for open strings.


The $\mass$ squared of a state is one of the Casimir values
that label the representation of the Poincar\'e group
under which the state transforms.
The remaining data that fully specify the Poincar\'e
representation are those given by the representation of
the stabilizer group (i.e. the subgroup
that leaves the momentum invariant).
This information is contained in the expression of the
Lorentz generators of the string.
$$
\eqalign{
J^{\mu\nu}=q^u p^\nu -q^\nu &p^\mu -i\sum\limits_{n=1}^\infty
{1\over n}(x_{-n}^\mu x_n^\nu -x_{-n}^\nu x_n^\mu)\cr
&\left(-i\sum\limits_{n=1}^\infty{1\over n}
(\xtil_{-n}^\mu \xtil_n^\nu-\xtil_{-n}^\nu
\xtil_{-n}^\mu)\quad\hbox{for closed strings}\right)\cr}
\eqno{(2.47)}
$$
It follows that the string ground state
$\vert\left.0,k\right>$ transforms as a scalar (i.e.
trivial representation under the stabilizer).

We observed in the first lecture that, at long
distances compared to the Planck length $\ell_P$,
the size of strings effectively becomes unobservable. 
We can then approximate the dynamics of the string by that of
point particles, and string theory by quantum field theory.
Thus, we shall seek to identify string states with states
in quantum field theory.
To do so, we require that string and quantum field states
transform under the same representations of the
Poincar\'e group, as well as of
other symmetry groups that are available. 
Also their interactions should agree in this 
long distance limit.

We examine under what conditions the Virasoro
constraints eliminate all negative norm states, so that
$\scrF_k^{\phys}$ has positive norm.
Before stating the general result, let's look at the lowest
level cases (for the closed string; for the open string
it suffices to let $M^2\to M^2/4$).

\medskip\noindent
(1) \ Ground state $\vert\left.0,k\right>_L$ with
mass square $M^2=-2a$ (actually
  $\vert\left.0,k\right>_L\otimes\vert
\left.0,k\right>_R$ for the closed string).
The Virasoro conditions are automatically satisfied.

\medskip\noindent
(2) \ First excited state
$\vert\left.\eps,k\right>_L=\eps\cdot
x_{-1}\vert\left.0,k\right>_L$ with mass square
$M^2=-2a+2$:

$$
L_1 \vert\left.\eps,k\right>_L=0  \Longrightarrow
  \eps\cdot k=0\,\,.
$$
The remaining Virasoro conditions
$L_n \vert\left.\eps,k\right>_L=0,\,\, n\Ge 2$,
are automatically satisfied.

\noindent
There are three possibilities:

{\narrower{\narrower{\narrower{\medskip\noindent
\item{(i)} $a>1$, so that $k^2$ is space-like. If we
 pick $k=(0,k^1,0\ldots 0)$ then we are allowed to
choose $\eps=(\eps^0,0\ldots 0)$, and
we see that ghosts are still present.

\smallskip
\item{(ii)} $a<1$, so that $k^2$ time-like. If we pick
$k=(k^0,0\ldots0)$ we see that we always have 
$\eps^2>0$ and there are $D-1$
independent states with positive norm.

\smallskip
\item{(iii)} $a=1$, so that
$k^2$ is light-like. If we pick
$k=(k^1,k^1,0\ldots0)$, we see that there are 
$(D-2)$ massless states of positive norm, and one
of zero norm.\medskip}}}}

\noindent
Case (i) is clearly eliminated: \ Virasoro constraints do
not suffice to eliminate all negative norm states.
Case (ii) has massive $M^2>0$ vector states (open
string), or massive rank $2$ tensor states (closed
string).
Case (iii) has massless $M^2=0$ vector states (open
strings), which suggests the presence of massless
Yang-Mills states, and has massless $M^2=0$ rank $2$
tensor states (closed strings) which suggests the
presence of gravity (for symmetric tensors).

\medskip\noindent
3) \ At the next level $\vert\left.\eps,k\right>_L=
\{\eps_{\mu \nu}x_{-1}^\mu x_{-1}^\nu +\eta_\mu
x_{-2}^\mu\}\vert\left.0,k\right>_L$, with mass
squared $M^2=-2a+4$ and
$$
\matrix{
L_1 &\null\kern-5pt\vert\left.\eps,k\right>_L=0\cr
\noalign{\smallskip}
L_2 &\null\kern-5pt\vert\left.\eps,k\right>_L=0\cr}
\qquad\cases{
\eps_{\mu\nu}k^\nu+\eta_\mu=0 &\cr
&\cr
\eps_\mu^\mu+2k_\mu\eta^\mu=0\,.\cr}
$$
The norm of the state is 
$$
{}_L\!\!\left<\eps,k\vert\eps,k'\right>_L=2\scrN
{}_L\!\!\left<0,k\vert 0,k'\right>_L
$$
with
$$
\eqalign{
\scrN &=\eps_{\mu \nu}\eps^{\mu\nu}+\eta_\mu \eta^\mu\cr
&=(\eps_{00})^2-2(\eps_{0i})^2+(\eps_{ij})^2-
  (\eta_0)^2+(\eta_i)^2.\cr}
$$
Choosing a rest frame where $k^0=M$, $k^i=0$,  the
Virasoro conditions become
$$
\cases{
\eps_{00}M+\eta_0=0 &\cr
\eps_{i0}M+\eta_i=0\cr
-\eps_{00}+\eps_{ii}+2M\eta_0=0\,\,.\cr}
$$
The Virasoro conditions impose no constraints on the
traceless part of $\eps_{ij}$, and its contribution to
$\scrN$ is $\Ge 0$.
Hence, we may retain only the trace part: \
$\eps_{ij}=\eps \delta_{ij}$.
Eliminating $\eta_i$ and $\eta_0$ and $\eps_{00}$ in
favor of $\eps_{0i}$ and $\eps$, we have
$$
\scrN={\eps^2(D-1)\over(2M^2+1)^2}\left\{
(2M^2+1)^2-(M^2-1)(D-1)\right\}+(M^2-2)(\eps_{0i})^2.
$$
Assuming $D>1$, we find the conditions
$$
\cases{
M^2\Ge 2 &\cr
&\cr
(2M^2+1)^2-(M^2-1)(D-1)\Ge 0\,\,. &\cr}
$$
the first of which is already satisfied since
$M^2=-2a+4$ and we have seen before that we must have
$a\Le 1$.  
For $M^2=2$ (corresponding to massless states at level
$1$), we find from the second inequality that $D\Le 26$.
In fact, for $D=26$, the second inequality implies $M^2\Le
2$, so that we must have $M^2=2$ for $D=26$.

For $D\Le 25$, the quadratic form
$(2M^2+1)^2-(M^2-1)(D-1)$ is non-negative
for all $M^2\Ge 2$, so 
the state $\vert\left.\eps,k\right>$ always has
non-negative norm.

For $D\Ge 27$, the quadratic form is negative for a
region in which $M^2\Ge 2$.
In fact, by analyzing higher level states it can be
shown that for $D\Ge 27$ no value of $a$ can be
chosen so as to eliminate all ghosts.
The complete result can be summarized in the
following

\Proclaim{No Ghost Theorem {\rm (R. Brower, P. Goddard
and C. Thorn)}.}

\noindent
$\bullet$ \ For $D>26$ {\bf or} $a>1$

{\narrower{\narrower\smallskip\noindent
$\scrF^{\phys}$ contains negative norm states;\medskip}}

\noindent
$\bullet$ \ For $D=26$ {\bf and} $a<1$

{\narrower{\narrower\smallskip\noindent
$\scrF^{\phys}$ contains negative norm states;\medskip}}

\noindent
$\bullet$ \ For $D=26$ {\bf and} $a=1$

{\narrower{\narrower\smallskip\noindent
$\scrF^{\phys}$ contains {\it only} non-negative norm
states, in particular, two towers of null states
and $\scrF^{\plus} \equiv \scrF^{\phys}/\scrF^{\Null}$
has positive definite norm;
$$
\scrF^{\Null} =\left\{(L_{-2}+{3\over 2}L_{-1}^2)\vert
  \left.\chi_2\right>\oplus
L_{-1}\vert\left.\chi_1\right>\right\}
$$
\medskip}}


\noindent
$\bullet$ \ For $D<26$ {\bf and} $a\Le 1$

{\narrower{\narrower\smallskip\noindent
$\scrF^{\phys}$ but contains only positive norm states for 
$a<1$ $($contains one tower of null states
$L_{-1}\vert\left.\chi_1\right>$ for $a=1$$)$.\smallskip}}
\finishproclaim

We shall not prove the no ghost theorem here, since later
on, using BRST quantization, a simple proof can be given.

The case $D\Le 25$ and $a\Le 1$ is contained in $D=26$,
$a=1$ by setting $26-D$ space-like operators to zero, and
adjusting the $26-D$ positive squared momenta
$p_i^2$, $i=1,\ldots,26-D$, so that 
$$
{1\over 2}\,\sum\limits_{i}\,p_i^2-a+1=0\,\,.
$$

\vfill\eject

\noindent
G) \ {\bf Spectrum of the critical bosonic string with
$D=26$, $a=1$}

\smallskip
Based on the masses and Lorentz tensor structure of the
lowest lying states of the spectrum, we may identify which
particles they naturally correspond to in field theory.

It is easy to see that the spectrum consists entirely
of bosons.

\bigskip\noindent
{\bf Open strings}

\medskip\noindent
(1) \ States corresponding to
$\vert\left.0,k\right>$ have $M^2=-{1\over 2}$, hence
they are {\it tachyons} (particles traveling faster
than light), and Lorentz scalars;

\smallskip\noindent
(2) \ States of the form $\eps\cdot
x\vert\left.0,k\right>$ with $M^2=0$ are vectors
under $\SO(1,D-1)$.
There are $(D-2)$ positive norm states and one null
state, which constitute a massless gauge
boson in $\dbR^D$.
Gauge transformations have the form
$\eps_\mu\to\eps_\mu+k_\mu\gamma$ (invariance under
these transformations is clear since they amount
to shifting the state by a null state $\gamma
L_{-1}\vert\left.0,k\right>=\gamma
k\cdot x_{-1}\vert\left.0,k\right>$);

\medskip\noindent
(3) \ $M^2>0$ states, with $M^2={1\over 2}n$,
$n\in\dbN$, transform under various tensor
representations of $\SO(1,D-1)$.

\bigskip\noindent
{\bf Closed strings}

\medskip\noindent
(1) \ States corresponding to $\vert
\left.0,k\right>=\vert\left.0,k\right>_L\otimes
\vert\left.0,k\right>_R$ have $M^2=-2$, so they are
{\it tachyons}, and Lorentz scalars;

\medskip\noindent
(2) \ States of the form
$\eps_{\mu\nu}x_{-1}^\mu\xtil_{-1}^\nu\vert\left.0,k\right>$
have $M^2=0$.
Corresponding to the trace-part of $\eps$ there is a
Lorentz scalar, the {\it dilaton}, with positive
norm.
The symmetric traceless part of $\eps$ gives a
Lorentz rank $2$ tensor, the {\it graviton}.
There are ${D(D+1)\over 2}-1$ such states,
${D(D-1)\over 2}-1$ of them having positive norm.
As for the antisymmetric part of $\eps$, it gives a
Lorentz rank $2$ tensor called the {\it
anti-symmetric tensor} with ${D(D-1)\over 2}$ states,
out of which ${(D-1)(D-2)\over 2}$ have positive
norm.
These states exhibit gauge invariance under 
$$
\eps_{\mu\nu}\to\eps_{\mu\nu}+\gamma_\mu
k_\nu+\gamma'_\nu k_\mu\eqno{(2.48)}
$$
(which amounts to shifting by null states);

\medskip\noindent
(3) \ $M^2>0$ states, with $M^2=-2n$, $n\in\dbN$,
transform under various tensor representations of
$\SO(1,D-1)$.



\bigskip\noindent
\noindent
H) \ {\bf Lightcone gauge, density of states}

\smallskip
For $D=26$ and $a=1$, the null states in $\scrF^{\Null}$
may be ignored completely.
We may retain positive norm states only at the cost of
giving up manifest Lorentz invariance, $\SO(1,25)$, by
retaining $D-2=24$ transverse oscillators only.
Let $M'$ be a space-like sub-vector space of $M$, with
$\dim\,M'=24$, and let $\eps$, $\epstil$ be in $M'$;
then $\eps\cdot x_{-n}$ and $\widetilde{\eps}\cdot
\xtil_{-n}$ for $n>0$ create only positive norm states.
This construction has manifest $\SO(24)$ invariance.
It can be checked that the full Lorentz algebra 
$\SO(1,25)$ closes in this realization.
Also, it can be shown in this case that there is a 
one-to-one
correspondence between those states constructed
above and the states in $\scrF^{\plus}=\scrF^{\phys}/
\scrF^{\Null}$ in the covariant approach.

If the same construction were attempted for $D\not= 26$
or $a\not=1$, one would find that the Lorentz algebra in
this realization does not close.

Using the above correspondence, it is very easy to
estimate the total number of states at each mass level.
Introducing a generating function
$$
\eqalign{
G(w) 
&=\Tr_{\scrF^{\plus}}(w^{2M^2+2})=\prod\limits_{n=1}^\infty
  (1-w^n)^{-24}\cr
&\equiv\sum\limits_{n=1}^\infty d_n w^{2n}\,\,.\cr}
\eqno{(2.49)}
$$
We find the number of states asymptotically for large
$M^2$: $d_{M^2}=M^{-25/2}e^{M/4\pi\sqrt{2}}$.

\bigskip\noindent
I) \ {\bf Primary fields and Vertex Operators for
Physical States}

\smallskip
The construction given above for the physical Hilbert
space has the disadvantage that the operators
$x_{-n}^\mu$ with $n>0$,
create states in Fock space that are not
in general physical.
We now construct {\it vertex operators}, which have
the property that they send physical states into
physical states, and spurious states into other
spurious states.

Since physical states obey
$(L_n-\delta_{n,0})\vert\left.\psi\right>
=(\Ltil_n-\delta_{n,0})\vert\left.\psi\right>=0$,
$n\Ge 0$, it is clear that any operator commuting
with the entire Virasoro algebra will yield such a
vertex operator.  
Also, since $[L_n,V]=[\Ltil_n,V]=0$,
$V$ automatically sends spurious states into
spurious states.

Since $L_n$ generates conformal reparametrizations, $V$
should be conformal invariant.
To construct $V$'s, we consider general tensors,
then construct invariants.
We use the action of $L_n$ on any {\it primary
field}:
$$
\eqalign{
[L_n,\phi(z,\zbar)] &=z^{n+1}\partial_z\phi(z,\zbar)
  +h(n+1)z^n\phi(z,\zbar)\cr
&=\partial_z(z^{n+1}\phi(z,\zbar))+(h-1)(n+1)
  z^n\phi(z,\zbar)\cr
[\Ltil_n,\phi(z,\zbar)] &=\partial_{\zbar}
  (\zbar^{n+1}\phi(z,\zbar))+(\barh-1)(n+1)
  \zbar^n\phi(z,\zbar)\,\,.\cr}
\eqno{(2.50)}
$$
It is clear that amongst primary fields, the only
operator that commutes with all of $L_n$ and $\Ltil_n$ is the
unit operator $\phi=1$.

However, integrals of $\phi$ may be invariant.
For any primary field of conformal weight $(h,\barh)=(1,1)$,
its commutators with $L_n$, $\Ltil_n$ are
total derivatives, and so its integral will be
Virasoro invariant.
In addition, primary fields $\phi$ of weight $(1,1)$ form
natural volume forms on $\Sigma$.
Hence the integral of $\phi$
$$
V=\int \phi (z,\zbar)\qquad\hbox{\rm satisfies}
\qquad [L_n,V]=[\Ltil_n,V]=0\,\,.
\eqno{(2.51)}
$$
This is not the only way to construct vertex
operators though: if $\phi$ is a holomorphic primary
field (e.g. $\partial_z x^\mu$), then it will commute
with $\Ltil_n$ provided $\barh=0$ and its commutator
with $L_n$ is a total derivative if  $h=1$.
In addition $(1,0)$ and $(0,1)$ forms are naturally
integrated along contours.
Thus, the operators
$$
\eqalignno{
&\oint dz\,\phi_{(1,0)}\cr
\noalign{\hbox{and}}
&\oint d\zbar\,\phi_{(0,1)}\cr}
$$
are also vertex operators provided that
$\partial_{\zbar}\phi_{(1,0)}=0$ and $\partial_z
\phi_{(0,1)}=0$.
The operators $\phi_{(1,0)}$ and $\phi_{(0,1)}$
 are precisely what we
defined previously as conformal currents; their
charges are conserved.
Such operators do not in fact create new physical
states, as will become clear by considering examples
below.

Vertex operators for the closed bosonic string are
now easily reconstructed from the basic 
field $x^\mu$ associated with $D$ copies of the $c=1$ CFT.
We have the following primary fields
$$
\matrix{
\partial_z x^\mu 
   &\hbox{\rm holomorphic, weight}\hfill &(1,0)\cr
\noalign{\medskip}
\partial_{\zbar}x^\mu 
   &\hbox{\rm anti-holomorphic, weight}\hfill &(0,1)\cr
\noalign{\medskip}
:\,e^{ik\cdot x}:  
&\hbox{weight }
   \left({k^2\over 2},{k^2\over 2}\right)\hfill &\cr}
$$
{}From the holomorphic current $\partial_x x^\mu$ we
get the conserved charge of {\it momentum}:
$$
p^\mu={1\over 2\pi}\oint dz\,\partial_z x^\mu\,\,.
\eqno{(2.52)}
$$
Physical states in $\scrF_k^{\phys}$ will just be
eigenstates of $p^\mu$, so $p^\mu$ does not really
create any new states, but simply measures their
momentum $k$.

A single exponential yields a vertex operator for
$k^2=2$, i.e. when the conformal weight is $(1,1)$:
$$
V_T(k)=\int\nolimits_{\Sigma} d^2z\,:e^{ikx}:
\eqno{(2.53)}
$$
This scalar operator creates a state of mass
$M^2=-2$, which is the tachyon discovered previously.
But the exponential may also be combined with
derivatives for $k^2=0$
$$
V_{G,B,\Phi}(k)=\int\nolimits_{\Sigma}
d^2z \, \eps_{\mu\mubar}(k)\,\partial_z x^\mu
\partial_{\zbar}x^{\mubar}\,:e^{ik\cdot x}:
\eqno{(2.54)}
$$
which reproduces the states for the graviton, the
antisymmetric tensor, and the dilaton
as symmetric traceless, anti-symmetric,
and trace part respectively, with
$k_\mu\eps^{\mu\mubar}(k)=k_{\mubar}\eps^{\mu\mubar}(k)=0$.
For higher powers of $\partial_z x^\mu$ and
$\partial_{\zbar}x^{\mubar}$, we will have masses of
the form $M^2=2n-2$ in general.
However, higher derivatives $\partial_z^2 x^\mu$ may
also appear.
While these are not primary fields (they mix with
$\partial_z x$ under conformal reparametrizations),
they may combine with other operators and still yield
primary fields.

When we obtained the Virasoro conditions on the
polarization tensors $\eps_{\mu\mubar}$, we
established an invariance of the states in
$\scrF^{\phys}$ under the transformations
$$
\eps_{\mu\mubar}\to \eps_{\mu\mubar}+
k_\mu\gammabar_{\mubar}+k_{\mubar}\gamma_\mu\quad
\eqno{(2.55)}
$$
(We shall shortly interpret this invariance as a
manifestation of $\Diff(M)$-invariance for the closed
string.)
This invariance is also easily established for the
vertex operators: for example a shift by $\gammabar$ yields
$$
\eqalign{
\int\nolimits_{\Sigma} d^2z\, k_\mu\gammabar_{\mubar}
\partial_z x^\mu\partial_{\zbar}x^{\mubar}\,:e^{ikx}:
&=-i\int\limits_{\Sigma}d^2z\gammabar_{\mubar}
\partial_{\zbar}x^{\mubar}(\partial_z e^{ikx})\cr
&=i \int\nolimits_{\Sigma}d^2z \gammabar_{\mubar}
 (\partial_z\partial_{\zbar}x^{\mubar})e^{ik\cdot x}\cr}
\eqno{(3.56)}
$$
But we have $\partial_z\partial_{\zbar}x^\mu=0$ by the field
equations --- away from any other singularities ---
and thus the integral vanishes.
(The effects of the singularities may also be taken
into account, as we shall show later on and do not
modify the conclusion above.)

\bigskip\noindent
J) \ {\bf Identifying the graviton: vertex operators
from background fields}

\smallskip
One of the most important discoveries in string
theory is that the $k^2=0$, $\eps_{\mu\mubar}(k)$-symmetric 
traceless physical state, found above, should be
identified with the graviton. 
This is the particle in quantum
field theory that represents excitations of the
space-time metric and that mediates the force of
gravity.
Here, we provide further evidence for this
identification.
Along the same lines, we provide evidence for the
identification of the open string $k^2=0$,
$\eps_\mu(k)$-state with the Yang-Mills particle in
this section and in \S{K}.

Consider a space-time $M$ which is almost flat with
the following metric
$$
G_{\mu \nu}(x)=\eta_{\mu \nu}+\eps_{\mu \nu}(x)
\eqno{(2.57)}
$$
where $\eps_{\mu \nu}(x)$ is everywhere small compared
to $\eta_{\mu \nu}$.
The tensor 
$\eps_{\mu \nu}(x)$ may be viewed as a superposition of
plane waves by Fourier transform:
$$
\eps_{\mu \nu}(x)=\int d^{26}k\,\, 
\eps_{\mu \nu}(k)e^{ik\cdot x}
\eqno{(2.58)}
$$
Now, consider string theory in this background:
the action on the worldsheet is given by
$$
S_G={1\over 8\pi}\int\nolimits_{\Sigma}
d\mu_g\,g^{mn}\partial_m x^\mu\partial_n x^\nu
G_{\mu \nu}(x)
\eqno{(2.59)}
$$
and the transition amplitude involves the exponential
of $S_G$.
Since $\eps_{\mu \nu}$ is small, we may expand in
powers of $\eps_{\mu \nu}$:
$$
e^{-S_G} =e^{-S_\eta}\sum\limits_{n=0}^\infty
  {(-1)^n\over n!}S_n {1\over (8\pi)^n}\eqno{(2.60)}
$$
with $S_\eta$ the action
corresponding to the flat metric and $S_n$ defined by
$$
S_n =\int d^{26}k_1\ldots d^{26}k_n 
  V(k_1)\ldots V(k_n) \eqno{(2.61)}
$$
where
$$
V(k) =\eps_{\mu \nu}(k)\int\nolimits_{\Sigma}
d\mu_g g^{mn}\partial_m x^\mu\partial_n x^\nu
e^{ikx} 
$$
The operator $V(k)$ is precisely the vertex operator we
constructed for the graviton --- written now in
general metric $g$ --- for a definite momentum $k$.

Thus, we see that strings couple to the fluctuations
in the background metric precisely by the graviton
vertex operator.
The plane waves may be regarded as creating
excitations in the background space-time metric,
which make it deviate from the flat metric.
(A completely analogous correspondence holds for the
other massless states, the dilaton and the anti-symmetric
tensor.)

We have however established more precise information:
in order to maintain conformal invariance and
eliminate negative norm states, the momentum $k$ and
the polarization tensor $\eps_{\mu \nu}$ had to satisfy
{\it on-shell conditions}
$$
\eqalign{
k^2\eps_{\mu \nu}(k)=0 &\Longrightarrow
  \square \eps_{\mu \nu}(x)=\square G_{\mu \nu}(x)=0\cr
k^\mu\eps_{\mu \nu}(k)=0 &\Longrightarrow
\partial^\mu\eps_{\mu\nu}(x)=\partial^\mu G_{\mu \nu}(x)=0\cr
\eps_\mu^\mu(k)=0 &\Longrightarrow
  \eps_\mu ^\mu(x)=0\,\,,\cr}
\eqno{(2.62)}
$$
where we have listed to the right of the double
arrows the restatement of the conditions on the
metric.

Now, since $G_{\mu \nu}$ is almost flat 
($\eps$ being small), corrections  quadratic and higher
in $\eps$ may be neglected.
Considering the Ricci tensor in this
approximation, we have
$$
\eqalign{
R_{\mu \nu} &=R_{\mu\lam\nu}^\lam=\partial_\nu
  \Gamma_{\mu\lam}^\lam-\partial_\lam\Gamma_{\mu \nu}^\lam+
O(\eps^2)\cr
2R_{\mu \nu} &=\partial_\mu\partial_\nu \eps_\lam^\lam-
 \partial_\lam\partial_\mu\eps_\nu^\lam-\partial_\lam
  \partial_\nu \eps_\mu^\lam+\square\eps_{\mu \nu}\cr}
\eqno{(2.63)}
$$
The conditions on $G_{\mu \nu}$ obtained from the
on-shell conditions of $\eps_{\mu \nu}(k)$ now imply
that
$$
R_{\mu \nu}=0\,\,,
\eqno{(2.64)}
$$
within the almost flat (or linearized)
approximation.

In addition, the fact that shifts by null states are
physically unobservable implies the gauge invariance
of the physical states in $\scrF^{\plus}$ under the
shifts
$$
\eps_{\mu \nu}(k)\longrightarrow \eps_{\mu \nu}(k)
+k_\mu\gamma_\nu+k_\nu\gamma_\mu
\eqno{(2.65)}
$$
(with $k^\mu\gamma_\mu=0$ for on-shell states).
In terms of the metric, this gauge invariance
translates into
$$
\eps_{\mu \nu}(x)\to\eps_{\mu \nu}(x)+\partial_\mu
\gamma_\nu(x)+\partial_\nu\gamma_\mu(x).
\eqno{(2.66)}
$$
We recognize this transformation as an infinitesimal
diffeomorphism generated by the
vector field $\gamma^\mu(x)$, in
the approximation where $O(\eps^2)$ terms are
neglected.
$R_{\mu \nu}=0$ is invariant under these
transformations, as can be checked explicitly.

In fact $R_{\mu \nu}=0$ is the only 
$\Diff(M)$-invariant equation that 
reduces to $\square G_{\mu \nu}(x)=0$ 
in the linearized limit.
Thus, we find that --- to leading order in metric
fluctuations -- the conformal invariance constraints
translate into Ricci flatness, i.e. Einstein's
equations in the absence of matter.
This discovery supports our proposed identification
of $\eps_{\mu \nu}(k)$, $k^2=0$ string states with
gravitons.

Later on (Lecture VIII), we shall further confirm this
identification in another important limit, that of
small variations in the derivatives of the metric,
the so-called low energy limit.

\bigskip\noindent
K) \ {\bf Internal degrees of freedom of open strings:
Chan-Paton rules}

\smallskip
The end points of the open string are distinguished
points.
It is consistent with Poincar\'e and $2$-$d$ conformal
invariance to put degrees of freedom or {\it charges}
at the end points.
During interactions of several strings, the charges
simply flow along the boundary, and should thus be
{\it conserved}.
$$
\vbox{\epsfxsize=2.75in\epsfbox{fig6.eps}}
$$
The simplest case corresponds to adding {\it electric
charge} to the end points of oriented strings:
$$
\vbox{\epsfxsize=2.75in\epsfbox{fig7.eps}}
$$
\noindent
Consistency with the interactions requires that the
charges be opposite (i.e. corresonding to conjugate
representatives) at the two ends.
An external $U(1)$ gauge field $A_\mu$ couples to
these charges just in the same way as it couples to
point particles with electric charge $\pm e$, moving
on worldlines:
$$
S_A=ie\int\nolimits_{B}d\tau\,\xdot^\mu(\tau)A_\mu(x(\tau)).
\eqno{(2.67)}
$$
Following the derivation of vertex operators from the
coupling to background fields that we used for the
closed string, we easily obtain the vertex operators
for the open string gauge states, by decomposing
$A_\mu$ into Fourier modes
$$
A_\mu(x)=\int d^Dk\,\eps_\mu(k)e^{ik\cdot x}\,\,,
\eqno{(2.68)}
$$
where $\eps_\mu(k)$ is the polarization tensor.
Thus, the vertex operators are
$$
V(\eps,k)=\eps_\mu(k)\int\nolimits_{B}
d\tau\,\xdot^\mu(\tau)e^{ik\cdot x(\tau)}
\eqno{(2.69)}
$$
with conformal invariance for $k^2=0$.

We may add further degrees of freedom to the end
points of the string in the following way.
Let $\lam^i$, $i=1,\ldots,N$ be Grassmann valued
fields that live only on the boundary $B$ 
of the string worldsheet 
(recall that Neumann b.c. were imposed on $B$ for the map
$x\colon\,\Sigma\to M$).
A natural candidate for the action for such degrees
of freedom is
$$
S_0(\lam)=\int\nolimits_{B}d\tau\,\lambar^i
{d\over d\tau}\lam_i\,\,.
\eqno{(2.70)}
$$
This action is $\Diff(B)$ invariant, and does not
depend upon the $\Sigma$ metric.
The equation of motion for $\lam_i$ simply requires
that $\lam_i$ be $\tau$-indepenent, so that $\lam_i$
may indeed be interpreted as conserved charges.
The action $S_0$ automatically has $\SO(N)$ symmetry
when $\lam_i$ are real (and
$\lambar^i=\delta^{ij}\lam_j$) and $\SO(2N)$ symmetry
when $\lam_i$ is complex.
As a result, quantization automatically produces a
Hilbert space of states on which a spinor
representation of $\SO(N)$ (for $\lam_i$ real) or
$\SO(2N)$ (for $\lam_i$ complex) acts.

To restrict to a subgroup $G$ of $\SO(N)$ or
$\SO(2N)$, and/or a subrepresentation $\Lam$ of the
corresponding spinor representations, it is necessary
to restrict the Hilbert space of states.
This is conveniently achieved by introducing a
Lagrange multiplier, $\sigma\colon\,B\to\dbR$.
For $G=U(N)$, for example, the restriction to the
rank $p$ ($0\Le p\Le N$) totally anti-symmetric
tensor representation of $U(N)$, we use the following
modification of $S_0$:
$$
S(\lam)=\int\nolimits_{B} d\tau\left(\lambar^i\,
{d\over d\tau}\,\lam_i+\sigma(\tau)(\lambar^i\lam_i-
p)\right)
\eqno{(2.71)}
$$
Analogous modifications may be used for $G=\SO(N)$
and $\Sp(N)$.
Let us now restrict to $G=U(N)$, and $p=1$, so that
$\Lam$ equals the defining representation.
Quantization of $\lam_i$ generates $N$ conserved
charges attached to the boundaries of the string
$$
\vbox{\epsfxsize=2.75in\epsfbox{fig8.eps}}
$$
The string for $U(N)$ gauge group should be {\it
oriented}, since the charges at both ends of the
string transform under inequivalent representations
of $U(N)$.
The string state now transforms under the
representation $\Lambar\otimes\Lam$ of $U(N)$, which
is the adjoint of $U(N)$.
The representation matrices $T^a$, with $a=1,\ldots,
\dim\,U(N)=N^2$ describe the string state belonging
to the representation $\Lambar\otimes\Lam$, by the
index $a$.
The matrix elements $(T^a)_i^j$ of $T^a$ specify
which $\lam_i$ and $\lambar^j$ are created at the
boundary $B$ of the open string.
The full Fock space of the open string is given by
$\scrF_{\Open}\otimes\Lambar\otimes\Lam$, and the
physical state space is
$\scrF_{\Open}^{\phys}\otimes\Lambar\otimes\Lam$.

We observe that for $G=U(N)$ and $\Lam$ the defining
representation, $\Lambar\otimes\Lam$ transforms in
the adjoint representation and the open string states
are thus natural candidates for Yang-Mills particles
with gauge group $G=U(N)$.
Had we considered any other representation for
$\Lam$ (other than the defining or its complex 
conjugate), the
string states would not have tranformed precisely
under the adjoint representation of $U(N)$, and could
not have been interpreted as Yang-Mills particles.
Thus the defining representation is singled
out by its relevance to describing Yang-Mills
particles.

The strings for $\SO(N)$ and $\Sp(N)$ should be
unoriented, since the charges at both ends transform
under equivalent representations of $\SO(N)$ and
$\Sp(N)$, respectively.
Considering $\Lam$ to be the defining representation
of $\SO(N)$, we obtain string states in the adjoint
of $\SO(N)$ by the anti-symmetrized tensor product of
$\Lam$ by itself $(\Lam\otimes\Lam)_A$.
On the other hand, for $\Lam$ the defining
representation of $\Sp(N)$, we obtain the string
states in the adjoint of $\Sp(N)$ by the symmetrized
tensor product of $\Lam$ by itself
$(\Lam\otimes\Lam)_S$.
In each case we recover precisely the string states
suitable for Yang-Mills particles.

For exceptional groups, no construction is known that
produces string states in the adjoint representation
only, and it is widely believed that it is impossible
to obtain Yang-Mills particles from open strings with
these groups.

A non-Abelian gauge field can couple to the
boundaries of the open string, much in the same way
as this happens for Abelian gauge fields.
Here, however, we must include the coupling of the
charges of the gauge field $A_\mu^a$ to those
of the string.
Let us restrict again to the defining representation
of $U(N)$.
The coupling of $A_\mu^a$ is given as follows:
$$
S_A(\lam)=\int\nolimits_{B}d\tau
\lambar^i\left(\delta_i^j{d\over d\tau}+iA_\mu^a
\xdot^\mu(T^a)_i^j\right)\lam_j+\sigma(\tau)
  (\lambar^i\lam_i-1)\,\,.\eqno{(2.72)}
$$
The coupling to the gauge field $A_\mu^a$ may again
be used to construct the vertex operators for the
creation and annihilation of gauge states of the
string.
One finds upon quantization that the vertex opertor
for the $U(1)$ case is simply to be multiplied by the
matrices $(T^a)_i^j$:
$$
V^a(\eps,k)=T^aV(\eps,k)\,\,.
\eqno{(2.73)}
$$
As the internal degree of freedom and the matrices
$T^a$ commute with Virasoro, the physical state
conditions on $V^a$ are satisfied whenever they are
on $V(\eps,k)$.

It is instructive to examine the effects of these
degrees of freedom on interactions as well.
For simplicity, we restrict to $A_\mu^a=0$.
Charges then simply continue to be translated along the
boundary $B$ in a conserved way, as shown
schematically in the figure below.

\line{\hfill\epsfxsize=1.50in\epsfbox{fig9.eps}
\qquad
\vtop{\vskip-50pt\hbox{Interactions are only
allowed}
\hbox{if charges are conserved.}}\hfill}

\noindent
In a full string diagram, the effect will be to
include a trace over the matrices $T$, just as in
Yang-Mills Feynman rules, as shown schematically
below.

\line{\hfill\epsfxsize=1.50in\epsfbox{fig10.eps}
\quad
\vtop{\vskip-50pt\hbox{$\sim
\tr\,T^a\,T^b\,T^c\,T^d$}}\hfill}

\noindent
With the help of the Yang-Mills vertex operators
$V^a$ of (2.73) and (2.69), one recovers Yang-Mills
scattering amplitudes in the limit where momenta are
small compared to the Planch length.


\bye

