\input amstex
\magnification=1200
\documentstyle {amsppt}
\pagewidth{6.5 true in}
\pageheight{8.0 true in}
\nologo

\noindent
{\bf Fall Term Exam, N$^{\text{0}}$. 6}\qquad\qquad\qquad
\qquad\qquad\qquad\qquad\qquad(solution by D. Freed)
\smallskip
\hbox to \hsize{\hrulefill}
\bigskip
\noindent
{\bf Problem:}
\medskip
Let $\Cal O(x)$ be a local operator which annihilates the
vacuum:
$$
\Cal O(x)|\Omega\rangle=0.\tag1
$$
Show at least at a physical level of rigor that $\Cal
O(x)=0$.
\medskip
Now extend this slightly.  Suppose that for some distinct
points $x$ and $y\langle\Omega|\Cal O^*(y)\Cal
O(x)|\Omega\rangle$\hfill\break 
$=0$.  ($\Cal O^*$ is the hermitian
adjoint of $\Cal O$.)  Prove that $\Cal O=0$.
\bigskip
\noindent
{\bf Solution:}
\medskip
We recall that in a quantum field theory the correlation
function (better distribution) 
$F(x_1,\ldots,x_n)=\langle\Cal O_1(x_1)\Cal
O_2(x_2)\ldots\Cal O_n(x_n)\rangle$ is initially defined
for $x_1,\ldots,x_n$ distinct points in Minkowski space
$V$.  It is the boundary value of an {\it analytic}
function on a connected domain $\Cal D\subset(V_{\Bbb
C})^{\times n}$ where $\Cal D$ includes:
\roster
\item"{(1)}"  The subspace of $E^{\times n}$ consisting of $n$
distinct points where $E$ is Euclidean space; and
\smallskip
\item"{(2)}"  an open set in $V^{\times n}$ consisting of $n$
spacelike separated points.  
\endroster
\smallskip
\noindent
(The precise description of
some admissible open sets in (2) is given in the lemma of
Jost.) The restriction of $F$ to either of the two sets
described determines $F$.
\medskip
Now suppose $\Cal O$ is an operator which annihilates the
vacuum:  $\Cal O(x)|\Omega\rangle=0$.  We would like to
show that $\Cal O=0$, which means that any correlation
function involving $\Cal O$ vanishes identically.  For this
we restrict the correlation function to a set of the form
(2) above, and then
$$
\align
\langle\Cal O_1(x_1)\cdots\Cal O(x)\cdots\Cal
O_n(x_n)\rangle&=\langle\Cal O_1(x_1)\cdots\Cal
O_n(x_n)\Cal O(x)\rangle\\
&=\langle\Omega|\Cal O_1(x_1)\cdots\Cal O_n(x_n)\Cal
O(x)|\Omega\rangle\\
&=0.
\endalign
$$
\medskip
For the second assertion, we suppose $\langle\overline{\Cal
O(y)}\Cal O(x)\rangle=0$ for distinct points $x,y$ in
Euclidean space; we want to show that any correlation
function with $\Cal O$ vanishes (in Euclidean space).  Let
$\pi$ denote the perpendicular bisecting hyperplane to the
segments joining $x$ and $y$, and denote by $\overline p$
the reflection of $p$ in $\pi$.  Thus $y=\overline x$.
Now let
$$
\align
\Cal O'&=\Cal O_1(p_1)\cdots\Cal O_n(p_n),\\
\overline{{\Cal O}'}&=\overline{\Cal O_1(\overline p_1)}
\cdots\overline{\Cal O_n(\overline p_n)},
\endalign
$$
denote some product of operators.  We must show
$\langle\Cal O(x)\Cal O'\rangle=0$.
Consider
$$
\langle(\overline{\Cal O(\overline x)}+\lambda\Cal
O')\,\,(\Cal O(x)+\lambda\overline{\Cal O'})\rangle.
$$
By reflection positivity this is nonnegative.
Set
$$
\lambda=\frac{-\text{Re}\langle\Cal O(x)\Cal
O'\rangle}{\langle\Cal O'\overline{\Cal O'}\rangle}
$$
to conclude
$$
(\text{Re}\langle\Cal O(x)\Cal O'\rangle^2\leq\langle\overline
{\Cal O(\overline x)}\Cal O(x)\rangle\,\,\langle\Cal
O'\overline{\Cal O'}\rangle=0.
$$
Thus Re$\langle\Cal O(x)\Cal O'\rangle=0$.  The arguments
applied to $\sqrt{-1}\Cal O'$ show the imaginary part
vanishes as well.
\bye
