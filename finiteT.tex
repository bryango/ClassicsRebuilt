\section{Theories at Finite Temperature}
\label{FiniteT}

As discussed in section \ref{tests}, 
the quantities that can be most successfully compared between gauge
theory and string theory are those with some protection from
supersymmetry and/or conformal invariance --- for instance, dimensions of
chiral primary operators.  Finite temperature breaks both
supersymmetry and conformal invariance, and the insights we gain from
examining the $T>0$ physics will be of a more qualitative nature.
They are no less interesting for that: we shall see in
section~\ref{ConstT} how the entropy of near-extremal D3-branes comes
out identical to the free field theory prediction up to a factor of
a power of $4/3$; 
then in section~\ref{TPhaseT} we explain how a phase transition
studied by Hawking and Page in the context of quantum gravity is
mapped into a confinement-deconfinement transition in the gauge theory,
driven by finite-size effects; and in section~\ref{adsqcd} we will
summarize the attempts to use holographic duals of finite-temperature
field theories to learn about pure gauge theory at zero temperature
but in one lower dimension.



\subsection{Construction}
\label{ConstT}

The gravity solution describing the gauge theory at finite temperature
can be obtained by starting from the general black three-brane solution
(\ref{solution}) and taking the decoupling limit of section 
\ref{correspondence} keeping the energy density above extremality finite.
The resulting metric can be written as
\eqn{NearDThree}{
\eqalign{
   ds^2 &=R^2 \left[ u^2( - h dt^2 + dx_1^2 + dx_2^2 + dx_3^2 ) 
+ { d u^2 \over h  u^2} + d \Omega_5^2 \right]
\cr
 h & = 1 - { u_0^4 \over u^4} ~,~~~~~~~ u_0 = \pi T .
}}
%The near-extremal D3-brane metric, describing D3-branes with a finite
%energy density, is 
%  \eqn{NearDThree}{\eqalign{
%   ds^2 &= f^{-1/2} \left( -h dt^2 + dx_1^2 + dx_2^2 + dx_3^2 \right) +
%    f^{1/2} \left( h^{-1} dr^2 + r^2 d\Omega_5^2 \right),  \cr
%   f &= 1 + {R^4 \over r^4},  \qquad  h = 1 - {r_0^4 \over r^4} \ .
%  }}
%The arguments of the previous sections 
%suggest that the near-horizon limit of this metric
%should correspond to the field theory at a finite energy density.
It will often be useful to Wick rotate by setting $t_E = it$, and use
the relation between the finite temperature theory and the Euclidean
theory with a compact time direction.  

The first computation which indicated that finite-temperature $U(N)$
Yang-Mills theory might be a good description of the microstates of $N$
coincident D3-branes was the calculation of the entropy
\cite{Gubser:1996de,sunp}.  On the supergravity side, the entropy of
near-extremal D3-branes is just the usual Bekenstein-Hawking result, $S =
A/{4 G_N}$, and it is expected to be a reliable guide to the entropy of the
gauge theory at large $N$ and large $g_{YM}^2 N$.  There is no problem on
the gauge theory side in working at large $N$, but large $g_{YM}^2 N$ at
finite temperature is difficult indeed.  The analysis of
\cite{Gubser:1996de} was limited to a free field computation in the field
theory, but nevertheless the two results for the entropy agreed up to a
factor of a power of $4/3$.  In the canonical ensemble, where temperature
and volume are the independent variables, one identifies the field theory
volume with the world-volume of the D3-branes, and one sets the field
theory temperature equal to the Hawking temperature in supergravity.  The
result is
  \eqn{CanS}{\eqalign{
   F_{SUGRA} &= -{\pi^2 \over 8} N^2 V T^4,  \cr
   F_{SYM} &= {4 \over 3} F_{SUGRA} \ .
  }}
 The supergravity result is at leading order in $l_s/R$, and it would
acquire corrections suppressed by powers of $T R$ if we had considered the
full D3-brane metric rather than the near-horizon limit, \NearDThree.
These corrections do not have an interpretation in the context of CFT
because they involve $R$ as an intrinsic scale.  Two equivalent methods to
evaluate $F_{SUGRA}$ are a) to use $F = E - TS$ together with standard
expressions for the Bekenstein-Hawking entropy, the Hawking temperature,
and the ADM mass; and b) to consider the gravitational action of the
Euclidean solution, with a periodicity in the Euclidean time direction
(related to the temperature) which eliminates a conical deficit angle at
the horizon.\footnote{The result of \cite{Gubser:1996de}, $S_{SYM} =
(4/3)^{1/4} S_{SUGRA}$, differs superficially from \CanS, but it is only
because the authors worked in the microcanonical ensemble: rather than
identifying the Hawking temperature with the field theory temperature, the
ADM mass above extremality was identified with the field theory energy.}

The $4/3$ factor  is a long-standing puzzle into which we still have
only qualitative insight.  The gauge theory computation was performed
at zero 't~Hooft coupling, whereas the supergravity is supposed to be
valid at strong 't~Hooft coupling, and unlike in the 1+1-dimensional
case where the entropy is essentially fixed by the central charge,
there is no non-renormalization theorem for the coefficient of $T^4$ in the
free energy.  Indeed, it was suggested in \cite{Gubser:1998nz} that
the leading term in the $1/N$ expansion of $F$ has the form
  \eqn{InterpolateS}{
   F = -f(g_{YM}^2 N) {\pi^2 \over 6} N^2 V T^4,
  }
 where $f(g_{YM}^2 N)$ is a function which smoothly interpolates between a
weak coupling limit of $1$ and a strong coupling limit of $3/4$.  It was
pointed out early \cite{Horowitz:1997nw} that the quartic potential
$g_{YM}^2 \tr [\phi^I,\phi^J]^2$ in the ${\cal N}=4$ Yang-Mills action
might be expected to freeze out more and more degrees of freedom as the
coupling was increased, which would suggest that $f(g_{YM}^2 N)$ is
monotone decreasing.  An argument has been given \cite{Itzhaki:1999ge},
based on the non-renormalization of the two-point function of the stress
tensor, that $f(g_{YM}^2 N)$ should remain finite at strong coupling.

The leading corrections to the limiting value of $f(g_{YM}^2 N)$ at
strong and weak coupling were computed in \cite{Gubser:1998nz} and
\cite{Fotopoulos:1999es}, respectively.  The results are 
  \eqn{WeakStrong}{\seqalign{\span\TL & \span\TR \qquad & \span\TT}{
   f(g_{YM}^2 N) &= 1 - {3 \over 2\pi^2} g_{YM}^2 N + \ldots
     & for small $g_{YM}^2 N$,  \cr
   f(g_{YM}^2 N) &= {3\over 4} + {45 \over 32} 
    {\zeta(3) \over (g_{YM}^2 N)^{3/2}} + \ldots
     & for large $g_{YM}^2 N$.
  }}
 The weak coupling result is a straightforward although somewhat tedious
application of the diagrammatic methods of perturbative finite-temperature
field theory.  The constant term is from one loop, and the leading
correction is from two loops.  The strong coupling result follows from
considering the leading $\alpha'$ corrections to the supergravity action.
The relevant one involves a particular contraction of four powers of the
Weyl tensor.  It is important now to work with the Euclidean solution, and
one restricts attention further to the near-horizon limit.  The Weyl
curvature comes from the non-compact part of the metric, which is no longer
$AdS_5$ but rather the AdS-Schwarzschild solution which we will discuss in
more detail in section~\ref{TPhaseT}.  The action including the
   $\alpha'$ corrections no longer has the
Einstein-Hilbert form, and correspondingly the Bekenstein-Hawking
prescription no longer agrees with the free energy computed as $\beta I$
where $I$ is the Euclidean action.  In keeping with the basic prescription
for computing Green's functions, where a free energy in field theory is
equated (in the appropriate limit)
with a supergravity action, the relation $I = \beta F$ is regarded
as the correct one.
(See \cite{Wald:1993nt}.)
It has been conjectured that the interpolating function $f(g_{YM}^2
N)$ is not smooth, but exhibits some phase transition at a finite
value of the 't~Hooft coupling.  We regard this as an unsettled
question.  The arguments in \cite{Li:1999kd,Gao:1998ww} 
seem as yet incomplete.  In
particular, they rely on analyticity properties of the perturbation
expansion which do not seem to be proven for finite temperature field
theories.



\subsection{Thermal Phase Transition}
\label{TPhaseT}

The holographic prescription of \cite{Gubser:1998bc,Witten:1998qj}, applied at
large $N$ and $g_{YM}^2 N$
where loop and stringy corrections are negligible, involves extremizing
the supergravity action subject to particular
asymptotic boundary conditions.  We
can think of this as the saddle point approximation to the path
integral over supergravity fields.  That path integral is ill-defined
because of the non-renormalizable nature of supergravity.  String
amplitudes (when we can calculate them) render on-shell quantities
well-defined.  Despite the conceptual difficulties we can use some
simple intuition about path integrals to illustrate an important point
about the AdS/CFT correspondence: namely, there can be more than one
saddle point in the range of integration, and when there is we should
sum $e^{-I_{SUGRA}}$ over the classical configurations to obtain the
saddle-point approximation to the gauge theory partition function.
Multiple classical configurations are possible because of the general
feature of boundary value problems in differential equations: there
can be multiple solutions to the classical equations satisfying the
same asymptotic boundary conditions.  The solution which globally
minimizes $I_{SUGRA}$ is the one that dominates the path integral.

When there are two or more solutions competing to minimize
$I_{SUGRA}$, there can be a phase transition between them.  An example
of this was studied in \cite{Hawking:1983dh} long before the AdS/CFT
correspondence, and subsequently resurrected, generalized, and
reinterpreted in \cite{Witten:1998qj,Witten:1998zw} as a
confinement-deconfinement transition in the gauge theory.  Since the
qualitative features are independent of the dimension, we will restrict
our attention to $AdS_5$.  It is worth noting however that if the
$AdS_5$ geometry is part of a string compactification, it doesn't
matter what the internal manifold is except insofar as it fixes the
cosmological constant, or equivalently the radius $R$ of anti-de
Sitter space.

There is an embedding of the Schwarzschild black hole solution into
anti-de Sitter space which extremizes the action
  \eqn{EinAct}{
   I = -{1 \over 16 \pi G_5} 
    \int d^5 x \, \sqrt{g} \left( {\cal R} + {12 \over R^2} \right) \ .
  }
 Explicitly, the metric is
  \eqn{AdSSch}{\eqalign{
   ds^2 &= f dt^2 + {1 \over f} dr^2 + r^2 d\Omega_3^2,  \cr
   f &= 1 + {r^2 \over R^2} - {\mu \over r^2} \ .
  }}
 The radial variable $r$ is restricted to $r \geq r_+$, where $r_+$ is the
largest root of $f=0$.  The Euclidean time is periodically identified, 
$t \sim t +
\beta$, in order to eliminate the conical singularity at $r = r_+$.  This
requires
  \eqn{HawkingBeta}{
   \beta = {2\pi R^2 r_+ \over 2 r_+^2 + R^2} \ .
  }
 Topologically, this space is $S^3 \times B^2$, and the boundary is $S^3
\times S^1$ (which is the relevant space for the field theory on $S^3$
with finite temperature).  
We will call this space $X_2$.  Another space with the same
boundary which is also a local extremum of \EinAct\ is given by the metric
in \AdSSch\ with $\mu = 0$ and again with periodic time.  This space, which
we will call $X_1$, is not only metrically distinct from the first (being
locally conformally flat), but also topologically $B^4 \times S^1$ rather
than $S^3 \times B^2$.  Because the $S^1$ factor is not simply connected,
there are two possible spin structures on $X_1$, corresponding to thermal
(anti-periodic) or supersymmetric (periodic) boundary conditions on
fermions.  In contrast, $X_2$ is simply connected and hence admits a unique
spin structure, corresponding to thermal boundary conditions.  For the
purpose of computing the twisted partition function, $\tr (-1)^F e^{-\beta
H}$, in a saddle-point approximation, only $X_1$ contributes.  But, $X_1$
and $X_2$ make separate saddle-point contributions to the usual thermal
partition function, $\tr e^{-\beta H}$, and the more important one is the
one with the smaller Euclidean action.

Actually, both $I(X_1)$ and $I(X_2)$ are infinite, so to compute
$I(X_2)-I(X_1)$ a regulation scheme must be adopted.  The one used in
\cite{Witten:1998zw,Gubser:1998nz} is to cut off both
$X_1$ and $X_2$ at a definite coordinate radius $r=R_0$.  For $X_2$, the
elimination of the conical deficit angle at the horizon fixes the period of
Euclidean time; but for $X_1$, the period is arbitrary.  In order to make
the comparison of $I(X_1)$ and $I(X_2)$ meaningful, we fix the period of
Euclidean time on $X_1$ so that the proper circumference of the $S_1$ at
$r=R_0$ is the same as the proper length on $X_2$ of an orbit of the Killing
vector $\partial/\partial t$, also at $r=R_0$.  In the limit $R_0\to\infty$,
one finds
  \eqn{IDiff}{
   I(X_2)-I(X_1) = {\pi^2 r_+^3 (R^2 - r_+^2) \over
    4 G_5 (2 r_+^2 + R^2)} \ ,
  }
 where again $r_+$ is the largest root of $f=0$.  The fact that \IDiff\ (or
more precisely its $AdS_4$ analog) can change its sign was interpreted in
\cite{Hawking:1983dh} as indicating a phase transition between a black hole
in $AdS$ and a thermal gas of particles in $AdS$ (which is the natural
interpretation of the space $X_1$).  The black hole is the
thermodynamically favored state when the horizon radius $r_+$ exceeds the
radius of curvature $R$ of $AdS$. In the gauge theory we interpret this
transition as a confinement-deconfinement transition.  Since the theory is
conformally invariant, the transition temperature must be proportional to
the inverse radius of the space $S^3$ which the field theory lives on.
Similar transitions, and also local thermodynamic instability due to
negative specific heats, have been studied in the context of spinning
branes and charged black holes in
\cite{Gubser:1998jb,Landsteiner:1999gb,Cai:1998ji,
Cvetic:1999ne,Chamblin:1999tk,Cvetic:1999rb,Caldarelli:1999ar}.  Most of
these works are best understood on the CFT side as explorations of exotic
thermal phenomena in finite-temperature gauge theories.  Connections with
Higgsed states in gauge theory are clearer in
\cite{Kraus:1998hv,Tseytlin:1998cq}.  The relevance to confinement is
explored in \cite{Cvetic:1999rb}.  See also
\cite{Birmingham:1998nr,Louko:1998hc,Hawking:1998kw,Peet:1998cr} for other
interesting contributions to the finite temperature literature.

Deconfinement at high temperature can be characterized by a spontaneous
breaking of the center of the gauge group. In our case the gauge group is
$SU(N)$ and its center is $\IZ_N$.
The order parameter for the breaking of the center is the
expectation value of the Polyakov (temporal) loop $\langle W(C) \rangle$.
The boundary of the spaces $X_1,X_2$ is $S^3 \times S^1$, and 
the path $C$ wraps around the circle.
An element of the center $g \in \IZ_N$ acts on the Polyakov loop
by  $\langle W(C) \rangle \rightarrow g \langle W(C) \rangle$.
The expectation value of the Polyakov loop measures the change of the
free energy of the system $F_q(T)$ induced by the presence
of the external charge $q$, 
$\langle W(C) \rangle \sim exp \left(-F_q(T)/T \right)$.
In a confining phase  $F_q(T)$ is infinite and therefore  
$\langle W(C) \rangle  = 0$.
In the deconfined phase  $F_q(T)$ is finite and therefore  
$\langle W(C) \rangle  \neq 0$.

As discussed in section \ref{wilsonloops}, 
in order to compute $\langle W(C) \rangle$ we have to 
evaluate the partition function of strings with a 
worldsheet $D$ that is bounded
by the loop $C$. 
Consider first the low temperature phase. The relevant space 
is $X_1$ which, as discussed
above,  has
the topology $B^4 \times S^1$.
The contour $C$ wraps the circle and is not homotopic to zero in $X_1$.
Therefore $C$ is not a boundary of any $D$, which immediately
implies that  $\langle W(C) \rangle  = 0$.
This is the expected behavior at low temperatures (compared to the
inverse radius of the $S^3$), where the center
of the gauge group is not broken.

For the high temperature phase the relevant space is $X_2$, which has the
topology $S^3 \times B^2$. The contour $C$ is now a boundary of a
string worldsheet $D=B^2$ (times
a point in $S^3$). This seems to be in agreement with the fact that
in the   high temperature phase $\langle W(C) \rangle  \neq 0$
and the center
of the gauge group is broken. It was pointed out in \cite{Witten:1998zw}
that there is a subtlety with this argument, since the
center should not be broken in finite volume ($S^3$), but only
in the infinite volume limit ($\IR^3$).
Indeed, the solution $X_2$ is not unique
and we can add to it an expectation value for
the integral of the NS-NS 2-form field $B$ on $B^2$, with vanishing
field strength. This is an angular parameter $\psi$
with period $2 \pi$, which contributes $i\psi$ to the string
worldsheet action. The string theory partition function
%i.e. the integral over string
%worldsheets $D$, 
includes now an integral over all values of
$\psi$, making  $\langle W(C) \rangle  = 0$ on $S^3$.
In contrast, on $\IR^3$ one integrates over the local fluctuations
of $\psi$ but not over its vacuum expectation value. Now
$\langle W(C) \rangle  \neq 0$ and depends on the value of $\psi \in U(1)$,
which may be understood as the dependence on the center $\IZ_N$ in the
large $N$ limit.
Explicit computations of Polyakov loops at finite temperature 
were done in \cite{Rey:1998wp,Brandhuber:1998bs}.

In \cite{Witten:1998zw} the Euclidean black hole solution \AdSSch\ 
was suggested to
be holographically dual to a theory related to
pure QCD in three dimensions.  In the large
volume limit
the solution corresponds to the ${\cal N}=4$ gauge theory on $\IR^3
\times S^1$ with thermal boundary conditions, and when the $S^1$ is made
small (corresponding to high temperature $T$) the theory at distances
larger than $1/T$ effectively reduces to pure Yang-Mills on
$\IR^3$.  Some of the non-trivial successes of this approach to QCD will be
discussed in section \ref{adsqcd}.


