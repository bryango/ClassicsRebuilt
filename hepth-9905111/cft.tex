
\section{Conformal Field Theories} 
\label{cft}

Symmetry principles, and in particular Lorentz and \Poincare
invariance, play a major role in our understanding of quantum field
theory. It is natural to look for possible generalizations of
\Poincare invariance in the hope that they may play some role in
physics; in \cite{Coleman:1967ad} 
it was argued that for theories with a non-trivial
S-matrix there are no such bosonic generalizations. An interesting
generalization of \Poincare invariance is the addition of a scale
invariance symmetry linking physics at different scales (this is
inconsistent with the existence of an S-matrix since it does not allow
the standard definition of asymptotic states). Many interesting field
theories, like Yang-Mills theory in four dimensions, are
scale-invariant; generally this scale invariance does not extend to
the quantum theory (whose definition requires a cutoff which
explicitly breaks scale invariance) but in some special cases (such as
the $d=4,\cn=4$ supersymmetric Yang-Mills theory) it does, and even
when it does not (like in QCD) it can still be a useful tool, leading
to relations like the Callan-Symanzik equation. It was realized in the
past 30 years that field theories generally exhibit a renormalization
group flow from some scale-invariant (often free) UV fixed point
to some scale-invariant (sometimes trivial) IR fixed point, and
statistical mechanics systems also often have non-trivial IR
scale-invariant fixed points. Thus, studying scale-invariant theories
is interesting for various physical applications.

It is widely believed that unitary interacting scale-invariant
theories are always invariant under the full conformal group, which is
a simple group including scale invariance and \Poincare
invariance. This has only been proven in complete generality for two
dimensional field theories
\cite{Zamolodchikov:1986gt,Polchinski:1988dy}, but there are no known
counter-examples. In this section we will review the conformal group
and its implications for field theories, focusing on implications
which will be useful in the context of the AdS/CFT
correspondence. General reviews on conformal field theories may be
found in \cite{Mack:1988nf,Fradkin:1996is,Fradkin:1997df} 
and references therein.

\subsection{The Conformal Group and Algebra}
\label{conformal_group}

The conformal group is the group of transformations which preserve the
form of the metric up to an arbitrary scale factor, $g_{\mu
\nu}(x) \to \Omega^2(x) g_{\mu \nu}(x)$ (in this section greek letters
will correspond to the space-time coordinates,
$\mu,\nu=0,\cdots,d-1$). It is the minimal group that includes the
\Poincare group as well as the inversion symmetry $x^\mu \to x^\mu /
x^2$.

The conformal group of Minkowski space\footnote{More precisely, some of
these transformations can take finite points in Minkowski space to
infinity, so they should be defined on a compactification of Minkowski
space which includes points at infinity.} is generated by the
\Poincare transformations, the scale transformation
\eqn{scale}{x^\mu \to \lambda x^\mu,}
and the special conformal transformations
\eqn{specconf}{x^\mu \to {{x^\mu + a^\mu x^2} \over {1 + 2 x^\nu a_\nu
+ a^2 x^2}}.}
We will denote the generators of these transformations by $M_{\mu
\nu}$ for the Lorentz transformations, $P_\mu$ for translations, $D$
for the scaling transformation \eno{scale} and $K_\mu$ for the
special conformal transformations \eno{specconf}. The vacuum of a
conformal theory is annihilated by all of these generators.
They obey the conformal algebra
\eqn{algebra}{\eqalign{[M_{\mu\nu},P_\rho]=-i(\eta_{\mu\rho}P_\nu-
\eta_{\nu\rho}P_\mu); &\qquad
[M_{\mu\nu},K_\rho]=-i(\eta_{\mu\rho}K_\nu-\eta_{\nu\rho}K_\mu); \cr
[M_{\mu\nu},M_{\rho\sigma}]=-i\eta_{\mu\rho}M_{\nu\sigma} 
\pm permutations; &\qquad
[M_{\mu\nu},D]=0; \qquad
[D,K_\mu]=iK_\mu; \cr
[D,P_\mu]=-iP_\mu; &\qquad
[P_\mu,K_\nu]=2iM_{\mu\nu}-2i\eta_{\mu\nu}D, }}
with all other commutators vanishing.
This algebra is isomorphic to the algebra of $SO(d,2)$, and can be put
in the standard form of the $SO(d,2)$ algebra (with signature
$-,+,+,\cdots,+,-$) with generators $J_{ab}$ ($a,b=0,\cdots,d+1$)
by defining
\eqn{bigso}{J_{\mu\nu} = M_{\mu\nu};\qquad J_{\mu d} = {1\over 2}
(K_\mu - P_\mu);\qquad J_{\mu (d+1)} = {1\over 2}(K_\mu +
P_\mu);\qquad J_{(d+1)d} = D.}
For some applications it is useful to study the conformal theory in
Euclidean space; in this case the conformal group is 
$SO(d+1,1)$,\footnote{Strictly speaking, $SO(d+1,1)$ is the connected 
component of the conformal group which includes the identity, and it does 
not include $x^\mu \to x^\mu / x^2$.  We will hereafter ignore such 
subtleties.}
 and
since $\IR^d$ is conformally equivalent to $S^d$ the field theory on
$\IR^d$ (with appropriate boundary conditions at infinity) is isomorphic
to the theory on $S^d$. Much of what we say below will apply also to
the Euclidean theory.

In the special case of $d=2$ the conformal group is larger, and in
fact it is infinite dimensional. The special aspects of this case will
be discussed in chapter \ref{ChapAdS3} where they will be needed.

\subsection{Primary Fields, Correlation Functions, and Operator Product
Expansions}
\label{primaries}

The interesting representations (for physical applications) of the
conformal group involve operators (or fields) which are eigenfunctions
of the scaling operator $D$ with eigenvalue $-i\Delta$ ($\Delta$ is
called the {\it scaling dimension} of the field). This means that
under the scaling transformation \eno{scale} they transform as
$\phi(x) \to \phi^\prime(x) = \lambda^{\Delta} \phi(\lambda x)$. 
The commutation
relations \eno{algebra} imply that the operator $P_\mu$
raises the dimension of the field, while the operator $K_\mu$ lowers
it. In unitary field theories there is a lower bound on the dimension
of fields (for scalar fields it is $\Delta \geq (d-2)/2$ which is the
dimension of a free scalar field), and, therefore, each representation
of the conformal group which appears must have some operator of lowest
dimension, which must then be annihilated by $K_\mu$ (at
$x=0$). Such operators are called {\it primary operators}. The action
of the conformal group on such operators is given by \cite{Mack:1969rr}
\eqn{confaction}{\eqalign{
[P_\mu, \Phi(x)] &= i\del_\mu \Phi(x),\cr
[M_{\mu\nu}, \Phi(x)] &= [i(x_\mu \del_\nu - x_\nu \del_\mu) +
\Sigma_{\mu\nu}] \Phi(x), \cr
[D,\Phi(x)] &= i(-\Delta + x^\mu \del_\mu) \Phi(x),\cr
[K_\mu, \Phi(x)] &= [i(x^2\del_\mu - 2x_\mu x^\nu \del_\nu + 2x_\mu
\Delta) - 2x^\nu \Sigma_{\mu\nu}] \Phi(x), }}
where $\Sigma_{\mu\nu}$ are the matrices of a finite dimensional
representation of the Lorentz group, acting on the indices of the
$\Phi$ field. The representations of the conformal group corresponding
to primary operators are classified by the Lorentz representation and
the scaling dimension $\Delta$ (these determine the Casimirs of the
conformal group).  These representations include the primary field and
all the fields which are obtained by acting on it with the generators
of the conformal group (specifically with $P_\mu$).  Since the
operators in these representations are eigenfunctions of $D$, they
cannot in general 
be eigenfunctions of the Hamiltonian $P_0$ or of the mass
operator $M^2 = -P^\mu P_\mu$ (which is a Casimir operator of the
\Poincare group but not of the conformal group); in fact, they have a
continuous spectrum of $M^2$ ranging from $0$ to $\infty$ (there are
also representations corresponding to free massless fields which have
$M^2=0$).

Another possible classification of the representations of the
conformal group is in terms of its maximal compact subgroup, which is
$SO(d)\times SO(2)$. The generator of $SO(2)$ is $J_{0(d+1)} = {1\over
2}(K_0+P_0)$, and the representations of the conformal group described
above may be decomposed into representations of this subgroup. This is
useful in particular for the oscillator constructions of the
representations of superconformal algebras 
\cite{Gunaydin:1982yq,Bars:1983ep,Gunaydin:1985fk,Gunaydin:1985vz,
Gunaydin:1985wc,Gunaydin:1986tc,Gunaydin:1987hb}, which we will not
describe in detail here (see \cite{Minic:1999eq} for a recent review).
This subgroup is also useful in the radial quantization of the
conformal field theory on $S^{d-1}\times \IR$, which will be related to
AdS space in global coordinates.

Since the conformal group is much larger than the \Poincare group, it
severely restricts the correlation functions of primary fields, which
must be invariant under conformal transformations. 
%We will discuss
%here correlation functions in Minkowski space; 
It has been shown by
Luscher and Mack \cite{Luscher:1974ez} that the Euclidean Green's
functions of a CFT may be analytically continued to Minkowski space,
and that the resulting Hilbert space carries a unitary representation
of the Lorentzian conformal group. The formulas we will write here for
correlation functions apply both in Minkowski and in Euclidean space.
It is easy to show using the
conformal algebra that the 2-point functions of fields of different
dimension vanish, while for a single scalar field of scaling dimension
$\Delta$ we have
\eqn{twopoint}{\vev{\phi(0)\phi(x)} \propto 
{1\over |x|^{2\Delta}} \equiv {1\over (x^2)^\Delta}.}
3-point functions are also determined (up to a constant) by the
conformal group to be of the form
\eqn{threepoint}{\vev{\phi_i(x_1) \phi_j(x_2) \phi_k(x_3)} = {c_{ijk}
\over 
|x_1-x_2|^{\Delta_1+\Delta_2-\Delta_3}
|x_1-x_3|^{\Delta_1+\Delta_3-\Delta_2}
|x_2-x_3|^{\Delta_2+\Delta_3-\Delta_1}}. }
Similar expressions (possibly depending on additional constants) arise
for non-scalar fields.
With 4 independent $x_i$ one can construct two combinations of the $x_i$
(known as harmonic ratios)
which are conformally invariant, so the correlation function can be any
function of these combinations; for higher $n$-point functions there are
more and more independent functions which can appear in the
correlation functions. Many other properties of conformal field
theories are also easily determined using the conformal invariance;
for instance, their equation of state is necessarily of the form $S =
c V (E / V)^{(d-1)/d}$ for some constant $c$.

The field algebra of any conformal field theory includes the
energy-momentum tensor $T_{\mu\nu}$ which is an operator of dimension
$\Delta=d$; the Ward identities of the conformal algebra relate
correlation functions with $T$ to correlation functions without
$T$. Similarly, whenever there are global symmetries, their
(conserved) currents $J_\mu$ are necessarily operators of dimension
$\Delta=d-1$. The scaling dimensions of other operators are not
determined by the conformal group, and generally they receive quantum
corrections. For any type of field there is, however, a lower bound on
its dimension which follows from unitarity; as mentioned above, for
scalar fields the bound is $\Delta \geq (d-2)/2$, where equality can
occur only for free scalar fields.

A general property of local field theories is the existence of an {\it
operator product expansion} (OPE). As we bring two operators
$\co_1(x)$ and $\co_2(y)$ to the same point, their product creates a
general local disturbance at that point, which may be expressed as a
sum of local operators acting at that point; in general all operators
with the same global quantum numbers as $\co_1 \co_2$ may appear. The
general expression for the OPE is $\co_1(x) \co_2(y) \to \sum_n
C_{12}^n(x-y) \co_n(y)$, where this expression should be understood as
appearing inside correlation functions, and the coefficient functions
$C_{12}^n$ do not depend on the other operators in the correlation
function (the expression is useful when the distance to all other
operators is much larger than $|x-y|$). In a conformal theory, the
functional form of the OPE coefficients is determined by conformal
invariance to be $C_{12}^n(x-y) = c_{12}^n /
|x-y|^{\Delta_1+\Delta_2-\Delta_n}$, where the constants $c_{12}^n$
are related to the 3-point functions described above. The leading
terms in the OPE of the energy-momentum tensor with primary fields are
determined by the conformal algebra. For instance, for a scalar
primary field $\phi$ of dimension $\Delta$ in four dimensions,
\eqn{emope}{T_{\mu \nu}(x) \phi(0) \propto \Delta \phi(0) \del_\mu
\del_\nu({1\over {x^2}}) + \cdots.}

One of the basic properties of conformal field theories is the
one-to-one correspondence between local operators $\co$ and states
$|\co \rangle$ in the radial quantization of the theory. In radial
quantization the time coordinate is chosen to be the radial direction
in $\IR^d$, with the origin corresponding to past infinity, so that
the field theory lives on $\IR \times S^{d-1}$. The Hamiltonian in
this quantization is the operator $J_{0(d+1)}$ mentioned above. An
operator $\co$ can then be mapped to the state $|\co \rangle =
\lim_{x\to 0} \co(x) |0\rangle$. Equivalently, the state may be viewed
as a functional of field values on some ball around the origin, and
then the state corresponding to $\co$ is defined by a functional
integral on a ball around the origin with the insertion of the
operator $\co$ at the origin. The inverse mapping of states to
operators proceeds by taking a state which is a functional of field
values on some ball around the origin and using conformal invariance
to shrink the ball to zero size, in which case the insertion of the
state is necessarily equivalent to the insertion of some local
operator.

\subsection{Superconformal Algebras and Field Theories}
\label{superconfalg}

Another interesting generalization of the \Poincare algebra is the
supersymmetry algebra, which includes additional fermionic operators
$Q$ which anti-commute to the translation operators $P_\mu$. It is
interesting to ask whether supersymmetry and the conformal group can
be joined together to form the largest possible simple algebra
including the \Poincare group; it turns out that in some dimensions and
for some numbers of supersymmetry charges this is indeed
possible. The full classification of superconformal algebras was
given by Nahm \cite{Nahm:1978tg}; 
it turns out that superconformal algebras exist
only for $d\leq 6$. In addition to the generators of the conformal
group and the supersymmetry, superconformal algebras include two other
types of generators. There are fermionic generators $S$ (one for each
supersymmetry generator) which arise in the commutator of $K_\mu$ with
$Q$, and there are (sometimes) R-symmetry generators forming some Lie
algebra, which appear in the anti-commutator of $Q$ and
$S$ (the generators $Q$ and $S$ are in the fundamental representation
of this Lie algebra). 
Schematically (suppressing all indices), 
the commutation relations of the superconformal
algebra include, in addition to \eno{algebra}, the relations
\eqn{superalgebra}{\eqalign{
[D,Q] &= -{i\over 2}Q;\qquad [D,S] = {i\over 2}S;\qquad [K,Q] \simeq
S;\qquad [P,S] \simeq Q;\cr
\{Q,Q\} &\simeq P;\qquad \{S,S\} \simeq K;\qquad \{Q,S\} \simeq
M+D+R. }}
The exact form of the commutation relations is different for different
dimensions (since the spinorial representations of the conformal group
behave differently) and for different R-symmetry groups,
and we will not write them explicitly here.

For free field theories without gravity, which do not include fields
whose spin is bigger than one, the maximal possible number of
supercharges is 16 (a review of field theories with this number of
supercharges appears in \cite{Seiberg:1997ax}); it is believed that
this is the maximal possible number of supercharges also in
interacting field theories. Therefore, the maximal possible number of
fermionic generators in a field theory superconformal algebra is
32. Superconformal field theories with this number of supercharges
exist only for $d=3,4,6$ ($d=1$ may also be possible but there are no
known examples). For $d=3$ the R-symmetry group is $Spin(8)$ and the
fermionic generators are in the $\bf(4,8)$ of $SO(3,2)\times Spin(8)$;
for $d=4$ the R-symmetry group is $SU(4)$ and\footnote{Note that this
is different from the other $\cn$-extended superconformal algebras in
four dimensions which have a $U(\cn)$ R-symmetry.} the fermionic
generators are in the $\bf(4,4)+\bf({\overline 4},{\overline 4})$ of
$SO(4,2)\times SU(4)$; and for $d=6$ the R-symmetry group is
$Sp(2)\simeq SO(5)$ and the fermionic generators are in the $\bf(8,4)$
representation of $SO(6,2)\times Sp(2)$.

Since the conformal algebra is a subalgebra of the superconformal
algebra, representations of the superconformal algebra split into
several representations of the conformal algebra. Generally a primary
field of the superconformal algebra, which is (by definition)
annihilated (at $x=0$) by the generators $K_\mu$ and $S$, will include
several primaries of the conformal algebra, which arise by acting with
the supercharges $Q$ on the superconformal primary field. The
superconformal algebras have special representations corresponding to
{\it chiral primary operators}, which are primary operators which are
annihilated by some combination of the supercharges. These
representations are smaller than the generic representations,
containing less conformal-primary fields. A special property of chiral
primary operators is that their dimension is uniquely determined by
their R-symmetry representations and cannot receive any quantum
corrections. This follows by using the fact that all the $S$
generators and some of the $Q$ generators annihilate the field, and
using the $\{Q,S\}$ commutation relation to compute the eigenvalue of
$D$ in terms of the Lorentz and R-symmetry representations
\cite{Kac:1977hp,Dobrev:1987qz,Dobrev:1985qv,Seiberg:1997ax,
Minwalla:1998ka}. The
dimensions of non-chiral primary fields of the same representation are
always strictly larger than those of the chiral primary fields. A
simple example is the $d=4,\cn=1$ superconformal algebra (which has a
$U(1)$ R-symmetry group); in this case a chiral multiplet (annihilated
by $\overline Q$) which is a primary is also a chiral primary, and the
algebra can be used to prove that the dimension of the scalar
component of such multiplets is $\Delta={3\over 2}R$ where $R$ is the
$U(1)$ R-charge. A detailed description of the structure of chiral
primaries in the $d=4,\cn=4$ algebra will appear in section \ref{tests}.

When the R-symmetry group is Abelian, we find a bound of the form
$\Delta \geq a|R|$ for some constant $a$, ensuring that there is no
singularity in the OPE of two chiral ($\Delta=aR$) or anti-chiral
($\Delta=a|R|=-aR$) operators. On the other hand, when the R-symmetry
group is non-Abelian, singularities can occur in the OPEs of chiral
operators, and are avoided only when the product lies in particular
representations.








