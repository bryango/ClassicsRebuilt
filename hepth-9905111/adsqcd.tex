\section{QCD} 
\label{adsqcd}

The proposed extension of the duality conjecture between field
theories and superstring theories to field theories at finite 
temperature, as described in section \ref{FiniteT}, opens up the
exciting possibility of studying the physically relevant non
supersymmetric gauge theories.  Of particular interest are non
supersymmetric gauge theories that exhibit asymptotic freedom and
confinement.  In this section, we will discuss an approach to studying
pure (without matter fields) QCD$_p$ in $p$ dimensions using a dual
superstring description.  We will be discussing mainly the cases
$p=3,4$.

The approach proposed by Witten \cite{Witten:1998zw} was to start with
a maximally supersymmetric gauge theory on the $p+1$ dimensional
worldvolume of $N$ $Dp$ branes. One then compactifies the
supersymmetric theory on a circle of radius $R_0$ and imposes
anti-periodic boundary conditions for the fermions around the
circle. Since the fermions do not have zero frequency modes around the
circle they acquire a mass $m_f \sim 1/R_0$.  The scalars then acquire
a mass from loop diagrams, and at energies much below $1/R_0$ they
decouple from the system. The expected effective theory at large
distances compared to the radius of the circle is pure QCD in $p$
dimensions.  Note that a similar approach was discussed in the
treatment of  gauge theories at finite
temperature $T$ in section \ref{FiniteT}, where the radius of the circle is
proportional to $1/T$.  The high temperature limit of the
supersymmetric gauge theory in $p+1$ dimensions is thus described by a
non supersymmetric gauge theory in $p$ dimensions.

The main obstacle to the analysis is clear from the discussions of the
duality between string theory and quantum field theories in the
previous sections.  The string approach to weakly coupled gauge
theories is not yet developed.  Most of the available tools are
applicable in the supergravity limit that describes the gauge theory
with a large number of colors and large 't Hooft parameter. In this
regime we cannot really learn directly about QCD, since the typical
scale of candidate 
QCD states (glueballs) is of the same order of magnitude (for
QCD$_4$, or a larger scale for QCD$_3$) as the scale $1/R_0$ of the
mass of the ``extra'' scalars and fermions. A related issue is that
at
short distances asymptotically free gauge theories are weakly coupled
and the dual supergravity description is not valid.  Therefore, we
will be limited to a discussion in the strong coupling region of the
gauge theories and in particular we will not be able to exhibit
asymptotic freedom.

One may hope that a full solution of the classical ($g_s=0$) string
theory will provide a description of large $N$ gauge theories for all
couplings (in the 't Hooft limit).  To study the gauge theories with a
finite number of colors requires the quantum string theory.  However,
there is also a possibility that the gauge description is valid for
weak coupling and the string theory description is valid for strong
coupling with no smooth crossover between the two descriptions.  In
such a scenario there is a phase transition at $\lambda = \lambda_c$ 
\cite{Li:1999kd,Gao:1998ww}.
This will prevent us from using the string description to study QCD,
and will prevent classical string theory from being the master field
for large $N$ QCD.

In the last part of this section we will briefly discuss another
approach, based on a suggestion by Polyakov \cite{Polyakov:1998ju}, to
study non supersymmetric gauge theories via a non supersymmetric
string description. In this approach one can exhibit asymptotic
freedom qualitatively already in the gravity description.  In the IR
there are gravity solutions that exhibit confinement at large
distances as well as strongly coupled fixed points.


 

\subsection{QCD$_3$}


The starting point for studying QCD$_3$ is the $\cN=4$ superconformal
$SU(N)$ gauge theory in four dimensions which is realized as the low
energy effective theory of $N$ coinciding parallel D3 branes.  As
outlined above, the three-dimensional non-supersymmetric theory is
constructed by compactifying this theory on $\IR^3 \times {\bf S}^1$
with anti-periodic boundary conditions for the fermions around the
circle. The boundary conditions break supersymmetry explicitly and as
the radius $R_0$ of the circle becomes small, the fermions decouple
from the system since there are no zero frequency modes.  The scalar
fields in the four dimensional theory will acquire masses at one-loop,
since supersymmetry is broken, and these masses become infinite as
$R_0 \rightarrow 0$.  Therefore in the infrared we are left with only
the gauge field degrees of freedom and the theory should be
effectively the same as pure QCD$_3$.

We will now carry out the same procedure in the dual superstring
(supergravity) picture.  As has been extensively discussed in the
previous sections, the ${\cal N}=4$ theory on $\IR^4$ is
conjectured to be dual to type IIB superstring theory on AdS$_5 \times
{\bf S}^5$ with the metric (\ref{nearhor}) or (\ref{metricu}).
 
Recall that the dimensionless gauge coupling constant $g_4$ of the
$\cN=4$ theory is related to the string coupling constant $g_s$ as
$g_4^2 \simeq g_s$.  In the 't Hooft limit, $N \rightarrow \infty$
with $g_4^2N \simeq g_s N$ fixed, the string coupling constant
vanishes, $g_s \rightarrow 0$.  Therefore, we could study the $\cN=4$
theory using the tree level string theory in the AdS space
(\ref{metricu}). If also $g_s N \gg 1$, the curvature of the AdS space is
small and the string theory is approximated by classical supergravity.

Upon compactification on ${\bf S}^1$ with
supersymmetry breaking boundary conditions, (\ref{metricu})
is replaced by the Euclidean black hole geometry \cite{Horowitz:1998pq, 
Witten:1998zw} \footnote{The stability issue of this background 
is discussed in \cite{Horowitz:1999ha}.} 

\beq
ds^2 = \alpha' \sqrt{4 \pi g_s N} 
\left( u^2 (h(u)
 d\tau^2 + \sum_{i=1}^3 dx_i^2)+ h(u)^{-1}
 \frac{du^2}{u^2} + d\Omega_5^2 \right) \ ,
\label{3dmetric}
\eeq
where
 $\tau$ parametrizes the compactifying circle (with radius 
$R_0$ in the field theory) and
\beq
h(u)=1-{u_0^4\ov u^4} \ .
\eeq

The $x_{1,2,3}$ directions correspond to the $\IR^3$
coordinates of QCD$_3$. The horizon of this geometry is
located at $u=u_0$ with
\beq
u_0 = {1 \over 2 R_0} \ .
\label{horizonlocation}
\eeq
The supergravity approximation is applicable
for $N \rightarrow \infty$ and $g_s N \gg 1$, so that all the curvature
invariants are small.
The metric (\ref{3dmetric}) describes the Euclidean theory, the 
Lorentzian theory is obtained by changing $\sum_{i=1}^3 dx_i^2 \to
 -dt^2 + dx_1^2 + dx_2^2 $. Notice that this is not the same
as the Wick rotation that leads to the near extremal 
black hole solution
\NearDThree .



{}From the point of view of QCD$_3$, the radius $R_0$ of the
compactifying circle provides  the ultraviolet cutoff
scale. To obtain large $N$
QCD$_3$ itself (with infinite cutoff), 
one has to take $g_4^2 N \rightarrow 0$  
as  $R_0 \rightarrow 0$ so that the three dimensional effective
coupling $g_3^2 N = g_4^2N/( 2\pi R_0) $ remains
at the intrinsic energy scale of QCD$_3$. 
%More precisely, we would like to take the radius $R_0$ 
%to be small enough but not strictly
%zero in order to avoid the string winding modes around the circle.
$g_3^2$ is the classical dimensionful coupling of  QCD$_3$. The effective 
dimensionless gauge coupling of QCD$_3$ at the 
distance scale $R_0$ is therefore $g_sN$. 


The proposal is that Type IIB string theory on the AdS black hole background
(\ref{3dmetric}) provides a dual description to QCD$_3$ (with the UV
cutoff described above).
The limit in which the classical supergravity description is valid, 
$g_sN \gg 1$,
is the limit where the typical mass scale of QCD$_3$, $g_3^2 N$, is much
larger than the cutoff scale $1/R_0$.
%is strongly coupled.
It is the opposite of the limit that is required in order 
to see the ultraviolet freedom
of the theory.
Therefore, with the currently available techniques, we 
can only study large $N$ QCD$_3$ with a fixed ultraviolet
cutoff $R_0^{-1}$ in the strong coupling regime.
It should be emphasized that by strong coupling we mean here that the
coupling is large compared to the cutoff scale, so we really have many
more degrees of freedom than just those of QCD$_3$.
QCD$_3$ is the theory which we would get 
in the limit of vanishing bare coupling, which
is the opposite limit to the one we are taking.

This is analogous to, but not the same as, 
the lattice strong coupling expansion with a fixed 
cutoff given by the
 lattice spacing $a$ (which is analogous to $R_0$ here).
There, QCD$_3$ is
 obtained in the limit $g_3^2 a \rightarrow 0$ while strong 
coupling lattice QCD$_3$ is the theory at large $g_3^2 a$.
An important  difference in the approach that we take, 
compared to the lattice description,
is that 
we have full Lorentz 
invariance in the three gauge theory coordinates.  
The regularization of the gauge theory
in the dual string theory description
is provided by a one higher dimensional theory, the theory on D3 branes.

In the limit 
$R_0 \rightarrow 0$ the geometry (\ref{3dmetric}) is singular.
As discussed 
above, in this limit the supergravity description is not valid
and we have to use the string theory description.
 

\subsubsection{Confinement}

As we noted before,
the gauge coupling of QCD$_3$ $g_3^2$ has dimensions of mass, and 
it provides a scale
already for the classical theory.
The effective dimensionless expansion parameter at a 
length scale $l$, $g_3^2(l)\equiv
 l g_3^2$, goes to zero as $l \rightarrow 0$.
Therefore, like QCD$_4$, the theory is free at short distances.
Similarly, at a large length scale $l$ the effective coupling 
becomes strong. Therefore,
the interesting IR physics is non-perturbative.

In three dimensions the Coulomb potential is already confining. This is a 
logarithmic confinement $V(r) \sim ln(r)$. 
Lattice simulations provide evidence that in QCD$_3$
at large distances there is confinement
with a linear potential $V(r) \sim \sigma r$.



To see confinement in the dual description
we will consider the spatial Wilson loop.
In a confining theory the vacuum expectation value
of the Wilson loop operator exhibits an area law behavior \cite{Wilson:1974co}
\beq
\langle W(C) \rangle \simeq exp(-\sigma A(C)) \ ,
\label{area}
\eeq
where $A(C)$ is the area enclosed by the loop $C$. The constant 
$\sigma$ is called the string tension.
The area law (\ref{area}) is equivalent to the quark-antiquark confining
linear potential $V(L) \sim \sigma L$.
This can be simply seen by considering a rectangular loop $C$ with sides of 
length $T$ and $L$ in Euclidean
space as in figure \ref{arealaw}. 
For large $T$ we have, when $V(L) \sim \sigma L$ and interpreting $T$ as
the time direction,
\beq
\langle W(C) \rangle \sim  exp(-TV(L)) \sim exp(-\sigma A(C))  \ .
\eeq 

\begin{figure}[htb]
\begin{center}
\epsfxsize=1.5in\leavevmode\epsfbox{arealaw.eps}
\end{center}
\caption{A confining quark-antiquark linear potential $V(L) \sim \sigma L$ 
can be extracted from the Wilson loop obeying an area law
$\langle W(C) \rangle \sim  exp(-\sigma TL)$.
}
\label{arealaw}
\end{figure} 


The prescription to evaluate the vacuum expectation value
of the Wilson loop operator in the dual string description
has been introduced in section \ref{wilsonloops}.
It amounts to computing 
\beq
\langle W(C) \rangle = \int exp(-\mu(D)) \ ,
\label{sumarea}
\eeq
where $\mu(D)$ is the regularized area of the worldsheet of a string
$D$ bounded at infinity by $C$. 
%More precisely, we have to include the
%fermions too and the integral is on the space of superstring
%worldsheets bounded by $C$, and evaluates the full superstring
%partition function.

We will work in the supergravity approximation in which (\ref{sumarea})
is approximated 
by
\beq
\langle W(C) \rangle = exp(-\mu(D)) \ ,
\label{minarea}
\eeq
where $\mu(D)$ is the minimal area of a string worldsheet $D$ 
bounded at infinity by $C$.

This prescription has been applied in section \ref{wilsonloops} 
to the calculation of the
Wilson loop in the $\cN=4$ theory which is not a confining theory. 
Indeed, it has been
found there that it exhibits a Coulomb like behavior.
The basic reason was that when we scaled up the loop $C$ by 
$x^i \rightarrow \alpha x^i$ with a positive number $\alpha$, 
we could use conformal invariance to 
scale up $D$ without changing its (regularized) area. 
Therefore $D$ was not proportional
to $A(C)$.
When scaling up the loop the surface D bends in the interior of the AdS space. 
In the case when such a bending is limited by the range of the radial 
coordinate
one gets an area law. This is the case in the models at hand, in which
the coordinate $u$ in (\ref{3dmetric}) 
is bounded from below by $u_0$ as in figure \ref{string_fig}.

\begin{figure}[htb]
\begin{center}
\epsfxsize=3in\leavevmode\epsfbox{string.eps}
\end{center}
\caption{The worldsheet of the string $D$ is bounded at infinite $u$ by the 
loop $C$. The string tends to minimize its length by going to the 
region with smallest
metric component $g_{ii}$, which in this case 
is near the horizon $u=u_0$. The energy between the quark and the antiquark
is proportional to the distance $L$ between them and to the string tension
which is $\sigma = \frac{1}{2\pi}g_{ii}(u_0)$.
}
\label{string_fig}
\end{figure} 

The evaluation of the classical action of the string worldsheet 
bounded by the loop $C$ at infinite $u$ is straightforward, 
as done in section \ref{wilsonloops}
\cite{Rey:1998wp,Brandhuber:1998wl}.
The string minimizes its length by going to the region with the
smallest possible metric component $g_{ii}$ (where $i$ labels the
$\IR^3$ directions), from which it gets the contribution to the string
tension.  The smallest value of $g_{ii}$ in the metric
(\ref{3dmetric}) is at the horizon.  Thus, we find that the Wilson
loop exhibits an area law (\ref{area}), where the string tension is
given by the $g_{ii}$ component of the metric (\ref{3dmetric})
evaluated at the horizon $u=u_0$ times a numerical factor
$\frac{1}{2\pi}$ :
\beq
\sigma = \frac{1}{2 \pi} \sqrt{4 \pi g_s N} u_0^2 = 
\frac{(g_sN)^{1/2}}{4\sqrt{\pi} R_0^2} \ .
\label{tension}
\eeq


The way supergravity exhibits confinement has an analog in the lattice
strong coupling expansion, as first demonstrated by Wilson
\cite{Wilson:1974co}.  The leading contribution in the lattice strong
coupling expansion to the string tension is the minimal tiling by
plaquettes of the Wilson loop $C$ as we show in figure \ref{strong}.
This is analogous to the minimal area of the string worldsheet $D$
ending on the loop $C$ in figure \ref{string_fig}. One important
difference is that in the supergravity description the space is
curved.  Of course, a computation analogous to the Wilson loop
computations we described in section \ref{wilsonloops}
which would be done in flat
space would also
%is the simplest background that
exhibit confinement, since the minimal area of the string worldsheet
$D$ ending on the loop $C$ is simply the area enclosed by the loop
itself.


\begin{figure}[htb]
\begin{center}
\epsfxsize=1.5in\leavevmode\epsfbox{strong.eps}
\end{center}

\caption{The leading contribution in the lattice 
strong coupling expansion to the string tension
is the minimal tiling by plaquettes of the Wilson loop $C$.
}
\label{strong}
\end{figure} 

The quark-antiquark linear potential $V = \sigma L$ can have
corrections arising from the fluctuations of the thin tube (string)
connecting the quark and antiquark.  L\"uscher studied a leading
correction to the quark-antiquark potential at large separation $L$.
Within a class of bosonic effective theories in flat space that
describe the vibrations of the thin flux tubes he found a universal
term, $-c/L$, called a L\"uscher term \cite{Luscher:1981sb}~:
\beq
V = \sigma L - c/L \ .
\eeq
For a flux tube in $d$ space-time dimensions $c= (d-2)/24 \pi$.
Lattice QCD calculations of the heavy quark potential have not
provided yet a definite confirmation of this subleading term.  This
term can also not been seen order by order in the lattice strong
coupling expansion.  Subleading terms is this expansion are of the non
minimal tiling type, as in figure \ref{strong2}, and correct only the
string tension but not the linear behavior of the potential.


\begin{figure}[htb]
\begin{center}
\epsfxsize=2in\leavevmode\epsfbox{strong2.eps}
\end{center}
\caption{Subleading contribution in the lattice strong coupling 
expansion to the string tension,
which is 
a non minimal tiling of the Wilson loop. This is the lattice analog
of the fluctuations of the string worldsheet. 
}
\label{strong2}
\end{figure} 


The computation of the vacuum expectation value of the Wilson loop
(\ref{minarea}) based on the minimal area of the string worldsheet $D$
does not exhibit the L\"uscher term \cite{Greensite:1998ro}.  This is
not surprising. Even if the the L\"uscher term exists in QCD$_3$, it
should originate from the fluctuations of the string worldsheet
(\ref{sumarea}) that have not been taken into account in
(\ref{minarea}).  Some analysis of these fluctuations has been done
in \cite{Greensite:1999wf}, 
but the full computation has not been carried out yet.  


Other works on confinement as seen by a dual supergravity description are
\cite{Dorn:1998tu,Kinar:1998xx,Danielsson:1999wt,
Naik:1999bs}.

\subsubsection{Mass Spectrum}

If the dual supergravity description is in the same universality class
as QCD$_3$ it should exhibit a mass gap.  In the following we will
demonstrate this property. We will also compute the spectrum of lowest
glueball masses in the dual supergravity description.  They will
resemble qualitatively the strong coupling lattice picture.  We will
also discuss a possible comparison to lattice results in the continuum
limit.

The mass spectrum in pure QCD can be obtained by computing the
correlation functions of gauge invariant local operators (glueball
operators) or Wilson loops, and looking for the particle poles.  As we
discussed extensively before, correlation functions of local operators
are related (in some limit)
to tree level amplitudes in the dual supergravity
description.  We will consider the two point functions of glueball
operators $\cO$ (for instance, we could take $\cO = \tr(F^2)$).  
%It takes the general form
For large $|x-y|$ it has an expansion of the form
\beq
\langle \cO(x) \cO(y) \rangle \simeq \sum c_i exp(-M_i |x-y|) \ ,
\label{corr}
\eeq
where $M_i$ are called the glueball masses.

We will classify the spectrum of
glueballs by $J^{PC}$ where $J$ is the glueball spin, $P$ its parity
and $C$ its charge conjugation eigenvalue. 
The action of $C$ on the gluon fields is \cite{Mandula:1983cl}
\beq
C: A_{\mu}^a T^a_{ij} \rightarrow -A_{\mu}^a T^a_{ij} \ ,
\eeq
where the $T^a$'s are the hermitian generators of the gauge group.
In string theory, charge conjugation corresponds to the worldsheet parity
transformation changing the orientation of the open strings attached to the
D-branes.

Consider first the lowest mass glueball state.
It carries $0^{++}$ quantum numbers.
One has to identify a corresponding glueball
operator, namely a local gauge invariant operator with these quantum numbers.
The lowest  dimension operator with these properties is $\tr(F^2)$,
and we have to compute its two point function. 
To do that we need to identify first
the corresponding supergravity field that couples to it as a source
at infinite $u$.
This is the Type IIB dilaton field $\Phi$. 

The correspondence  between the gauge theory and the dual string 
theory picture asserts that in the SUGRA limit the 
computation of the correlation function amounts
to solving the field equation for $\Phi$ in the AdS black hole background
(\ref{3dmetric}),
\beq
\partial_{\mu}(\sqrt{g}g^{\mu\nu}\partial_{\nu} \Phi) = 0 \ .
\label{Laplace}
\eeq

In order to find the lowest mass modes we consider solutions of $\Phi$
which are independent of the angular coordinate $\tau$ and take the
form $\Phi = f(u)e^{ikx}$.  Plugging this in (\ref{Laplace}) we obtain
the differential equation
\beq
\partial_u [u(u^4-u_0^4)\partial_u f(u)] + M^2 uf(u) = 0,~~~~~ M^2 = -k^2 \ .
\label{difeq}
\eeq
The eigenvalues $M^2$ of this equation are the glueball masses squared.


At large $u$ equation (\ref{difeq}) has two independent solutions, 
whose asymptotic behavior is $f
\sim constant$ and $f\sim 1/u^4$. We consider normalizable solutions
and choose the second one.  Regularity requires the vanishing of the
derivative of $f(u)$ at the horizon.  The eigenvalues $M^2$ can be
determined numerically \cite{Csaki:1998gm,Koch:1998eo, Zyskin:1998an}, 
or approximately via WKB techniques \cite{Csaki:1998gm, Minahan:1999tm}.

One finds that:\\
(i) There are no solutions with eigenvalues $M^2 \leq 0$.\\
(ii) There is a discrete set of eigenvalues $M^2 > 0$.

This exhibits the mass gap property of the supergravity picture.  In
fact, even without an explicit solution of the eigenvalues $M^2$ of
equation (\ref{difeq}), the properties (i) and (ii) can be deduced
from the structure of the equation and the requirement for
normalizable and regular solutions \cite{Witten:1998zw}.

The $0^{++}$ mass spectrum in the WKB approximation closely agrees with the
more accurate numerical solution.
It takes the form
\beq
M^2_{0^{++}} \approx \frac{1.44 n(n+1)}{R_0^2},~~~~ n=1,2,3,\cdots \ .
\label{spectrum}
\eeq


The mass spectrum (\ref{spectrum}), that corresponds to a massless
mode of the string in ten dimensions, is proportional to the cutoff
$1/R_0$ and not to $\sigma^{1/2}$, which is bigger by a power of $g_s
N$ (\ref{tension}).  This is qualitatively similar to what happens in
strong coupling lattice QCD with lattice spacing $a$. As we will
discuss in the next section, in the strong coupling lattice QCD
description the lowest masses of glueballs are proportional to $1/a$.
Note that in a stringy description of QCD we would expect the
glueballs to correspond to
string excitations, which  are expected to have masses of order
$\sigma^{1/2}$. Therefore in the supergravity limit, $g_sN \gg 1$, the
glueballs that correspond to the string excitations are much heavier
than the ``supergravity glueballs'' which we analyzed.

The natural scale for the glueball masses of continuum QCD$_3$ is  
$g_3^2N$. Therefore to get to the continuum QCD$_3$ region we have to
require $g_3^2N \ll 1/R_0$ which implies $g_s N \ll 1$.
As discussed above, our computation is performed in
the opposite limit $g_s N \gg 1$.
In particular, we do not have control 
over possible mixing between glueball states
and the other scalars and fermionic degrees of freedom which are 
at the same mass scale $1/R_0$ in the field theory.   


We can attempt a numerical comparison of the supergravity computations
with the continuum limit of lattice QCD, obtained by taking the bare
coupling to zero. Since these are computations at two different limits
of the coupling value (of the original $\cn=4$ theory) there is
apriori no reason for any agreement.  Curiously, it turns out that
ratios of the glueball excited state masses with $n > 1$ in
(\ref{spectrum}) and the lowest mass $n=1$ state are in reasonably
good agreement with the lattice computations (within the systematic
and statistical error bars)
\cite{Csaki:1998gm,Teper:1998sg}. 


As a second example consider the spectrum of $0^{--}$ glueball masses.
It can be computed via the field equations of the NS-NS
2-form field. 
The details of the computation can be found in \cite{Csaki:1998gm} and,
as in the $0^{++}$ case, the
ratios of the glueball masses 
are found to be in good agreement
with the lattice computations. 

In closing the numerical comparison we note another curious agreement
between the supergravity computation and the weak coupling lattice
computations. This is for the
ratio of the lowest mass
$0^{++}$ and $0^{--}$ glueball
states,
\beqar
&\left(\frac{M_{0^{--}}}{M_{0^{++}}}\right)_{{\rm supergravity}}&= 1.50, 
\nonumber\\
&\left(\frac{M_{0^{--}}}{M_{0^{++}}}\right)_{{\rm lattice~~~~~}}& =
 1.45\pm 0.08 \ .
\eeqar

As stressed above, the regime where we would have liked to compute the
mass spectrum is in the limit of small $g_s N$ (or large ultraviolet
cutoff $1/R_0$).  In this limit the background is singular and we have
to use the string theory description,  which we lack.  We can compute
the subleading correction in the strong coupling expansion to the
masses. This requires the inclusion of the $\alpha'^3$ corrections to
the supergravity action.  The typical form of the masses is
\beq
M^2 = \frac{c_0 + c_1 \alpha'^3/R^6}{R_0^2}  \ ,
\eeq
with $c_0$ as in (\ref{spectrum}).
The background metric is modified by the inclusion of the $\alpha'^3 \cR^4$
string correction to the supergravity action. The modified
metric has been derived in
\cite{Gubser:1998nz,Pawelczyk:1998pb}.
Based on this metric
the corrections to the masses $c_1$ have been computed in \cite{Csaki:1998gm}.
While these corrections significantly change the glueball masses, 
the corrections to the
mass ratios turn out to be relatively small.

Lattice computations may exhibit lattice artifacts due to the finite
lattice spacing. Removing them amounts to taking a sufficiently small
lattice spacing such that effectively the right physics of the
continuum is captured. Getting close to the continuum means, in
particular, that deviations from Lorentz invariance are minimized.

Analogous ``artifacts'' are seen in the dual supergravity description.
They correspond to Kaluza-Klein modes that are of the same mass scale
as the glueball mass scale.  There are Kaluza-Klein modes from the
circle coordinate $\tau$ in (\ref{3dmetric}) that provides the cutoff
to the three dimensional theory.  They have a typical mass scale of
order $1/R_0$.  There are also $SO(6)$ non-singlet Kaluza-Klein modes
from the five-sphere in (\ref{3dmetric}). In the field theory they
correspond to operators involving the $SO(6)$ non-singlet scalar and
fermion fields of the high-energy theory. They have a mass scale of
order $1/R_0$ too.


The inclusion of the subleading $\alpha'^3$ correction does not make
the Kaluza-Klein modes sufficiently heavy to decouple from the
spectrum \cite{Ooguri:1998ga,Csaki:1998gm}.  This means that the dual
supergravity description is also capturing physics of the higher
dimensions, or of the massive scalar and fermion fields from the point
of view of QCD$_3$.  One hopes that upon inclusion of all the
$\alpha'$ corrections, and taking the appropriate limit of small
$g_sN$ (or large cutoff $1/R_0$), these Kaluza-Klein modes will
decouple from the system and leave only the gauge theory degrees of
freedom.  Currently, we do not have control over the $\alpha'$
corrections, which requires an understanding of a two dimensional sigma
model with a RR background.  In section \ref{diffsuba} 
we will use an analogy with
lattice field theory to improve on our supergravity description and
remove some of the Kaluza-Klein modes.


\subsection{QCD$_4$}

One starting point for obtaining QCD$_4$ is the $(2,0)$ superconformal
theory in six dimensions realized on $N$ parallel coinciding
M$5$-branes, which was discussed in section \ref{adsmore}.  The
compactification of this theory on a circle of radius $R_1$ gives a
five-dimensional theory whose low-energy effective theory is the
maximally supersymmetric $SU(N)$ gauge theory, with a gauge coupling
constant $g_5^2 = 2 \pi R_1$.  To obtain QCD$_4$, one compactifies
this theory further on another ${\bf S}^1$ of radius $R_0$. The
dimensionless gauge coupling constant $g_4$ in four dimensions is
given by $g_4^2 = g_5^2/(2 \pi R_0) =  R_1/R_0$. As in the previous case,
to break supersymmetry one imposes the anti-periodic boundary
condition on the fermions around the second ${\bf S}^1$. And, as in
the previous case, to really get QCD$_4$ we need to require that the
typical mass scale of QCD states, $\Lambda_{QCD}$, will be much smaller
than the other mass scales in our construction ($1/R_1$ and $1/R_0$),
and this will require going beyond the supergravity
approximation. However, one can hope that the theory obtained from the
supergravity limit will be in the same universality class as QCD$_4$,
and we will give some evidence for this.

As discussed in section \ref{adsmore}, the large $N$ limit of the
six-dimensional theory is $M$ theory on AdS$_7 \times {\bf S}^4$. Upon
compactification on the two circles and imposing anti-periodic
boundary conditions for the fermions on the second ${\bf S}^1$, we
get $M$ theory on a black hole background \cite{Witten:1998zw}.
Taking the large $N$ limit while keeping the 't Hooft parameter $2 \pi
\lambda = g_4^2 N$ finite requires $R_1 \ll R_0$. We can now use the
duality between M theory on a circle and Type IIA string theory, and
the M$5$ brane wrapping on the ${\bf S}^1$ of radius $R_1$ becomes a
D$4$ brane. The large $N$ limit of QCD$_4$ then becomes Type IIA
string theory on the black hole geometry given by the metric
\beq
ds^2 = \frac{2 \pi \lambda}{3u_0}u
\left(4u^2 \sum_{i=1}^4 dx_i^2
+ \frac{4}{9u_0^2}u^2(1- \frac{u_0^6}{u^6})  d \tau^2 +
4\frac{du^2}{u^2(1- \frac{u_0^6}{u^6})} + 
d\Omega_4^2 \right) \ ,
\label{D4}
\eeq
with a non constant dilaton background
\beq
e^{ 2\phi} = \frac{8 \pi \lambda^3 u^3}{27 u_0^3  N^2} \ .
\label{dilaton_qcd4}
\eeq
The coordinates $x_i, i=1,..,4$, parametrize the 
$\IR^4$ gauge theory space-time, the coordinate
$u_0 \leq u \leq \infty$, 
and $\tau$ is an angular coordinate with period $2 \pi$.   
The location of the
horizon is  at $u=u_0$, 
which is related to the radius $R_0$
% = (2 \pi T)^{-1}$ 
of the compactifying circle as
\beq
u_0 = {1 \over 3 R_0} \ .
\eeq

Equivalently, we could have started with the five dimensional 
theory on the worldvolume 
of $N$ D4 branes and heated it up to a finite 
temperature $T=1/2 \pi R_0$. 
Indeed, the geometry (\ref{D4}) with the dilaton
background (\ref{dilaton_qcd4}) is the near horizon geometry
of the non-extremal D4 brane background. But again,  when we 
Wick rotate (\ref{D4}) back to Lorentzian signature we take one of 
the coordinates $x_i$ as time. Notice that the string coupling
(\ref{dilaton_qcd4}) goes as $1/N$.


\subsubsection{Confinement}

QCD$_4$ at large distances is expected to confine with a linear
potential $V(r) \sim \sigma r$ between non-singlet states.  Therefore,
the vacuum expectation value of the Wilson loop operator is expected
to exhibit an area law behavior.  In order to see this in the dual
description we follow the same procedure as in QCD$_3$.
 
The string tension $\sigma$ is given by the coefficient of the term
$\sum_{i=1}^4 dx_i^2$ in the metric (\ref{D4}), evaluated at the
horizon $u=u_0$, times a $\frac{1}{2\pi}$ numerical factor :
\beq
\sigma = \frac{4}{3} \lambda u_0^2 = \frac{ 4\lambda}{27 R_0^2} \ .
\label{tension4}
\eeq

In QCD$_4$ it is believed that confinement is a consequence of the
condensation of magnetic monopoles via a dual Meissner effect.  Such a
mechanism has been shown to occur in supersymmetric gauge theories in
four dimensions
\cite{Seiberg:1994mc}.
This has also been demonstrated to some extent on the lattice via the
implementation of the 't Hooft Abelian Projection \cite{Hooft:1981to}.
We will now see that this appears to be the mechanism also in the dual
string theory description \cite{Gross:1998gk}.

Consider the five dimensional theory on the world volume of the D4
branes.  A magnetic monopole is realized as a D2 brane ending on the
D4 brane \cite{Douglas:1996du}. It is a string in five dimensions.
Upon compactification on a circle, the four dimensional monopole is
obtained by wrapping the string on the circle. We can now compute the
potential between a monopole and anti-monopole.  This amounts to
computing the action of a D2-brane interpolating between the monopole
and the anti-monopole, which mediates the force between them as in
figure \ref{Monopole}(a).  This is the electric-magnetic dual of the
computation of the quark-anti-quark potential described above.

If the pair is separated by a distance $L$ in the $x_1$ direction,
and stretches along the $x_2$ direction (which we can interpret as the
Euclidean time), the D2 brane coordinates are
${\tau,x_1, x_2}$. 
The action per unit length in the $x_2$ direction is given by 
\beq
V= \frac{1}{(2 \pi)^2 {\alpha'}^{3/2}}\int_0^L d\tau dx_1 e^{-\phi} 
\sqrt{det G} \ ,
\label{V}
\eeq
where $G$ is the induced metric on the D2 brane worldvolume.
We have to find a configuration of the D2-brane that minimizes (\ref{V}).
For $L > L_c$ where (up to a numerical constant) $L_c \sim R_0$,
there is no minimal volume D2 brane configuration that connects
the monopole and the anti-monopole and the energetically favorable
configuration is as in figure \ref{Monopole}(b).
Therefore there is no force between the monopole and the anti-monopole, which 
means that the magnetic charge is screened.
At length scales $L \gg R_0$ 
we expect pure QCD$_4$ as the effective description.
We see that in this region confinement is accompanied by 
monopole condensation, as we expect.   


\begin{figure}[htb]
\begin{center}
\epsfxsize=3in\leavevmode\epsfbox{monopole.eps}
\end{center}
\caption{The magnetic monopole is a string in five dimensions and
the four dimensional monopole is obtained by wrapping
the string on the circle. 
The potential between a monopole (wrapped on $c_1$) and an anti-monopole
(wrapped in the opposite orientation on $c_2$),
separated by a distance $L$ in the $x_1$ direction,
amounts to computing the action of a D2-brane which 
mediates the force between them
as in figure (a). 
For $L > L_c$ 
there is no minimal volume D2 brane configuration that connects
the monopole and the anti-monopole and the energetically favorable
configuration is as in figure (b), and then
the magnetic charge is screened.
}
\label{Monopole}
\end{figure} 


\subsubsection{$\theta$ Vacua}

In addition to the gauge coupling, four dimensional gauge theories
have an additional parameter $\theta$ which is the coefficient of the
$\tr(F\wedge F)$ term in the Lagrangian.  The $\theta$ angle
dependence of asymptotically free gauge theories captures non trivial
dynamical information about the theory.  Unlike in spontaneously broken
gauge theories, it cannot be analyzed by an instanton expansion.  What
is required is an appropriate effective description of the theories at
long wavelengths.  Such an effective description is provided by the
lattice.  However, since the Lorentz invariance is lost by the
discretization of space time, it is very difficult to study questions
such as the behavior of the system under $\theta \rightarrow \theta +
2\pi$.  Also, the construction of instantons which are the relevant
objects in the analysis of the $\theta$ dependence is a rather non
trivial task and involves delicate cooling techniques.


Another effective description may be provided by the description of
the four dimensional gauge theories by the M5 brane wrapping a non
supersymmetric cycle.  Indeed, in this formalism, one sees that the
vacuum energy exhibits the correct $\theta$ angle behavior in softly
broken supersymmetric gauge theories \cite{Oz:1998ba}.

In this subsection we use the dual string theory description to
analyze the $\theta$ angle dependence in large $N$ $SU(N)$ gauge
theory \cite{Witten:1998td}.  Since the amplitude for an instanton is
weighted by a factor $exp(-8 \pi^2 N/ \lambda)$ where $\lambda$ is the
't Hooft parameter (which we keep fixed), it naively seems that the
instanton effects vanish as $N\rightarrow \infty$.  However, unlike
the $\cN=4$ gauge theory for instance, here one expects this not to be
the case due to IR divergences in the theory.

Let us first review what we expect the behavior of the $\theta$
dependence to be from the field theory viewpoint.
The Yang-Mills action is
\beq 
I_{YM} = \int d^4x \tr( {N \over 4 \lambda} F^2 + {\theta
\over 16 \pi^2} F \tilde{F} ) \ .
\label{YM}
\eeq
At large $N$ we expect the energy of the vacuum 
to behave like
$E(\theta) = N^2 C(\theta/N)$.
The $N^2$ factor is due to the fact that this is the order of the number
of degrees of freedom (this also follows from the standard scaling of
the leading diagrams in the 't Hooft limit).
The dependence on $\theta/N$ follows from (\ref{YM}) 
as is implied by the large $N$ limit.
$\theta$ is chosen to be periodic with period $2 \pi$. Since the physics
should not change under 
$\theta \rightarrow \theta + 2 \pi$ we require
that 
$E(\theta + 2 \pi ) = E(\theta)$.

These conditions cannot be satisfied by a smooth function
of $\theta/N$. They can be satisfied by a multibranched function with 
the interpretation 
that there are $N$ inequivalent
vacua, and all of them are stable in the large $N$ limit.
The vacuum energy is then given by a minimization of 
the energy of the $k^{th}$ vacuum $E_k$
with respect 
to $k$
\beq
E(\theta) =  \min_k E_k(\theta) =  N^2 \min_k C((\theta+ 2 \pi k)/N) \ ,
\eeq
for some function $C(\theta)$ which is quadratic in $\theta$ for small
values of $\theta$.

\begin{figure}[htb]
\begin{center}
\epsfxsize=4in\leavevmode\epsfbox{theta.eps}
\end{center}
\caption{The energy of the vacuum is expected to be a multibranched function.
}
\label{theta}
\end{figure} 


The function $E(\theta)$ is periodic in $\theta$ and jumps at some
values of $\theta$ between different branches. The CP transformation
acts by $\theta \rightarrow -\theta$ and is a symmetry only for
$\theta=0,\pi$. Therefore, $C(\theta) = C(-\theta)$.  One expects an
absolute minimum at $\theta=0$ and a non vanishing of the second
derivative of $E(\theta)$ with respect to $\theta$, which corresponds
to the topological susceptibility $\chi_t$ of the system as we will
discuss later.  Taking all these facts into account one conjectures in
the leading order in $1/N$ that \cite{Witten:1980ln}
\beq
E(\theta) =  \chi_t \min_k (\theta+ 2 \pi k)^2 + O(1/N) \ ,
\label{vacuumenergy}
\eeq 
where $\chi_t$ is positive and independent of $N$.  At $\theta=\pi$
the function exhibits the jump between the vacua at $k=0$ and $k=-1$
and the spontaneous breaking of CP invariance.

In order to analyze the $\theta$ dependence in the dual string theory
description with the background (\ref{D4}) we have to identify the
$\theta$ parameter.  This is done by recalling that the effective
Lagrangian of $N$ D4 branes in Type IIA string theory has the coupling
\beq
\frac{1}{16 \pi^2}\int d^5x \varepsilon^{\rho\alpha\beta\gamma\delta}
\cA_{\rho} \tr(F_{\alpha\beta} F_{\gamma\delta}) \ ,
\label{FFcoupling}
\eeq
where $\cA$ is the Type IIA RR 1-form and $F$ is the $U(N)$ gauge
field strength on the five dimensional brane worldvolume.  Upon
compactification of the D4 brane theory on a circle we see that the
four dimensional $\theta$ parameter is related to the integral of the
RR 1-form on the circle.  Since it is a ten dimensional field it is 
a parameter from the worldvolume  point of view.
% The requirement that $\theta$ is a parameter
%that modifies the vacuum but is not a dynamical field amounts to
%vanishing of the field strength $\cF$ of $\cA$.

In the dual description we define the parameters at infinite $u$.  The
$\theta$ parameter is defined as the integral of the RR 1-form
component on the circle at infinite $u$
\beq
\theta = \int d \tau \cA_{\tau} = 2 \pi \cA_{\tau}^{\infty} \ ,
\eeq
which is defined modulo $2\pi k, k \in \IZ$.
%In  this description
%the vanishing of the field strength 
%is imposed at infinite $u$.
%corresponds to the vanishing of
%the field strength at infinite $u$.

The action for the RR 1-form takes the form
\beq
I = \frac{1}{2 \kappa_{10}^2} \int d^{10}x \sqrt{g} \frac{1}{4}
g^{\alpha\alpha'}g^{\beta\beta'}(\partial_{\alpha}
\cA_{\beta} - \partial_{\beta}\cA_{\alpha})
(\partial_{\alpha'}
\cA_{\beta'} - \partial_{\beta'}\cA_{\alpha'}) \ ,
\label{actionRRone}
\eeq
and the equation of motion for $\cA$ is 
\beq
\partial_{\alpha}[\sqrt{g}g^{\beta\gamma}g^{\alpha\delta}(\partial_{\gamma}
A_{\delta} - \partial_{\delta}A_{\gamma})] = 0 \ .
\label{mode}
\eeq

The required solution $A_{\tau}(u)$ to (\ref{mode}), regular at $u=u_0$ and 
with vanishing field strength
at infinite $u$ (in order to have finite energy), takes the form 
\beq
A_{\tau}(u) = A_{\tau}^{\infty}(1- \frac{u_0^6}{u^6}) \ .
\label{solution_a}
\eeq

   

Evaluating the Type IIA action for the RR 1-form
(\ref{actionRRone}) 
with the solution (\ref{solution_a}) and recalling the $2 \pi 
\IZ$ ambiguity
we get the vacuum energy (\ref{vacuumenergy})
where $\chi_t$ is independent of $N$ \cite{Witten:1998td}.


In order to check that the vacua labeled by $k$ are all stable in the
limit $N\rightarrow \infty$ we need a way to estimate their lifetime.
The domain wall separating two adjacent vacua is constructed by
wrapping a D6 brane of Type IIA string theory on the $S^4$ part of the
metric \cite{Witten:1998td}.  Since the energy density of the brane at
weak coupling is of order $1/g_s$ where $g_s$ is the Type IIA
string coupling, as $N \rightarrow \infty$ (with fixed $g_s N$)
it is of order $N$. If we
assume a mechanism for the decay of a $k$-th vacuum via a D6 brane
bubble, its decay rate is of the order of $e^{-N}$.  Thus, there is an
infinite number of stable vacua in the infinite $N$ limit.
 
One can repeat the discussion of confinement in the previous subsection
for $\theta \neq 0$.
When $\theta = 2 \pi p/q$ with co-prime integers $p,q$ the confinement is
associated with a condensation of $(-p,q)$ dyons and realizes the
mechanism of oblique confinement.

\subsubsection{Mass Spectrum}

The analysis of the mass spectrum of QCD$_4$ as seen by  the dual
description in the supergravity limit is similar to the one we carried
out for QCD$_3$.
It is illuminating to consider an analogous picture of 
strong coupling lattice QCD \cite{Gross:1998gk}.

In strong coupling lattice QCD
the masses of the lightest glueballs 
are of order $1/a$ where $a$ is the lattice spacing.
The reasoning is that in strong coupling lattice
QCD the leading contribution to the correlator of
two Wilson loops separated by distance $L$
is from a tube with the size of one plaquette, as in figure \ref{glueball},
that connects the loops. 
With the Wilson lattice action the $0^{++}$ glueball mass is given by 
\cite{Creutz:1987xx}
\beq
M_{0^{++}}  =  -4 \log(g_{4}^2 N) a^{-1} \ .
\label{lattice}
\eeq


\begin{figure}[htb]
\begin{center}
\epsfxsize=1.6in\leavevmode\epsfbox{glueball.eps}
\end{center}
\caption{The leading contribution in strong coupling lattice
QCD to the correlator of
two Wilson loops, separated by distance $L$,
is from a tube with the size of one plaquette
that connects the loops. 
This leads to the lowest mass glueballs having a mass of the order of
$1/a$, where $a$ is the lattice spacing.
}
\label{glueball}
\end{figure} 



To make the connection with continuum QCD$_4$ we would like to sum the
lattice strong coupling expansion $M_{0^{++}} = F(g_{4}^2 N) a^{-1}$,
and take the limit $a \to 0$ and $g_4 \to 0$ with
\beq
g_{4}^2 N \simeq 
\frac{1}{b log(1/a\Lambda_{QCD})}~~~as~~~~a \rightarrow 0 \ ,
\label{limit}
\eeq
where $g_4$ is the four dimensional coupling and $b$ is the first coefficient
of the $\beta$-function.
We hope that in the limit (\ref{limit}) we will get a finite glueball mass
measured in $\Lambda_{QCD}$ units.

In the dual string theory description the analog of $a$ is $R_0$.
The strong coupling expansion is analogous to
the $\alpha'$ expansion of string theory.
Supergravity is the leading contribution in this expansion. 
The lowest glueball masses $M_g$ correspond to the zero modes
of the string, and their mass is proportional to $1/R_0$.
Another way to see that this limit 
resembles the strong coupling lattice QCD picture is 
to consider the Wilson loop correlation function
$\langle W(C_1) W(C_2) \rangle$ as in figure \ref{wilson}(a).

For $L > L_c$, where $L$ is the distance
between the loops and $L_c$ is determined by the size of the
loops, there is no stable 
string worldsheet configuration
connecting the two loops, as in figure \ref{wilson}(b).
The string worldsheet that connects the loops as in figure \ref{graviton}(a)
collapses and 
the two disks are now connected by a tube of string scale size as in 
figure \ref{graviton}(b),
resembling the strong coupling lattice QCD picture.
The correlation
function is then mediated by a supergraviton exchange between the disks.
Thus, the  supergravitons are identified with the glueball states and the 
lowest glueball masses turn out to be proportional to $1/R_0$ 
\cite{Gross:1998gk}.


\begin{figure}[htb]
\begin{center}
\epsfxsize=5in\leavevmode\epsfbox{wilson.eps}
\end{center}
\caption{The Wilson loop correlation function
in figure (a) is computed by minimization of the string worldsheet
that interpolates between them.  When the distance between the loops
$L$ is larger than $L_c$ there is no stable string worldsheet
configuration connecting the two loops as in figure (b).  }
\label{wilson}
\end{figure} 



\begin{figure}[htb]
\begin{center}
\epsfxsize=2.8in\leavevmode\epsfbox{graviton.eps}
\end{center}
\caption{The string worldsheet that connects the loops in figure (a)
collapses and 
the two disks are now connected by a tube of a string scale as in figure (b).
The correlation
function is mediated by a supergraviton exchange between the disks
and the  supergravitons are identified with the glueball states.
}
\label{graviton}
\end{figure} 

As in strong coupling lattice QCD,
to make the connection with the actual QCD$_4$ theory we need 
to sum the strong coupling expansion $M_g= F(g_{4}^2 N) / R_0$
and take the limit of $R_0 \to 0$ and $g_4 \to 0$ with
\beq
g_{4}^2 N \rightarrow \frac{1}{b log(1/R_0 \Lambda_{QCD})}~~~
as~~~~R_0\rightarrow 0 \ .
\label{limitsg}
\eeq
Again, 
we hope that in the limit (\ref{limitsg}) we will get a finite glueball mass
proportional to $\Lambda_{QCD}$.


In the limit (\ref{limitsg}) the background (\ref{D4}) is singular.
Thus, to work at large $N$ in this limit we need the full
tree level string theory description and not just the SUGRA limit.
The supergravity description will provide us with information analogous
to that of strong coupling lattice QCD with a finite cutoff.
However, since as discussed before
the regularization here is done via a higher dimensional theory,
we will have the advantage of a full Lorentz invariant description
in four dimensions.
What we should be worried about is whether we capture the physics of the
higher dimensions as well (which from the point of view of QCD$_4$ 
correspond to additional charged fields).

In order to compute the mass gap 
we consider the scalar glueball $0^{++}$.
The  $0^{++}$ glueball mass  spectrum is obtained by solving
the supergravity equation for any mode $f$ that couples
to $0^{++}$ glueball operators; we expect (and this is verified by the
calculation) that the lightest glueball will come from a mode that couples
to the operator
$\tr(F^2)$. There are several steps to be taken 
in order to identify this mode and its supergravity equation.
First, we consider small
fluctuations of the supergravity 
fields on the background (\ref{D4}), (\ref{dilaton_qcd4}).
The subtlety that arises is the need to disentangle the 
mixing between the dilaton field and
the volume factor which has been done in \cite{Hashimoto:1998ao}. 
One then plugs the appropriate ``diagonal''
combinations of these fields into the supergravity equations of motion.
The field/operator identification can then be 
done by considering the Born-Infeld action of
the D4 brane in the gravitational background. 

To compute the lowest mass modes
we consider solutions of the form $f = f(u) e^{ikx}$ 
which satisfy the equation
\beq
\frac{1}{u^3}\partial_{u}[u(u^6-u_0^6)\partial_{u}f(u)] + M^2 f(u) = 0 \ .
\eeq
The eigenvalues $M^2$ are the glueball masses.  The required solutions
are normalizable and regular at the horizon.  The eigenvalues $M^2$
can be determined numerically \cite{Hashimoto:1998ao} or approximately
via WKB techniques \cite{Minahan:1999tm}.

As in QCD$_3$ one finds that:\\
(i) There are no solutions with eigenvalues $M^2 \leq 0$.\\
(ii) There is a discrete set of eigenvalues $M^2 > 0$.

This exhibits the mass gap property of the supergravity picture.

The $0^{++}$ mass spectrum in the WKB approximation closely agrees with the
more accurate numerical solution.
It takes the form
\beq
M^2 \approx \frac{0.74 n(n+2)}{R_0^2},~~~~ n = 1,2,3,\cdots \ .
\label{spectrum4}
\eeq
As in QCD$_3$,  the ratios of the glueball excited state masses with $n > 1$
in (\ref{spectrum4}) and the lowest mass $n=1$ state are in good agreement
with the available lattice computations \cite{Hashimoto:1998ao, Csaki:1998gm}. 


As another example consider the $0^{-+}$ glueballs.
The lowest dimension operator with these quantum numbers is $\tr(F \tilde{F})$.
As we discussed previously, 
on the D4 brane worldvolume it couples to the RR 1-form
$\cA_{\tau}$ (\ref{FFcoupling}).
Its equation of motion is given by (\ref{mode}).
We look for solutions of the form $\cA_{\tau} = f_{\tau}(u)e^{ikx}$.
Plugging this into (\ref{mode}) we get
\beq
\frac{1}{u^5}(u^6-u_0^6) \partial_u[u^7\partial_u f_{\tau}(u)] 
+ u^4 M^2 f_{\tau}(u)=0 \ .
\eeq
As for the $0^{++}$ glueball states, the 
ratios of the $0^{-+}$ glueball masses 
are found to be in good agreement
with the lattice computations 
\cite{Hashimoto:1998ao}.

Finally, we note that the
ratio of the lowest masses
$0^{++}$ and $0^{-+}$ glueball
states \cite{Hashimoto:1998ao}
\beqar
\left(\frac{M_{0^{-+}}}{M_{0^{++}}}\right)_{{\rm supergravity}}&= &1.20, 
\nonumber\\
\left(\frac{M_{0^{-+}}}{M_{0^{++}}}\right)_{{\rm lattice~~~~~~}}& =&
1.36 \pm 0.32 \ ,
\eeqar
agrees with the lattice results too.
Similar types of agreements 
in mass spectrum computations were claimed 
in strong coupling lattice QCD \cite{Munster:1983ps}.
However, note that (as discussed above for QCD$_3$)
other ratios, such as the ratio of the glueball masses
to the square root of the string tension, are very different in the SUGRA
limit from the results in QCD.

The computation of the mass gap in the dual supergravity picture is in
the opposite limit to QCD.  As in the supergravity description of
QCD$_3$, also here the Kaluza-Klein modes do not decouple.  In this
approach, in order to perform the computation in the QCD regime we
need to use string theory.  The surprising agreement of certain mass
ratios with the lattice results may be a coincidence. Optimistically,
it may have an underlying dynamical reason.

\subsubsection{Confinement-Deconfinement Transition}

We will now put the above four dimensional QCD-like theory 
at a finite temperature $T$ (which should not be confused with $\frac{1}
{2 \pi R_0}$).
We will  see that there is a deconfinement transition. 
In order to consider the theory at finite temperature we go to Euclidean
space and we compactify the time direction $t_E$ on a circle of radius 
$\beta$ with antiperiodic fermion 
boundary conditions. Since we already had
one circle (labeled by $\tau$ in (\ref{D4})), we now have two circles
with antiperiodic boundary conditions. So, we can have several possible
gravity solutions. One is the original extremal D4 brane, another
is the solution  (\ref{D4}) and a third one is the same solution
 (\ref{D4}) but with $\tau$ and $t_E$ interchanged. 
These last two solutions are possible only when the fermions have
antiperiodic boundary conditions on the corresponding circles. 
One of the last two solutions always has lower free energy than 
the first, so we concentrate on these last two.
 
It turns out that the initial solution (\ref{D4}) has the lowest free
energy for low temperatures, when $\beta = 1/T > 2 \pi R_0$,
 while the one with 
 $\tau \leftrightarrow t_E$  has the lowest free energy for 
$\beta = 1/T < 2 \pi R_0$ (high temperatures). 
The entropy of these two solutions is very different, and therefore
there is a first order phase transition, in complete analogy with
the discussion in section \ref{FiniteT}.
We do not know of a proof that there are no other solutions, but 
these two solutions have different topological properties, so
there cannot be a smoothly interpolating solution. In any case,
for very low and very high temperatures they are expected to be
the dominant configurations (see \cite{Horowitz:1999ha})\footnote{
There are other singular solutions  
 \cite{Russo:1998ze},
 but the general philosophy
is that we do not  allow singular solutions unless we can give a 
physical interpretation for the singularity.}.
 The entropy of the 
the high  temperature  phase is of order $N^2$, while the entropy of the 
low temperature  phase is essentially zero since the number of states in
the gravity picture 
is independent of the Newton constant. 

If we compute the potential between a quark and an antiquark then
in the low temperature phase it grows linearly, so that we have 
confinement, while in the high temperature phase the strings
coming from the external quarks can end on the horizon, so that
the potential vanishes beyond a certain separation. Thus, this
is  a confinement-deconfinement transition. It might seem a bit
surprising at first sight that essentially the same solution can 
be interpreted as a confined and a deconfined phase at the same time. The 
point it that quark worldlines are timelike, therefore they select
one of the two circles, and the physical properties depend crucially 
on whether this circle
is contractible or not in the full ten-dimensional geometry. 


\subsubsection{Other Dynamical Aspects}
\label{other_dynamical}

In this subsection we comment 
on various aspects of QCD$_4$  as seen by the
string description. 
We first show  how the baryons appear in the dual string theory (M
theory) picture.
We will then compute other properties of the QCD vacuum, the topological
susceptibility and the gluon condensate, as seen in the dual description.

\medskip

{\it\bf Baryons}

The baryon is an $SU(N)$ singlet bound state of 
$N$ quarks. Since we do not have quarks in our theory, we need to put
in external quarks as described in section \ref{wilsonloops}, 
and then there is a
baryon operator coupling $N$ external quarks.
As in the conformal case, also here it can be constructed as $N$ 
open strings that end on a D4 brane that is wrapped on 
$S^4$ 
\cite{Gross:1998gk,Witten:1998xy}, 
as in figure \ref{baryon}.  
If we view this geometry as arising from M-theory, then the strings
are M2 branes wrapping the circle with periodic fermion 
boundary conditions
 and the D4 brane is an M5 fivebrane also wrapping 
this circle. Then, $N$ M2 branes can end on this M5 brane as in 
\cite{Witten:1998xy}.
%
%%
%
%The strings may be viewed as M2 branes wrapping a circle in the 
%AdS black hole geometry. 
%Like in the conformal case, we can have an M5-brane wrapping the $S^4$ as
%well as the circle which the M2-branes wrap, and the couplings in the
%M5-brane worldvolume force $N$ M2-branes to end on such an M5-brane, giving
%the baryon vertex.
There is a very similar picture of a baryon in strong coupling
lattice QCD as is depicted in 
figure \ref{baryonlat}, where quarks are connected by flux links to a vertex.
%
\begin{figure}[htb]
\begin{center}
\epsfxsize=2.8in\leavevmode\epsfbox{baryon.eps}
\end{center}
\caption{The baryon is an $SU(N)$ singlet bound state of 
$N$ quarks. It is constructed as $N$ open strings that join together
at a point in the bulk AdS black hole geometry.
}
\label{baryon}
\end{figure}

\begin{figure}[htb]
\begin{center}
\epsfxsize=2in\leavevmode\epsfbox{baryonlat.eps}
\end{center}
\caption{A baryon state in strong coupling lattice QCD.
The quarks located at lattice sites 
are connected by flux links to a vertex. A similar picture is obtained
by projecting the baryon vertex in figure \ref{baryon} on $x$ space.}
\label{baryonlat}
\end{figure}

Several aspects of baryon physics can be seen from the string picture of
figure \ref{baryon} 
\cite{Witten:1998xy, Gross:1998gk}.
The baryon energy is proportional to the string tension (\ref{tension4})
and (in the limit of large distances between the quarks) to the sum of
the distances between the $N$ quark locations and the location of the
baryon vertex in the four dimensional $x$-space
\cite{Gross:1998gk,Brandhuber:1998xy,Imamura:1998hf}. 
(There is some subtlety
in evaluating the baryon energy, and it was clarified
in \cite{Imamura:1998gk} in the case of ${\cal N}=4$ gauge theory. 
See also \cite{Callan:1998iq,Callan:1999zf}.)  
We may consider the baryon vertex 
as a fixed (non-dynamical) point in the Born Oppenheimer
approximation. In such an approximation, the $N$ quarks move
independently in the potential due to the string stretched between
them and the vertex.  The baryon mass spectrum can be computed by
solving the one body problem of the quark in this potential.
Corrections to this spectrum can be computed by taking into account
the potential between the quarks and the dynamics of the vertex.  A
similar analysis has been carried out in the flux tube model
\cite{Isgur:1985fs} based on the Hamiltonian strong coupling lattice
formulation \cite{Kogut:1975xx}.


In a confining theory we do not expect to see a baryonic configuration
made from $k < N$ quarks. This follows for the above description. If
we want to separate a quark we will be left with a string running to 
infinity, which has infinite energy.
% In other words, separating to infinity $N-k$
%quarks from the baryon should cost infinite energy.  This can be seen
%from the string picture (see figure \ref{baryon1} for $k=2$).  The $N-k$
%strings have to end at $\infty$ in $x$-space and it costs infinite
%energy to create such a configuration.
%
%\begin{figure}[htb]
%\begin{center}
%\epsfxsize=2.8in\leavevmode\epsfbox{baryon1.eps}
%\end{center}
%\caption{Separating to infinity
%$N-2$ quarks from the baryon should cost infinite energy in a
%confining theory.  Indeed, the $N-2$ strings have to end at $\infty$
%in $x$-space and it costs infinite energy to create such a
%configuration.  }
%\label{baryon1}
%\end{figure}


\medskip

{\it\bf Topological Susceptibility}

The topological susceptibility $\chi_t$ measures the fluctuations of the
topological charge of the QCD vacuum. 
It is defined by
\begin{equation}
\chi_t=\frac{1}{(16\pi^2)^2} \int d^4x \langle \tr(F\tilde{F}(x))
  \tr(F\tilde{F}(0))\rangle \ .
  \label{topol}
\end{equation}
  
At large $N$ the Witten-Veneziano formula \cite{Witten:1979ca,
Veneziano:1979uv} relates the mass $m_{\eta'}$ in $SU(N)$ Yang-Mills
gauge theory with $N_f$ quarks to the topological susceptibility of
$SU(N)$ Yang-Mills theory without quarks:
\beq m_{\eta'}^2 = \frac{4N_f}{f_{\pi}^2} \chi_t \ .
\label{mass}
\eeq
Equation (\ref{mass})
is applicable at large $N$ where $f_{\pi}^2 \sim N$. In this limit
$m_{\eta'}$ goes to zero and we have the $\eta'-\pi$ degeneracy.

Nevertheless, plugging the phenomenological values $N_f=3, N=3,
m_{\eta'}\sim 1~ GeV,f_{\pi} \sim 0.1~ GeV$ in (\ref{mass}) leads to a
prediction $\chi_t \sim (180~ MeV)^4$, which is in surprising
agreement with the lattice simulation for a finite number of colors
\cite{Teper:1997pf}.

Evaluating the 2-point function from the type IIA SUGRA 
action for the RR 1-form
(\ref{actionRRone}) with the solution (\ref{solution_a}), we get the topological
susceptibility
\beq
\chi_t = \frac{2 \lambda^3}{729 \pi^3 R_0^4} \ .
\label{chit}
\eeq


The supergravity result (\ref{chit})
 depends on two parameters, $\lambda$ and
$R_0$. This is the leading  asymptotic
behavior in $1/\lambda$ of the full string theory expression 
$\chi_t \sim (F(\lambda) / R_0)^4$.
We would have liked to compute $F(\lambda)$, take the limit (\ref{limitsg})
and compare to the lattice QCD result.
However, this goes beyond the currently available calculational tools. 

It may be instructive, though, to consider the following comparison.
Let us assume that there is a cross-over between the supergravity
description and the continuum QCD description. We can estimate the
cross-over point.  In perturbative QCD we find $F(\lambda) \sim e^{-12 \pi/
11 \lambda}$, therefore the cross-over point (to the $F \sim
\lambda^{3/4}$ behavior of (\ref{chit})) can be estimated to be at
$\lambda \sim 12\pi/11$.  Also, since the mass scale in the
QCD regime is $\Lambda_{QCD}$, at the cross-over point $T = 1/2 \pi R_0 \sim
\Lambda_{QCD}\sim 200~ MeV$.  Of course, we should bear in mind that
at the cross-over point both the supergravity and perturbative QCD are
not applicable descriptions.  If we compare the topological
susceptibility (\ref{chit}) at the correspondence point with the
lattice result we get
\beq
\left({\chi_t^{{\rm SUGRA}}  \over 
\chi_t^{{\rm Lattice}}}\right)^{1/4} =  1.7 \ .
\eeq
It may be an encouraging sign that the number we get is of order one,
though its level of agreement is not as good as the mass ratios of the
glueball spectrum.

\medskip

{\it\bf Gluon Condensation}

The gluon condensate $\langle {1 \over 4 g_{4}^2}\tr(F^2(0)) \rangle$
is related by the trace anomaly to the energy density $T_{\mu\mu}$ of
the QCD vacuum.  In the supergravity picture the one point function of
an operator corresponds to the first variation of the supergravity
action. This quantity is expected to vanish by the equations of
motion. However, the first variation is only required to vanish up to
a total derivative term. Since asymptotically anti-de Sitter space has
a time-like boundary at infinity, there is a possible boundary
contribution. Indeed, unlike the $\cN=4$ case, the one point function
of the $\tr(F^2)$ operator in the dual string theory description of
QCD does not vanish.

It can be computed either directly or by using the relation between
the thermal partition function and the free energy $Z(T)=\exp(-{\cal
F}/T)$. This relates the free energy associated with the string theory
(supergravity) background to the expectation value of the operator
$\tr(F^2) $.  One gets \cite{Hashimoto:1998ao}
\beq
\langle {1\over 4g_{\rm 4}^2 }\tr(F^2_{\mu\nu}(0)) \rangle =
{1\over 8\pi } {N^2\over\lambda} \sigma^2  \ .
\label{ggl2}
\eeq

The relation (\ref{ggl2}) between the gluon condensate and the string
tension is rather general and applies for other regular backgrounds
that are possible candidates for a dual description
\cite{Csaki:1999ln}. 

If we attempt again a numerical comparison with the lattice computation 
\cite{Campostrini:1989gc,DElia1997fs} we find at the
cross-over point
\beq
\left({{({\rm Gluon\ condensate})}^{{\rm SUGRA}}  
\over {({\rm Gluon\ condensate})}^{{\rm Lattice}}}
\right)^{1/4}  =  0.9 \ .
\eeq
We should note that in field theory the gluon condensate is divergent, 
and there are subtleties (which are not
completely settled) as to the relation between the lattice regularized result
and the actual property of the QCD vacuum.


Finally, for completeness of the numerical status,
we note that if we compare the string tension (\ref{tension4}) 
at the cross-over point and the lattice result 
we get 
\beq
\left({{({\rm QCD\ string\ tension})}^{{\rm SUGRA}}  
\over {({\rm QCD\ string\ tension})}^{{\rm Lattice}}}\right)^{1/2}  = 2 \ .
\eeq
\subsection{Other Directions}

In this subsection we briefly review other possible ways of describing
non supersymmetric asymptotically free gauge theories via a dual
string description. Additional possibilities are described in section
\ref{deformations}.

\subsubsection{Different Background Metrics}
\label{diffsuba}

The string models dual to QCD$_p$ that we studied exhibit the required
qualitative properties, such as confinement, a mass gap and the
$\theta$ dependence of the vacuum energy, already in the supergravity
approximation.  We noted that besides the glueball mass spectrum there
exists a spectrum of Kaluza-Klein modes at the same mass scale.  This
indicates that the  physics of the higher
dimensions is not decoupling from the four dimensional 
physics\footnote{From the field
theory point of view it indicates that $SU(4)$-charged fields and KK
modes of five dimensional fields contribute
in addition to the four dimensional gluons.}.
The 
Kaluza-Klein states did not decouple upon the inclusion of the
$\alpha'^3$ correction, but one hopes that they do decouple in the
full string theory framework.  In the following we discuss an approach
to removing some of them already at the supergravity level. It should
be stressed, however, that this does not solve the issue of a possible
mixing between the glueball states and states that
correspond to the scalar and fermion fields,
which for large
$\lambda$ are at the same mass scale in the field theory.

Again, the analogy with lattice gauge theory is useful.  It is well
known in the lattice framework that the action one starts with has a
significant effect on the speed at which one gets to the continuum
limit.  One can add to the lattice action deformations which are
irrelevant in the continuum limit and arrive at an appropriate
effective description of the continuum theory while having a larger
lattice spacing.  Such actions are called improved actions.

A similar strategy in the dual supergravity description amounts to a
modification of the background metric. The requirement is that the
modification will better capture the effective description of the
gauge theory while still having a finite cutoff (corresponding to
finite $\lambda$ in our case).  On the lattice a criterion for
improvement is Lorentz invariance.  Here, since the cutoff is provided
by a higher dimensional theory we have the full Lorentz invariance in
any case. The improvement will be measured by the removal of the
Kaluza-Klein modes.  Note that we are attempting at an improvement in
the strong coupling regime.  Such ideas have only now begun to be
explored on the lattice \cite{Dalley:1998gc}. Till now, the effort of
lattice computations was directed at the computation of the strong
coupling expansion series.

Models that generalize the above background by the realization of the
gauge theories on non-extremal rotating branes have been studied in
\cite{Russo:1998nc, Csaki:1999ln,Cvetic:1999rb}.  The deformation of the background
is parametrized by the angular momentum parameter.  Kaluza-Klein modes
associated with the circle have the form $\Phi = f(u)
e^{ikx}e^{in\tau}, n > 0$.  It has been shown that as one varies the
angular momentum one decouples these Kaluza-Klein modes, while
maintaining the stability of the glueball mass spectrum.  This
deformation is not sufficient to decouple also the Kaluza-Klein modes
associated with the sphere part of the metrics (\ref{3dmetric}) and
(\ref{D4}), so we are still quite far from QCD.

The number of non-singular backgrounds is limited by the no hair
theorem. One may consider more angular momenta, for instance.
However, this does not seem to be sufficient to decouple all the
Kaluza-Klein states \cite{Russo:1999rd,Russo:1999sm}.  It is possible
that we will need to appeal to non regular backgrounds in order to
fully decouple the higher dimensional physics.  Some non
supersymmetric singular backgrounds of Type II supergravity that
exhibit confinement were constructed and discussed in
\cite{Kehagias:1999tr,Gubser:1999pk,Girardello:1999hj,Constable:1999ch}.


\subsubsection{Type 0 String Theory}

The Type 0 string theories have worldsheet supersymmetry but no
space-time supersymmetry as a consequence of a non-chiral GSO
projection \cite{Dixon:1986ba,Seiberg:1986ss}.  Consider two types of
such string theories, Type 0A and Type 0B.  They do not have
space-time fermions in their spectra. Nevertheless, they have a
modular invariant partition function.  The bosonic fields of these
theories are like those of the supersymmetric Type IIA and Type IIB
string theories, with a doubled set of Ramond-Ramond fields.  Type 0
string theories can be formally viewed as the high temperature limit of the
Type II string theories.  They contain a tachyon field $\cT$.



Type 0 theories have D-branes. As in the Type II case, we can consider
the gauge theories on the worldvolume of $N$ such branes.  These
theories do not contain an open string tachyon.  Moreover,  the usual
condensation of the tachyon could be avoided in the near horizon region
as we explain below.

One particular example studied in \cite{Klebanov:1998db} is the theory
on $N$ flat D3 branes in Type 0B theory.  Since there is a doubled set
of RR 4-form fields in Type 0B string theory, the D3 branes can carry
two charges, electric and magnetic.  The worldvolume theory theory of
$N$ flat electric D3 branes is a $U(N)$ gauge theory with six scalars
in the adjoint representation of the gauge group.  There are no
fermionic fields.  The classical action is derived by a dimensional
reduction of the pure $SU(N)$ gauge theory action in ten dimensions.
The six scalars are the components of the gauge fields in the reduced
dimensions. The
 classical theory has an $SO(6)$ global symmetry that rotates the
six scalars.  This allows several possible parameters (from the point
of view of renormalizable field theory)~: a gauge coupling $g_{YM}$, a
mass parameter for the scalars $m$ and various scalar quartic
potential couplings $g_i$, one of which appears in the classical 
  Lagrangian.  In the classical worldvolume 
 action, the mass
parameter is zero and the $g_i$ are fixed in terms of $g_{YM}$, it
is just the dimensional reduction of the ten dimensional 
bosonic Yang-Mills theory.
Quantum mechanically, the parameters are corrected differently and can
take independent values.  The theory has a phase diagram depending on
these parameters.  Generically we expect to see in the diagram
Coulomb-like (Higgs) phases, confinement phases and maybe non trivial
RG fixed points arising from particular tunings of the parameters.

As in the case of D branes in Type II theories, one
can conjecture here that the low-energy theory on the electric D3 branes 
has a dual non supersymmetric
string description. At first sight this should   involve a
solution of $AdS_5\times S^5$ type. 
The closed string tachyon might be allowed in $AdS$ if the curvature
is of the order of the string scale, since in that case the 
tachyon would obey the  Breitenlohner-Freedman bound (\ref{posbound}).
The fact that the curvatures are of the order of the string scale
renders the gravity analysis invalid. In principle we should
solve the worldsheet string theory. Since we  do not know how to 
do that at present
we can just do a gravity analysis and hope that the full string
theory analysis will give similar results. 
It was observed  in \cite{Klebanov:1998db} that the tachyon 
potential includes the terms
\beq
 {1 \over 2} m^2 e^{-2 \Phi} {\cT}^2 + 
|\cF|^2 \left( 1 + \cT  + { {\cT}^2 \over 2}  \right) \ ,
\label{Tcouple}
\eeq
where $\cF$ is the electric  RR five form field strength (the magnetic
one  
couples in a similar way but with $\cT \to - \cT$). 
The fact that the RR fields contribute positively to the mass
allows  curvatures which, numerically, 
are a bit less than the string scale. Furthermore, 
it has been noticed  in 
\cite{Klebanov:1998yy} that the first string correction to this
background seems to vanish. These conditions on the curvature 
translate into the condition  $g_sN < O(1)$ which is precisely what
we expect to get in QCD.
% From (\ref{Tcouple})
%that we need  $g_s^2 l_s^2 |\cF|^2 > O(1)$.
% Since $\cF \sim N$ we get $g_sN < O(1)$. \cite{Klebanov:1998af}
%
%  The coupling of $\cT$
%and $\cF$ in (\ref{Tcouple}) can shift the effective tachyon mass to
%be positive.  Assuming the overall scale of the metric to be
%$(g_sN)^{1/2}$ results in a stability condition $g_s^2 l_s^2 |\cF|^2 >
%O(1)$.  Since $\cF \sim N$ we get $g_sN < O(1)$.  Therefore the
%stability condition appears to hold for small bare 't Hooft coupling,
%which is the correct limit to take in order to get to QCD.  However,
%in this limit the curvatures in string units are not small and we can
%have large corrections to the supergravity computations.


An interesting feature is that,
due to the potential (\ref{Tcouple}) the tachyon would have a nonzero
expectation value and that 
 induces a variation of the dilaton field $\Phi$
in the radial coordinate via the equation 
\cite{Minahan:1999tm,Klebanov:1998yy}
\beq
\nabla^2 \Phi = \frac{1}{8}m^2 e^{\Phi/2}\cT^2,~~~~~~m^2 = 
-\frac{2}{\alpha'} \ .
\eeq
Since the radial coordinate is associated with the energy scale of the
gauge theory, this variation may be interpreted as the flow of the
coupling.  In the UV (large radial coordinate) the tachyon is constant
and one finds a metric of the form $AdS_5 \times S^5$. This indicates
a UV fixed point. The coupling vanishes at the UV fixed point, and
this makes the curvature of the gravity solution infinite in the UV, but
that is precisely what is expected since the field theory is UV free.
The running of the coupling is logarithmic, though it  goes
like $ 1/(\log E)^2$. However,  the quark-antiquark potential goes as
 $1/\log E$ due to the square root in \energy .  

%The gauge theory is asymptotically free. At short distances
%the theory is weakly coupled and we expect the dual gravity 
%description not to be valid.
%This can happen if the gravity background is singular, 
%as in the models 
%based on heating up SCFTs to high temperature that we 
%discussed previously.
%In the case at hand the signal for the gravity description not being 
%adequate
%is that the $\alpha'$ corrections are of the same order as the 
%leading order gravity contributions \cite{Klebanov:1998af}.
%For large $N$ the curvature in string units is small but the
%$\alpha'$ corrections
%associated with the Weyl tensor are not suppressed. 
%The effective string coupling is small and
% therefore we expect the classical
%string description to be applicable.
%This still has an advantage over the finite temperature case in 
%that we can see
%qualitatively the asymptotic freedom property already in the leading order
%gravity description.


In the IR (small radial coordinate) the tachyon vanishes and one finds
again a solution of the form $AdS_5 \times S^5$.  In the IR the
coupling is infinite.  Therefore this solution seems to exhibit a
strong coupling IR fixed point.  However, since the dilaton is large,
classical string theory is not sufficient to study the fixed point
theory.  The gravity solution at all energy scales $u$ has not been
constructed yet.

Generically one expects
the gauge theory to have different phases parametrized by the possible  
couplings.
The IR fixed point should occur as a particular tuning of the 
couplings.
Indeed, other solutions at small radial coordinate were constructed
in \cite{Minahan:1998af} that  exhibit confinement and a mass gap.
Moreover they were argued to be more generic 
than the IR fixed point solution.


It was pointed out in \cite{Nekrasov:1999mn} that the theories on
 the D3 branes
of Type 0B string theory are particular examples of the orbifold models 
of $\cN=4$ theory that we studied in section \ref{orbifolds}.
The R-symmetry of $\cN=4$ theory is $SU(4)$, the spin cover of $SO(6)$.
It has a center $\IZ_4$ and one can orbifold with respect to it or 
its subgroups $\Gamma$.
The theory on $N$ flat electric D3 branes arises when the action
 of $\Gamma$
on the Chan-Paton (color) indices is in a trivial representation.
This orbifold is not in the class of ``regular representations'' which
we discussed in section \ref{orbifolds}; in particular, in this case the
beta function does not vanish in the planar diagram limit. If we study
instead the theory arising on $N$ self-dual D3-branes of type 0 (which may
be viewed as bound states of electric and magnetic D3-branes) we find
a theory which is in the class of 
``regular representation orbifolds'' \cite{Klebanov:1999ch},
and behaves similarly to type II D3-branes in the large $N$ limit. We will
not discuss this theory here.
    
As with the D branes in Type II string theory, we can 
construct a large number of non supersymmetric
models in Type 0 theories by placing the D branes at singularities.
One example is the theory of
D3 branes of Type 0B string theory at a conifold singularity.
As discussed in section \ref{conifolds},
when placing $N$ D3 branes of Type IIB string theory at a conifold
the resulting low-energy worldvolume theory is $\cN=1$ supersymmetric
$SU(N) \times SU(N)$ gauge theory  
with chiral superfields $A_k,k=1,2$ transforming in the $(N,\bar{N})$
representation and $B_l, l=1,2$ transforming in the $(\bar{N},N)$ 
representation, and with some superpotential.


On the worldvolume of $N$ electric D3 branes of Type 0B string theory
at a conifold there is a truncation of the fermions and one gets an
$SU(N) \times SU(N)$ gauge theory with complex scalar fields
$A_k,k=1,2$ transforming in the $(N,\bar{N})$ representation and $B_l,
l=1,2$ transforming in the $(\bar{N},N)$ representation.  This theory
(at least if we set to zero the coefficient of the scalar potential
which existed in the supersymmetric case) is asymptotically free.  The
gravity description of this model has been analyzed in
\cite{Mohsen:1999ba}.  In the UV one finds a solution of the form
$AdS_5 \times T^{1,1}$ which indicates a UV fixed point.  The
effective string coupling vanishes in accord with the UV freedom of
the gauge theory.  In the IR one finds again a solution of the form
$AdS_5 \times T^{1,1}$ with infinite coupling that points to a strong
coupling IR fixed point.  Of course, one expects the gauge theory to
have different phases parametrized by the possible couplings.  Indeed,
there are other more generic solutions that exhibit confinement and
a mass gap \cite{Mohsen:1999ba}.

Other works on dual descriptions of gauge theories via the Type 0 D branes
are \cite{Ferretti:1998xu,Zarembo:1999hn,Kogan:1999gi,%
Tseytlin:1999ii,Armoni:1999fb,%
Ferretti:1999gj,Costa:1999qx}.
\newpage











