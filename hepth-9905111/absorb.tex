\subsection{Greybody Factors and Black Holes} 
\label{gbFactorsBH}

An important precursor to the AdS/CFT correspondence was the calculation of
greybody factors for black holes built out of D-branes.  It was noted in
\cite{Callan:1996dv} that Hawking radiation could be mimicked by processes
where two open strings collide on a D-brane and form a closed string which
propagates into the bulk.  The classic computation of Hawking (see, for
example, \cite{Hawking:1974df} for details) shows in a semi-classical
approximation that the differential rate of spontaneous emission of
particles of energy $\omega$ from a black hole is
  \eqn{HawkingRadiation}{
   d\Gamma_{\rm emit} = {v 
\sigma_{\rm absorb} \over e^{\omega/T_H} \pm 1} 
     {d^n k \over (2\pi)^n} \ ,
  }
where $v$ is the velocity of the emitted particle in the 
transverse directions, and
 the sign in the denominator is minus for bosons and plus for
fermions.  We use $n$ to denote the number of spatial dimensions around the
black hole (or if we are dealing with a black brane, it is the number of
spatial dimensions perpendicular to the world-volume of the brane).  $T_H$
is the Hawking temperature, and $\sigma_{\rm absorb}$ is the cross-section
for a particle coming in from infinity to be absorbed by the black hole.
In the differential emission rate, the emitted particle is required to have
a momentum in a small region $d^n k$, and $\omega$ is a function of
$k$. To obtain a total emission rate we
would integrate \HawkingRadiation\ over all $k$.

If $\sigma_{\rm absorb}$ were a constant, then \HawkingRadiation\ tells us
that the emission spectrum is the same as that of a blackbody.  Typically,
$\sigma_{\rm absorb}$ is not constant, but varies appreciably over the
range of finite $\omega/T_H$.  The consequent deviations from the pure
blackbody spectrum have earned $\sigma_{\rm absorb}$ the name ``greybody
factor.''  A successful microscopic account of black hole thermodynamics
should be able to predict these greybody factors.  In \cite{Das:1996wn} and
its many successors, it was shown that the D-branes provided an account of
black hole microstates which was successful in this respect.

Our first goal will be to see how greybody factors are computed in the
context of quantum fields in curved spacetime.  The literature on this
subject is immense.  We refer the reader to \cite{Matzner} for an overview
of the General Relativity literature, and to
\cite{Maldacena:1997ix,Gubser:1997yh,Peet:1997es} and references therein
for a first look at the string theory additions.

In studying scattering of particles off of a black hole (or any fixed
target), it is convenient to make a partial wave expansion.  For
simplicity, let us restrict the discussion to scalar fields.  Assuming that
the black hole is spherically symmetric, one can write the asymptotic
behavior at infinity of the time-independent scattering solution as
  \eqn{PartialWave}{\eqalign{
   \phi(\vec{r}) &\sim e^{ikx} + 
f(\theta) {e^{ikr} \over r^{n/2}}  \cr
    &\sim \sum_{\ell = 0}^\infty \tf{1}{2} \tilde{P}_\ell(\cos \theta)
     {S_\ell e^{i k r} + (-1)^\ell i^n e^{-i k r} \over
      (i k r)^{n/2}} \ ,
  }}
 where $x = r\cos \theta$.  The term $e^{ikx}$ represents the incident
wave, and the second term in the first line represents the scattered wave.
The $\tilde{P}_\ell(\cos\theta)$ are generalizations of Legendre
polynomials.  The absorption probability for a given
partial wave is given by
$P_\ell = 1-|S_\ell|^2$.  An application of the Optical
Theorem leads to the absorption cross section \cite{Gubser:1997qr}
  \eqn{OptTheorem}{
   \sigma_{\rm abs}^\ell = 
    {2^{n-1} \pi^{{n-1 \over 2}} \over k^n}
    \Gamma\left( {n-1 \over 2} \right) \left( \ell + {n-1 \over 2} \right) 
     {\ell+n-2 \choose \ell} P_\ell \ .
  }
 Sometimes the absorption probability $P_\ell$ is called the greybody
factor.

The strategy of absorption calculations in supergravity is to solve a
linearized wave equation, most often the Klein-Gordon
 equation $\square \phi =
0$, using separation of variables, $\phi = e^{-i\omega t}
P_\ell(\cos\theta) R(r)$.  Typically the radial function cannot be
expressed in terms of known functions, so some approximation scheme is
used, as we will explain in more detail below.  Boundary conditions are
imposed at the black hole horizon corresponding to infalling matter.  Once
the solution is obtained, one can either use the asymptotics \PartialWave\
to obtain $S_\ell$ and from it $P_\ell$ and $\sigma_{\rm abs}^\ell$, or
compute the particle flux at infinity and at the horizon and note that
particle number conservation implies that $P_\ell$ is their ratio.

One of the few known universal results is that for $\omega/T_H \ll 1$,
$\sigma_{\rm abs}$ for an $s$-wave massless scalar approaches the horizon
area of the black hole \cite{Das:1997we}.  This result holds for any
spherically symmetric black hole in any dimension.  For $\omega$ much
larger than any characteristic curvature scale of the geometry, one can use
the geometric optics approximation to find $\sigma_{\rm abs}$.

We will be interested in the particular black hole geometries for which
string theory provides a candidate description of the microstates.  Let us
start with $N$ coincident D3-branes, where the low-energy world-volume
theory is $d=4$ ${\cal N}=4$ $U(N)$ gauge theory.  The equation of motion
for the dilaton is $\square\phi = 0$ where $\square$ is the laplacian for
the metric 
  \eqn{D3Metric}{
   ds^2 = \left( 1 + {R^4 \over r^4} \right)^{-1/2} 
    \left( -dt^2 + dx_1^2 + dx_2^2 + dx_3^2 \right) + 
    \left( 1 + {R^4 \over r^4} \right)^{1/2} 
    \left( dr^2 + r^2 d\Omega_5^2 \right) \ .
  }
 It is convenient to change radial variables: $r = R e^{-z}$, $\phi =
e^{2z} \psi$.  The radial equation for the $\ell^{\rm th}$ partial wave is
  \eqn{LthPartial}{
   \left[ \partial_z^2 + 2 \omega^2 R^2 \cosh 2z - (\ell+2)^2 \right]
    \psi_\ell(z) = 0 \ ,
  }
 which is precisely Schrodinger's equation with a potential $V(z) = - 2
\omega^2 R^2 \cosh 2z$.  The absorption probability is precisely the
tunneling probability for the barrier $V(z)$: the transmitted wave at large
positive $z$ represents particles falling into the D3-branes.  At leading
order in small $\omega R$, the absorption probability for the $\ell^{\rm
th}$ partial wave is
  \eqn{LowCross}{
   P_\ell = {4\pi^2 \over (\ell+1)!^4 (\ell+2)^2} 
    \left( {\omega R \over 2} \right)^{8 + 4\ell} \ .
  }
 This result, together with a recursive algorithm for computing all
corrections as a series in $\omega R$, was obtained in \cite{Gubser:1998iu}
from properties of associated Mathieu functions, which are the solutions of
\LthPartial.  An exact solution of a radial equation in terms of known
special functions is rare.  We will therefore present a standard
approximation technique (developed in \cite{Unruh:1976fm} and applied to
the problem at hand in \cite{Klebanov:1997kc}) which is sufficient to
obtain the leading term of \LowCross.  Besides, for comparison with string
theory predictions we are generally interested only in this leading term.

The idea is to find limiting forms of the radial equation which can be
solved exactly, and then to match the limiting solutions together to
approximate the full solution.  Usually a uniformly good approximation can
be found in the limit of small energy.  The reason, intuitively speaking,
is that on a compact range of radii excluding asymptotic infinity and the
horizon, the zero energy solution is nearly equal to solutions with very
small energy; and outside this region the wave equation usually has a
simple limiting form.  So one solves the equation in various regions and
then matches together a global solution.

It is elementary to show that this can be done for \LthPartial\ using two
regions:
  \eqn{MatchSoln}{\seqalign{\span\TT \qquad & \span\TR}{
   far region: $z \gg \log \omega R$ \qquad & 
    \eqalign{& \left[ \partial_z^2 + 
      \omega^2 R^2 e^{2z} - (\ell+2)^2 \right] \psi = 0  
       \cr\noalign{\vskip-0.5\jot}
      & \quad \psi(z) = H^{(1)}_{\ell+2}(\omega R e^z)}
     \cr\noalign{\vskip2\jot}
   near region: $z \ll -\log \omega R$ \qquad & 
    \eqalign{& \left[ \partial_z^2 + 
      \omega^2 R^2 e^{-2z} - (\ell+2)^2 \right] \psi = 0  
       \cr\noalign{\vskip-0.5\jot}
      & \quad \psi(z) = a J_{\ell+2}(\omega R e^{-z})}
  }}
 It is amusing to note the $\IZ_2$ symmetry, $z \to -z$, which exchanges
the far region, where the first equation in \MatchSoln\ is just free
particle propagation in flat space, and the near region, where the second
equation in \MatchSoln\ describes a free particle in $AdS_5$.  This
peculiar symmetry was first pointed out in \cite{Klebanov:1997kc}.  It
follows from the fact that the full D3-brane metric comes back to itself,
up to a conformal rescaling, if one sends $r \to R^2/r$.  A similar duality
exists between six-dimensional flat space and $AdS_3 \times S^3$ in the
D1-D5-brane solution, where the Laplace equation again can be solved in
terms of Mathieu functions \cite{ghUnp,Cvetic:1999fv}.  To our knowledge
there is no deep understanding of this ``inversion duality.''

 For low energies $\omega R \ll 1$, the near and far regions overlap in a
large domain, $\log \omega R \ll z \ll -\log \omega R$, and by comparing
the solutions in this overlap region one can fix $a$ and reproduce the
leading term in \LowCross.  It is possible but tedious to obtain the
leading correction by treating the small terms which were dropped from the
potential to obtain the limiting forms in \MatchSoln\ as perturbations.
This strategy was pursued in \cite{Gubser:1998kv,Taylor-Robinson:1998tk}
before the exact solution was known, and in cases where there is no exact
solution.  The validity of the matching technique is discussed in
\cite{Matzner}, but we know of no rigorous proof that it holds in all the
circumstances in which it has been applied.

The successful comparison of the $s$-wave dilaton cross-section in
\cite{Klebanov:1997kc} with a perturbative calculation on the D3-brane
world-volume was the first hint that Green's functions of ${\cal N}=4$
super-Yang-Mills theory could be computed from supergravity.  In
summarizing the calculation, we will follow more closely the conventions of
\cite{Gubser:1997yh}, and give an indication of the first application of
non-renormalization arguments \cite{Gubser:1997se} to understand why the
agreement between supergravity and perturbative gauge theory existed
despite their applicability in opposite limits of the 't~Hooft coupling.

Setting normalization conventions so that the pole in the propagator of the
gauge bosons has residue one at tree level, we have the following action
for the dilaton plus the fields on the brane:
  \eqn{BBAction}{
   S = {1 \over 2\kappa^2} \int d^{10} x \sqrt{g} \, \left[ {\cal R} - 
    \tf{1}{2} (\partial\phi)^2 + \ldots \right] +
    \int d^4 x \, \left[ -\tf{1}{4} e^{-\phi} \tr F_{\mu\nu}^2 + 
      \ldots \right] \ ,
  }
 where we have omitted other supergravity fields, their interactions with
one another, and also terms with the lower spin fields
in the gauge theory action.  A
plane wave of dilatons with energy $\omega$ and momentum perpendicular to
the brane is kinematically equivalent on the world-volume to a massive
particle which can decay into two 
gauge bosons through the coupling ${1 \over
4} \phi \tr F_{\mu\nu}^2$.  In fact, the absorption cross-section is given
precisely by the usual expression for the decay rate into $k$ particles:
  \eqn{CrossFeynman}{
   \sigma_{\rm abs} = {1 \over S_f} {1 \over 2\omega} 
    \int {d^3 p_1 \over (2\pi)^3 2\omega_1} \ldots
     {d^3 p_k \over (2\pi)^3 2\omega_k} (2\pi)^4 \delta^4(P_f - P_i)
     \overline{|{\cal M}|}^2 \ .
  }
 In the Feynman rules for ${\cal M}$, a factor of $\sqrt{2\kappa^2}$
attaches to an external dilaton line on account of the non-standard
normalization of its kinetic term in \BBAction.  This factor gives
$\sigma_{\rm abs}$ the correct dimensions: it is a length to the fifth
power, as appropriate for six transverse spatial dimensions.  In
\CrossFeynman, $\overline{|{\cal M}|}^2$ indicates summation over
distinguishable processes: in the case of the $s$-wave dilaton there are
$N^2$ such processes because of the number of gauge bosons.  One easily
verifies that $\overline{|{\cal M}|}^2 = N^2 \kappa^2 \omega^4$.  $S_f$ is
a symmetry factor for identical particles in the final state: in the case
of the $s$-wave dilaton, $S_f = 2$ because the outgoing gauge bosons are
identical.

Carrying out the $\ell=0$ calculation explicitly, one finds
  \eqn{CrossFinal}{
   \sigma_{\rm abs} = {N^2 \kappa^2 \omega^3 \over 32\pi} \ ,
  }
 which, using \OptTheorem\ and the relation between $R$ and $N$, 
can be shown to be in precise agreement with the
leading term of $P_0$ in \LowCross.  This is now understood to be due to a
non-renormalization theorem for the two-point function of the operator
${\cal O}_4 = {1 \over 4} \tr F^2$.

To understand the connection with two-point functions, note that an
absorption calculation is insensitive to the final state on the D-brane
world-volume.  The absorption cross-section is therefore related to the
discontinuity in the cut of the two-point function of the operator to which
the external field couples.  To state a result of some generality, let us
suppose that a scalar field $\phi$ in ten dimensions couples to a gauge
theory operator through the action
  \eqn{SintDef}{
   S_{\rm int} = \int d^4 x \, \partial_{y_{i_1}} \cdots 
    \partial_{y_{i_\ell}} \phi(x,y_i)\Big|_{y_i = 0} 
    {\cal O}^{i_1 \ldots i_\ell}(x) \ ,
  }
 where we use $x$ to denote the four coordinates parallel to the
world-volume and $y_i$ to denote the other six.  An example where this
would be the right sort of coupling is the $\ell^{\rm th}$ partial wave of
the dilaton \cite{Gubser:1997yh}.  The $\ell^{\rm th}$ partial wave
absorption cross-section for a particle with initial momentum $p = \omega
(\hat{t} + \hat{y}_1)$ is obtained by summing over all final states
that could be created by the operator ${\cal O}^{1\ldots
1}$:\footnote{There is one restriction on the final states: for a process
to be regarded as an $\ell^{\rm th}$ partial wave absorption cross-section,
$\ell$ units of angular momentum must be picked up by the brane.  Thus
${\cal O}^{i_1 \ldots i_\ell}$ must transform in the irreducible
representation which is the $\ell^{\rm th}$ traceless symmetric power of
the ${\bf 6}$ of $SO(6)$.}
  \eqn{SigmaPartial}{\eqalign{
   \sigma_{\rm abs} &= {1 \over 2\omega} \sum_n \prod_{i=1}^n {1 \over S_f}
    {d^3 p_i \over (2\pi)^3 2\omega_i} (2\pi)^4 \delta^4(P_f - P_i)
    \overline{|{\cal M}|}^2  \cr
    &= {2\kappa^2 \omega^\ell \over 2i\omega} {\rm Disc} \, 
     \int d^4 x \, e^{i p \cdot x} \langle {\cal O}^{1\ldots 1}(x)
      {\cal O}^{1\ldots 1}(0) \rangle \Big|_{p=(\omega,0,0,0)} \ .
  }}
 In the second equality we have applied the Optical Theorem (see
figure~\ref{OptTh}).  
  \begin{figure}
   \vskip0cm
   $$
    \sum_X \left| 
       \ooalign{\lower0.25in\hbox{\smash{\psfig{figure=lhsOptTh.eps}}}}\ 
      \right|^2 = 
     {2\kappa^2 \over i} {\rm Disc} \, 
       \ooalign{\lower0.34in\hbox{\smash{\psfig{figure=rhsOptTh.eps}}}}
   $$
   \vskip0cm
 \caption{An application of the optical theorem.}\label{OptTh}
  \end{figure}
  \comment{had to hack at this one with ooalign.  Anyone with TeX tricks is
welcome to show me a better way.  SSG}
 The factor of $2\kappa^2$ is the square of the external leg factor for the
incoming closed string state, which was included in the invariant amplitude
${\cal M}$.  The factor of $\omega^\ell$ arises from acting with the $\ell$
derivatives in \SintDef\ on the incoming plane wave state.  The symbol
${\rm Disc}$ indicates that one is looking at the unitarity cut in the
two-point function in the $s$ plane, where $s = p^2$.  The two-point
function can be reconstructed uniquely from this cut, together with some
weak conditions on regularity for large momenta.  Results analogous to
\SigmaPartial\ can be stated for incoming particles with spin, only it
becomes more complicated because a polarization must be specified, and the
two-point function in momentum space includes a polynomial in $p$ which
depends on the polarization.

Expressing absorption cross-sections in terms of two-point functions helps
illustrate why there is ever agreement between the tree-level calculation
indicated in \CrossFeynman\ and the supergravity result, which one would
{\it a priori} expect to pick up all sorts of radiative corrections.
Indeed, it was observed in \cite{Gubser:1997se} that the $s$-wave graviton
cross-section agreed between supergravity and tree-level gauge theory
because the correlator $\langle T_{\alpha\beta} T_{\gamma\delta} \rangle$
enjoys a non-renormalization theorem.  One way to see that there must be
such a non-renormalization theorem is to note that conformal Ward
identities relate this two-point function to $\langle T^\mu_\mu
T_{\alpha\beta} T_{\gamma\delta} \rangle$ (see for example
\cite{Erdmenger:1997yc} for the details), and supersymmetry in turn relates
this anomalous three-point function to the anomalous VEV's of the
divergence of R-currents in the presence of external sources for them.  The
Adler-Bardeen theorem \cite{Adler:1969er} protects these anomalies, hence
the conclusion.

Another case which has been studied extensively is a system consisting of
several D1 and D5 branes.  The D1-branes are delocalized on the four extra
dimensions of the D5-brane, which are taken to be small, so that the total
system is effectively 1+1-dimensional.  We will discuss the physics of this
system more extensively in chapter~\ref{ChapAdS3}, and the reader can also
find background material in \cite{Peet:1997es}.  For now our purpose is to
show how supergravity absorption calculations relate to finite-temperature
Green's functions in the 1+1-dimensional theory.

Introducing momentum along the spatial world-volume (carried by open
strings attached to the branes), one obtains the following ten-dimensional
metric and dilaton:
  \eqn{TenSol}{\eqalign{
   ds_{10,{\rm str}}^2 &= H_1^{-1/2} H_5^{-1/2} \left[ -dt^2 + dx_5^2 + 
    {r_0^2 \over r^2} (\cosh\sigma dt + \sinh\sigma dx_5)^2 \right]  \cr 
      & \qquad + 
    H_1^{1/2} H_5^{-1/2} \sum_{i=1}^4 dy_i^2 +
    H_1^{1/2} H_5^{1/2} 
    \left[ \left( 1 - {r_0^2 \over r^2} \right)^{-1} dr^2 + 
    r^2 d\Omega_3^2 \right]  \cr
   e^{\phi-\phi_\infty} &= H_1^{1/2} H_5^{-1/2}  \cr
   H_1 &= 1 + {r_1^2 \over r^2} \qquad H_5 = 1 + {r_5^2 \over r^2} \ .
  }}
 The quantities $r_1^2$, $r_5^2$, and $r_K^2 = r_0^2 \sinh^2\sigma$ are
related to the number of D1-branes, the number of D5-branes, and the net
number of units of momentum in the $x_5$ direction, respectively.  The
horizon radius, $r_0$, is related to the non-extremality.  For details, see
for example \cite{Maldacena:1997ix}.  If $r_0 = 0$ then there are only
left-moving open strings on the world-volume; if $r_0 \neq 0$ then there
are both left-movers and right-movers.  The Hawking temperature can be
expressed as ${2 \over T_H} = {1 \over T_L} + {1 \over T_R}$, where
  \eqn{TLR}{
   T_L = {1 \over \pi} {r_0 e^\sigma \over 2 r_1 r_5} \qquad
   T_R = {1 \over \pi} {r_0 e^{-\sigma} \over 2 r_1 r_5} \ .
  }
 $T_L$ and $T_R$ have the interpretation of temperatures of the left-moving
and right-moving sectors of the 1+1-dimensional world-volume theory.  There
is a detailed and remarkably successful account of the Bekenstein-Hawking
entropy using statistical mechanics in the world-volume theory.  It was
initiated in \cite{Strominger:1996sh}, developed in a number of subsequent
papers, and has been reviewed in \cite{Peet:1997es}.

The region of parameter space which we will be interested in is
  \eqn{DiluteGas}{
   r_0,r_K \ll r_1,r_5
  }
 This is known as the dilute gas regime.  The total energy of the open
strings on the branes is much less than the constituent mass of either the
D1-branes or the D5-branes.
 We are also interested in low energies
$\omega r_1, \omega r_5 \ll 1$, but $\omega/T_{L,R}$ can be arbitrary
thanks to \DiluteGas , \TLR . 
% The reason for considering this regime \DiluteGas\ will become
% clearer in section \ref{adsthree}.
  The D1-D5-brane system is not stable because
left-moving open strings
   can run into right-moving open string and form a
closed string: indeed, this is exactly the process we aim to quantify.
Since we have collisions of left and right moving excitations we
expect that the answer will contain the left and right moving occupation
factors, so that the emission rate is 
\eqn{emission}{
d\Gamma = g^2_{eff} { 1 \over (e^{ \omega \over 2 T_L} -1) }
 { 1 \over (e^{ \omega \over 2 T_R} -1) }{d^4k \over (2\pi)^4 }
}
where $g_{eff}$ is independent of the temperature and measures the 
coupling of the open strings to the closed strings. 
The functional form of \emission\ seems, at first sight,
 to be different from \HawkingRadiation . But 
in order to compare them we should calculate 
the absorption cross section
appearing in \HawkingRadiation .

%This calculation is most easily done by 
%compactifying down to five dimensions.  We arrive at the Einstein-frame
%metric 
%  \eqn{FiveEinstein}{\eqalign{
%   ds_{5,{\rm Ein}}^2 &= -{h \over F^{2/3}} dt^2 + 
%    F^{1/3} \left[ {1 \over h} dr^2 + r^2 d\Omega_3^2 \right],  \cr
%   F &= H_1 H_5 H_K, \qquad H_K = 1 + {r_K^2 \over r^2}, \qquad
%   h = 1 - {r_0^2 \over r^2},
%  }}
% with $H_1$ and $H_5$ as above.  

Off-diagonal gravitons $h_{y_1 y_2}$
(with $y_{1,2}$ in the compact directions) reduce
to scalars in six dimensions which obey the massless Klein
Gordon  equation.  These
so-called 
minimal scalars have been the subject of the most detailed study.
We will consider only the $s$-wave and we take the momentum along the 
string to be zero.   Separation of variables leads to the
radial equation
  \eqn{RadialLaplace1}{\eqalign{
   & \left[ {h \over r^3} \partial_r hr^2 \partial_r + 
     \omega^2 f \right] R = 0 \ , \cr
  & ~~h = 1 - {r_0^2 \over r^2},~~~~f =
 \left( 1 + {r_1^2\over r^2} \right) 
 \left( 1 + {r_5^2\over r^2} \right)  \left( 1 + {r_0^2\sinh^2 \sigma
\over r^2} \right)  ~ .
  }}
 Close to the horizon, a convenient radial variable is $z = h = 1 -
r_0^2/r^2$.  The matching procedure can be summarized as follows:
  \eqn{MatchSolnAgain1}{\seqalign{\span\TT \qquad & \span\TR}{
   far region:  & 
    \eqalign{& \left[ {1 \over r^3} \partial_r r^3 \partial_r + 
      \omega^2 \right] R = 0  
       \cr\noalign{\vskip-0.5\jot}
      & \quad R = A {J_1(\omega r) \over r^{3/2}},}
     \cr\noalign{\vskip2\jot}
   near region:  & 
    \eqalign{& \left[ z(1-z) \partial_z^2 + 
      \left( 1 - i {\omega \over 2\pi T_H} \right) (1-z) \partial_z + 
      {\omega^2 \over 16 \pi^2 T_L T_R} \right] 
       z^{i\omega \over 4\pi T_H} R = 0
       \cr\noalign{\vskip-0.5\jot}
      & \quad R = z^{-{i \omega \over 4\pi T_H}}
       F\left( -i {\omega \over 4\pi T_L}, -i {\omega \over 4\pi T_R};
        1 - i {\omega \over 2\pi T_H}; z \right).  }
  }}
 After matching the near and far regions together and comparing the
infalling flux at infinity and at the horizon, one arrives at
  \eqn{SigmaSwave}{
   \sigma_{\rm abs} = \pi^3 r_1^2 r_5^2 \omega
    {e^{\omega \over T_H} - 1 \over 
     \left( e^{\omega \over 2 T_L} - 1 \right) 
     \left( e^{\omega \over 2 T_R} - 1 \right)} \ .
  }
This has precisely the right form to ensure the matching of 
\emission\ with \HawkingRadiation\  (note that for a massless particle
with no momentum along the black string $v =1$ in \HawkingRadiation ).
It is possible to be more precise and  calculate the coefficient in 
\emission\  and actually check that the matching is precise 
\cite{Das:1996wn}. We leave 
this to chapter \ref{ChapAdS3}.

\begin{figure}[htb]
\begin{center}
\epsfxsize=3.5in\leavevmode\epsfbox{nature.eps}
\end{center}
\caption{ Low energy dynamics of extremal or near-extremal black branes. 
$r_5$ denotes the typical gravitational size of the system, namely
the position where the metric significantly deviates from 
Minkowski space. The Compton wavelength of 
the particles we scatter is much larger than the gravitational size,
$ \lambda \gg r_5$. In this situation we replace the whole black 
hole geometry (a)  by a point-like system in the transverse coordinates 
 with localized excitations (b).
These excitations are the ones described by the field theory living on
the brane. 
}
\label{nature}
\end{figure} 

Both
in the D3-brane analysis and in the D1-D5-brane analysis, one can see that
all the interesting physics is resulting from the near-horizon region: the
far region wave-function describes free particle propagation.  For quanta
whose Compton wavelength is much larger than the size of the black hole,
the effect of the far region is merely to set the boundary conditions in
the near region. See figure \ref{nature}. 
This provides a motivation for the prescription for
computing Green's functions, to be described in
section \ref{correlators}: as the calculations of this section
demonstrate, cutting out the near-horizon region of the supergravity
geometry and replacing it with the D-branes leads to an identical response
to low-energy external probes.

