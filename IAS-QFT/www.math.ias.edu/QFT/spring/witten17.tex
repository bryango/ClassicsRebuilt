%From: John W Morgan <jmorgan@math.ias.edu>
%Date: Tue, 10 Jun 1997 14:31:56 -0400
%Subject: Lecture II-17

\documentclass[10pt]{article}
\usepackage{amssymb}


%These are the macros which are in common with all of the
% sections in the paper mmr
% Each section, for now, should begin with \documentstyle[11pt,cd]{article}
% and then have \input{mmrmacros} followed by \begin{document}
% The only exception is that the \Label macro is slightly different
% in each file and should be put in separately.
%New CD macros
\newcommand{\cdrl}{\cd\rightleftarrows}
\newcommand{\cdlr}{\cd\leftrightarrows}
\newcommand{\cdr}{\cd\rightarrow}
\newcommand{\cdl}{\cd\leftarrow}
\newcommand{\cdu}{\cd\uparrow}
\newcommand{\cdd}{\cd\downarrow}
\newcommand{\cdud}{\cd\updownarrows}
\newcommand{\cddu}{\cd\downuparrows}
% (S) Proofs.
% (S-1) Head is automatically supplied by \proof.

\def\proof{\vspace{2ex}\noindent{\bf Proof.} }
\def\tproof#1{\vspace{2ex}\noindent{\bf Proof of Theorem #1.} }
\def\pproof#1{\vspace{2ex}\noindent{\bf Proof of Proposition #1.} }
\def\lproof#1{\vspace{2ex}\noindent{\bf Proof of Lemma #1.} }
\def\cproof#1{\vspace{2ex}\noindent{\bf Proof of Corollary #1.} }
\def\clproof#1{\vspace{2ex}\noindent{\bf Proof of Claim #1.} }
% End of Proof Symbol at the end of an equation must precede $$.

\def\endproof{\relax\ifmmode\expandafter\endproofmath\else
  \unskip\nobreak\hfil\penalty50\hskip.75em\hbox{}\nobreak\hfil\bull
  {\parfillskip=0pt \finalhyphendemerits=0 \bigbreak}\fi}
\def\endproofmath$${\eqno\bull$$\bigbreak}
\def\bull{\vbox{\hrule\hbox{\vrule\kern3pt\vbox{\kern6pt}\kern3pt\vrule}\hrule}}
\addtolength{\textwidth}{1in}                  % Margin-setting commands
\addtolength{\oddsidemargin}{-.5in}
\addtolength{\evensidemargin}{.5in}
\addtolength{\textheight}{.5in}
\addtolength{\topmargin}{-.3in}
\addtolength{\marginparwidth}{-.32in}
\renewcommand{\baselinestretch}{1.6}
\def\hu#1#2#3{\hbox{$H^{#1}(#2;{\bf #3})$}}          % #1-Cohomology of #2
\def\hl#1#2#3{\hbox{$H_{#1}(#2;{\bf #3})$}}          % #1-Homology of #2
\def\md#1{\ifmmode{\cal M}_\delta(#1)\else  % moduli space, delta decay of #1
{${\cal M}_\delta(#1)$}\fi}
\def\mb#1{\ifmmode{\cal M}_\delta^0(#1)\else  %moduli space, based, delta
					      %decay of #1
{${\cal M}_\delta^0(#1)$}\fi}
\def\mdc#1#2{\ifmmode{\cal M}_{\delta,#1}(#2)\else    %moduli space, delta
						      %decay, chern class #1
						      %of #2
{${\cal M}_{\delta,#1}(#2)$}\fi}
\def\mbc#1#2{\ifmmode{\cal M}_{\delta,#1}^0(#2)\else   %as before, based
{${\cal M}_{\delta,#1}^0(#2)$}\fi}
\def\mm{\ifmmode{\cal M}\else {${\cal M}$}\fi}
\def\ad{{\rm ad}}
\def\msigma{\ifmmode{\cal M}^\sigma\else {${\cal M}^\sigma$}\fi}
\def\cancel#1#2{\ooalign{$\hfil#1\mkern1mu/\hfil$\crcr$#1#2$}}
\def\dirac{\mathpalette\cancel\partial}
\newtheorem{thm}{Theorem}
\newtheorem{theorem}{Theorem}[subsection]
\newtheorem{proposition}[theorem]{Proposition}
\newtheorem{lemma}[theorem]{Lemma}
\newtheorem{claim}[theorem]{Claim}
\newtheorem{example}[theorem]{Example}
\newtheorem{corollary}[theorem]{Corollary}
\newtheorem{D}[theorem]{Definition}
\newenvironment{defn}{\begin{D} \rm }{\end{D}}
\newtheorem{addendum}[theorem]{Addendum}
\newtheorem{R}[theorem]{Remark}
\newenvironment{remark}{\begin{R}\rm }{\end{R}}
\newcommand{\note}[1]{\marginpar{\scriptsize #1 }} 
\newenvironment{comments}{\smallskip\noindent{\bf Comments:}\begin{enumerate}}{\end{enumerate}\smallskip}

\renewcommand{\thesection}{\Roman{section}}
\def\eqlabel#1{\addtocounter{theorem}{1}
\write1{\string\newlabel{#1}{{\thetheorem}{\thepage}}}
\leqno(\rm\thetheorem)}
\def\cS{{\cal S}}
\def\ov{\overline}












\title{Lecture II-17: $N=2$ Super-symmetric Yang-Mills theories in
dimension four: Part 1} 
\author{Edward Witten\thanks{Notes by John Morgan}}
\date{}
\begin{document}
\maketitle


\section{Introduction}

Today we shall be studying the dynamics of  $N=2$ super-symmetric
Yang-Mills theories in 
four-dimensions. The underlying manifold on which we shall study these
theories will be Minkowski four-space $M^4={\bf R}^{3,1}$. We are
mainly interested in the case when the gauge group $G$ is
$SU(2)$. Since low energy limits of $SU(2)$ theories can
generate $U(1)$-theories, we shall begin by
studying $U(1)$-Yang-Mills theory. One should note that these
$U(1)$-theories are purely unrenormalizable (all terms are
unrenormalizable) in dimension four. Thus, they  don't make sense
as fundamental theories, but they are free in the infra-red.
But, by the same token, we can study their low
energy  limits by doing semi-classical calculations.
Of course, when we get to the $SU(2)$-theory we will be dealing with
an asymptotically free, renormalizable theory which has strong quantum
effects. This is then a reasonable fundamental theory, but its low
energy limit is hard to understand. 

We shall eventually use the analysis of low energy limits of these
gauge theories on Minkowski space as a tool in the analysis of low
energy limits of these
theories on compact Riemannian four-manifolds. Understanding these
limits will give another mechanism for computing certain correlation
functions associated to cycles in the four-manifold.  These
correlation functions can be computed in the low energy limit because
one shows that they are invariant under deformation of the parameters.
Computing in  the high energy limit, one identifies these correlation
functions with differentiable invariants, namely the Donaldson invariants
of the four-manifold. In this way we obtain totally different
expressions for the Donaldson invariants of a four-manifold.  In 
most cases these invariants can be directly computed from a
$U(1)$-monopole theory. This then produces the link between the
Donaldson polynomial invariants and the Seiberg-Witten $U(1)$-monopole
invariants. All this is the subject of the next several lectures.
Today we study low energy limits of $U(1)$-gauge theories.


\section{Low Energy $U(1)$ $N=2$ super Yang-Mills Theories}

\subsection{The basic fields}

Consider pure $N=2$ super Yang-Mills
theory in four-dimensions with gauge group 
$U(1)$. This means that the  fields in the Lagrangian
will form a single $N=2$ vector super-multiplet.  That is to say, we
will have the gauge 
field,  and its $N=2$ super-symmetric partners.
All the fields will be massless.
Let us recall from the super-homework what the $N=2$ massless vector
super-multiplet looks like in dimension $4$.
Massless particles are associated to
representations of 
the `little' subgroup $Spin(2)$ in the Lorentz group $Spin(3,1)$ fixing a
nontrivial vector $v\in R^{3,1}$ of norm zero. The `helicity' of such
a particle (i.e., its spin around its direction of motion) is the
representation of its stabilizer.  These are indexed so that the
fundamental representation of $Spin(2)$ is labeled $1/2$.
The irreducible representations of the Poincar\'e group that comprise
the $N=2$ vector super-multiplet in dimension four are arrayed as
pictured below:

$$\matrix{{\rm helicity} & -1 & -1/2 & 0 &\oplus  & 0 & 1/2 & 1 \cr
\ {\rm multiplicity\ of \ representation} \ & 1 & 2 & 1 & &  1 & 2 & 1.}$$
The action of CPT conjugation on these representations is to interchange the
right and left halves of this table. This multiplet
has a $U(2)_R$ symmetry, acting so as to 
preserve helicity.  Thus, the $SU(2)_R\subset U(2)_R$ acts trivially
except on those particles 
of helicities $\pm(1/2)$. On each of these it acts via the fundamental
two-dimensional representation. The $U(1)_R$ acts trivially on the
states of helicity $\pm 1$ and acts by (two copies of) the standard
representation on the states of helicity $\pm 1/2$ and by the $\pm 2$ power of
the fundamental representation on  the states of helicity
zero. The CPT involution normalizes the $U(1)_R$ and acts on it by
conjugation. Thus, the representation of $U(1)_R$ on the other half of
the multiplet is given by the negative of the above.

The fields of an $N=2$ vector super-multiplet  on $N=1$
super-space in dimension four decompose into an $N=1$ vector multiplet
and an $N=1$ chiral multiplet.
The $N=1$ vector super-multiplet is the gauge $N=1$
super-multiplet ${\cal A}$.
 The field ${\cal A}$ is the super-connection and (after partial gauge
fixing to Wess-Zumino gauge) consists of a triple 
$${\cal A}=(A,\lambda,D)$$
where $A$ is the usual $U(1)$-gauge field, $\lambda$ is a fermion
field, and $D$ is an auxiliary field.
The curvature of ${\cal A}$ is a two-form on $N=1$
super-space. The entire curvature form is determined by certain
components $W_\alpha$ which are
 $(\frac{1}{2},0)$ spinor chiral super-fields. (Representations of
$Spin(3,1)=SL_2({\bf C})$ are indexed by an ordered pair $(a,b)$ of
half-integers. The pair $(1/2,0)$ indexes the defining two-dimensional
complex representation.  It is 
chiral because the second half-integer is zero.) The only way that the
super-connection  ${\cal
A}$ enters the action is through the field $W_\alpha$.
The other fields  form an $N=1$ super-multiplet called $\Phi$.
This is a (scalar) chiral super-multiplet. 
It consists of
$$\Phi=(\phi,\psi,F)$$
where $\phi$ is a complex-valued scalar field,
$\psi$ is a fermion, and $F$ is an auxiliary field.  The auxiliary
fields $D$ and $F$ enter the action quadratically and without
derivatives so that 
they can be integrated out using the equations of motion.
Thus, they play no role in the analysis.

\subsection{The Lagrangian}

Now we are ready to write the Lagrangian for the pure $N=2$
$U(1)$-gauge theory in four-dimensions.  Though we
require impose $N=2$ super-symmetry, we choose to write the Lagrangian
first as a Lagrangian for fields on $N=1$ super-space and then study the
consequences of the extra super-symmetry. The Lagrangian is:
\begin{equation}\label{1steqn}
{\cal L}=\left(\frac{i}{4\pi}\int d^4xd^2
\theta\tau(\Phi)W_\alpha W^\alpha+{\rm c.c.}\right)
+\frac{1}{4\pi}\int_{R^{3,1|4}}d^4xd^4\theta K(\Phi,\overline\Phi).
\end{equation}

As we remarked above, $W$ is a piece of the curvature of the $N=1$ super
connection ${\cal A}$.  It 
is a chiral super-field. The function $\tau(\Phi)$ is a holomorphic
function on the complex plane (or more generally the complex curve)
where $\Phi$ takes its values.  Since 
$\Phi$ is a chiral super-field it follows  that 
$\tau(\Phi)$ is also a chiral super-field.
We write
$$\tau=\frac{\theta}{2\pi}+\frac{4\pi i}{e^2}.$$
 Lastly,
$K(\Phi,\overline\Phi)$ is a K\"ahler potential on $\Phi$-space.


We have encountered this Lagrangian before, with one essential
difference.
That difference is that here we are not assuming $\tau$ is constant.
Rather we allow it to be a holomorphic function of $\Phi$. 
Briefly, the reason that we consider this generalization is that this
theory will arise as a low energy limit of some $SU(2)$-theory. That
theory will have a complex plane of classical vacua (i.e., the
space of minima of the potential energy  will be ${\bf C}$).
We find terms in the low energy limit which involve
functions on this space of classical vacua (cf, sigma-models arising
as low energy limits).  It is in this way that the non-constant holomorphic
function $\tau$ arises.
The main application we have in mind for this study is to the low
energy limits of $N=2$ super-symmetric pure $SU(2)$ gauge theory.
Of course, in the low energy limit $\Phi$ will be  slowly varying,
i.e., almost constant.  To the extent that $\Phi$ is almost constant
the same will be true of $\tau$, and we find ourselves expanding
around a low energy effective theory with constant $\tau_{\rm eff}$.



Another difference of this Lagrangian from the general $N=1$
super-symmetric Lagrangian is 
that there is no super-potential term.  The reason is that one can have
such a term in an $N=1$ super-symmetric theory on $N=1$ super-space in
four-dimensions, but the extra conditions imposed by requiring the
theory to have $N=2$ super-symmetry imply that this term must be zero.
Another related, but more delicate, consequence of the same idea is
that $N=2$ super-symmetry imposes a relation between $\tau$ and $K$.
To see this, let us write out the Lagrangian in components.  We get
\begin{eqnarray}\label{2ndeqn}
{\cal L} & = &
\frac{1}{4\pi}\int_{R^{3,1}} d^4x\left(i\left[  \overline\tau(F^+\wedge
F^+)-\tau(F^-\wedge F^-)\right]
+K_{\phi\overline\phi}\partial\phi\overline{\partial\phi}\right. \\
& & \hskip.5in \left. + {\rm Im}\tau\overline\lambda i\dirac\lambda +
K_{\phi\overline\phi} \overline\psi i\dirac\psi\right) \nonumber
\end{eqnarray}
One thing which is clear from this formula, which we have seen before,
is that it is not the K\"ahler potential $K(\Phi,\overline\Phi)$ {\sl
per se} that enters into 
the Lagrangian; only the K\"ahler metric
$K_{\phi\overline\phi}\partial\phi \overline{\partial\phi}$ of the
potential enters. 


To see the relation of $\tau$ and $K$ we use the  $SU(2)_R$-symmetry
which is a consequence of $N=2$ super-symmetry. 
Notice that if $K_{\phi\overline\phi}={\rm Im}\tau$, then 
the pair
$(\lambda,\psi)$ form a multiplet for the $SU(2)_R$ which is
isomorphic
to the  standard representation. 
Conversely if there is an $SU(2)_R$-symmetry, then with an 
appropriate holomorphic redefinition of the
coordinates the $(\lambda,\psi)$ form a standard multiplet for the
$SU(2)_R$ action.  Thus, in these coordinates the 
coupling terms for $\lambda,\lambda$ and $\psi,\psi$ must have the
same strength, and hence with this choice of coordinates we have
${\rm
Im}\tau=K_{\phi\overline\phi}$.
One immediate consequence of this is that ${\rm Im}\tau>0$. This seems
like a problematic condition.  Locally it is not hard to ensure that
it holds, but for a global holomorphic function on all of ${\bf C}$ it
cannot  be achieved. There are two possibilities to contemplate.  One
is that as $\tau$ approaches the real axis,  for some reason
the physics breaks down. One would then be faced with studying what
happens at these `bad' limits.  Another possibility is that $\tau$ is
multi-valued. 
After all, we already know that the transformation $\tau\to \tau +1$
(i.e., increasing $\theta$ by $2\pi$) does not affect the theory.
So at least to that extent one must consider $\tau$ as multi-valued.  Of
course, this does not solve the problem we are addressing since any
non-constant, entire holomorphic function $\tau$ to ${\bf
C}/\{\tau\cong\tau+1\}$ 
still cannot have ${\rm Im}\tau>0$.
To arrange this we need a larger group of indeterminacy for the values
of $\tau$.  
For example, recall that it is easy to construct global holomorphic
functions into 
the upper half-plane if we permit them to be multi-valued, say under
the standard action of
$SL_2({\bf Z})$. As we shall see, this is exactly what happens in this
case. 

\subsection{Indeterminacy of $\tau$}


In order to make the notation consistent with that of [Seiberg-Witten
???] we set $\phi=a$.
The  term in the Lagrangian derived from the K\"ahler potential 
becomes
$$\frac{i}{4\pi}\int_{R^{3,1}}d^4x(\tau-\overline\tau)\frac{\partial
a}{\partial x^i}\frac{\partial\overline{a}}{\partial \overline x^i},$$
which means that the induced K\"ahler metric on the $\Phi$-plane, now
called $U$, is
$$ds^2=\frac{i}{4\pi}(\tau da
d\overline{a}-da\overline{\tau}d\overline{a}),$$ 
where $a$ is a local holomorphic coordinate on the $U$.
It is convenient to introduce another holomorphic function $a_D$
(called {\sl the dual of} $a$) 
locally on an open subset $V$ of  $U$ with the property
that $da_D=\tau da$. 
Given this, we can re-write the metric on $V$ as
$$ds^2=\frac{i}{4\pi}(da_Dd\overline{a}-dad\overline{a}_D).$$
We denote by $\Omega$ the K\"ahler form of this metric:
$$\Omega=\frac{-1}{8\pi}\left(da_D\wedge d\overline{a}-da\wedge
d\overline{a}_D\right).$$
We can view this more symmetrically in $a$ and $a_D$ by forming the
complex two-space with coordinates $(a,a_D)$. 
Then there is an embedding of the  complex curve $f\colon V\to
{\bf C}^2$ into this complex
two-space so that the image is a solution curve for  the differential
equation $da_D=\tau da$.  This embedding will be the graph of
either of the variables as a function of the other.
Let $\omega=\frac{-1}{8\pi}\left(da\wedge d\overline{a}_D-da_D\wedge
\overline{a}\right)$. It is a closed, non-positive $(1,1)$-form on
complex two-space. 
Of course, $f$ must be such that $f^*\omega=\Omega>0$.

The question is: How much freedom do we have in the representation of
the theory in terms of the parameters $a,a_D$ (and implicitly $\tau$)?
Said another way, In what ways can we find other Lagrangian's of this
form with different $a$ and $a_D$ (and hence different $\tau$) which
represent the same theory? 
Just considering the K\"ahler potential term we see an obvious
geometric change of coordinates that brings the Lagrangian back into
the same form with different values of the parameters.  Namely, we can
act by an extended version $ISL_2({\bf R})$ of $SL_2({\bf R})$. 
We define
$$0\to {\bf C}^2\to ISL_2({\bf R})\to SL_2({\bf R})\to 0$$
to be an extension which is a semi-direct product with the natural
action of $SL_2({\bf R})$ on ${\bf C}^2$. 
An element $(M,\pmatrix{c_D \cr c})$ in this group acts by
$$(M,\pmatrix{ c_D \cr c})\cdot\pmatrix{a_D \cr a}=M\cdot\pmatrix{a_D
\cr a}+\pmatrix{c_D \cr c}.$$ 
Since $\tau=\frac{da_D}{da}$, one sees that the action of this group
on $\tau$ is the usual fractional linear transformation of $M$ on
$\tau\in {\bf H}$:
$$\pmatrix{r & s \cr t & u}\cdot \tau=\frac{u\tau +t}{s\tau +r}.$$ 
In this way this change of variables brings the $d^4\theta$ term of
the Lagrangian back into the same form with a different $a,a_D,\tau$.


This computation with the K\"ahler metric leads us to ask whether the
non-uniqueness in our 
description of the holomorphic function $\tau$ should be that we are
allowed to compose $\tau$ with any $SL_2({\bf R})$  fractional-linear
transformation. 
But 
we must examine the other term in the Lagrangian where $\tau$ appears. 
That is in the term involving the gauge field (or photon). 
Recall from Lecture II-8 what happens  for constant $\tau$.
Since $\tau =\frac{\theta}{2\pi}+\frac{4\pi
i}{e^2}$ and the action is given by $e^{i{\cal L}}$, if 
we consider strictly upper triangular matrices the only ones that
leave invariant this part of the action are those of the form
$$\pmatrix{ 1 & b \cr 0 & 1}$$
with $b\in{\bf Z}$. 
As we have seen in Lecture II-8, this action of the strictly upper
triangular matrices 
in $SL_2({\bf Z})$ extends to an action of $SL_2({\bf Z})$ on
representations of the theory. 
That is to say, if $\tau$ and $\tau'$ in ${\bf H}$ differ by the action
of $SL_2({\bf Z})$, then the pure $U(1)$-gauge theory with $\tau$ as
parameter has another representation as a $U(1)$-gauge theory (with
different basic fields) where the parameter is $\tau'$.
The transformation $\tau\to (-1)/\tau$ is the duality transformation
in $U(1)$-gauge theory interchanging the electric and magnetic charges.
The upshot of all this is that
the full symmetry group of representations  of the theory by
Lagrangian's of the given form is 
$$ISL_2({\bf Z})={\bf C}^2\ltimes SL_2({\bf Z}).$$
The way this group transforms the parameter  $\tau$ is through
composing with the usual 
fraction linear transformation action of $SL_2({\bf Z})$ on the upper
half-plane. 

We now see how to describe our low energy effective  theory globally.  There
is a complex curve $U$ of vacua and there is a global holomorphic function
$\overline\tau\colon U\to {\bf H}/PSL_2({\bf Z})$.  Furthermore, for
each $u\in U$ there is a neighborhood $V$ of $u$, a lifting of
$\overline\tau$ to a  
holomorphic function $\tau\colon V\to {\bf H}$, and $N=1$ super-fields
${\cal A},\Phi$ such that there is a
Lagrangian in the form 
given in Equation~\ref{1steqn} (or Equation~\ref{2ndeqn} in
components)  which represents the theory in $V$.
As we pass from one of these open subsets to another, the
representation of the low energy effective theory locally by
Lagrangian's changes. 
The function $\tau$ changes by  the
natural action of $SL_2({\bf Z})$ on ${\bf H}$.
The chiral super-field changes by the action of $ISL_2({\bf Z})$
described above, and
the gauge super-field ${\cal A}$ changes by the duality action
of $SL_2({\bf Z})$ action on the representations of pure $N=1$
super-symmetric $U(1)$-gauge theory as described in Lecture II-8.
Because there are global holomorphic functions to ${\bf H}/SL_2({\bf
Z})$,  this is a completely consistent picture of a global theory in
which ${\rm Im}\tau$ never goes to zero. 


\subsection{The Family of elliptic curves associated to $\tau$}


We have now established all the relevant general features of the
low energy effective theories we want to study. We have an $N=2$
super-symmetric pure 
$U(1)$-gauge theory with a family of vacua parameterized by
a complex curve $U$, where the scalar field $\phi=a$ takes its values. This
curve has an open covering $\{V^\alpha\}$ and on 
each open subset there is a pair of dual holomorphic coordinates
$a^\alpha,a_D^\alpha$ so that
$\tau^\alpha=\frac{da^\alpha_D}{da^\alpha}$ is the parameter appearing
in the Lagrangian. We assume that the different local representations of
the theory differ by automorphisms in $ISL_2({\bf Z})$.
We have just seen that the local holomorphic function $\tau$ on $U$
which appears 
in the Lagrangian is in fact a global function of $U$ to ${\bf
H}/SL_2({\bf Z})$. 
Thus, for each $u\in U$ it is natural to consider $\tau(u)$ as the
$j$-invariant of an elliptic curve $E(u)$. Of course, it is possible
that at a discrete set of points $\{p_i\}_i$ in $U$ that $\tau$ goes off to
infinity, or equivalently that the elliptic curves $E(u)$ develop  a
node as $u\to p_i$.
At these points the physics changes and there are more massless
fields, fields that become massive at nearby $u\in U$. 

There is a natural way to
fit these elliptic curves together. Namely, there is a complex analytic
surface ${\cal E}$ together with a proper holomorphic map
${\cal E}\to U$ whose fibers form the  family of
elliptic curves $\{E(u)\}_{u\in U}$
The simplest way to construct such a family is to take the Weierstrass
form. That is to say, there are functions $A,B$ on $U$ such that
${\cal E}$ is defined by
$$Y^2=X^3+AX+B,$$
with an appropriate completion of the missing point at infinity in
each elliptic curve. (For globally non-trivial bases $U$, we will have a
line bundle $L$ over $U$ and $A$ and $B$ will be sections of
$L^{\otimes 4}$ and $L^{\otimes 6}$, respectively.) This family of
elliptic curves has a natural section (the section at infinity in the
Weierstrass description), and thus, it can be viewed as a family of
one-dimensional Abelian varieties. There are, of course, other
families of  holomorphically varying genus-one curves over the same base with
the same $j$-invariant function but without a section.  But given the
information of the $j$-invariant function only, there is no
distinguished family except the one with a section. 


Suppose that we have  a family of elliptic curves ${\cal E}\to U$ with
a section $\sigma$. Furthermore, suppose that
we have an open covering $\{V^\alpha\}_\alpha$ of $U$ such that on
each $V^\alpha$ there are a pair of dual holomorphic coordinates
$a^\alpha,a^\alpha_D$ such that $\tau^\alpha=
\frac{da^\alpha_D}{da^\alpha}$ is a
function to the upper half-plane.
We suppose that these two sets of data are related as above.  That is
to say, 
${\cal E}|_{V^\alpha}$ 
is isomorphic to the family
$V^\alpha\times{\bf C}/(1,\tau^\alpha(v))$.
This allows us to define one-cycles
$E_1^\alpha(u),E_2^\alpha(u)\subset E(u)$ for all $u\in V^\alpha$ as
the images of the arcs $\{u\}\times [0,1]$ and $\{u\}\times
[0,\tau^\alpha(u)]$ in $\{u\}\times {\bf C}$. 
Then, there is a differential of the
second kind $\lambda^\alpha$ (i.e., a meromorphic one-form with trivial
residue along its polar locus) on ${\cal E}|_{V^\alpha}$ such that
$$\int_{E^\alpha_1(u)}\lambda^\alpha=a^\alpha(u)$$
$$\int_{E^\alpha_2(u)}\lambda^\alpha =a^\alpha_D(u).$$
In general, we can not fit these differentials $\lambda^\alpha$ of the
first kind together to produce a global object.  (We will see the precise
conditions when this can be done soon.) But 
we claim that we can fit together the two-forms $d\lambda^\alpha$.
Namely, we claim that we there is a unique global holomorphic two-form
$\eta$ on ${\cal E}$ 
such that on  ${\cal E}|_{V^\alpha}$ we have
$$\int_{E^\alpha_1}\eta=da^\alpha$$
$$\int_{E^\alpha_2}\eta=da^\alpha_D.$$
Of course, the uniqueness assertion immediately implies that the
restriction of any such global two-form $\eta$ 
to  ${\cal E}^\alpha$ is equal to 
$d\lambda^\alpha$.  


Recall that  ${\cal E}^\alpha$ is identified with $V^\alpha\times {\bf
C}/(1,\tau^\alpha(u))$. On the universal covering space
$V^\alpha\times {\bf C}$ we have the holomorphic one-form $dz$ induced
from the standard holomorphic coordinate on ${\bf C}$.
This one-form does not descend to the quotient because $\tau$ is not
a constant  function of $a^\alpha$. Nevertheless, the two-form
$da^\alpha \wedge dz$ does descend to give a well-defined two-form on
${\cal E}^\alpha$.
It is clear that the 
only holomorphic two-form on ${\cal
E}|_{V^\alpha}$ which satisfies 
the first integral equation above is 
$$\eta^\alpha=-da^\alpha\wedge dz.$$

This two-form $\eta^\alpha$ also clearly satisfies the second integral
equation. Now we need to see that these local two-forms fit together
to give a global form $\eta$.
But this is clear from the uniqueness of the local forms.

Now let us compute
$\pi_*(\eta\wedge\overline\eta)$, where $\pi$ is the projection from
${\cal E}\to U$. Let us restrict attention to ${\cal
E}|_{V^\alpha}$. From our local description we have 
$$\pi_*(\eta\wedge\overline\eta)=\int_{{\rm
fibers}}d\lambda^\alpha \wedge d\overline{\lambda}^\alpha.$$
It follows immediately from the formulas for the integrals of
$d\lambda$ over $E^\alpha_1$ and $E^\alpha_2$ and the fact that
$E^\alpha_1\cdot E^\alpha_2=1$ in the
homology of $E(u)$ that the result is
$$\pi_*(\eta\wedge \overline\eta)=da^\alpha\wedge
d\overline{a}^\alpha\overline{\tau}^\alpha+da^\alpha\wedge
d\overline{a}^\alpha  \tau^\alpha 
=(\tau^\alpha-\overline{\tau}^\alpha)da^\alpha\wedge
d\overline{a}^\alpha=8\pi\Omega|_{V^\alpha}.$$   

Since the two-form $\eta$ is holomorphic, it integrates trivially on
any fiber and on the section $\sigma$ of the elliptic fibration.
Thus, its homology class is determined by the integrals
of $\eta$ along two-tori which lie over circles in $U$.

The deRham cohomology class of $\eta$ measures whether or not the
automorphisms required in passing between  our local descriptions in
terms of various sets of $(a,a_D,\tau)$ lie in $SL_2({\bf Z})$ or in
the extended group $ISL_2({\bf Z})$. 
We claim that if the deRham homology class of $\eta$ is trivial,
then the automorphisms we use lie in $SL_2({\bf Z})$. The point is the
following. We already know that the change in $da, da_D$ is by
elements in $SL_2({\bf Z})$.  We need to compute the changes in
$a,a_D$. 
If we go around any loop $\gamma$ in $U$ with the
property that $E_1$ comes back to itself, then the cycles
$E_1(\gamma(t))$ fit together to make a torus and the integral around
this torus of $\eta$ is equal to $\int_\gamma da$.  If this integral
is trivial, then $a$ can be analytically continued around this loop to
a single valued function.
Similar arguments apply to $a_D$. Thus, if all the periods of $\eta$
are trivial, then $a,a_D$ transform by $SL_2({\bf Z})$.


In general, the deRham cohomology class of a holomorphic two-form such
as $\eta$ is trivial if and only if there is a global differential of
the second kind $\lambda$ on ${\cal E}$ with 
$d\lambda=\eta$.
In the more general case, we can write the holomorphic two-form $\eta$
as $d\lambda$ for some 
global differential  $\lambda$ of the third kind on
${\cal E}$ (i.e., $\lambda$ is a meromorphic one-form on ${\cal E}$
with constant residue along its polar locus.) 
In turns out in our application to pure $SU(2)$-gauge theory,
 that the automorphisms between our various descriptions lie in
$SL_2({\bf Z})$, so that  we shall
not need to consider this more general case.
The proof of this fact is the subject of the next section.


\subsection{The BPS formalism}

Our goal here is to show that if the low energy effective $U(1)$-gauge
theory that we have 
been considering is the low energy limit of a pure $N=2$ super-symmetric
$SU(2)$ gauge-theory, then the transformations between the various local
descriptions in terms of $(a,a_D,\tau)$ of the $U(1)$-theory lie in
$SL_2({\bf Z})$. 
We shall establish this by using the BPS formalism.  In any
$U(1)$-theory we have two conserved charges, the electric and
magnetic charges: 
$$Q_e=\frac{1}{4\pi}\int_{S^2}*F$$
$$Q_m=\frac{1}{4\pi}\int_{S^2}F.$$
Here, we are doing the integrals on the sphere at infinity in some
time slice in Minkowski four-space. 
In pure $U(1)$-theory on four-space, these charges are both zero.
But in the case that our $U(1)$-theory is the low energy limit of an
$SU(2)$-theory these charges can be non-zero.
Recall from Lecture II-9 that our $SU(2)$-theory has a $U(1)$-symmetry
whose conserved charge is 
$$N_e=Q_e+\frac{\theta Q_m}{2\pi}.$$
As a result, this operator has integral eigenvalues.  It is called the
electric charge.
There is another conserved charge, the magnetic charge, given by
$$N_m=Q_m.$$
This is the first Chern class of the line bundle at infinity and
hence is clearly has integral eigenvalues as well.
Of course, as we saw in Lecture II-9 and II-9 these two charges are
interchanged by duality. Thus, both charges come from
$U(1)$-symmetries of the theory. 


At least
in the case of constant $\tau$ they are related to the $Q_e,Q_m$ by
\begin{eqnarray*}
N_m & = & Q_m \\
N_e & = & Q_e+\frac{\theta Q_m}{2\pi}
\end{eqnarray*}

The $N=2$ super-symmetry generators $Q_\alpha^i, \ \ i=1,2$ 
satisfy the relations of the super Poincar\'e group
$$\{Q_\alpha^i,Q_{\dot\alpha}^j\}=\delta^i{}_jP_{\alpha\dot\alpha}$$
where $P_{\alpha\dot\alpha}$ is infinitesimal translation in the
spatial direction $\{\alpha\dot\alpha\}$. Also, we have
$$\{Q_\alpha^i,Q_\beta^j\}=\epsilon^{ij}\epsilon_{\alpha\beta}U$$
where $U$ is  multiplication by a scalar, the central charge of the
theory. 

In general, there is the BPS bound on the mass of any state
which says
$$M\ge |U|.$$
In addition, there is a good understanding of the BPS saturated
states, i.e., those for which this inequality is an equality.

In our case we wish to compute the central charge $U$.
To do this recall that the Lagrangian (for $\tau$ constant) is:
$${\cal L}=\frac{1}{4\pi}\int d^4xd^2\theta i\tau
W_\alpha W^\alpha+\int d^4xd^4\theta{\rm Im}\tau \overline\Phi\Phi.$$ 
{}From this one knows (from Lecture II-9) that for this value of
$\tau$ 
$$U=aN_e+a_DN_m$$
where $\Phi=a+\ldots$.
(On dimensional grounds, this formula holds even if $\tau$ is not
constant.) 
The interpretation is that different realizations of the low energy
theory correspond to different values of $a$.  In  these
realizations $U$ depends on the values of $a, a_D$ by the above
formula.
Recall from Lecture II-9 that $(N_m,N_e)$ transform under $SL_2({\bf
Z})$ as we take different representations of our theory. Since $U$ has
to be unchanged, we see that if 
$$\pmatrix{a_D \cr a}\mapsto M\cdot\pmatrix{a_D \cr a},$$
then
$$\pmatrix{N_m & N_e}\mapsto \pmatrix{N_m & N_e}\cdot M^{-1}$$
for any $M\in SL_2({\bf Z})$. 

The first important point to notice is that we can not compensate for
the operation of adding a constant to $\pmatrix{a_D \cr a}$ by a
change in $(N_m,N_e)$.  This shows that the automorphisms between the
various representations of the $U(1)$-theory in terms of
$(a,a_D,\tau)$ lie in $SL_2({\bf Z})$. 
Thus, in this case  we can construct a differential
$\lambda$ of the second kind on ${\cal E}$ such that for each $\alpha$
we have
$$\int_{E_1}\lambda=a^\alpha$$
$$\int_{E_2}\lambda =a^\alpha_D.$$
on ${\cal E}|_{V^\alpha}$.

More generally, if the $SU(2)$-theory has extra symmetries (as might
be the case if we add matter), then the form of the central charge
becomes 
$$U=aN_e+a_DN_m+S$$
where $S$ is some other conserved charge.
In this case it is possible to compensate for the affine translations
in $(a_d,a)$.  Namely, we can have
$$\pmatrix{a_D \cr a \cr 1}\mapsto \pmatrix{M & \pmatrix{c_D \cr c} \cr
\pmatrix{ 0 & 0 } & 1}\cdot \pmatrix{a_D  \cr a \cr 1}$$
and
$$\pmatrix{N_m & N_e & S}\mapsto \pmatrix{N_m & N_e & S}\cdot
\pmatrix{M^{-1} & \pmatrix{* \cr *} \cr \pmatrix{0 & 0} & 1}.$$

In all these transformations no derivatives of $\tau$ enter.
In fact, what we did was to  compute the transformation laws for
$\tau$ constant and 
just extended them in the obvious way to the case when $\tau$ is
holomorphic.  The reason that we know that this mechanism  works is
that on dimensional grounds alone we know that no derivatives of
$\tau$ can enter the formulas.
While one can consider the low energy limits of these more complicated
$SU(2)$-theories with matter, we shall restrict ourselves to the case
when there are no extra symmetries and the transformations between the
various representations of our low energy theory lie in $SL_2({\bf
Z})$. 


We gave a derivation of the formula for the central charge above from
the form of the Lagrangian.  Let us give another derivation of the
same formula (or at least a piece of it). Let us consider $U(1)$-gauge
theory with a charged $N=2$ vector hyper-multiplet. In the $N=1$
language, this hyper-multiplet is a pair of chiral super-fields
$(T,\tilde T)$ of equal but opposite charges.  Super-symmetry implies
that the charges must be opposite, let us denote them by  $\pm
n_e$. In the Lagrangian we have added a 
super-potential  
$$W=n_e\Phi T\tilde T.$$ 
Since $\Phi=a+\ldots$, we see that the mass of the pair $(T,\tilde T)$
is $aN_e$. 
Thus, we see this term in the central charge of the theory from this
point of view.


\subsection{Jumping of the BPS spectrum}

Let us return to our low energy $N=2$ super-symmetric $U(1)$-gauge
theory given locally by $a,a_D$. 
In such theories the BPS mass inequality is
$$M\ge |n_ea+n_ma_D|.$$
Since  $N_e$ and $N_m$ are
integers, this is a good bound in the sense that it implies that there
is a positive constant which is a lower bound for $M$ as long as the
lattice in ${\bf C}$ generated by $a$ and $a_D$ is a honest lattice.
Let us examine what happens as the lattice  generated by
$a$ and $a_D$ degenerates, i.e., when $a$ and $a_D$ become collinear in
${\bf C}$.  
At such points the triangle inequality is no long strict and this
permits states to decay to other lower states.  In this way BPS states can
disappear (or appear), accounting for the jumping in the BPS spectrum.

 Generically, we expect this to happen along a real curve in
$U$. (Notice the lattice generated by $(1,\tau)$ is not degenerating,
rather it is the lattice of values of $a$ and $a_D$ which is
degenerating.) 
This phenomenon is similar to that in Vafa's lecture for $N=2$
super-symmetric theories in dimension two. There the jumping occurred
when the values of the central charge became collinear. 
In the example of interest, we shall obtain explicit information about
the curve in  $U$ where the BPS spectrum jumps.



\subsection{More about the application to $SU(2)$-gauge theory}

As we have remarked in passing several times, we will study an
$N=2$ super-symmetric $SU(2)$-gauge theory with no matter
hyper-multiplets. We shall do this by studying the theory as an $N=1$
super-symmetric theory with $\Phi$ an adjoint-valued chiral
super-field. 
The microscopic (or high energy) effective Lagrangian is given by:
$${\cal L}_{\rm mac}=\frac{1}{4\pi}\int d^4xd^2\theta
\tau_0(\Lambda_0){\rm Tr}W_\alpha W^\alpha+{\rm c.c.} +\frac{1}{4\pi}
\int d^4xd^4\theta {\rm Im}\tau_0(\Lambda_0){\rm
Tr}\overline\Phi\Phi$$ 
where $W$ is a piece of the super-curvature of the super-connection
${\cal A}$ on a principal $SU(2)$-bundle over Minkowski space, and
$\Phi=\phi+\theta\psi+\cdots$ with $\phi$ being a 
field with values in the complexification of the adjoint bundle.  The $\Lambda_0$ refers to the
ultraviolet momentum cut-off that 
we have chosen to regulate the $SU(2)$-theory.  Here, $\tau_0$ is a
constant (in the fields) which depends on the value of $\Lambda$.

The space of classical vacua is given by configurations consisting of
trivial connections 
and constant (covariantly constant) sections $\phi$ of ${\bf C}\otimes
su(2)$ which are zeros for the potential, up to gauge equivalence.
The potential for this  Lagrangian is the norm-squared of the moment map
$$V=\frac{{\rm Im} \tau_0}{4\pi}\int d^4x {\rm
Tr}\left([\phi,\overline\phi]^2\right).$$
If $\phi$ is a zero of the potential function, then clearly 
$\phi$ and $\overline\phi$ commute.
Writing $\phi=\gamma+i\delta$ with $\gamma,\delta\in su(2)$, we see
that the condition that $\phi$ and $\overline\phi$ commute is simply
that $\gamma$ and $\delta$ commute.  This means that $\gamma$ and
$\delta$  can be simultaneously
diagonalized. That is to say, up to gauge transformation,
$$\phi=\frac{1}{\sqrt{2}}\pmatrix{ a & 0 \cr 0 & -a}$$
for $a\in {\bf C}$.
There is a further conjugation, namely the action of the Weyl group,
which in this case is ${\bf Z}/2{\bf Z}$ and acts by sending $a\mapsto
-a$. Thus, the moduli space of classical vacua is a complex plane
parameterized by the complex coordinate 
$$u={\rm Tr}\phi^2=a^2.$$
In the classical theory, for  $u\not=0$ the $SU(2)$-gauge symmetry is
broken to $U(1)$ and the 
low energy theory is as we have discussed today.
It has a  single massless multiplet, which becomes Higgs'ed and
acquires a mass.
The low energy effective theory at the exceptional point $u=0$
remains an unbroken $SU(2)$-gauge theory.  


Let us examine it in light of our discussion today about low energy
effective $U(1)$-theories.  
As we have just explained the space of classical vacua is the complex
plane -- the $u$-plane.
Let us examine the metric on the $u$-plane. 
All points of this plane except zero correspond to a $U(1)$-theory as
described in this lecture.
The metric on the space of classical vacua is induced from the natural
metric on the 
space of trivial connections and covariantly constant scalars in the
complexified Lie algebra.  This is simply  the ad-invariant metric on the
complexified Lie algebra. Restricting to the diagonal matrices as
above, we see that the induced metric is $da\wedge d\overline{a}$. 
Since $u=a^2$, this means that the metric on the $u$-plane is
$$ds^2=\frac{du d\overline{u}}{|u|^2},$$
which has a singularity at the origin.  This singularity reflects the
fact that at the origin there are other masses fields so that the
the low energy effective theory is not a pure $N=1$ super-symmetric
$U(1)$-gauge theory at that point.

Now we can describe the family of tori over the $u$-plane punctured at
$0$, and the holomorphic two-form
$\eta$ on the total space of this family.
The function $\tau$ is constant, and hence the family of tori has
monodromy contained in $\{\pm 1\}\subset SL_2({\bf Z})$. Since the
metric on the $u$-plane minus the origin is given by ${\rm Im}\tau
da\wedge d\overline{a}$ we see that the parameter $a=\sqrt{u}$ is the
parameter of the same name in the $U(1)$-theory (i.e., the first of
the dual pair of local holomorphic coordinates on the base) and the
dual parameter $a_D$ is $\tau a$. This means that the monodromy around
infinity (or equivalently around zero) sends $a$ to $-a$ and $a_D$ to
$-a_D$. This means that the family of tori over the puncture
$u$-plane is obtained from the
product family $({\bf C}-\{0\})\times E$ over the punctured $a$-plane
by dividing out by the involution $(a,z)\mapsto (-a,-z)$. 
(Here, $E$ is the elliptic curve ${\bf C}/L(1,\tau)$, the quotient of
the complex plane by the lattice generated by $1$ and $\tau$.)
The holomorphic two-form $\eta$ is the image of $da\wedge dz$ where
$dz$ is a global holomorphic one-form on $E$.


This description is of the classical theory.  In the next lecture we
shall discuss the quantum version of this theory. It turns out that
things change somewhat in going to the quantum theory, but the
changes are in some ways not as drastic as one might imagine. For 
example the space 
of quantum vacua will be the same $u$-plane, though the singular
points will be different, and they will correspond to a different
type of object becoming massless.
Also, the monodromy will be different.


\end{document}







