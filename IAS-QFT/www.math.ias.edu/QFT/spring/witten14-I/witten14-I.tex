\documentstyle{article}

\input epsf.tex


%These are the macros which are in common with all of the
% sections in the paper mmr
% Each section, for now, should begin with \documentstyle[11pt,cd]{article}
% and then have \input{mmrmacros} followed by \begin{document}
% The only exception is that the \Label macro is slightly different
% in each file and should be put in separately.
%New CD macros
\newcommand{\cdrl}{\cd\rightleftarrows}
\newcommand{\cdlr}{\cd\leftrightarrows}
\newcommand{\cdr}{\cd\rightarrow}
\newcommand{\cdl}{\cd\leftarrow}
\newcommand{\cdu}{\cd\uparrow}
\newcommand{\cdd}{\cd\downarrow}
\newcommand{\cdud}{\cd\updownarrows}
\newcommand{\cddu}{\cd\downuparrows}
% (S) Proofs.
% (S-1) Head is automatically supplied by \proof.

\def\proof{\vspace{2ex}\noindent{\bf Proof.} }
\def\tproof#1{\vspace{2ex}\noindent{\bf Proof of Theorem #1.} }
\def\pproof#1{\vspace{2ex}\noindent{\bf Proof of Proposition #1.} }
\def\lproof#1{\vspace{2ex}\noindent{\bf Proof of Lemma #1.} }
\def\cproof#1{\vspace{2ex}\noindent{\bf Proof of Corollary #1.} }
\def\clproof#1{\vspace{2ex}\noindent{\bf Proof of Claim #1.} }
% End of Proof Symbol at the end of an equation must precede $$.

\def\endproof{\relax\ifmmode\expandafter\endproofmath\else
  \unskip\nobreak\hfil\penalty50\hskip.75em\hbox{}\nobreak\hfil\bull
  {\parfillskip=0pt \finalhyphendemerits=0 \bigbreak}\fi}
\def\endproofmath$${\eqno\bull$$\bigbreak}
\def\bull{\vbox{\hrule\hbox{\vrule\kern3pt\vbox{\kern6pt}\kern3pt\vrule}\hrule}
}
\addtolength{\textwidth}{1in}                  % Margin-setting commands
\addtolength{\oddsidemargin}{-.5in}
\addtolength{\evensidemargin}{.5in}
\addtolength{\textheight}{.5in}
\addtolength{\topmargin}{-.3in}
\addtolength{\marginparwidth}{-.32in}
\renewcommand{\baselinestretch}{1.6}
\def\hu#1#2#3{\hbox{$H^{#1}(#2;{\bf #3})$}}          % #1-Cohomology of #2
\def\hl#1#2#3{\hbox{$H_{#1}(#2;{\bf #3})$}}          % #1-Homology of #2
\def\md#1{\ifmmode{\cal M}_\delta(#1)\else  % moduli space, delta decay of #1
{${\cal M}_\delta(#1)$}\fi}
\def\mb#1{\ifmmode{\cal M}_\delta^0(#1)\else  %moduli space, based, delta
                                              %decay of #1
{${\cal M}_\delta^0(#1)$}\fi}
\def\mdc#1#2{\ifmmode{\cal M}_{\delta,#1}(#2)\else    %moduli space, delta
                                                      %decay, chern class #1
                                                      %of #2
{${\cal M}_{\delta,#1}(#2)$}\fi}
\def\mbc#1#2{\ifmmode{\cal M}_{\delta,#1}^0(#2)\else   %as before, based
{${\cal M}_{\delta,#1}^0(#2)$}\fi}
\def\mm{\ifmmode{\cal M}\else {${\cal M}$}\fi}
\def\ad{{\rm ad}}
\def\msigma{\ifmmode{\cal M}^\sigma\else {${\cal M}^\sigma$}\fi}
\def\cancel#1#2{\ooalign{$\hfil#1\mkern1mu/\hfil$\crcr$#1#2$}}
\def\dirac{\mathpalette\cancel\partial}
\newtheorem{thm}{Theorem}
\newtheorem{theorem}{Theorem}[subsection]
\newtheorem{proposition}[theorem]{Proposition}
\newtheorem{lemma}[theorem]{Lemma}
\newtheorem{claim}[theorem]{Claim}
\newtheorem{example}[theorem]{Example}
\newtheorem{corollary}[theorem]{Corollary}
\newtheorem{D}[theorem]{Definition}
\newenvironment{defn}{\begin{D} \rm }{\end{D}}
\newtheorem{addendum}[theorem]{Addendum}
\newtheorem{R}[theorem]{Remark}
\newenvironment{remark}{\begin{R}\rm }{\end{R}}
\newcommand{\note}[1]{\marginpar{\scriptsize #1 }}
\newenvironment{comments}{\smallskip\noindent{\bf Comments:}\begin{enumerate}}{
\end{enumerate}\smallskip}

\renewcommand{\thesection}{\Roman{section}}
\def\eqlabel#1{\addtocounter{theorem}{1}
\write1{\string\newlabel{#1}{{\thetheorem}{\thepage}}}
\leqno(\rm\thetheorem)}
\def\cS{{\cal S}}
\def\ov{\overline}












\title{Lecture II-14, part I: The Landau--Ginzburg Description of $N=2$
Minimal Models}
\author{Edward Witten\thanks{Notes by David R. Morrison}}
\date{May 1, 1997}

\begin{document}
\maketitle

%Dave's macros
% automatic loading of rsfs fonts, if present
\batchmode
  \newfont{\footscrfont}{rsfs10}
  \newfont{\footbbbfont}{msbm10}
\errorstopmode

\newif\ifscrf\scrftrue
\ifx\footscrfont\nullfont
  \scrffalse
\fi

\ifscrf
  \newfont{\scrfont}{rsfs10 scaled\magstep1}  % rsfs12 does not exist
  \newfont{\smallscrfont}{rsfs7}
  \newfont{\tinyscrfont}{rsfs7}
% \newfont{\footscrfont}{rsfs10}
  \newfont{\smallfootscrfont}{rsfs7}
  \newfont{\tinyfootscrfont}{rsfs7}
\fi
\makeatletter

\newif\iffn\fnfalse

\@ifundefined{reset@font}{\let\reset@font\empty}{} %needed for ancient LaTeX
\long\def\@footnotetext#1{\insert\footins{\reset@font\footnotesize
    \interlinepenalty\interfootnotelinepenalty
    \splittopskip\footnotesep
    \splitmaxdepth \dp\strutbox \floatingpenalty \@MM
    \hsize\columnwidth \@parboxrestore
   \edef\@currentlabel{\csname p@footnote\endcsname\@thefnmark}\@makefntext
    {\rule{\z@}{\footnotesep}\ignorespaces
      \fntrue#1\fnfalse\strut}}}
\makeatother

\ifscrf
  \newcommand{\Scr}[1]{\iffn
    \mathchoice{\mbox{\footscrfont #1}}{\mbox{\footscrfont #1}}
    {\mbox{\smallfootscrfont #1}}{\mbox{\tinyfootscrfont #1}}\else
    \mathchoice{\mbox{\scrfont #1}}{\mbox{\scrfont #1}}
    {\mbox{\smallscrfont #1}}{\mbox{\tinyscrfont #1}}\fi}
\else
  \def\Scr{\cal}
\fi
%line bundle (lb) and Lagrangian (lg) -- should have distinct fonts
\newcommand{\lb}{{\Scr L}}
\renewcommand{\lg}{{\cal L}}
%\numberwithin{equation}{section}
\newcommand{\operatorname}[1]{\mathop{\rm #1}\nolimits}
\newcommand{\Tr}{\operatorname{Tr}}
\newcommand{\ndex}{\operatorname{index}}

\section{Landau--Ginzburg models}

The topic for the first part of today's lecture
is a more detailed discussion of the known evidence for the
predicted equivalence between IR limits of certain Landau--Ginzburg models, and
the algebraically constructible conformal field theories known as ``minimal
models.''

Consider first the simplest Landau--Ginzburg model: a two-dimensional $N{=}2$
theory of a
single chiral superfield $\Phi$ with Lagrangian
$$\lg=\int d^2x\,d^4\theta\,\overline{\Phi}\Phi+\left(\int
d^2x\,d\theta^+\,d\theta^-\,\Phi^k
+ \mbox{c.c.}\right) .$$
Our claim in lecture II-12 was that this model flows in the IR to a conformal
field
theory with $\widehat{c}=1-\frac2k$.  If true, then the theory it flows to
must be an
algebraically solvable theory, due to the known classification of
representations of
the $N{=}2$ superconformal algebra with $\widehat{c}<1$.  More precisely,
fixing $\widehat{c}=1-\frac2k$,
there turn out to be a finite number of
 irreducible representations ${\cal H}_\alpha$ of the algebra of
central charge $\widehat{c}$ (up to isomorphism); a conformal field theory
must be built as
a modular-invariant combination
$$\bigoplus ({\cal H}_\alpha\otimes\overline{\cal H}_\beta)^{e_{\alpha\beta}}$$
of these representations.  The simplest modular-invariant combination, and the
one which will
occur here, is the unweighted diagonal sum over all of the representations:
$$\bigoplus_\alpha\,{\cal H}_\alpha\otimes\overline{\cal H}_\alpha.$$

More general Landau--Ginzburg models involve several chiral superfields
$\Phi_1$, \dots,
$\Phi_n$ and a quasi-homogeneous polynomial $W(\Phi_1,\dots,\Phi_n)$ which
serves as the
superpotential of the theory.  We should assume that the polynomial is
transverse in the
sense that $W=dW=0$ holds only at the origin; the quasi-homogeneity condition
can be expressed
as the existence of positive rational numbers $\alpha_j$ such that
$$W=\sum \alpha_j\Phi_j\frac{\partial W}{\partial \Phi_j}.$$
The Lagrangian can be written
$$\lg=\int d^2x\,d^4\theta\,\overline{\Phi}_j\Phi_j+\left(\int
d^2x\,d\theta^+\,d\theta^-\,W(\Phi_j)
+ \mbox{c.c.}\right) .$$


Intuitively, the quasi-homogenity is important because it allows the
possibility of extending
one of the $R$-symmetries, namely
$$\theta^+\to e^{-i\varepsilon}\theta^+,\qquad \theta^-\to\theta^-,$$
to act on the chiral fields as $\Phi_j\to e^{i\varepsilon\alpha_j}\Phi_j$ so
that the superpotential
transforms as $W\to e^{i\varepsilon}W$ and hence the Lagrangian is invariant.
If we {\it assume}\/
that these microscopic $R$-symmetries {\it become}\/ the $R$-symmetries of a
limiting conformal
field theory in the IR, then we can calculate the central charge $\widehat{c}$
 of that limiting
theory via operator product expansions.

This works as follows.  Recall that the coefficient $k$ which we encountered
in lecture II-13 as
measuring the anomaly which arises if one attempts to gauge the $R$-symmetry
$J_L$ appears in
the operator product expansion as
$$J_L(x)\,J_L(x')=\frac{k}{(x^--x'^-)^2}+\cdots .$$
The interpretation of $k$ as the obstruction to gauging makes it clear that it
is a topological
invariant which can be reliably measured or computed at any scale.
 In particular, it can be calculated in the UV, where the superpotential
doesn't matter.  On the other hand, if $J_L$ is the same $R$-symmetry which
appears in
the limiting $N=2$
superconformal field theory, then the  operator product expansion of the
superconformal algebra
asserts that
$$J_L(x)\,J_L(x')=\frac{\widehat{c}}{(x^--x'^-)^2}+\cdots ,$$
where $\widehat{c}$ is the central charge in the $N=2$ algebra.
Thus, $k=\widehat{c}$ and we can measure the IR
central charge by computing in the UV.



To compute $k$ in the ultraviolet, one treats the fields as free fields.
The obstruction to gauging comes from the fact that the $R$-symmetries are
chiral symmetries for the fermions in the free supermultiplets.
One gets $k=\sum_j(1-2\alpha_j)$.
Assuming the flow to an $N=2$ theory in the infrared, with the same
$R$-symmetry appearing
in the $N=2$ algebra, we have $k=\widehat c$ and hence $\widehat
c=\sum_j(1-2\alpha_j)$.
Notice
that in the case of a single field with $W(\Phi)=\Phi^k$, we must have
$\alpha=1/k$ and so we
get the desired result $\widehat{c}=1-2/k$ for the minimal model.

We now want to consider  a more sophisticated argument in favor of this
equivalence, one
which reveals more of the structure of the (limiting) conformal field theory.
Recall the
structure of the $N{=}2$ algebra:
$$\{Q_\pm,\overline{Q}_\pm\}=P_\pm$$
$$\{Q_\pm,Q_\mp\}=0.$$
We have $\overline{Q}_+^2=0$ so we can consider the cohomology of
$\overline{Q}_+$, i.e., consider
local operators ${\cal O}={\cal O}(x)$ such that $\{\overline{Q}_+,{\cal
O}\}=0$.  Then
$$\partial_+{\cal O}=\left[\left\{Q_+,\overline{Q}_+\right\},{\cal O}\right]
=\left[\overline{Q}_+,\left\{Q_+,{\cal O}\right\}\right],$$
i.e., in Euclidean language (identifying $(x^+,x^-)$ with $(z,\overline{z})$)
we would say that
the class of
${\cal O}(x)$ in $\overline{Q}_+$-cohomology varies holomorphically.

Taking two such operators, using holomorphic language, we write
$${\cal O}(z){\cal O}'(z')=\sum f_k(z-z'){\cal O}_k(z')+\{\overline{Q}_+,\cdots
\},$$
with the $f_k$'s being holomorphic functions.  From this formula, we can see
that the
$\overline{Q}_+$-cohomology has the general struture of a conformal field
theory in which
the operators are ``purely left-moving,'' that is, they vary holomorphically.

An important remark is that the $\overline{Q}_+$-cohomology is invariant under
Weyl rescaling,
and hence it depends only on the complex structure of the Riemann surface (not
the metric).
To see this invariance, we must show that $T_{z\overline{z}}$ is
$\overline{Q}_+$-exact.
A natural way to prove this is to modify  the usual definition of scale
invariance, employing the
quasi-homogeneity of our polynomial, so that the superpotential term is
preserved under
the modified scaling.  Then the kinetic term in the action fails to be
scale-invariant,
but the change in the kinetic term is of the form $\int d^4\theta(\cdots)$ and
hence is
$\overline{Q}_+$-exact.

We have formulated this discussion in terms of $\overline{Q}_+$-cohomology in
order to include
very general $N{=}2$ theories.  However, whenever there is a superspace
realization of the theory
(as is true of the Landau--Ginzburg theories we are studying), it is more
convenient to
study the $\overline{D}_+$-cohomology rather than the
$\overline{Q}_+$-cohomology, where
$$\overline{D}_+=e^{-\theta^+ \overline{\theta}^+\partial_+}\,\overline{Q}_+\,
e^{\theta^+\overline{\theta}^+\partial_+}.$$

In our Landau--Ginzburg theory, the equations of motion are
$$2\overline{D}_+\overline{D}_-\Phi^j=\frac{\partial W}{\partial\Phi^j}.$$
Consider the operator
$${\cal T}=\sum_j\left(\left(\frac{1-\alpha_j}2\right)D_-\Phi^j\,\,\overline{D}
_- \,\overline{\Phi}_j
-i\alpha\overline{\Phi}_j\partial_-\Phi_j\right);$$
then $\overline{D}_+{\cal T}=0$ so that this defines a
$\overline{D}_+$-cohomology class.
\footnote{We should worry about anomalies here, but in fact there is no
quantum anomaly in the statement
that $\overline D_+{\cal T}=0$, though there are such anomalies in somewhat
similar statements
in two-dimensional models with $(0,2)$ supersymmetry.  See E. Silverstein and
E. Witten,
``Global $U(1)$ $R$ Symmetry And Conformal Invariance Of $(0,2)$ Models,''
Phys. Lett.
{\bf B328} (1994) 307.}

In the free theory, with $W=0$, ${\cal T}$ is an operator of dimension
$(1,0)$.  If we have
a superconformal theory, we will be able to write
$${\cal T}=J+\theta^-G+\overline{\theta}^-\overline{G}+\theta^-\overline{\theta
^-}T.$$
If we do arrive at an $N{=}2$ conformal field theory in the limit, then we
should expect
an OPE of the form
$${\cal T}(x)\,{\cal T}(y)=\frac{\widehat c}{(x-y)^2}
+\frac{\cal T}{(x-y)}+ \mbox{holomorphic in ${\cal O}$'s}.$$
We can check this structure in the free field theory:  the first term comes
from a diagram of
the form

\centerline{\quad}
\centerline{\epsfxsize=.6in\epsfbox{fey1.eps}}
\centerline{\quad}


\noindent
with two propagators exchanged between the two operators
and the second term comes from a diagram of the form

\centerline{\quad}
\centerline{\epsfxsize=1in\epsfbox{fey2.eps}}
\centerline{\quad}

\noindent
with exchange of only one propagator.

Note that, in computing these terms in the operator product expansion, we can
set $W=0$,
since the superpotential term is too soft (superrenormalizable) to contribute
to these
singularities.  Once this is done, the different superfields are decoupled, so
the computation of $\widehat c$ is a sum over different superfields.  So
 there is an immediate conclusion that $\widehat{c}$ is {\it additive}\/ in the
$\alpha_j$'s, that is, $\widehat{c}=\sum f(\alpha_j)$ for some function $f$.

In summary,
even in our microscopic Lagrangian, we can see an operator ${\cal T}$, which
at the level
of $\overline D_+$ cohomology obeys at $N=2$ superconformal algebra.
So presumably this theory
flows in the IR to a conformal field theory whose $\overline{D}_+$-cohomology
contains the superconformal algebra.

If we have only one chiral superfield,
so $\widehat c<1$, the $N=2$ algebra acts almost irreducibly in the quantum
Hilbert space.
Hence one should expect that
${\cal T}$ will generate the $\overline{Q}_+$-cohomology.
For $\widehat c>1$, the $N=2$ algebra acts in a way that is far from
irreducible,
and one
should expect to require additional generators.


\section{The elliptic genus}

Our refined evidence for the equivalence of these theories will come from a
computation of
the {\it elliptic genus}.  We work on a Riemann surface
of genus $1$ with period lattice spanned by $1$ and $\tau$.

\setlength{\unitlength}{.4 true in}

\begin{center}
\begin{picture}(1,2.2)(0,-.3)
\put(0,0){\line(1,0){1.5}}
\put(0,0){\line(3,5){1}}
\put(0,-.3){\mbox{$0$}}
\put(1.5,-.3){\mbox{$1$}}
\put(.7,1.6){\mbox{$\tau$}}
\end{picture}
\end{center}


\noindent
We let $q=e^{2\pi i\tau}$, and compute $\Tr
q^{(H-P)/2}\overline{q}^{(H+P)/2}(-1)^{F_R}
.$

The insertion of $(-1)^{F_R}$ means that the right-moving fermions $\psi_+$,
$\overline{\psi}_+$
should be given the {\it odd}\/ spin structure (the one periodic in both
directions) while the
left-moving fermions $\psi_-$, $\overline{\psi}_-$ should be be given the {\it
even}\/ spin
structure (anti-periodic in one direction):

\setlength{\unitlength}{.4 true in}

\begin{center}
\begin{picture}(1,2.2)(0,-.3)
\put(0,0){\line(1,0){1.5}}
\put(0,0){\line(3,5){1}}
\put(.75,-.3){\mbox{$+$}}
\put(.15,.85){\mbox{$-$}}
\end{picture}
\end{center}


For $N{=}1$ supersymmetry, that was the right thing to compute---it turns out
to be an index
$$\Tr q^{(H-P)/2}\overline{q}^{(H+P)/2}(-1)^{F_R}=\ndex(Q_++\overline{Q}_+),$$
 where $(Q_++\overline{Q}_+)^2=2P_+=H+P$.

The story is more interesting in the case of $N{=}2$ supersymmetry, since
there is the possibility
of inserting $e^{i\gamma J_L}$, which commutes with both $Q_+$ and
$\overline{Q}_+$.\footnote{We use
the same symbol $J_L$ for the left $R$-current and the conserved charge
derived from it.}  We
introduce the function
$$F(q,\gamma)=\Tr q^{(H-P)/2}\overline{q}^{(H+P)/2}e^{i\gamma J_L}(-1)^{F_R}.$$
The insertion of $e^{i\gamma J_L}$ means: when calculating the trace, glue the
data on the
torus using
$e^{i\gamma J_L}$.

The first observation is that the quantity $F(q,\gamma)$ is independent of any
continuously variable
parameters of the theory that preserve the $N=2$ structure.
This is so  as it is a linear combination of terms each of which is an index
of an operator.
For the same reason (or because the trace of the stress tensor of the theory
is $\overline Q_+$-exact)
$F(q,\gamma)$ is independent of the area of the torus, and only depends on the
complex structure.

We want to calculate $F(q,\gamma)$ in two contexts: (i) an algebraic
calculation for the known
algebraically constructed models in the IR, and (ii) directly for the
Landau--Ginzburg models
in the UV.  The comparison of these two will then provide additional evidence
that the Landau--Ginzburg
theory indeed flows to the ``minimal model'' in the IR.

\subsection{Algebraic calculation in the IR}

The Hilbert space of a ``minimal model'' takes the form
${\cal H}=\bigoplus_\alpha\,{\cal H}_\alpha \otimes \overline{\cal H}_\alpha$,
with the left-movers in ${\cal H}_\alpha$ and the right-movers in
$\overline{\cal H}_\alpha$.
Thus, the elliptic genus can be written
$$F(q,\gamma)=\sum_\alpha\left(\Tr_{{\cal H}_\alpha} q^{(H-P)/2}e^{i\gamma
J_L}\right)
\left(\Tr_{\overline{\cal H}_\alpha}\overline{q}^{(H+P)/2}(-1)^{F_R}\right).$$

The right-moving contribution $\Tr_{\overline{\cal
H}_\alpha}\overline{q}^{(H+P)/2}(-1)^{F_R}$
evaluates to $1$ if
$\overline{\cal H}_\alpha$ has highest weight with $H+P=0$ and $0$ otherwise.
This has the
effect of selecting a distinguished subset from among the irreducible
representations.  Let us introduce a restricted sum
$\sum'$ to denote the sum over that subset.  Then our elliptic genus can be
written
$$F(q,\gamma)=\sum^{\,}_\alpha\vphantom{\sum}'\, \Tr_{{\cal
H}_\alpha}q^{(H-P)/2}e^{i\gamma J_L}.$$
(Note that the eigenvalues of the charge $J_L$ are not integers in general.
Rather, the object
$\exp(2\pi i J_L)$ commutes with the $N=2$ algebra and is a constant in each
representation;
the constant depends on the $R$-charges of the chiral primary fields, which
are rational numbers.
So $F(q,\gamma)$ is periodic in $\gamma$, but the period is not $2\pi$.)

\subsection{Direct calculation in the UV}

For the Landau--Ginzburg theory itself, we can try to directly calculate in
the UV.
We use the fact that $F(q,\gamma)$ is independent of the area of the torus.
We take
the area to be very small.  Roughly, in the UV
the interaction is negligible and we should be able to set $W=0$.
The only reason for this to fail has to do with the behavior for large fields.
The bosonic part of the action is
$$\int|d\phi|^2+\int\left|\frac{dW}{d\phi}\right|^2.$$
In the presence of the $W$ term, the unique minimum of the action is $\phi=0$.
If we would set $W$ to zero, the action would be minimized for any constant
$\phi$.
One would have to integrate over the space of constant $\phi$'s even in a
leading approximation
to the path integral, and then when $\phi$ becomes sufficiently large, the $W$
term, if
its coefficient is not strictly zero, is important.

The conclusion is, then, that as long as $\phi$ is a {\it function} on the
torus, the $W$
term cannot quite be ignored.  If, however, $\phi$ were not a function but a
section of a nontrivial
flat bundle, then even at $W=0$, the $|d\phi|^2$ part of the action would have
a unique and
nondegenerate minimum.  In that situation, it would be straightforward, in the
limit
of small area on the torus, to set $W$ to zero.

In the path integral evaluation of $F(q,\gamma)$ for generic $\gamma$, we get
just such a situation
in which $\phi$ is a section of a nontrivial flat bundle.
The reason is that  $J_L$ acts on $\phi$
by $\delta\phi=\alpha\phi$.  So in evaluating the elliptic genus
we are making the identification
$$\phi(P+\tau)=e^{i\gamma\alpha}\phi(P)$$
(where we regard the torus as the quotient of the complex plane by the lattice
generated
by $1$ and $\tau$, and $P$ is any point on the complex plane).
$\phi$ is a  section of the flat bundle on the torus specified by this gluing.
If $\gamma$ is generic, the zero-section is isolated and moreover
nondegenerate minimum
of the action even if we set $W=0$.  The nondegeneracy makes computations
straightforward.
The calculation of $F(q,\gamma)$ can then be carried out in detail, setting
$W=0$ and
expressing $F$ as a ratio of determinants.
Explaining how this goes would take us too far afield,
but the result\footnote{E. Witten, ``On the Landau--Ginzburg Description of
$N{=}2$ Minimal
Models,'' Int. J. Mod. Phys. A {\bf 9} (1994) 4783--4800; {\tt
hep-th/9304026}.} is
$$
F(q,\gamma)=e^{-i\gamma k\alpha/2}\cdot
\frac{1-e^{i\gamma(k+1)\alpha}}{1-e^{i\gamma\alpha}}
\cdot \prod_{n=1}^\infty \frac{(1-q^ne^{i\gamma(k+1)\alpha})(1-q^ne^{-i\gamma(k
+1)\alpha})}
{(1-q^ne^{i\gamma\alpha})(1-q^ne^{-i\gamma\alpha})} .
$$
When compared with the algebraic calculation in the IR, one can extract
formulas for
certain characters of the  $N{=}2$ algebra.  These formulas can be verified
independently,
giving strong support to the claim that the Landau-Ginzburg theory flows to
the $N=2$ minimal
model in the infrared.

\subsection{Connection to Calabi--Yau models}

Consider now the $U(1)$ gauge theory with chiral superfields $\Phi_j$ of
charge $\alpha_j$
and $P$ of charge $-1$, such that $\sum \alpha_j=1$.  (We are no longer
assuming that all
of the $\alpha_j$'s are equal, and we are using a different normalization for
the generator
of $U(1)$ than in lecture II-12.)  Denote the gauge supermultiplet by
$$\Sigma=\sigma+\cdots ,$$
let $W(\Phi_1,\dots,\Phi_n)$ be a transverse, quasi-homogeneous polynomial as
above
(i.e., satisfying $W=\sum\alpha_j\Phi_j\frac{\partial W}{\partial\Phi_j}$,
with $W=dW=0$ only at the origin),
and consider the Lagrangian
$$\lg=\int d^2x\,d^4\theta\,\left(\frac{\overline{\Sigma}\Sigma}{4e^2}+
\overline{\Phi}^j\Phi^j\right)
+\left(\int d^2x\,d^2\theta\, PW(\Phi)+ \mbox{c.c.}\right)
+\left(\int d^2x\,d\theta^+\,d\theta^-\,t\Sigma + \mbox{c.c.}\right) $$
such that $t=\frac\theta{2\pi}-ir$.

The polynomial $PW(\Phi)$ is gauge invariant, and we can repeat our earlier
story:  for $r\gg0$ we classically
get a Calabi--Yau hypersurface $\{W=0\}$ in weighted projective space ${\bf
WCP}^{n-1}_{(\alpha_1,\dots,\alpha_n)}$
(which may inherit some singularities
from those of the weighted projective space); for $r\ll0$ we get a
Landau--Ginzburg orbifold whose
superpotential is $W$.  The quantum theory is singular only when $r=r_0$,
$\theta=0$.  ($r_0$ is
a constant computed as in lecture II-12.)
The computation of the central charge in Landau--Ginzburg theory gives
$$\widehat{c}=\sum_j(1-2\alpha_j);$$
the computation for the $\sigma$-model gives $\widehat{c}=\dim_{\bf C}X$, and
these agree since
$\sum\alpha_j=1$.

\centerline{\quad}
\centerline{\epsfysize=1.35in\epsfbox{vertcyl1.eps}}
\centerline{\quad}


The elliptic genus should be independent of $t$ since it is a topological
invariant; this leads to a
formula which relates the elliptic genus of this type of Calabi--Yau manifold
to the explicit
elliptic genus which can be computed via characters of minimal models on the
Landau--Ginzburg side.

Although the elliptic genus is independent of $t$, one of the other quantities
we have discussed for
$N{=}2$ theories---the $\overline{Q}_t$-cohomology---varies holomorphically
with $t$.  The
{\it instantons}\/ in this problem---which are holomorphic curves in the
$\sigma$-model---will
contribute to the path integral in the form
$e^{-i\theta\int\frac{F}{2\pi}}\cdot
 e^{-r\int d^2x\,D}$
where $D=\sum\alpha_j|\Phi_j|^2-r$, i.e., $e^{-2\pi it\int_\Sigma
\Phi^*(c_1({\cal L}))}$.
(I have used the fact that the instanton equation of the $N=2$ model -- the
condition
for the field to be invariant under some supersymmetry -- gives $\int d^2x
D=\int F/2\pi$.)
We will explore the instanton sums in the second half of the lecture.

\end{document}
