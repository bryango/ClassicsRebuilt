%Date: Thu, 05 Dec 1996 16:26:41 -0500
%From: Edward Witten <witten@IAS.EDU>

%\input iasmacros
%\twelvepoint
\input harvmac

{\it Problem Set Seven}

(1) (a) Show that in $\phi^4$ theory the anomalous dimension of $\phi$ vanishes
in the one loop approximation and that the one-loop beta function is therefore
given by a one-loop diagram that we have computed before.  The one-loop
beta function is in particular $\beta(\lambda)=A\lambda^2$ where $\lambda$ is
the $\phi^4$ coupling, and $A$ is the coefficient of the one-loop divergence.

(b) Convince yourself that the sign of the logarithmic divergence in this
one-loop diagram (whose significance we by now appreciate) can be determined
without any computation; it is determined by the fact that in {\it free}
scalar field theory the two point function $\langle \phi^2(x)\phi^2(0)\rangle$
is positive.  (This is actually a point that has been made previously.)


(2) Consider a four-dimensional
scalar field theory with $n$ scalar fields $\phi_i$ and
a Lagrangian

\eqn\hh{\eqalign{L=&\int d^4x\left({1\over 2}\sum_{i,\alpha}\left(
{\partial\phi_i\over\partial x^\alpha}
\right)^2\right.\cr
&+ \left.\sum_{ijkl}\lambda_{ijkl}\phi^i\phi^j\phi^k\phi^l+\dots\right),\cr}}
where 
the $\dots$ are possible quadratic and cubic terms in $\phi_j$ that are
irrelevant for the present purposes.

 The renormalization group operator will read
\eqn\hobo{D=\mu{\partial\over\partial\mu}+\sum_{ijkl}\beta_{ijkl}
{\partial\over\partial \lambda_{ijkl}}+\dots}
(the $\dots$ are anomalous dimension terms) 
with some functions $\beta_{ijkl}$.

(a) Write down the one-loop approximation to $\beta_{ijkl}$.
You are not expected to do any real calculation here;   write down
the $\beta_{ijkl}$ in terms of the $\lambda_{ijkl}$ and the coefficient
called $A$ is problem 1 above.  

(b) Show that in this approximation the renormalization group flow
\eqn\unsolved{{d\lambda_{ijkl}\over dt}=\beta_{ijkl}}
is a gradient flow (with respect to the obvious metric on the space of 
couplings).

As Gross mentioned, proving that this is true in general (or at least
that  there are no closed orbits) is a longstanding open problem.

The special case of this problem for two-dimensional sigma models
is actually quite significant in string theory.  

(More precisely, what
one wants there is that if $g$ is the metric on a sigma model target space,
then the renormalization group flow for $g$ should be, after possibly
redefining $g $ in a local fashion by a substitution such as $g_{ij}\to g_{ij}
+R_{ij}+\dots$, with $R_{ij}$ the Ricci tensor and $\dots$ analogous
terms of higher degree, of the form
\eqn\pl{{dg\over dt}={\delta\over\delta g} \Gamma}
with $\Gamma$ the integral over the target of a polynomial in the Riemann
tensor and its covariant derivatives.)

At the end of one of Gawedzki's lectures, when I asserted that locally
there is no obstruction to perturbing a Ricci-flat metric to give a zero
of $\beta$, I omitted to mention that this is so if and only if 
it is true in sigma models that the renormalization group flow is a gradient
flow in the above sense. Otherwise, there is an obstruction coming from
the Bianchi identity.


(3) Consider in 
four dimensions a Yukawa theory with a spin one-half field 
$\psi$
and a scalar $\phi$.  The part of the Lagrangian involving $\psi$ is
\eqn\kk{L=\int d^4x\left(\bar\psi(i\Gamma\cdot\partial -\phi)\psi\right).}
Compute the anomalous dimension of the operator $\bar\psi \,\psi$ in
the one-loop approximation.

(4) The goal here is to explain the claim that the magnetic coupling
is responsible for the negative beta function of nonabelian gauge theories.

We consider scalar, spinor, and vector fields in some representation $R$
of a compact gauge group $G$, on a four-manifold $M$.

The Laplacian for a scalar field, a spinor field, or a section of the
tangent bundle will be called $\nabla_0$, $\nabla_{1/2}$. and $\nabla_1$,
respectively.  The operator governing small fluctuations of the scalar,
spinor, or vector field will be called $X_0$, $X_{1/2}$, or $X_1$.
In each case $X_j=\nabla_j + G_j$ where $G$ is a   term of order zero
(absent for $j=0$) which Gross called the magnetic moment coupling.
In the case of spin one, the vector fields are really gauge fields,
and the operator $X_1$ is accompanied by some operators $X_0$ for
ghost fields that are obtained upon quantization. (This was explained by
Fadde'ev.) 

The one-loop beta function comes from the logarithmic divergence in the 
determinant
of $X_j$.  As we are interested in terms that depend on the gauge field
and not the metric of $M$, we may as well take $M$ to be flat.
In this case, $\nabla_{1/2}$ is just four copies of $\nabla_0$, and $\nabla_1$
is likewise just four copies of $\nabla_0$ (as the spinor and tangent bundles
of $M$ are trivial rank four bundles with flat Levi-Civita connection 
- -- at least, let's agree to consider the full rank four spin bundle).

Here is now to determine the $X_j$ for $j=1/2,1$.

(a) For $j=1/2$, $X_j$ is the square of the Dirac operator $D$.
(The log of the determinant of $D$ is conveniently studied as one-half
the log of the determinant of $X_{1/2} =D^2$.)  A standard calculation
shows that

\eqn\jk{X_{1/2}=D^2=\nabla_{1/2}+\Gamma\cdot F}
where $F$ is the curvature of the gauge connection and $\Gamma\cdot F$ is
Clifford multiplication.

If the second ``magnetic moment'' term were absent, the log determinant
of $X_{1/2}$ would be four times that of $X_0$ but because one want a
minus sign for the fermion loop, the contribution of the fermions to the
beta function would be negative.  The $\Gamma\cdot F$ term makes a contribution
of the opposite sign, which is larger in magnitude, so fermions actually
make a positive contribution.

(b) Now consider the gauge  fields.

Maxwell's equations read

\eqn\max{d*dA=0.}

You might therefore believe that if we have a non-abelian gauge connection
$B$ and we write $A=B+a$, with $a$ a small fluctuation, and we write $d_B$
for the covariant derivative with the connection $B$, the linearization
of the Yang-Mills equations would be
\eqn\nax{d_B*d_Ba=0.}
Not only is that not so, but the equations \nax\ are inconsistent,
in the sense that they imply the first order constraint 
$d_Bd_B(*d_Ba)=0$ which is trivial when $B=0$ but generically non-trivial;
thus the space of solutions is not ``flat'' (if I may risk the expression)
in a deformation away from $B=0$. That certainly corresponds to a pathology.
By linearizing the Yang-Mills equations you get not \nax\
but
\eqn\axx{(d_B*d_B)a+(*F_B)\wedge a=0.}
(Verify this!) 
Equations \axx\ have a ``flat'' behavior (the constraint got by applying
$d_B$ to the equations is trivial) if $B$ obeys the Yang-Mills equations.
(Verify this!) More fundamentally, \axx\ has the gauge invariance
$a\to a+d_B\epsilon$ (for any $\epsilon\in \Omega^0(M,{\rm ad G})$)
if $B$ obeys the Yang-Mills equations.  (Verify this too!)

In gauge fixing the theory, one would have introduced a gauge fixing
term as Fadde'ev explained.  This can conveniently be taken to be
$(d_B^*a)^2$, and if so the   gauge-fixed version of \axx\ is
\eqn\baxx{\nabla_1a+(*F_B)\wedge a=0.}
In \baxx\ we see finally the gauge-fixed operator $X_1$ of the vector
field.  The ``magnetic coupling'' $a\to (*F_B)\wedge a$ is the term
responsible for asymptotic freedom, in the sense that without this
term, $X_1$ would be four copies of $X_0$ and would make a positive
contribution to the beta function.

\end


























------- End of Forwarded Message




