\documentstyle[12pt,epsf]{amsart}


\setlength{\textwidth}{450pt}
\setlength{\textheight}{600pt}
\setlength{\topmargin}{0pt}
\setlength{\oddsidemargin}{0pt}
\setlength{\evensidemargin}{0pt}

% automatic loading of rsfs fonts, if present
\batchmode
  \newfont{\footscrfont}{rsfs10}
  \newfont{\footbbbfont}{msbm10}
\errorstopmode

\newif\ifscrf\scrftrue
\ifx\footscrfont\nullfont
  \scrffalse
\fi

\ifscrf
  \newfont{\scrfont}{rsfs10 scaled\magstep1}  % rsfs12 does not exist
  \newfont{\smallscrfont}{rsfs7}
  \newfont{\tinyscrfont}{rsfs7}
% \newfont{\footscrfont}{rsfs10}
  \newfont{\smallfootscrfont}{rsfs7}
  \newfont{\tinyfootscrfont}{rsfs7}
\fi

\makeatletter

\newif\iffn\fnfalse

\@ifundefined{reset@font}{\let\reset@font\empty}{} %needed for ancient LaTeX
\long\def\@footnotetext#1{\insert\footins{\reset@font\footnotesize
    \interlinepenalty\interfootnotelinepenalty
    \splittopskip\footnotesep
    \splitmaxdepth \dp\strutbox \floatingpenalty \@MM
    \hsize\columnwidth \@parboxrestore
   \edef\@currentlabel{\csname p@footnote\endcsname\@thefnmark}\@makefntext
    {\rule{\z@}{\footnotesep}\ignorespaces
      \fntrue#1\fnfalse\strut}}}
\makeatother

\ifscrf
  \newcommand{\Scr}[1]{\iffn
    \mathchoice{\mbox{\footscrfont #1}}{\mbox{\footscrfont #1}}
    {\mbox{\smallfootscrfont #1}}{\mbox{\tinyfootscrfont #1}}\else
    \mathchoice{\mbox{\scrfont #1}}{\mbox{\scrfont #1}}
    {\mbox{\smallscrfont #1}}{\mbox{\tinyscrfont #1}}\fi}
\else
  \def\Scr{\cal}
\fi


%integral sign with cirlce is \oint

%GIT quotient
\newcommand{\catquot}{\mathchoice
{\mathrel{\mskip-4.5mu/\!/\mskip-4.5mu}}
{\mathrel{\mskip-4.5mu/\!/\mskip-4.5mu}}
{\mathrel{\mskip-3mu/\mskip-4.5mu/\mskip-3mu}}
{\mathrel{\mskip-3mu/\mskip-4.5mu/\mskip-3mu}}
}

%Hodge star (hstar) should probably not be \star
\newcommand{\hstar}{\mathop{*}}

% a wedgeproduct of a more reasonable size and spacing (from Robert Bryant)
\def\wdg{{\mathchoice{\,{\scriptstyle\wedge}\,}{{\scriptstyle\wedge}}
{{\scriptscriptstyle\wedge}}{{\scriptscriptstyle\wedge}}}}
%should it be a mathbin?

%line bundle (lb) and Lagrangian (lg) -- should have distinct fonts
\newcommand{\lb}{{\Scr L}}
\renewcommand{\lg}{{\cal L}}

%same font for circle bundles
\newcommand{\cb}{{\Scr S}}

%coupling constant in gauge theory (\ee) should not be in mathitalic, to avoid
%confusion with exponential
\newcommand{\ee}{\text{e}}
\newcommand{\gee}{\text{g}}

%script A, the space of connections
\newcommand{\cA}{{\cal A}}

%script D, for use in pathspace measures
\newcommand{\cD}{{\cal D}}

%script H, for Hilbert space
\newcommand{\cH}{{\cal H}}

%notation for operators
\newcommand{\cO}{{\cal O}}

%standard blackboard bold items
\newcommand{\IR}{{\Bbb R}}
\newcommand{\IZ}{{\Bbb Z}}

%operatornames
\newcommand{\Hol}{\operatorname{\rm Hol}}
\newcommand{\Maps}{\operatorname{\rm Maps}}
\renewcommand{\Re}{\operatorname{\rm Re}}
\newcommand{\Tr}{\operatorname{\rm Tr}}
\newcommand{\vol}{\operatorname{\rm vol}}

\numberwithin{equation}{section}

\begin{document}

\title[]{Lecture II-7: Abelian Duality}
\author[]{Edward Witten$^*$}
\thanks{$^*$Notes by David R. Morrison}
\date{27 February 1997}
\maketitle

\section{Introduction}

Today, we will discuss abelian duality in two and three dimensions (with a
brief mention of four dimensions, which will be further developed in the next
lecture).
In two dimensions, abelian duality is
often referred to as the ``$R$ goes to $1/R$'' equivalence.  In a certain
supersymmetric version, it leads to a linear version of mirror symmetry.

In three dimensions, after studying the duality we will give an application
to Polyakov's model of confinement.

Abelian duality in four dimensions will eventually have an application to
Donaldson theory, that is, to $N=2$ supersymmetric theories in dimension 4.
This again gives a model of confinement.  In fact, these two applications
of duality are the most concrete models of the phenomenon of confinement
which are known.

We begin with the classical statements of duality.  In two dimensions, consider
a theory which involves fields $\phi$, $\sigma$, both obeying the
Laplace equation
\begin{equation}
\nabla^2\phi=0; \qquad \nabla^2\sigma=0,
\end{equation}
and which are related by
\begin{equation}
d\phi=\hstar d\sigma.
\end{equation}
Classically, either of these fields can be taken as the fundamental field of
the
theory.  For example, if we begin with $\sigma$ such that
$\nabla^2\sigma=\hstar d\hstar d\sigma=0$ then $d\hstar
d\sigma=0$ so locally we can write $\hstar d\sigma=d\phi$ for some field
$\phi$,
i.e., $\sigma$ determines $\phi$ (locally and up to an additive constant).

Likewise in three dimensions, consider a theory with a field $\phi$ obeying
the Laplace
equation, as well as a connection $A$ on some line bundle $\lb$ with curvature
$F=dA$ obeying Maxwell's equations
\begin{equation}
dF=d\hstar F=0,
\end{equation}
whose duality relationship is
\begin{equation}
\hstar d\phi=F.
\end{equation}
Again, as above, either of these fields can be taken as fundamental.

In four dimensions, the analogue is two connections $A$ and $B$ on line bundles
$\lb_A$ and $\lb_B$, each satisfying Maxwell's equations, and related by
\begin{equation}
F_A=\hstar F_B.
\end{equation}
For abelian duality, we could keep going to higher dimensions if we wish, but
we quickly run out of field theory applications.  (There are some applications
in string theory.)

\section{Duality in two dimensions}

We wish to study quantum versions of these classical statements.  We begin with
the two-dimensional case.  We identify $S^1$ with $\IR/2\pi\IZ$, and use
additive coordinates on the circle.  We take $\phi$ to be a map
$\phi:\Sigma\to S^1$ where
$\Sigma$ is a compact oriented surface equipped with a Riemannian metric
$g_{\alpha\beta}$.
(We will consider a variant later on, in
which $\phi$ is not required to be defined at some specified points $P_i$, that
is, $\phi$ will map $\Sigma-\{ P_i\}$ to $S^1$.)  Our Lagrangian\footnote{All
Lagrangians in this lecture are written in Euclidean signature.} is
\begin{equation}
\lg(\phi) = \frac{R^2}{4\pi}\int d^2x\,\sqrt{g}\,\partial_\alpha\phi \,
\partial^\alpha\phi = \frac{R^2}{4\pi}\int d\phi\wdg\hstar d\phi.
\end{equation}
The equations of motion $dd\phi=d\hstar d\phi=0$ reproduce the classical theory
discussed above.

We will study this theory in various ways.
The usual trick is to introduce new variables with the
property that
integrating them out would lead back to the original theory, and then integrate
out the old variables instead of the new ones in order to produce a dual
formulation of the theory.

So we actually wish to study a different theory, one which will contain fields
$\phi$ and $A$, with $\phi$ a section of a trivial $S^1$-bundle $\cb$ and $A$
a connection
on $\cb$.  Choosing a trivialization $\phi_0$ of $\cb$ (so that
$A=\phi_0^*\tilde{A}$ for some $1$-form $\tilde{A}$
on the total space of the bundle), we can define a
covariant derivative
\begin{equation}
D_A\phi = d\phi + A ,
\end{equation}
and introduce a new Lagrangian
\begin{equation}
\lg(\phi,A)=\frac{R^2}{4\pi}\int d^2x\,\sqrt{g}(\partial_\alpha\phi +
A_\alpha)(\partial^\alpha\phi + A^\alpha) =
\frac{R^2}{4\pi}\int D_A\phi\wdg\hstar D_A\phi.
\end{equation}
While we can recover the old Lagrangian by setting $A$ to zero, there is no
mechanism which enforces this, unless we introduce yet a third field $\sigma$
which plays the role of a Lagrange multiplier.  We take $\sigma$ to be a map
from $\Sigma$ to $S^1$, and write a Lagrangian
\begin{equation}
\lg(\phi,A,\sigma)=\frac{R^2}{4\pi}\int D_A\phi\wdg\hstar D_A\phi
-\frac{i}{2\pi}\int \sigma\wdg F_A,
\end{equation}
where $F_A$ is the curvature of $A$.

This last term requires some comment.  Since we are assuming that the circle
bundle $\cb$ is
trivial, we can globally write $F_A=dA$ and the last term should be interpreted
as $\frac{i}{2\pi}\int d\sigma \wdg A$ (after integration by parts); this
step is needed because $\sigma$ is not single-valued.  More
generally, to
define such a term even when $\cb$ is nontrivial, one can use a bit of
topology to define $\exp(\frac{i}{2\pi}\int\sigma\wdg F_A)$, similar to
defining a Chern--Simons form.

The point of writing this Lagrangian is that $\lg(\phi,A,\sigma)$ is equivalent
to $\lg(\phi)$, as we will now show.  A na\"{\i}ve analysis goes as follows:
$\sigma$ appears without derivatives, and its equation of motion is
$F_A=0$; imposing this, we can then go to a gauge where $A=0$ and recover
the original theory.

More globally, we consider the path integral
\begin{equation}\label{pathintegral}
Z=\frac1{\vol(G)}\int \cD \phi\,\cD A\,\cD\sigma\,\exp\left[
-\frac{R^2}{4\pi}\int D_A\phi\wdg\hstar D_A\phi
+ \frac{i}{2\pi}\int\sigma\wdg F_A
\right].
\end{equation}
(We focus on the partition function for now, but path integrals with operator
insertions can also be treated this way, as we will discuss later.)
In light of the standard formula
\begin{equation}
\int \frac{dx}{2\pi} e^{ixy}=\delta(y),
\end{equation}
we would like to say that doing the $\sigma$-integral will set $F_A$ to zero.
We should treat this statement with care, since we are studying circle-valued
functions.

Bearing in mind our identification of $S^1$ with $\IR/2\pi\IZ$, our map
$\sigma:\Sigma\to S^1$ can be locally
written as a real-valued function, but it might not be globally single-valued;
however, $d\sigma$ will be a (single-valued) real $1$-form on $\Sigma$.
Choose a circle-valued function $\sigma_h:\Sigma\to S^1$ such that
$d\sigma_h$ is the {\it harmonic}\/ representative in the de~Rham cohomology
class
of $d\sigma$.  (We normalize the choice of $\sigma_h$ by picking some point
$P\in\Sigma$
and demanding that $\sigma_h(P)\in 2\pi\IZ$.)
Then we can write $\sigma=\sigma_h+\sigma_{\IR}$, with $\sigma_{\IR}$
a single-valued real function.

Notice that $\frac1{2\pi}d\sigma$, or equivalently $\frac1{2\pi}d\sigma_h$,
must have integral periods.
In particular, if
we choose a basis $\lambda_j$ of integral harmonic $1$-forms, and write
$d\sigma_h=\sum2\pi m_j\lambda_j$, then $m_j\in \IZ$.

Now we compute:
\begin{equation}
\begin{align}
\int\cD\sigma\, e^{\frac{i}{2\pi}\int\sigma\wdg F_A}
&=\int \cD\sigma_{\IR}\,e^{\frac{i}{2\pi} \int \sigma_{\IR}\wdg F_A}
\sum_{d\sigma_h\in H^1(\Sigma,2\pi\IZ)}e^{-\frac{i}{2\pi}d\sigma_h\wdg A}
\\&= \delta(F_A)\,\prod_j\left(\sum_{m_j\in\IZ} e^{-i
m_j\int\lambda_j\wdg A}\right).
\end{align}
\end{equation}
The first factor tells us that $A$ is a flat connection.  Moreover, among
flat connections the gauge equivalence classes $[A]$ are labeled by the
holonomies,
or equivalently by the quantities $\int \lambda_j\wdg A$.  Since the
remaining part of the integrand is gauge invariant, we can gauge fix (omitting
the
factor of $(\det G)^{-1}$) and
integrate over the space of gauge equivalence classes (a finite dimensional
integral):
\begin{equation}
\begin{align}
Z&=\int\cD\phi\,\cD[A]\,e^{-\frac{R^2}{4\pi}\int D_A\phi\wdg\hstar D_A\phi}
\delta(F_A)\,\prod_j\left(\sum_{m_j\in\IZ} e^{-i
m_j\int\lambda_j\wdg A}\right)
\\&=\int\cD\phi\,e^{-\frac{R^2}{4\pi}\int D_A\phi\wdg\hstar D_A\phi}
\delta(F_A)\,\delta(\int \lambda_i\wdg A=0\mod 2\pi\IZ).
\end{align}
\end{equation}
So there are delta functions setting the holonomies as well
as curvature to zero.
So $A$ is zero modulo gauge transformations, and our new theory
is indeed equivalent to the original theory, with partition function
\begin{equation}
Z=\int\cD\phi\,e^{-\frac{R^2}{4\pi}\int d\phi\wdg\hstar d\phi}.
\end{equation}


Now let us integrate out in the opposite order, integrating out $\phi$ and $A$
but keeping $\sigma$.  To integrate out $\phi$, we fix the gauge in such a
way that
$\phi=0$, and suppress the factor of $(\vol(G))^{-1}$ from the path integral.
Then the path integral \eqref{pathintegral} reduces to
\begin{equation}
\int \cD A\,\cD\sigma\,\exp\left[
-\frac{R^2}{4\pi}\int A\wdg\hstar A
+ \frac{i}{2\pi}\int\sigma \wdg F_A
\right].
\end{equation}
An exercise you might enjoy is verifying that the Faddeev-Popov determinant
associated with this gauge fixing is
\begin{equation}
\int\cD c\,\cD\overline{c}\,\exp\left[-\frac{R^2}{4\pi}\int d^2x\,
\overline{c}c \right].
\end{equation}


To do the integral over $A$, we need to complete the square, thinking of
the second
term in the exponent as
$-\frac{i}{2\pi}\int A\wdg d\sigma$,
i.e., it is the term linear in $A$.  We make a change of variables
$A'=A+\frac{i}{R^2}\hstar d\sigma$;
the result, including the Faddeev-Popov integral, is
\begin{equation}
\left(
\int\cD c\,\cD\overline{c}\,e^{-\frac{R^2}{4\pi}\int d^2x\,
\overline{c}c }\right)
\left(
\int \cD A' e^{-\frac{R^2}{4\pi}\int A'\wdg\hstar A'}\right)
\left(
\int \cD \sigma e^{-\frac1{4\pi R^2}\int
d\sigma\wdg\hstar d\sigma}\right).
\end{equation}
(The Gaussian integral over $A'$, like the Faddeev-Popov determinant, can
be thought of as a
normalization factor.)

The Faddeev-Popov integral gives a factor of $(\int dt\,e^{-tR^2/4\pi})^{-1}$
for
each $0$-form and
each $2$-form on $\Sigma$; the integral over $A'$ gives a factor of $\int
dt\,e^{-tR^2/4\pi}$
for each $1$-form on $\Sigma$.  Thus, if $n_j$ denotes the ``number of
$j$-forms
on $\Sigma$'', those terms combine to give an overall factor of
\begin{equation}
(\sqrt{\pi/(R^2/4\pi)})^{(-n_0+n_1-n_2)} = (R/2\pi)^{\chi(\Sigma)}.
\end{equation}
That is, the transformation rule which relates these dual formulations is
\begin{equation}
\int \cD\phi\,\exp\left[-\frac{R^2}{4\pi}\int d\phi\wdg\hstar d\phi\right]
= (R/2\pi)^{\chi(\Sigma)}
\int \cD\sigma \exp\left[-\frac1{4\pi R^2}\int d\sigma\wdg\hstar
d\sigma\right].
\end{equation}
(The factor of $(R/2\pi)^{\chi(\Sigma)}$ did not show up in previous explicit
calculations
we have done because they were done in genus 1.)  A coupling of $R$ in the
first
theory maps to a coupling of $1/R$ in the second theory, which is why this
duality
is sometimes referred to as ``$R$ goes to $1/R$.''

Now we would like to follow the operators in this theory through the duality
transformation, and determine the effect on correlation functions as well as
the
partition function.  The easy case is $d\phi$, which we expect to map to
$\hstar d\sigma$, as in the classical theory (actually it will map to a
multiple
of $\hstar d\sigma$, when normalizations are taken into account).
We need to repeat the above calculation with an insertion of operators
$\prod \cO_i(\phi)$, one of which is $d\phi$.  In order to do this, we need
a gauge-invariant extension of the operator $d\phi$, which is provided by
the covariant derivative:
\begin{equation}
d\phi(x) \text{ in first theory} \longrightarrow D_A\phi(x) \text{ in big
theory}.
\end{equation}
One then easily checks that this maps back to $d\phi$ when we integrate out
$\sigma$ (and gauge fix $A$ to zero).

The difficult thing about mapping operators in general will be finding the
appropriate extension to an operator in the larger theory, which reduces to
the original operator when $\sigma$ has been integrated out.

How will our calculation be modified?  Earlier, when we completed the square,
we made a change of variables $A'=A+\frac{i}{R^2}\hstar d\sigma$.  This
means that if $D_A\phi$
has been inserted in the path integral, it now becomes $D_{A'}\phi
-\frac{i}{R^2}\hstar d\sigma$.
Thus, when $A'$ is integrated out, we are left with an insertion of
$-\frac{i}{R^2}\hstar d\sigma$
in the dual theory.

Some care must be used in manipulating these mappings between operator
insertions.  For example, even though $d\phi$ maps to  $-\frac{i}{R^2}\hstar
d\sigma$,
$(d\phi)^2$ will {\it not}\/ map to $(\frac{i}{R^2}\hstar d\sigma)^2$ in the
dual theory, due to
nonlinearities introduced when we complete the square.

What made this case easy was that the covariant derivative is a natural
gauge-invariant extension of the ordinary derivative.

The hard operator insertion to deal with is $e^{i\phi}$.  There is no
gauge-invariant
version of this.  On the other hand, we don't really need it, because
$\langle e^{i\phi}(P)\rangle=0$.  On the other hand, the two-point correlators
$\langle e^{i\phi}(P)\, e^{-i\phi}(Q)\rangle$ are not zero in general, so
we should try to dualize such insertions, when $P\ne Q$.

We are going to do something rather strange.
It is impossible to construct a gauge-invariant extension {\it locally}\/ for
this pair of operator insertions, so we work non-locally during the duality
(i.e., in the intermediate theory), introducing a term in the path integral
of the form
\begin{equation}\label{nonlocal}
e^{i\phi}(P) e^{-i\phi}(Q) e^{\frac{i}{2\pi}\int \theta\wdg A}
\end{equation}
where $\theta$ is a $1$-form such that
\begin{equation}
\frac1{2\pi}d\theta=\delta_P-\delta_Q
\end{equation}
and all periods of $\theta $ are $0$ modulo $2\pi \IZ$.

To find such a $\theta$, take a path $\ell$ from $P$ to $Q$ and let $\theta$ be
Poincar\'e dual to $\ell$ (regarded as a
distribution).  We can rewrite our term \eqref{nonlocal} as
\begin{equation}
e^{i\phi}(P) e^{\frac{i}{2\pi}\int_\ell  A} e^{-i\phi}(Q) ,
\end{equation}
from which we see that this is a gauge-invariant expression.  (This is an
example of a choice of $\theta$, but we allow
more general $\theta$'s than that.)  We will eventually think of $\theta $
as a connection form on a trivial $S^1$-bundle on $\Sigma- P- Q$.

Thus, we have found a gauge-invariant nonlocal object \eqref{nonlocal}.
Suppose we try to insert this into the path integral.  First, if we set $A$ to
zero, we recover our original operator insertion $e^{i\phi}(P)
e^{-i\phi}(Q)$, so we have correctly
extended this pair of operators to the gauge theory.

To integrate out the other way, we gauge fix $\phi$ to zero, which reduces
\eqref{nonlocal} to $e^{\frac{i}{2\pi}\int \theta\wdg A}$.  This is a new
contribution to the term linear in $A$ in the exponent of the path
integral, of the form $-\frac{i}{2\pi}\int A\wdg\theta$,
so to complete the square we now must make the change of variables
$A'=A+\frac{i}{R^2}\hstar(d\sigma+\theta)$,
and this makes our dualized path integral take the form
\begin{equation}
\int \cD\sigma\,\exp\left[-\frac1{4\pi R^2}\int
(d\sigma+\theta)\wdg\hstar(d\sigma+\theta)\right].
\end{equation}
That is, we have obtained the same theory as before, but now formulated on
$\Sigma-P-Q$, and now performing the path integral over maps to ${ S}^1$
that do {\it not} extend over $P$ or $Q$.  That is, we write
$\widetilde{\sigma}=\sigma+\alpha$, with $d\alpha=\theta$, $\alpha$ being a map
from $\Sigma-P-Q$ to $S^1$, and then $\widetilde{\sigma}$ is the map
of $\Sigma-P-Q$ to $S^1$ that does not extend over $P$ or $Q$.
In fact, it has winding numbers 1 and $-1$ around $P$ and $Q$,
respectively.


\epsfxsize3in
\vskip.2in
\centerline{\epsfbox{winding.eps}}
\vskip.2in

There are two kinds of operators now, one local in one picture, and the other
local in the other picture.  The operator $e^{in\phi}(P)$ in one
description is mapped to an instruction ``delete $P$ from $\Sigma$,
and perform the path integral over $\sigma$'s that have winding number
$n$ around $P$'' in the other description.  Conversely, there is
an ordinary local operator $e^{in\sigma}(P)$ in the second description,
that corresponds to a nonlocal recipe in the first description.


\section{Duality in three dimensions}

We now turn to three-dimensional theories.  Our initial theory is the theory
of an $S^1$-valued function $\phi$ on a (fixed) compact $3$-manifold $M$
(possibly
with boundary),
governed by the Lagrangian
\begin{equation}
\lg=\frac{\Lambda}{4\pi} \int
d^3x\,\sqrt{g} \, \partial_\alpha\phi \, \partial^\alpha\phi
= \frac{\Lambda}{4\pi} \int d\phi\wdg\hstar d\phi .
\end{equation}
One novel feature of three dimensions is that the prefactor
$\Lambda$ has the dimensions of mass.
Since $\Lambda$ is a dimensionful quantity rather than a constant, it will be
impossible for anything special to happen at a numerical value of $\Lambda$.

We reinterpret $\phi$ as a section of a trivial $S^1$-bundle $\lb_B$ with a
connection
$B$, and write a Lagrangian which includes the covariant derivative
$D_B\phi=d\phi+B$:
\begin{equation}
\lg(\phi,B)=\frac{\Lambda}{4\pi}\int d^3x\,
(\partial_\alpha\phi+B_\alpha)(\partial^\alpha\phi+B^\alpha)
=\frac{\Lambda}{4\pi}\int D_B\phi\wedge\hstar D_B\phi
\end{equation}
This would be a trivial theory if left as it is.  To get somewhere, we
introduce
a line bundle $\lb$ with connection $A$, and the Lagrangian
\begin{equation}
\lg(\phi,B,A)=\frac{\Lambda}{4\pi}\int D_B\phi\wedge\hstar D_B\phi
-\frac{i}{2\pi}\int A\wdg F_B,
\end{equation}
interpreting the last term as $\frac{-i}{2\pi}\int F_A\wdg B$ (after
integrating
by parts) which makes sense since  $\lb_B$ is trivial.
As before, the extension to the case of $\lb_B$ not being trivial is a term in
the
path integral
$e^{-\frac{i}{2\pi}\int A\wdg F_B}$,
the Chern--Simons form for the structure
group $U(1)\times U(1)$ of the bundle
$\lb\oplus \lb_B$.

In order to carry out the duality transformation, we need to sum over line
bundles $\lb$, producing a path integral
\begin{equation}
\frac1{\vol(G)\vol(G')}\sum_{\lb}\int\cD\phi\,\cD A\,\cD B\,
\exp\left[-\frac{\Lambda}{4\pi}\int D_B\phi\wedge\hstar D_B\phi
+ \frac{i}{2\pi}\int F_A\wdg B\right].
\end{equation}
where $G$ and $G'$ are gauge groups for $\lb$ and $\lb_B$.

We would like to integrate out $A$.  The crude statement is that we get
$\delta(F_B)$,
which implies that $B$ is a flat connection.  As in the computation we
did in two dimensions, the complications arise from
the possibility of nontrivial holonomies for $B$.
In fact, if $F_B=0$ we can regard $B$ as an element of $H^1(M,\IR/\IZ)$ and
define for each $x=c_1(\lb)$, the quantity
$e^{i\int x\wdg [B]}$.  When this is summed over $[x]$
in doing the path integral,  the result is $\delta([B])$, showing that
$B$ is gauge-equivalent to the trivial flat connection.
Thus, we reduce back to the original theory;
we've learned that the extended theory with $A$ and ${\cal L}$ is equivalent
to the original theory.

Now we do the integral in the opposite direction, by doing the $\phi$ and $A$
integrals.  As in the two-dimensional case, we set $\phi$ to zero by a gauge
transformation, removing the normalization factor $(\vol(G))^{-1}$.  (Unlike
in two dimensions, the Faddeev-Popov determinant does not contribute anything
interesting, so we will not explicitly include it.)  The square can be
completed in the resulting path integral
\begin{equation}
\begin{align}
&\frac1{\vol(G')}\sum_{\lb}\int\cD A\,\cD B\,
\exp\left[-\frac{\Lambda}{4\pi}\int  B\wdg\hstar B
+ \frac{i}{2\pi}\int B\wdg F_A\right]
\\&=
\frac1{\vol(G')}\sum_{\lb}\int\cD A\,\cD B'\,
\exp\left[-\frac{\Lambda}{4\pi}\int  B'\wdg\hstar B'
-\frac1{4\pi\Lambda}\int F_A\wdg\hstar F_A\right],
\end{align}
\end{equation}
where we shifted $B'=B-\frac{i}{\Lambda}\hstar F_A$.

\noindent
The integral in $B'$ is Gaussian, and can be absorbed into the normalization
of the path integral, leaving us with
\begin{equation}
\frac1{\vol(G')}\sum_{\lb}\int\cD A\,
\exp\left[
-\frac1{4\pi\Lambda}\int F_A\wdg\hstar F_A\right],
\end{equation}
the familiar path-integral from Yang--Mills theory!  (With gauge coupling
$\ee=\sqrt{\Lambda}$.)

Now we want to map operators.  An analogous calculation to the one we did in
two dimensions shows that $d\phi$ extends to the covariant derivative
$D_B\phi=d\phi+B$ in the larger theory, and becomes
$\frac{i}{\Lambda}\hstar F_A$ in the dual theory.
This is the meaning in the quantum theory of the duality between solutions
of the Laplace equation and solutions of Maxwell's equation.

What about  operators such as $e^{i\phi}(P)$?  We will not do this, though
I will state the answer. $e^{in\phi}(P)$ is mapped, in the dual gauge theory,
to the instruction: delete the point $P$ from the 3-manifold $M$ and consider
the path integral for connections on a line bundle ${\cal L}$ whose
first Chern class evaluates to $n$ on a small sphere around $P$.  The argument
leading to this is quite along the lines of what we did in two dimensions.

I will not present that argument (which I recommend as an exercise)
and instead discuss what is roughly the reverse.  We
consider Wilson loop operators in the gauge theory, and try to map those
back to the scalar field theory.

Let $C$ be a circle in $M$ which is a boundary, and let $\lambda\in\IR$.
We have an operator $\exp(i\lambda \oint _C A)$  in the gauge theory.
Let us insert this into the path integral of the big theory (the one with
Lagrangian $\lg(\phi,B,A)$).
(More generally, we can do this even if $C$ is not a boundary, but then
$\lambda$ must be restricted to be an integer.  $C$ being a boundary, say
of a two-surface $D$, lets us write $\exp(i\lambda \oint _C A)
= \exp(i\lambda\int_DF)$, showing gauge-invariance for any $\lambda$. )

When we integrate out $A$ with this insertion, instead of getting
$\delta(F_B=0)$
we get $\delta(F_B=2\pi\lambda[C])$ with $[C]$ being the Poincar\'e dual to
$C$.    In other words,
performing the $A$ integral determines $B$ to be such that
$F_B=2\pi\lambda[C]$,
and the modified Lagrangian after integrating out is $\int D_B\phi\wedge\hstar
D_B\phi$.
Here $B$ is a
flat $U(1)$-connection with monodromy $e^{2\pi i \lambda}$ around the circle.
So the Wilson loop operator for a circle $C$ is dual to a recipe
``delete $C$ and interpret $\phi$ as a section of a flat bundle with
monodromy around $C$.''
In other words, we modify $\phi$ to $\widetilde{\phi}$ such that
$d\widetilde{\phi}=d\phi+B$, and interpret $\widetilde{\phi}$ as a section of
a trivial $S^1$-bundle on $M-C$ with a flat connection of monodromy
$e^{2\pi i\lambda}$
around $C$.  (To get such a trivial $S^1$-bundle, we either need to assume that
$C$ is a boundary, or that $\lambda$ is an integer.)

\section{Application to the Polyakov model}

We will describe a model constructed by Polyakov around 1975.  It was the first
use of duality in a nonlinear relativistic theory.  The model exhibits the
phenomenon known as confinement.

We work in three dimensions, and study $SO(3)$ gauge theory with a scalar
field $s$ in the $3$-dimensional representation, governed by the Lagrangian
\begin{equation}
\lb=\frac1{4\pi \ee^2}\int d^3x\, \Tr|F|^2 + \int d^3x\,(D\vec{s})^2
+\int d^3x\,\lambda(\vec{s}\,{}^2-a^2)^2.
\end{equation}
This model exhibits the Higgs mechanism at tree level.

Classically, the vacuum can be rotated to
\begin{equation}
\langle\vec{s}\rangle=\begin{pmatrix}a\\0\\0\end{pmatrix}
\end{equation}
by a gauge transformation.  In the vacuum state, the $SO(3)$ symmetry of
the Lagrangian is broken to an $SO(2)=U(1)$ symmetry, and the only massless
field is the $U(1)$-connection.  (This is the so-called Higgs mechanism.)
The low-energy theory looks like the $U(1)$ theory in three dimensions.

{\bf First Step}.
This theory looks classically, at low energies, like a $U(1)$ gauge theory with
Lagrangian
\begin{equation}
 \lg = \frac1{4\ee^2}\int F\wdg\hstar F.
\end{equation}
Is that the answer?  Is the Gaussian fixed point of the free $U(1)$ theory
stable?

One possible source of instability is a Chern--Simons interaction
\begin{equation}
\frac{-in}{2\pi}\int A\wdg F_A
\end{equation}
which could be added to the original Lagrangian $\lg$.  (In the flow to the
infrared, this term is more relevant than those appearing in $\lg$, hence
the instability.)

There are two obstacles to this being a source of instability in our problem.
The first is that the coefficient $n$ in the Chern--Simons term must be an
integer.  This implies that even if you can only approximately calculate the
theory, you can determine $n$ if the approximation can be made arbitrarily
accurate. One can, in particular, calculate $n$ in perturbation theory,
and -- as higher loop terms would involve positive powers of $e$ -- it could
only arise from a one-loop term.
So the effective $n$ in any three-dimensional gauge theory can be determined
by an explicit one-loop computation.
In the specific example we are considering here,
one  can simply notice that the Chern--Simons term is odd under parity
(i.e., it depends on a choice of orientation of the $3$-manifold) whereas
our original theory was not.  The parity-invariance is  a symmetry of the
one-loop determinants (whether or not parity ultimately is spontaneously
broken at low energies) and ensures that $n=0$.



Thus, anything which could make this theory unstable will be hard to
describe in terms of $A$.  However, we know that the $U(1)$ gauge theory is
equivalent (dual) to a scalar theory
\begin{equation}
\lg=\frac{\ee^2}{4\pi}\int d^3x\,\partial_\alpha\phi\partial^\alpha\phi
\end{equation}
In this theory, we could add a term
\begin{equation}
g\int d^3x\,\cos n\phi
\end{equation}
for some $n$.
We have chosen the potential to be periodic as $\phi$ is really a map
to a circle.
Notice that the theory with $g\ne0$ has a mass gap, the theory with $g=0$
does not.
Thus, we have identified something which qualitatively changes the physics,
but can only be conveniently interpreted by means of the dual variables.
This is strange, because doing the duality requires going to a low
energy description to begin with!

How do we see this effect in the original $SO(3)$ theory?  Consider the Feynman
diagrams of the $SO(3)$ theory.  We represent these with massless modes given
by wavy lines, massive ones by solid lines.  A typical Feynman diagram such as

\vskip.2in
\centerline{\epsfxsize=1in\epsfbox{feyn3.eps}}
\vskip.2in

\noindent
will be completely tame: we have massive propagators, which are analytic in the
momenta at low momentum.  Such diagrams merely give corrections to the
effective action for the $U(1)$ gauge field $A$ which can be described
as additional local, gauge-invariant terms in the Lagrangian.

We need something completely different: {\it instantons}.  Let us work on
$\IR^3$.
We have an $SO(3)$ bundle on $\IR^3$ whose structure group has been
reduced to $U(1)$ at infinity.  On the $S^2$ at infinity, a $U(1)$ bundle $\lb$
can have a nontrivial first Chern class, $c_1(\lb)\ne0$.



Pick a trivialization of the $SO(3)$ bundle, so that we can identify the
spatial $\IR^3$ with the bundle $\IR^3$, and simply treat our section $s$
as a  map $s:\IR^3\to\IR^3$.  To construct such a map with nontrivial topology
on the $S^2$ at infinity, we take $s$ of the form
\begin{equation}
s(\vec{x})=a\frac{\vec{x}}{|x|}f(|x|),
\end{equation}
with $f(|x|)$ an increasing function satisfying $f(0)=0$,
$\lim_{|x|\to\infty}f(|x|)=a$, e.g.,

\centerline{\epsfxsize=2in\epsfbox{plot4.eps}}

\noindent
Such a section is invariant under combined
 rotations of space and gauge rotations.

We would like the minimum action solution $s$ in this class.  It should be
spherically symmetric, with asymptotic behavior $Ds\sim\frac1{|x|^2}$ for
$|x|\to\infty$.  This leads to an ODE, which has a unique
solution.  The action
can be written as $(I/\ee^2)$ for some constant $I$, so it will diverge as
the coupling goes to $0$.

Note that since $\vec s$ vanishes precisely
at the origin, the structure group is
reduced to $U(1)$ away from the origin, but this reduction does not
extend over the origin.  In fact, over a sphere surrounding the origin,
the line bundle has a nonzero first Chern class (which actually is 2
if we work in $SO(3)$; that is, the adjoint bundle of $SO(3)$ decomposes
over a two-sphere surrounding the origin as ${\cal O}\oplus {\cal L}
\oplus {\cal L}^{-1}$ where ${\cal L}$ has degree 2).

{}From this point of view, the difference between a pure $U(1)$ theory
and a theory that looks like a $U(1)$ theory only at long distances
is that the latter can have bundles that over a large ${\bf S}^2$
at infinity have a nonzero first Chern class.  The Chern class can
be any multiple of the 2 that was found in the explicit solution that
we just described.


Note that in abelian gauge theory, it was not possible to use a bundle $\lb$
for which $c_1(\lb)\ne0$ over a sphere around the origin.  The qualitative
difference between an $SO(3)$ theory broken to $U(1)$ and a $U(1)$ theory
is that the former admits singularities where the $U(1)$ description
breaks down in this way.

If we did a similar thing in $3+1$ dimensions, we would find a time-independent
solution of finite energy (rather than finite action).  This solution looks
like a particle sitting there, and in fact is a magnetic monopole.  (The
nonzero magnetic charge comes from the fact the $c_1(\lb)\ne0$ which implies
that the magnetic field integrates to a nonzero amount.)  The point of making
this analogy is that we can think of the instanton as behaving like a zero-time
slice of a magnetic monopole.


What do instantons look like at long distances?  A monopole has a field which
behaves like
\begin{equation}
F=\text{const}\,\frac{\vec{x}}{|x|^3}=\text{const}\,\frac{\hat{x}}{|x|^2}
\end{equation}
(by Maxwell's equations).

The contribution to the path integral from each instanton is $e^{-I/\ee^2}$,
which is small.

To return to our theory, we now ask how it behaves at long distance.  For
example, how does a two-point function $\langle F(x)\,F(0)\rangle$ behave as
$|x|\to\infty$?

The answer in the free theory, by dimensional analysis, is $-1/|x|^3$.  (One
sees that the curvature $F$ has dimension $3/2$ in three-dimensional theories
by considering the basic term $\frac1{4\pi\ee^2}\int F\wdg\hstar F$ in the
Lagrangian.)  A bit more precisely, we are asserting that
\begin{equation}
\langle
F_{ij}(x)\,F_{k\ell}(0)\rangle=\frac1{|x|^3}(\delta_{ik}\delta_{j\ell}+\dots).
\end{equation}

The Feynman diagrams don't affect this asymptotic behavior: there is a slight
renormalization of $\ee^2$, plus other corrections which are unimportant at big
 distances.

How does an instanton affect this analysis?  Consider an instanton localized
near $y$, and its effect on the two-point function between $x$ and $0$.

\epsfxsize=4in
\centerline{\epsfbox{instanton.eps}}

\noindent
The leading approximation to the path integral in the instanton sector is
\begin{equation}
e^{-I/\ee^2}\int d^3y\,\frac{(\vec{x}-\vec{y})}{|x-y|^3}\,
\frac{\vec{y}}{|y|^3}.
\end{equation}
Thus, the overall behavior is
\begin{equation}
\begin{align}
\langle F(\vec{x})\,F(\vec{0})\rangle
&\cong
\frac1{Z}\left(\text{pert. theory} + e^{-I/\ee^2}(\text{instanton
sector})\right)
\\&\cong
\frac1{Z}\left((1+\dots)\frac1{|x|^3}+e^{-I/\ee^2}(\frac1{|x|}+\dots)\right).
\end{align}
\end{equation}
Since the instanton contribution is more important in the infrared than the
free theory term ($1/|x|$ compared to $1/|x|^3$),
the instanton triggers an instability.

To see in more detail what is happening in the infrared, we first note that
instantons are {\it rare}: the probability to have an instanton in
a small volume $V_0$ is proportional to $V_0e^{-I/\ee^2}$, so
the volume of space per
 instanton is given by $V\sim
e^{I/\ee^2}$, and  the spatial separation between two of them is $R\sim
{V}^{1/3}\sim e^{I/3\ee^2}$.

In the infrared, though, if we consider a large enough volume of space,
we will get lots of instantons, which we can treat as
a gas of particles of definite size.  The particles are {\it charged}, so
we can't ignore the interactions between them, given by Coulomb potentials.
The picture is as follows

\epsfxsize=3in
\centerline{\epsfbox{gas.eps}}

\noindent
We have labeled the positions of instantons (which are positively charged) with
$s_i$'s, and the positions of anti-instantons (which are negatively charged) by
$t_j$'s.  The sum over all of these takes the form
\begin{equation}
\sum_{n,m=0}^\infty \frac1{n!m!}\int d^3s_i|_{i=1}^{n}
\int d^3t_j|_{j=1}^{m} e^{-\frac{I}{\ee^2}(n+m)}
e^{
\sum_{i<j}\left(\frac1{|s_i-s_j|}+\frac1{|t_i-t_j|}\right)
-\sum_{i,j}\frac1{|s_i-t_j|}
}.
\end{equation}
We also need an operator insertion, of
\begin{equation}
F(x)=\sum \frac{x-s_i}{|x-s_i|^3}-\sum \frac{x-t_j}{|x-t_j|^3},
\end{equation}
and similarly for $F(0)$.


The physics involved is the classical statistical mechanics of a plasma in
space, with chemical potential $I/\ee^2$; energy $E$ given by the Coulomb
potential between the instantons and anti-instantons, and temperature
 $T=4\pi\ee^2$.

The phenomenon we need is known as ``Debye screening''---a plasma screens
external charges.  As a result of this screening, the system will have a mass
gap.

Here is a quick mathematical derivation of Debye screening (in this context).
We go back to our low-energy theory, writing in dual variables---a scalar
theory with $F=\hstar d\phi$.  We will add a term to the Lagrangian to account
for the instanton effect: the term we will use (justified by the results of the
calculation to come) is $\int e^{-I/\ee^2}(e^{2i\phi}+e^{-2i\phi})$.
(The 2 is present in the exponent because the basic instanton has first
Chern class 2.)
The operator
insertions of $F$ become insertions of $\hstar d\phi$, and the quantity we are
calculating can be written:
\begin{equation}
\Omega=\int \cD\phi\,\left(\hstar d\phi(x)\right)\left(\hstar d\phi(0)\right)
\exp\left[-\frac1{4\pi\ee^2}\int|d\phi|^2+\int
e^{-I/\ee^2}(e^{2i\phi}+e^{-2i\phi})\right].
\end{equation}
First we want to show that this is a correct description in the dual variables,
then we will analyze this version.

We will expand $\Omega$ in perturbation theory:
\begin{equation}\label{expanded}
\Omega=\sum_{n,m=0}^\infty \frac1{n!m!}\int \cD\phi\,
e^{-\frac1{4\pi\ee^2}\int d\phi\wdg\hstar d\phi}
\left(\hstar d\phi(x)\right)\left(\hstar d\phi(0)\right)
\left(\int e^{-I/\ee^2}e^{2i\phi(y)}d^3y\right)^n
\left(\int e^{-I/\ee^2}e^{-2i\phi(z)}d^3z\right)^n.
\end{equation}
We expand further, using the principle that $\left(\int dy\,f(y)\right)^n
=\int dy_1\dots dy_n\,f(y_1)\dots f(y_n)$.  Thus, we can rewrite
\eqref{expanded} as
\begin{equation}
\Omega=\sum_{n,m=0}^\infty \int \cD\phi\int d^3s_i|_{i=1}^n d^3t_j|_{j=1}^m
\left(\hstar d\phi(x)\right)\left(\hstar d\phi(0)\right)
e^{-\frac{I}{\ee^2}(n+m)} e^{2i\sum(\phi(s_i)-\phi(t_j))}
e^{-\frac1{4\pi\ee^2}\int|d\phi|^2}.
\end{equation}

The $\phi$-integral is Gaussian after the square has been completed; that
integral will contribute
\begin{equation}
e^{\sum_i\sum_j G(s_i,t_j)}
\end{equation}
to the overall answer, where $G(s,t)=1/|s-t|$ is the Green's function for the
Laplacian.  We thus recover the previous formula.
Note that the
$i=j$ terms contributed a divergence which is renormalized, so they do not
appear.

\bigskip\bigskip

The lesson we have learned is this: our original problem ($SO(3)$ gauge theory)
can be described in the infrared, using dual variables, by means of the
Lagrangian
\begin{equation}
\frac1{4\pi\ee^2}\int|d\phi|^2+e^{-I/\ee^2}\int d^3x\,\cos 2\phi.
\end{equation}
The second term in unrenormalizable, but is well-behaved in the infrared.
Expanding around the minimum of the potential, we see that $\phi$ has
a mass.  Thus, what used to be a massless $U(1)$ photon was
(i) dualized and reinterpreted as a scalar, and (ii) got a mass
that was more easily described in that language.

Here is another way to understand the above derivation.  Start
with the massless scalar field and dualize it, as in section 3
of this lecture, to a massless gauge field $A$.  Now perturb the
theory of the massless scalar by adding a weak perturbation
$\int d^3x \epsilon(e^{22i\phi}+e^{-2i\phi})$.  In terms of the gauge
field, the operator $e^{2i\phi(P)}$ becomes, as we noted at the end of
section 3, an instruction ``delete the point $P$, and consider bundles
with  first Chern class 2 on a small sphere around $P$.''  In other words,
from the point of view of an $SO(3)$ that looks like $U(1)$ at low energies,
the instruction is ``include an instanton at $P$.''
If we take the interaction term $\int d^3x \epsilon(e^{2i\phi}+e^{-2i\phi})$,
and expand in powers of $\epsilon$, we simply generate the instanton
gas, with each insertion of $e^{2i\phi}$ or $e^{-2i\phi}$ corresponding
to an instanton or antiinstanton.

Now we want to study confinement.
An important preliminary is to note the symmetries of the problem.
In general, in the duality from an abelian gauge theory, $\phi$
is an angular variable, a map to a circle, so $\phi$ is equivalent to
$\phi+2\pi$.  However, because the instanton-induced interaction is
a trigonometric function of $2\phi$, there is a symmetry under $\phi\to
\phi+\pi$.


In the original description, we could
consider a curve $C\subset M$

\epsfxsize=1.5in
\centerline{\epsfbox{trefoil.eps}}

\noindent
and the associated Wilson line operator
\begin{equation}
\langle \Tr_R\Hol(A,C)\rangle
\end{equation}
where $R$ is the $2$-dimensional representation of $SU(2)$ and $\Hol$
denotes the
holonomy. What does this operator translate to in our infrared description?
As we have seen at the end of section 3, it translates into a recipe ``delete
$C$ from spacetime and interpret $\phi$ as a section of a flat circle
bundle with monodromy around $C$.'' The monodromy is given by the angle
$2\pi \lambda$ where $\lambda$, introduced in the discussion in section
three,
is the $U(1)$ charge appearing in the Wilson loop, modulo 1.  We have chosen
a Wilson line in the two-dimensional representation of $SU(2)$; the
$U(1)$ charges are $\pm 1/2$ in units of the weight lattice of $SO(3)$.
Hence our example has $\lambda=1/2$.
The monodromy around the circle is thus $\phi\to\phi+\pi$, which,
as we noted a moment ago, is a symmetry of the theory even with
the instanton-induced interaction.

The problem is now classical: we want to find the minimum of the action
\begin{equation}
 \frac1{4\pi\ee^2}\int|d\phi+B|^2-e^{-I/\ee^2}\int
d^3x\,\cos 2\phi
\end{equation}
with $\phi$ required to have monodromy $\pi$ around $C$.
Thus, in particular, $\phi$ cannot be constant.
   In fact, the least action solution will be given
approximately in terms of a
surface $\Sigma$ with $\partial\Sigma=C$ of least area; $\phi$ will be constant
 far
from $\Sigma$, and will jump by $\pi $ in crossing $\Sigma$.

The easiest case is when $C$ is a large curve in the plane, say    given by
$x=0$, with $x$ a linear function on ${\bf R}^3$.
We want $\phi$ to jump by $\pi$ in going from negative $x$ to positive
$x$.   In the limit that $C$ is very large, $\phi$, if observed
somewhere deep in the interior of $C$, becomes a function of
$x$ only.
  What needs to be minimized is then
\begin{equation}
\int dx\,((\frac{d\phi}{dx}+B(x))^2-(\cos 2\phi-1))
\end{equation}
with the boundary conditions that $\phi\to 0$ for $x\to -\infty$ and
$\phi\to \pi$ for $x\to+\infty$.
This variational problem,
with the boundary conditions, has a solution that is unique
up to translation of $x$, with $\phi$ approaching its asymptotic
value exponential fast (because of the mass gap) and with some action $L$.

\noindent

Now in general, if $D$ is the minimal area surface with boundary
$C$, and $A(D)$ is its area, the minimum action $\phi$ that is a section
of the appropriate flat bundle has the property that $\phi$ is very
near zero or $\pi$ except near $D$, jumps by $\pi$ in crossing $D$,
and looks in profile near $D$ just like the solution of the idealized
one-dimensional problem discussed in the last paragraph.
Its action is very nearly $A(D)L$, so the expectation value of the Wilson
line is approximately $e^{-A(D)L}$, showing the area law and confinement.

\end{document}
