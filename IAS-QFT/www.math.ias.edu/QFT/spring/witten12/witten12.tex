\documentstyle{article}

\input epsf.tex

%These are the macros which are in common with all of the
% sections in the paper mmr
% Each section, for now, should begin with \documentstyle[11pt,cd]{article}
% and then have \input{mmrmacros} followed by \begin{document}
% The only exception is that the \Label macro is slightly different
% in each file and should be put in separately.
%New CD macros
\newcommand{\cdrl}{\cd\rightleftarrows}
\newcommand{\cdlr}{\cd\leftrightarrows}
\newcommand{\cdr}{\cd\rightarrow}
\newcommand{\cdl}{\cd\leftarrow}
\newcommand{\cdu}{\cd\uparrow}
\newcommand{\cdd}{\cd\downarrow}
\newcommand{\cdud}{\cd\updownarrows}
\newcommand{\cddu}{\cd\downuparrows}
% (S) Proofs.
% (S-1) Head is automatically supplied by \proof.

\def\proof{\vspace{2ex}\noindent{\bf Proof.} }
\def\tproof#1{\vspace{2ex}\noindent{\bf Proof of Theorem #1.} }
\def\pproof#1{\vspace{2ex}\noindent{\bf Proof of Proposition #1.} }
\def\lproof#1{\vspace{2ex}\noindent{\bf Proof of Lemma #1.} }
\def\cproof#1{\vspace{2ex}\noindent{\bf Proof of Corollary #1.} }
\def\clproof#1{\vspace{2ex}\noindent{\bf Proof of Claim #1.} }
% End of Proof Symbol at the end of an equation must precede $$.

\def\endproof{\relax\ifmmode\expandafter\endproofmath\else
  \unskip\nobreak\hfil\penalty50\hskip.75em\hbox{}\nobreak\hfil\bull
  {\parfillskip=0pt \finalhyphendemerits=0 \bigbreak}\fi}
\def\endproofmath$${\eqno\bull$$\bigbreak}
\def\bull{\vbox{\hrule\hbox{\vrule\kern3pt\vbox{\kern6pt}\kern3pt\vrule}\hrule}
}
\addtolength{\textwidth}{1in}                  % Margin-setting commands
\addtolength{\oddsidemargin}{-.5in}
\addtolength{\evensidemargin}{.5in}
\addtolength{\textheight}{.5in}
\addtolength{\topmargin}{-.3in}
\addtolength{\marginparwidth}{-.32in}
\renewcommand{\baselinestretch}{1.6}
\def\hu#1#2#3{\hbox{$H^{#1}(#2;{\bf #3})$}}          % #1-Cohomology of #2
\def\hl#1#2#3{\hbox{$H_{#1}(#2;{\bf #3})$}}          % #1-Homology of #2
\def\md#1{\ifmmode{\cal M}_\delta(#1)\else  % moduli space, delta decay of #1
{${\cal M}_\delta(#1)$}\fi}
\def\mb#1{\ifmmode{\cal M}_\delta^0(#1)\else  %moduli space, based, delta
                                              %decay of #1
{${\cal M}_\delta^0(#1)$}\fi}
\def\mdc#1#2{\ifmmode{\cal M}_{\delta,#1}(#2)\else    %moduli space, delta
                                                      %decay, chern class #1
                                                      %of #2
{${\cal M}_{\delta,#1}(#2)$}\fi}
\def\mbc#1#2{\ifmmode{\cal M}_{\delta,#1}^0(#2)\else   %as before, based
{${\cal M}_{\delta,#1}^0(#2)$}\fi}
\def\mm{\ifmmode{\cal M}\else {${\cal M}$}\fi}
\def\ad{{\rm ad}}
\def\msigma{\ifmmode{\cal M}^\sigma\else {${\cal M}^\sigma$}\fi}
\def\cancel#1#2{\ooalign{$\hfil#1\mkern1mu/\hfil$\crcr$#1#2$}}
\def\dirac{\mathpalette\cancel\partial}
\newtheorem{thm}{Theorem}
\newtheorem{theorem}{Theorem}[subsection]
\newtheorem{proposition}[theorem]{Proposition}
\newtheorem{lemma}[theorem]{Lemma}
\newtheorem{claim}[theorem]{Claim}
\newtheorem{example}[theorem]{Example}
\newtheorem{corollary}[theorem]{Corollary}
\newtheorem{D}[theorem]{Definition}
\newenvironment{defn}{\begin{D} \rm }{\end{D}}
\newtheorem{addendum}[theorem]{Addendum}
\newtheorem{R}[theorem]{Remark}
\newenvironment{remark}{\begin{R}\rm }{\end{R}}
\newcommand{\note}[1]{\marginpar{\scriptsize #1 }}
\newenvironment{comments}{\smallskip\noindent{\bf Comments:}\begin{enumerate}}{
\end{enumerate}\smallskip}

\renewcommand{\thesection}{\Roman{section}}
\def\eqlabel#1{\addtocounter{theorem}{1}
\write1{\string\newlabel{#1}{{\thetheorem}{\thepage}}}
\leqno(\rm\thetheorem)}
\def\cS{{\cal S}}
\def\ov{\overline}








%Dave's macros
%\numberwithin{equation}{section}
\newcommand{\operatorname}[1]{\mathop{\rm #1}\nolimits}
\newcommand{\Tr}{\operatorname{Tr}}


\title{Lecture II-12: $N=2$ SUSY theories in Dimension Two: Part I}
\author{Edward Witten\thanks{Notes by John Morgan and David R. Morrison}}
\date{April 10, 1997}


\begin{document}
\maketitle

\section{Introduction}

In this lecture and the next two we consider $N=2$ supersymmetric field
theories in  dimension two. Today, after some general introductory
remarks about such theories, we consider two-dimensional
$U(1)$-gauge theories with complex-valued chiral superfields charged
under the gauge group. These theories depend on two functions --
a superpotential
$W$ and a twisted superpotential $\widetilde{W}$.
The superpotential $W$ is a complex polynomial in the superfields.
This polynomial must satisfy a certain `weight' conditions, but is otherwise
free.
For gauge group $U(1)$,
the twisted superpotential
$\widetilde{W}$ of the sort of model we will look at
is determined by a single
complex parameter $t=-ir+\frac{\theta}{2\pi}$, or more precisely by ${\rm
exp}(-2\pi it)$.
Today we will keep the superpotential is fixed and we
allow the twisted superpotential to vary.  Thus, for each (generic)
complex polynomial $W$ satisfying the weight conditions, we get a
family of theories parameterized by the point ${\rm exp}(2\pi it)$ in
the cylinder ${\bf C}/{\bf Z}$.
We are interested in what happens at the ends of the cylinder ${\rm
Im}(t)\mapsto +\infty$ and ${\rm Im}(t)\mapsto-\infty$.
In the cases we consider, the limit as  ${\rm Im}(t)\mapsto -\infty$ is
described in terms of a
$\sigma$-model on the projective variety $X$
defined by the polynomial
$W$ giving the superpotential.  In this region, $-{\rm Im}(t)$ is the Kahler
class of $X$, and the fact that there is only one parameter in the
superpotential is related to the fact that $H^{1,1}(X)$ (or at least
the part of it that is pulled back from a certain ambient weighted
projective space) is one-dimensional.
The limit as ${\rm
Im}(t)\mapsto +\infty$ is described in terms of an orbifold
version of a Landau-Ginzburg model.

When $c_1(X)=0$, we will find that the beta function vanishes for
all $t$, and we  get a family of conformally
invariant theories.   For $c_1(X)\not= 0$, we get instead
a non-trivial renormalization group flow in $t$.
The flow  increases ${\rm Im}\, t$  when $c_1(X)>0$ and
decreases ${\rm Im}\, t$ when $c_1(X)<0$. (Furthermore, in these latter two
cases there are ``extra'' vacua at the  end of the cylinder to which one
flows in the infrared.)  This allows
us to see various
Landau-Ginzburg models as infra-red limits of $\sigma$-models and
other $\sigma$-models as infra-red limits of Landau-Ginzburg  models.
This is interesting because the descriptions of these types of models
are very different at the classical level.   Most of today's lecture
will deal with special cases
(the cases that $c_1(X)=0$, and some of the cases with
$c_1(X)>0$ with flow to a massive infrared theory) that have been extensively
studied, but the general picture that  will be presented
at the end of the lecture is actually new, as far as I know.


\section{Generalities on $N=2$ SUSY Theories in Two Dimensions}

\subsection{The  $\beta$-function of $N=2$ SUSY $\sigma$-models}

The first remark concerns $N=2$ supersymmetric $\sigma$-models with target a
compact K\"ahler manifold $X$, which we assume has a K\"ahler-Einstein
metric. In Gawedski's lectures we have seen that the one-loop
$\beta$-function is determined by the sign of the Ricci curvature of
the metric
and in fact when the Ricci tensor is positive or negative-definite, the sign
of the one-loop $\beta$-function is opposite that of the Ricci curvature.
Thus,
if $c_1(X)>0$ then the one-loop $\beta$-function is negative.
Thus, at small coupling the $\beta$-function is negative  so that the
$\sigma$-model under consideration is a
`good'  fundamental theory which is
asymptotically free. Such a theory is well-defined at the quantum level.
The question arises of what it flows to in the infrared.
Today we shall see examples of
such $X$ for which the infra-red
theory is massive and trivial, and also examples with flow to a non-trivial
infra-red fixed point.
In the latter case, the
limiting superconformal theories that we will get can be
 described in terms of what are known as
supersymmetric Landau-Ginzburg theories.
The Landau-Ginzburg models are sometimes
explicitly soluble by algebraic methods.

On the other hand, if $c_1(X)<0$, then the one-loop $\beta$-function is
positive and hence $\beta$ is positive at weak coupling.
Thus, we have
no reason to expect
$\sigma$-models for such $X$ to be `good' fundamental theories. Rather
in this case the $\sigma$-model is an
effective theory, free in the infra-red, and one should
ask whether one can describe a concrete, well-defined ultraviolet
theory that flows to these sigma models in the infrared.
We will solve this problem for this class of sigma models
by taking ${\rm Im}t\,\to\infty$, where we will find
a Landau-Ginzburg  theory (which is superconformal), which after a relevant
perturbation (to ${\rm Im}\,t$ large and positive but not $\infty$)
flows to the sigma model with target space $X$ (plus some extra massive vacua).


The case when $c_1(X)=0$, i.e., the case when $X$ is  Calabi-Yau, is
different. Here, the one loop $\beta$-function vanishes and
one-loop contributions do not change the K\"ahler metric. It turns out
that there are higher loop corrections to the $\beta$-function so that
it is not identically zero and hence the $\sigma$-model on $X$ (with
its Ricci-flat metric) is not conformal.
Nevertheless, standard invariant theory arguments show that there is
no higher loop correction to the K\"ahler class of the metric. Using Yau's
theorem about existence of K\"ahler metrics on $X$ with prescribed
Ricci curvature ($\partial\overline\partial$-exact) we can modify (at
least order-by-order in perturbation theory) the original Ricci-flat
K\"ahler metric without changing the complex structure or the K\"ahler
class until the $\sigma$-model for the new K\"ahler structure on $X$
has vanishing $\beta$-function.  This process will converge if the
K\"ahler class is sufficiently ample and far from the boundaries of the
K\"ahler cone; for today's models these conditions simply mean that
$-{\rm Im}\,t$ should be large enough.  What we will be exploring
today is really the question of what happens when $-{\rm Im}\,t$ is not large.

Today we shall see examples of $X$ with $c_1(X)=0$ for which there is
a family  of theories, parametrized by a punctured cylinder that approaches
the $\sigma$-model on $X$ as ${\rm Im}t\mapsto -\infty$ and approaches  a
Landau-Ginzburg model as ${\rm Im}t\mapsto +\infty$.  To get such a
family, the cylinder must be punctured at one point, where the theory
has a sort of pole.   It is believed
that the punctured cylinder parametrizes a smooth family of theories,
which flow in the infrared to  a punctured cylinder of
superconformal field theories. As one check of this belief, we argue that
the central charge of the Landau-Ginzburg theory at $+\infty$ and the
central charge of the $\sigma$-model at $-\infty$ are equal.
We will present more evidence for the scenario described today
 in a later lecture.


\subsection{Reasons for considering $N=2$ SUSY theories}

One advantage of studying $N=2$ supersymmetric theories is that they are much
more rigid than theories with less (or no) supersymmetry.
For example if we are considering two-dimensional $\sigma$-models into
a compact Riemannian manifold $X$, then without supersymmetry we can
add terms to the Lagrangian of the form
$$\int d^2y\,\varphi^*(h)$$
for a potential function $h\colon X\to {\bf R}$.
(Here, $\varphi\colon \Sigma\to X$ is the basic field in the
Lagrangian.) In $N=1$ supersymmetric $\sigma$-models, such a term
is not possible, but there can still
be a term
$$\int d^2y\,d^2\theta\,\Phi^*(h)$$
where again $h$ is a real-valued function on $X$ (called the superpotential)
and $\Phi$ is the basic
superfield in the Lagrangian.
In both cases this is a relevant perturbation.
When we require $N=2$ supersymmetry, the Riemannian manifold must be K\"ahler
and the possible potential term can still exist but must be of the
form $h={\rm Re}(f)$ for $f$ a holomorphic function.
Since our Riemannian manifold is assumed to be compact, this means
that $f$ is constant and hence, after performing the $\theta$ integrals,
this term vanishes.
Simple power counting shows that at the classical level
this is the only possible relevant
perturbation over a $2|4$-superspace.

\subsection{Revelant and Marginal Perturbations of the conformal
two-dimensional $\sigma$-model}

Let us study the marginal perturbations of these conformal
$\sigma$-models.
Part of the data of the $\sigma$-model is the complex structure on $X$.
Varying this is a marginal perturbation of the theory.
Let us examine this perturbation
in local coordinates. The metric tensor on $X$ is a $(1,1)$-tensor of the
form $h_{i\overline j}dz^i\otimes d\overline z^j$. A change in the complex
structure corresponds to a perturbation of the metric by a term of the
form
$$\delta h_{\overline i\overline j}+\delta h_{ij}$$
of type $(2,0)+(0,2)$. Of course, this form must be real  in order
that the new form be a metric tensor, and $\delta
h_{\overline i\overline j}$ must be $\overline\partial$-closed in order that
the
new almost complex structure is integrable.
Let us compute the change in the Lagrangian that goes with this
perturbation of the K\"ahler metric. We write things in local
coordinates using bosonic coordinates $y^\alpha, \alpha=1,2$ on the Riemann
surface and fermionic coordinates $\theta^\pm,\overline\theta^\pm$. We
write superfields as $X=X(y,\theta,\overline\theta)$,
chiral superfields as $\Phi(y,\theta)$ and anti-chiral superfields as
$\overline\Phi(y,\overline\theta)$.
The change in the Lagrangian is given by
$$\Delta {\cal L}=\left(\int d^2y\,d^2\overline\theta \delta h_{\overline
i\overline j}\overline D_+\overline X^{\overline i}\overline
D_-\overline X^{\overline j}\right)+{\rm c.\ c.}.$$

It is a nice exercise to verify that the argument of the $\overline{\theta}$
integral  in $\Delta {\cal L}$ is
anti-chiral. (This uses the fact that $\delta h_{\overline i \overline j}$ is
$\overline\partial $-closed.) This  is one of the two
possible types of marginal perturbations.

The other type of marginal perturbation is to fix the complex
structure $X$ and vary the complexified K\"ahler class of the K\"ahler
metric. One piece of this deformation is an ordinary  variation
$\delta h^{\rm herm}_{\overline i j}d\overline z^{\overline i}\wedge
dz^j$ by a
hermitian  symmetric closed
two-form. This ordinary metric perturbation is the real part of a
complex perturbation whose imaginary part is called the ``$B$-field''
perturbation (which, as we shall see, is analogous to the $\theta$-angle
which appears in gauge theories).  Thus, we consider a perturbation
$\delta h=\delta h_{\overline i j}d\overline z^{\overline i}\wedge
dz^j$  where $\delta h$ is a closed complex two-form.
This leads to a perturbation of the Lagrangian of the form
$$\Delta{\cal L}=\left(\int d^2yd\overline\theta^+d \theta^-(\delta
h_{\overline i
j}\overline D_+\overline X^{\overline i}D_-X^u\right)+{\rm c.\ c.}.$$

Other than the marginal deformations of the theory described above, there
can only be deformations of the form
$$\int d^2y d^4\theta({\rm something}).$$
For these the $\theta$-integration has weight $2$, and hence all such
terms are marginal (of dimension two) classically.
(Notice this count requires four $\theta$'s and
hence uses the fact that we have $N=2$ SUSY).  But at the quantum level,
they have anomalous dimensions (proportional to nonzero eigenvalues of the
Laplacian and so strictly positive) and are irrelevant.
 These types of deformations can affect the
K\"ahler metric without changing the complex structure or K\"ahler
class.

These computations show that
for the  $\sigma$-model with
$c_1(X)=0$, the theory has no relevant perturbations and the
marginal perturbations are obtained by deforming the complex structure
and complexified K\"ahler metric.

Let us begin with such a $\sigma$-model with a Kahler metric that is smooth and
has a large radius of curvature close to,
but not equal to, the metric that actually gives zero beta function.
The renormalization group flow will bring us to a metric that gives
conformal invariance, as all other deformations are irrelevant.
Were there relevant perturbations these would dominate in the infrared limit
and the limit would be in general impossible to control.





\subsection{Central Charge}

Let us review some material on central charges in conformal and
super-conformal field theories in dimension two. There are two basic
types of  $N=0$,
i.e., of ordinary, conformal field theories -- those with central
charge $c<1$, and those with central charge $c\ge 1$. The
former are called minimal models, exist only for a discrete set of
values of $c$, and are explicitly
algebraically described in terms of representation theory of the Virasoro
algebra.
The theories with $c=1$ can also be explicitly described, in terms of
a free boson.  When $c\leq 1$, the Virasoro algebras of left- and right-
movers act almost irreducibly in the quantum Hilbert space, which
is why these theories are known rather explicitly.  For $c>1$, the
Virasoro action is very far from being irreducible (the number of  highest
weight vectors grows exponentially as one goes to higher energies), and
conformal field theory has a completely different flavor.

When we go to the $N=1$ super-conformal algebra in dimension two,
the central charge again lies either in a
discrete set or in  a continuous  part. The lowest value of the
central charge in the continuous part is realized by the free scalar
super-field. Being the sum of a free scalar boson and a free scalar
fermion, considered as a representation of the ordinary conformal
algebra  this field has
central charge  $c=3/2$. For this reason one usually defines $\widetilde
c=2c/3$ so that the continuous values for $\widetilde c$ begin at
$\widetilde c=1$.

The story is similar for the $N=2$ super-conformal algebra in
dimension two. Here the lowest value in the continuous part is
realized by the free chiral super-field. Since this field consists of
a complex  scalar boson and a complex fermion, its $N=0$ central
charge  is $3$. Hence, we set $\widehat c=c/3$ so that once  again the
lowest value in the continuous part is $\widehat c=1$.
The discrete set of $N=2$ theories with $\widehat c<1$
are the ``$N=2$ minimal models''
which we will consider in detail later on.  Their central charges take the
form $\widehat{c}=1-\frac2n$ for $n=3, 4, 5, \dots$, and they
can be described algebraically and studied in a variety of attractive ways.

A conformal $N=2$ supersymmetric
$\sigma$-model with target space a $k$-dimensional complex manifold
$X$ of $c_1(X)=0$
has central
charge $\widehat c= k$.   This is proved by using the fact that $\widehat c$ is
constant in a family of conformal field theories, and expanding the metric
until $X$ can be approximated, in a local region, to any desired
accuracy, to ${\bf C}^k$ with a flat metric.  Thus, this sigma model
has the same $\widehat c$ as a sigma model with target a flat ${\bf C}^k$.
That is a free theory with $k$ chiral superfields, and so has $\widehat c=k$,
as claimed.

\section{The $U(1)$ Theories}

Let us turn now to the  study of $U(1)$-gauge theory with complex-valued
chiral superfields $A_1,\ldots,A_n$. These superfields are charged
under the $U(1)$ with charges $q_1,\ldots,q_n\in {\bf Z}$.
Recall from the superhomework that connections we use on $2|4$-space are
constrained to be flat in all pairs on odd directions except the pair
$\{\overline\theta^+,\theta^-\}$ (and its complex conjugate).
The basic invariant of such a connection is
$$\Sigma=\{\overline{\cal D}_+,{\cal D}_-\}$$
which is a section of the complexification of the adjoint bundle over
$2|4$-space.
The minimal pure gauge theory Lagrangian that we can write down is
$$\int d^2yd^4\theta\frac{1}{2e^2}\overline\Sigma \Sigma.$$
Adding in a kinetic term for
the $n$ chiral superfields, the minimal Lagrangian becomes
\begin{equation}\label{minlag}
{\cal L}_0=\int d^2yd^4\theta\left(\frac{1}{2e^2}\overline\Sigma
\Sigma+\sum_{i=1}^n\overline A_iA_i\right).
\end{equation}
Eventually, we will add a superpotential and a twisted superpotential
term to this Lagrangian.


\subsection{$R$-symmetries}
The above
$N=2$ supersymmetric gauge theory in two dimensions (or $2|4$ supertheory)
comes by dimensionally reducing from $N=1$ gauge theory in
four-dimensions (or $4|4$ supertheory). As such, there is an
$R$-symmetry for the $N=2$ two-dimensional theory induced by spatial
rotation in the remaining two dimensions in four-space.
But, in addition, the $N=1$ supersymmetric version of
$U(1)$-gauge theory in four-dimensions itself has an $R$-symmetry and
that symmetry continues to exist even if we add a superpotential.
This $R$-symmetry from dimension four dimensionally reduces to give a
second $R$-symmetry for our $2$-dimensional theory.

Independent of their sources, it is easy to see directly in the
$2|4$-theory what the $U(1)\times U(1)$-group of $R$-symmetries is.
We have $J_R$, the circle action
$$\theta^+\mapsto e^{i\alpha}\theta^+$$
$$\overline\theta^+\mapsto e^{-i\alpha}\overline\theta^+$$
with $\theta^-$ left fixed, and we have $J_L$, the circle action on
$\theta^-$ with $\theta^+$ left fixed. The $R$-symmetry that comes
from the $R$-symmetry of four-dimensional gauge theory is the product
of these two (i.e., it acts by the standard representation on both
$\theta^\pm$). Spatial rotation in the omitted two dimensions induces the
circle action which is the
product of the first times the inverse of the second of these.
Potentially there is an anomaly in one of these $R$-symmetries. To
have a super-conformal theory this anomaly must vanish (or be
cancelled) since these symmetries are part of the super conformal
algebra.

Let us make more explicit the earlier remark that the anomaly of the
$\sigma$-model is
proportional to the first Chern class $c_1(X)$. The point is that when
we write the Lagrangian out in coordinates we have an expression of
the form
$${\cal L}=\int d^2y\left(g_{i\overline j}\partial X^i\partial
\overline X^{\overline
j}+\overline\psi_-^{(0,1)}\dirac_+\overline\psi_-^{(1,0)}+
\psi_+^{(0,1)}\dirac_-\psi_+^{(1,0)}\right)$$
(plus a four-fermion term that will not affect the present discussion).
One of the  basic $R$-symmetries $J_L$ rotates the $\psi_-^{(0,1)}$ and
$\psi_-^{(1,0)}$  and the other $J_R$ rotates in the same manner
$\psi_+^{(0,1)}$ and $\psi_+^{(1,0)}$. Thus, under a chiral
rotation generated by $a_RJ_R+a_LJ_L$,
 the measure of integration in the path integral ${\cal
D}\psi_+{\cal D}\psi_-$ changes by an amount proportional to
$$a_R{\rm index}(\dirac_+)+a_L{\rm index}\dirac_-.$$
The  operators $\dirac_\pm$ are Dirac operators of positive or negative
chirality acting on the pullback to the Riemann surface of the holomorphic
tangent bundle $T^{(1,0)}(X)$.
By the index theorem both of these operators
have index given by $c_1(X)$ evaluated on the fundamental class of
the Riemann surface.
Thus, we see that the $R$-symmetry induced from the $R$-symmetry in
four-dimensions is not anomalous, since
${\rm index}\dirac_+-{\rm index}\dirac_-=0$,
but that the $R$-symmetry coming
from the spatial rotation has an anomaly:
${\rm index}\dirac_++{\rm index}\dirac_-=2{\rm index}\dirac_+$.

We have just given a computation for the anomaly for the
$\sigma$-model. Let us make an analogous computation in $U(1)$-gauge
theories with chiral superfields $A_i, 1\le i\le n$.
The superfield $A_i$ is a section of
the $q_i$-power of the line bundle associated to the connection on
$2|4$-space.   It can be expanded in terms of ordinary fields, both
bosons and fermions.
The $R$-symmetries act as chiral symmetries on the fermions.  So the
same type of  anomaly-index computation as above shows that the
anomaly of $J_R$ is given by
$$\sum_{i}q_i\int d^2y\frac{F_A}{2\pi}$$
where $F_A$ is the curvature of the ordinary connection on the Riemann
surface.
(The anomaly for $J_L$ is the negative of this number.)
Thus, in order to get a superconformal field theory we need
$\sum_iq_i=0$.


\section{One Example}

To make things as simple and concrete as possible let us focus on one
special case of our $U(1)$-gauge theory: $n+1$ chiral superfields
$$A_1,\ldots, A_n\ \ {\rm of\ weight}\ \ q=1$$
$$A_{n+1}=P\ \ {\rm of\ weight}\ \ q=-n.$$
This choice of weights assures us that our $R$-symmetries are
anomaly-free.
Now it is time to supplement the minimal Lagrangian ${\cal L}_0$ of
Equation~\ref{minlag} by a
superpotential and a twisted superpotential.
That is to say we consider Lagrangians of the form
\begin{equation}\label{full}
{\cal L}={\cal L}_0+\left(\int d^2yd\theta^+d\theta^-
W(A,P)+{\rm c.\ c.}\right) + \left(\int
d^2yd\overline\theta^+d\theta^-\widetilde{W}(\Sigma)+{\rm c.\ c.}\right).
\end{equation}
Here, $W$ is called the superpotential; it must be   a holomorphic
function on ${\bf C}^{n+1}$. The function $\widetilde{W}(\Sigma)$ is called
the twisted chiral
superpotential term.

Let us examine the consequence of assuming that each
of these two
terms in the Lagrangian is invariant under the $R$-symmetries. We
consider the twisted chiral superpotential term $\widetilde{W}$ first.
Here, it is important to use the fact that our theory is obtained by
dimensionally reducing a gauge theory  on $4|4$-space.
Starting with a connection
$A=\sum_{\mu=0}^3A_\mu dx^\mu$ on $4|4$-space we form
$$\sum_{\alpha=0}^1dy^\alpha+\sigma(dx^2+idx^3)+{\rm c.\ c.}.$$
This shows us how $\Sigma$ transforms under the $R$-symmetries.
For the extra term in Equation~\ref{full} coming from the twisted
superpotential to be invariant, it must have the same transformation
law under the $R$-symmetries as $\Sigma$ does. That is to say
$\widetilde{W}(\Sigma)$ must be  a linear function of $\Sigma$, i.e.,
$\widetilde{W}(\Sigma)=t\Sigma$ for some $t\in {\bf C}$.
Writing
$$t=\frac{\theta}{2\pi}-ir,$$
we have
$$\left(\int d^2yd\overline\theta^+d\theta^-t\Sigma\right) +{c.\
c.}=\theta\int \frac{F}{2\pi}+r\int D.$$
Thus, we see that adding such a linear twisted superpotential produces
a Fayet-Iliopulous term $r\int D$ plus a $\theta$-angle term which is
$\theta$ times an integral topological quantity. In this case the
Fayet-Iliopulous term corresponds to adding a constant to the moment
map for the $U(1)$-action on ${\bf C}^{n+1}$.

Now let us turn to the superpotential $W(A,P)$. For the term
$$\int d^2yd\theta^+d\theta^-W(A,P)$$
to be invariant, we need $W(A,P)$ to transform with weight two under
the $R$-symmetry induced from the $R$-symmetry in four dimensions and
be invariant under the $R$-symmetry induced by spatial rotation in
four-space.
In order to arrange this, we choose the action of the $R$-symmetry
groups on the line bundles of which the $A_i$  and $P$ are sections as
follows:
the line bundles $L_i$ with sections $A_i$  are invariant under the
$R$-symmetries and the line bundle $L_{n+1}$ with section $P$ is
invariant under  the $R$-symmetry induced
by spatial rotation and has weight two under the $R$-symmetry coming
from the $R$-symmetry of four-space. With this choice of $R$-symmetry
actions any superpotential of the form
$$W=P\cdot F(A_i)$$
for $F$ an arbitrary polynomial of $A_1,\ldots, A_n$, is invariant under
the $R$-symmetries.
There is of course, one extra condition, namely that $W$ must be gauge
invariant. This implies that $F$ must be homogeneous of degree $n$.
Thus, any $W$ of this form will produce a gauge invariant
term in the Lagrangian which is invariant under the $R$-symmetries.
This is the most general interaction with these symmetries.










\subsection{Classical analysis}

Now that we have specified our theory in detail, we
 want to determine what this theory is, particularly in the limits
$r\to\pm\infty$.  We begin with a classical
analysis.  Write each of the superfields appearing in our Lagrangian
in components
$$A=a+\theta(\cdots)+\cdots$$
$$P=p+\theta(\cdots)+\cdots$$
$$\Sigma=\sigma+\theta(\cdots)+\cdots$$
The superfield $\Sigma$ includes auxiliary fields $D$ and $F$ which can be
eliminated by means of their equations of motion.

The bosonic potential is a function $V(a,p,\sigma)$ of the bosonic
components of these fields.  For the Lagrangian we have described, the
potential takes the form
\begin{equation}\label{bospot}
V(a,p,\sigma)
=\frac1{2e^2}D^2+|dW|^2+|\sigma|^2\left(\sum|a_i|^2+n^2|p|^2\right).
\end{equation}
The last term
in eq.~(\ref{bospot}) arises as follows: in four dimensions, a
chiral field $\Phi=\phi+\theta(\cdots)+\cdots$ leads to a term in the
potential of the form
$$\sum_{\mu=0}^3\left|\frac{D\phi}{Dx^\mu}\right|^2$$
which upon reduction to two dimensions becomes
$${} \sum_{\alpha=0}^1\left|\frac{D\phi}{Dx^\alpha}\right|^2
+n^2|\sigma|^2|\phi|^2$$
if $\phi$ has charge $n$.

In general, the ``$D$-term'' (the auxiliary field in $\Sigma$), is
equal by its equations of motion to $\sum q_i|\phi_i|^2-r$.
This is actually a familiar function mathematically; it is the ``moment
function'' generating the $U(1)$ gauge action on the flat Kahler manifold
(a copy of ${\bf C}^s$, with $s$ the number of chiral superfields)
in which the chiral superfields take their values.
 In our example, this becomes
\begin{equation}\label{dterm}
D=\sum|a_i|^2-n|p|^2-r.
\end{equation}
Since $W=P\cdot F(A)$, we also have
\begin{equation}\label{dW}
|dW|^2=|F|^2+|p|^2|dF|^2.
\end{equation}
We will focus on the case in which $F$ is {\it transverse}, i.e., $F=dF=0$
only at the origin.

We want to find all classical zero-energy solutions, in other words, to
solve $V=0$.  Thanks to the form of the potential, this requires that
(i) $D=0$, (ii) $|dW|^2=0$ and (iii) either $\sigma=0$ or
$\sum|a_i|^2+n^2|p|^2=0$.
We will also need to take the gauge equivalence under $U(1)$
into account.

First, note that thanks to eq.~(\ref{dW}), setting $|dW|^2=0$ when $p\ne0$
implies
$F=dF=0$ so by our transversality condition, all $a_i=0$.  Thus, there are
two cases: $p=0$ or $a_i=0$ for all $i$.  On the other hand, if $r\ne0$,
then by eq.~(\ref{dterm}) since $D=0$ we cannot have both $p=0$ and $a_i=0$ for
all $i$.  (In fact the sign of $r$ will determine which one of these
holds.)  Thus, when $r\ne0$ we have $\sum|a_i|^2+n^2|p|^2\ne0$ and so
$\sigma$ must be $0$.

Consider now the case $r>0$.  Our equations for classical vacua become
$p=0$, $\sum|a_i|^2=r$, and $F=0$, and we must divide by the action of the
gauge group $U(1)$.  This gives the hypersurface $X$ defined by the equation
$F=0$ in ${\bf
CP}^{n-1}$, with K\"ahler class $r$. Thus, classically our theory can
be described as a nonlinear $\sigma$-model whose target space is
this hypersurface $X$, using a metric whose K\"ahler class has volume
proportional to $r^{n-2}$.  As we noted above, such a theory
has central
charge $\widehat{c}=n-2$.

This classical description will be a good approximation to the quantum
theory when $r\gg0$, since the nonlinear
$\sigma$-model is weakly coupled there.
In fact, since the nonlinear $\sigma$-model is stable (in the sense of
having no
relevant perturbations), and marginal perturbations merely vary the
complexified
K\"ahler and complex structures,  we should expect our quantum theory to
coincide with one member of this family of $\sigma$-models.

On the other hand, when $r<0$, the space of classical vacua satisfies
$a_i=0$ and $n|p|^2=-r$.  We can use a gauge transformation to fix
$p=\sqrt{-r/n}$, leaving a residual gauge invariance of ${\bf Z}_n$, i.e.,
the original $U(1)$ gauge group is broken to ${\bf Z}_n$.  This will
therefore be what is known as
an ``orbifold'' theory, in which the ${\bf Z}_n$ shows up in a
global analysis of the theory (in a manner which we shall describe later).
The local description of the theory (for which we can ignore the orbifolding
 issue) is this:
for $r\ll0$, the field $P$ has a large mass and can be integrated out,
leaving an effective
theory of $n$ massless chiral superfields $A_1$, \dots, $A_n$
with an effective interaction
$$W_{\mbox{\scriptsize eff}}=\mbox{constant}\cdot F(A_1,\dots,A_n),$$
where $F$ is a homogenous polynomial of degree $n$.

Such a theory of $n$ massless fields with a polynomial interaction,
 is called a {\it Landau--Ginzburg theory}.  It should apparently
 flow in the IR to a conformal field theory with $\widehat{c}=n-2$.
(We expect this since that is what happened for $r\gg0$, and neither the
conformality nor the central charge of the IR theory should change when we
vary $r$.)

Let's check this prediction, following work of Greene, Vafa, Warner, and
Martinec among others.
Consider a special $F$ of ``Fermat type:''
$$F(A_1,\dots,A_n)=\sum A_i^n.$$
The Landau--Ginzburg theory with this interaction will (at least locally)
factor as a product of $n$ identical theories, each with a simple $A^n$
interaction.  Our prediction implies that each of these theories must have
$\widehat{c}=\frac{n-2}n$.  But that is exactly the central charge of a minimal
model!  So if we can identify the minimal model with this Landau--Ginzburg
theory, we will have established the result.
Note that we would have been in difficulties if we had obtained a
value of $\widehat c$ less than one and not of the form $1-2/k$ for
some integer $k$!
Greene, Vafa, Warner  and
Martinec presented evidence that the Landau--Ginzburg theory with
superpotential $A^n$ does flow in the infrared to an $N=2$ minimal
model.  More arguments have been found subsequently. We will see some of the
evidence for this claim in Lecture II-14.

\subsection{Orbifolding}

As we pointed out above, the original gauge symmetry group of $U(1)$
of our theory was broken
to ${\bf Z}_n$ in the effective theory when $r\ll0$.  Thus,
if we consider the path integral for the low energy effective theory
on a Riemann surface $\Sigma$, the $U(1)$ gauge group is reduced to ${\bf
Z}_n$,
and we need only sum over flat ${\bf Z}_n$ bundles.
  On any surface  $\Sigma$,  a flat ${\bf Z}_n$ bundle
can be specified by its
holonomies $\gamma_1$, \dots, $\gamma_k\in{\bf Z}_n$ around various loops
in $\Sigma$.  The process of summing over all flat ${\bf Z}_n$ bundles in
a theory with a ${\bf Z}_n$ symmetry is known as {\it orbifolding}\/ the
theory.

In a Hamiltonian approach, formulated on a circle $S^1$, we need to
specify the holonomy $\gamma\in {\bf Z}_n$ around the circle as part of the
data
determining a state.  Let ${\cal H}_\gamma$ be the Hilbert space of states
whose holonomy is $\gamma$.  (Then Riemann surfaces with boundary
 on which flat ${\bf Z}_n$ bundles have been
specified will determine operators mapping among these various Hilbert spaces,
depending on the holonomies on the boundary circles.)
${\bf Z}_n$ acts on each ${\cal H}_\gamma$, since as the group ${\bf Z}_n$
is abelian, twisting as one goes around the circle by an element of
${\bf Z}_n$ is an operation that commutes with ${\bf Z}_n$.

We claim that the Hilbert space for the orbifolded theory is
$${\cal H}=\bigoplus_\gamma({\cal H}_\gamma)^{{\bf Z}_n}.$$
To see this, consider a cylinder with holonomy $\gamma$ on the ends:

\centerline{\quad}
\centerline{\epsfxsize=1.35in\epsfbox{horizcyl.eps}}
\centerline{\quad}

There is still as we have noted
an action of ${\bf Z}_n$ on each sector ${\cal H}_\gamma$;
let $\gamma'\in{\bf Z}_n$ determine an operator $\gamma':{\cal
H}_\gamma\to{\cal H}_\gamma$.
The trace of this operator can be evaluated as a partition function
$Z_{\gamma'\gamma}$
on a torus

\centerline{\quad}
\centerline{\epsfxsize=1.25in\epsfbox{cuttorus.eps}}
\centerline{\quad}

\noindent
constructed by using $\gamma'$ to twist the identification of bundles on
the boundary circles of the cylinder, i.e.,
$$Z_{\gamma'\gamma}=\Tr_{{\cal H}_\gamma}\gamma'q^H .$$



If we  sum the operators $\gamma'$, the result is $n\pi$ where
$\pi$ is the projection onto the  ${\bf Z}_n$-invariant subspace:
$$\frac1n\sum_{\gamma'}Z_{\gamma'\gamma}=\Tr_{{\cal H}_\gamma}\pi
q^H=\Tr_{({\cal H}_\gamma)^{{\bf Z}_n}}q^H.$$
Now the partition function for the ${\bf Z}_n$ gauge theory\footnote{
In performing the path integral in a gauge theory, one is supposed
to divide by the volume of the gauge group.  When the gauge group
is a Lie group of positive dimension, the group of gauge transformations
is infinite-dimensional and making sense of ``dividing by the volume
of the gauge group'' requires the Faddeev--Popov construction and
introduction of ghosts.  In the present case, we are considering
a low energy effective theory with  gauge group ${\bf Z}_n$.  The only
gauge transformations on $\Sigma$ are constant transformations by elements
of ${\bf Z}_n$; the volume of the gauge group is therefore $n$, the number
of elements of ${\bf Z}_n$.   The only
gauge connections are the flat connections.  So in the formula below,
the sum over $\gamma$ and $\gamma'$ is the path integral over the ${\bf Z}_n$
connections, and the factor of $1/n$ results from dividing by the
volume of the gauge group.} is obtained by summing further
on $\gamma$:
$$\frac1n\sum_{\gamma,\gamma'}Z_{\gamma'\gamma}=\Tr_{\oplus ({\cal
H}_\gamma)^{{\bf Z}_n}}q^H,$$
from which we conclude that the Hilbert space for our problem is indeed
${\cal H}=\bigoplus_\gamma ({\cal H}_\gamma)^{{\bf Z}_n}$.


\subsection{Interpolation from positive to negative $r$}

Our classical analysis can be summarized as follows: we needed to set
$D=dW=0$ and divide by $U(1)$.
The step of setting $D=0$ and dividing by $U(1)$ is the familiar
mathematical operation of
{\it symplectic reduction}, in which $D=0$ defines a level set for the
moment map of the $U(1)$ action
(with the choice of $r$ specifying the level).  There is another
mathematical interpretation
of this process, as a quotient in the sense of Geometric Invariant Theory
(GIT):  we complexify
the group $U(1)$ to ${\bf C}^*$, and consider the action of ${\bf C}^*$ on
${\bf C}^{n+1}$ with
the same weights as before (the $A_i$'s have weight $1$ and $P$ has weight
$-n$).  There are
two possible quotients (topologically): for $r>0$ the quotient can be
interpreted as the total
space of the bundle ${\cal O}_{{\bf P}^{n-1}}(-n)$ (in which $p$ serves as
a fiber coordinate),
while for $r<0$ the quotient is ${\bf C}^n/{\bf Z}_n$.

In either case we must still impose $dW=0$: in the $r>0$ case,
$W=PF(A_1,\dots,A_n)$ is a
nondegenerate function (in a generalized sense introduced by Bott)
and the the space of
critical points in
${\cal O}_{{\bf P}^{n-1}}(-n)$ is the variety in ${\bf P}^{n-1}$
defined by the equation $F=0$; in the
$r<0$ case we have a homogenous polynomial $F(A_1,\dots,A_n)$ with a highly
degenerate critical point
at the origin of  ${\bf C}^n/{\bf Z}_n$.

The natural parameter in our Lagrangian is $t=-ir+\frac\theta{2\pi}$.
It is the possibility of going to $\theta\not= 0$
that will enable us to interpolate from positive $r$ to
negative $r$ without meeting the singularity that one would find
in classical geometry.

\centerline{\quad}
\centerline{\epsfysize=1.35in\epsfbox{vertcyl1.eps}}
\centerline{\quad}

Classically, there is a singularity at $r=0$ with arbitrary $\theta$, and
interpolation is not
possible.  Quantum mechanically, we claim that the singularity will be
isolated, located at $r=r_0$ (for some $r_0$)
and $\theta=0$.  (Classically, the singularity is at $r=0$, but
quantum mechanically, it is shifted away from $r=0$
by a one-loop correction.)  It is believed that there are
no other singularities; we will not prove this claim
rigorously, but only explain
some of the reasons physicists believe it to be true.  The more
rigorous part of our analysis will be
the demonstration that there {\it is}\/ a singularity at this location; the
less rigorous part will
be the argument that this is the only singularity.

To make this argument, we again work on a circular space $S^1$ of finite
radius.  The compactness means that the story is rather similar to
ordinary quantum mechanics in zero space dimensions, so
let us first recall what happens in ordinary quantum mechanics of a point
particle.  We suppose that the particle is moving on a manifold $X$
with a potential function $V$.  If $X$ is compact, or if $V$ grows
at the ``ends'' of $X$, then one can vary the parameters upon which
$X$ and $V$ depend without meeting a singularity in the ground state
of the quantum mechanics.  However, one will get a singularity if
one varies in the parameters to a point at which $V$ no longer
grows at infinity.  For instance, if $X={\bf R}$, and $V={1\over 2}k
x^2$ (with $x$ a linear function on ${\bf R}$), then the ground state
wave is a smooth function of $k$ as long as $k>0$, but develops a singularity
at $k=0$.  The $k=0$ problem has no normalizable ground state wave
function.

Something like that will happen in our problem on a circle.
Recall that the form of our potential is
$$V=|P|^2|F|^2+|dF|^2+\frac{e^2}2\left(\sum|A_i|^2-n|P|^2-r\right)^2
+|\sigma|^2\left(\sum|A_i|^2+n^2|P|^2\right).$$
For generic $r$, this theory has the property that the potential
grows at infinity.  This ensures that the theory has a discrete
spectrum, and a ground state wave function that varies smoothly
with the parameters.  However, at $r=0$, we see that one can go to infinity
in $\sigma$, at no cost in energy, as long as $A_i$ and $P$ vanish.
This suggests that at $r=0$ there might be a continuous spectrum
and a singularity of the ground state wave function.



To explore this question in more detail, we need to understand  how
the theory behaves in the dangerous region, that is, very large $\sigma$
with other fields small.
When $A$ and $P$ are very close to zero, and $\sigma$ is large, $A$ and $P$
have large masses (on the order of $|\sigma|^2$) and they can be
integrated out, leaving us with a pure gauge theory with
an effective bosonic potential which at the classical level is merely
a constant
\begin{equation}\label{effbospot}
V_{\mbox{\scriptsize eff}}=\frac{e^2}2r^2.
\end{equation}
If this is the correct answer, quantum states decay exponentially
in the large  $\sigma$ region if their energy is less than $e^2r^2/2$.
If so, this region could be  dangerous for a supersymmetric ground
state of zero energy -- and could lead to a singularity in the wave function
of such a state -- only if $r=0$.

To understand the situation better, we must make the analogous
argument quantum mechanically.
For this, we consider the superfield to which $\sigma$ belongs, i.e., we write
$$\Sigma=\sigma+\theta\lambda+\theta\overline{\theta}(F+iD).$$
The action for the supersymmetric gauge theory is
\begin{equation}\label{efflag}
\frac1{2e^2}\int d^4\theta\,\overline{\Sigma}\Sigma +\left(\int
d\overline{\theta}^+\,d\theta^-\, t\Sigma
+ \mbox{c.c.}\right).
\end{equation}
If we perform the
$\theta$-integrals in the effective Lagrangian (\ref{efflag}), we get
$${1\over e^2}\int d^2x\,\left(
\frac{F^2}{2}+|d\sigma|^2+\overline{\lambda}\dirac
\lambda\right)
+\int d^2x\left(\frac{e^2r^2}2\right)+i\theta\int_\Sigma\frac{F}{2\pi},$$
which includes a curvature term (multiplied by a ``$\theta$-angle'') as well
as the bosonic potential previously discussed.

Luckily, we know the $\theta$-dependence of energy in $U(1)$ gauge theories
in two dimensions as computed in lecture II-4:
$$\frac{e^2}2\,\min_{n\in{\bf Z}}\left(n-\frac\theta{2\pi}\right)^2.$$
So as $\sigma\to\infty$, we have $V(\sigma)\sim\frac{e^2}2|\widetilde{t}|^2$
where $\widetilde{t}\equiv -ir+\frac\theta{2\pi}\pmod {\bf Z}$.
In other words, while the classical energy at large $\sigma$ is
$e^2r^2/2$, quantum mechanically $r^2$ is replaced by $|\widetilde t|^2$.
Thus, while classically the singularity is at $r=0$ and $\theta$ is
invisible, quantum mechanically the singularity is at $\widetilde t=0$.
For any other value of $\widetilde t$, there is a positive energy for
$\sigma\to 0$, and this region is not dangerous for supersymmetric
ground states.

Their is actually one more quantum effect of relevance: a one loop
correction gives a finite renormalization of $r$, and shifts the
position of the singularity from $\widetilde t=0$ to $\widetilde t=-ir_0$
for  a certain constant $r_0$.  This correction is important for
comparing to certain predictions of classical geometry, but not
important for what we will say today.

\section{Another example: flops}

As a second example, we consider a model with gauge group $G=U(1)$, chiral
fields
$A_1$ and $A_2$ of charge 1,  chiral fields $B_1$ and $B_2$ of charge $-1$,
with
{\it no}\/ superpotential, and no ``$P$.''  The Lagrangian takes the form
$$\int d^2x\,d^4\theta\left(\frac1{2e^2}\overline{\Sigma}\Sigma
+\sum(|A_i|^2+|B_i|^2)\right)
+\left(t\int d\overline{\theta}^+\,d\theta^-\,\Sigma + \mbox{c.c.}\right).$$
The $D$ term is
$$D=\sum(|A_i|^2-|B_i|^2)-r,$$
and the bosonic potential also contains a term
$$|\sigma|^2\sum(|A_i|^2+|B_i|^2).$$

As before, we must set $D=0$ and divide by $U(1)$.
When $r>0$ this yields the total space of the bundle ${\cal
O}(-1)\oplus{\cal O}(-1)$
over ${\bf P}^1_A$, whereas
when $r<0$ this yields the total space of the bundle ${\cal
O}(-1)\oplus{\cal O}(-1)$
over ${\bf P}^1_B$.
Here ${\bf P}^1_A$ is a copy of ${\bf P}^1$ obtained by requiring
$A_1, A_2$ to be not both zero, and dividing the pair $(A_1,A_2)$ by
${\bf C}^*$.  Likewise ${\bf P}^1_B$ is a copy of ${\bf P}^1$ obtained
by projectivizing $(B_1,B_2)$.

The transformation from the bundle over ${\bf P}^1_A$ to the bundle
over ${\bf P}^1_B$  is a simple model of a birational transformation
known as a
``flop.''  As in the previous example, the singularity only occurs at
$r=\theta=0$ and
one can interpolate between these two models.  This contrasts with
classical geometry, where one passes through a singularity in going
from one model of the quotient ${\bf C}^4/{\bf C}^*$ to the other.

\centerline{\quad}
\centerline{\epsfysize=1.35in\epsfbox{vertcyl2.eps}}
\centerline{\quad}

More generally, consider two Calabi--Yau manifolds $X$ and $X'$ which are
birationally
equivalent, in such a way that their ample cones meet.

\centerline{\quad}
\centerline{\epsfxsize=1in\epsfbox{cones.eps}}
\centerline{\quad}

\noindent
For simplicity assume $h^{2,0}=0$, and note that birational transformations
won't affect
the value of $h^{p,0}$.
Classically, to pass from $X$ to $X'$, one goes through a singularity
on the wall of the Kahler cone.  The singularity is rather similar to
the singularity just found in our ``flop'' example at $r=0$.
In quantum field theory, one can
go around the singularity by taking $\theta\not= 0$; thus
 one can smoothly continue from $X$ to $X'$ by varying the parameters
 of the conformal field theory.

 For example, we might construct such a pair of
Calabi--Yau
manifolds with $h^{1,1}=2$ by starting with a gauge group $G=U(1)\times
U(1)$, and
a superpotential of the form $W=PF(A_i)$.  If one sets things up
correctly, one of the $U(1)$'s puts us in the world of projective varieties
$X$ and $X'$, and by varying the moment map for the other $U(1)$,
we can make a ``flop'' between $X$ and $X'$ that is quite like
our above discussion with the       four chiral superfields
    fields $A_i$ and $B_j$.


\section{Cases in which $c_1\ne0$}

We now wish to consider linear sigma models in which the $R$-symmetry is
anomolous, so that we expect a nontrivial renormalization group flow.

\subsection{Negative $\beta$-function}

  We begin with the case of negative $\beta$-function, which for
nonlinear sigma models corresponds to target spaces $X$ for which
$c_1(TX)>0$.   We can build examples of these exactly as before, with
$U(1)$ acting on $n+1$ fields $A_i$, $P$, but this
time we give $P$ charge $-k$ and take a superpotential $W=P\cdot F$ with
$F$ homogeneous of degree $k$.   The beta function of the sigma model
is negative, zero, or positive for $k<n$, $k=n$, or $k>n$.
For $k\not= n$, even though there is still classically a singularity
at $r=0$, the quantum theory has no singularity on the $t$-cylinder.
(This is shown by analyzing the behavior at large $\sigma$, as we do
presently.)
For $k\not=n$, there is a nontrivial renormalization group flow on the
cylinder.
The flow is holomorphic and singularity-free, so it is a constant flow
``downward'' or ``upward'' on the cylinder, depending on the sign of $n-k$.

\centerline{\quad}
\centerline{\epsfysize=1.35in\epsfbox{vertcyl4.eps}}
\centerline{\quad}

When we repeat the previous analysis, we find that the $R$-symmetry which
comes from the four-dimensional $R$-symmetry is anomalous, although the
one which comes from rotations in the missing directions is anomaly-free.

The classical analysis can also be repeated:  for $r\gg0$ we find a space
of classical vacua desribed by $\{F=0\}$ in ${\bf P}^{n-1}$, and we expect
the nonlinear sigma model on this hypersurface to be a good approximation
to our theory.  This hypersurface has positive first Chern class if $k<n$,
so we
expect a well-defined quantum field theory from looking at the nonlinear
sigma model.  On the other hand, since $k<n$, the sigma model
is infrared-unstable.  The infrared flow will take us away from the
sigma model in the infrared  and toward the
 theory we will find
at the $r\to-\infty$ limit.  Note that the sigma model has an $n-2$
dimensional target space, so for $k<n$ the effective central charge
in the ultraviolet limit is $\widehat c=n-2$.

On the other hand, if $r\ll0$ then on the space of classical vacua $P\ne0$
but $A_i=\sigma=0$.  Giving a nonzero expectation value to $P$ breaks
$U(1)$ to ${\bf Z}_k$, and (in the $r\to-\infty$ limit) we get a
Landau-Ginzburg orbifold with an effective interaction
$$W_{\rm eff}(A_i)=F(A_i).$$
The central charge of this model is
$$\widehat{c}=n(1-\frac2k).$$
This is less than $n-2$ if $k<n$.
We will argue shortly that, for $k<n$, the Landau-Ginzburg theory
is the infrared limit of the renormalization group flow from the sigma model;
and the fact that its central charge is smaller than that of the sigma
model illustrates a general theorem by Zamolodchikov, which asserts
that the central charge always diminishes along a renormalization group flow:
This infrared theory is behaving as if it had fewer degrees of freedom
$$c_{\rm IR}\le c_{\rm UV}.$$
The intuition behind this theorem is that the central charge measures
the total number of degrees of freedom of the theory.  As one flows
toward the infrared, massive degrees of freedom are ``integrated out,'' and
no longer contribute to the central charge, which therefore can only
diminish.


As we will see, the Landau-Ginzburg model is only one possibility for
what the infrared flow can lead to for $k<n$.
 There are in fact $n-k$ other vacuum states at $r=-\infty$, with a
mass gap, which can't be seen classically but will require a $1$-loop
calculation.  Being massive, they have $\widehat c =0$, which is certainly
less than $n-2$.

\subsection{The ${\bf CP}^{n-1}$ model}

An extreme case of the situation we are considering is the case $k=0$,
i.e., the case of no $P$ field and no superpotential.  This is a $U(1)$
gauge theory with $n$ free charged superfields $A_1$, \dots, $A_n$ (all of
charge $1$).  The $D$-term takes the form
$$D=\sum|A_i|^2-r,$$
and we find the supersymmetric ${\bf CP}^{n-1}$ model which we studied
earlier.  It will be very instructive to re-examine this case in detail
before going back to the general case.

At first glance, there would appear to be no supersymmetric
vacua at all in the
$r\to-\infty$ limit of this ${\bf CP}^{n-1}$ model, since the moment
map is strictly positive.  However, it cannot
be so that there are no supersymmetric vacua at all in this limit.
An obstruction is provided by the supersymmetric index $\Tr\,(-1)^F$.
This, when computed for large positive $r$, is seen to coincide with the
Euler characteristic of ${\bf CP}^{n-1}$, which is $n$.  The supersymmetric
index must be invariant under deformations of $r$, so there must
be $n$ supersymmetric vacua even if $r$ is large and negative.
We must find them somewhere!

To see what is happening, we must consider the quantum mechanics
more carefully.  The classical treatment of the  $r\to -\infty$ limit
is valid in any compact region of field space, so if there are quantum
mechanical vacua for arbitrarily negative $r$ that cannot be seen
classically, they must disappear from the classical field of view by
going off to infinity in field space for $r\to-\infty$.
So again we must consider the
quantum-mechanical behavior at large $\sigma$.  For our ${\rm CP}^{n-1}$
model, the bosonic potential takes the form
$$V=\frac{e^2}2\,D^2+|\sigma|^2\sum|A_i|^2.$$
The theory is weakly coupled for $r\ll0$.

What happens if $|\sigma|\to\infty$?  As earlier, the fields $A_i$ will then
be massive and can be integrated out to yield an effective theory which is
an abelian gauge theory with Lagrangian
$${\cal L}_{\rm eff}=
\int d^2x\,d^4\theta\,\frac1{2e^2}\,\overline{\Sigma}\Sigma
+\left(\int d^2x\,d\overline{\theta}^+\,d\theta^-\,\widetilde{W}_{\rm
eff}(\Sigma)
+ c.c. \right).$$
The effective twisted superpotential depends on the renormalization
mass scale $\mu$, and we claim that it takes the unusual ($R$-symmetry
violating) form
\begin{equation}\label{effsup}
\widetilde{W}_{\rm eff}(\Sigma)=t\Sigma +
\frac{in}{2\pi}\left(\Sigma-\Sigma\ln(\Sigma/\mu)\right)
\end{equation}
which we shall verify by making a $1$-loop calculation below.  Notice that
a change
in the choice of logarithm in eq.~(\ref{effsup}) can be compensated for by
changing the value of
$t$ by an integer; such a change has no effect on the physics.
Of course, we must keep $\Sigma$ away from zero in order to define this
term, but that is consistent with our assumption that $\sigma$ is large.



With this new term in the twisted superpotential, the equation for a
critical point of $\widetilde{W}_{\rm eff}$ is
$$0=t-\frac{in}{2\pi}\ln(\Sigma/\mu),$$
or in components,
$$0=-ir   + \frac\theta{2\pi}-\frac{in}{2\pi}\ln(\sigma/\mu)
\pmod1.$$
This has $n$ solutions
\begin{equation}\label{sigmavacua}
\sigma =\mu\exp\left(-\frac{2\pi r}n-\frac{i\theta}n+\frac{2\pi
ik}n\right),
\end{equation}
$k=0,\dots,n-1$,
corresponding to $n$ different vacua in the $r\to-\infty$ limit.  $n$ is
of course, the expected number, the Euler characteristic of ${\bf CP}^{n-1}$.
Notice that $|\sigma|$ is indeed large when $r\to-\infty$, so the
assumptions made in deriving these vacua hold.
The fact that the $n$ vacua go off to infinity for $r\to-\infty$,
along with the fact that their existence depends on the one-loop quantum
correction, explains why one cannot see them classically.

Another very important consequence of the logarithm is that it means
that for $\sigma\to\infty$, the energy in the quantum theory grows
like $|\ln\sigma|^2$.  Hence, in particular, low energy states are limited
to a bounded region of field space, and there is no singularity in the
vacuum behavior for any value of $t$ on the cylinder.  As we will
see, the models with $n\not= k$ all have such a logarithm, and hence
they are all nonsingular throughout the $ t$ cylinder.

\subsection{The $1$-loop calculation}

To understand the origin of the term
$\frac{in}{2\pi}\left(\Sigma-\Sigma\ln(\Sigma/\mu)\right)$ in the
effective twisted superpotential, note that its presence is equivalent to
the effective Lagrangian having a term
$$\int d^2x\, r_{\rm eff}\, D,$$
where
$$r_{\rm eff}=r+\frac n{2\pi}\ln(|\sigma|/\mu).$$
In this theory, we have $D=\sum\overline{A}_iA_i-r$.  Classically, $A_i=0
=\overline A_iA_i$
for large $\sigma$, but quantum mechanically $\overline A_iA_i$ has
an expectation value at large $\sigma$; this expectation value can
be interpreted in an effective classical description as a shift
in $-r$.  So we need to
calculate the expectation value of $\overline{A}A$ by means of a
$1$-loop diagram

\centerline{\quad}
\centerline{\epsfxsize=.4in\epsfbox{oneloop.eps}}
\centerline{\quad}

\noindent
which contributes
$$\int\frac{d^2k}{(2\pi)^2}\,\frac1{k^2+m_A^2}
=\int\frac{d^2k}{(2\pi)^2}\,\frac1{k^2+|\sigma|^2}$$
where we got the mass $m_A$ from the term $|\sigma|^2|A|^2$ in the
potential $V$.  Cutting off the divergent integral, we find
$$\int^\Lambda\frac{d^2k}{(2\pi)^2}\,\frac1{k^2+|\sigma|^2}
=-\frac{1}{2\pi}\ln\frac{|\sigma|}{\mu}.$$
This is the shift in $-r$, and accounts for the claimed formula for
$r_{eff}$.
There are, by the way, no higher loop corrections to this formula.
This can be proved using holomorphy, or by noting that, in this
superrenormalizable theory, higher loop diagrams would involve
negative powers of $\sigma$ and would vanish for large $\sigma$.

Let us now repeat this for a general case with chiral superfields $B_i$ of
charges $q_i$.   (The $B_i$ are to include all of the chiral superfields,
including the $P$ field of the sigma model discussion, if it is present.)
The $D$-term is $\sum q_i|B_i|^2$ and for its expectation value at large
$\sigma$ we get
the $1$-loop correction
$$\sum q_i\int^\Lambda\frac{d^2k}{(2\pi)^2}\,\frac1{k^2+q_i^2|\sigma|^2}
=-\frac{1}{2\pi}\sum q_i \ln |q_i|
-\frac{1}{2\pi}(\sum q_i)\ln\frac{|\sigma|}{\mu}, $$
leading to an effective twisted superpotential
$$\widetilde{W}_{\em eff}(\Sigma)=
\left(t-\frac i{2\pi}\sum q_i\ln |q_i|\right)\Sigma
+\frac{i\sum q_i}{2\pi}(\Sigma-\Sigma\ln(\Sigma/\mu)).$$
In the Calabi-Yau case, $\sum_i q_i=0$ and the coefficient of the logarithm
vanishes.
Hence $\widetilde{W}_{\em eff}$ is linear in $\Sigma$, as a result
of which its contribution to the potential energy is independent of $\sigma$.
As a result of this, it is possible for $\sigma$ to go to infinity at
no cost in energy if $t$ has the correct value. At this value of
$t$, one gets a singularity.  The correct value
is the value at which the coefficient of $\Sigma$ in the twisted chiral
superpotential vanishes; the condition is not ${\rm Im}\,t=0$, as one
would expect classically, nor even $t=0$, but rather
$t-(i/2\pi)\sum_iq_i\ln|q_i| = 0$.
This is the shift in the position of the singularity that was mentioned
in our discussion of the Calabi-Yau case.  In order to compare the
present discussion to, for instance, the counting of rational curves on
the Calabi-Yau manifold, it is important to know about this shift.
When $\sum_i q_i\not=0$, the logarithm prevents $\sigma$ from going to
infinity, and hence prevents a singularity on the $t$ cylinder, as we have
already asserted.

  On the other hand, if $\sum q_i\ne0$, then there are
$|\sum q_i|$ extra vacua (as in the ${\bf CP}^n$ model) with $\sigma$-values
\begin{equation}\label{sigmavacuabis}
\sigma =\mu\exp\left(-\frac{2\pi r}{\sum q_i}-\frac{i\theta}{\sum
q_i}-\frac{\sum q_i\ln q_i}{\sum q_i} +\frac{2\pi
ik}{\sum q_i}\right),
\end{equation}
However, the prediction of these vacua
 is only valid if they occur  at large $\sigma$.
Hence, these vacua are only trustworthy  for $r\to-\infty$ if
$\sum_iq_i>0$, or for $r\to +\infty$ if $\sum_iq_i<0$.

\subsection{Hypersurfaces}

Let us now return to the situation of a hypersurface defined by a
homogeneous polynomial $F$ of degree $k$ in $n$ variables.
Thus $\sum_iq_i=n-k$.  Hence, vacua at large $\sigma$ are present
for $r\to-\infty$ if $n>k$, or at $r\to +\infty$ if $n<k$.
The vacua at large $\sigma$ are the ones that one cannot see classically.
In addition to those, one has vacua that one can see classically
-- a sigma model for $r>>0$ and a Landau-Ginzburg orbifold for $r<<0$.

\centerline{\quad}
\centerline{\epsfysize=1.35in\epsfbox{vertcyl3.eps}}
\centerline{\quad}

Let us now consider the two cases in detail.  For $k<n$, the sigma
model has for its target space a Fano variety, a hypersurface in
${\bf CP}^{n-1}$ of degree $k<n$.   This is a good description in the
ultraviolet, and gives a well-posed quantum field theory whose
infrared behavior we would like to determine.
The infrared flow brings us to $r\ll0$, where we find two types of
vacua:
(1) vacua which are conformal field theories of central charge
$\widehat{c}=n(1-\frac2k)$, and
(2) $n-k$ new vacua at large $\sigma$.  The corresponding effective twisted
superpotential for large $\sigma$ is
$$\widetilde{W}_{\rm eff}=t\Sigma + \frac{i}{2\pi}(n-k)(\Sigma -
\Sigma\ln\Sigma).$$
Both of these types of vacua
 have central charge less than the value $\widehat c=n-2$ of
the UV theory.
Thus, in this example, a single quantum field theory in the ultraviolet
can flow to quite different possibilities in the infrared.
Renormalization group flow for $N=2$ sigma models in two dimensions
has many invariants (such as $\Tr\,(-1)^F$, the elliptic genus, and
some of their cousins).  To reproduce the invariants of the UV sigma
model with Fano target, one must sum over the infrared vacua of types (1)
and (2).

\subsection{The case $c_1<0$}

Consider now the opposite case $k>n$.  A hypersurface of degree $k>n$
in ${\bf CP}^{n-1}$ is an algebraic variety of general type.
 In this case $\beta>0$, and there is
not a well-defined quantum field theory of maps from spacetime to this
hypersurface.  However, the sigma model with this hypersurface as target
makes sense as an effective infrared theory.  One can ask what well-defined
ultraviolet theory can flow to this effective theory in the infrared.

For this question, we will now give an answer.
For $k>n$, the ultraviolet
stable end of the cylinder is the Landau-Ginzburg end.  The Landau-Ginzburg
model has central charge $\widehat c= n(1-2/k)$, which
for $k>n$ is greater than the value $\widehat c = n-2$ of the sigma model.
Thus the Zamolodchikov theorem permits a renormalization group flow
from the LG model to the sigma model.  That is precisely what happens,
since for $n<k$ the renormalization group flow on the cylinder
is ``upwards,'' away from the LG model and towards the sigma model.

If $k>n$, there are no large $\sigma$ vacua at the LG end of the cylinder.
Instead, $k-n$ such vacua accompany the sigma model.  Thus,
a well-defined ultraviolet fixed point, given by the Landau-Ginzburg model,
can flow in the infrared either to the sigma model of the hypersurface,
or to a nonclassical massive vacuum at large $\sigma$.  Just as for $k<n$,
to equate renormalization group invariants such as $\Tr\,(-1)^F$
between the ultraviolet and infrared theories, one must sum over both
types of vacua in the infrared limit.

\centerline{\quad}
\centerline{\epsfysize=1.35in\epsfbox{vertcyl5.eps}}
\centerline{\quad}

Note that, regardless of the sign of $n-k$, the nonclassical vacua
always appear in the infrared, not the ultraviolet.  There is a good
intuitive reason for this.  In the ultraviolet,
where we are defining our theory, we can select what theory we wish
to study (from all of the possible UV fixed points), so the UV theory
has a vacuum of just one type.  Going to the infrared means solving
the equations, and at this stage things are out of our hands:
a definite quantum field theory may have more than one solution for
its infrared behavior.  That is what we have found today.  In fact,
we have seen that a definite UV field theory (based on a Fano variety
if $k<n$ or a Landau-Ginzburg model if $k>n$) can flow to quite different
possibilities in the infrared.

Both of the $k\not = n$ cases that we have studied today
are of considerable methodological interest.
For $k<n$, we have seen how the quantum theory with a Fano target
gives, in general, an answer which is a sort of mixture of the behavior
of a Calabi-Yau manifold and the behavior of complex projective space.
For $k>n$, we have found, in a concrete situation, a well-defined UV
fixed point with flow to the effective theory based on an algebraic
variety of general type.

\end{document}
