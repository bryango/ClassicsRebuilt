\input amstex

\documentstyle{amsppt}
\magnification=1200
\pagewidth{6.5 true in}
\pageheight{8.9 true in}
\loadeufm
\loadeusm

\catcode`\@=11
\def\logo@{}
\catcode`\@=13

\NoRunningHeads

\font\boldtitlefont=cmb10 scaled\magstep2

\def\dspace{\lineskip=2pt\baselineskip=18pt\lineskiplimit=0pt}

\def\uphook{\sssize\cup\kern-4.35pt\raise4.50pt%
     \hbox{$\big\uparrow$}}
\def\oplusop{\operatornamewithlimits{\oplus}\limits}
\def\otimesop{\operatornamewithlimits{\otimes}\limits}
\def\w{{\mathchoice{\,{\scriptstyle\wedge}\,}
  {{\scriptstyle\wedge}}
  {{\scriptscriptstyle\wedge}}{{\scriptscriptstyle\wedge}}}}
\def\Le{{\mathchoice{\,{\scriptstyle\le}\,}
  {\,{\scriptstyle\le}\,}
  {\,{\scriptscriptstyle\le}\,}{\,{\scriptscriptstyle\le}\,}}}
\def\Ge{{\mathchoice{\,{\scriptstyle\ge}\,}
  {\,{\scriptstyle\ge}\,}
  {\,{\scriptscriptstyle\ge}\,}{\,{\scriptscriptstyle\ge}\,}}}
\def\vrulesub#1{\hbox{\,\vrule height7pt depth5pt\,}_{#1}}
\def\rightsubsetarrow#1{\subset\kern-6.50pt\lower2.85pt
     \hbox to #1pt{\rightarrowfill}}
\def\mapright#1{\smash{\mathop{\,\longrightarrow\,}\limits^{#1}}}
\def\rmapdown#1{\Big\downarrow\kern-1.0pt\vcenter{
     \hbox{$\scriptstyle#1$}}}
\def\Amod{\text{\rm $A$-mod}}
\def\modA{\text{\rm mod-$A$}}
\def\HOM{\scrH om}
\def\END{\scrE nd}
\def\SMan{S\scrM an}

\def\vphi{{\varphi}}
\def\vphibar{\overline{\vphi}}

\def\zerobar{\overline{0}}
\def\onebar{\overline{1}}
\def\Cbar{\overline{C}}
\def\Lbar{\overline{L}}
\def\Mbar{\overline{M}}
\def\scrMbar{\overline{\scrM}}
\def\Qbar{\overline{Q}}
\def\grGbar{\overline{\grG}}

\def\Ttil{\widetilde{T}}
\def\Ytil{\widetilde{Y}}
\def\ytil{\widetilde{y}}
\def\xitil{\widetilde{\xi}}
\def\scrMtil{\widetilde{\scrM}}

\def\Stab{\text{\rm Stab}} \def\Mat{\text{\rm Mat}}
\def\SO{\text{\rm SO}} \def\Spin{\text{\rm Spin}}
\def\SU{\text{\rm SU}} \def\Lie{\text{\rm Lie}}
\def\SL{\text{\rm SL}} \def\Sym{\text{\rm Sym}}
\def\str{\text{\rm str}} \def\tr{\text{\rm tr}}
\def\Id{\text{\rm Id}} \def\Fr{\text{\rm Fr}}
\def\Hom{\text{\rm Hom}} \def\Mor{\text{\rm Mor}}
\def\Sets{\text{\rm Sets}} \def\GL{\text{\rm GL}}
\def\Vect{\text{\rm Vect}} \def\Ext{\text{\rm Ext}}
\def\Ber{\text{\rm Ber}} \def\Vol{\text{\rm Vol}}
\def\Aut{\text{\rm Aut}} \def\SUSY{\text{\rm SUSY}}
\def\Lie{\text{\rm Lie}}

\def\dbC{{\Bbb C}} 
\def\dbR{{\Bbb R}}
\def\dbZ{{\Bbb Z}} 


\def\scr#1{{\fam\eusmfam\relax#1}}

\def\scrA{{\scr A}}   \def\scrB{{\scr B}}
\def\scrC{{\scr C}}   \def\scrD{{\scr D}}
\def\scrE{{\scr E}}   \def\scrF{{\scr F}}
\def\scrG{{\scr G}}   \def\scrH{{\scr H}}
\def\scrI{{\scr I}}   \def\scrJ{{\scr J}}
\def\scrK{{\scr K}}   \def\scrL{{\scr L}}
\def\scrM{{\scr M}}   \def\scrN{{\scr N}}
\def\scrO{{\scr O}}   \def\scrP{{\scr P}}
\def\scrQ{{\scr Q}}   \def\scrR{{\scr R}}
\def\scrS{{\scr S}}   \def\scrT{{\scr T}}
\def\scrU{{\scr U}}   \def\scrV{{\scr V}}
\def\scrW{{\scr W}}   \def\scrX{{\scr X}}
\def\scrY{{\scr Y}}   \def\scrZ{{\scr Z}}

\def\gr#1{{\fam\eufmfam\relax#1}}

%Euler Fraktur letters (German)
\def\gra{{\frak a}}
\def\grG{{\frak G}}
\def\grg{{\frak g}}
\def\grp{{\frak p}}


\topmatter
\title\nofrills
{\boldtitlefont Lectures on supersymmetry}
\endtitle
\author
{\rm Joseph Bernstein}
\endauthor
\endtopmatter



\NoBlackBoxes
\document
\dspace

\bigskip\bigskip
\head
{\bf Lecture 2}
\endhead

\bigskip
\subhead
1. Super liner algebra
\endsubhead

\subhead
1.1
\endsubhead
A super vector space (over a field) is a $\dbZ_2$-graded vector space,
standard notation: $V=V_{\zerobar}\oplus V_{\onebar}$.
We let $p$ denote the function of parity defined on homogeneous elements.
Super vector spaces form an abelian category endowed with a functor
of change of parity that we denote as $\Pi$.
There is also a tensor structure on the category of super
vector spaces:
$$
\left(V_{\zerobar}\oplus V_{\onebar}\right)\otimes
\left(W_{\zerobar}\oplus W_{\onebar}\right):=
\underbrace{V_{\zerobar}\otimes W_{\zerobar}\oplus
     V_{\onebar}\otimes
W_{\onebar}}_{\thickfrac\thickness0{\big\uparrow}
{\dsize\text{even component}}}\,\, \oplus\,\,
\underbrace{V_{\zerobar}\otimes W_{\onebar}\oplus
V_{\onebar}\otimes 
     W_{\zerobar}}_{\thickfrac\thickness0{\big\uparrow}
  {\dsize\text{odd component}}}
\tag{$*$}
$$
The braiding operation $S:V\otimes W\to W\otimes V$ is defined by the rule:
$$
S_{V\otimes W}(X\otimes Y)=(-1)^{p(X)p(Y)}Y\otimes
X.\tag{$**$}
$$
If $V=V_{\zerobar}\oplus V_{\onebar}$ is a super vector
space, we set
$(m,n):=(\dim\,V_{\zerobar},\dim\,V_{\onebar})$ to be its
dimension.
All types of algebras in the super world are defined as
algebras in the tensor category of super vector spaces.
We will give below a more down-to-Earth definition of basic
algebraic structures.

\subhead
1.2
\endsubhead

Associative super algebra is a $\dbZ_2$-graded algebra (of
course, the product operation must respect the
$\dbZ_2$-grading).
Left and right modules over an associative super algebra 
are defined in an obvious manner, and the corresponding
categories will be denoted by \break 
$A-\mod$ and $\mod-A$, respectively.
On each of these categories we have the parity change functor
$\Pi$.

An associative algebra is said to be commutative if
$\forall\,a,b$ $a\cdot b=(-1)^{p(a)p(b)}b\cdot a$.

\proclaim{Lemma-Construction}
Let $A$ be a commutative super algebra.
Then every left $A$-module $M$ acquires in a functorial way
a right $A$-action.
Moreover, the right and the left $A$-actions on $M$ commute.
\endproclaim

\demo{Proof}
For $a\in A$, $m\in M$ we let $m\cdot a:=(-1)^{p(a)p(m)}am$.
This is a right action and we have:
$$
(b\cdot m)a=(-1)^{p(a)(p(b)+p(m))}(a\cdot b)(m)=
(-1)^{p(a)p(m)}b\cdot(am)=b\cdot(m\cdot a)\qquad\blacksquare
$$
\enddemo

\proclaim{Corollary}
If $M,N\in \Amod$, we can define new left modules 
$\HOM_{\modA}(M,N)$; $M\otimesop_{A}N$.
\endproclaim
Therefore, super commutative algebras play the same role in
the super world as usual commuttive algebras do in the usual
mathematics.

Finally, a Lie super algebra is a vector space
$L=L_{\zerobar}\oplus L_{\onebar}$ with a bracket operation
$[\quad,\quad]:\,\,L\otimes L\to L$ with:
$$
\align
&[X,Y]=(-1)^{p(X)p(Y)}[Y,X]\\
[X,&[Y,Z]]+(-1)^{p(Z)(p(X)+p(Y))}[Z,[X,Y]]+
(-1)^{p(X)(p(Y)+p(Z))}[Y,[Z,X]]=0
\endalign
$$
Note that, if $X\in L_{\onebar}$, $[X,X]$ does not
necessarily vanish. 
We have a functor:
$$
\text{Associative algebras} \longrightarrow \text{Lie
algebras.}
$$
This functor preserves the underlying vector space and
$$
[X,Y]=XY+(-1)^{p(X)p(Y)}YX.
$$
In particular, if $X\in A_{\onebar}$, $[X,X]=2X^2$

\subhead
1.3
\endsubhead

Let $V$ be a super vector space, or more generally a module
over a  commutative super aglebra $A$.

We have a natural action of the symmetric group $S^n$ on
$$
\underbrace{V\otimesop_{A}V\otimes\cdots
 \otimesop_{A}V}_{\text{$n$ times}}
\qquad\text{(Remember, $\sigma(X\otimes Y):=\,(-1)^{p(X)p(Y)}
Y\otimes X$)}
$$

$\Sym^n(V)$ is defined to be $(V^{\otimes n})^{S^n}$.

The space $\Sym(V) = \oplus \Sym^n(V)$ has a natural
structure of graded commutative superalgebra.

If $V=V_{\zerobar}$, $\Sym^n(V)$ in the super sense
identifies with the usual $\Sym^n(V)$, whereas if
$V=V_{\onebar}$, $\Sym^n(V)$ in the super sense coincides
with $\Lambda^n(V)$.

We define the  exterior algebra $\Lambda(V)$ as
$\Sym(\Pi(V))$. This again is a commutative superalgebra.


\subhead
1.4
\endsubhead

Let $A$ be a commutative super algebra and let $M$ and $N$
be two free $A$-modules, i.e.
$$
\align
M &=\underbrace{A\oplus A\oplus\cdots\oplus A\oplus A}_{p}\,\,
\oplus\,\,\underbrace{\Pi A\oplus \Pi A\oplus\cdots\oplus \Pi A}_{q}\\
N &=\underbrace{A\oplus A\oplus\cdots\oplus A\oplus
A}_{p'}\,\,\oplus\,\,\underbrace{\Pi A\oplus \Pi A\oplus\cdots
   \oplus \Pi A}_{q'}
\endalign
$$
and let $(e_1,\dotsc,e_p,e_{p+1},\dotsc,e_{p+q})$,
$(f_1,\dotsc,f_{p'},f_{p'+1},\dotsc,f_{p'+q'})$ denote their
bases.

Then every element in $\HOM(M,N)$ (cf. Corollary 1.2) can be
presented in a matrix form
$$
T=\quad
\bordermatrix{\offinterlineskip
     &\overbrace{\qquad}^{p} &\overbrace{\qquad}^{q}\cr
p'\Bigl\{ &A &B\cr
\noalign{\smallskip}
q'\Bigl\{ &C &D\cr}
$$

$T$ is even if all the entries of $A$, $D$ are even and those
of $B$ and $C$ are odd.

$T$ is odd if the entries of $A$, $D$ are odd and those of
$B$ and $C$ are even.

When $M=N$, $\END(M)$ will be identified with the matrix
algebra which we denote as $Mat_{p,q}(A)$

\proclaim{Lemma}
Consider an even morphism  $str: \Mat_{p,q}(A) \to {\str} A$ 
given by the formula
$\str\left(\lower7pt\hbox{\vbox{\offinterlineskip
\halign{\hfill #\hfill &\vrule# &\kern3pt\hfill #\hfill\cr
$A$ &&$B$\cr \phantom{$a$} &&\phantom{$a$}\cr
\noalign{\hrule}
\phantom{$a$} &&\phantom{$a$}\cr
$C$ &&$D$\cr}}}\right) =\tr\,A- (-1)^{p(D)}\tr\,D$.
Then $\str([X,Y])=0$.

$\blacksquare$
\endproclaim

The function $(\str)$ can also be defined  using the tensor
structure mentioned in 1.1.

\bigskip
\subhead
2. Super calculus
\endsubhead

\subhead
2.1. Super domains
\endsubhead

A super domain of dimension $(p,q)$ is an open subset
$U\subseteq \dbR^p$ with a super algebra of functions 
$C^\infty(U)\otimes\Lambda^{\cdot}(\xi_1,\dotsc,\xi_q)$.
For each $x\in U$ we have the evaluation map $ev_x\colon\,
C^\infty(U)\otimes\Lambda^{\cdot}(\xi_1,\dotsc,\xi_q)\to\dbR$
defined as 
$$
\cases
ev_x(f\otimes 1)=f(X) &\\
ev_x(1\otimes\xi)=0 &.
\endcases
$$
The triple $(U,C^\infty(U)\otimes
\Lambda^{\cdot}(\xi_1,\dotsc,\xi_q),ev_x)$ will be denoted by
calligraphic $\scrU$ and the algebra
$C^\infty(U)\otimes\Lambda^{\cdot}(\xi_1,\dotsc,\xi_q)$ will be
denoted by $C^\infty(\scrU)$.

If $\scrU$ and $\scrV$ are two super domains, a morphism
$\vphi\colon\,\scrU\to\scrV$ is a pair $(\vphi_0,\vphi^*)$,
where $\vphi_0\colon\,U\to V$ is a map of the underlying
topological spaces and $\vphi^*\colon\,C^\infty(\scrV)\to
C^\infty(\scrU)$ is a super algebra map, s.t. $\forall\,x\in
U$, $\forall\,f\in C^\infty(\scrV)$,
$ev_x(\vphi^*(f))=ev_{\vphi_0(x)}(f)$.

\remark{\bf Remarks}
Note that neither a function $\in C^\infty(\scrU)$ nor a
morphism $\scrU\to\scrV$ are determined by their values
at $x\in U$.
\endremark

\example{Example}

Let $\scrU=(\dbR^1, C^\infty(\dbR^1)\otimes\Lambda
(\xi_1,\xi_2))$.
Consider a morphism $\vphi\colon\,\scrU\to\scrU$ given by
$\vphi_0=\Id$; $\vphi^*(\xi_1)=\xi_1$,
$\vphi^*(\xi_2)=\xi_2$; $\vphi^*(x)=x+\xi_1\xi_2$.
This is an example of an automorphism which is trivial on
the underlying topological space.
\endexample

\proclaim{Lemma}

\roster
\item"{\rm (a)}"
A map of super domains $\vphi\colon\scrU\to\scrV$ is
completely determined by the image of the coordinate
functions
$$
y_1,\dotsc,y_{p'},\xi_1,\dotsc,\xi_{q'}\in C^\infty
(\scrV).
$$

\item"{\rm (b)}"
For any $\ytil_1,\dotsc,\ytil_{p'},\xitil_1,\dotsc,
\xitil_{q'}\in C^\infty(\scrU)$ $\exists !$ map
$\vphi\colon\,\scrU\to\scrV$ with $\vphi^*(y_i)=\ytil_i$:
$\vphi^*(\xi_j)=\xitil_j$ provided that $\forall\,x\in U$,
$(\ytil_1(x),\ytil_2(x),\dotsc,\ytil_{p'}(x))\in V$.
\endroster
\endproclaim

\subhead
2.2. Supermanifolds
\endsubhead

A supermanifold $\scrM$ is a topological space $M$ endowed
with a sheaf of super algebras $C^\infty(\scrM)$ 
which is locally  isomorphic to a super domain.

Morhisms between two supermanifolds $\scrM$ and $\scrN$ are
defined as in the case of super domains.

Analogous definitions can be also given in the context of real
analytic, complex, or algebraic manifolds.

Again, we emphasize that neither of the geometric constructions
on supermanifolds that will be described is determined by its
action on the points of $M$.
Nevertheless, the following construction is often convenient:

\proclaim{Lemma-Construction}
For every supermanifold $\scrM$ of dimension $(p,q)$ there
exists a canonical sub-supermanifold $\scrMbar$ of dimension
$(p,0)$ (i.e. a usual manifold) with a property that 
every map $\scrN @>{\vphi}>>\scrM$ with $\scrN$ being
completely even must factor as
$$
\matrix
 &\raise4pt\hbox{$\sssize\vphi$}\kern-5pt\nearrow 
     &\raise8pt\hbox{$\scrM$}\\
\scrN &&\uphook\\
&\searrow\kern-6pt\raise4pt\hbox{$\sssize\vphibar$}
&\lower8pt\hbox{$\scrMbar$}\\
\endmatrix
$$
\endproclaim

\demo{Proof}
We set $\scrMbar$ to have the same underlying topological
space as $\scrM$ does.
$C^\infty(\scrMbar)$ is the quotient of $C^\infty(\scrM)$ by
the ideal generated by all odd functions.\qquad
$\blacksquare$
\enddemo

$\scrMbar$ is called the underlying manifold of the
supermanifold $\scrM$.

We let $\SMan$ denote the category of supermanifolds. As in the case of
usual manifolds, $\SMan$ admits the direct product construction:
$\scrM,\scrN\to\scrN\times\scrM$.

As in the usual geometry, many important constructions for
supermanifolds are carried out in families, where by a
family $\scrM_S$ of supermanifolds over a base $S$ which is a
supermanifold too, we mean a submersive morphism
$$
\CD
\scrM_S\\
@VVV\\
S
\endCD
\qquad\text{(cf. 2.3)}.
$$
We will often consider sheaves of $C^\infty(\scrM)$-modules
on a supermanifold $\scrM$.
A sheaf is called locally free, if locally on $M$, we have
$\scrF\simeq C^\infty(\scrM)\otimesop_{\dbR}V$, where $V$ is
a super vector space.

If $x$ is a point of a supermanifold $\scrM$ and if $\scrF$
is a sheaf on $\scrM$, the fiber $\scrF_x$ of $\scrF$ at $x$
is defined as a super vector space $\scrF/m_x\scrF$, where
$m_x$ is the maximal ideal in $C^\infty(\scrM)$
corresponding to $x$.

\subhead
2.3. Tangent sheaf
\endsubhead

Consider the sheaf of all (even and odd) derivations of
$C^\infty(\scrM)$.
This is a sheaf of locally free $C^\infty(\scrM)$-modules,
called the tangent sheaf and we denote it by $\scrT(\scrM)$.

The dual sheaf to $\scrT(M)$ denoted  by
$\scrT(\scrM)^*$  is called the cotangent sheaf. 
We have a canonical even morphism $D: \scrO(\scrM) \to 
\scrT(\scrM)$ defined by $D(f)(X) = (-1)^{p(f)p(X)}X(f)$.

   The sheaf  $\Omega^1(\scrM) = \Pi \scrT(\scrM)^*$ is called the
sheaf of 1-forms on $\scrM$. We have canonical odd morphism
$d: \scrO(\scrM) \to \Omega^1(\scrM)$.

\proclaim{Lemma-Definition}
The sheaf $\oplusop_{i\Ge 0}\Lambda^i(\Omega^1(\scrM)
   \overset\text{def}\to\simeq \oplusop_{i\Ge 0}
\Omega^i(\scrM)$ has a structure of a  commutative super
algebra such that:

\roster
\item"{\rm 1)}"
The differential
$d\colon\,\Omega^i(\scrM)\to\Omega^{i+1}(\scrM)$ is odd and
satisfies the Leibnitz rule.

\item"{\rm 2)}"
For $f\in C^\infty(\scrM)\overset\text{def}\to\simeq
\Omega^0(\scrM)$ and $X \in\Gamma(\scrM,\scrT(\scrM))$ we have
$$
df(X)=(-1)^{p(x)}X(f).\qquad\qquad\blacksquare
$$
\endroster
\endproclaim

Note that, unlike the usual situation, the De Rham complex
$\Omega^0(\scrM)\to\Omega^1(\scrM)\to\,\,\ldots$ is infinite
unless $\scrM$ is even.

Let $\vphi\colon\,\scrM\to\scrN$ be a morphism between two
supermanifolds.
If $x\in M$ and $y=\vphi_0(x)\in N$ we have a map of
superspaces called the differential of $\vphi$ at $x$:
$$
T_x\vphi\colon\,\scrT_x(\scrM)\to\scrT_y(\scrN).
$$

As in the classical case, we have the ``implicit function
theorem'':

\proclaim{Theorem}

Let $\vphi\colon\,\scrM\to\scrN$ be a submersion (i.e.
$T_x\vphi$ is onto $\forall\,x$).
Then locally on $\scrM$, it can be identified with a
projection 
$\scrM = \scrN\times\scrN' \to \scrN$
\endproclaim

\subhead
2.4. Distributions
\endsubhead

A distribution on a manifold $\scrM$ is a subsheaf 
$\gra\hookrightarrow\scrT(\scrM)$ which locally on $M$ can be
realized as a direct summand.
Such $\gra$ leads to considering the Frobenius pairing:
$$
\Fr\colon\,\gra\otimes\gra\to\scrT(\scrM)/\gra.
$$
It is easy to see that $\Fr$ is a map of
$C^\infty(\scrM)$-modules.
We have an analog of the Frobenius theorem:

\proclaim{Theorem}
$\Fr\equiv 0$ if and only if the distribution $\gra$ is
integrable, i.e. if locally on $M$ there exists a
submersive map $\pi: \scrM \to \scrN$
 such that $\gra$ identifies with the kernel
$$
0 @>>> \gra @>>> \scrT(\scrM) @>>> \pi^*\scrT(\scrN)
@>>> 0\qquad\qquad\blacksquare
$$
\endproclaim

\subhead
2.5
\endsubhead

A convenient way to think about supermanifolds is to view
them as contravariant functors on the category $\SMan$.
In other words, to a supermanifold $\scrM$ associate a
functor
$$
F_\scrM\colon\,\,\text{\SMan}^0\to\text{Sets}
$$
given by $F_\scrM(S)=\Hom(S,\scrM)$.
By the general category theory, such a functor determines
$\scrM$ completely.
(Sometimes, the appropriate supermanifold does not exist and
such a functor is a good substitute for it.)
Let us consider several examples.

\example{Example 1}
$\dbR^{1,0}$, $\dbR^{0,1}$.

It is easy to see that
$$
\align
F_{\dbR^{1,0}}(S) &\simeq C^\infty(S)_{\zerobar}\\
F_{\dbR^{0,1}}(S) &\simeq C^\infty(S)_{\onebar}.
\endalign
$$
\endexample

\example{Example 2}

Let $V$ be a finite dimensional super vector space. Then one can canonically 
define a supermanifold which corresponds to $V$.
This supermanifold corresponds to the functor
$$
F_V(S)=(C^\infty(S)\otimes V)_{\zerobar}.
$$
More generally, let $E$ be a locally free sheaf of
$C^\infty(\scrM)$-modules on a supermanifold $\scrM$.
We define the total space of the corresponding super vector
bundle $V(E)$ on $\scrM$ by means of the functor
$$
F_{V(E)}(S)=\{\text{pairs } \alpha\colon\,S\to M,\sigma\},
$$
where $\alpha$ is a morphism of supermanifolds and $\sigma$
is an even section of $\alpha^*(E)$ on $S$. (Here $\alpha^*$
denotes the functor of pull-back. )
\endexample

\example{Example 3}

We will consider a particular case of the previous example
for $E=\Pi(\scrT(\scrM))$.
The supermanifold $V(E)$ will be also denoted as
$\scrMtil$.
\endexample

\proclaim{Lemma}
$F_{\scrMtil}(S)\simeq\Hom(S\times\dbR^{0,1},\scrM)
\qquad\qquad\blacksquare$

\noindent
In particular, the points of $\scrMtil$ bijectively
correspond to morphisms $\dbR^{0,1}\to\scrM$.
\endproclaim
Supermanifolds obtained in this way will be sometimes
denoted as $\Mor(\dbR^{0,1},\scrM)$.

\bigskip
\subhead
3. Lie supergroups and Lie superalgebras
\endsubhead

\subhead
3.1. Lie supergroups
\endsubhead

Let $\grG$ be a supermanifold, which we would like to endow with
a structure of a Lie supergroup on $\grG$. It is very convenient to phrase 
the corresponding definition in terms of the corresponding functors.

\definition{Definition}\quad 1)\enspace
We say that $\grG$ is a Lie supergroup if the functor
$F_\grG\colon{\SMan}^0\to\Sets$ possesses a group
structure (i.e. if for every $B\in\SMan$ the set $F_\grG(B)$ is
a group structure and for any morphism
$B_1\to B_2$ of supermanifolds the corresponding morphism
$F_\grG(B_2)\to F_\grg(B_1)$ is a homomorphism of groups.)

2)\enspace
We say that a Lie supergroup $\grG$ acts on a supermanifold
$\scrM$ if the functor $F_\grG$ acts on the functor
$F_\scrM$ (i.e. if for every supermanifold $B$ we are given an
action of the group $F_\grG(B)$ on the set $F_\scrM(B)$ such
that the obvious compatibility condition holds.)
\enddefinition

\remark{\bf Remark}
One can give also the following (equivalent)
definition of a Lie supergroup, or of its action on a
supermanifold:

We can say that a structure of a Lie supergroup
on a supermanifold $\grG$ is a pair of morphisms
$m\colon\,\grG\times\grG\to\grG$ (multiplication) and
$i\colon\,\grG\to\grG$ (taking the inverse) such that the
obvious axioms hold.
However, we will see below that in most examples 
the definition via functors is more convenient.
\endremark

\example{Example}
Let us define the general linear  Lie group $\grG\scrL(p,q)$, which is
a super analog of $\GL(n)$.
It should be thought of as the group of linear automorphisms
of $\dbR^{p,q}$.
We define a functor $F_\grG\colon\,{\SMan}^0\to\Sets$ by
$$
F_\grG(B)=\Aut(C^\infty(B)\otimes\dbR^{p,q})
\qquad\text{(cf. 1.4)}
$$

This functor is endowed with the obvious group structure and it is not 
difficult to see that this functor is represented by a supermanifold. 
\endexample

More generally, let $V$ be any superspace of dimension
$(p,q)$.
Then one can define a Lie supergroup $\grG\scrL(V)$ by
putting
$$
F_\grG(B)=\GL_{C^\infty(B)}(V\otimes C^\infty(B))
$$
(here $\GL_{C^\infty(B)}(V\otimes C^\infty(B))$ means the
group of automorphisms of the module $V\otimes C^\infty(B)$
over the superalgebra $C^\infty(B))$.

\subhead
3.2. Lie superalgebras
\endsubhead

If $\grG$ is a supergroup, it possesses a distinguished point
$e$ called the unit element.
Therefore we can consider the tangent space $\scrT_e\grG$ to
$\grG$ at the point $e$.
It is easy to see that the superspace $\grg=\scrT_e\grG$ 
has a canonical structure of a Lie superalgebra (as
in usual Lie theory, one can identify $\grg$ with the
space of left invariant vector fields on $\grG$ and then
consider their commutator).

Let $G$ denote $\grGbar$.
Then the Lie algebra $\Lie\,G$ of $G$ is canonically
identified with $\grG_{\zerobar}$ and we see that 
the pair $(G,\grg)$ forms a
Harish-Chandra pair.
Let us recall this notion.

\definition{Definition}
A Harish-Chandra pair is a pair $(G,\grg)$ where $G$ is a
Lie group and $\grg$ is a Lie superalgebra together with an
action of $G$ on $\grg$ by automorphisms and a Lie-algebra map
$i\colon\,\Lie\,G\to \grg_0$, such that the standard
compatibility condition saying that the action of $\Lie\,G$ on $\grg$ 
obtained by deriving the $G$-action coincides with the adjoint action of
$\Lie\,G=\grg_0$ on $\grg$ holds.
\enddefinition


It is easy to see that the assignment
$\grG\to(\grGbar,\Lie\,\grG)=(G,\grg)$ is a correctly
defined functor from the category of Lie supergroups to the
category of Harish-Chandra pairs.

\proclaim{Proposition}
This functor is an equivalence between the category of Lie supergroups
and the category of Harish-Chandra pairs for which the map
$$i\colon\,\Lie\,G\to \grg_0$$
is an isomorphism.
\endproclaim

This proposition allows to reduce any question
about Lie supergroups to a question about Harish-Chandra
pairs, which are intuitively easier to handle.
For example, let us explain how to describe an
action of a Lie supergroup on a supermanifold in terms of
Harish-Chandra pairs.

\definition{Definition}\quad 1)\enspace
An action of a Lie superalgebra $\grg$ on a supermanifold
$\scrM$ is a morphism of Lie superalgebras
$\grg\to\Vect\,\scrM$  (here $\Vect\,\scrM$ denotes the
algebra of global vector fields on $\scrM$).

2)\enspace
An action of a Harish-Chandra pair $(G,\grg)$ on $\scrM$ is
an action of the Lie group $G$ on $\scrM$ together with an
action of the Lie superalgebra $\grg$ on $\scrM$, satisfying
the obvious compatibility condition (i.e. that the two
actions of $\Lie\,G$ on $\scrM$ obtained by deriving the
$G$-action on $\scrM$ and by identifying $\Lie\,G$ with
$\grg_0\subset\grg$, coincide).
\enddefinition

It is easy to see that to give an action of a Lie supergroup
$\grG$ on $\scrM$ is the same as to give an action of the
corresponding Harish-Chandra pair.

\example{Example}
Let $\grG=\dbR^{0,1}$.
This is a Lie supergroup (with respect to addition) and as
any Lie supergroup it acts on itself by translations.
More precisely, for any supermanifold $B$ the set 
$\text{Hom}(B,\dbR^{0,1})$ is canonically identified with
the vector space $C^\infty(B)_{\onebar}$,, which endows
$F_\grG$ with a group structure.
\endexample

For any supermanifold $\scrM$ consider
$\scrMtil=\Mor(\dbR^{0,1},\scrM)$.
$\scrM$ is a supermanifold (cf. 2.5) which admits a natural
action of $\dbR^{0,1}$ (since it acts on itself).
The Lie algebra $\Lie\,\dbR^{0,1}$ is generated by the
standard odd vector field $\frac{\partial}{\partial\xi}$.
The action of its image in $\Vect\,\scrMtil$ on
$\Omega^*(\scrM)\subset C^\infty(\scrMtil)$ coincides with the De Rham
differential.

\subhead
3.3. Theory of connections
\endsubhead

We want to define the notion of connection in the
superworld.
Let us first explain what we mean by principal $\grG$-bundle
over $\scrM$, where $\grG$ is a Lie supergroup.
A principal $\grG$ bundle over $\scrM$ is a supermanifold
$P$ together with a free $\grG$-action and a natural
isomorphism $\scrM\simeq P/\grG$ (the notions of quotient
and free actions are again easily formulated in terms of the
corresponding functors).

As in the usual differential geometry, one can define a connection
on $P$ to be a distribution $\gra\subset\scrT(P)$ of codimension
equal to $\dim\,\scrM$ and transversal to the fibers of the
projection $P\to\scrM$ which is, moreover, $\grG$-invariant.

A connection on a vector bundle $\scrE$ on $\scrM$ is by definition a
first-order differential operator 
$$
\nabla\colon\,\scrE\otimes\Omega^*(\scrM)
\to\scrE\otimes\Omega^*(\scrM)
$$
which is odd, satisfies the Leibnitz rule and has degree $1$
with respect to the natural $\dbZ$-grading on
$\Omega^*(\scrM)$.

A connection on a principle $\grG$-bundle $P$ on $\scrM$ can be defined
as a compatible (w.r. to tensor products) family of connections
on every vector bundle associated with a representation
$\rho\colon\,\grg\to\grG\scrL(V)$ of $\grG$.
 

\bigskip
\subhead
4. Integration and determinants
\endsubhead

\subhead
4.0
\endsubhead
So far we have discussed only differential geometry in the
supercase.
Now we want to treat integration theory.
It is easy to see that in supercase one cannot integrate
differential forms over a supermanifold (since there is no
natural notion of highest-degree forms).
Instead, we can integrate sections of some vector bundle
over $\scrM$.
In order to define it we first have to discuss the notion of
superdeterminant (or Berezinian).
So, we are going back to linear algebra.


\subhead
4.1
\endsubhead
Let $C$ be a commutative superalgebra and let $\scrM$ be a
free $C$-module of rank $(p,q)$.
We want to define an analog of the notion of highest
exterior power of $\scrM$.
We have the following definition: 

Let $S(\scrM)$ denote the symmetric algebra of $\scrM$.
\proclaim{Lemma}
$$
\Ext_{S(\scrM)}^i(C,S(\scrM))=
\left\{\aligned
&0\quad \text{\rm if $i\not=p$}\\
&\text{\rm free $C$-module of rank $1$ and parity $q$ if $i=p$}
\endaligned
\right.
$$
\endproclaim

We define $\Ber_C(\scrM):=\Ext_{S(\scrM)}^p(C,S(\scrM))^*$.

\smallskip\noindent
{\bf Properties of $\Ber$}

\roster
\item"1)"
If $\scrM$ is of rank $(p,0)$ then
$$
\Ber\,\scrM=\Lambda_C^p\scrM
$$

\item"2)"
Suppose that $0\to\scrM_1\to\scrM\to\scrM_2\to 0$ is a short
exact sequence of $C$-modules.
Then 
$$
\Ber\,\scrM=\Ber\,\scrM_1\otimes\Ber\,\scrM_2.
$$

\item"3)"
Suppose that $\scrM$ is $(0,q)$-dimensional.
Then
$$
\Ber\,\scrM=S^q(\scrM)^{-1}.
$$
\endroster

Let now $g\in\Aut_C(\scrM)$ --- an automorphism of $\scrM$.
Then one can define $\Ber(g)$ to be the invertible element of
$C_{\zerobar}$ by which $g$ acts on $\Ber(\scrM)$.
We obtain a homomorphism $\Aut_C(\scrM)\to C_{\zerobar}^*$.

Berezin actually wrote a formula for $\Ber(g)$ (this was his
original definition of Berezinian).

Namely, let us choose a basis of $\scrM$ over $C$.
Suppose that in this basis $g$ looks like
$g=\left(\smallmatrix A &B\\ C
&D\endsmallmatrix\right)$.
Then
$$
\Ber\,g=\det(A - B\ D^{-1}C)(\det\,D)^{-1}
$$
Looking at this formula, it is not easy (although possible)
to prove that it is multiplicative.

Let us emphasize that, unlike the usual situation,
Berezinian is defined only on even invertible matrices.

There is yet another definiton of the Berezinian:
by proposition 3.2 there exists a canonical morphism of
Lie supergroups 
$$\Ber\colon\,\grG\scrL(p,q)\to \grG\scrL(1,0)$$
given by the usual determinant 
$GL(n)\to GL(1)$ and by the supertrace
$$\str:\Lie(\grG\scrL(p,q))\simeq\Mat(p,q,\dbR)\to\dbR\simeq 
\Lie(\grG\scrL(1,0))$$ 
and we can define $\Ber$ on even invertible $B$-valued matrices
as the corresponding homomorphism
$$F_{\grG\scrL(p,q)}(B)\to F_{\grG\scrL(1,0)}(B).$$ 

Let us now list some further properties of $\Ber(\scrM)$.

\roster
\item"4)"
$\Ber(\Pi(\scrM))=\Ber(\scrM)^{-1}$

\item"5)"
$\Ber(\scrM^*)=\Ber(\scrM)^{-1}$

\item"6)"
(very important in integration theory).
\endroster

Fix a basis $e_1\ldots e_p$, $e_{p+1}\ldots e_{p+q}$ of
$\scrM$.
Then it defines a canonical element\break
$e_1\Lambda e_2\dotsc\Lambda e_{p+q})\in\Ber(\scrM)$.
Moreover, if $e_1^1\ldots e_{p+q}^1$ is another basis of
$\scrM$ and $g$ is the transition matrix from one basis to
the other, then $\Delta(e_1^1\ldots e_{p+q}^1)=\Ber(g)
\Delta(e_1\ldots e_{p+q})$.
In fact, this property may serve as a definition of
$\Ber(\scrM)$.

Let now $\scrF$ be a locally free sheaf of
$C^\infty(\scrM)$-modules on a supermanifold $\scrM$.
We can associate to it an invertible sheaf $\Ber(\scrF)$ on
$\scrM$.

\subhead
4.2 Integration
\endsubhead

Let $\scrM$ be a supermanifold.
We define an invertible sheaf $\Vol(\scrM)$ on it as
$\Ber(\Pi(\scrT(\scrM)))$.
When $\scrM$ is even, $\scrT(\scrM)$ is even and 
$\Vol(\scrM)\simeq\Omega^{\text{top}}(\scrM)$ coincides with
the sheaf of usual volume forms.
Let us also fix an orientation on the underlying manifold.

\proclaim{Theorem}
We have a unique map
$\Gamma_c(\scrM,\Vol(\scrM))@>{\int}>>
\dbR$ of parity $p$ satisfying the following properties:

\roster
\item"{\rm 1)}"
If $\scrM$ is even $\int$ is the usual
integration of a volume form
$\omega\to\int\limits_{M}\omega$

\item"{\rm 2)}"
If $\scrM\simeq\scrM_1\times\scrM_2$;
$\omega_i\in\Gamma_c(\scrM_i,\Vol(\scrM_i))$, we have
$$
\int\limits_{\scrM}\omega=\int\limits_{\scrM_1}
\omega_1\times\int\limits_{\scrM_2}\omega_2
$$

\item"{\rm 3)}"
If $\scrM\simeq\dbR^{0,1}$ and $\omega_0\in\Vol(\dbR^{0,1})$
is given by
$\xi\cdot d\xi$, $\int\limits_{\dbR^{0,1}}\omega_0=1$.
\endroster
\endproclaim

\proclaim{Lemma}
For every $X \in\Gamma(\scrM,\scrT(\scrM))$;
$\omega\in\Gamma_c(\scrM,\Vol(\scrM))$ we have
$\int\limits_{\scrM}(X \omega)=0$.\qquad$\blacksquare$
\endproclaim

As in the classical case, everything said above can be put
also in the relative framework.
Namely, consider a submersive morphism $\pi: \scrM \to \scrN$ 
and let $\omega$ be a volume form on $\scrM$ whose support
is proper w.r. to $\pi$.
We can then ``integrate it along the fibers'', i.e.
$\exists$ a well defined volume form
$\pi_!\omega\in\Gamma(\scrN,\Vol(\scrN))$ and if
$\omega\in\Gamma_c(\scrM,\Vol(\scrM))$,
$$
\int\limits_{\scrM}\omega=\int\limits_{\scrN}\pi_{!}\omega.
$$

In particular, in order to compute
$\int\limits_{\scrM}\omega$ for a supermanifold $\scrM$, one
must choose a projection $p\colon\,\matrix \scrM\\
\big\downarrow\\ \scrMbar\endmatrix$ which reduces
$\int\limits_{\scrM}\omega$ to the usual integration
$\int\limits_{\scrMbar}p_!(\omega)$.

Note that the integration $\omega\to p_!(\omega)$ is
``non-analytic''.
 
\subhead
4.3. Integral forms
\endsubhead

The following formalism will be useful for us in the sequel.
We introduce sheaves $\sum^{-i}$ on a supermanifold $\scrM$
as $\sum^{-i}=\HOM(\Omega^i(\scrM),\Vol(\scrM))$.
We have an action of vector fields on $\scrM$ on
$\sum^{-i}$'s by Lie derivatives.
The direct sum $\sum(\scrM):=\oplusop_{i}\sum^i$ is a $\dbZ$-graded
module over the sheaf of superalgebras
$\Omega(\scrM):=\oplusop_{i}\Omega^i(\scrM)$.
It is easy to check that $\forall\, X \in\Gamma(\scrM,\scrF(\scrM))$
extends uniquely to a derivation of degree $-1$ on
$\oplusop_{i}\sum^i$ compatible with the module structure.

\proclaim{Lemma}
$\exists!$ odd differential of degree $1$
$d:\,\sum\to\sum$ which makes it
a $DG$-module over $\Omega(\scrM)$ such that
$\forall\,X \in\Gamma(\scrM,\scrT(\scrM))$ we have 
$[i_{X},d]=\Lie_{X}$.\qquad $\blacksquare$
\endproclaim

\bigskip
\subhead
5. SUSY Manifolds
\endsubhead

\subhead
5.1
\endsubhead

\definition{Definition}
A pre-SUSY manifold is a supermanifold $\scrM$ of dimension
$(p,q)$ equipped with a distribution
$\gra\hookrightarrow\scrT(\scrM)$ of dimension $(0,q)$.
It is called a SUSY manifold if the Frobenius pairing
$$
\Fr\colon\,\gra\otimes\gra\to\scrT(\scrM)/\gra
$$
is sufficiently non-degenerate.
\enddefinition

Let us consider first the case of SUSY curves, i.e.
$(p,q)=(1,1)$.
In this case the non-degeneracy condition reads as:
$\Fr\colon\gra\otimes\gra\to\scrT(\scrM)/\gra$ is an
isomorphism.
In what follows, our assertions apply both to the case of
usual supermanifolds and complex analytic ones.

\definition{Definition}
A Spin curve (complex or real) is a curve $C$ with a line
bundle $S$ and an isomorphism $S\otimes S=\scrT(C)$.
\enddefinition

\proclaim{Proposition}
The category of SUSY curves is equivalent to the category 
of Spin curves.
\endproclaim

\demo{Proof}
Let $(C,\gra)$ be a SUSY curve.
Consider the restriction $\gra\vert\Cbar$.
By definition,
$$
\gra\vert\Cbar\otimes\gra\vert\Cbar\simeq
\scrT(C)\vert\Cbar/\gra\vert\Cbar\simeq\tau(\Cbar).
$$
Therefore, $(\Cbar,\gra\vert\Cbar)$ is a Spin curve and we
have constructed a functor SUSY curves $\to$ Spin curves.

Since locally on a curve all Spin structures are
isomorphic, it is sufficient to prove the local faithfulness
of this functor.

So, let $C=\dbR^{1,1}$ be with coordinates $(x,\xi)$ and let
$\gra$ be given by $C^\infty(C)\cdot(\partial_\xi+
x\partial_x)$.
It is easy to see then that there are no non-trivial
automorphisms of this SUSY structure that induce the trivial
automorphism of $(\Cbar,\gra\vert\Cbar)$.\qquad
$\blacksquare$
\enddemo

Let $S\scrM_g$ be the moduli space of Spin curves of genus
$g$.
The proposition proved above implies that it admits a
natural super extension, namely the supermanifold of moduli
of SUSY curves of genus $g$.
$$
S\scrM_g\hookrightarrow \SUSY\,\scrM_g.
$$
Moreover, $S\scrM_g$ identifies with
$\overline{\SUSY\,\scrM_g}$.

\subhead
5.2
\endsubhead
In the sequel we will consider the example of a SUSY
$(2,2)$-manifold.

Let $\scrV$ be a SUSY $(2,2)$-manifold and let us assume,
moreover, that $\scrV$ is locally a product of two SUSY
curves.
This condition is equivalent to the existence of a
decomposition:
$$
\align
\gra &=\gra_1\oplus\gra_2\\
\scrT(\scrV)/\gra &=(\scrT(\scrV)/\gra)_1\oplus
     (\scrT(\scrV)/\gra)_2\\
\Fr(\gra_1\otimes\gra_1) &\simeq (\scrT(\scrV)/\gra)_1\\
\Fr(\gra_2\otimes\gra_2) &\simeq (\scrT(\scrV)/\gra)_2\\
\Fr(\gra_1\otimes\gra_2) &=\Fr(\gra_2\otimes\gra_1)=0.
\endalign
$$

Such a decomposition is canonical if it exists.


\proclaim{Lemma}
In the above situation, $\Vol(\scrV)$ canonically identifies
with $\gra_1^*\otimes\gra_2^*$.
\qquad $\blacksquare$
\endproclaim


\enddocument















