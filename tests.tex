\section{Tests of the AdS/CFT Correspondence} 
\label{tests}

In this section we review the direct tests of the AdS/CFT
correspondence. In section \ref{correspondence} 
we saw how string theory on $AdS$
defines a partition function which can be used to define a field
theory. Here we will review the evidence showing that this field
theory is indeed the same as the conjectured dual field theory. We
will focus here only on tests of the correspondence between the
$\cn=4$ $SU(N)$ SYM theory and the type IIB string theory compactified
on $AdS_5\times S^5$; most of the tests described here can
be generalized also to cases in other dimensions and/or with less
supersymmetry, which will be described below.

As described in section \ref{correspondence}, 
the AdS/CFT correspondence is a
strong/weak coupling duality. In the 't Hooft large $N$ limit, it
relates the region of weak field theory coupling $\lambda=g_{YM}^2 N$
in the SYM theory to the region of high curvature (in string units) in
the string theory, and vice versa. Thus, a direct comparison of
correlation functions is generally not possible, since (with our
current knowledge) we can only compute most of them perturbatively in
$\lambda$ on the field theory side and perturbatively in
$1/\sqrt{\lambda}$ on the string theory side. For example, as
described below, we can compute the equation of state of the SYM
theory and also the quark-anti-quark potential both for small
$\lambda$ and for large $\lambda$, and we obtain different answers,
which we do not know how to compare since we can only compute them
perturbatively on both sides. A similar situation arises also in many
field theory dualities that were analyzed in the last few years (such
as the electric/magnetic $SL(2,\IZ)$ duality of the $\cn=4$ SYM theory
itself), and it was realized that there are several properties of
these theories which do not depend on the coupling, so they can be
compared to test the duality. These are:

\begin{itemize}
\item{} The global symmetries of the theory, which cannot change as we
change the coupling (except for extreme values of the coupling). As
discussed in section \ref{correspondence}, 
in the case of the AdS/CFT correspondence we
have the same supergroup $SU(2,2|4)$ (whose bosonic subgroup is
$SO(4,2)\times SU(4)$) as the global symmetry of both theories. Also,
both theories are believed to have a non-perturbative $SL(2,\IZ)$
duality symmetry acting on their coupling constant $\tau$. These are
the only symmetries of the theory on $\IR^4$. Additional $\IZ_N$
symmetries arise when the theories are compactified on
non-simply-connected manifolds, and these were also successfully
matched in \cite{Aharony:1998qu,Witten:1998wy}\footnote{Unlike 
most of the other tests described
here, this test actually tests the finite $N$ duality and not just the
large $N$ limit.}.

\item{} Some correlation functions, which are usually related to
anomalies, are protected from any quantum corrections and do not
depend on $\lambda$. The matching of these correlation functions will
be described in section \ref{anomalies} below.

\item{} The spectrum of chiral operators does not change as the
coupling varies, and it will be compared in section \ref{chiralops}
below.

\item{} The moduli space of the theory also does not depend on the
coupling. In the $SU(N)$ field theory the moduli space is
$\IR^{6(N-1)}/S_N$, parametrized by the eigenvalues of six commuting
traceless $N\times N$ matrices. On the AdS side it is not clear
exactly how to define the moduli space. As described in section 
\ref{multicenter_sols},
there is a background of string theory corresponding to any point in
the field theory moduli space, but it is not clear how to see that
this is the exact moduli space on the string theory side (especially
since high curvatures arise for generic points in the moduli space).

\item{} The qualitative behavior of the theory upon deformations by
relevant or marginal operators also does not depend on the coupling
(at least for chiral operators whose dimension does not depend on the
coupling, and in the absence of phase transitions). This will be
discussed in section \ref{deformations}. 

There are many more
qualitative tests of the correspondence, such as the existence of
confinement for the finite temperature theory \cite{Witten:1998zw},
which we will not discuss in this section. We will also not discuss
here tests involving the behavior of the theory on its moduli space
\cite{Douglas:1998tk,Gonzalez-Rey:1998uh,Das:1999ij}.

\end{itemize}

\subsection{The Spectrum of Chiral Primary Operators}
\label{chiralops}

\subsubsection{The Field Theory Spectrum}
\label{fieldspect}

The $\cn=4$ supersymmetry algebra in $d=4$ has four generators
$Q_\alpha^A$ (and their complex conjugates ${\bar
Q}_{\dot{\alpha}A}$), where $\alpha$ is a Weyl-spinor index
(in the $\bf 2$ of the $SO(3,1)$ Lorentz group) and $A$ is an index in
the $\bf 4$ of the $SU(4)_R$ R-symmetry group (lower indices $A$ will be
taken to transform in the $\bf {\bar 4}$ representation). 
They obey the algebra
\eqn{nfouralgebra}{\eqalign{
\{Q_{\alpha}^A, {\bar Q}_{\dot{\alpha}B} \} &= 2(\sigma^\mu)_{\alpha
\dot{\alpha}} P_\mu \delta^A_B, \cr
\{Q_\alpha^A, Q_\beta^B \} &= \{{\bar Q}_{\dot{\alpha}A}, {\bar
Q}_{\dot{\beta}B} \} = 0, }
}
where $\sigma^i$ ($i=1,2,3$) are the Pauli matrices and
$(\sigma^0)_{\alpha \dot{\alpha}}=-\delta_{\alpha \dot{\alpha}}$
(we use the conventions of Wess and Bagger \cite{Bagger}).

$\cn=4$ supersymmetry in four dimensions has a unique multiplet which
does not include spins greater than one, which is the vector
multiplet. It includes a vector field $A_\mu$ ($\mu$ is a vector
index of the $SO(3,1)$ Lorentz group), four complex
Weyl fermions $\lambda_{\alpha A}$ (in the $\bf {\bar 4}$ of
$SU(4)_R$), and six real scalars
$\phi^{I}$ (where $I$ is an index in the $\bf 6$ of $SU(4)_R$). The
classical action of the supersymmetry generators on these fields is
schematically given (for on-shell fields) by
\eqn{nfouraction}{\eqalign{
[Q_\alpha^A, \phi^I] &\sim \lambda_{\alpha B}, \cr
\{Q_\alpha^A, \lambda_{\beta B}\} &\sim
(\sigma^{\mu \nu})_{\alpha \beta} F_{\mu \nu} + \epsilon_{\alpha
\beta} [\phi^I, \phi^J], \cr
\{Q_\alpha^A, {\bar \lambda}_{\dot{\beta}}^B \} &\sim
({\sigma}^\mu)_{\alpha \dot{\beta}} {\cal D}_\mu \phi^I, \cr
[Q_\alpha^A, A_\mu] &\sim ({\sigma}_\mu)_{\alpha \dot{\alpha}} {\bar
\lambda}_{\dot{\beta}}^A \epsilon^{\dot{\alpha} \dot{\beta}}, 
}}
with similar expressions for the action of the $\bar Q$'s, where
$\sigma^{\mu \nu}$ are the generators of the Lorentz group in the
spinor representation, ${\cal D}_\mu$ is the covariant derivative,
the field strength
$F_{\mu \nu} \equiv [{\cal D}_\mu, {\cal D}_\nu]$, and we have suppressed
the $SU(4)$ Clebsch-Gordan coefficients corresponding to the products
$\bf{4\times 6 \to {\bar 4}}$, $\bf{4\times {\bar 4} \to 1+15}$ and
$\bf{4\times 4 \to 6}$ in the first three lines of \eno{nfouraction}.

An $\cn=4$ supersymmetric field theory is uniquely determined by
specifying the gauge group, and its field content is a vector
multiplet in the adjoint of the gauge group. Such a field theory is
equivalent to an $\cn=2$ theory with one hypermultiplet in the adjoint
representation, or to an $\cn=1$ theory with three chiral multiplets
$\Phi^i$ in the adjoint representation (in the ${\bf 3}_{2/3}$ of the
$SU(3)\times U(1)_R
\subset SU(4)_R$ which is left unbroken by the choice of a single
$\cn=1$ SUSY generator) and a superpotential of the form $W \propto
\epsilon_{ijk} \tr(\Phi^i \Phi^j \Phi^k)$. The interactions of the
theory include a scalar potential proportional to $\sum_{I,J}
\tr([\phi^I,\phi^J]^2)$, such that the moduli space of the theory is
the space of commuting matrices $\phi^I$ ($I=1,\cdots,6$).

The spectrum of operators in this theory includes all the gauge
invariant quantities that can be formed from the fields described
above. In this section we will focus on local operators which involve
fields taken at the same point in space-time. For the $SU(N)$ theory
described above, properties of the adjoint representation of $SU(N)$
determine that such operators necessarily involve a product of traces
of products of fields (or the sum of such products). It is natural to
divide the operators into single-trace operators and multiple-trace
operators. In the 't Hooft large $N$ limit correlation
functions involving multiple-trace operators are suppressed by powers
of $N$ compared to those of single-trace operators involving the same
fields. We will discuss here in detail only the single-trace
operators; the multiple-trace operators appear in operator product
expansions of products of single-trace operators.

As discussed in section \ref{cft}, it is natural to classify the
operators in a conformal theory into primary operators and their
descendants. In a superconformal theory it is also natural to
distinguish between chiral primary operators, which are in short
representations of the superconformal algebra and are annihilated by
some of the supercharges, and non-chiral primary operators.
Representations of the superconformal algebra are formed by starting
with some state of lowest dimension, which is annihilated by the
operators $S$ and $K_\mu$, and acting on it with the operators $Q$ and
$P_\mu$.  The $\cn=4$ supersymmetry algebra involves 16 real
supercharges.  A generic primary representation of the superconformal
algebra will thus include $2^{16}$ primaries of the conformal algebra,
generated by acting on the lowest state with products of different
supercharges;
%(acting twice with the same supercharge gives zero);
acting with additional supercharges always leads to descendants of the
conformal algebra (i.e. derivatives). Since the supercharges have
helicities $\pm 1/2$, the primary fields in such representations will
have a range of helicities between $\lambda-4$ (if the lowest
dimension operator $\psi$ has helicity $\lambda$) and $\lambda+4$
(acting with more than 8 supercharges of the same helicity either
annihilates the state or leads to a conformal descendant). In
non-generic representations of the superconformal algebra a product of
less than 16 different $Q$'s annihilates the lowest dimension
operator, and the range of helicities appearing is smaller. In
particular, in the small representations of the $\cn=4$ superconformal
algebra only up to 4 $Q$'s of the same helicity acting on the lowest
dimension operator give a non-zero result, and the range of helicities
is between $\lambda-2$ and $\lambda+2$. For the $\cn=4$ supersymmetry
algebra (not including the conformal algebra) it is known that medium
representations, whose range of helicities is 6, can also exist (they
arise, for instance, on the moduli space of the $SU(N)$ $\cn=4$ SYM
theory \cite{Bergman:1997yw,Hashimoto:1998zs,Kawano:1998bp,
Bergman:1998gs,Hashimoto:1998nj,Lee:1998nv,Sasakura:1998cx,Tong:1999mg}); 
it is not clear if such medium representations of the
superconformal algebra \cite{Gunaydin:1998jc} 
can appear in physical theories or not
(there are no known examples). More details on the structure of
representations of the $\cn=4$ superconformal algebra may be found in
\cite{Gunaydin:1985fk,Andrianopoli:1998jh,Andrianopoli:1998nc,
Ferrara:1998pr,Gunaydin:1998sw,Andrianopoli:1998ut,Gunaydin:1998jc}
and references therein.

In the $U(1)$ $\cn=4$ SYM theory (which is a free theory), the only
gauge-invariant ``single trace'' operators are the fields of the
vector multiplet itself (which are $\phi^I,\lambda_A,
{\bar \lambda}^A$ and $F_{\mu \nu}
= \del_{[\mu} A_{\nu]}$). These operators form an ultra-short
representation of the $\cn=4$ algebra whose range of helicities is
from $(-1)$ to $1$ (acting with more than two supercharges of the same
helicity on any of these states gives either zero or derivatives,
which are descendants of the conformal algebra). All other local gauge
invariant operators in the theory involve derivatives or products of
these operators. This representation is usually called the doubleton
representation, and it does not appear in the $SU(N)$ SYM theory
(though the representations which do appear can all be formed by
tensor products of the doubleton representation). In the context of
AdS space one can think of this multiplet as living purely on the
boundary of the space \cite{Flato:1978qz,Fronsdal:1980aa,
Flato:1981we,Angelopoulos:1981wg,Nicolai:1984gb,Gunaydin:1985wc,
Gunaydin:1985vz,Flato:1986bf,Ferrara:1997dh,Ferrara:1998bv,Ferrara:1998jm}, as
expected for the $U(1)$ part of the original $U(N)$ gauge group of the
D3-branes (see the discussion in section \ref{correspondence}).

There is no known
simple systematic way to compute the full spectrum of
chiral primary operators of the $\cn=4$ $SU(N)$ SYM theory, so we will
settle for presenting the known chiral primary operators. The lowest
component of a superconformal-primary multiplet is characterized by
the fact that it cannot be written as a supercharge $Q$ acting on any
other operator. Looking at the action of the supersymmetry charges
\eno{nfouraction} suggests that generally operators built from the
fermions and the gauge fields will be descendants (given by $Q$ acting
on some other fields), so one would expect the lowest components of
the chiral primary representations to be built only from the scalar
fields, and this turns out to be correct.

Let us analyze the behavior of operators of the form ${\cal O}^{I_1
I_2 \cdots I_n} \equiv \tr(\phi^{I_1} \phi^{I_2} \cdots \phi^{I_n})$.
First we can ask if this operator can be written as $\{Q,\psi\}$ for
any field $\psi$. In the SUSY algebra
\eno{nfouraction} only commutators of $\phi^I$'s appear on the right
hand side, so we see that if some of the indices are antisymmetric the
field will be a descendant. Thus, only symmetric combinations of the
indices will be lowest components of primary multiplets. Next, we
should ask if the multiplet built on such an operator is a (short)
chiral primary multiplet or not. There are several different ways to
answer this question. One possibility is to use the relation between
the dimension of chiral primary operators and their R-symmetry
representation \cite{Kac:1977hp,Dobrev:1987qz,Dobrev:1985qv,
Seiberg:1997ax,Minwalla:1998ka}, and to check if this relation is
obeyed in the free field theory, where $[{\cal O}^{I_1 I_2 \cdots
I_n}] = n$. In this way we find that the representation is chiral
primary if and only if the indices form a symmetric traceless product
of $n$ $\bf 6$'s (traceless representations are defined as those who
give zero when any two indices are contracted). This is a representation
of weight $(0,n,0)$ of $SU(4)_R$; in this section we will refer to
$SU(4)_R$ representations either by their dimensions in boldface or by
their weights.

Another way to check this is to see if by acting with $Q$'s on these
operators we get the most general possible states or not, namely if
the representation contains ``null vectors'' or not (it turns out that
in all the relevant cases ``null vectors'' appear already at the first
level by acting with a single $Q$, though in principle there could be
representations where ``null vectors'' appear only at higher
levels). Using the SUSY algebra \eno{nfouraction} it is easy to see
that for symmetric traceless representations we get ``null vectors''
while for other representations we do not. For instance, let us
analyze in detail the case $n=2$. The symmetric product of two $\bf
6$'s is given by $\bf{6\times 6 \to 1 + 20'}$. The field in the $\bf
1$ representation is $\tr(\phi^I \phi^I)$, for which $[Q_\alpha^A,
\tr(\phi^I \phi^I)] \sim C^{AJB} \tr(\lambda_{\alpha B} \phi^J)$ where
$C^{AIB}$ is a Clebsch-Gordan coefficient for ${\bf {\bar 4}\times 6
\to 4}$. The right-hand side is in the $\bf 4$ representation, which is
the most general representation that can appear in the product
$\bf{4\times 1}$, so we find no null vectors at this level. On the
other hand, if we look at the symmetric traceless product
$\tr(\phi^{\{I} \phi^{J\}})\equiv \tr(\phi^I \phi^J) - {1\over 6}
\delta^{IJ} \tr(\phi^K \phi^K)$ in the ${\bf 20'}$ representation, we
find that $\{Q_{\alpha}^A, \tr(\phi^{\{I} \phi^{J\}}) \} \sim
\tr(\lambda_{\alpha B} \phi^K)$ with the right-hand side being in the
$\bf 20$ representation (appearing in $\bf{{\bar 4}\times 6 \to 4 +
20}$), while the left-hand side could in principle be in the
$\bf{4\times 20' \to 20+60}$. Since the $\bf 60$ does not appear on
the right-hand side (it is a ``null vector'') we identify that the
representation built on the $\bf 20'$ is a short representation of the
SUSY algebra. By similar manipulations (see \cite{Witten:1998qj,
Ferrara:1998ej,Andrianopoli:1998jh,Gunaydin:1998sw} for more details)
one can verify that chiral primary representations correspond exactly
to symmetric traceless products of $\bf 6$'s.

It is possible to analyze the chiral primary spectrum also
by using $\cn=1$ subalgebras of the $\cn=4$ algebra. If we use an
$\cn=1$ subalgebra of the $\cn=4$ algebra, as described above, the
operators ${\cal O}_n$ include the chiral operators of the form
$\tr(\Phi^{i_1} \Phi^{i_2} \cdots \Phi^{i_n})$ (in a representation of
$SU(3)$ which is a symmetric product of $\bf 3$'s), but for a
particular choice of the $\cn=1$ subalgebra not all the operators
${\cal O}_n$ appear to be chiral (a short multiplet of the $\cn=4$
algebra includes both short and long multiplets of the $\cn=1$
subalgebra).

The last issue we should discuss is what is the range of values of
$n$. The product of more than $N$ commuting\footnote{We can limit the
discussion to commuting matrices since, as discussed above,
commutators always lead to descendants, and we can write any product
of matrices as a product of commuting matrices plus terms with
commutators.} $N\times N$ matrices can
always be written as a sum of products of traces of less than $N$ of
the matrices, so it does not form an independent operator. This means
that for $n > N$ we can express the operator ${\cal O}^{I_1 I_2 \cdots
I_n}$ in terms of other operators, up to operators including
commutators which (as explained above) are descendants of the SUSY
algebra. Thus, we find that the short chiral primary representations
are built on the operators ${\cal O}_n = {\cal O}^{\{I_1 I_2 \cdots
I_n\}}$ with $n=2,3,\cdots,N$, for which the indices are in the
symmetric traceless product of $n$ $\bf 6$'s (in a $U(N)$ theory we would
find the same spectrum with the additional representation
corresponding to $n=1$). The superconformal algebra determines the
dimension of these fields to be $[{\cal O}_n] = n$, which is the same
as their value in the free field theory. We argued above that these
are the only short chiral primary representations in the $SU(N)$ gauge
theory, but we will not attempt to rigorously prove this here.

The full chiral primary representations are obtained by acting on the
fields ${\cal O}_n$ by the generators $Q$ and $P$ of the supersymmetry
algebra. The representation built on ${\cal O}_n$ contains a total of
$256\times {1\over 12}n^2(n^2-1)$ primary states, of which half are
bosonic and half are fermionic. Since these multiplets are built on a
field of helicity zero, they will contain primary fields of helicities
between $(-2)$ and $2$. The highest dimension primary field in the
multiplet is (generically) of the form $Q^4 {\bar Q}^4 {\cal O}_n$,
and its dimension is $n+4$. There is an elegant way to write these
multiplets as traces of products of ``twisted chiral $\cn=4$
superfields'' \cite{Ferrara:1998ej,Andrianopoli:1998jh}; see also
\cite{Ferrara:1998bp} which checks some components of these
superfields against the couplings to supergravity modes predicted on
the basis of the DBI action for D3-branes in anti-de Sitter space
\cite{Das:1998ei}.

It is easy to find the form of all the fields in such a multiplet by
using the algebra \eno{nfouraction}.  For example, let us analyze here
in detail the bosonic primary fields of dimension $n+1$ in the multiplet. To
get a field of dimension $n+1$ we need to act on ${\cal O}_n$ with two
supercharges (recall that $[Q]={1\over 2}$).  If we act with two
supercharges $Q_{\alpha}^A$ of the same chirality, their Lorentz
indices can be either antisymmetrized or symmetrized.
In the first case we get a Lorentz scalar field in the
$(2,n-2,0)$ representation of $SU(4)_R$, which is of the schematic
form 
\eqn{nplusone}{
\epsilon^{\alpha \beta} \{Q_\alpha, [Q_\beta, {\cal O}_n] \} \sim
\epsilon^{\alpha \beta} \tr(\lambda_{\alpha A} \lambda_{\beta B}
\phi^{J_1} \cdots \phi^{J_{n-2}}) + \tr([\phi^{K_1}, \phi^{K_2}]
\phi^{L_1} \cdots \phi^{L_{n-1}}).} 
Using an $\cn=1$ subalgebra some of these operators may be written as
the lowest components of the chiral superfields $\tr(W_{\alpha}^2
\Phi^{j_1} \cdots \Phi^{j_{n-2}})$. In the second case we get an
anti-symmetric 2-form of the Lorentz group, in the $(0,n-1,0)$
representation of $SU(4)_R$, of the form 
\eqn{nplusonet}{
\{Q_{\{\alpha}, [Q_{\beta\}}, {\cal O}_n] \} \sim
\tr((\sigma^{\mu
\nu})_{\alpha \beta} F_{\mu \nu} \phi^{J_1} \cdots \phi^{J_{n-1}}) +
\tr(\lambda_{\alpha A} \lambda_{\beta B} \phi^{K_1} \cdots
\phi^{K_{n-2}}).} 
Both of these fields are complex, with the complex
conjugate fields given by the action of two ${\bar Q}$'s. Acting with
one $Q$ and one ${\bar Q}$ on the state ${\cal O}_n$ gives a (real)
Lorentz-vector field in the $(1,n-2,1)$ representation of $SU(4)_R$,
of the form 
\eqn{nplusonev}{
\{Q_\alpha, [{\bar Q}_{\dot{\alpha}}, {\cal O}_n] \} \sim
\tr(\lambda_{\alpha A} {\bar \lambda}_{\dot{\alpha}}^B
\phi^{J_1} \cdots \phi^{J_{n-2}}) + (\sigma^\mu)_{\alpha \dot{\alpha}}
\tr(({\cal D}_\mu \phi^J) \phi^{K_1} \cdots \phi^{K_{n-1}}).}

At dimension $n+2$ (acting with four supercharges) we find :
\begin{itemize}
\item{} A complex scalar field in the $(0,n-2,0)$ representation,
given by $Q^4 {\cal O}_n$, of the form $\tr(F_{\mu \nu}^2 \phi^{I_1}
\cdots \phi^{I_{n-2}}) + \cdots$.
\item{} A real scalar field in the $(2,n-4,2)$ representation, given
by $Q^2 {\bar Q}^2 {\cal O}_n$, of the form $\epsilon^{\alpha \beta}
\epsilon^{\dot{\alpha} \dot{\beta}} \tr(\lambda_{\alpha A_1}
\lambda_{\beta A_2} {\bar \lambda}_{\dot{\alpha}}^{B_1} {\bar
\lambda}_{\dot{\beta}}^{B_2} \phi^{I_1} \cdots \phi^{I_{n-4}}) +
\cdots$.
\item{} A complex vector field in the $(1,n-4,1)$ representation,
given by $Q^3 {\bar Q} {\cal O}_n$, of the form $\tr(F_{\mu \nu} {\cal
D}^\nu \phi^{J} \phi^{I_1} \cdots \phi^{I_{n-2}}) + \cdots$.
\item{} An complex anti-symmetric 2-form field in the $(2,n-3,0)$
representation, given by $Q^2 {\bar Q}^2 {\cal O}_n$, of the form
$\tr(F_{\mu \nu} [\phi^{J_1},\phi^{J_2}] \phi^{I_1} \cdots
\phi^{I_{n-2}}) + \cdots$.
\item{} A symmetric tensor field in the $(0,n-2,0)$ representation,
given by $Q^2 {\bar Q}^2 {\cal O}_n$, of the form $\tr({\cal D}_{\{\mu}
\phi^J {\cal D}_{\nu\}} \phi^K \phi^{I_1} \cdots \phi^{I_{n-2}}) + \cdots$.
\end{itemize}

The spectrum of primary fields at dimension $n+3$ is similar to that
of dimension $n+1$ (the same fields appear but in smaller $SU(4)_R$
representations), and at dimension $n+4$ there is a single primary
field, which is a real scalar in the $(0,n-4,0)$ representation, given
by $Q^4 {\bar Q}^4 {\cal O}_n$, of the form $\tr(F_{\mu \nu}^4
\phi^{I_1} \cdots \phi^{I_{n-4}}) + \cdots$. Note that fields with
more than four $F_{\mu \nu}$'s or more than eight $\lambda$'s are
always descendants or non-chiral primaries. 

For $n=2,3$ the short multiplets are even shorter since some of the
representations appearing above vanish. In particular, for $n=2$ the
highest-dimension primaries in the chiral primary multiplet have
dimension $n+2=4$. The $n=2$ representation includes the currents of
the superconformal algebra. It includes a vector of dimension 3 in the
$\bf 15$ representation which is the $SU(4)_R$ R-symmetry current, and
a symmetric tensor field of dimension 4 which is the energy-momentum
tensor (the other currents of the superconformal algebra are
descendants of these). The $n=2$ multiplet also includes a complex
scalar field which is an $SU(4)_R$-singlet, whose real part is the
Lagrangian density coupling to ${1\over {4g_{YM}^2}}$ (of the form
$\tr(F_{\mu \nu}^2) + \cdots$) and whose imaginary part is the
Lagrangian density coupling to $\theta$ (of the form $\tr(F\wedge
F)$). For later use we note that the chiral primary multiplets which
contain scalars of dimension $\Delta \leq 4$ are the $n=2$ multiplet
(which has a scalar in the ${\bf 20'}$ of dimension 2, a complex
scalar in the $\bf 10$ of dimension 3, and a complex scalar in the $\bf
1$ of dimension 4), the $n=3$ multiplet (which contains a scalar in
the $\bf 50$ of dimension 3 and a complex scalar in the $\bf 45$ of
dimension 4), and the $n=4$ multiplet which contains a scalar in the
$\bf 105$ of dimension 4.

\subsubsection{The String Theory Spectrum and the Matching}
\label{stringy_spect}

As discussed in section \ref{field_operator}, 
fields on $AdS_5$ are in a one-to-one
correspondence with operators in the dual conformal field
theory. Thus, the spectrum of operators described in section
\ref{fieldspect} should agree with the spectrum of fields of type IIB
string theory on $AdS_5\times S^5$. Fields on AdS naturally lie in the
same multiplets of the conformal group as primary operators; the
second Casimir of these representations is $C_2=\Delta(\Delta-4)$ for
a primary scalar field of dimension $\Delta$ in the field theory, and
$C_2=m^2 R^2$ for a field of mass $m$ on an $AdS_5$ space with a
radius of curvature $R$. Single-trace operators in the field theory
may be identified with single-particle states in $AdS_5$, while
multiple-trace operators correspond to multi-particle states.

Unfortunately, it is not known how to compute the full spectrum of
type IIB string theory on $AdS_5\times S^5$. In fact, the only known
states are the states which arise from the dimensional reduction of
the ten-dimensional type IIB supergravity multiplet. These fields all
have helicities between $(-2)$ and $2$, so it is clear that they all
lie in small multiplets of the superconformal algebra, and we will
describe below how they match with the small multiplets of the field
theory described above. String theory on $AdS_5\times S^5$ is expected
to have many additional states, with masses of the order of the string
scale $1/l_s$ or of the Planck scale $1/l_p$. Such states would
correspond (using the mass/dimension relation described above) to
operators in the field theory with dimensions of order $\Delta \sim
(g_s N)^{1/4}$ or $\Delta \sim N^{1/4}$ for large $N, g_s N$. 
Presumably none of these
states are in small multiplets of the superconformal algebra (at
least, this would be the prediction of the AdS/CFT correspondence).

The spectrum of type IIB supergravity compactified on $AdS_5\times
S^5$ was computed in \cite{Kim:1985ez}. 
The computation involves expanding the ten
dimensional fields in appropriate spherical harmonics on $S^5$,
plugging them into the supergravity equations of motion, linearized
around the $AdS_5\times S^5$ background, and diagonalizing the
equations to give equations of motion for free (massless or massive)
fields\footnote{The fields arising from different spherical harmonics
are related by a ``spectrum generating algebra'', 
see \cite{Berglund:1999}.}. 
For example, the ten dimensional dilaton field $\tau$ may be
expanded as $\tau(x,y) = \sum_{k=0}^{\infty} \tau^k(x) Y^k(y)$ where
$x$ is a coordinate on $AdS_5$, $y$ is a coordinate on $S^5$, and the
$Y^k$ are the scalar spherical harmonics on $S^5$. These spherical
harmonics are in representations corresponding to symmetric traceless
products of $\bf 6$'s of $SU(4)_R$; they may be written as $Y^k(y)
\sim y^{I_1} y^{I_2} \cdots y^{I_k}$ where the $y^I$, for
$I=1,2,\cdots,6$ and with $\sum_{I=1}^6 (y^I)^2 = 1$, are coordinates
on $S^5$. Thus, we find a field $\tau^k(x)$ on $AdS_5$ in each such
$(0,k,0)$ representation of $SU(4)_R$, and the equations of motion
determine the mass of this field to be $m_k^2 = k(k+4)/R^2$. A similar
expansion may be performed for all other fields.

If we organize the results of \cite{Kim:1985ez} into representations
of the superconformal algebra \cite{Gunaydin:1985fk}, 
we find representations of the form
described in the previous section, which are built on a lowest
dimension field which is a scalar in the $(0,n,0)$ representation of
$SU(4)_R$ for $n=2,3,\cdots,\infty$. The lowest dimension scalar field
in each representation
turns out to arise from a linear combination of spherical harmonic
modes of the $S^5$ components of the graviton $h^a_a$ (expanded around
the $AdS_5\times S^5$ vacuum) and the 4-form field $D_{abcd}$, where
$a,b,c,d$ are indices on $S^5$. The scalar fields of dimension $n+1$
correspond to 2-form fields $B_{ab}$ with indices in the $S^5$.  The
symmetric tensor fields arise from the expansion of the
$AdS_5$-components of the graviton. The dilaton fields described above
are the complex scalar fields arising with dimension $n+2$ in the
multiplet (as described in the previous subsection).

In particular, the $n=2$ representation is called the supergraviton
representation, and it includes the field content of $d=5,\cn=8$
gauged supergravity. The field/operator correspondence matches this
representation to the representation including the superconformal
currents in the field theory. It includes a massless graviton field,
which (as expected) corresponds to the energy-momentum tensor in the
field theory, and massless $SU(4)_R$ gauge fields which correspond to
(or couple to) the global $SU(4)_R$ currents in the field theory.

In the naive dimensional reduction of the type IIB supergravity
fields, the $n=1$ doubleton representation, corresponding to a free
$U(1)$ vector multiplet in the dual theory, also appears. However, the
modes of this multiplet are all pure gauge modes in the bulk of
$AdS_5$, and they may be set to zero there. This is one of the reasons
why it seems more natural to view the corresponding gauge theory as an
$SU(N)$ gauge theory and not a $U(N)$ theory. It may be possible (and
perhaps even natural) to add the doubleton representation to the
theory (even though it does not include modes which propagate in the
bulk of $AdS_5$, but instead it is equivalent to a topological theory
in the bulk) to obtain a theory which is dual to the $U(N)$ gauge
theory, but this will not affect most of our discussion in this review
so we will ignore this possibility here.

Comparing the results described above with the results of section
\ref{fieldspect}, we see that we find the same spectrum of chiral
primary operators for $n=2,3,\cdots,N$. The supergravity results
cannot be trusted for masses above the order of the string scale
(which corresponds to $n\sim (g_s N)^{1/4}$) or the Planck scale
(which corresponds to $n\sim N^{1/4}$), so the results agree within
their range of validity. The field theory results suggest that the
exact spectrum of chiral representations in type IIB string theory on
$AdS_5\times S^5$ actually matches the naive supergravity spectrum up
to a mass scale $m^2 \sim N^2 / R^2 \sim N^{3/2} M_p^2$ which is much
higher than the string scale and the Planck scale, and that there are
no chiral fields above this scale. It is not known how to check this
prediction; tree-level string theory is certainly not enough for this
since when $g_s=0$ we must take $N=\infty$ to obtain a finite value of
$g_s N$. Thus, with our current knowledge the matching of chiral
primaries of the $\cn=4$ SYM theory with those of string theory on
$AdS_5\times S^5$ tests the duality only in the large $N$ limit. In
some generalizations of the AdS/CFT correspondence the string coupling
goes to zero at the boundary even for finite $N$, and then classical
string theory should lead to exactly the same spectrum of chiral
operators as the field theory. This happens in particular for 
%string
%theory on $AdS_3\times S^3\times {\cal M}$ with NS-NS
%charges\footnote{After performing a change of variables in the
%worldsheet theory.} 
%\cite{Giveon:1998ns}, and for 
the near-horizon limit of NS5-branes, 
in which case the exact spectrum was successfully compared in
\cite{Aharony:1998ub}. In other instances of the AdS/CFT
correspondence (such as the ones discussed in
\cite{Witten:1998xy,Klebanov:1998hh,Gubser:1998fp}) 
there exist also additional
chiral primary multiplets with $n$ of order $N$, and these have been
successfully matched with wrapped branes on the string theory side.

The fact that there seem to be no non-chiral fields on $AdS_5$ with a
mass below the string scale suggests that for large $N$ and large $g_s
N$, the dimension of all non-chiral operators in the field theory,
such as $\tr(\phi^I \phi^I)$, grows at least as $(g_s N)^{1/4} \sim
(g_{YM}^2 N)^{1/4}$. The reason for this behavior on the field theory
side is not clear; it is a prediction of the AdS/CFT correspondence.

\subsection{Matching of Correlation Functions and Anomalies}
\label{anomalies}

The classical $\cn=4$ theory has a scale invariance symmetry and an
$SU(4)_R$ R-symmetry, and (unlike many other theories) these
symmetries are exact also in the full quantum theory. However, when
the theory is coupled to external gravitational or $SU(4)_R$ gauge
fields, these symmetries are broken by quantum effects. In field
theory this breaking comes from one-loop diagrams and does not receive
any further corrections; thus it can be computed also in the strong
coupling regime and compared with the results from string theory on
AdS space.

We will begin by discussing the anomaly associated with the $SU(4)_R$
global currents. These currents are chiral since the fermions
$\lambda_{\alpha A}$ are in the $\bf {\bar 4}$ representation while
the fermions of the opposite chirality ${\bar
\lambda}_{\dot{\alpha}}^A$ are in the $\bf 4$ representation. Thus, if
we gauge the $SU(4)_R$ global symmetry, we will find an
Adler-Bell-Jackiw anomaly from the triangle diagram of three $SU(4)_R$
currents, which is proportional to the number of charged fermions.
In the $SU(N)$ gauge theory this number is $N^2-1$. The anomaly can be
expressed either in terms of the 3-point function of the $SU(4)_R$
global currents, 
\eqn{anomcorr}{
\left\langle J_{\mu}^a(x) J_{\nu}^b(y) J_\rho^c(z) \right\rangle_- =
-{{N^2-1}\over {32\pi^6}} id^{abc} {{\tr \left[ 
\gamma_5 \gamma_{\mu} (\not{x} - \not{y})
\gamma_{\nu} (\not{y} - \not{z}) \gamma_{\rho} (\not{z} - \not{x})
\right]}\over {(x-y)^4 (y-z)^4 (z-x)^4}},}
where 
$d^{abc} = 2\tr(T^a \{T^b, T^c\})$ and
we take only the negative parity component of the correlator, 
or in terms of the
non-conservation of the $SU(4)_R$ current when the theory is coupled to
external $SU(4)_R$ gauge fields $F_{\mu \nu}^a$,
\eqn{nonconcorr}{
({\cal D}^\mu J_\mu)^a = {{N^2-1}\over {384 \pi^2}} i d^{abc}
\epsilon^{\mu \nu \rho \sigma} F_{\mu \nu}^b F_{\rho \sigma}^c.}

How can we see this effect in string theory on $AdS_5\times S^5$ ? One
way to see it is, of course, to use the general prescription of
section \ref{correlators} 
to compute the 3-point function \eno{anomcorr}, and indeed
one finds \cite{Freedman:1998tz,Chalmers:1998xr} 
the correct answer to leading order in the large $N$
limit (namely, one recovers the term proportional to $N^2$). It is more
illuminating, however, to consider directly the meaning of the anomaly
\eno{nonconcorr} from the point of view of the AdS theory 
\cite{Witten:1998qj}. In the AdS
theory we have gauge fields $A_\mu^a$ which couple, as explained
above, to the $SU(4)_R$ global currents $J_\mu^a$ of the gauge theory,
but the anomaly means that when we turn on non-zero field strengths
for these fields the theory should no longer be gauge invariant. This
effect is precisely reproduced by a Chern-Simons term which exists in
the low-energy supergravity theory arising from the compactification
of type IIB supergravity on $AdS_5\times S^5$, which is of the form
\eqn{csterm}{{iN^2\over 96\pi^2} \int_{AdS_5} d^5x (d^{abc} \epsilon^{\mu \nu
\lambda \rho \sigma} A_\mu^a \del_\nu A_\lambda^b \del_\rho A_\sigma^c
+ \cdots).} 
This term is gauge invariant up to total derivatives,
which means that if we take a gauge transformation $A_\mu^a \to
A_\mu^a + ({\cal D}_\mu \Lambda)^a$ for which $\Lambda$ does not
vanish on the boundary of $AdS_5$, the action will change by a
boundary term of the form
\eqn{boundary}{
-{{iN^2}\over {384\pi^2}} \int_{\del AdS_5} d^4x \epsilon^{\mu \nu
\rho \sigma} d^{abc} \Lambda^a F_{\mu \nu}^b F_{\rho \sigma}^c.} 
{}From this we can read off the anomaly in $({\cal D}^\mu J_{\mu})$
since, when we have a coupling of the form $\int d^4x A^\mu_a
J_\mu^a$, the change in the action under a gauge transformation is
given by $\int d^4x ({\cal D}^\mu \Lambda)_a J_\mu^a = -\int d^4x
\Lambda_a ({\cal D}^\mu J_\mu^a)$, and we find exact agreement with
\eno{nonconcorr} for large $N$.

The other anomaly in the $\cn=4$ SYM theory is the conformal (or Weyl)
anomaly (see \cite{Deser:1993yx,Duff:1994wm} 
and references therein), indicating the breakdown of
conformal invariance when the theory is coupled to a curved external
metric (there is a similar breakdown of conformal invariance when the
theory is coupled to external $SU(4)_R$ gauge fields, which we will
not discuss here). The conformal anomaly is related to the 2-point and
3-point functions of the energy-momentum tensor 
\cite{Osborn:1994cr,Anselmi:1997mq,Erdmenger:1997yc,Anselmi:1997am}.  
In four dimensions, the general form of the conformal anomaly is
\eqn{confanom}{\vev{g^{\mu \nu} T_{\mu \nu}} = -aE_4 -cI_4,}
where
\eqn{confcoeff}{\eqalign{
E_4 &= {1\over {16\pi^2}} ({\cal R}_{\mu \nu \rho \sigma}^2 - 4 
{\cal R}_{\mu \nu}^2
+ {\cal R}^2), \cr
I_4 &= -{1\over {16\pi^2}} ({\cal R}_{\mu \nu \rho \sigma}^2 - 2 
{\cal R}_{\mu
\nu}^2 + {1\over 3}{\cal R}^2), }}
where ${\cal R}_{\mu \nu \rho \sigma}$ is the curvature tensor, 
${\cal R}_{\mu \nu}
\equiv {\cal R}_{\mu \rho \nu}^{\rho}$ is the Riemann tensor, and 
${\cal R} \equiv
{\cal R}^{\mu}_{\mu}$ is the scalar curvature.
A free field computation in the $SU(N)$ $\cn=4$ SYM theory leads to
$a=c=(N^2-1)/4$. In supersymmetric theories the supersymmetry algebra
relates $g^{\mu \nu}T_{\mu \nu}$ 
to derivatives of the R-symmetry current, so it is
protected from any quantum corrections. Thus, the same result should
be obtained in type IIB string theory on $AdS_5\times S^5$, 
and to leading order in the
large $N$ limit it should be obtained
from type IIB supergravity on $AdS_5\times S^5$. This
was indeed found to be true in
\cite{Henningson:1998gx,Henningson:1998ey,Balasubramanian:1999re,
Mueck:1999nf}\footnote{A generalization with more varying fields may be found
in \cite{Nojiri:1998dh}.}, 
where the conformal anomaly was
shown to arise from subtleties in the regularization of the
(divergent) supergravity action on $AdS$ space. The result of 
\cite{Henningson:1998gx,Henningson:1998ey,Balasubramanian:1999re,Mueck:1999nf}
implies that a computation using 
gravity on $AdS_5$ always gives rise to theories
with $a=c$, so generalizations of the AdS/CFT correspondence which
have (for large $N$) a supergravity approximation
are limited to conformal theories which have $a=c$ in the large
$N$ limit. Of course, if we do not require the string theory to have a
supergravity approximation then there is no such restriction.

For both of the anomalies we described the field theory and string
theory computations agree for the leading terms, which are of order
$N^2$. Thus, they are successful tests of the duality in the large $N$
limit. 
%The field theory results imply that there should be a
%correction to this of order $1$, which corresponds to a correction of
%order $1/N^2 = g_s^2 / (g_s N)^2 \sim g_s^2 (\alpha'/R^2)^4$ to the
%five dimensional action. However, this correction has not yet been
%computed directly. 
For other instances of the AdS/CFT correspondence there are
corrections to anomalies at order $1/N \sim g_s (\alpha'/R^2)^2$;
such corrections were discussed in \cite{Anselmi:1998zb} and
successfully compared in 
\cite{Aharony:1999rz,Blau:1999vz,Nojiri:1999mh}\footnote{Computing
such corrections tests the conjecture that the correspondence holds
order by order in $1/N$; however, this is weaker than the statement
that the correspondence holds for finite $N$, since the $1/N$
expansion is not expected to converge.}.
It would be interesting to compare other corrections to the large $N$
result.

Computations of other correlation functions 
\cite{Lee:1998bx,D'Hoker:1999tz,Gonzalez-Rey:1999ih}, 
such as 3-point
functions of chiral primary operators and correlation functions which
have only instanton contributions (we will discuss these in section
\ref{baryons}), have suggested that they are
also the same at small $\lambda$ and at large $\lambda$, even though
they are not related to anomalies in any known way. Perhaps there is
some non-renormalization theorem also for these correlation functions,
in which case their agreement would also be a test of the AdS/CFT
correspondence. 
As discussed in \cite{Intriligator:1998ig,
Intriligator:1999ff} (see also \cite{Ferrara:1998zt})
the non-renormalization theorem for 3-point
functions of chiral primary operators would follow from a conjectured
$U(1)_Y$ symmetry of the 3-point functions of $\cn=4$ SCFTs involving
at least two operators which are descendants of chiral 
primaries\footnote{A proof of this, using the analytic harmonic superspace
formalism which is conjectured to be valid in the $\cn=4$ theory,
was recently given in \cite{Eden:1999}.}. This
symmetry is a property of type IIB supergravity on $AdS_5\times S^5$
but not of the full type IIB string theory.


