%From: Pavel I Etingof <etingof@IAS.EDU>
%Date: Tue, 17 Dec 1996 10:01:33 -0500

\input amstex
\documentstyle{amsppt}
\magnification 1200
\NoRunningHeads
\NoBlackBoxes
\document

\def\top{{\text{top}}}
\def\Re{\text{Re}}
\def\tW{\tilde W}
\def\Aut{\text{Aut}}
\def\tr{{\text{tr}}}
\def\ell{{\text{ell}}}
\def\Ad{\text{Ad}}
\def\u{\bold u}
\def\m{\frak m}
\def\O{\Cal O}
\def\tA{\tilde A}
\def\qdet{\text{qdet}}
\def\k{\kappa}
\def\RR{\Bbb R}
\def\be{\bold e}
\def\bR{\overline{R}}
\def\tR{\tilde{\Cal R}}
\def\hY{\hat Y}
\def\tDY{\widetilde{DY}(\g)}
\def\R{\Bbb R}
\def\h1{\hat{\bold 1}}
\def\hV{\hat V}
\def\deg{\text{deg}}
\def\hz{\hat \z}
\def\hV{\hat V}
\def\Uz{U_h(\g_\z)}
\def\Uzi{U_h(\g_{\z,\infty})}
\def\Uhz{U_h(\g_{\hz_i})}
\def\Uhzi{U_h(\g_{\hz_i,\infty})}
\def\tUz{U_h(\tg_\z)}
\def\tUzi{U_h(\tg_{\z,\infty})}
\def\tUhz{U_h(\tg_{\hz_i})}
\def\tUhzi{U_h(\tg_{\hz_i,\infty})}
\def\hUz{U_h(\hg_\z)}
\def\hUzi{U_h(\hg_{\z,\infty})}
\def\Uoz{U_h(\g^0_\z)}
\def\Uozi{U_h(\g^0_{\z,\infty})}
\def\Uohz{U_h(\g^0_{\hz_i})}
\def\Uohzi{U_h(\g^0_{\hz_i,\infty})}
\def\tUoz{U_h(\tg^0_\z)}
\def\tUozi{U_h(\tg^0_{\z,\infty})}
\def\tUohz{U_h(\tg^0_{\hz_i})}
\def\tUohzi{U_h(\tg^0_{\hz_i,\infty})}
\def\hUoz{U_h(\hg^0_\z)}
\def\hUozi{U_h(\hg^0_{\z,\infty})}
\def\hg{\hat\g}
\def\tg{\tilde\g}
\def\Ind{\text{Ind}}
\def\pF{F^{\prime}}
\def\hR{\hat R}
\def\tF{\tilde F}
\def\tg{\tilde \g}
\def\tG{\tilde G}
\def\hF{\hat F}
\def\bg{\overline{\g}}
\def\bG{\overline{G}}
\def\Spec{\text{Spec}}
\def\tlo{\hat\otimes}
\def\hgr{\hat Gr}
\def\tio{\tilde\otimes}
\def\ho{\hat\otimes}
\def\ad{\text{ad}}
\def\Hom{\text{Hom}}
\def\hh{\hat\h}
\def\a{\frak a}
\def\t{\hat t}
\def\Ua{U_q(\tilde\g)}
\def\U2{{\Ua}_2}
\def\g{\frak g}
\def\n{\frak n}
\def\hh{\frak h}
\def\sltwo{\frak s\frak l _2 }
\def\Z{\Bbb Z}
\def\C{\Bbb C}
\def\d{\partial}
\def\i{\text{i}}
\def\ghat{\hat\frak g}
\def\gtwisted{\hat{\frak g}_{\gamma}}
\def\gtilde{\tilde{\frak g}_{\gamma}}
\def\Tr{\text{\rm Tr}}
\def\l{\lambda}
\def\I{I_{\l,\nu,-g}(V)}
\def\z{\bold z}
\def\Id{\text{Id}}
\def\<{\langle}
\def\>{\rangle}
\def\o{\otimes}
\def\e{\varepsilon}
\def\RE{\text{Re}}
\def\Ug{U_q({\frak g})}
\def\Id{\text{Id}}
\def\End{\text{End}}
\def\gg{\tilde\g}
\def\b{\frak b}
\def\S{\Cal S}
\def\L{\Lambda}

\topmatter
\title On integration in a space with complex dimension
\endtitle
\author {\rm {\bf Pavel Etingof} }\endauthor
\endtopmatter

These are comments on dimensional regularization of Feynman integrals, which 
was discussed in the lectures of K.Gawedzki and D.Gross. 
This material can be found in physical literature, but in a slightly 
different form. 

We will first define a certain remarkable operator 
(the D-dimensional integral), which acts on functions of nonnegative 
symmetric bilinear forms, and then show how it applies to dimensional 
regularization.  

{\bf Acknowledgement: } I am grateful to Pierre Deligne,
who helped me improve this text. 

{\bf 1. The $D$-dimensional integral.}

Let $E$ be a real vector space of dimension $N$
with a volume element $dv_E$. Denote by 
$S^2E^*$ the space of symmetric bilinear forms on $E$, and by
$S^2_+E^*\subset S^2E^*$ the set of nonnegative symmetric bilinear forms. 
We think of elements of $S^2E^*$ as of selfadjoint operators 
$E\to E^*$. For any such operator $A$, we can canonically define 
its determinant $\det(A)\in \Bbb R$.  

Let $dA$ be a 
volume element on $S^2E^*$ defined by $dv_E$. 
That is: identify $E^*$ with $E$ 
by any positive operator $X:E\to E^*$ with deteminant $1$,
regard $A\in S^2E^*$ as a selfadjoint operator on $E$
using this identification,
 and define $dA$ to be the volume element on $S^2E^*$ corresponding 
to the norm $||A||^2=\Tr(A^2)$. (It is clear that this definition 
does not depend on $X$). 

Fix a complex number $D$ with $\Re D>N-1$. 

\proclaim{Lemma 1} For any $B\in S^2E$ such that $B>0$
$$
\gather
\int_{S^2_+E^*}(\det A)^{{(D-N-1)/2}}e^{-\tr(AB)}dA=
\Gamma_N(D/2)(\det B)^{-D/2},\\
\Gamma_N(D/2)=
(2\pi)^{N(N-1)/4}\prod_{l=0}^{N-1}\Gamma\biggl(\frac{D+l}{2}
\biggr).\tag 1
\endgather
$$
\endproclaim

\demo{Proof} 
Denote this integral by $f(B)$.
It is easy to see that for any $S\in GL(E)$,
$f(S^*BS)=f(B)|\det S|^{-D}$. 
{}From this it follows that $f(B)=C(D)(\det B)^{-D/2}$. 

It remains to compute $C(D)$. Let us 
pick $B$ such that $\det B=1$, and identify $E^*$ with $E$ 
using the isomorphism $B:E^*\to E$. Then $E$ becomes a 
Euclidean space with the inner product $B^{-1}$. 
Thus the constant $C(D)$ has the form
$$
\int_{A:E\to E,A^*=A,A\ge 0}(\det A)^{(D-N-1)/2}e^{-\tr A}dA
$$
Since this integral is invariant 
under conjugation by $SO(E)$, we can reduce it to the quotient
space $S^2_+E^*/SO(E)=[0,\infty)^N/S_N$.
It is known from the theory 
of symmetric spaces 
that the pushforward of $dA$
from $S^2_+E^*$ to $S^2_+E^*/SO(E)$ has the form 
 $b_N\prod_{1\le i<j\le N}|s_i-s_j|
ds_1...ds_N$, where $s_i$ are the eigenvalues of $A\in S^2_+E^*/SO(E)$, and
$$
b_N=\frac{2^{N(N-1)/4}\pi^{N(N+1)/4}}{\prod_{l=1}^N\Gamma(l/2)},
$$

Thus, $C(D)$ has the form
$$
b_N N!^{-1}\int_{[0,\infty)^N}\prod_i t_i^{{(D-N-1)/2}}\prod_i e^{-t_i}
\prod_{1\le i<j\le N}|t_i-t_j|
dt_1...dt_N.\tag 2
$$

To compute this integral, we use (a special case of) Selberg's integral 
formula:
$$
\gather
\int_{0\le t_i\le 1}\prod_i t_i^a\prod_i (1-t_i)^b
\prod_{1\le i<j\le N}|t_i-t_j|
dt_1...dt_N=\\
\prod_{l=1}^N\frac{\Gamma(a+\frac{l+1}{2})\Gamma(b+\frac{l+1}{2})
\Gamma(\frac{l}{2}+1)}{\Gamma(a+b+\frac{2N-l+3}{2})\Gamma(\frac{3}{2})}.
\tag 3\endgather 
$$
Let us set $s_i=bt_i$ in Selberg's integral, 
and send $b$ to $+\infty$. In the limit, using Stirling's formula, 
and setting $a={(D-N-1)/2}$, we get
$$
\gather
C(D)=b_N 
N!^{-1}\prod_{l=1}^N\frac{\Gamma(\frac{D+l-N}{2})\Gamma(\frac{l}{2}+1)}
{\Gamma(\frac{3}{2})}=\\ b_N\pi^{-N/2}
\prod_{l=1}^N\Gamma\biggl(\frac{D+l-N}{2}\biggr)
\Gamma\biggl(\frac{l}{2}\biggr)=\Gamma_N(D/2).
.\tag 4
\endgather
$$
The Lemma is proved. 
\enddemo

Now consider the function $\rho^E_D$ on $S^2E^*$ which is
supported at $S^2_+E^*$ and is given by the formula
$$
\rho^E_D(A)=
 \pi^{ND/2}\Gamma_N(D/2)^{-1}
(\det A)^{{(D-N-1)/2}}.\tag 5
$$
By Lemma 1, we have
$$
\int_{S^2_+E^*}\rho^E_D(A)e^{-\tr(AB)}dA=\pi^{ND/2}(\det B)^{-D/2},\tag 6
$$
$\Re D>N-1$.

If $\Re D>N-1$, the function $\rho^E_D(A)$ is a locally $L^1$-function, so 
it defines a distribution on $S^2E^*$.  
Let us compute the Fourier transform of $\rho^E_D(A)$, i.e. 
$$
\hat \rho^E_D(B)=\int_{S^2E^*}\rho^E_D(A)e^{-i\tr(AB)}dA,
$$
$B\in S^2E$ (the integral is understood in the sense of distributions).

\proclaim{Lemma 2} 
$$
\hat \rho^E_D(B)=
\lim_{J\to +0}\pi^{ND/2}\det (iB+J)^{-D/2}.\tag 7
$$
where $J\in S^2E^*,J>0$.
\endproclaim

{\bf Remark.} Here we use the branch of the function $z^{-D/2}$ 
which is equal to $e^{-D\ln z/2}$ for real positive $z$. 

\demo{Proof} By analytic continuation, formula (6) holds 
for arbitrary $B\in (S^2E^*)_\C$ with $\Re B>0$.
In particular, we can replace $B$ with $iB+J$,
where $B\in S^2E^*$. In the limit when $J\to +0$, we obtain formula (7).
\enddemo

Let $\mu: S^2E\to \Hom(E^*,E)$ be the tautological embedding. 
Let $\Delta$ 
be the polynomial of degree $N$ on $S^2E$,
defined by the equation $\Delta(B)=\det(\mu(B))$. We regard $\Delta$ as a 
differential operator of order $N$ on $S^2E^*$
with constant coefficients.  

\proclaim{Corollary 3} If $\Re D>N+1$, then 
$$
\Delta \rho^E_D(A)=\pi^N\rho^E_{D-2}(A).\tag 8
$$
\endproclaim

\demo{Proof} The proof is obvious after passing to Fourier 
transforms.
\enddemo

Let $\Cal S(S^2E^*)$ be the space of complex-valued Schwarz 
functions on $S^2E^*$. 
For any $f\in \Cal S(S^2E^*)$ and $\Re D>N-1$, define
$$
I_D^E(f)=\int_{S^2E^*}\rho^E_D(A)f(A)dA.\tag 9
$$

\proclaim{Corollary 4} 
$I_D^E(f)$ extends to an entire function of $D$.
\endproclaim

\demo{Proof} For $Re D>N-1-2K$, $K\in \Z_+$, we can define 
$I_D^E(f)$ by the formula
$$
I_D^E(f)=(-\pi)^{-KN}\int_{S^2E^*}\rho^E_{D+2K}(A)\Delta^Kf(A)dA.\tag 10
$$
According to Corollary 3, if $\Re D>>0$, formula (10) defines the same
function as (9). Thus, (10) defines an analytic continuation of $I_D^E$. 
\enddemo

\proclaim{Proposition 5} Let $D\ge 0$ be a nonnegative integer, 
and $V$ be a vector space of dimension $D$, 
with a positive definite symmetric bilinear form $\beta$. 
Then for any $f\in \Cal S(S^2E^*)$ 
one has
$$
I_D^E(f)=\int_{\Hom(E,V)}f(x^*(\beta))dx,\tag 11
$$
where $x^*(\beta)$ denotes the inverse image of $\beta$ under $x$
(a nonnegative symmetric bilinear form on $E$).  
\endproclaim

\demo{Proof} It is enough to check
that (11) holds for functions $e^{-\tr(AB)}1_{S^2_+E^*}(A)$
where $1_Y$ is the characteristic function of the set $Y$, and
$\text{Re}(B)>0$. This follows from the fact 
that the function $f(A)$ admits a Fourier representation. 
 
In the case 
$$
f(A)=e^{-\tr(AB)}1_{S^2_+E^*}(A),\tag 12
$$
 evaluating the Gaussian integral 
on the right hand side of (11), we get $\pi^{ND/2}(\det B)^{-D/2}$.
On the left hand side of (11), by Lemma 1, we get the same. 
The proposition is proved. 
\enddemo

{\bf Remark.} From Proposition 5 it is clear why formula (9) works only for
$\Re D>N-1$. Indeed, for integer $D$ between $0$ and $N-1$ 
the distribution $\rho_D(A)dA$ is concentrated on forms of
rank $\le D$, so it is a purely singular distribution. 
Thus, the measure $\rho_D(A)dA$ is not absolutely continuous with respect
to $dA$ and hence cannot be represented as a function times $dA$. 
In fact, it turns out that this distribution is singular
(not locally $L^1$) 
for all $D$ with $\Re D\le N-1$. 

Proposition 5 motivates the following definition.

\proclaim{Definition} $I_D^E(f)$ is called the D-dimenional
integral of $f$.
\endproclaim

Another way of computing the D-dimensional integral is using 
Fourier transforms. By Plancherel formula, 
it follows from (9) that 
$$
I_D^E(f)=(2\pi)^{-N(N+1)/2}
\int_{S^2E^*}\hat f(B)\overline{\hat\rho^E_{\bar D}(B)}dB,\tag 13
$$
where $\hat f$ is the Fourier transform of $f$.
(It is clear that
$\overline
{\hat\rho^E_{\bar D}(B)}=\lim_{J\to +0}\pi^{ND/2}\det(-iB+J)^{-D/2}$).  

For $\Re D<2$, this formula can be written as
$$
I_D^E(f)= 2^{-N(N+1)/2}\pi^{N(D-N-1)/2}
\int_{S^2E^*}\hat f(B)e^{\frac{\pi iD}{4}\sigma(B)}|\det (B)|^{-D/2}dB,\tag 14
$$
where $\sigma(B)$ is the signature of $B$ (the number of pluses 
minus the number of minuses). 
If $\Re D\ge 2$, the distribution $\overline{\hat\rho^E_{\bar D}(B)}$ 
is singular, so in order to define $I_D^E(f)$ by a direct integration
formula, we need to use the operator $\Delta$. 
Namely, from Lemma 1 it follows that
$$
\Delta \hat \rho^E_D(B)=(-1)^Np_N(D)
\hat\rho^E_{D+2}(B),\ \  p_N(D)=\pi^{-N}\prod_{l=1}^N\frac{D+l-N}{2}\tag 15
$$ 
So if $\Re D<2+2K$, we can define $I_D^E$ by the formula
$$
\gather
I_D^E(f)=2^{-N(N+1)/2}\pi^{N(D-N-1)/2}\prod_{J=1}^Kp_N(D-2J)^{-1}\times\\
\int_{S^2E^*}\Delta^K\hat f(B)
e^{\frac{\pi i(D-2K)}{4}\sigma(B)}|\det (B)|^{(-D+2K)/2}dB,
\tag 16\endgather
$$

{\bf Example.} If $N=1$, the D-dimensional integral of $f$ is 
(up to normalization) just the Mellin transform of $f$:
$$
I_D^E(f)=\pi^{D/2}\Gamma(D/2)^{-1}\int_{0}^\infty x^{\frac{D}{2}-1}f(x)dx.
$$

Let $\Cal S(S^2_+E^*)$ be the space of smooth functions $f$ on 
$S^2_+E^*$, such that all the derivatives of $f$ are rapidly decaying.
It is easy to see that the restriction map $\Cal S(S^2E^*)\to
\Cal S(S^2_+E^*)$ is surjective. So, $I_D^E$ defines a linear functional
$\Cal S(S^2_+E^*)\to \C$. 

{\bf 2. D-dimensional integral with parameters.}

Now we will extend our construction to a more general situation.
Let $F\subset E$ be a subspace of dimension $M$
with a volume element $dv_F$.
Any symmetric bilinear form
$A$ on $E$ defines a form $A_F$ on $F$ (the restriction of $E$).

Also, for any 
symmetric bilinear form $B$ on $E^*$ 
such that $B_{F^\perp}$ is nondegenerate
we can define a form $B^{F^*}$ on $F^*$ as follows.  
Let $(F^\perp)^{\perp_B}$ be the B-orthogonal complement of 
$F^\perp$. As $B_{F^\perp}$ is nondegenerate, 
this space is naturally identified with $F^*=E^*/F^\perp$. 
By the definition, the form $B^{F^*}$ is the restriction of $B$ to
$(F^\perp)^{\perp_B}=F^*$.

Let us find an explicit expression of $B^{F^*}$. Choose 
a positive definite inner product on $E$ and thus identify
$E^*$ with $E$. Then we have a decomposition 
$E=F\oplus F^\perp$, and $B$ becomes a selfadjoint operator
$E\to E$ with a decomposition
$$
B=\biggl(\matrix B_{11}&B_{12}\\ B_{21}&B_{22}\endmatrix\biggr),
$$
where $B_{ij}^*=B_{ji}$. Now the operator 
corresponding to $B_{F^\perp}$ is $B_{22}$, and the 
operator corresponding to $B^{F^*}$ is 
$B_{11}-B_{12}B_{22}^{-1}B_{21}$. From this formula
it is obvious that for this operator 
to be defined, it is sufficient for $B_{22}$ to be invertible. 
 
The following Lemma is a generalization of Lemma 1. 

\proclaim{Lemma 6} For $\Re D>N-1$ and any positive $B\in S^2E$, 
$C\in S^2F^*$
$$
\gather
\int_{S^2_+E^*}\delta(A_F-C)(\det A)^{
(D-N-1)/2}
e^{-\tr(AB)}dA=\\
\frac{\Gamma_N(D/2)}{\Gamma_M(D/2)}e^{-\tr(CB^{F^*})}
(\det C)^{(D-M-1)/2}(\det B_{F^\perp})^{-D/2},
\tag 17\endgather 
$$
\endproclaim

{\bf Remark.} The space $S^2F^*$ inherits a volume element from $F$,
so the delta-function $\delta(A_F-C)$ is well defined. 

\demo{Proof} Denote the left hand side of (17) by $I(B,C)$. 
Let $S\in GL(E)$ be an automorphism which preserves $F$.
Make a change of variable $A\to S^*AS$. Then 
$A_F\to S_F^*A_FS_F$, where $S_F$ is the restriction of $S$ to $F$.
Thus we get
$$
I(B,C)=|\det S|^{N-M}I(SBS^*,(S_F^*)^{-1}CS_F^{-1}).\tag 18
$$
Now choose $B$ with $\det B_{F^\perp}=1$, and 
identify $E^*$ with $E$ using the map $B:E\to E^*$.
Then $E$ becomes a 
Euclidean space with the inner product $B^{-1}$.
Now we can identify $F^*$ with $F$ using the map $B^{F^*}: F^*\to F$,
thus $F$ is also a Euclidean space. 

After these identifications, integral (17) takes the form
$$
\int_{A\in S^2_+E^*}\delta(A_F-C)(\det A)^{
(D-N-1)/2}
e^{-\tr A}dA.\tag 19 
$$
Using the decomposition $E=F\oplus F^\perp$, 
we can write $A$ in the form
$$
A=\biggl(\matrix A_{11}&A_{12}\\ A_{21}&A_{22}\endmatrix\biggr),\tag 20
$$
where $A_{ij}^*=A_{ji}$. 
This allows us to rewrite (19) as
$$
\int_{A\in S^2_+E^*:A_{11}=C}(\det A)^{
(D-N-1)/2}
e^{-\tr A}dA.\tag 21
$$
To compute this integral, 
let us use the identity $\det A=\det A_{11}\det (A_{22}-
A_{21}A_{11}^{-1}A_{12})$. Since we are integrating 
over such $A$ that $A_{11}=C$, we get 
$\det A=\det C\det (A_{22}-A_{21}C^{-1}A_{12})$.
After the change of variable
$X=A_{22}-A_{21}C^{-1}A_{12}$, 
integral (21) takes the form
$$
(\det C)^{(D-N-1)/2}e^{-\tr C}\int_{X\ge 0}(\det X)^{(D-N-1)/2}
e^{-\tr X}dX\int_{\Hom(F^\perp,F)}e^{-\tr A_{12}^*C^{-1}A_{12}}dA_{12}
\tag 22
$$
The second integral in (22) is Gaussian and equals 
$$
(2\pi)^{M(N-M)/2}(\det C)^{(M-N)/2}.\tag 23
$$
 The first integral in (22) 
was computed in Lemma 1 and equals $\Gamma_{N-M}((D-M)/2)$. 
Using the identity 
$$
\frac{\Gamma_N(D/2)}{\Gamma_M(D/2)}=
(2\pi)^{M(N-M)/2}\Gamma_{N-M}((D-M)/2).\tag 24
$$
 we obtain the statement 
of the Lemma. 
\enddemo

Now consider the distribution on $S^2E^*$ (depending on $C$ as a parameter)
$$
\gather
\rho_D^{EF}(A,C)=\delta(A_F-C)\frac{\rho_D^E(A)}{\rho_D^F(C)}=\\
\pi^{(N-M)D/2}\frac{\Gamma_M(D/2)}{\Gamma_N(D/2)}
\delta(A_F-C)(\det C)^{-(D-M-1)/2}(\det A)^{
(D-N-1)/2}, \tag 25\endgather
$$
for $\Re D>N-1$. 

For $f\in \Cal S(S^2_+E^*)$, define
a function $I_D^{EF}(f)$ 
on $C>0$, by setting
$$
I_D^{EF}(f)(C)=\int_{S^2_+E^*}\rho_D^{EF}(A,C)f(A)dA.\tag 26
$$
Since $(\det C)^{-1}\le (\det A)^{-1} (\tr A)^{N-M}$, 
the function $I_D^{EF}(f)$ extends continuously to 
$C\ge 0$, i.e. defines a continuous, rapidly decaying function 
on $S^2_+F^*$. 

{\bf Remark.} Although the power of $(\det C)^{-1}$ in (25) 
is more than that of $\det A$, for small $\det C$ the 
intersection of the region of integration in (26) 
with any fixed ball has area $\sim (\det C)^{(N-M)/2}$ (because 
of presence of the delta-function), which helps compensate 
the negative exponent of $\det C$. 
 
Using Lemma 6, it is easy to compute the Fourier transform 
$\hat\rho_D^{EF}(B,C)$ of $\rho_D^{EF}(A,C)$ 
with respect to $A$.

\proclaim{Lemma 7}  
$$
\hat \rho_D^{EF}(B,C)=
\lim_{J\to +0}\pi^{(N-M)D/2}e^{-i\tr (CB^{F^*})}
\det (iB_{F^\perp}+J)^{-D/2}.\tag 27
$$
\endproclaim

\demo{Proof} Analogous to the proof of Lemma 2.
\enddemo

Let $\Delta_{2}$ be the operator $\Delta$ 
acting on $\Cal S(S^2F^\perp)$.

\proclaim{Corollary 8} If $\Re D>N+1$, then 
$$
\Delta_2 \rho_D^{EF}(A,C)=\pi^N\rho_{D-2}^{EF}(A,C).\tag 28
$$
(here $\Delta_2$ acts on $A$). 
\endproclaim

\demo{Proof} The proof is immediate after passing to Fourier 
transforms.
\enddemo

\proclaim{Corollary 9} For any $f\in \Cal S(S^2_+E^*)$, $I_D^{EF}(f)$ 
is a Schwarz function of $C$, 
and for any $C\ge 0$
the function $I_D^{EF}(f)(C)$, regarded as a function
 of $D$, analytically continues
to an entire function. 
\endproclaim

\demo{Proof} Similar to the Proof of Corollary 4. 
\enddemo
 
\proclaim{Proposition 10} 
 Let $D\ge 0$ be a nonnegative integer, 
and $V$ be a vector space of dimension $D$
with a positive definite symmetric bilinear form $\beta$. 
Then for any $f\in \Cal S(S^2_+E^*)$ and $y:F\to V$
one has
$$
I_D^{EF}(f)(y^*(\beta))=\int_{x\in \Hom(E,V):x|_F=y}
f(x^*(\beta))dx.\tag 29
$$
\endproclaim

\demo{Proof} As in Proposition 5, it is enough to consider $f(A)=
e^{-\tr(AB)}$. In this case, on the r.h.s. of 
(19) is a Gaussian integral, whose value is 
$$
\pi^{(N-M)D/2}e^{-\tr (y^*(\beta)B^{F^*})}
(\det B_{F^\perp})^{-D/2}.
$$
Thus, the Proposition follows from Lemma 6.
\enddemo

\proclaim{Definition} $I_D^{EF}$ is called the D-dimensional 
integral with external parameters. 
\endproclaim

As before, the D-dimensional integral can be computed using Fourier 
transform. 

Let $f\in \Cal S(S^2E^*)$.
By Plancherel formula, 
$$
I_D^{EF}(f)(C)=(2\pi)^{-N(N+1)/2}
\int_{S^2E^*}\hat f(B)\overline{\hat\rho_{\bar D}^{EF}(B,C)}dB,\tag 30
$$
where $\hat f$ is the Fourier transform of $f$.
Here
$$
\overline
{\hat\rho_{\bar D}^{EF}(B,C)}=\lim_{J\to +0}\pi^{(N-M)D/2}
e^{i\tr (CB^{F^*})}\det(-iB_{F^\perp}+J)^{-D/2}.
$$  

For $\Re D<2$, this formula can be written as
$$
\gather
I_D^{EF}(f)(C)= \\
2^{-N(N+1)/2}\pi^{((N-M)D-N(N+1))/2}
\int_{S^2E^*}\hat f(B)e^{\frac{\pi iD}{4}\sigma(B_{F^\perp})}
e^{i\tr (CB^{F^*})}
|\det (B_{F^\perp})|^{-D/2}dB.\tag 31\endgather
$$
If $\Re D\ge 2$, the distribution $\overline{\hat\rho_{\bar D}^{EF}(B,C)}$ 
is singular, so in order to define $I_D^{EF}(f)$ by a direct integration
formula, we need to use the operator $\Delta$. 
Namely, from Lemma 6 it follows that
$$
\gather
(\det C)^{-1}\Delta \hat \rho^{EF}_D(B,C)=(-1)^{N-M}p_{NM}(D)
\rho_{D+2}^{EF}(B),\\
 p_{NM}(D)=\pi^{M-N}\prod_{l=M+1}^N\frac{D+l-N}{2}\tag 32
\endgather
$$ 
So if $\Re D<2+2K$, we can define $I_D^{EF}$ by the formula
$$
\gather
I_D^{EF}(f)(C)=2^{-N(N+1)/2}\pi^{((N-M)D-N(N+1))/2}(\det C)^{-K}
\prod_{J=1}^Kp_N(D-2J)^{-1}\times\\
\int_{S^2E^*}\Delta^K\hat f(B)
e^{i\tr (CB^{F^*})}
e^{\frac{\pi i(D-2K)}{4}\sigma(B)}
|\det (B)|^{(-D+2K)/2}dB.
\endgather
$$

Thus we have defined an operator $I_D^{EF}: \Cal S(S^2_+E^*)\to \Cal S(S^2_+F^*)$. 
It is clear that $I_D^{FG}I_D^{EF}=I_D^{EG}$ if $E\supset F\supset G$,
and $I_D^{E0}=I_D^E$.


Now we can define the
 D-dimensional integral of functions with polynomial growth.
Let $f$ be a smooth function on $S^2E^*$ such that 
there exists $a\in \R$ for which $f$ and all its derivatives
satisfy the condition
$\phi(A) =O(||A||^a)$, $||A||\to \infty$
(for any norm on $S^2E^*$). 
In this case, the integral $I_D^{EF}(f)$ is defined 
for $\Re D<<0$. Indeed, 
the condition on $f$ implies that
the Fourier transform $\hat f$ is a distribution defined 
on functions with sufficiently many 
continuous derivatives having polynomial growth 
at infinity. Thus, we can use formula (30)
to define $I_D^{EF}(f)$ for $\Re D<<0$.

{\bf Remark.} The operator $I_D^{EF}$ has rather strange properties.
For example, if $P$ is any polynomial, then
$I_D^E(P(A))=0$. Indeed, the Fourier
transform of $P$ is a distribution supported at $0$, 
so for sufficiently negative $D$ formula (14) gives $0$. 
Thus, $I_D^E$ is not continuous in any topology in which polynomials 
are dense. 

{\bf Example.} Let $N=1$, $M=0$, and $f$ be a rational function, regular
at $x\ge 0$. Write $f$ as a sum of simple fractions:
$f(x)=f_0(x)+\sum_i\alpha_i(x+a_i)^{-n_i}$, where 
$f_0$ is a polynomial, $a_i\in\Bbb C\setminus
\Bbb R_-$, and $n_i$ are positive integers.
Then
$$
I_D^E(f)=\pi^{D/2}\sum_i\alpha_i\frac{\Gamma(n_i-D/2)}{\Gamma(n_i)}
a_i^{n_i-\frac{D}{2}}.
$$
 
Thus we see that if $f$ is satisfies the above condition
of polynomial growth then $I_D^{EF}(f)(C)$ is
holomorphic in $D$ for $\Re D<<0$. 
In general, 
we cannot expect that 
this function will analytically continue to the 
whole complex plane. 
However, for functions of a certain algebraic nature, it extends 
meromorphically to the whole complex plane.  

{\bf 3. D-dimensional integral of functions arising from Feynman diagrams.}

\proclaim{Definition} A function $f$ on $S^2_+E^*$ is called a function 
of Feynman type if it has the form
$$
f(A)=\frac{P(A)}{\prod_{j=1}^l(\tr (AB_j)+m_j^2)}, 
$$
where $P$ is a polynomial, $B_j\ge 0$, 
such that $(\sum t_iB_i)_{F^\perp}$ is nondegenerate 
for $t_i>0$, and $m_j\ne 0$
are real numbers.
\endproclaim

\proclaim{Proposition 11} If $f$ is of Feynman type then 
$I_D^{EF}(f)$ extends meromorphically to the whole complex plane. 
\endproclaim

\demo{Proof} We will prove the statement for 
$P=1$. For general $P$, the proof is similar to the case $P=1$. 

Using the formula $b^{-1}=\int_0^\infty e^{-tb}dt$, 
we can rewrite $I_D^{EF}(f)$ in the form
$$
I_D^{EF}(f)=\int_{t_1,...,t_l\in [0,\infty)}
e^{-\sum t_jm_j^2}I_D^{EF}
( e^{-\tr(A(\sum t_jB_j))})dt.\tag 33
$$
Set $B(t)=\sum t_jB_j$. 
Using Lemma 6, we can rewrite this in the form
$$ 
I_D^{EF}(f)(C)=\pi^{(N-M)/2}\int_{t_1,...,t_l\in [0,\infty)}
e^{-\sum t_jm_j^2-\tr(CB^{F^*}(t))}
(\det B_{F^\perp}(t))^{-D/2}dt.\tag 34
$$

Now we will use the following theorem of Bernstein, which is proved
with the help of the theory of D-modules. 

{\bf Bernstein's theorem.} Let $Q(t)$ be any polynomial in $l$ variables. 
Then there exists a polynomial differential operator 
$L(D)$ in $l$ variables, 
with coefficients depending polynomially on $D$, and a 
polynomial $q$ in $D$, such that $L(D)Q^{-D/2}=q(D)Q^{-1-D/2}$.

This theorem implies the following statement, which is 
a special case of a result of Bernstein.

{\bf Corollary.} Assume that $Q$ takes positive values for $t_j>0$, and
let $g$ be a rapidly decaying smooth function defined for $t_i\ge 0$,
such that its derivatives are also rapidly decaying.  
Then the integral
$$
I(D,g)=\int_{t_i\in [0,\infty)}g(t)Q(t)^{-D/2}dt,\tag 35
$$
convergent for $\Re D<< 0$, extends meromorphically to the whole
complex plane. 

The corollary is proved by induction in $l$ (the number of $t_j$).
The base of  induction is trivial. Now,
using Bernstein's theorem and integration by parts,
we can write
$$
\gather
I(D+2,g)=q(D)^{-1}\int g(t)L(D)(Q(t)^{-D/2})dt=
\\ q(D)^{-1}\int (L(D)^*g(t))Q(t)^{-D/2}dt+
C(D)=
q(D)^{-1}I(D,L(D)^*g)+C(D),\tag 36
\endgather
$$
where $C(D)$ is the integral over the boundary, which is meromorphic 
by the assumption of induction. 

Now we apply the corollary to our situation, and obtain the proof 
of the Proposition for $P=1$. 
\enddemo

{\bf Remark 1.} Using Hironaka's theorem about the existence 
of a smooth resolution of singularities for a compact projective variety,
it is possible to prove that $I_D^{EF}$ is meromorphic in $D$ in a much
more general situation than in Proposition 11. 
For example, as was explained to me by M.Kontsevich,
one can prove that if $f$ is a rational 
function without singularities in $S^2_+E^*$, 
then $I_D^{EF}(f)$ is meromorphic. 

It is also possible to show that for rational $f$ as
above the poles of $I_D^{EF}(f)$ are all rational, and 
belong to a finite number of arithmetic progressions. 
 
{\bf 4. Dimensional regularization of Feynman integrals.} 

Proposition 11 allows to define the procedure of dimensional regularization 
of Feynman integrals in quantum field theory. 

Suppose that we have a quantum field theory given by some Lagrangian,
and we want to compute its correlation functions. 
According to Feynman rules, such a correlation function 
equals to the sum of amplitudes of all Feynman diagrams (graphs),
which are given by some finite-dimensional integrals. 

We will work in the Euclidean setting and in momentum space. 
For simplicity we will assume that our theory contains only  
scalar massive fields. 
Then the amplitude of any Feynman graph $\Gamma$ 
is given by the formula
$$
\int_{x\in\Hom(E,V):x|_F=y}g(x)dx, y\in \Hom(F,V),\tag 37
$$
where $V$ is the momentum space 
(a Euclidean space
of dimension $d$ with inner product $\beta$),
$E=H^1_c(\Gamma\setminus\d\Gamma,\Bbb R)$
(the cohomology with compact supports), 
$F$ is the image of the map $H^0(\d\Gamma,\Bbb R)\to
H^1_c(\Gamma\setminus\d\Gamma,\Bbb R)$ in the long exact sequence
of cohomology, and 
$g$ is a function on $\Hom(E,V)$ of the form
$g(z)=f(z^*(\beta))$, 
where $f$ is a certain function of Feynman type
attached to $\Gamma$.

{\bf Remark.} It is clear 
from the long exact sequence
that $E/F=H^1(\Gamma,\Bbb R)$. 

Since cohomology groups are defined over $\Z$, they have
natural volume elements. This defines the volume element
$dx$ in (37). 

The problem is that integral (37) is often divergent, since the function
$g$ does not decay rapidly enough. 

Dimensional regularization is a method to give meaning to the integral (37).
This is done as follows. 
Recall that since $f$ is of Feynman type, 
$I_D^{EF}(f)$ is meromorphic in $D$.

\proclaim{Definition} Dimensional regularization 
of integral (37) is the regular part of
$I_D^{EF}(f)$ at $D=d$, i.e. 
$\frac{1}{2\pi i}\oint_{|D-d|=\e}I_D^{EF}(f)d\ln(D-d)$.  
\endproclaim

{\bf Remark.} In general, we have to consider functions 
$f$ which (even before the regularization) depend meromorphically on $D$. 
This is caused by the fact that our graph $\Gamma$ may contain divergent
proper subgraphs, which have to have been regularized 
before we start working with $\Gamma$. Since regularization
is performed by adding $D$-dependent counterterms, we will have to work
with functions which depend meromorphically on $D$. 
 
{\bf 5. D-dimensional Stokes formula.}

Let $F\subset E$ be vector spaces
with volume elements, $\dim E=N$, $\dim F=M$, and
$D$ be any complex number.
Define the space $\Omega_{EF}^{\top}$ 
of (Schwarz) top degree forms  
to be the space $\Cal S(S^2_+E^*)$, and the space 
$\Omega_{EF}^{\top-1}$ of forms of degree $\top-1$ to be the space
of Schwarz functions 
$\omega: S^2_+E^*\to \Hom(E/F,E)$. That is,
$\Omega_{EF}^{\top}=\Hom(E/F,E)\o \Cal S(S^2_+E^*)$. 

If $D$ is an integer, and $V$ is a Euclidean space of dimension
$D$ with inner product $\beta$, then we have a linear map
$\theta: \Hom(E,V)\to \Hom(F,V)$ given by restriction, and
any $f\in \Omega_{EF}^\top$ indeed defines a top differential
 form $f(x^*(\beta))dx$ on each fiber of $\theta$
($dx$ is the natural volume form on the fiber). 

Also, any $\omega\in\Omega_{EF}^{\top-1}$
defines a $\top-1$-form on the fiber.
Indeed, let $z: (E/F)^*\to V$ be a linear map. 
We regard $z$ as a constant 1-form on the fibers of $\theta$.  
Then the $\top-1$-form corresponding to 
$\Omega$ is defined by the formula 
$(\omega\wedge z)(x)=\tr (\omega \beta(x,z))dx$, 
where $\beta(x,z)\in \Hom(E\o (E/F)^*,\Bbb R)$ 
is regarded as an operator $E\to E/F$. 

Recall that the group $GL(E)$ acts on $S^2_+E^*$ by $A\to S^*AS$, 
$S\in GL(E)$. Thus, any element 
$X$ of the Lie algebra $\End(E)$ of $GL(E)$
defines a vector field on $S^2_+E^*$:
$L_Xf(A)=\frac{d}{dt}|_{t=0}f(e^{X^*t}Ae^{Xt})$. 

For any $\omega=\sum X_i\o f_i\in \Omega_{EF}^{\top -1}$, 
$X_i\in\Hom(E/F,E)$, $f_i\in \Cal S(S^2_+E^*)$, define 
the form $d\omega\in \Omega_{EF}^\top$ by the formula
$$
d\omega=\sum_i(L_{X_i}f_i+D\tr(X_i)f_i),\tag 38
$$
where $X_i$ are regarded as elements of $\End(E)$. 
Thus, we have defined a linear map $d:\Omega_{EF}^{\top-1}\to
\Omega_{EF}^\top$. It is easy to see 
that if $D$ is an integer, $d$ coincides 
with the usual differential. 

\proclaim{Proposition 12} (Stokes formula.) 
$I_D^{EF}(d\omega)=0$ for any 
$\omega\in \Omega_{EF}^{\top-1}$.
\endproclaim

\demo{Proof} It is enough to prove the formula for $\Re D>>0$;
for other values of $D$ the formula follows by analytic continuation. 

For $\Re D>>0$, the formula follows by integration by parts
 from the definition (26) of $I_D^{EF}$, as 
$L_X((\det A)^{(D-N-1)/2}dA)=D\tr(X)(\det A)^{(D-N-1)/2}dA$. 
\enddemo

\end


