\subsection{Conifold theories}
\label{conifolds}

In the correspondence between string theory on $AdS_5 \times S^5$ and
$d=4$ $\cn=4$ SYM theories, some of the most direct checks, such as
protected operator dimensions and the functional form of two- and
three-point functions, are determined by properties of the supergroup
$SU(2,2|4)$.  Many of the normalizations of two- and three-point
functions which have been computed explicitly are protected by
non-renormalization theorems.  And yet, we are inclined to believe that
the correspondence is a fundamental dynamical principle, valid
independent of group theory and the special non-renormalization
properties of ${\cal N}=4$ supersymmetry.

To test this belief we want to consider theories with reduced
supersymmetry.  Orbifold theories \cite{Kachru:1998ys} provide 
interesting examples;
however, as discussed in the previous sections,
it has been shown \cite{Bershadsky:1998mb,Bershadsky:1998cb} that
at large $N$ these theories are a projection of ${\cal N}=4$
super-Yang-Mills theory; in particular many of
their Green's functions are dictated
by the Green's functions of the ${\cal N}=4$ theory.  
The projection involved is onto
states invariant under the group action that defines the orbifold.
Intuitively,
%the key seems to be that 
this similarity with the $\cn=4$ theory arises because
the compact part of the geometry is
still (almost everywhere)
locally $S^5$, just with some global identifications. Therefore, 
to make a more
non-trivial test of models with reduced supersymmetry, we are
more interested in geometries of the form $AdS_5 \times M_5$ where the
compact manifold $M_5$ is not even locally $S^5$.

In fact, such compactifications have a long history in the
supergravity literature: the direct product geometry $AdS_5 \times
M_5$ is known as the Freund-Rubin ansatz \cite{Freund:1980xh}.
The curvature of the anti-de Sitter part of the geometry is supported
by the five-form of type~IIB supergravity.  Because this five-form is
self-dual, $M_5$ must also be an Einstein manifold, but with positive
cosmological constant: rescaling $M_5$ if necessary, we can write
${\cal R}_{\alpha\beta} = 4 g_{\alpha\beta}$.  For simplicity,
we are assuming that only the
five-form and the metric are involved in the solution.

A trivial but useful observation is that five-dimensional Einstein
manifolds with ${\cal R}_{\alpha\beta} = 
4 g_{\alpha\beta}$ are in one-to-one
correspondence with Ricci-flat manifolds $C_6$ whose metric has the
conical form
  \eqn{ConeForm}{
   ds_{C_6}^2 = dr^2 + r^2 ds_{M_5}^2 \ .
  }
 It can be shown that, given any metric of the form \ConeForm, the
ten-dimensional metric
  \eqn{KehagMetric}{
   ds_{10}^2 = \left( 1 + {R^4 \over r^4} \right)^{-1/2} 
     \left( -dt^2 + dx_1^2 + dx_2^2 + dx_3^2 \right) + 
    \left( 1 + {R^4 \over r^4} \right)^{1/2} ds_{C_6}^2
  }
 is a solution of the type IIB supergravity equations, provided one
puts $N$ units of five-form flux through the manifold $M_5$, where
  \eqn{NLRelation}{
   R^4 = {\sqrt{\pi} \over 2} {\kappa N \over \Vol M_5} \ .
  }
 Furthermore, it was shown in \cite{Kehagias:1998gn} that the number of
supersymmetries preserved by the geometry \KehagMetric\ is half the number
that are preserved by its Ricci-flat $R \to 0$ limit.  Preservation of
supersymmetry therefore amounts to the existence of a Killing spinor on
$ds_{C_6}^2$, which would imply that it is a Calabi-Yau metric.  Finally,
the $r \ll R$ limit of \KehagMetric\ is precisely $AdS_5 \times M_5$, and
in that limit the number of preserved supersymmetries doubles.

These facts suggest a useful means of searching for non-trivial
Freund-Rubin geometries: starting with a string vacuum of the form
$\IR^{3,1} \times C_6$, where $C_6$ is Ricci-flat, we locate
a singularity of $C_6$ where the metric locally has the form
\ConeForm, and place a large number of D3-branes at that point.  The
resulting near-horizon Freund-Rubin geometry has the same number of
supersymmetries as the original braneless string geometry.  The
program of searching for and classifying such singularities on
manifolds preserving some supersymmetry was enunciated most completely
in \cite{Morrison:1998cs}.

We will focus our attention on the simplest non-trivial example, which was
worked out in \cite{Klebanov:1998hh}\footnote{Additional aspects and
examples of
conifold theories were discussed in \cite{Uranga:1999vf,Dasgupta:1998su,
Gubser:1999ia,Lopez:1999zf,
vonUnge:1999hc,Erlich:1999rb}.}. 
$C_6$ is taken to be the standard conifold, which
as a complex 3-fold is determined by the equation
  \eqn{ConEq}{
   z_1^2 + z_2^2 + z_3^2 + z_4^2 = 0 \ .
  }
 The Calabi-Yau metric on this manifold has $SU(3)$ holonomy, so one
quarter of supersymmetry is preserved.  We will always count our
supersymmetries in four-dimensional superconformal field theory terms,
so one quarter of maximal supersymmetry (that is, eight real
supercharges) is in our terminology ${\cal N}=1$ supersymmetry
(superconformal symmetry).  The
supergravity literature often refers to this amount of supersymmetry
in five dimensions as ${\cal N}=2$, because in a flat space
supergravity theory with this much supersymmetry, reduction on $S^1$
without breaking any supersymmetry leads to a supergravity theory in
four dimensions with ${\cal N}=2$ supersymmetry.

The Calabi-Yau metric on the manifold \ConEq\ may be derived from 
the K\"ahler
potential $K = \left( \sum_{i=1}^4 |z_i|^2 \right)^{2/3}$, and can be
explicitly written as
  \eqn{ConMet}{
   ds_{C_6}^2 = dr^2 + r^2 ds_{T^{11}}^2,
  }
 where $ds_{T^{11}}^2$ is the Einstein metric on the coset space
  \eqn{TOneOne}{
   T^{11} = {SU(2) \times SU(2) \over U(1)} \ .
  }
In the quotient \TOneOne, the $U(1)$ generator is chosen to be 
the sum $\tf{1}{2}
\sigma_3 + \tf{1}{2} \tau_3$ of generators of 
the two $SU(2)$'s.  The manifolds $T^{pq}$,
where the $U(1)$ generator is chosen to be $\tf{p}{2} \sigma_3 + \tf{q}{2}
\tau_3$, with $p$ and $q$ relatively prime, were studied in
\cite{Romans:1985an}.  The topology of each of these manifolds is $S^2
\times S^3$.  They all admit unique Einstein metrics.  Only $T^{11}$ leads
to a six-manifold $C_6$ which admits Killing spinors.  In fact, besides
$S^5 = SO(6)/SO(5)$, $T^{11}$ is the unique five-dimensional coset space
which preserves supersymmetry.  The Einstein metrics can be obtained via a
rescaling of the Killing metric on $SU(2) \times SU(2)$ by a process
explained in \cite{Romans:1985an}.  The metric on $T^{11}$ satisfying
${\cal R}_{\alpha\beta} = 4 g_{\alpha\beta}$ can be written as
  \eqn{TMet}{
   ds_{T^{11}}^2 = \tf{1}{6} \sum_{i=1}^2 \left( d\theta_i^2 +
    \sin^2 \theta_i d\phi_i^2 \right) + 
    \tf{1}{9} \left( d\psi + \cos\theta_1 d\phi_1 + 
     \cos\theta_2 d\phi_2 \right)^2 \ .
  }
 The volume of this metric is $16\pi^3/27$, whereas the volume of the
unit five-sphere, which also has ${\cal R}_{\alpha\beta} = 4
g_{\alpha\beta}$, is $\pi^3$.

Perhaps the most intuitive way to motivate the conjectured dual gauge
theory \cite{Klebanov:1998hh} is to first consider the $S^5/\IZ_2$
orbifold gauge theory, where the $\IZ_2$ is chosen to flip the signs of
four of the six real coordinates in $\IR^6$, and thus has a fixed $S^1$ on
the unit $S^5$ in this flat space.  This $\IZ_2$ breaks $SO(6)$ down to
$SO(4) \times SO(2)$, which is the same isometry group as for $T^{11}$.  In
fact, it can also be shown that an appropriate blowup of the singularities
along the fixed $S^1$ leads to a manifold of topology $S^2 \times S^3$.
Since $T^{11}$ is a smooth deformation of the blown-up orbifold, one might
suspect that its dual field theory is some deformation of the orbifold's
dual field theory.  The latter field theory is well known
\cite{Kachru:1998ys}, as described in section \ref{orbifolds}.  
It has ${\cal N}=2$ supersymmetry.  The field
content in ${\cal N}=1$ language is
  \eqn{OrbField}{\seqalign{\span\TT \quad & \span\TC \quad & \span\TC}{
   gauge group & SU(N) & SU(N)  \cr
   chirals $A_1$, $A_2$ & \oalign{\idget\endyoung} & 
    \overline{\oalign{\idget\endyoung}}  \cr
   chirals $B_1$, $B_2$ & \overline{\oalign{\idget\endyoung}} &
    \oalign{\idget\endyoung}  \cr
   chiral $\Phi$ & \hbox{adj} & {\bf 1}  \cr
   chiral $\tilde\Phi$ & {\bf 1} & \hbox{adj}.
  }}
 The adjoint chiral fields $\Phi$ and $\tilde{\Phi}$, together with the
${\cal N}=1$ gauge multiplets, fill out ${\cal N}=2$ gauge multiplets.
The chiral multiplets $A_1$, $B_1$ combine to form an ${\cal N}=2$
hypermultiplet, and so do $A_2$, $B_2$.  The superpotential is
dictated by ${\cal N}=2$ supersymmetry:
  \eqn{OrbPot}{
   W = g \tr \Phi (A_1 B_1 + A_2 B_2) + 
       g \tr \tilde\Phi (B_1 A_1 + B_2 A_2) \ ,
  }
where $g$ is the gauge coupling of both $SU(N)$ gauge groups.
 A relevant deformation which preserves the global $SU(2) \times SU(2)
\times U(1)$ symmetry, and also ${\cal N}=1$ supersymmetry, is
  \eqn{DefPot}{
   W \to W + \tf{1}{2} m \left( \tr \Phi^2 - \tr \tilde\Phi^2 \right) \ .
  }
 There is a nontrivial renormalization group flow induced by these mass
terms.  The existence of a non-trivial
infrared fixed point can be demonstrated using
the methods of \cite{Leigh:1995ep}: having integrated out the heavy fields
$\Phi$ and $\tilde\Phi$, the superpotential is quartic in the remaining
fields, which should, therefore, all have dimension $3/4$ at the
infrared fixed point (assuming that we
do not break the symmetry between the two gauge groups).  The anomalous
dimension $\gamma = -1/2$ for the quadratic operators $\tr AB$ is precisely
what is needed to make the exact beta functions vanish.

The IR fixed point of the renormalization group 
described in the previous paragraph is the
candidate for the field theory dual to type IIB string theory on
$AdS_5 \times T^{11}$, or in
weak coupling terms the low-energy field theory of coincident
D3-branes on a conifold singularity.  There are several non-trivial
checks that this is the right theory.  The simplest is to note that
the moduli space of the $N=1$ version of the theory is simply the
conifold.  For $N=1$ the scalar fields $a_i$ and $b_j$ (in the chiral
multiplets $A_i$ and $B_j$) are just
complex-valued.  The moduli space 
%should 
can
be parametrized by the
%gauge-invariant
combinations $a_i b_j$, and if we write
  \eqn{RecoverCon}{
   \pmatrix{z_1 + i z_4 & i z_2 + z_3 \cr
            i z_2 - z_3 & z_1 - i z_4} = 
   \pmatrix{a_1 b_1 & a_1 b_2 \cr a_2 b_1 & a_2 b_2} \ ,
  }
 then we recover the conifold equation \ConEq\ by taking the
determinant of both sides.  In the $N>1$ theories, a slight
generalization of this line of argument leads to the conclusion that
the fully Higgsed phase of the theory, where all the D3-branes are
separated from one another, has for its moduli space the $N^{\rm th}$
symmetric power of the conifold.

The most notable prediction of the renormalization group analysis of the
gauge theory is that the operators $\tr A_i B_j$ should have dimension
$3/2$.  This is something we should be able to see from the dual 
description.  As a
warmup, consider first the ${\cal N}=4$ example.  There, as described
in section \ref{tests}, the lowest
dimension operators have the form $\tr \phi^{(I} \phi^{J)}$, 
and their dimension
is two.  Their description in supergravity is a Weyl deformation of the
$S^5$ part of the geometry with $h^a_a \propto Y^2(y)$, 
where
$h^a_a$ is the trace of the metric on $S^5$ and $Y^2(y)$ is a
$d$-wave spherical harmonic on $S^5$.  The four-form potential
%$a_{\alpha\beta\gamma\delta}$ 
$D_{abcd}$ is also involved in the deformation, and
there are two mass eigenstates in $AdS_5$ which are combinations 
of these two 
fields.  A simple way to
compute $Y^2$ is to start with the function $x_i x_j$ on $\IR^6$ and
restrict it to the unit $S^5$.  This suggests quite a general way to find
eigenfunctions of the Laplacian on an Einstein manifold $M_5$: we start by
looking for harmonic functions on the associated conical geometry
\ConeForm.  The Laplacian is
  \eqn{ConeLaplace}{
   \square_{C_6} = {1 \over r^5} \partial_r r^5 \partial_r + 
    {1 \over r^2} \square_{M_5} \ .
  }
 The operator $r^2 \square_{C_6}$ commutes with $r\partial_r$, so we can
restrict our search to functions $f$ on $C_6$ with $\square_{C_6} f = 0$
and $r\partial_r f = \Delta f$ for some constant $\Delta$.  Such
harmonic functions
restricted to $r=1$ have $\square_{M_5} f\Big|_{r=1} = -\Delta(\Delta+4)
f\Big|_{r=1}$.  Following through the analysis of \cite{Kim:1985ez} one
learns that the mass of the lighter of the two scalars in $AdS_5$
corresponding to $h^a_a \propto f\Big|_{r=1}$ is $m^2 R^2 =
\Delta(\Delta-4)$.  So, the dimension of the corresponding operator is
$\Delta$.  In view of \RecoverCon, all we need to do to verify
in the supergravity approximation
the renormalization group prediction $\Delta = 3/2$ for $\tr A_i B_j$ is to
show that $r \partial_r z_i = \tf{3}{2} z_i$.  This follows from scaling
considerations as follows.  The dilation symmetry on the cone is $r \to
\lambda r$.  Under this dilation, $ds_{C_6}^2 \to \lambda^2 ds_{C_6}^2$.
The K\"ahler form should have this same scaling, and that will follow if also
the K\"ahler potential $K \to \lambda^2 K$.  As mentioned above, the
Calabi-Yau metric follows from $K = \left( \sum_{i=1}^4 |z_i|^2
\right)^{2/3}$, which has the desired scaling if $z_i \to \lambda^{3/2}
z_i$.  Thus, indeed $r\partial_r z_i = \tf{3}{2} z_i$.

It is straightforward to generalize the above line of argument to operators
of the form $\tr A_{(i_1} B^{(j_1} \ldots A_{i_\ell)} B^{j_\ell)}$.
Various aspects of the matching of operators in the conformal field theory
to Kaluza-Klein modes in supergravity have been studied in
\cite{Klebanov:1998hh,Gubser:1999vd,Jatkar:1999zk}.  But there is another
interesting type of color singlet operators, which are called dibaryons
because the color indices of each gauge group are combined using an
antisymmetric tensor. The dibaryon operator is
  \eqn{Dibaryon}{
   \epsilon_{\alpha_1 \ldots \alpha_N} 
    \epsilon^{\beta_1 \ldots \beta_N} 
    A^{\alpha_1}{}_{\beta_1} \ldots A^{\alpha_N}{}_{\beta_N} \ ,
  }
 where we have suppressed $SU(2)$ indices.  Let us use the notation
$SU(2)_A$ for the global symmetry
group under which $A_i$ form a doublet, and $SU(2)_B$ for
the group under which $B_j$ form a doublet.  Clearly, \Dibaryon\ is a
singlet under $SU(2)_B$.  This provides the clue to its string theory dual,
which must also be $SU(2)_B$-symmetric: it is a D3-brane wrapped on
$T^{11}$ along an orbit of $SU(2)_B$ \cite{Gubser:1998fp}.  Using the explicit
metric \TMet, it is straightforward to verify that $mR = \tf{3}{4} N$ in
the test brane approximation.  Up to corrections of order $1/N$, the
mass-dimension relation is $\Delta = mR$, so we see that again the 
field theory
prediction for the anomalous dimension of $A$ is born out. The 3-cycle
which the D3-brane is wrapped on may be shown to be the unique
homologically non-trivial 3-cycle of $T^{11}$. There is also
an anti-dibaryon, schematically $B^N$, which is a D3-brane wrapped on an
orbit of $SU(2)_A$.  The two wrappings are opposite in homology, so the
dibaryon and anti-dibaryon can annihilate to produce mesons.  This
interesting process has never been studied in any detail, no doubt because
the dynamics is complicated and non-supersymmetric.  It is possible to
construct dibaryon operators also in a variety of orbifold theories
\cite{Gubser:1998fp,Gukov:1998kn}.

The gauge theory dual to $T^{11}$ descends via renormalization group flow
from the gauge theory dual to $S^5/\IZ_2$, as described after \DefPot.
The conformal anomaly has been studied extensively for such flows (see for
example \cite{Anselmi:1997am}), and the coefficient $a$ in (\ref{confanom}) is
smaller in the IR than in the UV for every known flow that connects UV 
and IR
fixed points.  
        Cardy has conjectured that this must always be the case
        \cite{Cardy:1988cw}.  To describe the field theoretic attempts
        to prove such a c-theorem would take us too far afield, so
        instead we refer the reader to \cite{Forte:1998dx}
        and references therein.
        In section~\ref{cTheorem} we will demonstrate that a limited c-theorem
        follows from elementary properties of gravity if the AdS/CFT
        correspondence is assumed.

In the presence of ${\cal N}=1$ superconformal invariance, one can compute
the anomaly coefficients $a$ and $c$ in (\ref{confanom}) if one knows $\langle
\partial_\mu R^\mu \rangle_{g_{\mu\nu}, B_\lambda}$, where $R_\mu$ is the
R-current which participates in the superconformal algebra, and the
expectation value is taken in the presence of an arbitrary metric
$g_{\mu\nu}$ and an external gauge field source $B_\mu$ for the R-current.
The reason $a$ and $c$ can be extracted from this anomalous one-point
function is that $\partial_\mu R^\mu$ and $T^\mu_\mu$ are superpartners in
the ${\cal N}=1$ multiplet of anomalies.  It was shown in
\cite{Anselmi:1997am} via a supergroup argument that
  \eqn{acTR}{\eqalign{
   \langle (\partial_\mu R^\mu) T_{\alpha\beta} T_{\gamma\delta} \rangle
     &= (a-c) \big[ \ \big]_{\alpha\beta\gamma\delta}  \cr
   \langle (\partial_\mu R^\mu) R_\alpha R_\beta \rangle
     &= (5a-3c) \big[ \ \big]_{\alpha\beta} \ ,
  }}
 where now the correlators are computed in flat space.  The omitted
expressions between the square brackets are tensors depending on the
positions or momenta of the operators in the correlator.  Their form is not
of interest to us here because it is the same for any theory: we are
interested instead in the coefficients.  These can be computed
perturbatively via the triangle diagrams in figure~\ref{figBssg}.
  \begin{figure}
   \vskip0cm
   \centerline{\psfig{figure=figBssg.eps,width=4in}}
   \vskip-2cm
   \centerline{$\displaystyle{a-c \propto \sum_\psi r(\psi)
     \qquad\qquad\qquad\quad
    5a-3c \propto \sum_\psi r(\psi)^3}$}
   \vskip1.5cm
 \caption{Triangle diagrams for computing the anomalous contribution to
$\partial_\mu R^\mu$.  The sum is over the chiral fermions $\psi$ which run
around the loop, and $r(\psi)$ is the R-charge of each such
fermion.}\label{figBssg}
  \end{figure}
 The Adler-Bardeen theorem guarantees that the one loop result is exact,
provided $\partial_\mu R^\mu$ is non-anomalous in the absence of external
sources (that is, it suffers from no internal anomalies).  The constants of
proportionality in the relations shown in figure~\ref{figBssg} can be
tracked down by comparing the complete Feynman diagram amplitude with the
explicit tensor forms which we have omitted from \acTR.  We are mainly
interested in ratios of central charges between IR and UV fixed points, so
we do not need to go through this exercise.

The field theory dual to $S^5/\IZ_2$, expressed in ${\cal N}=1$
language, has the field content described in \OrbField.  The R-current of
the chosen ${\cal N}=1$ superconformal algebra descends from a $U(1)$ in
the $SO(6)$ R-symmetry group of the ${\cal N}=4$ algebra, and it assigns
a $U(1)_R$ charge
$r(\lambda) = 1$ to the $2N^2$ gauginos (fermionic components of the vector
superfield) and $r(\chi) = -1/3$ to the $6N^2$ ``quarks'' (fermionic
components of the
chiral superfields)\footnote{We will ignore here the distinction
between $U(N)$ and $SU(N)$ groups which is subleading in the $1/N$
expansion.}.  We have $\sum_\psi r(\psi) = 0$, which means that the
R-current has no gravitational anomalies \cite{Alvarez-Gaume:1984ig}.

For the field theory dual to $T^{11}$, the R-current described in the
previous paragraph is no longer non-anomalous because we have added a mass
to the adjoint chiral superfields.  
There is, however, a non-anomalous combination $S_\mu$ of
this current, $R_\mu$, with the Konishi currents, $K^i_\mu$, which by
definition assign charge~$1$ to the fermionic
fields in the $i^{\rm th}$ chiral
multiplet and charge~$0$ to the fermionic fields in the vector multiplets:
  \eqn{SDef}{
   S_\mu = R_\mu + \tf{2}{3} \sum_i 
    \left( \gamma_{\rm IR}^i - \gamma^i \right) K^i_\mu \ .
  }
 Here $\gamma^i$ is the anomalous dimension of the $i^{\rm th}$ chiral
superfield.  At the strongly interacting ${\cal N}=1$ infrared fixed point,
$S_\mu$ is the current which participates in the superconformal algebra.
However, to compute correlators $\langle (\partial_\mu S^\mu) \ldots
\rangle$ it is more convenient to go to the ultraviolet, where $\gamma^i =
0$ and the perturbative analysis in terms of fermions running around a loop
can be applied straightforwardly.  Using the fact that $\gamma_{\rm IR}^A =
\gamma_{\rm IR}^B = -1/4$ and $\gamma_{\rm IR}^\Phi = \gamma_{\rm
IR}^{\tilde{\Phi}} = 1/2$, we find that $s_{\rm UV}(\lambda) = 1$ for the
gauginos, $s_{\rm UV}(\chi) = -1/2$ for the quarks which stay light (i.e.,
the bifundamental quarks), and $s_{\rm UV}(\eta) = 0$ for the quarks which
are made heavy (that is, the adjoint quarks).  Note that it is immaterial
whether we include these heavy quarks in the triangle diagram, which is as
it should be since we can integrate them out explicitly.  As before,
$\sum_\psi s_{\rm UV}(\psi) = 0$, so there are no gravitational
anomalies and $a_{IR} = c_{IR}$.
Combining the information in the past two paragraphs, we have a field
theory prediction for the flow from the $S^5/\IZ_2$ theory to the
$T^{11}$ theory:
  \eqn{ChargeRatio}{
   {a_{\rm IR} \over a_{\rm UV}} = {c_{\rm IR} \over c_{\rm UV}} 
     = {5a_{\rm IR} - 3c_{\rm IR} \over 5a_{\rm UV} - 3c_{\rm UV}}
     = {2N^2 + 4N^2 \left( -{1 \over 2} \right)^3 \over
        2N^2 + 6N^2 \left( -{1 \over 3} \right)^3}
     = {27 \over 32} \ .
  }

This analysis was carried out in \cite{Gubser:1999vd}, where it was also
noted that these numbers can be computed in the supergravity
approximation.  To proceed,
let us write the ten-dimensional Einstein metric as
  \eqn{EinForm}{
   ds_{10}^2 = R^2 \widehat{ds}_5^2 + R^2 ds_{M_5}^2,
  }
 where $R$ is given by \NLRelation\ and $\widehat{ds}_5^2$ is the metric of
$AdS_5$ scaled so that $\hat{\cal R}_{\mu\nu} = -4 \hat{g}_{\mu\nu}$.  We will
refer to $\widehat{ds}_5^2$ as the dimensionless $AdS_5$ metric.  Reducing
the action from ten dimensions to five results in
  \eqn{FiveActForm}{
   S = {\pi^3 R^8 \over 2 \kappa^2} \int d^5 x \, \sqrt{\hat{g}} 
    \left( \hat{\cal R} + 12 + \ldots \right)
     = {\pi^2 N^2 \over 8 \Vol M_5} \int d^5 x \, \sqrt{\hat{g}} 
    \left( \hat{\cal R} + 12 + \ldots \right) \ ,
  }
 where $\sqrt{\hat{g}}$ and $\hat{\cal R}$ under the integral sign refer to the
dimensionless $AdS_5$ metric, and in the second equality we have used
\NLRelation.  In \FiveActForm, $\kappa$ is the ten-dimensional
gravitational coupling.  In computing Green's functions using the
prescription of \cite{Gubser:1998bc,Witten:1998qj}, the prefactor ${\pi^2
N^2 \over 8 \Vol M_5}$ multiplies every Green's function.  In particular, it
becomes the normalization factor for the one-point function $\langle
T^\mu_\mu \rangle$ as calculated in \cite{Henningson:1998gx}.  Also, as
pointed out in section \ref{tests}, the supergravity calculation in
\cite{Henningson:1998gx} always leads to $a=c$.  Without further thought we
can write $a = c \propto (\Vol M_5)^{-1}$, and
  \eqn{HenSkenRatio}{
   {a_{\rm IR} \over a_{\rm UV}} = {c_{\rm IR} \over c_{\rm UV}} = 
     \left( {\Vol T^{11} \over \Vol S^5/\IZ_2} \right)^{-1}
    = {27 \over 32} \ ,
  }
 in agreement with \ChargeRatio.  It is essential that the volumes in
\HenSkenRatio\ be computed for manifolds with the same cosmological
constant.  Our convention has been to have ${\cal R}_{\alpha\beta} =
4g_{\alpha\beta}$.

It is possible to do better and pin down the exact normalization of the
central charges.  In fact, literally the first normalization check
performed in the AdS/CFT correspondence was the verification
\cite{Gubser:1998bc} that in the compactification dual to
${\cal N}=4$ $SU(N)$ Yang-Mills theory, the
coefficient $c$ had the value $N^2/4$ (to leading order in large $N$).
Thus, in general
  \eqn{FreundRubinCharge}{
   a = c = {\pi^3 N^2 \over 4 \Vol M_5}
  }
 (again to leading order in large $N$) for the CFT dual to a Freund-Rubin
geometry $AdS_5 \times M_5$ supported by $N$ units of five-form flux
through the $M_5$.  This is in a normalization convention where the CFT
comprised of a single free real scalar field has $c = 1/120$.  See, 
for example,
\cite{Gubser:1999vd} for a table of standard anomaly coefficients per
degree of freedom.  Even more generally, we can consider any
compactification of string theory or M-theory (or any other, as-yet-unknown
theory of quantum gravity) whose non-compact portion is $AdS_5$.  This
would include in particular type~IIB supergravity geometries which involve
the $B_{\mu\nu}^{NS,RR}$ fields, or the complex coupling $\tau$.  Say the
$AdS_5$ geometry has ${\cal R}_{\mu\nu} = 
-\Lambda g_{\mu\nu}$.  If we rescale the
metric by a factor of $4/\Lambda$, we obtain the dimensionless $AdS_5$
metric $\widehat{ds}_5^2$ with $\hat{\cal R}_{\mu\nu} = -4 \hat{g}_{\mu\nu}$.
In defining a conformal field theory through its duality to the $AdS_5$
compactification under consideration, the part of the action relevant to
the computation of central charges is still the Einstein-Hilbert term plus
the cosmological term:
  \eqn{CosmConstC}{
   S = {1 \over 2\kappa_5^2} \int d^5 x \, \sqrt{g} 
    \left( {\cal R} + 3\Lambda + \ldots \right)
     = {4 \over \kappa_5^2 \Lambda^{3/2}} \int d^5 x \, \sqrt{\hat{g}} 
    \left( \hat{\cal R} + 12 + \ldots \right) \ ,
  }
 where $\kappa_5^2 = 8\pi G_5$ is the five-dimensional gravitational
coupling.  Comparing straightforwardly with the special case analyzed in
\FreundRubinCharge, we find that the conformal anomaly coefficients, as
always to leading order in $1/N$, must be given by 
  \eqn{acCos}{
   a = c = {1 \over G_5 \Lambda^{3/2}} \ .
  }



