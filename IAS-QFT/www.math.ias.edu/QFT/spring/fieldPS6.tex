%Date: Fri, 11 Apr 1997 09:56:27 -0400
%From: Edward Witten <witten@sns.ias.edu>



\input harvmac


Homework, April 10

\def\Z{\bf Z}
In class we considered in two dimensions a theory with $n$ chiral
superfields $A_i$, $i=1,\dots, n$, and superpotential
 $F(A_1,\dots,A_n)$ a homogeneous function of degree $n$.
Then we constructed a $\Z_n$ orbifold of this theory in which
one divides by the $\Z_n$ group that acts as $A_i\to e^{2\pi i/n}A_i$.

The theory has a left and right moving $R$ symmetry group $U(1)\times U(1)$,
with which you should be familiar.  So the Hilbert space is bigraded.


Formulate this theory on a circle ${\bf S}^1$ 
(with supersymmetric, periodic boundary
conditions for all fields).  There is a finite-dimensional space of
supersymmetric ground states of the theory, in fact for reasons
that should be clear from the lecture this space is closely related
to the cohomology of the projective hypersurface defined by the equation
$F=0$ in ${\bf P}^{n-1}$.  Both are bigraded, the bigrading on one side
coming from the $R$ symmetries and the bigrading on the other side coming
from the Hodge decomposition of the cohomology of the hypersurface;
the two bigradings will match (after a shift by $n/2$, that is
what is usually called $H^{p,q}$ maps to states whose $R$ charges
are $p',q'=p-n/2,q-n/2$).

Compute the space of supersymmetric ground states of the orbifold
theory.  There are different pieces:

(1) There are $n-1$ twisted sectors.  How many zero energy states
come from each twisted sector?  To find the 
$p',q'$ values of the ground states in the twisted sectors, you can
note that (a) perturbation theory is straightforward as there are no zero
modes, (b) you can go to the weak coupling limit in computing $p'$ and $q'$,
(c) therefore you can use free field theory, (d) the computation is hence
extremely similar to computations that appear in superstring theory
in comparing Ramond and Neveu-Schwarz sectors, (e) you essentially
just have to compute an eta invariant of a rather trivial one-dimensional
operator.

(2) In the untwisted sector, perturbation theory is not straightforward.
However, in this case the non-zero modes do not make any important
contribution (there is no eta invariant as in the previous case) and
hence you lose nothing essential if you think of the ${\bf S}^1$ as very
small, drop the non-zero modes, and reduce to $0+1$ dimensions.
We are hence in the realm of classical mathematics.  Using whatever
knowledge you have of quantum mechanics, ${\bf L}^2$ cohomology,
singularity theory, or whatever, determine the spectrum of
zero energy states of this $0+1$ dimensional problem.

Now combine the answers in (1) and (2) and compare to the cohomology of
the projective hypersurface.




\end


