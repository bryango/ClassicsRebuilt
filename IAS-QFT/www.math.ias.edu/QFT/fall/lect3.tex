
%% This is an AMS-TeX file.
%% The command to compile it is: amstex <file>
%%
\input amstex
\documentstyle{amsppt}
\loadeusm
\magnification=1200
\pagewidth{6.5 true in}
\pageheight{8.9 true in}

\catcode`\@=11
\def\logo@{}
\catcode`\@=13

\NoRunningHeads

\font\boldtitlefont=cmb10 scaled\magstep1
\font\bigboldtitlefont=cmb10 scaled\magstep2

\def\dspace{\lineskip=2pt\baselineskip=18pt\lineskiplimit=0pt}


\def\SO{\text{\rm SO}} 
\def\plus{{\sssize +}}
\def\oplusop{\operatornamewithlimits{\oplus}\limits}
\def\Piop{\operatornamewithlimits{\Pi}\limits}
\def\prodop{\operatornamewithlimits{\prod}\limits}
%\def\w{{\mathchoice{\,{\scriptstyle\wedge}\,}
%  {{\scriptstyle\wedge}}
%  {{\scriptscriptstyle\wedge}}{{\scriptscriptstyle\wedge}}}}
\def\Le{{\mathchoice{\,{\scriptstyle\le}\,}
  {\,{\scriptstyle\le}\,}
  {\,{\scriptscriptstyle\le}\,}{\,{\scriptscriptstyle\le}\,}}}
\def\Ge{{\mathchoice{\,{\scriptstyle\ge}\,}
  {\,{\scriptstyle\ge}\,}
  {\,{\scriptscriptstyle\ge}\,}{\,{\scriptscriptstyle\ge}\,}}}
\def\vrulesub#1{\hbox{\,\vrule height7pt depth5pt\,}_{#1}}
%\def\rightsubsetarrow#1{\subset\kern-6.50pt\lower2.85pt
%     \hbox to #1pt{\rightarrowfill}}
%\def\mapright#1{\smash{\mathop{\,\longrightarrow\,}\limits^{#1}}}
\def\arrowsim{\smash{\mathop{\to}\limits^{\lower1.5pt
  \hbox{$\scriptstyle\sim$}}}}
\def\Fdots{\hbox{$:\!\!F\!\!:$}}

\def\Fbar{\overline{F}}
\def\hhbar{\bar{h}}
\def\Vbar{\overline{V}} \def\tr{{\text{\rm tr}}}
%\def\Qbar{\overline{Q}}
%\def\Ebar{\overline{E}}
%\def\fbar{\overline{f}}
%\def\gbar{\overline{g}}
%\def\Sbar{\overline{S}}
%\def\scrRbar{\overline{\scrR}}
%\def\scrSbar{\overline{\scrS}}

%\def\Stil{\widetilde{S}}
\def\scrEtil{\widetilde{\scrE}}
%\def\scrStil{\widetilde{\scrS}}
%\def\scrTtil{\widetilde{\scrT}}
%\def\rhotil{\tilde{\rho}}
\def\grStil{\widetilde{\grS}}
\def\tilSt{\widetilde{St}}
\def\tilG'{\widetilde{G'}}

\def\spinact{s}
\def\spinel{\tau}


\def\uhat{\widehat{u}}
\def\Phat{\widehat{P}}
\def\Ghat{\widehat{G}}

\def\QFT{\text{\rm QFT}}
\def\End{\text{\rm End}} \def\Spec{\text{\rm Spec}}
\def\Cl{\text{\rm Cl}} \def\Sol{\text{\rm Sol}}
\def\id{\text{\rm id}} \def\odd{\text{\rm odd}}
\def\im{\text{\rm Im}} \def\even{\text{\rm even}}
\def\const{\text{\rm const}} \def\Ker{\text{\rm Ker}}
\def\supp{\text{\rm supp}}
\def\Map{\text{\rm Map}}
\def\Sym{\text{\rm Sym}}
%\def\Stab{\text{\rm Stab}}
%\def\Hom{\text{\rm Hom}}
%\def\Vbarspace{\overline {V}_{space}}
%\def\Vspace{ V_{space}}
%\def\SO{\text{\rm SO}} 
\def\Spin{\text{\rm Spin}} \def\spin{\text{\rm spin}}
%\def\Res{\text{\rm Res}}
%\def\Aut{\text{\rm Aut}}
%\def\Out{\text{\rm Aut}}
%\def\QFT{\text{\rm QFT}}
%\def\SU{\text{\rm SU}} \def\Lie{\text{\rm Lie}}
%\def\SL{\text{\rm SL}} 

\def\dbC{{\Bbb C}} 
\def\dbR{{\Bbb R}}
\def\dbZ{{\Bbb Z}} 

\def\gr#1{{\fam\eufmfam\relax#1}}

%Euler Fraktur letters (German)
\def\grA{{\gr A}}	\def\gra{{\gr a}}
\def\grB{{\gr B}}	\def\grb{{\gr b}}
\def\grC{{\gr C}}	\def\grc{{\gr c}}
\def\grD{{\gr D}}	\def\grd{{\gr d}}
\def\grE{{\gr E}}	\def\gre{{\gr e}}
\def\grF{{\gr F}}	\def\grf{{\gr f}}
\def\grG{{\gr G}}	\def\grg{{\gr g}}
\def\grH{{\gr H}}	\def\grh{{\gr h}}
\def\grI{{\gr I}}	\def\gri{{\gr i}}
\def\grJ{{\gr J}}	\def\grj{{\gr j}}
\def\grK{{\gr K}}	\def\grk{{\gr k}}
\def\grL{{\gr L}}	\def\grl{{\gr l}}
\def\grM{{\gr M}}	\def\grm{{\gr m}}
\def\grN{{\gr N}}	\def\grn{{\gr n}}
\def\grO{{\gr O}}	\def\gro{{\gr o}}
\def\grP{{\gr P}}	\def\grp{{\gr p}}
\def\grQ{{\gr Q}}	\def\grq{{\gr q}}
\def\grR{{\gr R}}	\def\grr{{\gr r}}
\def\grS{{\gr S}}	\def\grs{{\gr s}}
\def\grT{{\gr T}}	\def\grt{{\gr t}}
\def\grU{{\gr U}}	\def\gru{{\gr u}}
\def\grV{{\gr V}}	\def\grv{{\gr v}}
\def\grW{{\gr W}}	\def\grw{{\gr w}}
\def\grX{{\gr X}}	\def\grx{{\gr x}}
\def\grY{{\gr Y}}	\def\gry{{\gr y}}
\def\grZ{{\gr Z}}	\def\grz{{\gr z}}


\def\scr#1{{\fam\eusmfam\relax#1}}

\def\scrA{{\scr A}}   \def\scrB{{\scr B}}
\def\scrC{{\scr C}}   \def\scrD{{\scr D}}
\def\scrE{{\scr E}}   \def\scrF{{\scr F}}
\def\scrG{{\scr G}}   \def\scrH{{\scr H}}
\def\scrI{{\scr I}}   \def\scrJ{{\scr J}}
\def\scrK{{\scr K}}   \def\scrL{{\scr L}}
\def\scrM{{\scr M}}   \def\scrN{{\scr N}}
\def\scrO{{\scr O}}   \def\scrP{{\scr P}}
\def\scrQ{{\scr Q}}   \def\scrR{{\scr R}}
\def\scrS{{\scr S}}   \def\scrT{{\scr T}}
\def\scrU{{\scr U}}   \def\scrV{{\scr V}}
\def\scrW{{\scr W}}   \def\scrX{{\scr X}}
\def\scrY{{\scr Y}}   \def\scrZ{{\scr Z}}

\def\gr#1{{\fam\eufmfam\relax#1}}

%Euler Fraktur letters (German)
\def\grg{{\frak g}}
\def\grp{{\frak p}}


%\NoBlackBoxes
\document

\centerline{\bigboldtitlefont Lecture 3}

\bigskip\centerline{\boldtitlefont Free QFT}

\centerline{David Kazhdan}


\dspace
\bigskip\bigskip





\subhead{3.1} {Some examples of free classical field theories}\endsubhead
As was explained in Bernstein's lectures, for any classical
field theory the space $\Sol$ of stationary points of
Lagrangian carries a canonical symplectic form, or at least 
a closed $2$-form.
%%%%%%%%%%It is obtained from a canonical 2-form on $M$ with values in
%%%%%%%%%%$H^{\text{top}-1}(M)$,  by integrating %%%%%%%%%%
%%%%%%%%%%the  $H^{\text{top}-1}(M)$-classes along a ``space-like''
%%%%%%%%%%codimension 1 cycle  on $M$.  
 
The theory is called free if $\Sol$ is isomorphic to a
linear space with a constant $2$-form and linear action of the
Poincar\'e group.
Let us consider some examples.

%\smallskip\noindent
\example{Example 1. 
A scalar boson field in dimension $(1,d-1)$}

\noindent
A classical field is a real-valued function on
$\dbR^{1,d-1}$.
The Langrangian is given by:
$$
\scrL(\varphi)=\vert d\varphi\vert^2-m^2\varphi^2.
$$
The space of stationary points is: $\Sol=\{\varphi\vert
\square \varphi+m^2\varphi=0\}$.
It is linear, the symplectic form is constant.
It is given by:
 $[\varphi_1,\varphi_2]=\int\limits_{\Gamma}
*(\varphi_1 d\varphi_2-\varphi_2 d\varphi_1)$ for
$\varphi_1,\varphi_2\in\Sol$, where $\Gamma$ is a space-type
hypersurface in $V$.
By a space-like hypersurface we mean a complete smooth
oriented hypersurface such that the normal vector at each
point lies in $V_+$ (for example the hyperplane
$\{t=\const\}$ is space-like).
Since $*(\varphi_1 d\varphi_2-\varphi_2 d\varphi_1)$ is
closed and sufficiently rapidly decaying the integral does
not depend on the choice of a hypersurface. (The condition of Schwartz
decay in space directions is tacitly imposed on the classical solutions
under consideration throughout the lecture). 
\endexample

%\smallskip\noindent 2)\enspace
\example{Example 2. Spinors}
Let $V$ be $10$-dimensional, or of any dimension $n=8k+2$.
Then the spinor representation $S$ decomposes as $S=S^+\oplus
S^-$, and $S^+$, $S^-$ are real.
There exists a unique symmetric pairing
$\pi\colon\,S^+\times S^+\to V$ (see \S {3.16} in Deligne's 
``Notes on spinors'' on this site). 
We also have a pairing $(\quad,\quad)\colon\, S^+\times
S^-\to\dbR$ and the action map $\spinact :S^\pm \otimes V\to
S^\mp$.
The space of classical fields is defined to be the space of maps:
$\Map(V,\dbR^{0,1}\otimes
S^+)=\Pi\overline{\Sol}$ where
$\overline{\Sol}=\Map(V,S^+)$.
We define Lagrangian by: $\scrL(\psi)=(\partial\psi,\psi)$.
Here $d\psi\in\Omega^1(V,\dbR^{0,1}\otimes
S^+)=\Gamma(V,\dbR^{0,1}\otimes S^+\otimes V)$, and
$\partial\psi \overset\text{def}\to=
\spinact d\psi\in\Gamma(V,\dbR^{0,1}\otimes S^-)$.
We have: $\Sol=\{\psi\vert \partial\psi=0\}$,
it is a purely odd vector space.
Giving a symplectic form on $\Sol$ is equivalent to giving a
symmetric form on $\overline{\Sol}$.
The symmetric form is constant and is given by:
$(\psi_1,\psi_2)=\int\limits _\Gamma\pi(\psi_1,\psi_2)$ where $\Gamma$
is a space-type hypersurface, and we use the
isomorphism $V\simeq\Lambda^{d-1}(V^*)$ provided by the
volume form on $V$ to interpret $\pi(\psi_1,\psi_2)$
as a $(d-1)$-form on $V$.
\endexample


\example{Example 3. Free electro-dynamics in any dimension}

\noindent
Let $M$ be the space of $1$-forms on $V$ modulo the exact
forms.
The Lagrangian $\scrL\colon\,M\to\Omega^d(V)$ is as follows:
$$
\scrL(\alpha)=\alpha\cdot (d*d)\alpha.
$$
We have: $\Sol=\{\alpha\vert d*d\alpha=0\}/\{\text{exact
$1$-forms}\}$.
We will write down the symplectic form using Fourier
transform.
The Fourier transform of the exact sequence
$$
0 @>>> \Omega^0(V)/\dbR @>d>> \Omega^1(V)
@>{*d*d}>> \Omega^1(V)
$$
looks as follows:
$$
0 @>>> \scrS(V) @>{\nu}>> \scrS(V)\otimes V
@>{\eta}>> \scrS(V)\otimes V
$$
where $\scrS(V)$ is the Schwartz space, and:
$$
\align
\nu(f)(p) &=f(p)p\quad\text{for $p\in V,\,f\in\scrS(V)$},\\
\eta(G)(p) &=(G(p),p)p-(p,p)G(p).
\endalign
$$
Let $\scrO_0=\{p\vert p^2=0,\, p\not=0\}$.
Let $\scrN$ denote a vector bundle over $\scrO_0$, such
that $\scrN_p=p^\perp/\dbR p$.

\proclaim{Lemma}
\roster
\runinitem"a)"
For any $G\in\Ker\,\eta$ we have $G(p)\in p^\perp$ for all
$p\in\scrO_0$.

\item"b)"
The map $\Ker\,\eta\to\Gamma(\scrO_0,\scrN)$ given by $G\to
G(p)/\dbR p$ is an isomorphism between $\Sol$ and
$\{\gamma\in\Gamma(\scrO_0,\scrN)\vert
\gamma(-p)=\overline{\gamma(p)}\}$.
The symplectic form $[\quad,\quad]$ on $\Sol$ is constant
and is given by
$[\gamma_1,\gamma_2]=\int\limits_{\scrO_0^+}\im\,
\gamma_1\overline{\gamma_2}$.
\endroster
\endproclaim
\endexample

\remark{Remark} It is a useful exercise to find an expression 
for the symplectic form $[\quad,\quad]$ in terms of Fourier transform
in examples 1,2 also.
\endremark

\subhead{3.2} {Clifford module}\endsubhead
Quantization of  a free classical field theory is based on the following
linear algebra construction.


Let $H$ be a real superspace, equipped with an even 
super-symplectic form $[\quad,\quad]: H\times H\to\dbR$.
Let $\Cl^H$ be the corresponding Clifford (=Heisenberg) algebra.

A complex polarization of $H$ is an automorphism
$I\colon\,H\to H$, preserving $[\quad,\quad]$ and such that
$I^2=-\id$.

Define the maps $i_\pm\colon\, H\to H_\dbC$ by:
$i_\pm(v)=v\pm i^{-1}Iv$.
Let $H_\pm=\im$ $i_\pm$.
Then $H_\pm$ is a complex isotropic subspace.
We also have an antilinear isomorphism $^{\hbox{--}}\colon\,
H_+\to H_-$.
Define $(\quad,\quad)\colon\, H_+\times H_+\to\dbC$ by
$(h_1,h_2)=\frac {i}{2}[h_1,\hhbar_2]$.

\proclaim{Claim}
\roster
\runinitem"a)"
$(\quad,\quad)$ is a super-hermitian form on $H_+$.

\item"b)"
$(\quad,\quad)$ is positive iff: $[h,h]>0$ for $h\in
H^{\odd}-\{0\}$ and $[Ih,h]>0$ for $h\in H^{\even}-\{0\}$.
\endroster
\endproclaim

\proclaim{Proposition}
Let $(H,[\quad,\quad],I)$ be as before such that
$(\quad,\quad)$ is positive.
Then
\roster
\item"a)"
there exists a unique up to unique isomorphism triple
$(j_I,\scrD,\Omega)$ where: $j_I$ is a complex representation of
$\Cl^H$ on $\scrD$ and $\Omega\in \scrD-\{0\}$ such that:

%%%%%%%%%%\itemitem
$j_I(\Cl^H)(\Omega)=\scrD$

%%%%%%%%%%\itemitem
$j_I(H_-)(\Omega)=0$.

\item"b)"
There exists a unique positive Hermitian form
$(\quad,\quad)_\scrD$ on $\scrD$ such that
$(\Omega,\Omega)=1$ and $j_I(h)$ is an antiHermitian operator
on $\scrD$ for $h\in H$.
\endroster
\endproclaim

\remark{Remark}
We have an embedding $\Sym(H_+)\hookrightarrow \Cl^H$, which
induces an isomorphism:\break
 $j_{H_+}\colon\,\Sym(H_+)\arrowsim
\scrD$, where $j_{H_+}(h)=j_I(h)(\Omega)$.
The algebra $\Cl^H$ acts on $\scrD=\Sym(H_+)$ by
differential operators.
\endremark

\remark{Remark}
If $H$ is infinite-dimensional, then the representation
depends essentially on the complex polarization $I$.
For finite dimensional $H$ the $L^2$-completions of
representations constructed via different polarizations are
isomorphic.
\endremark

\remark{Remark}
The above construction for the finite dimensional case is discussed in 
Deligne's notes on quantization \S 3--7.
\endremark

\subhead{3.3} {Examples of free QFT's} \endsubhead
\subsubhead
Scalars
\endsubsubhead  Let $H$ be the space of real solutions of:
$\square\varphi+m^2\varphi=0$ and $[\quad,\quad]$ be the
form on $H$ given by $[\varphi_1,\varphi_2]=\int_\Gamma *(\varphi_1
d\varphi_2-\varphi_2 d\varphi_1)$ (Example 1 above).

To  $f\in \scrS(V)$ there corresponds a classical observable,
i.e. a function on $H$, given by $h\to \langle h, f\rangle _{L^2(V)}$. So in
quantum theory  an operator
 $\phi(f)$  should be defined for $f\in \scrS(V)$.

Let us construct a positive complex polarization $I$ on
$(H,[\quad,\quad])$.
We describe $I$ in terms of Fourier transform.
The Fourier transform gives an isomorphism
$H=\{f\in\scrS_\dbC(\scrO_m)\vert f(-p)=\overline{f(p)}\}$,
where $\scrO_m=\{p\in V-\{0\}\vert p^2=m^2\}$.
We have: $\scrO_m=\scrO_m^+\sqcup \scrO_m^-$ (union of $2$
connected components).
Restriction to $\scrO_m^+$ defines an isomorphism
$H\arrowsim \scrS_\dbC(\scrO_m^+)$.
Since $\scrS_\dbC(\scrO_m^+)$ has natural complex structure
and
$\scrS_\dbC(\scrO_m^+)\subset\scrS_\dbC(\scrO_m)=H_\dbC$
is an isotropic subspace, we get a positive complex
polarization of $H$.
Consider the ``canonical'' representation
$j_I\colon\,\Cl^H\to\End\,\scrD$ described above.
Let $\scrH=\overline{\scrD}$ be the Hilbert space completion
with respect to the invariant Hermitian form (see
Proposition).
Since the action of $P$ on $H$ commutes with $I$, we get an
action of $P$ on $\scrH$.

We define  the field operators
$\phi\colon\scrS(V)\to\End(\scrD)$ as the composition 
$$\scrS(V)\owns f @>>> \scrF(f) @>>> \scrF(f)|_{\scrO_m}
@> j_I>> \End \scrD$$
where we consider the restriction $ \scrF(f)|_{\scrO_m}$ as
an element of $H$. 


Let us finally analyze the support of the spectral measure
of $\scrH$ considered as a $V$-module.
We obviously have: $\Spec\,\rho\vrulesub{V}=\scrO_m^+$,
where $\rho$ denotes the action of $P$ on $H_+$.
Recall that $j_{H_+}\colon\,\Sym\,H_+\arrowsim \scrD$ is an
isomorphism.

Hence $\Spec\,U\vrulesub{V}$ is the closure of the set 
$\{0\}\cup \scrO_m^+\cup
\scrO_m^++\scrO_m^+\cup\ldots$ where we use the notation
$A+B=\{a+b\, \vert \, a\in A, b\in B\}$, and $U$ stands for the
representation of $P$ on $\scrH$.
It is easy to see that $\Spec\,U\vrulesub{V}=\{0\}\sqcup
\scrO_m^+\sqcup\{p\in V_+\vert p^2\Ge 2m\}$.

\subsubhead
Spinors
\endsubsubhead
We keep the notations of Example 2 above.
Let $H=\Pi(\{\psi\vert \partial\psi=0\})$.
(Recall that $\Pi$ denotes change of parity of a
superspace).

A classical observable is associated with $f\in \scrS^{\scrS^-}(V)$,
 and is given
by $h\to \int _V \left( f(v), h(v) \right) dv$.

In Fourier coordinates the classical field equation becomes:
$s(p,\scrF (\psi)(p) )=0$.
Since $\scrD^2=\square$ we see that for any $\psi\in H$:
$\supp(\scrF(\psi))\subset\scrO_0$.

Let $\scrA$ be a subbundle of the trivial bundle over $\scrO_0$
with fiber $S^+$, such that $\scrA_p\subset
S^+=\Ker\,\spinact(p)$.

Then $H\cong \{f\in\scrS_\dbC(\scrO_0,\scrA)\vert f(-p)=
\overline{f(p)}\}$.
The symplectic form on $\Sol$ is the same as symmetric
pairing on $H$, and it is given by:
$(f_1,f_2)=\int\limits_{\scrO_0^+}\frac{\pi(f_1(p),f_2(-p))}{p}
\mu_0$, where we use linear dependence of the vectors
$\pi(f_1(p),f_2(-p))\,$, $p$, and denote by
$\frac{\pi(f_1(p),f_2(-p))}{p}$ the coefficient of
proportionality. (This formula is equivalent to the one given in Example
2).

Using the decomposition $\scrO_0=\scrO_0^+\sqcup
\scrO_0^-$, we obtain  a polarization and the
canonical representation $j_I\colon\,\Cl^H\to\End\,\scrD$.
If $U$ is, as usually, the action of $P$ on $\scrD$, then
$\Spec\,U\vrulesub{V}=\Vbar_+$.

The field operators are also defined similarly to the above. 
Namely, for $f\in \scrS(V,S^-)$ the section
 $\varkappa(f)$ such that $\varkappa (f)(p)=s(p,\scrF(f)\vrulesub{\scrO_0}
(p) )$ 
 lies in $H$ because $s(p)^2=0$ for $p\in \scrO_0$.
We  define
$\phi(f)=j_I(\varkappa(f))\in \End\,\scrD$. 

\subsubhead
Photones {\rm (Free quantum electro-dynamics
= free abelian gauge theory)}
\endsubsubhead
Here $\Sol\cong \{f\in\scrS(\scrO_0,\scrN)\vert
f(-p)=\overline{f(p)}\}$.
Polarization comes again from the decomposition:
$\scrO_0=\scrO^+\sqcup\scrO^-$.
Thus we obtain the canonical representation
$j_I\colon\,\Cl^H\to\End\,\scrD$.

Recall that in classical situation (Example 3)
the space Sol is a subspace in $\Omega^1(V)/$
$d\Omega^0(V)$.
So to any 2-form $\omega \in \Omega^2(V)$ there corresponds
a classical gauge-invariant observable:
$$\Omega ^1/d\Omega ^0 \owns \alpha \to \int \limits _V d\alpha
\wedge *\omega$$

%%%%%%%%%%Thus we have to associate with $\omega \in \Omega ^2(V)$
%%%%%%%%%%a quantum observable $\phi (\omega)\in \End (\scrD)$.

Consider the map $\varkappa\colon\,\Omega^2(V)=\scrS^{\Lambda^2 V}\to H =
S_\dbC(\scrO_0,\scrN)$ given by
$\varkappa(\omega)(p)=i_p\scrF(\omega)(p)\in p^\perp$.
Now we define the field map: $\phi(\omega)=j_I(\varkappa(\omega))$.



\subhead{3.4} {Free QFT of arbitrary spin}\endsubhead
A QFT is called  free of mass $m$ if the field operator $\phi(f)$ depends
only on $\scrF(f)|_{\scrO_m}$. (In other words, the classical field equation
$\square +m^2=0$ is satisfied  on quantum level).  

All examples considered in the previous section are of this kind.

In this section we give a construction of a  free QFT of mass
$m$ which generalizes these examples. 

To describe it we need some preliminaries. 

Recall that  $G=\Spin(d-1,1)$, and $\spinel\in G$ is the spin-element.

 We fix the mass 
of theory $m \Ge 0$. Let $St$ denote the stabilizer of a point 
$p \in \scrO _m ^+$. 

 If $m>0$ then  $St=\Spin (p^\perp) \cong \Spin (d-1)$.
If $m=0$ then $St$ has nontrivial 
unipotent radical $N$ and $St/N =\Spin (p^\perp/\dbR p)\cong \Spin(d-2)$.
 Let $G'$ denote a maximal semisimple subgroup of $St$.
 (So $G'=St$ if $m>0$, and $G'=St/N$ if $m=0$). 


 We fix a real representation $R$ of $G$, and
$\rho$ of $G'$.
 We will view $R,\,\rho$ as representations of real algebraic
groups defined over $\dbR$.

  $\rho$ can be also viewed as representation of
the group $St$. We  fix a nonzero morphism
 $ i:R|_{St}\to \rho$.

We do not have to assume that  $R,\,\rho$ are irreducible, but it is
convenient to require that $\spinel$ acts on $R$ and $\rho$
by constant, i.e. $R$ and $\rho$ are sums of irreducible representations of
equal spin. This constant will be denoted by $(-1)^{\spin(R)}$, 
$(-1)^{\spin (\rho)}$ respectively. 
  Obviously  $(-1)^{\spin(R)}=(-1)^{\spin (\rho)}$.

Let $\tilSt$ denote the subgroup $\{g\in G| g(p)=\pm p\}$; let 
$\tilG'$ denote a maximal semisimple
 subgroup of $\tilSt$. (So $\tilG'=\tilSt$
if $m>0$, and  $\tilG'=\tilSt/N$ if $m=0$). Clearly $G'$ is a subgroup of
$\tilG'$ of index 2. 

More precisely, if $m>0$ then $\tilG' \cong Pin(n)$. If $m=0$ then $\tilG'
\cong  
(\Spin(d-2)\times \dbZ _4 )/(\spinel , -1)$.

 $\tilG'$ carries the structure of algebraic group  over
$\dbR$. Note that 
 the nontrivial connected component of $\tilG'$ has {\it no real points}.

We assume that complexification of $\rho$ extends to a representation
of $\tilG'$ in such a way that  the map $i$ is $\tilG'$-equivariant.
 (The representation extends automatically if $m=0$
or if $d$ is even).

Let $K$ be the normalizer in $\tilG' (\dbC)$ of the subgroup of real points
 $\tilG' (\dbR)=G'(\dbR)$. Then $K$ is a compact Lie subgroup in the
Lie group  $\tilG' (\dbC)$; moreover, $K$ has 2 connected components
and the connected component of 1 is $G'(\dbR)$. 

We have: $K = \{g\in \tilG'(\dbC) |\pi (g) \in
\SO(d-1,1)(\dbR)\}$ where $\pi: \Spin \to \SO$ is the projection. 

 Note that  $K$ is {\it not} the set of real points of $\tilG'$.
 More precisely, for  $g\in K-G'(\dbR)$ we have $\overline g = g\cdot \spinel$.


Since $K$ is compact, the space of $\rho_\dbC$ carries a $K$-invariant
Hermitian form $\left<\quad,\quad\right>$. It also yields a quadratic form
$(\quad,\quad)$, where $(x,y)=\left< x, \overline y \right>$. The quadratic
form restricted to the subspace of real vectors (the space of $\rho$)
is real and positive definite. 

\medskip

Now let us return to QFT.   


 
To $\rho$ there corresponds 
a $G$-equivariant real vector bundle $\scrA _\rho$ on $\scrO _m^+$;
its complexification will be denoted by $ \scrA _\rho^\dbC$.
Let  $H$ be the space of Schwartz sections  of  $ \scrA
_\rho^\dbC$ viewed as a real vector space. 

 As usual we get  the
structure of a super-space on $H$ by declaring the parity of $H$ to be 
$(-1)^{\spin(\rho)}$. 


The bundle  $\scrA_\rho^\dbC$ carries a
$G$-equivariant Euclidean and Hermitian metrics, obtained
from metrics $(\quad , \quad)$, $\left<\quad,\quad \right>$
on the space of $\rho_\dbC$ in the standard way. 
We will abuse notations by denoting the metrics on the vector bundle by
 $(\quad , \quad)$, $\left<\quad,\quad \right>$ also. Of course
  $(\quad , \quad)|_{\scrA_\rho}$ is real and positive.  

  A super-symplectic form on $H$ is  given by:
$[h_1,h_2]=\int\limits_{\scrO_m^+} Im \left<h_1,\overline{h_2} \right>$
if $H$ is even, and 
$[h_1,h_2]=\int\limits_{\scrO_m^+} Re \left<h_1,\overline{h_2} \right>$
if $H$ is odd. The obvious complex structure on $H$ yields the
 positive polarization to be denoted  $I$.
 Thus we get the canonical representation $\scrD\owns
\Omega$ and its Hilbert space completion $\scrH$. 



To  provide the field  map
$\phi : \scrS^R \to \End (\scrD) $ we first construct 
a map $\varkappa : \scrS^R \to H$ and then take $\phi = j_I \circ \varkappa$.


To the morphism $i: R|_{St} \to \rho$
there corresponds a map $\iota:R \to \Gamma (\scrA_\rho)$, where
$\Gamma(\scrA_\rho)$ is the space of sections.



Now we can define $\varkappa$. For $f\otimes r \in 
 \scrS^R = \scrS \otimes R $ we put  
$\varkappa (f\otimes r) = \scrF(f)|_{\scrO_m^+}\cdot \iota(r)$ 
 where $\scrF$ stands for  Fourier transform as usual.  

The data of QFT is constructed.

\remark{Remark} In fact the maps $\varkappa$, $\phi$ are defined
on a larger function space then the space of Schwartz section $\scrS^R$.
Indeed, let $F$ be a Schwartz function on $\dbR^{d-1}$. Let $f$ be a
generalized function on $V=\dbR \times \dbR^{d-1}$ of the form: $f =\delta
_{t=t_0} \times F$ where $t$ is the coordinate on $\dbR$.  Then it is easy
to see that $\scrF(f)|_{\scrO_m^+}$ is a Schwartz function on the
hyperboloid $\scrO _m^+$. Thus $\varkappa$ and $\phi$ are defined on the
generalized sections of $\scrR$ of the form $f\cdot r$ for $f$ as above
and $r\in \dbR$. 


\endremark

 Let us now check the axiom $\epsilon)$ of
Wightman QFT (the space-locality property).

We need the following

\proclaim{Key lemma} 
For any $r_1,\,r_2 \in R$ the real-valued function on $\scrO_m^+$
defined by $P_{r_1,r_2}(p)=(\iota(r_1),\iota(r_2))|_p$ extends to  a polynomial
function on $\scrO_m$. This polynomial function is even if $R$ and $\rho$
have trivial spin, and odd otherwise.
\endproclaim

\demo{Proof of the Lemma} We can view $\scrA_\rho$ as an algebraic
vector
bundle on the real algebraic variety $\sigma_m$, and $\iota(r)$
for $r\in R$ -- as  an algebraic section of  this vector bundle. 

Since the metric on $\scrA_\rho$ is algebraic as well, 
the first part of the claim is clear. 

For any $g\in G(\dbC)$ such that $g(p)=-p$
we have by the definition 
$$
(\iota(r_1),\iota(r_2))_{-p}=
(\iota(g(r_1)),\iota(g(r_2)))_{p}=
(i(g(r_1)),i(g(r_2)))
$$

In particular the last equality holds for any $g\in K-G'(\dbR)$.

As was mentioned above for
$g\in K-G'(\dbR)$ we have $g^{-1} \overline g
=\spinel$. So for such $g$:
$$
\multlinegap{0pt}
\multline
(i(g(r_1)),i(g(r_2)))=\langle i(g(r_1)),
\overline{i(g(r_2))}\rangle= \langle(i(r_1)),
\overline g(i(r_2))\rangle= \langle i(r_1),
g^{-1} \overline g(i(r_2))\rangle\\
= \langle i(r_1),
\spinel (i(r_2))\rangle=(-1)^{\spin (\rho)} \langle i(r_1),
i(r_2)\rangle=(-1)^{\spin (\rho)}(\iota(r_1),\iota(r_2))_{p}
\endmultline
$$

So  for $r_1,\,r_2 \in R$ 
we proved that $\left(\iota(r_1),\iota(r_2)\right)_{-p}=
(-1)^{\spin (\rho)}\left(\iota(r_1),\iota(r_2)\right)_{p}$. 
Since the map
$(r_1,r_2)\to P_{r_1,r_2}$ is $G$-equivariant it 
follows that the same
holds for any $p'\in \sigma _m ^+$, and the Lemma is proven. 
\enddemo

\demo{Proof of space-locality}
It is enough to check that if
$\varphi_1, \varphi  _2 \in \scrS^R$ are 
space-like separated then the
scalar product $[\varkappa (\varphi _1),\varkappa(\varphi_2)]$ 
vanishes.

Let us take $F_1,\, F_2 \in \scrS(\dbR^{d-1})$ such that
$\supp (F_1)\cap \supp (F_2) 
=\emptyset$. Let $f_j = \delta _{t=0} \times
F _j$ for $j=1,2$. For $r_1,\, r_2 \in R$ the expression $f_j\cdot
r_j$ is a generalized section of $\scrR$. As was mentioned above
$\varkappa$ is actually defined on such sections.

Since $\varkappa$ and $[\quad , \quad]$ are $P$-equivariant 
it sufficies
to check that $[\varkappa(f_1\cdot r_1),\varkappa(f_2\cdot r_2)]=0$
for $f_j, \, r_j$ as above.

Put $g_j=\scrF(f_j)$. 
Then the functions $g_j$ satisfy
$g_j(p+(t,\vec 0))=g_j(p)$, so we can write $g_j(t, q)=
g_j(q)$ where $t\in \dbR$, $q\in \dbR^{d-1}$.
We also have $g_j(-p)=\overline{g_j(p)}$ 
since  $f_j$ is a real function. So:
$$
\multlinegap{0pt}
\multline
[\varkappa (f_1\cdot r_1),\varkappa (f_2\cdot r_2)]
=\const \cdot 
\int\limits_{\sigma_m^+} 
\left(g_1 \overline{g_2} - (-1)^{\spin (\rho)} 
g_2\overline{g_1}\right)\left(\iota (r_1),\iota (r_2)\right)d\mu\\
=\const \cdot 
\int\limits_{\dbR^{d-1}} \left(g_1(q) \overline{g_2(q)} -
(-1)^{\spin (\rho)}g_2(q) \overline{g_1(q)}\right)
P_{r_1,r_2}(|q|,q) \cdot |q|^{-1} d q
\endmultline
$$
(notations of the Key Lemma). It can be rewritten as: 
$$
\multlinegap{0pt}
\multline
\int\limits
_{\dbR^{d-1}} \left(g_1(q) \overline {g_2(q)} -
 (-1)^{\spin (\rho)} 
g_1(-q) \overline {g_2(-q)}\right)
P_ {r_1,r_2}(|q|,q) \cdot |q|^{-1} d q\\
=\int\limits_{\dbR^{d-1}} (g_1(q) \overline {g_2(q)})
\left(P_{ r_1,r_2}(|q|,q)
- (-1)^{\spin (\rho)} P_{ r_1,r_2}
(|q|,-q) \right)
 \cdot |q|^{-1} d q
\endmultline
$$

By the Key Lemma 
$$ 
P_{ r_1,r_2}(|q|,q)
- (-1)^{\spin (\rho)} P_{ r_1,r_2}
(|q|,-q)  = P_{ r_1,r_2}(|q|,q)
-  P_{ r_1,r_2} (-|q|,q)
$$ 
and $ P_{ r_1,r_2}(|q|,q)$ is a polynomial function. 

But then it follows that the 
function $Q(q)=\vert q\vert^{-1}(P_{ r_1,r_2}(\vert q\vert,q)
-P_{r_1,r_2}(-\vert q\vert,q))$ is again polynomial. 

Fourier transform of
multiplication by a polynomial function is a differential operator,
in particular  is local. Thus we have:
$$
\align
[\varkappa(f_1\cdot r_1),\varkappa(f_2\cdot r_2)]&=
\int\limits _{\dbR^{d-1}}(g_1(q)\overline{g_2(q)})Q(q)dq
=(Q(q)\cdot g_1, g_2)_{L^2(\dbR^{d-1})}\\
&=(\hat{Q}(F_1), F_2)_{L^2(\dbR^{d-1})} = 0
\endalign
$$
where $\hat {Q}$ is the differential operator with constant coefficients
corresponding to $Q$ and the last equality is obvious because  
$F_1, \, F_2$ have separated supports.

Space-locality is proved.
\enddemo


\subhead{3.5} {Wightman functions of a free
field theory; truncated Wightman functions} \endsubhead
The next statement characterizes free theories in terms of Wightman
functions. 

\proclaim{Claim}
Let $\scrW_n$ be the Wightman functions of the free scalar
$\QFT$ of mass $m$.
Then we have
$\scrW_2(v_1,v_2)=\scrF(\delta_{\scrO_m^\plus})$ 
$(v_1-v_2)$; $\scrW_n=0$ if $n$ is odd, and
$$
\scrW_{2k}(v_1,\dotsc,v_{2k})=\sum
\scrW_2(v_{i_1},v_{j_1})\scrW_2(v_{i_2},v_{j_2})\ldots
\scrW_2(v_{i_k},v_{j_k}),
$$
where the sum is taken over all partitions of
$\{1,\dotsc,n\}$ in pairs $(i_1,j_1),\dotsc,(i_k,j_k)$,
and $i_r<j_r$.
\endproclaim

Proof is straightforward.

This statement can be easily generalized to other free
$\QFT$'s. 

It also can be reformulated using the so-called {\it truncated Wightman
functions}. 

Since these objects will be used in the next lecture, let us give their
definition. 
We set: $$\scrW_n^{\tr}\,\,{\overset\text{def}\to=}
\sum\limits_{\pi=(\pi_1,\dotsc,\pi_k)}(-1)^{k-1}
 (k-1)!
\Piop_{\ell=1}^k
\scrW_{ |\pi_\ell |}
(v_{i_1}, v_{i_2},\dotsc,
v_{i_{|\pi_\ell |}})$$

Here the sum is over all decompositions of $\{1,..,n\}$
into the union of disjoint nonempty subsets $\pi_1,..,\pi_k$,
(the order of  $\pi_1,..,\pi_k$ does not matter), 
and $i_1<i_2  \dotsc < i_{|\pi_\ell|}$
are the elements of $\pi_\ell$.
 
An equivalent equality is
$$
\scrW_n=\sum\limits_{\pi=(\pi_1,\dotsc,\pi_k)}
\prod\limits_{\ell=1}^k \scrW^\tr_{ |\pi_\ell|}
(v_{i_1}, \dotsc,
v_{i_{|\pi_\ell|}})$$

\proclaim{Corollary} For a free QFT we have $W^{\tr}_n = 0$ for $n> 2$. 
\endproclaim






\subhead{3.6} {Gaussian measures}\endsubhead
One can also easily describe free field theory in Euclidean
formulation.

For a free QFT one can make sense of the measure $\nu$ on the space of
tempered distributions discussed in lecture 2 (\S 2.3);
 the corresponding measure is
Gaussian. Here is the formal definition for the case of the scalar theory.

\proclaim{Claim}
Let $\grS$ ($e_1,\dotsc,e_n$) be the Schwinger functions of
the free Euclidean scalar $\QFT$ of mass $m$.
Then $\grS$ ($e_1,\dotsc,e_n$) is the moment of a Gaussian
measure $\nu_m$ on the space of tempered distributions
$\scrS'(E)$ with covariance  $ (-\square_E+m^2)^{-1}$.
[So for $f\in\scrS$ we have
$\scrF(\nu_m)(f)=e^{((-\square_E+m^2)^{-1} f,f)}$].
\endproclaim


\remark{Remark} Gaussian measure on a finite dimensional real vector space
$H$ corresponding  to a quadratic form $\left<\quad,\quad\right>$
is characterized by the equalities 
$$
\nu(1)=1;\ \  \ \,\nu\left(\frac{\partial\, P}{\partial h}\right)
=-h^{\sssize\vee}\nu(P)
$$
 for any $h\in H$, and $P\in \Sym(H')$ a polynomial function on $H$. 
(Here $\frac{\partial}{\partial h}$ is a constant vector field, and
$h^{\sssize\vee}=\left<h,\quad\right>$ is a linear functional on
$H$.)

In our setting of kernel spaces ($H=\scrS'$, $H'=\scrS$
and $\left<f,f\right>=
\left((-\square +m^2) f,f\right)$) this is still so, if one requires that
the equality holds for all $h$ for which it makes sense, namely for all
$h\in \scrS\subset \scrS'$. 
\endremark

Reconstruction of the Hilbert space of a free QFT in terms of the Gaussian
measure is based on the next Lemma.



Let $I$ be a positive polarization of $H$, $[\quad,\quad]$ and let
$\left<h_1,h_2\right>=[Ih_1,h_2]$. Then we have an imbedding
$\scrD=\Sym(H_+)\hookrightarrow A_H$ defined by the
decomposition $H_\dbC=H_+\oplus H_-$.

\proclaim{Lemma}
For $F_1,F_2\in\scrD$ we have
$(F_1,F_2)_\scrD=\nu(F_1\Fbar_2)$
\endproclaim

We finish the lecture with discussion of 

\subhead{3.7} {Normal ordering}\endsubhead
In general in QFT one is interested in behavior of correlation functions near
diagonals. Naively speaking, we would like to consider along with
 operator-valued generalized function $\phi(v)$ also an operator-valued
generalized
 function $\phi^n(v)$. As usual such operator-valued
functions should be encoded in correlation functions on $V_k$
whose ``value at $(v_1,\ldots,v_k)$'' equals to $\left(\Omega,
\phi^{n_1}(v_1)\cdot
\ldots \cdot \phi^{n_k}(v_k)\Omega\right)$. If the naive definition would work,
these correlators would obviously be the restrictions
 of the original Wightman function to the corresponding diagonal.
Of course this is not possible, since  Wightman functions are
 singular along diagonals. Techniques to overcome this difficulty are
 discussed in Witten's lecture 3; here we only present a simple
 device, which leads to an answer in the free theory case (see also
\S \S 3.2, 3.5 in Witten's lecture 3).

Let $(\scrH, \phi, \scrD, \Omega)$ be the free scalar QFT. (The assumption
 that the theory is scalar is not essential, and made for
 notational convenience).

 For any function $f\in \scrS$ let us write $f=f_++f_-$
where $f_\pm \in \scrS_\dbC$ and $\scrF(f_\pm)|_{\scrO_m^\mp}=0$.

Then define normally ordered power of the field operator by
$$:\phi(f)^n:{\overset\text{def}\to=}
 \sum \binom ni
\phi(f_+)^i\phi(f_-)^{n-i}.$$ 

\proclaim{Claim}

 \roster
\item"a)" For any $v\in V$, $x,y \in \scrD$ and $f\in \scrS$
the limit ${\underset{f\to \delta_v}\to
\lim} \left( :\phi( f)^n : (x),y\right)$ exists; we will write 
$\left(:\phi^n(v):(x),y\right){\overset\text{def}\to=}
{\underset{f\to \delta_v}\to \lim}\left( : \phi( f)^n : x,y\right)$. Thus the
correlator $\left(\phi(v_1),..,:\phi^n(v_i) :,..,\phi(v_n)\right)$ is 
a well-defined distribution.


\item"b)" If $d=2$, then for $f\in \scrS$ there exists a well-defined
endomorphism $:\phi^n(f):\in \End(\scrH)$ such that $\left(:\phi^n(f):
(x),y\right)=\int f(v)\left(:\phi^n(v):(x),y\right) dv$.
\endroster
\endproclaim 

In fact Claim b) is valid for much wider class of 2-dimensional theories
than free ones; see \S 8.6 of [Glimm-Jaffe]. 

One can define normal ordering more algebraically as follows.

Recall that we have $H=\{\varphi\vert\square\varphi + m^2\varphi=0\}$,
$[\varphi_1,\varphi_2]=\int *(\varphi_1
d\varphi_2-\varphi_2 d\varphi_1)$.
Consider a  Lagrangian subspace $L\subset H$,
$L=\left\{\varphi\colon\frac{\partial}{\partial t}\,\varphi
\vrulesub{t=0}=0\right\}$.
Since $L$ is isotropic we have an embedding:
$\Sym(L)\hookrightarrow\Cl^H$.
Then the map  $F\to j_I(F)\Omega$ induces an isomorphism
$j_L\colon\,\Sym(L)\arrowsim \scrD$ (because $H_\dbC=L\oplus
H_-$).
Also, the projection along $H_-$ induces an isomorphism
$u\colon H_+\arrowsim L$.

Now consider the endomorphism of $\Sym\,L$ given by the
composition of the arrows:
$$
\spreadmatrixlines{1\jot}
\matrix
\Sym\,L &\overset{u^*}\to\longrightarrow & \Sym\,H_+\\
\null\qquad\hbox{$\lower4pt\hbox{$\ssize j_L^{-1}$}\!\!\!\nwarrow$} 
     &&\swarrow\!\!{\ssize j_{_{H_+}}}\\
&\scrD&
\endmatrix
$$
We denote this endomorphism by $F\to \Fdots\,\,$.


\proclaim{Lemma}

\roster
\item"a)" For $\ell \in L$ we have $:\phi^n(\ell):=j_I(:\ell^n:)$.
(The LHS is defined by Remark in \S 3.4).

\item"b)"
$F-\Fdots\in\oplusop_{k=0}^{n-1}\Sym^k(L)$ for  $F\in\Sym^nL$.

\smallskip
\item"c)"
$\left( j_L(\Fdots ) \, ,j_L(G)\right)_\scrD=0$ for $F\in\Sym^nL$,
 $G\in\Sym^k(L)$ $k<n$.
\endroster
\endproclaim

\proclaim{Lemma}
For any $F_1,F_2\in\Sym(L)$ we have:
$$
\left( j_L(F_1),j_L(F_2)\right)_\scrD=
\nu_{\left<\,\,,\,\,\right>\vrulesub{L}}(F_1\cdot\Fbar_2)
$$
\endproclaim

Let $\varkappa\in\End(\Sym\, L)$ be the second order
differential operator corresponding to the form
$\left<\quad,\quad\right>$.
(So we have: $\varkappa(a^k)=k(k-1)$ $\left<a,a\right>a^{k-2}$
for $a\in L$.)

\proclaim{Lemma}
$\Fdots=\exp\left(-\frac{\varkappa}{2}\right)F$.
\endproclaim

\proclaim{Corollary}
$:\exp(\ell^{\sssize \vee}):=\exp\left(-\frac{(\ell , \ell)}{2}\right)
\exp(\ell^{\sssize\vee})$, for $\ell\in L$.
\endproclaim



\centerline{\bf References}

\medskip
\ref
\by B. Simon
\book The $P(\phi)_2$ Euclidean (quantum) field theory; \S II.5
\publ Princeton University Press
\yr 1974
\endref

For a discussion of the Gaussian measure approach to free QFT see Chapter 3
of {\it loc. cit.} or 
\medskip
\ref
\by J. Glimm, A. Jaffe
\book Quantum Physics. (A functional integral point of view), \S 6.2
\publ Springer-Verlag
\yr 1987
\endref





 
\enddocument




