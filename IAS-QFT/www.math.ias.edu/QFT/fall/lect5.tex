
%% This is an AMS-TeX file.
%% The command to compile it is: amstex <file>
%%
\input amstex
\documentstyle{amsppt}
\loadeusm
\magnification=1200
\pagewidth{6.5 true in}
\pageheight{8.9 true in}

\catcode`\@=11
\def\logo@{}
\catcode`\@=13

\NoRunningHeads

\font\boldtitlefont=cmb10 scaled\magstep1
\font\bigboldtitlefont=cmb10 scaled\magstep2

\def\dspace{\lineskip=2pt\baselineskip=18pt\lineskiplimit=0pt}

%\def\plus{{\sssize +}}
%\def\oplusop{\operatornamewithlimits{\oplus}\limits}
\def\otimesop{\operatornamewithlimits{\otimes}\limits}
%\def\Piop{\operatornamewithlimits{\Pi}\limits}
%\def\w{{\mathchoice{\,{\scriptstyle\wedge}\,}
%  {{\scriptstyle\wedge}}
%  {{\scriptscriptstyle\wedge}}{{\scriptscriptstyle\wedge}}}}
\def\Le{{\mathchoice{\,{\scriptstyle\le}\,}
  {\,{\scriptstyle\le}\,}
  {\,{\scriptscriptstyle\le}\,}{\,{\scriptscriptstyle\le}\,}}}
\def\Ge{{\mathchoice{\,{\scriptstyle\ge}\,}
  {\,{\scriptstyle\ge}\,}
  {\,{\scriptscriptstyle\ge}\,}{\,{\scriptscriptstyle\ge}\,}}}
\def\vrulesub#1{\hbox{\,\vrule height7pt depth5pt\,}_{#1}}
%\def\rightsubsetarrow#1{\subset\kern-6.50pt\lower2.85pt
%     \hbox to #1pt{\rightarrowfill}}
%\def\mapright#1{\smash{\mathop{\,\longrightarrow\,}\limits^{#1}}}
%\def\arrowsim{\smash{\mathop{\to}\limits^{\lower1.5pt
%  \hbox{$\scriptstyle\sim$}}}}
\def\[[{[\![}
\def\]]{]\!]}

\def\eps{{\varepsilon}}
\def\kap{{\kappa}}

\def\ln{{\ell n}}


%\def\pvec{\vec{p}}
\def\udot{\Dot{u}}

%\def\Ebar{\overline{E}}
%\def\fbar{\overline{f}}
%\def\gbar{\overline{g}}
%\def\Sbar{\overline{S}}
%\def\Vbar{\overline{V}}
%\def\scrRbar{\overline{\scrR}}
%\def\scrSbar{\overline{\scrS}}
%\def\grSbar{\overline{\grS}}

\def\Phat{\widehat{P}}

%\def\xtil{\widetilde{x}}
\def\btil{\tilde{b}}
\def\stil{\tilde{s}}
\def\Ptil{\widetilde{P}}
%\def\scrStil{\widetilde{\scrS}}
%\def\scrTtil{\widetilde{\scrT}}
%\def\rhotil{\tilde{\rho}}
\def\mutil{\widetilde{\mu}}
\def\tautil{\widetilde{\tau}}
\def\Ltil{\widetilde{L}}
\def\grbtil{\tilde{\grb}}
\def\btil{\tilde{b}}

%\def\Stab{\text{\rm Stab}} \def\det{\text{\rm det}}
%\def\space{\text{\rm space}} \def\const{\text{\rm const}}
%\def\Vbarspace{\overline{V}_{\space}}
%\def\Vspace{V_{\space}} \def\Spec{\text{\rm Spec}}
%\def\SO{\text{\rm SO}}  \def\tr{\text{\rm tr}}
%\def\Spin{\text{\rm Spin}}
\def\tr{\text{\rm tr}}
\def\Aut{\text{\rm Aut}} \def\Hom{\text{\rm Hom}}
\def\aut{\text{\rm aut}}
\def\con{{\text{\rm con}}}
%\def\Out{\text{\rm Out}}
%\def\End{\text{\rm End}}
%\def\supp{\text{\rm supp}}
%\def\SU{\text{\rm SU}} \def\Lie{\text{\rm Lie}}
%\def\SL{\text{\rm SL}} 
\def\Sym{\text{\rm Sym}} \def\char{\text{\rm char}}
\def\QFT{\text{\rm QFT}} \def\ex{\text{\rm ex}}
\def\In{\text{\rm in}} \def\pr{\text{\rm pr}}
\def\map{\text{\rm map}} \def\irr{\text{\rm irr}}
\def\det{\text{\rm det}}

\def\dbC{{\Bbb C}} 
%\def\dbR{{\Bbb R}}
%\def\dbZ{{\Bbb Z}} 

\def\gr#1{{\fam\eufmfam\relax#1}}

%Euler Fraktur letters (German)
\def\grA{{\gr A}}	\def\gra{{\gr a}}
\def\grB{{\gr B}}	\def\grb{{\gr b}}
\def\grC{{\gr C}}	\def\grc{{\gr c}}
\def\grD{{\gr D}}	\def\grd{{\gr d}}
\def\grE{{\gr E}}	\def\gre{{\gr e}}
\def\grF{{\gr F}}	\def\grf{{\gr f}}
\def\grG{{\gr G}}	\def\grg{{\gr g}}
\def\grH{{\gr H}}	\def\grh{{\gr h}}
\def\grI{{\gr I}}	\def\gri{{\gr i}}
\def\grJ{{\gr J}}	\def\grj{{\gr j}}
\def\grK{{\gr K}}	\def\grk{{\gr k}}
\def\grL{{\gr L}}	\def\grl{{\gr l}}
\def\grM{{\gr M}}	\def\grm{{\gr m}}
\def\grN{{\gr N}}	\def\grn{{\gr n}}
\def\grO{{\gr O}}	\def\gro{{\gr o}}
\def\grP{{\gr P}}	\def\grp{{\gr p}}
\def\grQ{{\gr Q}}	\def\grq{{\gr q}}
\def\grR{{\gr R}}	\def\grr{{\gr r}}
\def\grS{{\gr S}}	\def\grs{{\gr s}}
\def\grT{{\gr T}}	\def\grt{{\gr t}}
\def\grU{{\gr U}}	\def\gru{{\gr u}}
\def\grV{{\gr V}}	\def\grv{{\gr v}}
\def\grW{{\gr W}}	\def\grw{{\gr w}}
\def\grX{{\gr X}}	\def\grx{{\gr x}}
\def\grY{{\gr Y}}	\def\gry{{\gr y}}
\def\grZ{{\gr Z}}	\def\grz{{\gr z}}


\def\scr#1{{\fam\eusmfam\relax#1}}

\def\scrA{{\scr A}}   \def\scrB{{\scr B}}
\def\scrC{{\scr C}}   \def\scrD{{\scr D}}
\def\scrE{{\scr E}}   \def\scrF{{\scr F}}
\def\scrG{{\scr G}}   \def\scrH{{\scr H}}
\def\scrI{{\scr I}}   \def\scrJ{{\scr J}}
\def\scrK{{\scr K}}   \def\scrL{{\scr L}}
\def\scrM{{\scr M}}   \def\scrN{{\scr N}}
\def\scrO{{\scr O}}   \def\scrP{{\scr P}}
\def\scrQ{{\scr Q}}   \def\scrR{{\scr R}}
\def\scrS{{\scr S}}   \def\scrT{{\scr T}}
\def\scrU{{\scr U}}   \def\scrV{{\scr V}}
\def\scrW{{\scr W}}   \def\scrX{{\scr X}}
\def\scrY{{\scr Y}}   \def\scrZ{{\scr Z}}
\def\hatP{{\widehat{P}}}
\def\hatQ{{\widehat{Q}}}
\def\Qhat{{\widehat{Q}}}
\def\ten{\otimes}
\def\hatd{{\hat{d}}}
\NoBlackBoxes
\document

\centerline{\boldtitlefont Lecture 5.}

\medskip
\centerline{\bigboldtitlefont Feynman graphs}

\bigskip
\centerline{David Kazhdan}


\dspace
\bigskip\bigskip


\subhead{5.1}{Feynman graph expansion}\endsubhead
The technique of Feynman graphs allows one to write down an
asymptotic series for functional integral of a $\QFT$ in a
neighborhood of a free QFT (see Witten's lecture 1). 
Nevertheless, the technique itself can be applied to a
purely finite dimensional integral.
In this lecture we will discuss this application only.

Let $V$ be a finite dimensional real vector space, let $V'$
be its dual.
We will view elements of symmetric algebra $\Sym(V')$ as
polynomial functions on $V$.
We fix a nondegenerate positive definite quadratic form
$b\in\Sym^2(V')$.
Let $b^{-1}\in\Sym^2(V)$ be the corresponding quadratic form
on $V'$.

Let $\mu_0=e^{-b^{-1}/2}dv$ be the Gaussian measure on $V'$; we
have $\scrF(\mu_0)=e^{-b/2}$.

Let $P$ be a polynomial function on $V'$.
We want to study the ``perturbed'' measure $\mu=e^{\eps P}\mu_0$.
This is a well-defined measure if $\eps>0$, and $P$ is
negative.


We are going to write down an asymptotic expansion for the Fourier
transform $\scrF(\mu)$.

More precisely, we will obtain a formal Taylor series
$\mutil \in \dbC [[V', \eps]]$ in the following way.

Let $D(P)$ be the differential operator with constant
coefficients on $V$ corresponding to $P$. More precisely, $D$ is a
homomorphism of the algebra of polynomial functions on $V'$ 
to the algebra of  differential operators with constant
coefficients on $V$, such that $D(x)=i \partial /\partial x$ for $x \in V$. 

We write:
$$
\scrF(e^{\eps P}\mu_0)=e^{\eps D(P)}(e^{-b/2})
=\sum\limits_{n=0}^\infty \tfrac{1}{n!}\eps^n D(P)^n
(e^{-b/2})\,\,.
$$
This is obviously an element of $\dbC [[V', \eps]]$.



This formal power series is connected with the original
analytical problem via the notion of Borel summability.
However in this lecture we will be interested in the formal
expression only.

It is notationally convenient to perform ``Wick rotation''. 
So we consider the series
$$
\mutil (v)= \scrF(e^{\eps P}\mu_0)(iv)
=\sum\limits_{n=0}^\infty \tfrac{1}{n!}\eps^n \Phat^n
(e^{b/2})\,\,.
$$
where $\Phat = D(P(-ix))$, so that $\widehat x = \partial /\partial x$.

The series $\mutil$  is defined for arbitrary nondegenerate quadratic form
 $b$, and a polynomial function $P$ over any field of
characteristic $0$. (The condition on $b$ to be positive definite
is of course irrelevant in this setting.)




There exists a  combinatorial way to ``compute''
$\mutil$.
To describe it we first fix some notations.

By a graph we will mean the data consisting of: \
two sets $\Gamma_1$ (edges) and $\Gamma_0$ (vertices) and a
map $j\colon\,\Gamma_1\to\Gamma_0\times\Gamma_0$.
We also assume an involution $\sigma$ of $\Gamma_1$ is fixed
such that $j(\gamma)=(x,y)\Leftrightarrow j(\sigma(\gamma))=
(y,x)$, and $\sigma$ has no fixed points.
We will only consider graphs with no isolated vertices (i.e.
$\pr_1\circ j$ is surjective).
An edge of a graph is an element of $\Gamma_1/\sigma$.
For a vertex $\gamma_0\in\Gamma_0$ we define its ``star'' as
$\Gamma(\gamma_0)=\{\gamma_1\in\Gamma_1\vert \;
\pr_2\circ j(\gamma_1)=\gamma_0\}$.
The set of external vertices is $\Gamma_{\ex}=\{\gamma_0\in
\Gamma_0\vert \#(\Gamma(\gamma_0))=1\}$, and the set of
inner vertices is $\Gamma_{\In}=\Gamma_0-\Gamma_{\ex}$.
We put: \ $L(\gamma_0)=\Sym^{\#(\Gamma(\gamma_0))}(V)$;
$\Ltil(\Gamma)=\otimesop_{\gamma_0\in\Gamma_0}L(\gamma_0)$;
$L(\Gamma)=\otimesop_{\gamma\in\Gamma_{\In}}L(\gamma)$.

Assume that for any $\gamma_0\in\Gamma_0$ we are given an
element $s(\gamma_0)\in L(\gamma_0)$.
We then take $\ell_s=\otimesop_{\gamma_0\in\Gamma_0}
s(\gamma_0)\in\Ltil(\Gamma)$.
For any map $\stil\colon\,\Gamma_0\to V$ we define
$s_{\stil}(\gamma_0)=
\left(\stil(\gamma_0)\right)^{\#(\Gamma(\gamma_0))}\in
L(\gamma_0)$, and denote $\ell_{\stil}=\ell_{s_{\stil}}$.

\proclaim{Lemma 1}
There exists a unique linear function $\tautil_b^\Gamma$ on
$\Ltil(\Gamma)$ such that for any map $\stil\colon\,\Gamma_0\to
V$ we have:
$$
\tautil_b^\Gamma(\ell_{\stil})=
\tfrac{1}{\#\Aut(\Gamma)}\prod\limits_{\gamma_1\in\Gamma_1/\sigma}
b\left(\stil(\pr_1\circ j(\gamma_1)),\stil(\pr_2\circ
j(\gamma_1))\right)\,\,.
$$
\endproclaim

\medskip\noindent
Proof is clear.

We can view $\tautil_b^\Gamma$ as a map
$\tau_b^\Gamma\colon\,L(\Gamma)\to\Sym^{\#\Gamma_{\ex}}(V')$.

\example{Example}
$\Gamma_n$ is a disjoint union of $n$ copies of the graph: \
$\bullet\!\raise1.85pt
\hbox to 20pt{\hrulefill}\!\bullet\,\,\,$.
(The nonoriented edge on the picture corresponds to two
elements of $\Gamma_1$ permuted by $\sigma$.)
Then $L(\Gamma)=\dbC$  canonically, because
$\Gamma_{\In}$ is empty.
We have:
$$
\tau_b^{\Gamma_n}(1)=\tfrac{b^n}{2^n n!}\,\,.
$$

Note the equality $\sum\limits_{n=0}^\infty
\tau_b^{\Gamma_n}(1)
=e^{b/2}$ where $\Gamma _0$ is the empty graph.


Let $P$, $\mutil$ be as above.
We write $P=\sum\limits_{n\Ge 2}
P_n /n!$ where $P_n$ 
is homogeneous of degree $n$ (thus we assume for
convenience that $P$ contains no constant or linear term).
\endexample

\proclaim{Theorem}
We have an equality of formal series:
$$
\mutil=\biggl(\,\sum\limits_{\Gamma\in\grg}\eps^{\#\Gamma_{\In}}
\tau_b^\Gamma(s_P)\biggr)   e^{b/2}           %\mutil_0
$$
Here $\grg$ is the set of isomorphism classes of graphs and
$s_P(\gamma_0)=P_{\#\Gamma(\gamma_0)}$.
\endproclaim

\demo{Proof} [Another proof appears in \S 1.3 of Witten's lecture 1].
 
The idea of the proof is as follows. We want
to prove an equality between 
two expressions, which are power series in $\eps$
and in $P_n$'s. We will show
that both the left hand side and the right hand side
satisfy the same system of linear differential equations
in $P_n$'s as
variables, and then we will check
that the initial conditions coincide.

So, let $\Gamma$ be a graph and let $\lambda\subset \Gamma_{\ex}$,
be such that $|\lambda|=n$.
We denote by $\Gamma^{\lambda}$ the ``cone over $\lambda$''. 
This means that
$\Gamma^{\lambda}$ is a graph, such that $\Gamma_0^{\lambda}=
\Gamma_0\cup \{*\} -\lambda$ and
$\Gamma^{\lambda}_1=\Gamma_1-\{$all
edges of $\Gamma$ with an end in $\lambda\}
\cup\{$all edges from $*$ to vertices connected
with $\lambda\}$.   (We add all edges connecting $*$
to a vertex in $\lambda$, and then erase the vertices belonging
to $\lambda$.)
Assume now 
that $n>1$. Then $*$ is an internal vertex of $\Gamma^{\lambda}$.

{}For any $d\in S^n(V)$ and $s\in L(\Gamma)$ we define 
$s_{\lambda}^d=d\ten s\in L(\Gamma^{\lambda})$.

\proclaim{Lemma 2}
For any $d\in S^n(V)$, $\Gamma\in\grg$ and $s\in L(\Gamma)$ we have
%\eq{lemma}
$$
{\hatd\over n!}(\tau_b(\Gamma)(s))
=\sum_{\lambda\subset\Gamma_{\ex},|\lambda| =n}
\tau_b(\Gamma^{\lambda})(s_{\lambda}^d)
$$
%\end{equation}
(Here $\hatd$ is the differential operator with constant 
coefficients on $V$ corresponding\break
to $d$.)
\endproclaim

Now we can prove the theorem. Let $l(P)$ denote the left hand side 
 and let $r(P)$ be the right hand side of the
equality. 
We want to write down a certain system of differential equations
in $P_n$'s as variables, 
which will be satisfied by both sides. 
It is easy to see from the definitions
that $l(P)$ satisfies the system of equations
$$
{\partial l(P)\over\partial P_n}(Q)=(\eps{\Qhat}/ n!)l(P)
$$
where $Q$ is an arbitrary polynomial 
of degree $n$ on $V'$ and $\Qhat$
is the corresponding differential operator on $V$; 
in the left hand side
$Q$ is considered as a tangent vector to the space of polynomials
of degree $n$.
 
But now Lemma 2 tells us precisely that the right hand side of this
expression is equal to ${\partial r(P)\over\partial P_n}(Q)$. 

Therefore $l(P)$ and $r(P)$ satisfy the same system of differential
equations. Let's now check the initial conditions. 
By the definition, $l\vrulesub{P=0}=e^{b/2}$. 
On the other hand, one also has $r\vrulesub{P=0}=e^{b/2}$ by
the above Example. Hence
we have $l(P)=r(P)$ for any $P$, so the theorem is proved.
\enddemo

\remark{Remark 1} 
The theorem and its proof along with all the statements
below can be immediately generalized in the following way. 
We can take $P$
to be an element of $\dbC[V]\[[\eps\]]$ rather 
than a polynomial, i.e. 
$P=\sum\nolimits_{m=0}^\infty \eps^m P_{(m)}$ is a 
Taylor series in $\eps$ 
with polynomial functions on $V'$ as coefficients.  
\endremark

\remark{Remark 2}
This  way of writing an asymptotic expansion for a measure 
 uses Fourier transform, and hence the linear structure of
$V$ heavily.
For  example when one applies    the Feynman graph expansion to the 
gauge theory, $V$ being the space of connections modulo gauge
transformation,  then $V$ has no natural linear structure.
In this case one still can write an asymptotic expansion 
for functional integral, but an 
individual term assigned to a particular
graph is not canonically defined. 
\endremark
\medskip

We denote $Z=\sum\nolimits_{\Gamma\in\grg}\tau_b^\Gamma(s_P)
\eps^{\#\Gamma_{\In}}$.

One of the difficulties in using this expansion is that there
are ``too many'' terms; so we try to reduce it somehow.
We denote $F=\sum\nolimits_{\Gamma\in\grg_{\con}}\tau_b^\Gamma
(s_P)\eps^{\#\Gamma_{\In}}$, where $\grg_{\con}$ is the set
of connected nonempty graphs.

Note that $F$ (and all series that will be introduced below)
lie in a subring $\dbC[V']\[[\eps\]]\subset \dbC \[[V',\eps\]]$.


\proclaim{Claim 1}
$F=\log\,Z$.
\endproclaim

\demo{Proof} Decomposing a graph into union of its irreducible components
one can identify the two combinatorial expressions:
$$Z= \sum\limits _{\Gamma\in \grg} \tau _b^\Gamma
=  \sum\limits _{\Gamma_1,..,\Gamma_k \in \grg_\con ; n_1,..,n_k}
\frac {1}{n_1!n_2!..n_k!}\tau_b^{\Gamma_1}\dotsc \tau_b^{\Gamma_k}=
 \exp(F)$$
where the summation is over all sets of  distinct elements $\Gamma_i\in
\grg_\con$.  
\enddemo

We next want to study the ``quasi-classical approximation''
to our integral.
So we introduce another variable $\hbar$ and consider the
expression:
$$
F_{\hbar}(v)=\log\,\scrF\left(e^{(b^{-1}/2+\eps
P)/\hbar}dv\right)(\hbar v)\,\,.
$$
As an immediate corollary to the Theorem, we get:

\proclaim{Lemma 3}
$F_{\hbar}=\hbar^{-1}\sum\nolimits_{\Gamma\in\grg_{\con}}
\hbar^{h^1(\Gamma)}\tau_b^\Gamma(s_P)\eps^{\#\Gamma_{\In}}$
where $h^1(\Gamma)= \dim (H^1(\Gamma))= \# \Gamma_1/\sigma -\# 
\Gamma_0$. 
\endproclaim

\demo{Proof}
Apply the same formal procedure to the quadratic form
$\btil=\hbar b$ and polynomial $\Ptil=\hbar^{-1}P$, and note that for a
connected graph $\Gamma$ we have $\tau _{\hbar b}^\Gamma
(s_{P/\hbar})\left( \hbar v\right) = \hbar ^{h^1(\Gamma)-1} \tau _b^\Gamma
(s_P) (v)$.
\enddemo

\subhead{5.2} {Quasi-classical (low-loop) approximations}\endsubhead
Recall  that the ``classical''
approximation to Fourier transform is the Legendre transform. An appropriate
version of the definition of Legendre transform is as follows. 

If $f$ is a function on $V'$, then we can view its differential $df$ as a map
$df\colon\,V'\to V$.

If $d f$ is an isomorphism then the Legendre transform of $f$ is defined by
$$L(f)(v)= \left< v , (df)^{-1}(v) \right> - f\left( (df)^{-1}(v)\right)$$
i.e. $L(f)(v)$   is the critical value of the function $v-f$
 where $v$ is considered as a linear function on $V'$.

Let now $G\in \dbC \[[V, \eps\]]$ be a formal power series. Then its
differential $dG$ (in the $V'$-direction) 
is an element of $\Hom (V',V)\ten  \dbC \[[V, \eps\]]$.

Assume that   $G=g_0 +g_1+\eps G_1$ where $g_0$ is a nondegenerate
quadratic form on $V'$, and $g_1$ contains no terms of degree less then 3
in $V$.
 Then  $dG$ is invertible, i.e. there exists
 $H=(dG)^{-1}\in  \Hom (V,V')\ten  \dbC \[[V,\eps\]]$ with no constant term 
 such that $dG\circ
H=id$ and $H\circ dG = id$. In this situation the Legendre transform
$L(G) \in \dbC [[V', \eps]]$ can be defined by the same formula.

To describe the next term of the asymptotics 
we define for any function $f$
on $V'$ as above a new function $H(f)$ on $V$ as follows. Let
$Hess_f(p)=\det \bigl(\frac {\partial ^2 f}{\partial v_i \partial
v_j}\bigr)|_p$ 
be the Hessian of $f$ (we assume that a constant volume form 
on $V$ is fixed so that the
determinant of the quadratic form is taken with respect to that volume form).

We put:  
$H(f)(v)=Hess\left( (df)^{-1}(v)\right)$.
  
For $G=g_0 +g_1+\eps G_1 \in \dbC [[V', \eps]]$ as above we can define
$Hess(G)$ and $H(G) \in \dbC [[V', \eps]]$ by the same formula.

\proclaim{Claim 2} 
\roster
\runinitem"a)"
Let $F_0=\sum\nolimits_{\Gamma\in T}\tau_b^\Gamma(s_P)
\eps^{\#\Gamma_{\In}}$, where $T$ is the set of (nonempty) trees.
Then $F_0$ is the  Legendre transform of
$\frac{b^{-1}}{2}-\eps P$.

\smallskip
\item"b)" 
Let $F_1 =\sum\limits_{\Gamma\in \grg _1}\tau_b^\Gamma(s_P)
\eps^{\#\Gamma_{\In}}$ where $\grg _1$ is the 
set of one-loop connected
graphs (i.e. graphs with $h^0(\Gamma)=h^1(\Gamma)=1$. Then 
$F_1=\log\bigl(H\bigl(\frac{b^{-1}}{2}-\eps P\bigr)^{-1}\bigr)$. 
{\rm (We assume that
the background volume form is such 
that $\det (\frac{b^{-1}}{2})=1$. Then   
$H(\frac{b^{-1}}{2}-\eps P) \in 1 
+\eps \dbC \[[V', \eps\]] +V'\dbC \[[V',\eps\]]$, 
and  $\log (H(\frac{b^{-1}}{2}-\eps P))$ is a well-defined
element of  $\dbC \[[V', \eps\]]$).}
\endroster
\endproclaim


\demo{Proof of  the Claim} From  Lemma 3 we see
that $F_{\hbar}( v)= F_0(v)/\hbar +F_1(v)+ O(\hbar)$.
Thus  the claim follows from the stationary phase approximation applied
to the integral
 $$
\scrF(f)(iv)=\frac{1}{(2\pi)^{d/2}}\int\limits_{V'}
e^{(v,p)+(\eps P-b^{-1}/2)(p)}dp
$$
 
\medskip

One can also  directly show the equality of the two combinatorial 
expressions:
$$
\align
F_0 \left(d(\frac{b^{-1}}{2}-\eps P)|_p \right) &=
\frac{b^{-1}}{2}(p) -
\eps\sum\limits_n \frac{P_n}{(n-1)!} (p)
+\eps\sum\limits_n\frac{P_n}{n!}(p)\\
&=\langle p,d\Bigl(\frac{b^{-1}}{2}-\eps P\Bigr)|_p\rangle - 
(b^{-1}/2-\eps P)\Bigl(d(\frac{b^{-1}}{2}-\eps P)\Bigr)\tag5.1
\endalign
$$
and
$$
F_1\Bigl(d(\frac{b^{-1}}{2}-\eps P)|_p \Bigr)=
-\log Hess\Bigl(\frac{b^{-1}}{2}-\eps P\Bigr)\tag5.2
$$

Here (5.1) is equivalent to $F_0=L\bigl(\frac{b^{-1}}{2}-\eps
P\bigr)$ and (2) is equivalent to $F_1=\log \bigl(
H\bigl(\frac{b^{-1}}{2}-\eps P\bigr) ^{-1}\bigr)$.

Let us sketch the combinatorial proof of (5.1) and (5.2). 

Let us identify $V$ and $V'$ by means of $b$ and  use an 
equality $\eps^{\#\Gamma_{\In}}\tau _b^\Gamma(s_P)\circ
d\bigl(\frac{b^{-1}}{2}-\eps 
P\bigr)=\sum (-1)^{c_i}\eps^{\#\Gamma^i_{\In}} 
\tau_b^{\Gamma^i}(s_P)$ 
where the graph $\Gamma^i$ is obtained from the graph $\Gamma$
by adding $c_i$ vertices of arbitrary valencies, 
each one of which is connected
with an external vertex of $\Gamma$ in such a way that any  
external vertex of $\Gamma$ 
is connected with no more then one new vertex.

Now it is not hard to check that in the expansion of $F_0 \circ
\bigl(d\bigl(\frac{b^{-1}}{2}-\eps P\bigr)\bigr)$ 
all terms with $\eps ^i$ for
$i\Ge 2$ cancel, and 
identify the sum of remaining terms with the RHS of (5.1). 

Likewise in the expansion of 
$F_1  \circ \bigl(d\bigl(\frac{b^{-1}}{2}-\eps
P\bigr)\bigr)$ only the terms with one-particle irreducible 
one-loop graphs
(``circles'') do not cancel; the sum 
of these terms is identified with
$\sum\nolimits_{n=1}^\infty\eps^i\tr{(dP)^n}/n=-\log\det(Id - dP)$
which coincides with the RHS of (5.2).
\enddemo

\subhead{5.3} {Effective potential}\endsubhead
Let us call a nonempty graph {\it $1$-particle irreducible} (or just
 {\it 1-irreducible}) if it is
connected and remains connected after removal of any
internal edge.
To any connected graph there corresponds a unique tree with
a $1$-particle irreducible graph assigned to each vertex
together with an identification of the set of edges coming
to a vertex of a tree with the set of external vertices of
the corresponding graph.
Thus ``computation'' of $F_0$ can be reduced to summation
over trees and over $1$-particle irreducible graphs
separately.
Denote
$F_{1-\irr}=\sum\nolimits_{\Gamma\in\grg_{1-\irr}}
\eps^{\#\Gamma_{\In}}
\tau_b^\Gamma(s_P)$ where 
$\grg_{1-\irr}$ is the set of 1-particle irreducible
graphs. (Notice that the graph \ 
$\bullet\!\raise1.85pt\hbox to 20pt{\hrulefill}
\!\bullet\,\,\,$
is 1-particle irreducible by the definition).
 
\proclaim{Claim 3} 
$F_{1-\irr}=b-b^*(L(F))$ where $b$ is viewed as a map from $V$ to
$V'$ and $L$ is the Legendre transform.
\endproclaim

\demo{Proof}
By  Remark 1 after the proof of the theorem we can apply 
claim 2 to $P'=\eps ^{-1}(b^{-1})^*( F_{1-\irr}-b/2)
\in \dbC [V']\[[\eps\]]$.



From the  above combinatorial observation
we see that $F= F_P = (F_0)_{P'}$. Hence claim 2
implies that $F=L(b^{-1}/2 - \eps P')=L\left((b^{-1})^*(b-
F_{1-\irr})\right)$. Since Legendre transform
is involutive we get the claim. 
\enddemo

$F_{1-\irr}$ is called the effective potential of the
theory.



\enddocument





