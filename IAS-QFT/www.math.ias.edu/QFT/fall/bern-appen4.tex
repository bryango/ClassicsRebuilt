%From: Pierre Deligne <deligne@IAS.EDU>
%Date: Mon, 6 Jan 1997 14:42:07 -0500
%Subject: fourth appendix

\input amstex
\documentstyle{amsppt}
\magnification=1200
\loadbold
\loadeusm

\font\boldtitlefont=cmb10 scaled\magstep2

\NoRunningHeads
\pagewidth{6.5 true in}
\pageheight{8.9 true in}

\catcode`\@=11
\def\logo@{}
\catcode`\@=13

\def\eps{{\varepsilon}}

\def\dbR{{\Bbb R}}

\def\undertext#1{$\underline{\vphantom{y}\hbox{#1}}$}
\def\nspace{\lineskip=1pt\baselineskip=12pt%
     \lineskiplimit=0pt}
\def\dspace{\lineskip=2pt\baselineskip=18pt%
     \lineskiplimit=0pt}

%\def\upvee{{\sssize \vee}}
%\def\plus{{\sssize +}}
\def\wedgeop{\operatornamewithlimits{\wedge}\limits}
\def\w{{\mathchoice{\,{\scriptstyle\wedge}\,}
  {{\scriptstyle\wedge}}
  {{\scriptscriptstyle\wedge}}{{\scriptscriptstyle\wedge}}}}
\def\Le{{\mathchoice{\,{\scriptstyle\le}\,}
{\,{\scriptstyle\le}\,}
{\,{\scriptscriptstyle\le}\,}{\,{\scriptscriptstyle\le}\,}}}
\def\Ge{{\mathchoice{\,{\scriptstyle\ge}\,}
{\,{\scriptstyle\ge}\,}
{\,{\scriptscriptstyle\ge}\,}{\,{\scriptscriptstyle\ge}\,}}}
%\def\mapright#1{\smash{\mathop{\,\longrightarrow\,}%
%     \limits^{#1}}}
%\def\rmapdown#1{\Big\downarrow\kern-1.0pt\vcenter{
%     \hbox{$\scriptstyle#1$}}}
%\def\arrowsim{\smash{\mathop{\,\longrightarrow\,}%
%   \limits^{\lower1.5pt\hbox{$\scriptstyle\sim$}}}}
%\def\leftarrowsim{\smash{\mathop{\,\longleftarrow\,}%
%   \limits^{\lower1.5pt\hbox{$\scriptstyle\sim$}}}}

\def\red{\text{\rm red}}
%\def\Sym{\text{\rm Sym}} 
%\def\Hom{\text{\rm Hom}} 
%\def\Tor{\text{\rm Tor}} \def\Gr{\text{\rm gr}}
%\def\Ext{\text{\rm Ext}} \def\ev{\text{\rm ev}}
%\def\I{\text{\bf I}} \def\Ber{\text{\rm Ber}}
%\def\II{\text{\bf II}} \def\ber{\text{\rm ber}}

\def\bfs{\bold{s}}

%Script letters:
\def\scr#1{{\fam\eusmfam\relax#1}}
\def\scrE{{\scr E}} 
\def\scrL{{\scr L}} 
\def\scrM{{\scr M}} 
\def\scrO{{\scr O}} 

\def\scrLbar{\overline{\scrL}}


\topmatter
\title\nofrills
{\boldtitlefont Appendix:}\\
\hbox{\boldtitlefont Signs: \ 
  de Rham and integral forms complexes}
\endtitle
\author
P. Deligne
\endauthor
\endtopmatter

\NoBlackBoxes
\parindent=20pt
\frenchspacing
\document
\bigskip
\dspace
\subhead
1
\endsubhead
Some standard conventions for ordinary manifolds don't
sit well with the principle of treating $p$-forms as
objects of parity $p$.
Here are troublesome conventions:

\smallskip\noindent
(a) \ for a $p$-form $\alpha_1\w\ldots\w\alpha_p$, and
vector fields $X_1,\dotsc,X_p$, 
$$
\alpha_1\w\ldots\w\alpha_p(X_1,\dotsc,X_p)=\det(\alpha_i(X_j)),
$$
with typical term $\alpha_i(X_1)\ldots\alpha_p(X_p)$,
despite the fact that $X_i$ passed over $\alpha_j$ for
$i<j$.

\smallskip\noindent
(b) \ for $f$ a function and $X$ a vector field, 
$$
Xf=df(X)=i_Xdf=X\bot df.
$$
In the middle equality, $df$ and $X$ are permuted
without sign consequence.

\smallskip\noindent
(c) \ In $\dbR^n$, the density corresponding to the
$n$-form $dx_1\w\ldots\w dx_n$, when $\dbR^n$ is
oriented by $(e_1,\dotsc,e_n)$, is positive.

\smallskip
The bigraded points of view, by keeping separate the
``cohomological'' grading, used above, and the parity,
allows to keep those formuli while treating parity
consistently.

\subhead
2
\endsubhead
Let $M$ be a supermanifold.
The vector fields on $M$ are the derivations of the
structural sheaf $\scrO$:
$$
D(fg)=Df.g+(-1)^{p(D)p(f)}fDg
$$
The sheaf of $1$-forms is the dual, the duality being
defined by
$$
T\otimes\Omega^1\to\scrO\colon\,
D\otimes\alpha\longmapsto i_D\alpha.
$$
One defines
$$
\alpha(D)=(-1)^{p(D)p(\alpha)}i_D\alpha.
$$
To the extent possible, it is best to consider only
vector fields and forms of even parity, considering
instead of an odd $D$ (resp. $\alpha$) the product $\eps D$
(resp. $\eps\alpha$) for $\eps$ an auxiliary odd
constant.
If we work over a basis $B$, as we should, odd constant
means odd function on $B$.

The bilinearity of $i_D\alpha$ and of $\alpha(D)$ is as
dictated by the sign rule: with $(uD)(f)=u(Df)$, we have
$$
\alignat2
i_{fD}\alpha &=fi_D\alpha\,\,, &\qquad   i_Df\alpha
  &=(-1)^{p(D)p(f)}fi_D\alpha\\
(f\alpha)(D) &=f(\alpha(D))\,\,, &\qquad  \alpha(fD)
  &=(-1)^{p(\alpha)p(f)}f\alpha(D).
\endalignat
$$

The de Rham complex $\Omega^*$ is $\wedge \Omega^1$,
with $\Omega^p=\wedgeop^p \Omega^1$ in cohomological
degree $p$, and with a differential $d$ characterized by
the following identities

\medskip
\item{(1)}
$d$ is an even derivation of cohomological degree one;

\item{(2)}
from $\scrO$ to $\Omega^1$, $d$ is defined by\hfill\break
\null\qquad\qquad $i_Ddf=Df$;

\item{(3)}
$d^2=0$.

\medskip\noindent
One may prefer to say that $\Omega^*$ is the symmetric
algebra in the bigraded sense on $\Omega^1$, put in
cohomological degree $1$.
The defining formula (2) can be rewritten
$df(D)=(-1)^{p(f)p(D)}Df$.

\subhead
3
\endsubhead
An even derivation $D$ generate (at least locally) a
$1$-parameter group of automorphisms $\exp(tD)$, with
$$
Df=\partial_t(\exp(tD)^*f)\quad\text{at}\quad
t=0.\tag3.1
$$

By transport of structures, this group acts on all kinds
of tensor fields, and the Lie derivative $\scrL_D$ is
defined by the right side of (3.1).
If $D$ is odd, one defines $\scrL_D$ by
$$
\eps\scrL_D=\scrL_{\eps D}
\tag3.2
$$
for $\eps$ an odd constant.

One should beware that $f\mapsto\exp(tD)^*f$ is the
inverse of the action of $\exp(tD)$ on functions by
transport of structures.
Because of this,if derivations are viewed as forming the
Lie algebra of a group of diffeomorphisms, with
$D\mapsto$ infinitesimal diffeomorphism $\exp(\eps D)$,
i.e. $D$ generating the one-parameter subgroup
$\exp(tD)$, the bracket in this Lie algebra is the
opposite of the bracket of vector fields.

Instead of considering $\exp(tD)$ for $t\in\dbR$, one
can work $\mod\,t^2$: \ $\exp(tD)$
is then the family of automorphisms parametrized by
the scheme with coordinate ring $\dbR[t]/(t^2)$ defined
by
$$
\exp(tD)^*f:= f+tDf\qquad\qquad\text{(for $t^2=0$)}
\tag3.3
$$
and the Lie derivative of any tensor $g$ is given by 
$$
\exp(tD)^*g=g+t\scrL_D g\qquad\qquad\text{(for $t^2=0$)}
\tag3.4
$$
For $D$ an odd derivation, the same formuli hold when
$t$ is taken to be odd.

A vector field $D$ defines a derivation $i_D$ of
$\Omega^*$, of cohomological degree $-1$ and parity that
of $D$, characterized as extending the already defined
$i_D$ on $\Omega^1$.
Cartan's formula for the Lie derivative acting on
$\Omega^*$ is
$$
\scrL_D=i_D d+di_D.\tag3.5
$$
As $d$ is of parity $0$ and $i_D$ and $d$ are of odd
cohomological degree, the second member is the bracket
$[i_D,d]$ in the bigraded sense.

\subhead
4
\endsubhead
Let $V$ be a vector space of dimension $(0,1)$, with
generator $D$.
The free Lie aglebra generated by $V$, i.e. by $D$, has
$D,D^2=\frac12[D,D]$ as basis.
The corresponding Lie group has as points over a base
$B$ the $\exp(tD^2+\theta D)$, $t$ and $\theta$ being,
respectively, even and odd functions on $B$, with the
group law given by the Campbell-Hausdorff formula:
$$
(t_1,\theta_1)*(t_2,\theta_2)=(t_1+t_2-\theta_1\theta_2,\theta_1
+\theta_2).
\tag4.1
$$
The minus sign comes from
$\frac12[\theta_1D,\theta_2D]=-\frac12\theta_1\theta_2
[DD]=-\theta_1\theta_2 D^2$.

An odd vector field on a supermanifold $M$ generates an
action of that group.
More precisely, if $D^2$ is the vector field defined by
$DDf=D^2f$, i.e. $D^2=\frac12[D,D]$, the automorphisms
$\exp(tD^2+\theta D)$ (at least locally defined) form a
group, with
$$
\exp(t_1D^2+\theta_1 D)\exp(t_2D^2+\theta_2D)=
\exp((t_1+t_2+\theta_1\theta_2)D^2+(\theta_1+\theta_2)D).
\tag4.2
$$
The reason for the change of sign from (4.1) is
explained in 3.

\subhead
5
\endsubhead
A coordinate system $x^1,\dotsc,x^n$ (first $p$ even,
last $n-p$ odd) defines derivations $\partial_i$, with
$\partial_i x^j=\delta_i^j$.
Our conventions so far force the unusual looking
$$
df=\sum dx^i\partial_i f.\tag5.1
$$

\subhead
6
\endsubhead
The bundle of densities is defined by the property that
its sections are in duality with the space of functions
with compact support.
The duality pairing is written $\int \mu f$.
Accordingly, the $\scrO$-module structure is given by
$$
\int a\mu.f=(-1)^{p(a)p(\mu)}\int \mu.af\tag6.1
$$
According to the sign rule,
$$
\int f\mu:=(-1)^{p(f)p(\mu)}\int\mu f,\tag6.2
$$
and one also has $\int f(a\mu)=\int(fa)\mu$.

By ``density'', we mean here $C^\infty$-density: \ if,
in a local coordinate system $(x_1,
\dotsc,x_p$,\break
$\theta_1,\dotsc,\theta_q)$, $f$ is expanded as
$\sum\theta^I f_I$, we consider only the linear forms
$f\mapsto \int\mu f$ which are linear combinations of the
$\int
f_I(n_1,\dotsc,n_p)\varphi(x_1,\dotsc,x_p)dx_1\ldots
dx_p$ for $C^\infty$ functions $\varphi$.

A local coordinate system
$x_1,\dotsc,x_p,\theta_1,\dotsc,\theta_q$ defines a
density $[x_1,\dotsc,x_p,\theta_1,\dotsc,\theta_q]$ on
its domain of definition: \ for $f_{[1,q]}$ the (post)
coefficient of $\theta_1\ldots\theta_q$, as above, the
integral 
$$
\int f_{[1,q]}(x_1,\dotsc,x_p)dx_1\ldots dx_p.
\tag6.3
$$
It is the integral, for the density $dx_1\ldots dx_p$,
of the restriction to $\theta_1=\ldots=\theta_q=0$ of
$\partial_{\theta_q}\ldots\partial_{\theta_1}f$.
It is of parity $(-1)^q$ (and cohomological degree $0$).
It is a basis of densities, which is hence of dimension
$(1,0)$ for $q$ even and $(0,1)$ for $q$ odd.

\noindent
Warning. \ Densities $\mu_1$ on $M_1$ and $\mu_2$ on
$M_2$ give a density $\mu_1\boxtimes\mu_2$ on $M_1\times
M_2$ by
$$
\int\nolimits_{M_1\times M_2}\mu_1\boxtimes\mu_2
f_1\boxtimes f_2=(-1)^{p(f_1)p(\mu_2)}\int
\mu_1 f_1\int\mu_2 f_2.
$$
One has, in local coordinates,
$$
\split
[x_1,\dotsc, &x_{p(1)},\theta_1,\dotsc,\theta_{q(1)}]\boxtimes
[y_1,\dotsc,y_{p(2)},\eta_1,\dotsc,\eta_{q(2)}]=\\
&(-1)^{q(2)q(1)}[x_1,\dotsc,x_{p(1)},y(1),\dotsc,y_{p(2)},
\theta_1,\dotsc,\theta_{q(1)},\eta_1,\dotsc,\eta_{q(2)}].
\endsplit
$$

The densities given by two coordinate systems are
related by the Berezinian of a jacobian matrix, and a
sign: \ $+$ if the orientations given by
$x_1,\dotsc,x_p$ and $y_1,\dotsc,y_p$ on the reduced
space agree, $-$ otherwise.
This is best expressed by telling that the bundle of
densities is canonically isomorphic to the tensor
product Ber$\otimes$or of the Berezinian of $\Omega^1$
(cohomological degree $p$, parity $q$) by the
orientation line bundle (cohomological degree $-p$).

\subhead
7
\endsubhead
The generalized functions (distributions) on a
supermanifold can be defined as the dual of densities
with compact support, or as a suitable completion of the
space of functions.
It makes sense to take the pull back $f(g)$ of a
generalized function $f$ by a submersion $g$.

\subhead
Example
\endsubhead
Let $Y$ be the function on $\dbR$ with value $1$ for
$x<0$ and $0$ for $x>0$.
On $\dbR^{1,2}$, with coordinates
$(t,\theta_1,\theta_2)$, we have
$$
Y(t+\theta_1\theta_2)=Y(t)+\left(\tfrac{d}{dt}\,Y\right)(t).
\theta_1\theta_2=Y(t)-\delta(t).\theta_1\theta_2.
\tag7.1
$$
The integration up to $0$ of a density $\mu$ with support
bounded below is
$$
\int\nolimits_{-\infty}^0\mu:=\int Y(t)\mu.
$$
The formula (7.1) shows that it is not invariant by the
automorphism
$\alpha\colon\,(t,\theta_1,\theta_2)\mapsto(t-\theta_1\theta_2,
\theta_1,\theta_2)$ of $\dbR^{1,1}$, despite $\alpha$
being the identity on the reduced space: \ integration
on an open set requires the choice of a codimension
$(1,0)$ boundary submanifold.

\subhead
8
\endsubhead
Integration of a density $\mu$ with compact support is
defined, hence invariant by diffeomorphisms.
It follows that for any vector field $v$, one has
$$
\int\scrL_v(\mu)=0\tag8.1
$$

Applying (8.1) to a product $\mu f$, we get
$$
\int\scrL_v(\mu).f=-(-1)^{p(v)p(\mu)}\int\mu.
vf\,\,,\tag8.2
$$
from which results that for any $g$
$$
\scrL_{vg}(\mu)=\scrL_v(g\mu)\tag8.3
$$
(with $vg=(-1)^{p(v)p(g)}gv$).
The map $v\otimes\mu\mapsto\scrL_v\mu$ hence induces a
morphism
$$
d\colon\,\,\text{(tangent bundle) $\otimes$
densities $\to$ densities}\tag8.4
$$
with
$$
\int d\alpha=0\tag8.5
$$
if $\alpha$ has compact support.

\subhead
9
\endsubhead
For an ordinary manifold of dimension $n$, one could
construct $\Omega^*$ not from the bottom up, defining
$p$-forms as wedge of $1$-forms, but from the top down,
defining a $p$-form as obtained by contracting a top
form with $(n-p)$ tangent vectors.
This corresponds to two descriptions of $\Omega^*$ as a
Clifford module over the sum of the tangent and
cotangent bundle: \ the usual description is being
generated by $\scrO$, which is killed by the $i_X$, the other is
being generated by $\Omega^n$, which is 
killed by the $\alpha\w$.
In the second description, $d$ can be defined (as was
done in 8.) by its relation (3.4) with the Lie
derivative.
Instead of $\Omega^n$, one could use its twist by the
orientation bundle, i.e. the bundle of densities.
In the case of super varieties, one can similarly define
an {\it integral forms} complex $I^n$ ($n\Le 0$)
starting from the line bundle of densities.
It differs (more than by a twist) from the de Rham
complex, but one can arrange the sign conventions so
that it obeys the same formalism.
One disposes of contractions $i_v$, exterior products
$\alpha\w$ for $\alpha$ a $1$-form and exterior
differential $d$, of cohomological degree $-1,1$ and
$1$, as usual.
Iterated contractions are linear and induce an
isomorphism
$$
\wedgeop^{p}T\otimes I^0\to I^{-p}.
$$
Exterior products are linear,
$$
[i_v,\alpha\w]=i_v(\alpha)
$$
and iterated products give a module structure over
$\Omega^*$.
One has
$$
\align
&[d,\alpha\w]=(d\alpha)\w\qquad\qquad\text{and}\\
&[d,i_v]=\scrL_v
\endalign
$$
for $v$ a vector field.

\subhead
10
\endsubhead
The basic property of integral $(-p)$-forms is that they
can be integrated on $(p,0)$-codimensional cooriented
subvarieties.
If the orientation of the normal bundle to $W$ is
defined by the basis dual to $dt_1,\dotsc,dt_p$ for
$t_1,\dotsc,t_p$ a system of equations for $W$,
our convention is that for any function with compact
support on $V$, one has
$$
\int\nolimits_{W}\alpha\vert W.f=\int\nolimits_{V}
\delta(t_1)dt_1\w\ldots\w\delta(t_n)dt_n\w
\alpha.f
\tag10.1
$$

The rule (10.1) has the virtue that if
$t\colon\,V\to\dbR$ is submersive above $0$ and $1$, one
has for a codimension $1$-form with compact support
$\alpha$
$$
\int\nolimits_{0\Le t\le 1}d\alpha=
\int\nolimits_{t=1}\alpha-\int\nolimits_{t=0}\alpha
\tag10.2
$$
if $\chi$ is the characteristic function of the interval
$[0,1]$ of $\dbR$, the left side of (10.2) is by
definition $\int\chi(t)d\alpha$ (cf. 7.) and an
integration by part using (8.5), shows it equals $\int
-d(\chi(t))\alpha=\int (\delta(t-1)dt)\alpha-\int
\delta(t)dt\alpha$, as claimed by (10.2).



\enddocument





