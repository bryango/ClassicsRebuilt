%Date: Mon, 20 Apr 1998 16:20:45 -0400 (EDT)
%From: Pavel Etingof <etingof@abel.math.harvard.edu>

\input amstex
\documentstyle{amsppt}
\magnification 1200
\NoRunningHeads
\NoBlackBoxes
\document

\def\bb{\overline{\beta}}
\def\bga{\overline{\gamma}}
\def\bg{\overline{g}}
\def\eps{\epsilon}
\def\tW{\tilde W}
\def\Aut{\text{Aut}}
\def\tr{{\text{tr}}}
\def\ell{{\text{ell}}}
\def\Ad{\text{Ad}}
\def\u{\bold u}
\def\m{\frak m}
\def\O{\Cal O}
\def\tA{\tilde A}
\def\qdet{\text{qdet}}
\def\k{\kappa}
\def\RR{\Bbb R}
\def\be{\bold e}
\def\bR{\overline{R}}
\def\tR{\tilde{\Cal R}}
\def\hY{\hat Y}
\def\tDY{\widetilde{DY}(\g)}
\def\R{\Bbb R}
\def\h1{\hat{\bold 1}}
\def\hV{\hat V}
\def\deg{\text{deg}}
\def\hz{\hat \z}
\def\hV{\hat V}
\def\Uz{U_h(\g_\z)}
\def\Uzi{U_h(\g_{\z,\infty})}
\def\Uhz{U_h(\g_{\hz_i})}
\def\Uhzi{U_h(\g_{\hz_i,\infty})}
\def\tUz{U_h(\tg_\z)}
\def\tUzi{U_h(\tg_{\z,\infty})}
\def\tUhz{U_h(\tg_{\hz_i})}
\def\tUhzi{U_h(\tg_{\hz_i,\infty})}
\def\hUz{U_h(\hg_\z)}
\def\hUzi{U_h(\hg_{\z,\infty})}
\def\Uoz{U_h(\g^0_\z)}
\def\Uozi{U_h(\g^0_{\z,\infty})}
\def\Uohz{U_h(\g^0_{\hz_i})}
\def\Uohzi{U_h(\g^0_{\hz_i,\infty})}
\def\tUoz{U_h(\tg^0_\z)}
\def\tUozi{U_h(\tg^0_{\z,\infty})}
\def\tUohz{U_h(\tg^0_{\hz_i})}
\def\tUohzi{U_h(\tg^0_{\hz_i,\infty})}
\def\hUoz{U_h(\hg^0_\z)}
\def\hUozi{U_h(\hg^0_{\z,\infty})}
\def\hg{\hat\g}
\def\tg{\tilde\g}
\def\Ind{\text{Ind}}
\def\pF{F^{\prime}}
\def\hR{\hat R}
\def\tF{\tilde F}
\def\tg{\tilde \g}
\def\tG{\tilde G}
\def\hF{\hat F}
\def\bG{\overline{G}}
\def\Spec{\text{Spec}}
\def\tlo{\hat\otimes}
\def\hgr{\hat Gr}
\def\tio{\tilde\otimes}
\def\ho{\hat\otimes}
\def\ad{\text{ad}}
\def\Hom{\text{Hom}}
\def\hh{\hat\h}
\def\a{\frak a}
\def\t{\hat t}
\def\Ua{U_q(\tilde\g)}
\def\U2{{\Ua}_2}
\def\g{\frak g}
\def\n{\frak n}
\def\hh{\frak h}
\def\sltwo{\frak s\frak l _2 }
\def\Z{\Bbb Z}
\def\C{\Bbb C}
\def\d{\partial}
\def\i{\text{i}}
\def\ghat{\hat\frak g}
\def\gtwisted{\hat{\frak g}_{\gamma}}
\def\gtilde{\tilde{\frak g}_{\gamma}}
\def\Tr{\text{\rm Tr}}
\def\l{\lambda}
\def\I{I_{\l,\nu,-g}(V)}
\def\z{\bold z}
\def\Id{\text{Id}}
\def\<{\langle}
\def\>{\rangle}
\def\o{\otimes}
\def\e{\varepsilon}
\def\RE{\text{Re}}
\def\Ug{U_q({\frak g})}
\def\Id{\text{Id}}
\def\End{\text{End}}
\def\gg{\tilde\g}
\def\b{\frak b}
\def\S{\Cal S}
\def\L{\Lambda}

\topmatter
\title Lecture 5: Renormalization groups (continued)
\endtitle
\author {\rm {\bf David Gross} }\endauthor
\endtopmatter

\centerline{Notes by P.Etingof and D.Kazhdan}
\vskip .1in

In this lecture we will return to the Wilsonian point of view 
on the method of renormalization group, which was discussed in the first 
lecture. Namely, we will discuss the exact renormalization group equation, 
which describes the renormalization group flow in the space of all 
Lagrangians. This equation together with asymptotic freedom 
provides a tool for constructing nonperturbative field theories. 

We follow J.Polchinski's paper ``Renormalization and effective 
Lagrangians'', Nuclear Physics B231, 1984, p. 269-295. 

For simplicity we will consider the theory of a scalar bosonic field
in 4 dimensions, as 
in the first lecture. We consider the theory with a cutoff propagator 
$\frac{P(p^2/\Lambda^2)}{p^2+m^2}$, where $P$ is a smooth 
positive nonincreasing function, such that $P(a)=1$ for 
$a\le 1$ and $P(a)\to 0$ sufficiently fast as $a\to \infty$. 

The free action for a massive scalar associated to this cutoff
can be written in momentum space as
$$
S_0(\phi,\L_0)=\frac{1}{2}
\int\hat\phi(p)\hat\phi(-p)\frac{p^2+m^2}{P(p^2/\Lambda_0^2)}\frac{d^4p}{(2\pi)^4}
\tag 5.1
$$
(hat denotes the Fourier transform).
This action is very big if high Fourier modes in $\phi$ are present, so 
it ``suppresses'' high Fourier modes in the path integral. 
Also, when $\L_0\to\infty$, 
this action obviously tends to the standard action for a massive scalar.

Introduce a perturbation of the action $S_0$ by ``interaction'' terms: 
$$
S_I^0(\phi)=\int(\frac{1}{2}g_1^0\phi^2+\frac{1}{2}g_2^0(\nabla\phi)^2
+\frac{g_3^0}{4!}\phi^4)d^4x,\tag 5.2
$$
where $g_i^0$ are formal variables, and consider 
the total action $S(\phi,\L_0)=S_0(\phi,\L_0)+S_I^0(\phi)$. 
Consider the theory whose Green's functions are defined by the 
path integral 
$$
Z(J)=\int D\phi\ e^{-S(\phi,\L_0)+\int J\phi d^4x}.\tag 5.3
$$
Because we are considering a cutoff theory, all Feynman diagrams are well 
defined, so we get a perturbation series. 

Now consider some smaller scale $\L_R<\L_0$. 
We are interested in physics at the scale $\L_R$. 
This means, we assume that $m<<\L_R$ and we are interested in values of 
$Z(J)$ for such $J$ that $\hat J(p)=0$ for $p^2>\L_R^2$. 

The physics at the scale $\L_R$ will be described by an effective 
Lagrangian, which is obtained from the original Lagrangian by 
``integrating out'' degrees of freedom between $\L_R$ and $\L_0$. 

To study the effective Legrangian, 
we introduce an arbitrary interaction term  
$S_I(\phi,\L)$ and consider the integral
$$
Z(J,S,\L)=\int D\phi\ e^{-S_0(\phi,\L)+S_I(\phi,\L)+\int J\phi d^4x}\tag 5.4
$$

Let us look for such $S_I(\phi,\L)$ that $Z(J,S,\L)$ is independent 
on $\L$, and write down a differential equation for such $S_I(\phi,\L)$. 

The independence of $Z$ of $\L$ yields 
$$
\frac{d}{d\L}\int D\phi e^{-S_0(\phi,\L)+S_I(\phi,\L)+\int J\phi d^4x}=0.\tag 5.5
$$
Differentiating, we get
$$
\gather
\int D\phi \biggl(\frac{1}{2}
\int\hat\phi(p)\hat\phi(-p)(p^2+m^2)\frac{d}{d\L}P(p^2/\Lambda^2)\frac{d^4p}{(2\pi)^4}
-\frac{dS_I(\phi,\L)}{d\L}\biggr)\times\\
e^{-S_0(\phi,\L)-S_I(\phi,\L)+\int \hat J(p)\hat\phi(-p)\frac{d^4p}{(2\pi)^4}}=0.\tag 5.6
\endgather
$$

>From this one can get
$$
\gather
\frac{\d S_I}{\d \L}=-\frac{1}{2}\int d^4p (2\pi)^4
(p^2+m^2)^{-1}\frac{\d P(p^2/\L^2)}
{\d \L}\times \\
\left(\frac{\delta S_I}{\delta \phi(-p)}
\frac{\delta S_I}{\delta \phi(p)}+\frac{\delta^2S_I}{\delta \phi(-p)\delta 
\phi(p)}\right). \tag 5.7
\endgather
$$
Indeed, substituting (5.7) into (5.6), we get the integral of a complete 
derivative, i.e. zero; 
here we heavily use the fact that $\hat J(p)$ has disjoint 
support with $\frac{\d P (p^2/\L^2}{\d\L}$. And it can be shown that 
(5.7) follows from (5.6).

Note that equation (5.7) corresponds to a linear equation 
for $e^{S_I}$:  
$$
\frac{\d e^{S_I}}{\d \L}=-\frac{1}{2}\int d^4p (2\pi)^4
(p^2+m^2)^{-1}\frac{\d P(p^2/\L^2)}
{\d \L}\frac{\delta^2e^{S_I}}{\delta \phi(-p)\delta 
\phi(p)}. \tag 5.8
$$
Equation (5.8) can be regarded as an infinite-dimensional heat equation. 
It is called the exact renormalization group equation. If 
$S_I(\phi,\L)$ is the solution of this equation such that 
$S_I(\phi,\L_0)=S_I^0(\phi)$, then $S_I(\phi,\L_R)$ is the effective action 
at the scale $\L_R$. This equation allows to prove the renormalizability 
of $\phi^4$ theory without use of graph techniques, in particular of 
Weinberg's theorem (see Polchinski's paper). 


\end






