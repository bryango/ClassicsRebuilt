\section{Large $N$ Gauge Theories as String Theories}
\label{largen}

The relation between gauge theories and string theories has been an
interesting topic of research for over three decades. String theory
was originally developed as a theory for the strong interactions, due
to various string-like aspects of the strong interactions, such as
confinement and Regge behavior. It was later realized that there is
another description of the strong interactions, in terms of an $SU(3)$
gauge theory (QCD), which is consistent with all experimental data to
date. However, while the gauge theory description is very useful for
studying the high-energy behavior of the strong interactions, it is
very difficult to use it to study low-energy issues such as
confinement and chiral symmetry breaking (the only current method for
addressing these issues in the full non-Abelian gauge theory is by
numerical simulations). In the last few years many examples of the
phenomenon generally known as ``duality'' have been discovered, in
which a single theory has (at least) two different descriptions, such
that when one description is weakly coupled the other is strongly
coupled and vice versa (examples of this phenomenon in two dimensional
field theories have been known for many years). One could hope that a
similar phenomenon would apply in the theory of the strong
interactions, and that a ``dual'' description of QCD exists which
would be more appropriate for studying the low-energy regime where the
gauge theory description is strongly coupled.

There are several indications that this ``dual'' description could be
a string theory. QCD has in it string-like objects which are the flux
tubes or Wilson lines. If we try to separate a quark from an
anti-quark, a flux tube forms between them (if $\psi$ is a quark
field, the operator ${\bar \psi}(0) \psi(x)$ is not gauge-invariant
but the operator ${\bar \psi}(0) P\exp(i\int_0^x A_\mu dx^\mu)
\psi(x)$ is gauge-invariant). In many ways these flux tubes behave
like strings, and there have been many attempts to write down a string
theory describing the strong interactions in which the flux tubes are
the basic objects. It is clear that such a stringy description would
have many desirable phenomenological attributes since, after all, this
is how string theory was originally discovered. The most direct
indication from the gauge theory that it could be described in terms
of a string theory comes from the 't Hooft large $N$ limit
\cite{'tHooft:1974jz}, which we will now describe in detail.

Yang-Mills (YM) theories in four dimensions have no dimensionless
parameters, since the gauge coupling is dimensionally transmuted into
the QCD scale $\Lambda_{QCD}$ (which is the only mass scale in these
theories). Thus, there is no obvious perturbation expansion that can
be performed to learn about the physics near the scale
$\Lambda_{QCD}$. However, an additional parameter of $SU(N)$ gauge
theories is the integer number $N$, and one may hope that the gauge
theories may simplify at large $N$ (despite the larger number of
degrees of freedom), and have a perturbation expansion in terms of the
parameter $1/N$. This turns out to be true, as shown by 't Hooft based
on the following analysis (reviews of large $N$ QCD may be found in 
\cite{Coleman:1980nk,Manohar:1998xv}).

First, we need to understand how to scale the coupling $g_{YM}$ as we
take $N \to \infty$. In an asymptotically free theory, like pure YM
theory, it is natural to scale $g_{YM}$ so that $\Lambda_{QCD}$
remains constant in the large $N$ limit. The beta function equation
for pure $SU(N)$ YM theory is
\eqn{betaym}{\mu {dg_{YM}\over d\mu} = -{11\over 3} N {{g_{YM}^3}
\over {16 \pi^2}} + {\cal O}(g_{YM}^5),}
so the leading terms are of the same order for large $N$ if we take
$N \to \infty$ while keeping $\lambda \equiv g_{YM}^2 N$ fixed (one
can show that the higher order terms are also of the same order in
this limit). This is known as the {\it 't Hooft limit}. 
The same behavior is valid if we include also matter
fields (fermions or scalars) in the adjoint representation, as long as
the theory is still asymptotically free. If the theory is conformal,
such as the $\cn=4$ SYM theory which we will discuss in detail below,
it is not obvious that the limit of constant $\lambda$ is the only one
that makes sense, and indeed we will see that other limits, in which
$\lambda \to \infty$, are also possible. However, the limit of constant
$\lambda$ is still a particularly interesting limit
and we will focus on it in the remainder of this chapter.

Instead of focusing just on the YM theory, let us describe a general
theory which has some fields $\Phi_i^a$, where $a$ is an index in the
adjoint representation of $SU(N)$, and $i$ is some label of the field
(a spin index, a flavor index, etc.). Some of these fields can be
ghost fields (as will be the case in gauge theory). We will assume
that as in the YM theory (and in the $\cn=4$ SYM theory), the 3-point
vertices of all these fields are proportional to $g_{YM}$, and the
4-point functions to $g_{YM}^2$, so the Lagrangian is of the schematic
form 
\eqn{schemlag}{{\cal L} \sim \tr(d\Phi_i d\Phi_i) + g_{YM}
c^{ijk} \tr(\Phi_i \Phi_j \Phi_k) + g_{YM}^2 d^{ijkl} \tr(\Phi_i
\Phi_j \Phi_k \Phi_l),} 
for some constants $c^{ijk}$ and $d^{ijkl}$
(where we have assumed that the interactions are
$SU(N)$-invariant; mass terms can also be added and do not change the 
analysis). Rescaling the fields by $\tp_i \equiv g_{YM} \Phi_i$,
the Lagrangian becomes 
\eqn{newschemlag}{{\cal L} \sim {1\over g_{YM}^2} \left[ \tr(d\tp_i
d\tp_i) + c^{ijk} \tr(\tp_i \tp_j \tp_k) + d^{ijkl} \tr(\tp_i \tp_j
\tp_k \tp_l) \right],}
with a coefficient of $1/ g_{YM}^2 = N/ \lambda$ in front
of the whole Lagrangian.

Now, we can ask what happens to correlation functions in the limit of
large $N$ with constant $\lambda$. Naively, this is a classical limit
since the coefficient in front of the Lagrangian diverges, but in fact
this is not true since the number of components in the fields also
goes to infinity in this limit. We can write the Feynman diagrams of
the theory \eno{newschemlag} in a double line notation, in which an
adjoint field $\Phi^a$ is represented as a direct product of a fundamental
and an anti-fundamental field, $\Phi^i_j$, as in figure \ref{thooft}.
The interaction vertices we wrote are all consistent with this sort of
notation. The propagators are also consistent with it in a $U(N)$ 
theory; in an $SU(N)$ theory there is a small mixing term
\eqn{propsun}{\vev{\Phi^i_j \Phi^k_l} \propto (\delta^i_l \delta^j_k -
{1\over N} \delta^i_j \delta^k_l),} 
which makes the expansion slightly more complicated, but this involves
only subleading terms in the large $N$ limit so we will neglect this
difference here. Ignoring the second term the propagator for the
adjoint field is (in terms of the index structure) like that of a
fundamental-anti-fundamental pair.  Thus, any Feynman diagram of
adjoint fields may be viewed as a network of double lines. Let us
begin by analyzing vacuum diagrams (the generalization to adding
external fields is simple and will be discussed below). In such a
diagram we can view these double lines as forming the edges in a
%triangulation\footnote{We will use the word triangulation to denote a
%general description of a surface as a union of faces which are not
%necessarily triangles.} of a surface, if we view each single-line loop
simplicial decomposition (for example, it could be a triangulation)
of a surface, if we view each single-line loop
as the perimeter of a face of the simplicial decomposition. 
The resulting
surface will be oriented since the lines have an orientation (in one
direction for a fundamental index and in the opposite direction for an
anti-fundamental index). When we compactify space by adding a point at
infinity, each diagram thus corresponds to a compact, closed, oriented
surface.

\begin{figure}[htb]
\begin{center}
\epsfxsize=4in\leavevmode\epsfbox{thooft.eps}
\end{center}
\caption{Some diagrams in a field theory with adjoint fields in the
standard representation (on the left) and in the double line
representation (on the right). The dashed lines are propagators for
the adjoint fields, the small circles represent interaction vertices,
and  solid lines carry indices in the fundamental
representation.}
\label{thooft}
\end{figure}

What is the power of $N$ and $\lambda$ associated with such a
diagram? From the form of \eno{newschemlag} it is clear that each
vertex carries a coefficient proportional to $N/ \lambda$, while
propagators are proportional to $\lambda / N$. Additional powers of
$N$ come from the sum over the indices in the loops, which gives a
factor of $N$ for each loop in the diagram (since each index has $N$
possible values). Thus, we find that a diagram with $V$ vertices, $E$
propagators (= edges in the simplicial decomposition) 
and $F$ loops (= faces in
the simplicial decomposition) comes with a coefficient proportional to
\eqn{euler}{N^{V-E+F} \lambda^{E-V} = N^{\chi} \lambda^{E-V},}
where $\chi \equiv V-E+F$ is the Euler character of the surface
corresponding to the diagram. For closed oriented surfaces, $\chi =
2-2g$ where $g$ is the genus (the number of handles) of the
surface.\footnote{We are discussing here only connected diagrams, for
disconnected diagrams we have similar contributions from each
connected component.}  Thus, the perturbative expansion of any diagram 
in the field theory may be
written as a double expansion of the form 
\eqn{thooftexp}{\sum_{g=0}^{\infty}
N^{2-2g} \sum_{i=0}^{\infty} c_{g,i} \lambda^i = \sum_{g=0}^{\infty}
N^{2-2g} f_g(\lambda),} 
where $f_g$ is some polynomial in $\lambda$
(in an asymptotically free theory the $\lambda$-dependence will turn into
some $\Lambda_{QCD}$-dependence but the general form is similar; infrared
divergences could also lead to the appearance of terms which are not 
integer powers of
$\lambda$). 
In the large $N$ limit we see that any computation will be
dominated by the surfaces of maximal $\chi$ or minimal genus, which
are surfaces with the topology of a sphere (or equivalently a plane). 
All these {\it planar
diagrams} will give a contribution of order $N^2$, while all other
diagrams will be suppressed by powers of $1/N^2$. For example, the
first diagram in figure \ref{thooft} is planar and proportional to
$N^{2-3+3}=N^2$, while the second one is not and is proportional to 
$N^{4-6+2}=N^0$. We presented our
analysis for a general theory, but in particular it is true for any gauge
theory coupled to adjoint matter fields, like the $\cn=4$ SYM
theory. The rest of our discussion will be limited mostly to gauge
theories, where only gauge-invariant ($SU(N)$-invariant) objects are
usually of interest.

The form of the expansion \eno{thooftexp} is the same as one finds in
a perturbative theory with closed oriented strings, if we identify
$1/N$ as the string coupling constant\footnote{In the conformal case,
where $\lambda$ is a free parameter, there is actually a freedom of
choosing the string coupling constant to be $1/N$ times any function
of $\lambda$ without changing the form of the expansion, and this will
be used below.}. Of course, we do not really see
any strings in the expansion, but just diagrams with holes in them;
however, one can hope that in a full non-perturbative description of
the field theory the holes will ``close'' and the surfaces of the Feynman
diagrams will become actual closed surfaces. The analogy of
\eno{thooftexp} with perturbative string theory is one of
the strongest motivations for believing that field theories and string
theories are related, and it suggests that this relation would be more
visible in the large $N$ limit where the dual string theory may be
weakly coupled. However, since the analysis was based on perturbation
theory which generally does not converge, it is far from a rigorous
derivation of such a relation, but rather an indication that it might
apply, at least for some field theories (there are certainly also
effects like instantons which are non-perturbative in the $1/N$
expansion, and an exact matching with string theory would require a
matching of such effects with non-perturbative effects in string
theory).

The fact that $1/N$ behaves as a coupling constant in the large $N$
limit can also be seen directly in the field theory analysis of the 't
Hooft limit. While we have derived the behavior \eno{thooftexp} only
for vacuum diagrams, it actually holds for any correlation function of
a product of gauge-invariant fields $\vev{\prod_{j=1}^n G_j}$ such
that each $G_j$ cannot be written as a product of two gauge-invariant
fields (for instance, $G_j$ can be of the form ${1\over N}\tr(\prod_i
\Phi_i)$). We can study such a correlation function by adding to the
action $S \to S + N\sum g_j G_j$, and then, if $W$ is the sum of
connected vacuum diagrams we discussed above (but now computed with
the new action), 
\eqn{genvev}{\vev{\prod_{j=1}^n G_j} = (iN)^{-n} \left[ {{\del^n W}\over
{\prod_{j=1}^n \del g_j}} \right]_{g_j=0}.}
Our analysis of the vacuum diagrams above holds also for these
diagrams, since we put in additional vertices with a factor of $N$,
and, in the double line representation, each of the operators we
inserted becomes a vertex of the simplicial decomposition of the
surface (this would not
be true for operators which are themselves products, and which would
correspond to more than one vertex). Thus, the leading contribution to
$\vev{\prod_{j=1}^n G_j}$ will come from planar diagrams with $n$
additional operator insertions, leading to
\eqn{largencorr}{\vev{\prod_{j=1}^n G_j} \propto N^{2-n}}
in the 't Hooft limit. We see that (in terms of powers of $N$) the
2-point functions of the $G_j$'s come out to be canonically
normalized, while 3-point functions are proportional to $1/N$, so
indeed $1/N$ is the coupling constant in this limit (higher genus
diagrams do not affect this conclusion since they just add higher
order terms in $1/N$). In the string theory analogy the operators
$G_j$ would become vertex operators inserted on the string
world-sheet. For asymptotically free confining theories (like QCD) one
can show that in the large $N$ limit they have an infinite spectrum of
stable particles with rising masses (as expected in a free string
theory). Many additional properties of the large $N$ limit are
discussed in
\cite{Migdal:1977nu,Coleman:1980nk} and other references.

The analysis we did of the 't Hooft limit for $SU(N)$ theories with
adjoint fields can easily be generalized to other cases. Matter in the
fundamental representation appears as single-line propagators in the
diagrams, which correspond to boundaries of the corresponding
surfaces. Thus, if we have such matter we need to sum also over
surfaces with boundaries, as in open string theories. For $SO(N)$ or
$USp(N)$ gauge theories we can represent the adjoint representation as
a product of two fundamental representations (instead of a fundamental
and an anti-fundamental representation), and the fundamental
representation is real, so no arrows appear on the propagators in the
diagram, and the resulting surfaces may be non-orientable. Thus, these
theories seem to be related to non-orientable string theories
\cite{Cicuta:1982fu}. We will not discuss these cases in detail here,
some of the relevant aspects will be discussed in section
\ref{orientifolds} below.

Our analysis thus far indicates that gauge theories may be dual to
string theories with a coupling proportional to $1/N$ in the 't Hooft
limit, but it gives no indication as to precisely which string theory
is dual to a particular gauge theory. For two dimensional gauge
theories much progress has been made in formulating the appropriate
string theories
\cite{Gross:1993tu,Gross:1993hu,Minahan:1993sk,Gross:1993yt,Naculich:1993ve,
Ramgoolam:1993hh,Cordes:1994sd,Horava:1999wf}, 
but for four dimensional gauge
theories there was no concrete construction of a corresponding string
theory before the results reported below, since the planar diagram
expansion (which corresponds to the free string theory) is very
complicated. Various direct approaches towards constructing the
relevant string theory were attempted, many of which were based on the
loop equations \cite{Makeenko:1979pb} for the Wilson loop observables
in the field theory, which are directly connected to a string-type
description.

Attempts to directly construct a string theory equivalent to a four
dimensional gauge theory are plagued with the well-known problems of
string theory in four dimensions (or generally below the critical
dimension). In particular, additional fields must be added on the
worldsheet beyond the four embedding coordinates of the string to
ensure consistency of the theory. In the standard quantization of four
dimensional string theory an additional field called the Liouville
field arises \cite{Polyakov:1981rd}, 
which may be interpreted as a fifth space-time
dimension. Polyakov has suggested \cite{Polyakov:1997tj,
Polyakov:1998ju} that such a five dimensional string theory
could be related to four dimensional gauge theories if the couplings
of the Liouville field to the other fields take some specific
forms. As we will see, the AdS/CFT correspondence realizes this idea,
but with five additional dimensions (in addition to the radial
coordinate on AdS which can be thought of as a generalization of the
Liouville field), leading to a standard (critical) ten dimensional
string theory.




