\documentstyle{article}


%These are the macros which are in common with all of the
% sections in the paper mmr
% Each section, for now, should begin with \documentstyle[11pt,cd]{article}
% and then have \input{mmrmacros} followed by \begin{document}
% The only exception is that the \Label macro is slightly different
% in each file and should be put in separately.
%New CD macros
%\newcommand{\cdrl}{\cd\rightleftarrows}
%\newcommand{\cdlr}{\cd\leftrightarrows}
%\newcommand{\cdr}{\cd\rightarrow}
%\newcommand{\cdl}{\cd\leftarrow}
%\newcommand{\cdu}{\cd\uparrow}
%\newcommand{\cdd}{\cd\downarrow}
%\newcommand{\cdud}{\cd\updownarrows}
%\newcommand{\cddu}{\cd\downuparrows}
% (S) Proofs.
% (S-1) Head is automatically supplied by \proof.

\def\proof{\vspace{2ex}\noindent{\bf Proof.} }
\def\tproof#1{\vspace{2ex}\noindent{\bf Proof of Theorem #1.} }
\def\pproof#1{\vspace{2ex}\noindent{\bf Proof of Proposition #1.} }
\def\lproof#1{\vspace{2ex}\noindent{\bf Proof of Lemma #1.} }
\def\cproof#1{\vspace{2ex}\noindent{\bf Proof of Corollary #1.} }
\def\clproof#1{\vspace{2ex}\noindent{\bf Proof of Claim #1.} }
% End of Proof Symbol at the end of an equation must precede $$.

\def\endproof{\relax\ifmmode\expandafter\endproofmath\else
  \unskip\nobreak\hfil\penalty50\hskip.75em\hbox{}\nobreak\hfil\bull
  {\parfillskip=0pt \finalhyphendemerits=0 \bigbreak}\fi}
\def\endproofmath$${\eqno\bull$$\bigbreak}
\def\bull{\vbox{\hrule\hbox{\vrule\kern3pt\vbox{\kern6pt}\kern3pt\vrule}\hrule}
}
\addtolength{\textwidth}{1in}                  % Margin-setting commands
\addtolength{\oddsidemargin}{-.5in}
\addtolength{\evensidemargin}{.5in}
\addtolength{\textheight}{.5in}
\addtolength{\topmargin}{-.3in}
\addtolength{\marginparwidth}{-.32in}
\renewcommand{\baselinestretch}{1.6}
\def\hu#1#2#3{\hbox{$H^{#1}(#2;{\bf #3})$}}          % #1-Cohomology of #2
\def\hl#1#2#3{\hbox{$H_{#1}(#2;{\bf #3})$}}          % #1-Homology of #2
\def\md#1{\ifmmode{\cal M}_\delta(#1)\else  % moduli space, delta decay of #1
{${\cal M}_\delta(#1)$}\fi}
\def\mb#1{\ifmmode{\cal M}_\delta^0(#1)\else  %moduli space, based, delta
                                              %decay of #1
{${\cal M}_\delta^0(#1)$}\fi}
\def\mdc#1#2{\ifmmode{\cal M}_{\delta,#1}(#2)\else    %moduli space, delta
                                                      %decay, chern class #1
                                                      %of #2
{${\cal M}_{\delta,#1}(#2)$}\fi}
\def\mbc#1#2{\ifmmode{\cal M}_{\delta,#1}^0(#2)\else   %as before, based
{${\cal M}_{\delta,#1}^0(#2)$}\fi}
\def\mm{\ifmmode{\cal M}\else {${\cal M}$}\fi}
\def\ad{{\rm ad}}
\def\msigma{\ifmmode{\cal M}^\sigma\else {${\cal M}^\sigma$}\fi}
\def\cancel#1#2{\ooalign{$\hfil#1\mkern1mu/\hfil$\crcr$#1#2$}}
\def\dirac{\mathpalette\cancel\partial}
\newtheorem{thm}{Theorem}
\newtheorem{theorem}{Theorem}[subsection]
\newtheorem{proposition}[theorem]{Proposition}
\newtheorem{lemma}[theorem]{Lemma}
\newtheorem{claim}[theorem]{Claim}
\newtheorem{example}[theorem]{Example}
\newtheorem{corollary}[theorem]{Corollary}
\newtheorem{D}[theorem]{Definition}
\newenvironment{defn}{\begin{D} \rm }{\end{D}}
\newtheorem{addendum}[theorem]{Addendum}
\newtheorem{R}[theorem]{Remark}
\newenvironment{remark}{\begin{R}\rm }{\end{R}}
\newcommand{\note}[1]{\marginpar{\scriptsize #1 }} 
\newenvironment{comments}{\smallskip\noindent{\bf Comments:}\begin{enumerate}}{
\end{enumerate}\smallskip}

\renewcommand{\thesection}{\Roman{section}}
\def\eqlabel#1{\addtocounter{theorem}{1}
\write1{\string\newlabel{#1}{{\thetheorem}{\thepage}}}
\leqno(\rm\thetheorem)}
\def\cS{{\cal S}}
\def\ov{\overline}



\input epsf








\title{The Dirac Index on Manifolds and Loop Spaces: Part II}
\author{Edward Witten}
\date{December 17, 1996}

\begin{document}
\maketitle

%\tableofcontents

\section{Introduction}
In this lecture we shall extend the discussion from the last lecture
to the case of signature operators and Dirac operators on loop spaces.

\section{The Lagrangian formulation: $\sigma$-models in two dimensions}

We shall consider the space (or rather the super-space) of maps of
${\bf R}^{2|2}$ to a finite dimensional, complete riemannian manifold
$M$ of even dimension. 
Let $u,v$ be the coordinates on the underlying space ${\bf R}^2$ of
${\bf R}^{2|2}$ and
let $\theta_+,\theta_-$ be the odd coordinates on this super manifold.
The metric on the underlying ${\bf R}^2$ is  the flat Minkowski metric
$ds^2=du\,\,dv$.
On ${\bf R}^{2|2}$ we have the vector fields $D_+$ and $D_-$ given by
$$D_+=\frac{\partial}{\partial \theta_+}-\theta_+\frac{\partial}{\partial
u},$$
and
$$D_-=\frac{\partial}{\partial \theta_-}-\theta_-\frac{\partial}{\partial
v}.$$
The Lagrangian that we consider is
$${\cal L}=\frac{1}{2}\int_{{\bf
R}^{2|2}}d^2\theta dudv\ g_{IJ}D_+X^ID_-X^J,$$ 
or written invariantly using the metric $\langle\ \ ,\ \ \rangle_M$ on $TM$
$${\cal L}=\frac{1}{2}\int_{{\bf R}^{2|2}} d^2\theta dudv\langle
D_+X,D_-X\rangle _M=\frac{1}{2}\int_{{\bf R}^2}dudv\left(D_-D_+\langle
D_+X,D_-X\rangle_M\right)|_{\theta=0} .$$
Let us write the superfield out in coordinates as
$$X=x+\theta_+\psi_-+\theta_-\psi_++\theta_+\theta_-F$$
where $x=X|_{\theta=0}$, $\psi_\pm=D_\pm(X)|_{\theta=0}$, and
$F=D_-D_+(X)|_{\theta=0}$.
Integrating out the odd variables yields
$${\cal L}=\int_{{\bf R}^2}dudv\frac{1}{2}\left( \langle\frac{\partial
x}{\partial u},\frac{\partial x}{\partial v}\rangle +\langle
\psi_+,D_u\psi_+\rangle+\langle \psi_-,D_v\psi_-\rangle +\frac{1}{4}
R_{IJKL}\psi_-^I\psi_-^J\psi_+^K\psi_+^L +\langle F,F\rangle\right).$$
Of course, when we consider the action since $F$ appears purely
quadratically and with no derivatives we can simply integrate out
these degrees of freedom introducing the condition that $F=0$ as one
of the equations of motion.  From now on we drop this term.


The presence of the indefinite form  $\langle \frac{\partial
x}{\partial u},\frac{\partial x}{\partial v}\rangle$ indicates that
neither $u$ nor $v$ can be used as a time coordinate
on ${\bf R}^2$.
We
switch coordinates letting (time coordinate) $t=(u+v)/2$ and (space 
coordinate) $\sigma$ be $(u-v)/2$ with the metric $ds^2=dt^2-d\sigma^2$. 
We also allow the possibility that the spatial direction $\sigma$ has
been made periodic so that it is a circle instead of a copy of ${\bf
  R}$.
Under this change of coordinates we have
$$\frac{d}{du}=\frac{d}{dt}+\frac{d}{d\sigma}$$
$$\frac{d}{dv}=\frac{d}{dt}-\frac{d}{d\sigma}$$
and analogously for covariant derivatives.

\centerline{\quad}
\centerline{\epsfxsize=1in\epsfbox{fig2.1.eps}}
\centerline{\bf Figure 1.}
\centerline{\quad}

Rewriting the Lagrangian in these variables yields
\begin{eqnarray*}
\lefteqn{{\cal L}=\frac{1}{2}\int_{{\bf R}^2}dtd\sigma\left(\langle \dot x,\dot
x\rangle+\langle 
\psi_+,\dot \psi_+\rangle +\langle \psi_-,\dot\psi_-\rangle\right. }  \\
& &\left.-\langle\frac{\partial x}{\partial\sigma},\frac{\partial x}{\partial
\sigma}\rangle +\langle \psi_+,\frac{\partial \psi_+}{\partial
\sigma}\rangle -\langle\psi_-,\frac{\partial \psi_-}{\partial
\sigma}\rangle
+\frac{1}{4}R_{IJKL}\psi_-^I\psi_-^J\psi_+^K\psi_+^L\right).
\end{eqnarray*}
The first three terms, those  involving the time derivatives, are the
kinetic energy terms.  The other terms are potential terms.


As usual, we have symmetries of the Lagrangian associated with the right
invariant vector fields
$$Q_+=\frac{\partial}{\partial
  \theta^+}+\theta^+\frac{\partial}{\partial u}$$
$$Q_-=\frac{\partial}{\partial
  \theta^-}+\theta^+\frac{\partial}{\partial v}.$$
Of course we have
$$Q_+^2=\frac{\partial }{\partial u}$$
$$Q_-^2=\frac{\partial }{\partial v}.$$


\section{Quantization}

Now that we have a time direction $t$ it makes sense to quantize in
the Hamiltonian framework.
In order to quantize we restrict to the  slice   $t=0$.
The fields $\psi_\pm$ should be viewed as spinors on the slice $t=0$ with
values in $x^*(TM)$ , or equivalently as half-densities on the slice 
with values in $S_\pm\otimes x^*TM$.
In the case that the spatial direction is a circle, we can view
$\psi_\pm$ as half-density on $S^1$ with values in  $T({\cal L}M)$.
Since we are taking periodic boundary values for the $\psi_\pm$
the spin bundles $S_\pm$  are trivial over the circle (i.e., give the
non-bounding spin structure on $S^1$.)
We can write
$$\int_{t=0}(\psi_+,\psi_+)$$ as a well-defined quadratic form on
the space of sections on the slice. Quantizing this infinite
dimensional quadratic space yields a spin space. Since we are in infinite
dimensions now, the spin space depends not only on the underlying
quadratic space but also on a polarization (i.e., a self-annihilating
subspace in the complexification). The subspace we choose is given by
the operator $D/D\sigma$ 
acting on  the spinors. Since space is now a circle, this operator has
discrete spectrum. For the spinors $\psi_+$ we use the
polarization coming from the negative eigenspaces plus a polarization
of the zero eigenspace. The choice of this latter polarization is
irrelevant since it is finite dimensional.
For the spinors $\psi_-$ we choose the polarization consisting
of the positive eigenspaces of $D/D\sigma$ plus a polarization for the
zero eigenspace. The reason for these choices will be apparent presently.
We call the resulting spin spaces for the two types of spinors $S_R$
of right-moving spinors (the $\psi_+$) and $S_L$ of left-moving
spinors (the $\psi_-$).
The Hilbert space after quantizing is
the tensor product 
$S_R\otimes S_L$.
This gives us a model for the differential
forms on the loop space $\Omega^*({\cal L}M)$. They are a type of
middle dimensional forms in the sense that they are of neither finite
dimension nor codimension.




The Hamiltonian operator is a perturbation of the
operator given by the purely quadratic part of the kinetic plus
potential energy.
This operator derived from the quadratic energy terms has pieces
$$H_{\psi_+}=-\int_{S^1}(\psi_+,\frac{D}{D\sigma}\psi_+)$$
and
$$H_{\psi_-}=\int_{S^1}(\psi_-,\frac{D}{D\sigma}\psi_-).$$
Since we need these  two operators to be bounded below, 
the choices we have made are forced upon us (up to finite dimensional
change which does not affect the result spin space).

The supersymmetries $Q_\pm$ discussed above become operators after we
quantize satisfying the same relations they did before quantization.
In addition, 
the operators $Q_\pm$ commute with the Hamiltonian $H$, which is
associated with the vector field
$$\frac{\partial }{\partial t}=\frac{1}{2}\left(\frac{\partial}{\partial
u}+\frac{\partial}{\partial v}\right).$$

Associated to the vector fields  $Q_\pm$ are conserved quantities also
denoted $Q_\pm$ given by
$$Q_+=\int_{S^1}(\psi_+,D_ux)$$
$$Q_-=\int_{S^1}(\psi_-,D_vx).$$
Writing things out in local coordinates on $M$, we have
$$Q_+=\int_{S^1}d\sigma g_{IJ}\psi_+^I(\sigma)\left(\frac{\partial
x^J}{\partial t}+\frac{\partial x^J}{\partial \sigma}\right).$$
Under quantization $\partial x^I/\partial t$ becomes $-ig^{IJ}\delta/\delta
x^J(t)$ so that
$$Q_+=\int_{S^1}d\sigma\left(-i\psi_+^I\frac{\delta}{\delta
x^I(t)}+(\psi,\frac{dx}{d\sigma})\right).$$ 
The first term in this last formula for $Q_+$
is formally the Dirac operator for the free loop space
${\cal L}M$. The second term is needed to make the operator
well-behaved quantum mechanically. 
Recall that in the last lecture (in the finite dimensional
$S^1$-equivariant case) we
had the operator $D+t\Gamma\cdot V$ where $D$ was the Dirac operator
and $V$ was the Killing vector field of a circle action ($t$ is a real
number and $\Gamma$
represents Clifford multiplication). In the case of the free loop
space, there is a natural circle action which rotates the
parametrization on the domain circle. Its Killing vector field $V$ is
$\frac{\partial }{\partial \sigma}$ so that the analogue of
$t\Gamma\cdot V$ in this case is the second term in the last
expression for $Q_+$. Notice that there is no parameter here
playing the role of $t$ in the finite dimensional discussion. 
If we tried to vary our operators replacing $D/Du$ by $D/Dt+\epsilon
D/D\sigma$, this would work fine as long as $\epsilon>0$ (of course
once $\epsilon\not=1$ we would violate relativity, but that is not
relevant here), but when $\epsilon<0$ our Hamiltonian operators would
no longer be bounded below. To achieve this we would have to pick the
opposite polarizations for the spinors, so our operators
and index theory would jump discontinuously in passing through $\epsilon=0$.
 This is the basic reason that
the vanishing results we will get  for index theory
on free loop space are weaker than the finite dimensional
vanishing theorem.
Half of the $S^1$-indices vanish in the loop space case rather than all of them
as in the finite dimensional case. In the language of our discussion
from last week, we are forced to keep $t>0$ and hence can only show that
half the indices vanish.

Of course, for a  spin manifold $M$ of finite, even dimension we have that the
differential forms can be decomposed in terms of spinors (which we denote
$S(M)$) by
by $\Omega^*(M)=S(M)\otimes S(M)$. 
In general, in Riemannian geometry, there is a natural Dirac operator
acting on sections of $S(M)\otimes E$ where $E$ is any vector bundle
with connection.  By setting $E=S(M)$ and identifying $S(M)\otimes V$
with $S(M)\otimes S(M)$ in either of two ways (which differ by exchange
of the factors), 
we have two Dirac
operators acting on $\Omega^*(M)$.
 These coupled Dirac
operators commute and are given  by $D=d+d^*$, using the Dirac
operator on the first factor, and $\tilde D =
d-d^*$, using the Dirac operator on the second factor. 
If $M$ has a circle action generated by a Killing vector field $V$, then
we can form the operators
$$Q_+=D+\Gamma\cdot V$$
$$Q_-=\tilde D-\tilde\Gamma\cdot V,$$
where $\Gamma$ and $\tilde\Gamma$ represent Clifford multiplication
in the first and second factor respectively.
These operators commute and furnish finite-dimensional prototypes
for the supersymmetry generators $Q_\pm$ in the loop space case.

\section{The Index of $Q_+$}

Our goal is to calculate the ${\rm index}(Q_+)$ 
on the free loop space. Of course, this index
must be interpreted properly since the free loop space is of infinite
dimension.
One thing we need to do in this infinite dimensional context is define
the `degree' of our `forms' on the loop space, at least modulo two.
How to do this was  discussed at the end of the last lecture.
Recall from that discussion that in  our
situation there is a group ${\bf Z}/2{\bf Z}\times {\bf Z}/2{\bf 
Z}$ with generators acting via
$$\psi_+\mapsto -\psi_+;\ \ \ \psi_-\mapsto\psi_-$$
and
$$\psi_+\mapsto \psi_+;\ \ \ \ \psi_-\mapsto-\psi_-.$$
As we saw, if $M$ is spin then the full group of symmetries exists in
the quantum theory; in any event,
 the diagonal copy of ${\bf Z}/2{\bf Z}$ always
exists as a symmetry of the quantum theory.
It turns out that using the diagonal ${\bf Z}/2{\bf Z}$ produces an
index which plays the role of the Euler characteristic in finite
dimensions. (After all in our finite dimensional analogy, the
diagonal  ${\bf Z}/2{\bf Z}$ acts with $(S^+\otimes S^+)\oplus
(S^-\otimes S^-)$ as the fixed space and $(S^+\otimes S^-)\oplus
(S^-\otimes S^+)$ as the minus one eigenspace. Under the isomorphism of
$S\otimes S$ with $\Omega^*(M)$, this ${\bf Z}/2{\bf
Z}$-grading corresponds to the degree modulo two.
So it is not surprising that in the context of the loop space the
index graded by the diagonal involution gives an Euler
characteristic-type object.)
But if we 
 assume that $M$ is spin, then we can
take  one of these involutions, say the
involution $\rho$ sending $\psi_+$ to $-\psi_+$. In the finite dimensional
situation this involution is the Hodge $*$-operator, so it is
reasonable to view the
index computed using this involution as a generalization the usual
signature of a finite dimensional manifold.
This is the index that we shall study.
But just using this involution is not enough.
In order to get finite answers, we need to use the circle symmetry on
the free loop space in addition to the involution.
The way we set things up is to consider the index as a formal
character of the circle. That is to say we consider
$${\rm index}\,Q_+={\rm Tr}_{{\rm Ker}^+(Q_+)}q^P-{\rm Tr}_{{\rm
Ker}^-(Q_+)}q^P$$ 
where $q$ is a formal variable, the operator $P$ is given by
$$P=-\frac{\partial}{\partial\sigma},$$
and ${\rm Ker}^\pm(Q_+)$ refer to the 
decomposition of ${\rm Ker}(Q_+)$ into  $\pm 1$-eigenspaces under the
involution $\rho$.


We will compute this index by
quantizing the space of  maps of a $2|2$
supermanifold whose underlying geometric manifold is $(S^1\times {\bf
R})$ into a compact even dimensional spin manifold $M$.

\section{The computation around the fixed points of the $S^1$-action.}

As usual the computation localizes around the fixed points of the
supersymmetry $Q_+$. Let us see what these are.
The operator
$$Q_+^2=H+P$$
is the Laplacian on forms plus ${\cal L}_V+|V|^2$,
where $V$ is the vector field generating the natural circle action on
the free loop space.
Thus, as in the last lecture we compute the index by localizing around
the fixed points of $V$.
The zeros of the vector field $V$ acting on the loop space are easy to
determine: they 
are simply the constant loops. Hence, they make a copy of the manifold
$M$, embedded as a subspace of ${\cal L}M$. Thus, by an infinite dimensional
analogue of the Atiyah-Bott fixed point theorem we expect the
computation to yield an answer of the form
$$F(q)=\int_M\hat A(M){\rm ch}({\cal R})$$
where ${\cal R}$ is a bundle (or more precisely, a formal sum of
bundles) that will be found presently
by a computation on the normal bundle to $M$ in the 
free loop space.

We made a similar computation in the last lecture for circle actions
on finite dimensional manifolds. The method was to find the kernels of
the Hamiltonian operators in the normal direction to the fixed point
set.
That is what we do here as well.
Let us begin the discussion by parameterizing this normal bundle. We
consider a constant loop at $x_0\in M$. Then any nearby $x$ can be
written as
$$X(\sigma)=x_0+{\rm exp}(\varphi(\sigma))$$
where $\varphi(\sigma)\in TM_{x_0}$ satisfies
$\int_{S^1}d\sigma\varphi(\sigma)=0$ and $\varphi(\sigma)$ satisfies
the free wave equation:
$$\left(\frac{d^2}{dt^2}-\frac{d^2}{d\sigma^2}\right)\varphi=0.$$
This means that we can decompose $\varphi(\sigma)$ uniquely as
$$\varphi(\sigma)=\varphi_+(\sigma)+\varphi_-(\sigma),$$
with $\varphi_+$ be left-moving and $\varphi_-$ being right-moving.
This decomposition is unique since we have required that the zero
Fourier mode for  $\varphi$ vanishes.

Since we are considering `differential forms' on the loop space as
being the tensor product of two spin bundles $S_R\otimes S_L$,
we also have to consider the fermions $$\psi_-(v),\psi_+(u).$$
If we want to consider the Dirac operator $Q_+$ that acts on spinors
$S_R$, then the roles of $S_R$ and $S_L$ are not symmetrical.
$S_R$ is the bundle of ``spinors on free loop space,'' while
$S_L$ is an auxiliary bundle on free loop space that just happens
to be a spin bundle constructed with a different polarization.
(In fact, at the end of the lecture we will replace $S_L$ by a more
general vector bundle on free loop space.)  Let us then focus on the
role of $S_R$, which is obtained by quantizing $\psi_+$.  By quantizing
the zero modes of $\psi_+$, we get the spinors on 
$M$ (regarded as the fixed point subset of ${\cal L}M$), 
while quantization of the nonzero modes of $\psi_+$ gives the
spinors on the normal bundle $N$ to $M$.  The tensor product is
$S_R=S(M)\otimes S(N)$, where the two factors are $S(M)$, the spinors
on $M$, obtained by quantizing the constant modes of $\psi_+$,
and $S(N)$, the spinors of the normal bundle, obtained by quantizing
the part of $\psi_+$ that is orthogonal to the zero modes. 
Thus $\psi_+=\psi_{+,0}+\psi_{+,\perp}$, where
$\psi_{+,0}$ is constant and 
$$\int_{S^1}d\sigma \psi_{+,\perp}(u)=0.$$
$S(N)$ is obtained by quantizing $\psi_{+,\perp}$.

First suppose that $M$ is the flat manifold ${\bf R}^{2n}$. Then the
theory would have no interaction terms and the action would be purely
quadratic. The resulting Hilbert space would be the tensor product of
spinor-valued $L^2$ functions on ${\bf R}^{2n}$ (obtained by quantizing $x_0$,
$\dot x_0$, and $\psi_{+,0}$) 
 with a tensor product Hilbert space
$$\chi=\chi^+\otimes \chi^-,$$
where $\chi^+$ and $\chi^-$ are Hilbert spaces of right-movers and left-movers,
respectively.  Concretely
$\chi^+$ comes from quantizing   $\varphi_+$ and $\psi_{+,0}$,
and $\chi^-$ comes from quantizing $\varphi_-,\psi_-$.  (Because the
Dirac operator whose index we will compute is $Q_+$ -- a choice that
breaks the symmetry between left and right -- it is very helpful
to separate out the zero modes from $\psi_+$, as we have done, but there
is no need to similarly separate out the zero modes from $\psi_-$.)
 $\chi^+$  is a  Fock space since, for $M={\bf R}^{2n}$,
the theory is free.  $\chi^-$ is almost a Fock space; it is actually
the tensor product of a copy of $S(M)$ (obtained by quantizing the zero
modes in $\psi_-$) with a Fock space obtained by quantizing the nonzero
modes of $\psi_-$ and $\varphi_-$.  The Dirac operator $Q_+$ of the free
model is the sum of an operator that acts in $\chi^+$ and an operator
(namely $(\psi_{+,0}\dot x_0)$, which is interpreted as the ordinary
Dirac operator on ${\bf R}^{2n}$) that acts on the spinors on ${\bf R}^{2n}$.

Now suppose that $M$ is not flat, but scale up the metric of $M$ by a large
factor so that the curvature is everywhere very small.  Working near
the constant loops in ${\cal L}M$ and using the fact that the flat
model is everywhere a good approximation, we make a fiberwise version of
the description in the last paragraph.
The Hilbert
space of the theory consists of spinor-valued $L^2$ sections of a certain
bundle  of Fock-like spaces over $M$. The bundle over $M$
is a tensor product $\chi^+\otimes \chi^-$ where $\chi^+$ and $\chi^-$ are
constructed exactly as in the flat model.  (We call the factors Fock-like
only because $\chi^-$ is not quite a Fock space.)
One might think in terms of a projection from a small neighborhood
of $M\subset {\cal L}M$ back to $M$.  The idea is to compute the 
index of the Dirac operator by using this fibration; as usual the first
step is to solve the Dirac equation along the fibers, which are copies
of the normal bundle to $M$ in ${\cal L}M$.  We write
$Q_{+,\perp}$ for the part of $Q_+$ that acts in $\chi$.

The Dirac operator in the fibers is easily described.
The operator $Q_{+,\perp}$ acts in $\chi^+$  as the tensor algebra of its
action on the space generated by
 $\varphi_+(n),\psi_+(n)$ for $n>0$.  (As usual
we make Fourier expansions $\varphi_+=\sum_ne^{in\sigma}\varphi_+(n)$, and
likewise for other fields.)   In each such space, $Q_{+,\perp}$ 
has a one-dimensional
kernel,
so the kernel of $Q_{+,\perp}$ 
in the full Fock space $\chi^+$ is the infinite tensor
product $L$ of the one-dimensional invariant subspaces.
Hence its kernel in $\chi$ is identified with $\chi^-$ tensored with $L$:
$${\rm Ker}(Q_{+,\perp})=L\otimes \chi^-.$$

The involution that we are using acts nontrivially on $\chi^+$ but
trivially on $L$ and
trivially on $\chi^-$. It then follows that the involution acts
trivially on the kernel of $Q_{+,\perp}$.
To understand better that kernel,
let us examine $\chi^-$ in more detail. We have fields
$$\varphi_-(\sigma)=\sum_{n\not= 0}\varphi_ne^{in\sigma}$$
$$\psi_-(\sigma)=\sum_n\psi_ne^{in\sigma}.$$
These produce creation and annihilation operators in the Fock space
$\chi^-$ which satisfy:
$$[\varphi_n^i,\varphi_m^j]=\frac{1}{m}\delta_{n+m}\delta^{ij}$$
$$\{\psi_n^i,\psi_m^j\}=\delta^{ij}\delta_{n+m},$$
where $\delta^{ij}$ is the metric in the tangent space to $M$ at
$x_0$. 
We quantize so that $H$ is bounded below.  In particular, the bosons
can be quantized  with a vacuum state
$|\Omega\rangle$, such that
$$\varphi_n|\Omega\rangle =0, \ \ n>0,$$
and hence the bosonic states in the Fock space are
$$\prod_{\ell=1}^\infty\varphi_{-\ell}^{\epsilon_\ell}|\Omega\rangle$$
of energy $\sum_\ell\ell\epsilon_\ell$.


For the fermions $\psi_-$ the zero mode is allowed so that the
quantization of $\psi_0$ gives a copy of the spin bundle $S$ of the
manifold $M$. In particular, there is not a unique vacuum since any
state of $S$ is of zero energy. We use $|s\rangle$ for such zero
energy states. As before, we want $\psi_n|s\rangle=0,\ n>0$ and for
any zero energy state $|s\rangle$. Thus, the states in the Fock space
are 
$$\prod_{n=1}^\infty\psi_{-n}^{\delta_n}|s\rangle$$
of energy $\sum_{n=1}^\infty n\delta_n$.
Now we are in position to describe ${\rm Ker}(Q_+)=\chi^-$ as an
$S^1$-module. 
Clearly, from the above description we see that it is given by
$$\chi^-=S\otimes_{n=1}^\infty {\rm
Sym}_{q^n}(TM)\otimes_{n=0}^\infty\Lambda_{q^n}(TM).$$
Here we use the abbreviations
$${\rm Sym}_{q^n}(TM)= \oplus_{m=0}^\infty q^{nm}{\rm Sym}^m(TM),$$
and similarly 
$$\Lambda_{q^n}(TM)=\oplus_{m=0}^\infty q^{nm}\Lambda^m(TM).$$
Here ${\rm Sym}^m$ and $\Lambda^m$ denote as usual the $m^{th}$ symmetric
or exterior powers.
  Finally, a factor $q^n$ in front of a vector bundle is to
indicate that the action of the circle on this bundle is by the
$n^{th}$ power of the standard representation.

After solving the Dirac equation along the fibers to get the kernel
$L\otimes \chi^-$, the next step is to solve the Dirac equation
along $M$ with values in this kernel.
Thus, formally at least, we have
$$F(q)=q^{-{\rm dim}M/24}\int_M\hat A(M){\rm
ch}\left(S\otimes _{n-1}^\infty{\rm
Sym}_{q^n}(TM)\otimes_{m=1}^\infty\Lambda_{q^m}(TM)\right),$$
where $F(q)$ is ${\rm Signature}({\cal L}M)(q)$ as defined by the
index of $Q_+$.
Several comments are in order about this formula. First of all the
power of $q$ out front comes from the action of the circle on the fixed
complex line $L$;  we will explain later why this is
the correct factor. Secondly, notice that only positive powers of $q$
appear under the integral sign. But since there are positive powers
which do appear in this 
formula, the signature is not constant as a representation of $S^1$,
unlike the finite dimensional case. What replaces the constancy for
the character is the
fact that in this context the character is a holomorphic function of
$q$, with positive powers only (multiplying a fixed prefactor). We shall see 
shortly that setting $q=e^{2\pi i\tau}$, this character extends to a
holomorphic function of $\tau$ in the entire upper half-plane, a
function that is modular for the subgroup $\Gamma_0(2)$ of $SL_2({\bf Z})$.  


\section{Path Integral Approach}

In the path integral approach to computing the index under discussion
we consider, for real $\theta$, the expression
$${\rm Tr}\left(e^{-\beta H}e^{i\theta P}(-1)^{F_\rho}\right)$$
where $\rho$ is the involution sending $\psi_+$ to $-\psi_+$ and
$\psi_-$ to $\psi_-$. Since $Q_+^2=H+P=0$ on ${\rm Ker}(Q_+)$, on this
kernel $H=-P$. This trace is equal to 
$${\rm Tr}_{{\rm Ker}^+(Q_+)}e^{-(\beta+i\theta)H}-{\rm Tr}_{{\rm
Ker}^-(Q_+)}e^{-(\beta+i\theta)H}.$$
This of course is simply the supertrace
$$F(q)={\rm str}_{{\rm Ker}(Q_+)}q^H$$
for $q=e^{-(\beta+i\theta)}$. 
Notice that $|q|<1$ since $\beta>0$.
The path integral has a propagator $e^{-\beta H}$ (so that time is
Euclidean) for fields on a circle. The term $e^{i\theta P}$ says that
before we glue the ends of the cylinder together we rotate by an angle
$\theta$.
This yields a riemann surface diffeomorphic to $T^2$.
As we have seen from the Hamiltonian discussion previously, there is
a power series expression for this supertrace which is a holomorphic
function of $q={\rm exp}(2\pi i\tau)$  for $\tau$ in the upper
half-plane.
One can also prove directly using path integrals that this path
integral is a holomorphic function of $\tau$ in the upper half-plane. 
(The key step in the proof is to write certain components of the stress
tensor in the form $\{Q_+,\dots\}$; one then uses this to prove
that the $\overline \tau$ derivative of the index vanishes.)

\centerline{\quad}
\centerline{\epsfxsize=1.5in\epsfbox{fig2.2.eps}}
\centerline{\bf Figure 2.}
\centerline{\quad}



In this path integral
description, the fields $\psi_\pm$ are sections over $T^2$ of
the spin bundles $S_\pm$. Since there are four spin structures on
$T^2$, there are $16$ different possibilities for the pair of spin
bundles $S^\pm$ and hence $16$ different possibilities for the path
integral
$$\int_{X\colon T^2\to M}{\cal D}X{\cal D}\psi_+{\cal D}\psi_-
e^{-{\cal L}}.$$ 
We need to know which spin structures we should use. These are of
course determined by the boundary conditions, or gluing conditions on
the fields $\psi_\pm$. In the spatial direction for both $\psi_+$ and
$\psi_-$ we are using the
ordinary periodicity and hence the spin structure in that direction is
the non-bounding one for both types of spinors.
In the time direction for $\psi_+$ we are computing the supertrace,
this means that the spin structure for $S^+$ in that direction is the
non- bounding one.
In the time direction for $\psi_-$ we are computing the usual trace
and hence the spin structure for $S^-$ in that direction is the
bounding one.
Letting $+$ denote the non-bounding spin structure on $S^1$ and $-$
the bounding one we see that the spin structures are given as pictured
below:

\centerline{\quad}
\centerline{\epsfxsize=3in\epsfbox{fig2.3.eps}}
\centerline{\bf Figure 3.}
\centerline{\quad}




Let us consider the action of $SL_2({\bf Z})$ on these spin
structures. First of all, the action of an element in $SL_2({\bf Z})$
on a spin structure depends only on its reduction modulo two. The spin
structure for $S_+$ is in fact invariant under the full group
$SL_2({\bf Z})$ whereas the subgroup leaving second one invariant is
$\Gamma_0(2)$, which consists of
matrices whose lower left entry is zero modulo two. 
This is a subgroup of index three
in $SL_2({\bf Z})$ reflecting the fact that the orbit of this spin
structure under $SL_2({\bf Z})$ has three elements.
The fundamental domain for $\Gamma_0(2)$ acting on the upper
half-plane is:


\centerline{\quad}
\centerline{\epsfxsize=1.5in\epsfbox{fig2.4.eps}}
\centerline{{\bf Figure 4.} Fundamental Domain for $\Gamma_0(2)$}
\centerline{\quad}




Notice that the quotient of the upper half-plane by $\Gamma_0(2)$ has
three cusps.
The one represented at $\infty$ in the upper half-space corresponds to
$\beta\mapsto\infty$. Computing in this limit shows that the path
integral reduces to the supertrace on ${\rm Ker}(Q_+)$ of $q^H$, as
discussed above. The other cusps correspond to different computations.
The  cusps are interchanged by the action of $SL_2({\bf Z})$
and hence computing in the limit at one of these means computing  the
same path integrals but with different  spin structures.

As we have already mentioned, the action of $SL_2({\bf Z})$ leaves
invariant the spin bundle $S_+$, so that when we compute at the other
cusps $\psi_+$ is still a section of the same spin bundle.
What changes is the spin bundle $S_-$.
Computing at one of the two other cusps means taking $\psi_-$ as a
section of
$$x^*TM\otimes \epsilon$$
where $\epsilon$ is a real line bundle on $T^2$ which is non-trivial
over the  the spatial  circle.  The two other cusps differ by whether
$\epsilon$ is trivial in the time direction; we first take the
case that it is.

The previous analysis can be repeated to determine the index that
corresponds to a path integral with this spin structure.
 Namely, the Hilbert space of states $\chi$ for the theory is
a tensor product $\chi=\chi_+\otimes \chi_-$ of Fock spaces.
Nothing has changed with $\chi_+$, but let us examine $\chi_-$.
 As
before, we have, for example, the Fourier decomposition
$$\varphi_-(\sigma)=\sum_{n\not= 0}\varphi_ne^{in\sigma}.$$
The difference is that because of the change in the spin structure
$S_-$ around the spatial circle, the Fourier expansion for $\psi_-$ is now
$$\psi_-(\sigma)=\sum_{m\in {\bf Z}+\frac{1}{2}}\psi_ne^{im\sigma}.$$
In particular, in the Fock space there is no copy of $S$, since there
is no zero Fourier coefficient for $\psi_-$. (So things are now a bit
simpler: $\chi^{\pm}$ are true Fock spaces, while before this was not quite
so for $\chi^-$.)  Thus, the expansion at
this cusp for the index is
$$F(q)=q^{-{\rm dim} M/24}\int_M\hat A(M){\rm
ch}\left(\otimes_{n=1}^\infty{\rm
Sym}_{q^n}(TM)\otimes_{m=1/2,3/2,\ldots}\Lambda_{q^m}(TM)\right).$$ 

The third cusp is the image of the second cusp under the map
$\tau\mapsto \tau+1$. Thus, the expansion at the third cusp differs
from this one simply by changing the sign of $q^{1/2}$
and thus, in the last formula,
 changing the sign of $q^{m}$ for all $m$.  (Recall that $m$
is in ${\bf Z}+\frac{1}{2}$.)

Now there is one important point that we have not yet discussed. This
is the source of the factor $q^{-{\rm dim}M/24}$ in front of these
integral expressions. This comes from the $S^1$ action on
$$L={\rm Ker}(Q_{+,\perp})$$ 
In fact, $L$ is an infinite
tensor product of one dimensional spaces which are obtained by quantizing
$\varphi_+(n), \psi_+(n)$ for $n=1,2,3,\dots.$  On the $n^{th}$ such
space, the circle acts with character
$${n\over 2}{\rm dim}\, M.$$
Thus, on the tensor product the circle action contributes a
multiplicative factor
$$q^{\frac{1}{2}\sum_{n=1}^\infty n{\rm dim}M}.$$
Of course, this factor is regularized in the renormalization
procedure. It becomes
$$q^{\frac{1}{2}\zeta(-1){\rm dim}M}=q^{-{\rm dim}M/24}.$$

As an alternative to the Hamiltonian approach that we have followed,
the index of $Q_+$ can be evaluated
by  directly computing the path integral. One gets equivalent formulas,
of course.  Scaling up the metric of $M$
reduces the evaluation of the path integral to one-loop computations.
One requires computing Pfaffians for $D_+$ 
and $D_-$ on spinors in a non-trivial but flat bundle over $T^2$ and
 computing the determinant of the Laplacian for the
bosons.  We will not explore this direction today.


\section{Bundles with whose signature  or Dirac
operator has constant character
for circle actions}

Let us examine the formulas we get for the index of $Q_+$ at the
various cusps. For one cusp we get a formula of the form
$$F(q)=\int_M\hat A(M){\rm ch}\left(S\otimes(\oplus_{n=1}^\infty q^n 
R_n)\right).$$  
At the other cusps we get a formula of the form
$$F(q)=\int_M\hat A(M){\rm ch}(\oplus_{m\in {\bf Z}/2}
 q^m \tilde R_m).$$  
Both the bundles $R_n$ and $\tilde R_m$ are $SO$-bundles associated to
the tangent bundle of $M$ by certain representations which we can make
explicit. 
We claim that these bundles  have a special property.
Suppose that $M$
has an $U(1)$-symmetry $G\colon S^1\times M\to M$, with generating
vector field $K$. Then
$$F(q,\alpha)={\rm Tr}_{{\rm Ker}^+(Q_+)}q^He^{i\alpha K}-{\rm
Tr}_{{\rm Ker}^-(Q_+)}q^He^{i\alpha K}$$
is independent of $\alpha$. 
Of course, this is really a statement about the signature operator 
 coupled with the finite
dimensional vector bundles $ R_m$ (or the Dirac operator with
$\tilde R_m$) on a finite dimensional spin manifold with circle action. The 
statement is that for the special 
bundles arising from consideration of the loop space
index, the indices of these operators, viewed as characters on the
circle, are constant.

To establish this result, let us consider the path integrals where
instead of using maps of $T^2\to M$ we consider sections of a bundle
over an $M$-bundle over $T^2$ where as we pass from time $t=0$ to
$t=\beta$ we identify $M$ with itself by twisting by ${\rm exp}(\alpha K)$.
(Of course, we could do the same sort of twisting in the other
direction as well. That we will do presently.)
We expand the resulting index as
$$F(q,\alpha)=q^{-{\rm dim}M/24}\sum_{k,m}q^ke^{im\alpha}c_{k,m}$$
for constants $c_{k,m}\in {\bf Z}$.
We have already seen that
$$c_{k,m}=0\ \ {\rm for\ }k<0.$$
Next we claim that:
$$c_{k,m}=c_{k+m,m}.$$
If we can establish this claim, then together with the first fact, it
immediately follows that $c_{k,m}=0$ for $m\not=0$, which is exactly
the constancy of $F(q,\alpha)$ with respect to $\alpha$.

It remains to establish the above claim.
To do this, let us introduce another $\gamma\in S^1$, and let us
consider the Dirac operator $Q_+$ on a twisted loop space ${\cal L}_\gamma
M$, the $M$ bundle over the spatial circle obtained by gluing the ends
together using the automorphism $\exp(\gamma K)$. 
The $\sigma$-model makes sense as a function
of $\gamma$, and $F(q,\alpha)$, being an index, is independent of the
continuous parameter $\gamma$.
On ${\cal L}_\gamma M$, we have
$$e^{2\pi \frac{d}{d\theta}}=e^{\gamma K}.$$
Let us define subspaces 
${\cal H}_{\gamma,n,m}$
in the Hilbert space of states by the conditions
$${\cal H}_{\gamma,n,m}=\left\{\psi|K\psi=\psi \ {\rm and}\ 
P\psi=\left(n-\frac{\gamma m}{2\pi}\right)\psi\right\}.$$
The dimension of the space ${\cal H}_{{\rm Id},m,n}$ is the index $c_{n,m}$.
Once again, we know that the dimension of these spaces is independent
of $\gamma$ (the Fredholm property of the operators). As we let
$\gamma$ go from $0$ to $2\pi$ this has the effect of shifting the
$c_{n,m}$ to $c_{n+m,m}$. This, then, establishes the claim and hence
the constancy statement.

In this way we have found an infinite sequences of bundles $\tilde R_m$
(or $R_n$)
associated to the tangent bundle of a compact even dimensional spin
manifold $M$ with the following property.
If $M$ has an isometric circle action, then the index of the Dirac (or
signature)
operator coupled to these bundles is a constant character of
$S^1$. 
The first few of the $\tilde R_m$'s are:
\begin{eqnarray*}
\tilde R_0 & = & {\cal O}; \ \ {\rm the\ trivial\ bundle} \\
\tilde R_{1/2} & = & TM \\
\tilde R_1 & = & TM\oplus \wedge^2TM.
\end{eqnarray*}

By computing relevant bordism groups Landweber-Stong showed that for
general spin manifolds the $\tilde R_m$  are the only bundles
naturally associated to the tangent bundle with this constancy property.



\section{Generalization to Vector Bundles over the Loop Space}

There is an analogue of this theory  where one
considers spinors on the loop space with values in certain
 vector bundles ${\cal 
V}$
over the loop space.  The relevant ${\cal V}$'s are constructed
from ordinary finite-rank vector bundles $V\to M$.
Physically, this analogue is very important because it
arises in heterotic string theory.
Mathematically, one finds that
 the index of the Dirac operator with values in certain bundles made
 from $V$ (rather than from the tangent bundle of $M$)  is a constant
character of the circle if $M$ has a circle action obeying certain conditions.
Let us sketch briefly the relevant version of the nonlinear sigma model and
its mathematical application.

We use ``chiral'' superfields; we 
set $\theta^-$ and  $\psi_-$ to zero and consider maps
$$X\colon {\bf R}^{2|1}\to M.$$
The Lagrangian we consider is
$$\int d\theta_+dudv\left((D_vX,D_{\theta_+}X\right)=\int
dudv\left(g_{IJ}\partial_u x^I\partial_v x^J+\psi_+D_v\psi_+\right).$$
To get something interesting we add another superfield 
$$\Lambda\in \Gamma\left({\bf R}^{2|1},S_-\otimes x^*(V)\right),$$
where $V\to M$ is a real $SO(k)$-bundle with a connection.
We express $\Lambda$ in components as $\Lambda=\lambda+\theta_+ G$.
We form a new Lagrangian
$$\int  d\theta_+dudv\left((D_vX,D_{\theta_+}X)
+(\Lambda,D_{\theta_+}\Lambda)\right)$$
which after performing the theta integral and integrating out the auxiliary
field $G$ becomes schematically
$$\int dudv\left(g_{IJ}\partial_u X^I\partial_v x^J+\psi_+D_v\psi_+
+\lambda D_u \lambda+F\psi_+\psi_+\lambda\lambda\right).$$
Here $F$ is the curvature of the connection on $V$.

In this theory we have the operator $Q_+$ as before, but no longer do
we have $Q_-$. When we go to evaluate path integrals we find that
for a fixed choice of the Riemann surface $\Sigma$ and the map $X:\Sigma
\to M$,  the fermion path integral takes values in a product of
determinant line bundles
${\rm det}D_+(\lambda)\cdot {\rm det}D_-(\lambda)$.
In order for the path integral to be a real number, and not just a
section of a line bundle, we need to trivialize this product of
determinant line bundles. The first question then is a topological
one: when is this product of determinant line bundles topologically
trivial?  The answer depends a bit on what choices of spin structures
one wishes to use.  Assuming that we want to be able to 
make independent choices
of spin structure for both $\psi_+$ and $\lambda$ (this is what one needs
to do in the heterotic string theory, and it leads to the most interesting
mathematical results), one needs the theory to be invariant under the
full ${\bf Z}_2\times {\bf Z}_2$ of independent sign changes for $\psi_+$
and $\lambda$.  For this, we need
$M$ to be a spin manifold and $V$ a spin
bundle.
The triviality of the product of determinant line bundles for any
Riemann surface and any spin structures gives further
$$\frac{p_1(TM)}{2}=\frac{p_1(V)}{2}.~~~~~~(*)$$
(Here $p_1$ is the first Pontryagin class, which is naturally divisible
by two for spin manifolds and spin bundles.)


The next question is to find a flat connection on this product of
determinant line bundles.  Mathematically, there is a standard
definition of a connection on the determinant line bundle.  Though it is not
flat, it can be modified to be flat if one is 
given a $3$-form $H$ on $M$ such that
$$dH=\frac{{\rm tr} R\wedge R-{\rm tr}F\wedge F}{8\pi^2}.$$
Here, $R$ is the curvature of the riemannian manifold $M$.
The point of the formula is that $dH$ is the four-form that appears
in the standard mathematical formula for the curvature of the usual connection.
The equality of $p_1(TM)$ and $p_1(V)$ ensures that such $H$'s exist.
If one is given such an $H$, then one modifies the usual connection
on the determinant line bundle as follows.  The Dirac operators whose
product of determinants we wish to trivialize are defined when one
is given a choice of a Riemann surface $\Sigma$ and a map $X:\Sigma\to M$.
A one-parameter family of such data give a three-manifold $W=\Sigma\times
[0,1]$ and a map $X_W:W\to M$.  We multiply the usual parallel
transport from 0 to 1 in the determinant line by
the phase factor
$${\exp(i\int_W X^*_W(H)).}$$
The usual local formula for the curvature of the determinant line bundle
ensures that the modified connection is flat; its global holonomy is also
trivial if the formula $(*)$ holds integrally and not just at the level
of differential forms.

This construction is related to some very important physics.
Physically, $H$ is the curvature of the two-form field that is usually
called $B$, and the inclusion of the $H$-dependent phase
 factor is part of the Green-Schwarz
anomaly cancellation mechanism.


To give a mathematical application,
 we  suppose that $M$ and $V$ have $S^1$-actions and we ask, as
before, if the index is a constant character for the $S^1$.
Here a new question arises:
the action of $S^1$ induces a symmetry of the Lagrangian, but is it a
symmetry of the quantum field theory?
For this, we want not just that
the  product of determinant lines considered
before should be trivial, but that  
if $X:\Sigma\to M$ is an $S^1$-invariant map, the $S^1$ action on the
fiber of the product
of determinant lines over $X$ should be trivial.
Via some further index theory, one can show that the condition
for this is that $(*)$ must hold in the $S^1$-equivariant cohomology,
not just in ordinary integral cohomology:
$$\left(\frac{(p_1)_{S^1}(V)}{2}-\frac{(p_1)_{S^1}(TM)}{2}\right)^{S^1}=0.
~~~(**)$$  
(Here, $(p_1)_{S^1}$ refers to the Pontryjagin class in
$S^1$-equivariant cohomology.)
Once we have this, the arguments as in the previous case apply to show
that the Dirac index on $M$ with values in $\tilde R_m(V)$ or in
$S(V)\otimes R_n(V)$ is constant as a character of the circle.

One can actually make a somewhat similar statement if 
the left hand side of $(**)$ is not zero but is the pullback of
 a class in $H^4(BG)$.  For if this is so, then by adding or in the sense
 of $K$ theory subtracting from $V$ a trivial bundle with nontrivial
 $S^1$ action, one can reduce to the case that $(**)$ is satisfied.




\end{document}





