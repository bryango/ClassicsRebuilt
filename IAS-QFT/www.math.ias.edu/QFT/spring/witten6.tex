%Date: Wed, 26 Mar 1997 18:32:05 -0500 (EST)
%From: Pavel Etingof <etingof@abel.MATH.HARVARD.EDU>

\input amstex
\documentstyle{amsppt}
\magnification 1200
\NoRunningHeads
\NoBlackBoxes
\document

\def\opsi{\overrightarrow{\psi}}
\def\h{\frak h}
\def\tW{\tilde W}
\def\Aut{\text{Aut}}
\def\tr{{\text{tr}}}
\def\ell{{\text{ell}}}
\def\Ad{\text{Ad}}
\def\u{\bold u}
\def\m{\frak m}
\def\O{\Cal O}
\def\tA{\tilde A}
\def\qdet{\text{qdet}}
\def\k{\kappa}
\def\RR{\Bbb R}
\def\be{\bold e}
\def\bR{\overline{R}}
\def\tR{\tilde{\Cal R}}
\def\hY{\hat Y}
\def\tDY{\widetilde{DY}(\g)}
\def\R{\Bbb R}
\def\h1{\hat{\bold 1}}
\def\hV{\hat V}
\def\deg{\text{deg}}
\def\hz{\hat \z}
\def\hV{\hat V}
\def\Uz{U_h(\g_\z)}
\def\Uzi{U_h(\g_{\z,\infty})}
\def\Uhz{U_h(\g_{\hz_i})}
\def\Uhzi{U_h(\g_{\hz_i,\infty})}
\def\tUz{U_h(\tg_\z)}
\def\tUzi{U_h(\tg_{\z,\infty})}
\def\tUhz{U_h(\tg_{\hz_i})}
\def\tUhzi{U_h(\tg_{\hz_i,\infty})}
\def\hUz{U_h(\hg_\z)}
\def\hUzi{U_h(\hg_{\z,\infty})}
\def\Uoz{U_h(\g^0_\z)}
\def\Uozi{U_h(\g^0_{\z,\infty})}
\def\Uohz{U_h(\g^0_{\hz_i})}
\def\Uohzi{U_h(\g^0_{\hz_i,\infty})}
\def\tUoz{U_h(\tg^0_\z)}
\def\tUozi{U_h(\tg^0_{\z,\infty})}
\def\tUohz{U_h(\tg^0_{\hz_i})}
\def\tUohzi{U_h(\tg^0_{\hz_i,\infty})}
\def\hUoz{U_h(\hg^0_\z)}
\def\hUozi{U_h(\hg^0_{\z,\infty})}
\def\hg{\hat\g}
\def\tg{\tilde\g}
\def\Ind{\text{Ind}}
\def\pF{F^{\prime}}
\def\hR{\hat R}
\def\tF{\tilde F}
\def\tg{\tilde \g}
\def\tG{\tilde G}
\def\hF{\hat F}
\def\bg{\overline{\g}}
\def\bG{\overline{G}}
\def\Spec{\text{Spec}}
\def\tlo{\hat\otimes}
\def\hgr{\hat Gr}
\def\tio{\tilde\otimes}
\def\ho{\hat\otimes}
\def\ad{\text{ad}}
\def\Hom{\text{Hom}}
\def\hh{\hat\h}
\def\a{\frak a}
\def\t{\hat t}
\def\Ua{U_q(\tilde\g)}
\def\U2{{\Ua}_2}
\def\g{\frak g}
\def\n{\frak n}
\def\hh{\frak h}
\def\sltwo{\frak s\frak l _2 }
\def\Z{\Bbb Z}
\def\C{\Bbb C}
\def\d{\partial}
\def\i{\text{i}}
\def\ghat{\hat\frak g}
\def\gtwisted{\hat{\frak g}_{\gamma}}
\def\gtilde{\tilde{\frak g}_{\gamma}}
\def\Tr{\text{\rm Tr}}
\def\l{\lambda}
\def\I{I_{\l,\nu,-g}(V)}
\def\z{\bold z}
\def\Id{\text{Id}}
\def\<{\langle}
\def\>{\rangle}
\def\o{\otimes}
\def\e{\varepsilon}
\def\RE{\text{Re}}
\def\Ug{U_q({\frak g})}
\def\Id{\text{Id}}
\def\End{\text{End}}
\def\gg{\tilde\g}
\def\b{\frak b}
\def\S{\Cal S}
\def\L{\Lambda}
\def\tl{\tilde\lambda}

\topmatter
\title Lecture II-6: Gauge theory in 2 dimensions with 
self-interacting bosons, the Wilson line operator, and confinement
\endtitle
\author {\rm {\bf Edward Witten} }\endauthor
\endtopmatter

\centerline{Notes by Pavel Etingof and David Kazhdan}

\vskip .1in

In the last lecture we considered 2-d gauge theories with fermions. 
Today we will consider 2-dimensional gauge theories with bosons. 
As before, we will work with Euclidean Lagrangians. 

{\bf 6.1. Infrared behavior of $U(1)$ gauge theories with 
bosons in 2-dimensions.}

We consider a $U(1)$ gauge theory with bosons in 2 dimensions. 

Our spacetime is a Riemann surface $\Sigma$. 
Our fields are $A$ -- a $U(1)$ connection in some line bundle $L$, 
and $\phi_1,...,\phi_N$ -- complex scalar fields, which are sections of $L$. 

Our Lagrangian is
$$
\int d^2x\left(\frac{(*F)^2}{4e^2}+\frac{1}{2}|d\phi|^2+\frac{\l}{4}
(|\phi|^2-a^2)^2\right)-\frac{i\theta}{2\pi}\int F,\tag 6.1
$$
where $|\phi|^2=\sum |\phi_i|^2, |d\phi|^2=\sum |d\phi_i|^2$. 

We want to understand the infrared behavior of this theory.

Clasisically, the ``mass'' of the bosons is given by $\phi_i$ is $m^2=-
\l a^2$. 
If $\l\to 0$ and $m^2>0$ remains fixed, this Lagrangian becomes the Lagrangian
of a gauge theory with ``free'' massive bosons (i.e. bosons interacting
with the gauge field only and not interacting with each other), which was 
considered in Lecture II-4. 

The parameters of the theory are $m,\theta,\l,e$ (with $\l>0$). We will 
now keep $\l,e$ fixed, and vary $m,\theta$, to see how the behavior  
of the theory depends on them. 

First consider the situation $m^2\to +\infty$ (i.e. $m^2>>e^2$). In this 
situation the quartic term in the Lagrangian becomes unimportant, 
and the situation is essentially the same as in Lecture II-4. Namely,
the theory has a mass gap, and the interaction between bosons is approximated 
by the 1-dimensional Coloumb 
potential, which causes their confinement:  
all states in the theory are of total charge zero. 
Also, the space reversal symmetry is preserved for $\theta=0$, 
but broken for $\theta=\pi$, thus producing two vacua and
a cut at $\theta=\pi$. For example, the vacuum energy density, 
which is a $2\pi$-periodic function of $\theta$ 
given for $-\pi\le \theta\le \pi$ by formula (6.2) below,  
has a discontinuous derivative at $\theta=\pi$. 

Now consider the situation when $m^2\to -\infty$ (i.e. $m^2<<-e^2$). 
In this case, the situation is totally different: because of the presence 
of the quartic term, the classical potential for bosons has a minimum at
$|\phi|^2=a^2$, so the space of minima is $S^{2N-1}$. The gauge group 
$U(1)$ acts freely on this space, so the space of classical vacua
is $\C P^{N-1}=S^{2N-1}/U(1)$. Thus, the low energy effective theory 
is the theory of maps from $\Sigma$ to $\C P^{N-1}$ with the round
metric (the sigma-model). 

{\bf Remark.} In Lecture 3 we saw that the theory 
with the Lagrangian density $\frac{1}{2}(d\phi)^2+\l(\phi^2-a^2)^2$ 
flows in the infrared to the sigma-model of maps to the sphere. We
remarked that this means that this (superrenormalizable) theory 
is a good cutoff for the sigma-model to the sphere. Similarly, 
the superrenormalizable (and hence rigorously existing) theory 
defined by Lagrangian (6.1) is a good UV cutoff for the theory of maps to 
$\C P^{N-1}$. 

For Large $N$, we found that the sigma-model of maps 
to $\C P^{N-1}$ behaves in the infrared as the gauge theory  
of massive bosons. Today we will consider in more detail the case $N=1$. 
In this case $\C P^{N-1}$ is one point, so the theory defined by 
(6.1) has a mass gap and a unique vacuum. 

{\bf 6.2. The vacuum energy density.} 

Recall that for $m^2>>e^2$ in Lecture II-4 we found the following value 
of the vacuum energy density:
$$
E_{vac}(\theta)=\frac{e^2}{2}\left(\frac{\theta}{2\pi}\right)^2.\tag 6.2
$$
So the $\theta$-derivative of the energy density 
is proportional to the first power of the coupling $e^2$. 

In the case $m^2<<-e^2$, the situation is totally different. Namely, now 
the contributions to the path integral from nontrivial line bundles
(relative to infinity) 
is exponentially small in $e^2$, since any section of such a bundle 
has to vanish somewhere, and this will have a big action due to the presense
of the quartic potential. Thus, to any finite order of perturbation 
theory (in $e^2$), the path integral,
and hence the vacuum energy density are independent of $\theta$. 
More precisely, the theta-dependence is exponentially small, and comes from 
the sum over {\it instantons} -- lowest action configurations of nontrivial 
first Chern class. 

{\bf 6.3. Instantons.}

Let us compute the coefficient of $e^{i\theta}$ in the Fourier expansion 
of the path integral with Lagrangian (6.1). This is equivalent to taking 
the path integral over connections in line bundles with $c_1=1$. 

As usual, the main contribution to the path integral comes from field 
configurations which are close to the configuration minimizing 
the action. Such a configuration $(\phi,A)$ is called an instanton.

Let $(\phi,A)$ be an instanton. 
In order for its action to be finite, we must have
$$
|\phi(x)|\to a, F_A(x)\to 0, x\to \infty,\tag 6.3
$$
where $F_A$ is the curvature of $A$. Moreover,   
since $\phi$ is classically massive, this convergence 
is exponentially fast. Thus, an instanton in our problem 
has to be a highly localized field configuration. 

In our further discussion of instantons, we will assume that $\Sigma=\R^2$, 
and $c_1$ is measured with respect to trivialization at infinity. 
In this case, we expect that instantons are 
rotationally symmetric, with respect to some center of rotation $x_0$. 
Without loss of generality we can assume that $x_0=0$, so that the instanton 
has the form $\phi=ae^{i\alpha}f(r)$, $A=g(r)d\alpha$, where 
$f(r)$, $g(r)$ are some functions of the radius $r$ (where $r,\alpha$ 
are polar coordinates on $\R^2$). It is easy to show 
that in this case one must have $f(0)=0$, $f(\infty)=1$, 
and $g(0)=0$, $g(\infty)=1$. Thus, we get a boundary value problem for
ordinary differential equations. It can be proved by considering 
the corresponding ODE that this problem 
has a unique solution. Thus, the instanton is unique, up to translations. 
In particular, it has a definite size, and its action is 
a well-defined positive constant $I$.  

{\bf Remark.} In this respect, our instanton is different from 
the instantons of the $\C P^{N-1}$ model, which could be 
transformed by any conformal automorphism, and therefore had no 
definite size. 

By dimensional analysis 
it is clear that $I$ is of the form $a^2h(\frac{\l}{e^2})$, 
where $h$ is a dimensionless function. It is possible to show that
$h(z)\sim Cz$ as $z\to \infty$, where $C$ is a constant, 
so for small $e$ (compared to $\l$), $I\sim \frac{C\l a^2}{e^2}$.   
This calculation illustrates
the fact that the contribution of the instanton to the path integral 
is exponentially small with respect to $e^2$. 

{\bf 6.4. Instanton gas.}

Now we want to understand how to compute the contribution of the instantons 
to the path integral, and when such a computation gives a good approximation. 

First of all consider line bundles with $c_1=2$. It can be shown that 
there is no instanton in this topological class if $e^2/\lambda$ is 
sufficiently large. We will assume that there is no instanton 
for $c_1>1$, but what we will say can be generalized to the case 
when there is one (for small $e^2/\lambda$). In the case when there
is no instanton with $c_1>1$,  
the problem of minimization of action in the case $c_1=2$ has no global 
minimum (the infinum is not attained). However, 
we can consider approximate instantons, whose action approaches
the infinum arbitrarily closely. More precisely, if we take 
$A(x)=A_*(x-x_1)+A_*(x-x_2),\phi(x)=\phi_*(x-x_1)\phi_*(x-x_2)/a$, where
$(A_*,\phi_*)$ is the instanton centered at $0$, then the action 
of $(A,\phi)$ equals $2I$, plus a correction which is of order
$e^{-c|x_1-x_2|}$, where $c$ is a positive constant.
Thus, the infinum of actions for $c_1=2$ is $2I$. 

Similarly, the infinum of actions for $c_1=n$ is $nI$, for any
$n\in \Z_+$. This follows from the fact that the action of 
the field configuration 
$$\sum_{i=1}^n A_*(x-x_i), a\prod_{i=1}^n\frac{\phi_*(x-x_i)}{a}$$
has action which is exponentially close to $nI$ when $|x_i-x_j|$ are big. 

Now consider the situation when $c_1<0$. It is easy to see that 
$(-A_*,\bar \phi_*)$ is the instanton for $c_1=-1$. 
It is called the antiinstanton. Thus, 
the situation for $n<0$ is symmetric to the situation for $n\in \Z_+$:
the field configuration $$-\sum_{i=1}^{|n|}A_*(x-y_i),
a\prod_{i=1}^n\frac{\bar\phi_*(x-y_i)}{a}$$ has action exponentially
close to $|n|I$ when $|y_i-y_j|$ is big. 

More generally, one can consider field configurations
$$\sum_{i=1}^n A_*(x-x_i)-\sum_{i=1}^m A_*(x-y_i),
a\prod_{i=1}^n\frac{\phi_*(x-x_i)}{a}
\prod_{i=1}^m\frac{\bar\phi_*(x-y_i)}{a},$$ 
with $c_1=n-m$ which have action exponentially 
close to $(n+m)I$ when $x_i,y_j$ are distant from each other. 

{\bf Remark.} 
Such field configurations are called ``instanton gas''. 
The term ``instanton gas'' refers to a gas with long range
Coulomb forces (like the instantons we studied later in the Polyakov
model in 2+1 dimensions).  This gas with exponentially small forces
at long range is more like an ordinary gas of atoms, for instance
the air in the atmosphere.  It is an almost ideal gas, the ideal gas law
of thermodynamics is the case that the forces are exactly zero.  Any
real gas (hydrogen, oxygen) behaves as an almost ideal gas if the density
is small enough, because then the particles are generally at big distances
where the interactions are small.  That is the case for the instantons
in this model because the instanton action is big (and the instantons
have a definite size)

{\bf 6.5. Summing over instantons.}

As we remarked, the perturbation series with respect to powers of $e^2$
for the path integral with Lagrangian (6.1) does not involve contributions
from instantons, as they are exponentially small. Let us introduce 
a refined perturbation series, which will take instantons into account. 
For this purpose we will work on a Riemann surface of volume $V$, and 
introduce a new perturbation parameter $W=Ve^{-I}$.
We will consider the perturbation expansion with respect to both $W$ and $e$. 
The key fact is that in this
refined perturbation expansion, the only contributions of finite order
in $W$ and $e$ are from the instanton gas. Thus, the approximation 
to the partition function obtained this way has the form
$$
Z(e^2,\theta,W)=e^{VP_0}\sum_{n,m=0}^{\infty}
\frac{W^{m+n}P_+^nP_-^me^{i\theta(n-m)}}{m!n!},\tag 6.4
$$
where $P_0,P_+,P_-$ are the perturbation series around the zero
solution, the instanton, and the antiinstanton, respectively. 

Here the term with indices $m,n$ comes from a field configuration 
with $n$ instantons and $m$ antiinstantons, distant from each other. 
The factorials in the denominator arise from the fact that instantons 
and antiinstantons are not labeled, and their permutation does not change
the configuration.  

Summing (6.4) by ordinary calculus, we get
$$
Z(e^2,\theta,W)=e^{VP_0+W(P_+e^{i\theta}+P_-e^{-i\theta})}.\tag 6.5
$$
In our case, $P_+=P_-=P$, so we get
$$
Z(e^2,\theta,W)=e^{VP_0+2WP\cos\theta}.\tag 6.6
$$

Now we can compute the energy density of the vacuum in this approximation. 
This can be done from the equality $Z=e^{-VE_{vac}}$, which is 
the definition of the vacuum energy density $E_{vac}$. Thus, 
$$
E_{vac}=-P_0-2Pe^{-I}\cos\theta.\tag 6.7
$$
As we expected, the theta-dependence is exponentially suppressed, but we were 
able to compute the main term of this dependence. 

The next order correction to (6.7) is of order $e^{-2I}$, but it 
is hard to calculate. But if we had an instanton for $c_1=2$ with
$I'<2I$, then the correction would be of the form $e^{-2I'}\cos 2\theta$.  

We see that unlike the case $m^2>>0$, where $E_{vac}=e^2\theta^2/8\pi^2$
is a non-smooth function of $\theta$ (it has a cut at $\theta=\pm \pi$), 
in the case $m^2<<0$ the function $E_{vac}(\theta)=-P_0+2Pe^{-I}\cos\theta$
is smooth in the first approximation. Thus, for $m^2<<0$ there is no cut 
at $\theta=\pm \pi$, and there is a mass gap and a unique vacuum for any 
$\theta$. 

Thus, the cut at $\theta=\pm \pi$ has to add at some point $m^2=m^2_c\in \R$.  
At this point the effective potential of the theory (if it makes sense) 
has a quartic critical point at the origin, so that the theory has no 
mass gap. It is conjectured that this theory is conformal, and is the 
continuous limit of the 2-dimensional Ising model. 

{\bf 6.6. The Wilson line operator.}

In this and subsequent sections we will define the Wilson line operator, 
and try to understand its physical and formal properties. 

Suppose we have a gauge theory with gauge group $G$ on a spacetime $M$. 
Let $A$ denote the corresponding gauge field with values in the Lie algebra 
$\g$ of $G$. Let $C$ be a closed loop in $M$, and $Hol(A,C)\in G$ be the 
holonomy of the connection $A$ along $C$ (it is only defined up to 
conjugation). Let $R$ be a finite-dimensional representation of $G$. 
Define the classical Wilson line (or Wilson loop) functional to be
$$
W_R(C)(A)=\Tr|_R(Hol(A,C)).\tag 6.8
$$
It is clear that this functional is gauge invariant. 

If $C$ is a union of disconnected loops $C_i$ labeled with representations
$R=(R_i)$, then by definition $W_R(C)=\prod W_{R_i}(C_i)$. 

If $G$ is abelian, and $C$ is cotractible, then 
$W_R(C)=e^{i\int_D F}$, where $F$ is the curvature of $A$, and 
$D$ is a disk such that $\d D=C$. 

An important generalization of this is the following: 
$\hat G$ is the simply-connected cover of $G$, and $R$ is 
a representation of the universal covering 
$\hat G$ of $G$. In this case it is easy to see that 
$W_R(C)$ is still well-defined when $C$ bounds a disk $D$. 
(in the abelian case, it follows from the above integral representation
of $W$). 

Now we want to define an analogue of $W_R(C)$ in quantum theory. 
For this purpose we need to renormalize the classical functional
$W_R(C)$. This can be done by expanding $W_R(C)$ in powers of $A$ 
(like the exponential is expanded in Taylor series):
$$
W_R(C)=\text{dim}(R)+\int dl\Tr|_R(A(l))+
\frac{1}{2}\int dl dl'\Tr|_R(A(l)A(l'))+...,\tag 6.9
$$
where $l$ is some parameter on $C$, and 
$A(l)$ is the evaluation of the form $A$ on the 
tangent vector $\frac{d}{dl}$ to $C$ at the point $l$, with respect to this 
parametrization. Each term of this expansion
is polynomial and can be renormalized as usual. 

In fact, one can show in most cases that 
the operator $W_R(C)$ has only multiplicative renormalization. 
This follows from the fact that classically, 
$W_R(C)$ is the trace of the monodromy of the differential equation
$x'=Ax$ along the loop. This equation is renormalized to
$x'=(A+c)x$, where $c$ is an operator in the theory, invariant
under the same symmetries as $A$, in the adjoint representation of 
the gauge group, of dimension the same and lower than $A$. If there is
no such operators except for constants 
(constants come up for the $U(1)$-case)
then this equation will change to $x'=(A+c)x$ under 
renormalization, where $c$ is a scalar operator. This shows that $W_R(C)$ 
will have multiplicative renormalization. 

More precisely, one can show 
that in critical dimension 4, the divergent renormalization 
factor has the form $e^{L(C)\Lambda f(e^2)+o(\L)}$, 
where $\L$ is the cutoff and $L(C)$ is the length of $C$, while 
in the superrenormalizable case (in less than 4 dimensions), 
the divergent factor has power growth with respect to $\L$.

A more physical way of thinking of the Wilson loop operator 
is the following. We will use the Hamiltonian picture. 
Thus, $M=M_s\times \R$, where $M_s$ is the space manifold.
The Hilbert space of the theory is then $\Cal H$ the space of functions on 
$\Cal A$, where 
$\Cal A$ is the space of gauge classes of connections on
$M_s$. The space $\Cal H$ has the form $\Cal H=(\Cal H_0)^{\tilde G}$, 
where $\Cal H_0$ is the space of functions on all connections, 
and $\tilde G$ is the group of  gauge transformations.
 
Now let $C$ be a loop in $M$. 
Suppose first that the loop $C$ is in the submanifold 
$t=0$. In this case, the Wilson loop operator is obviously just 
the operator of multiplication by the function $W_R(C)$. 

Now consider the situation when $C$ is not in a horizontal section 
of the spacetime, but a general curve. In this case, $W_R(C)$ 
is no longer multiplication by a function.

  {\bf Remark. } 
We think of $C$ as a worldline of a ``charge'' 
which transforms in a representation $R$. The best is to think 
of two charges of type $R,R^*$ which are born at some time $t_0$ from nothing 
at the same point $x_0$, then fly around for a while (until $t_1$), and finally
recombine at the time $t_1$, back into nothing, at a point $x_1$. 
We think of these charges as classical, external objects. That is,
the expectations in the presence of $C$ are conditional 
expectations, given that the worldlines of the charges form the loop $C$. 
In this setting, we can regard $e^{iH(t_1-t_0)}W_R(C)$ 
as the evolution operator from time $t_0$ to time $t_1$, in the 
system with presence of $C$ (here $H$ is the Hamiltonian 
of the system). 

The operator $W_R(C)$ allows us to define the time ordered correlation
functions in the presence of $C$ for any set of local operators at points 
$(x,t)$ such that $t\notin [t_0,t_1]$, simply as correlation functions with 
the insertion of $W_R(C)$ in the right place. 
But what should we do for $t_0<t<t_1$? 

The most natural method is to set the Hilbert space of states 
at $t$ to be $\Cal H(t)=(\Cal H_0\o R(x(t))\o R^*(y(t)))^{\tilde G}$,
where $R(x)$, $R^*(y)$ are the evaluation representations of the group 
$\tilde G$ at the points $x.y$, and $x(t),y(t)$ are the positions
of the two charges at the time $t$. 
Define operators $W_R(C_+(t)): \Cal H\to \Cal H(t)$, $W_R(C_-(t)): 
\Cal H(t)\to \Cal H$ 
-- the Wilson ``open line'' operators corresponding to the upper and lower
half-loops $C_+(t),C_-(t)$ (the parts of $C$ lying above and below time $t$). 
The definition of the ``open line'' operators similar to the definition 
of the closed loop operators. 
Now, the expectation value of
a local operator $\O(x,t)$ in the presence of $C$ is just the usual 
expectation value of the operator $W_R(C_+(t))\phi(x,t)W_R(C_-(t))$.

{\bf 6.7. The path integral representation of the Wilson line operator.}

It is convenient to represent the classical Wilson line functional 
as a 1-dimensional path integral. Such a representation
 will be deduced in this section. 

Let $R$ be an irreducible, finite-dimensional representation of a  
simple compact Lie group $G$. By Borel-Weil theory, we can think of $R$ 
as $H^0(G/T,\Cal L_R)$, where $\Cal L_R$ is a suitable holomorphic 
line bundle. The bundle $\Cal L_R$ has a natural Hermitian metric, 
and therefore a natural connection $B$. For any map $\phi: S^1\to G/T$, 
define the action $I_R(\phi)=-\ln \text{Hol}(\phi^*(B))$ (Hol is for holonomy).
The number $I_R$ is only defined up to $2\pi in$, but 
$e^{-I_R(\phi)}=\text{Hol}(\phi^*(B))$ is well defined. 

Consider the path integral
$$
\int D\phi e^{-I_R(\phi)}.\tag 6.10
$$
It can be represented in the form
$$
\int D\phi e^{i\int_D \phi^*(F_B)},\tag 6.11
$$
where $F_B$ is the curvature of $B$, and $D$ is a disk whose boundary
is the image of $S^1$. 
The quantum theory defined by this path integral is the Hilbert space $R$ 
with the zero Hamiltonian (as the path integral is invariant under 
diffeomorphisms). 

Now let $P$ be the total space of 
a principal $G$-bundle over the circle, 
with a connection $A$, and let $P/T$ be the 
associated $G/T$ bundle. Since $\Cal L$ is a $G$-equivariant line bundle 
on $G/T$, it defines a complex line bundle on $P$, which we denote by 
$\Cal L_P$. The hermitian connection $B$ on $\Cal L$ defines a 
connection on $\Cal L_P$ in the vertical direction (along fibers of $P$). 
This connection can be naturally extended to a connection $B_P$ on $\Cal L_P$ 
using the connection $A$: in a local gauge where $P$ is trivial and $A=0$, 
$B$ is extended by the condition that constant paths   
are horizontal paths.

For any smooth section $\phi$ of $P/T$, define the action
$$
I(\phi,A)=-\ln \text{Hol}(\phi^*(B_P)).\tag 6.12
$$
As before, $I$ is defined up to $2\pi i n$. 

Let $W_R(A)$ denote the trace in $R$ of the holonomy of $A$ around 
the circle. 
It is now easy to see that the functional $W_R(A)$ has the following 
integral representation: 
$$
W_R(A)=\int_{\phi\in \Gamma(P/T)}e^{-I(\phi,A)}D\phi.\tag 6.13
$$

Now let $C$ is a loop in $M$, and $A$ is a connection in a principal 
$G$-bundle $P$ over $M$. Let $f: S^1\to C$ be a parametrization.
Then we have
$$
W_R(C)(A)=\int_{\phi\in \Gamma(f^*P/T)}e^{-I(\phi,f^*A)}D\phi.\tag 6.14
$$

This gives an integral representation of the Wilson line operator.
This representation allows us to represent the theory in the presence of $C$
(for instance, for pure gauge theory) by a double path integral
$$
\int DAD\phi e^{-\frac{1}{4e^2}\int \Tr |F^2|-I(\phi,A)}.\tag 6.15
$$

 {\bf Remark.} 
Note that the integrand of (6.13),(6.14) is the holonomy of an abelian 
connection, so it can be explicitly computed as the exponential 
of the integral of curvature. This, we have represented the monodromy 
of an arbitrary linear differential equation on the circle 
as an explicit integral over infinitely many auxiliary 
variables. It is well known in classical analysis 
that such a reopresentation with finitely many auxiliary variables
is, in general, impossible: it is only possible for some special 
equations of ``hypergeometric'' type. 

{\bf 6.8. The Higgs and the confinement regimes.} 

Consider a Wilson loop which is approximately a rectangle: 
two charges are born, move away from each other at distance $L$, 
stay at that distance from each other for a time $T>>L$, and then 
annihilate each other. Then 
The expectation value $\<W_R(C)\>$ of the Wilson loop operator 
(in the Euclidean setting) has the following meaning: it is approximately 
equal to $Ce^{-TV(L)}$, 
where $V(L)$ is the energy of interaction of the charges
at distance $L$, and $C$ is a constant. 
This is clear from the interpretation of the Wilson 
loop operator as an evolution operator, which is given above. 

So the asymptotics of $\<W_R(C)\>$ depends on the asymptotics 
of $V(L)$ as $L\to \infty$. 

Physicists believe that above 2 dimensions, in gauge theories 
with a mass gap, there are
two possibilities:

1. Higgs regime: $V{L}\to \text{const}, L\to \infty$.

In this regime, charges can separate from each other. 

2. Confinement regime: $V(L)\sim \text{const}L$ as $L\to \infty$,
where the constant is positive. 

In this regime charges are confined, and cannot separate without 
spending an arbitrarily large amount of energy. 

Now let us consider a Wilson loop $C$ of any shape, with circumference $S$ and 
minimal area of the spanning surface $A$. 
There is a general conjecture that patterns 1 and 2, if they hold
for $T>>L>>0$, hold for an arbitrary $C$ (large in all directions) 
in the following form:
the Higgs regime corresponds
to the asymptotics
$\<W_R(C)\>\sim e^{-wS}$ (the circumference law), 
and the confinement regime corresponds
to the asymptotics $\<W_R(C)\>\sim e^{-wA}$ (the area law), where $w>0$. 

{\bf Remark.} Actually, confinement can only occur if
$R$ is a representation of the universal cover $\hat G$ of $G$ and not 
of $G$ itself. Indeed, if $R$ is a representation of $G$, 
there are physical processes contributing to $\<W_R(C)\>$ 
in which large portions of the Wilson line have zero charge
(i.e. carry the trivial representation of $G$)
(i.e. some particles have annihilated the charges on the Wilson line).
These processes have an amplitude which is bigger than that
predicted by the area law. 
 
As a toy example, let us consider the 2-dimensional theory with Lagrangian
(6.1), which we studied in the first part of the lecture. 
In this case, $G=U(1)$. Let $R$ be the representation of $\hat G$ defined by 
$\l\in \R$: $x\to e^{i\l x}$. Then, classically, 
$W_R(C)=e^{i\l\int_D F_A}$. 
Thus, quantum mechanically
$$
\<W_R(C)\>=\int DAD\phi D\bar\phi e^{-\Cal L}e^{i\l\int_D F}. \tag 6.16
$$ 
Let $V$ be the volume of spacetime, and $A_C$ be the area 
enclosed by $C$. We will split the path integral (6.16) 
into a product of two -- the integral over values
of fields inside the loop $C$ and over the values outside. 
Observe that the last factor in (6.16) 
(the holonomy factor) is 
of the same type as the topological term in (6.1).
Therefore, according to the results 
of the first part of this lecture, (6.16) yields
$$
\<W_R(C)\>= \frac{1}{Z}e^{-(V-A_C)E(\theta)-A_CE(\theta+\l)},\tag 6.17
$$
up to boundary terms, which we will neglect here. 
(Here $E(\theta)$ is the vacuum energy density). 
Since $Z=e^{-VE(\theta)}$, we get
$$
\<W_R(C)\>=e^{A_C(E(\theta)-E(\theta+\l))}.\tag 6.18
$$

According to our calculations, for $\theta=0$
$$
E(\theta+\l)-E(\theta)=E(\l)-E(0)=2(1-cos \l)e^{-I/e^2}P. \tag 6.19
$$
This is always positive when $\l$ is not a multiple of $2\pi$. Thus,  
theory (6.1) for $\theta=0$ exhibits the confinement regime. 

In more than 2 dimensions, one expects that this theory obeys Higgs 
regime. 

{\bf 6.9. The confinement conjecture.}

The following conjecture is central in quantum field theory.

\proclaim{Conjecture} Let $G$ be a simple compact Lie group,
and $R$ be a representation of $G$ which is not a 
representation of the adjoint group $G_{ad}$. In 3 and 4 dimensions, 
the pure gauge theory with gauge group $G$ and Lagrangian
$\int \Tr F\wedge *F$ exhibits confinement
for charges with values in $R$ and $R^*$.
\endproclaim

The physically interesting case of this theory is $G=SU(3)$, $R=\C^3$. 
This special case of the conjecture would explain confinement of quarks. 

In 2 dimensions, this conjecture is true, as we saw in the previous lectures. 




\end

