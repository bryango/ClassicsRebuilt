%From: Lisa C Jeffrey <jeffrey@math.ias.edu>
%Date: Wed, 16 Apr 1997 16:26:54 -0400


%Latex file

\documentclass[12pt]{article} 

\newcommand{\printname}[1]
        {\smash{\makebox[0pt]{\hspace{-1.0in}\raisebox{8pt}{\tiny #1}}}}

\newcommand{\labell}[1] {\label{#1}}

\setlength{\textwidth}{6.5 truein}
\setlength{\textheight}{8.0 truein}
\begin{document}










\vfill\eject






\noindent
{\bf Fall Term Exam, N$^{\rm{0}}$. 1}\qquad\qquad\qquad
\qquad\qquad\qquad(solution by L. Jeffrey and S. Wu)

\smallskip
\hbox to \hsize{\hrulefill}

\bigskip
\noindent
{\bf Problem:}
\medskip



\noindent Prove via path integrals the character valued index theorem for
the Dirac operator. 
\medskip

\noindent That is, let $M$ be a spin manifold and 
let $g$ be an isometry of $M$, with a lift to the spin bundles. 
Calculate 
$$
Tr  (-1)^F g
$$
by path integrals. 

\medskip
\noindent That is, write down the path integral for sections of a suitable $M$ 
bundle over a circle. Taking the circumference of the circle to be small,
compute the integral explicitly and obtain the Atiyah-Bott formula 
for (1).

\newcommand{\RR}{{\mathbf{R}}}
\newcommand{\CC}{{\mathbf{C}}}
\newcommand{\ZZ}{{\mathbf{Z}}}
\newcommand{\calh}{ { \cal H} } 
\newcommand{\cals}{ { \cal S} } 
\newcommand{\cald}{ { \cal D} } 
\newcommand{\norm}{{\cal N}}


\bigskip
\noindent
{\bf Solution:}
\medskip 


We use a supersymmetric quantum field theory where the fields
are maps $X: \RR^{1,1} \to M$ satisfying a periodicity condition.
We write the maps as $X (t, \theta)$ where $t$ is the even variable 
and $\theta$ the odd variable: decomposing into bosonic and fermionic
fields we write
$$ X(t,\theta) = x(t) + \theta \psi(t).$$
The Lagrangian is
$$ L(X) = \int dt d\theta g_{IJ} \frac{dX^I }{dt} D X^J $$
where we have introduced a metric $g_{IJ}$ on $M$ and the 
 operator
$$ D = \frac{\partial}{\partial \theta} - i \theta \frac{\partial}{\partial t}
. $$
In components the action becomes
$$ L(X) = \int dt \{ g_{IJ}  \frac{dx^I}{dt} \frac{dx^J}{dt} 
+ \frac{i}{2} g_{IJ} \psi^I \frac{D\psi^J}{Dt} \}, $$
where $D \psi^J/Dt$ denotes the covariant derivative (with respect
to the Levi-Civita connection) in the direction of 
the tangent vector $x_* (d/dt)$ in  $M$.
The supersymmetry operator is 
$$Q = \frac{\partial }{\partial \theta} +i \theta \frac{\partial }{\partial t}.
$$
\newcommand{\dbar}{{ \cal D} }
Upon quantization the space of states is identified
with the sections $\calh_{\pm} $ of the 
spin bundle $\cals_{\pm} $ on $M$ and the operator $Q$ is identified
with the Dirac operator $\dbar$ on $M$.
We introduce the operator $(-1)^F$ on $\calh_{\pm}$ which is defined
to have the values $+1$ on $\calh_+$
(states with an even number of fermions) and $(-1)$ on 
$\calh_-$ (states with an 
odd number of fermions). The index of the Dirac operator 
$\dbar$ is then
\begin{equation} \labell{dirac}
{\rm Index} (\dbar) = {\rm Trace} (-1)^F  = 
{\rm Trace } (-1)^F e^{- \beta H}, 
\end{equation}
where we have introduced the Hamiltonian $H$ and a positive parameter
$\beta$. 
Notice that since the Dirac operator $\dbar$ gives an 
isomorphism between the
eigenspaces $\calh_+^\lambda$ of $H$ on $\calh_+$ with a positive
eigenvalue $\lambda$ and the corresponding
eigenspaces $\calh_-^\lambda$ of the action of $H$ on $\calh_-$,
the trace of $(-1)^F e^{- \beta H} $ is in fact independent of 
$\beta$. 
This can be computed by the path integral 
\begin{equation} \labell{lagrangian}
{\rm Trace } (-1)^F e^{- \beta H} = \int \cald X e^{ - \int_{0}^\beta
L(X) dt},
\end{equation}
where we integrate over fields $X(t) $ satisfying periodic
boundary conditions $X(t+1) = X(t)$. 

The corresponding character valued index is
\begin{equation} \labell{diracg}
{\rm Index}_g (\dbar) = {\rm Trace}  (-1)^Fg  = 
{\rm Trace } (-1)^Fg e^{- \beta H}.
\end{equation}
Since the isometry $g$ commutes with the Dirac operator, the trace 
${\rm Trace } (-1)^Fg e^{- \beta H}$ is again independent of $\beta$.
It  is given by  the path integral 
\begin{equation} \labell{lagrangiang}
{\rm Trace } (-1)^F g e^{- \beta H} = \int \cald X e^{ - \int_{0}^\beta
L(X) dt},
\end{equation}
which is the same as
(\ref{lagrangian}) but with the boundary 
conditions on the fields 
changed to 
\begin{equation} \labell{bc}
x(t+1) = g x(t), ~~~ \psi(t+1) = g \psi(t). 
\end{equation}

We now rescale the time variable $t $  by replacing it by 
$t' = \beta t$ where $t \in [0,1]$:
this is equivalent to replacing
the action $\int_{0}^\beta L(X) dt$ by 
\begin{equation}\labell{resclag}
\frac{1}{\beta}   \int_0^1 \Biggl \{ g_{IJ} \frac{dx^I}{dt} \frac{dx^J}{dt} 
+ \frac{i}{2} g_{IJ}\psi^I \frac{D\psi^J}{Dt}\Biggr \} , 
\end{equation}
where we have also rescaled the fermionic variables $\psi^I$ by
multiplying them by a factor $1/\sqrt{\beta}$.


\newcommand{\bose}{{\alpha } }
\newcommand{\bosep}{ { \bose}^{\parallel} }
\newcommand{\bosen}{ { \bose}^{\perp} }
\newcommand{\etap}{ { \eta}^{\parallel} }
\newcommand{\etan}{ { \eta}^{\perp} }


We must first identify the solutions of the Euler-Lagrange equation.
This equation is 
$$dX^I/dt = 0 ; $$
given the twisted periodicity condition  (\ref{bc}) the only solutions
are constant paths
$x_0(t) = x_0$ which send $\RR $ to  a point $x_0$ in the fixed point set 
$Z = M^g$ of $g$, and constant paths
$\psi_0(t) = \psi_0 $ which send
$\RR$ to  a point $\psi_0$ in the tangent space $\Pi T_{x_0} M$. 

If $X_0 = (x_0, \psi_0)$ is a solution of the Euler-Lagrange equation,
we write an infinitesimal variation
$\delta X $ at $X_0$ as 
$$ \delta X = (\bose, \eta)$$ 
(where $\bose$ is the bosonic piece, a path in 
$T_{x_0} M$,  and $\eta$ is the fermionic piece, a path in 
$\Pi T_{x_0} M$).
To avoid overcounting
 infinitesimal variations $\delta X$ which are simply deformations
within the class of solutions $X_0$ of the Euler-Lagrange equation, we 
impose the conditions 
\begin{equation} \labell{overcount}
  \int_0^1 \bose(t) dt = 0, ~~\int_0^1 \eta (t) dt = 0.
\end{equation}



We decompose the infinitesimal variations
$(\bose, \eta)$  further into 
$$ \bose = \bosep + \bosen, ~~ \eta = \etap + \etan $$
where $\bosep$ is a path in the tangent space $T_{x_0} Z$ at $x_0$
to the 
fixed point set $Z$, while
$\bosen$ is a path in the fibre $\norm_{x_0} $  of the normal 
bundle $\norm$ to $Z$ in $M$.
The fermionic piece $\eta$ is similarly decomposed into a path
$\etap$ in the tangent space $T_{x_0} Z$ and a path 
$\etan$ in the fibre $\norm_{x_0}$ of the normal bundle.
These components separately satisfy the condition 
(\ref{overcount}).
The endpoint conditions become
\begin{equation} \labell{endp}
\bosep(1) = \bosep(0), ~~ \etap(1) = \etap(0)
\end{equation}
\begin{equation} \labell{endn}
\bosen(1) = g \bosen(0), ~~ \etan(1) = g\etan(0)
\end{equation}
where in (\ref{endn}) $g$  acts as an automorphism
of the fibres of  the normal bundle to $Z$ in $M$.

\newcommand{\opbose}{{D_B}}
\newcommand{\opferm}{{D_F}}

Expanding the Lagrangian to second order in $(\bose, \eta)$ we find
$L = L^\parallel + L^\perp, $ where 
\begin{equation} \labell{lpar}
L^\parallel = \frac{1}{\beta} \int_{0}^1 dt 
\Biggl ( \frac{1}{2} |\frac{d\bosep}{dt}|^2 
- \frac{i}{4} \psi_0^I \psi_0^J R_{IJKL} (\bosep)^K \frac{d (\bosep)^L}{dt}
+ \frac{i}{2} 
(\etap, \frac{d\etap}{dt})  \Biggr )
\end{equation}
while 
\begin{equation} \labell{lperp}
L^\perp = \frac{1}{\beta} \int_{0}^1 dt 
\Biggl ( \frac{1}{2} |\frac{d\bosen}{dt}|^2 
- \frac{i}{4} \psi_0^I \psi_0^J R_{IJKL} (\bosen)^K \frac{d (\bosen)^L}{dt}
+ \frac{i}{2} 
(\etan, \frac{d \etan}{dt})  
\Biggr ).
\end{equation}
(Because the isometry $g$ preserves the Levi-Civita
connection, the fixed 
point set $Z$ is totally geodesic. The solutions 
$\psi_0 $ to the Euler-Lagrange equation are tangent
to the fixed point set $Z$.
Thus, in  (\ref{lpar}) the indices $I,J,K,L$ run over a basis for
the tangent bundle to the fixed point set $Z$, while in 
(\ref{lperp}) the indices $I,J$
 run over a basis for the tangent space to $Z$ 
while the indices $K,L$ run over a basis for the normal bundle $\norm$ to 
the fixed point set.)

The bosonic parts of these
Lagrangians are given by a differential operator $(1/\beta)\opbose$ 
of order 2 acting on $\bosep$ (resp. $\bosen$), where $\opbose$ is defined
by
\begin{equation} \labell{opbose}
(\opbose)_{KL}  =   - \frac{d^2}{dt^2} \delta_{KL} -  F_{KL} 
\frac{d}{dt}, 
\end{equation}
where $F = (F_{KL})$  is the curvature 
$(i/2)\psi_0^I \psi_0^J R_{IJKL} $ of the
Levi-Civita connection on $M$. 
Similarly the fermionic Lagrangians are given by a differential
operator $(1/\beta)$ $\opferm$ (where $\opferm= i\frac{D}{Dt}$
has order 1)
 acting on $\etap $ (resp. $\etan$).
(At this point we may use normal coordinates to identify the 
covariant derivative $D/Dt$ with the ordinary
derivative $d/dt$.)

We compute the path integral (\ref{lagrangiang}) perturbatively.
Since the result is independent of $\beta$, we may evaluate the
integral in the limit $\beta \to 0$: however, since the propagator is
multiplied by a factor $\beta$, this means that in the 
limit $\beta \to 0 $ the exact answer is given by  the leading order
term in perturbation theory (i.e. the ratio 
of determinants arising from the Gaussian integral). 
In other words, the path integral (\ref{lagrangiang}) becomes  
\begin{equation} \label{pathintres}
\frac{ {\rm Pf}(\opferm )}{\sqrt{\det \opbose} }. 
\end{equation}
Since the operators $\opbose$ and $\opferm$ preserve the tangent
bundle to $Z$ and the normal bundle to $Z$
(in other words $\opbose$  decomposes into operators 
$$\opbose = 
\opbose^\parallel \oplus \opbose^\perp,$$ and 
similarly for $\opferm$: this
will follow from the decomposition (\ref{lpar})-(\ref{lperp})
of the Lagrangian given below) 
 our path integral decomposes as 
$$\int_Z \frac{ {\rm Pf}(\opferm^\parallel )}{\sqrt{\det \opbose^\parallel} }
\frac{ {\rm Pf}(\opferm^\perp )}{\sqrt{\det \opbose^\perp} }. $$

The computation of 
$$\frac{ {\rm Pf}(\opferm^\parallel )}{\sqrt{\det \opbose^\parallel} } $$
is identical to the computation presented in the lecture and yields
\begin{equation} \labell{intpar}
\hat{A}(F^\parallel) = \frac{1}{(2\pi)^{\dim Z/2}}
 \prod_{a= 1}^{\dim Z/2}  \frac{F^\parallel_a/2}{\sinh F^\parallel_a/2}.
\end{equation}
$\hat{A}(Z)=\hat{A}(F^\parallel/(2\pi))$ represents the A-hat genus 
of the fixed point set $Z$.
Here, we have decomposed the curvature $F^\parallel$ of the Levi-Civita
connection of $Z$ (which is a matrix valued two-form on $Z$ with values
in skew-symmetric  square matrices of dimension $\dim Z$)
into its Chern roots: in other words  we have skew diagonalized it and 
written it in the form
\begin{equation} \labell{diagcurv} F^{\parallel} = 
\left ( \begin{array}{lccccr} 
0 & F^1 & 0  & 0 & 0 \dots& 0   \\
 - F^1 & 0 & 0  & 0& 0 \dots & 0 \\
0& 0& 0  & F^2 & 0 \dots & 0\\ 
0& 0& -F^2  & 0 & 0 \dots & 0\\ 
\vdots& \vdots& \vdots & \vdots & \dots & \dots \\
\end{array} \right ) \end{equation}

\newcommand{\eig}{\vartheta}

The computation of the determinants for $\opbose^\perp$ and
$\opferm^\perp$ differs from the computation presented in the lecture
only because new  periodicity conditions (\ref{bc}) for the fields  have
been introduced. 
We first observe that since $g$ is orientation preserving,
the normal bundle $\norm$ to each component
of the fixed point set $Z$ of $g$ is even dimensional.
Moreover, the multiplicity of the eigenvalue $-1$ is even.
Therefore the fiber $\norm_{x_0}$ over $x_0\in Z$ decomposes into
subspaces $V_b$ invariant under $g$, $b = 1,\dots,(\dim_\RR\norm)/2$.
Each $V_b$ has real dimension 2 and $g$ acts on $V_b$ as
\begin{equation}
g = \left (\begin{array}{lr} \cos \eig_b & \sin \eig_b \\
       - \sin \eig_b & \cos \eig_b \end{array} \right )
\end{equation}
Here a pair of eigenspaces of eigenvalue $-1$ corresponds to one $V_b$
with $\eig_b=\pi$.
(Alternatively, identifying each eigenspace 
$V_b$ with $\CC$ the isometry  $g$ acts as multiplication by 
$e^{i \eig_b}$. Of course, here the $\eig_b$ need not be distinct.)
Since the isometry $g$ preserves the Levi-Civita connection,
the basis of the normal bundle which skew-diagonalizes the curvature
(as in (\ref{diagcurv}))
may be chosen to coincide with the basis of eigenspaces of $g$.
Thus we may compute the action of 
$\opbose^\perp$ and $\opferm^\perp$  acting on paths 
$\bosen_b, \etan_b$ taking
values in one eigenspace $V_b$. 


In fact, it is convenient to  replace $\bosen_b$ 
and $\etan_b$ by  paths $\bosen_{b,\pm}, \etan_{b,\pm}$ in the 
complexification $V_b \otimes \CC$: this decomposes naturally 
as $V_b \otimes \CC = V_b^+ \oplus V_b^-$, where
$g$ acts on $V_b^\pm$ with the eigenvalue $e^{\pm i \eig_b}$.
Thus we 
may convert the endpoint conditions (\ref{bc}) to periodic endpoint
conditions by writing
\begin{equation} \label{transbose}
 \bosen_{b,\pm} (t) = \exp (\pm i t \eig_b) \zeta^\pm_b(t), 
\end{equation}
\begin{equation} \label{transferm}
 \etan_{b,\pm} (t) = \exp (\pm i t \eig_b) \xi^\pm_b(t), 
\end{equation}
where $\zeta^\pm_b: \RR \to V_b^\pm$ and
$\xi^\pm_b: \RR \to \Pi V_b^\pm$ satisfy the usual periodic
boundary conditions
$\zeta^\pm_b(t+1) = \zeta^\pm_b(t), $
$\xi^\pm_b(t+1) = \xi^\pm_b(t). $


If we substitute (\ref{transbose}) into the definition of $\opbose$ 
we find that the operator $\opbose$ acting on the 
variables $ \zeta^\pm_b$ becomes instead (using  the decomposition
of $F^\perp$ corresponding to 
(\ref{diagcurv}), where the eigenspaces are the eigenspaces
$V_b^\pm$: the subspaces which diagonalize the curvature coincide
with the eigenspaces of $g$ because the isometry $g$ is assumed to 
preserve the Riemannian metric on $M$)
\begin{equation} \labell{bosop}
(\opbose)^\pm_b = -  (\frac{d}{dt} \pm i \eig_b)^2 
\pm  F_b (\frac{d}{dt} \pm i \eig_b). 
\end{equation}
Since the eigenvalues of $\frac{d}{dt}$ on the 
$\zeta_b$ are $2 \pi i n$ for $n \in \ZZ$, we find that the eigenvalues
are given by 
\begin{equation} \labell{eigval}
(2 \pi i n \pm i \eig_b)^2 \pm F_b (2 \pi i n \pm i \eig_b). 
\end{equation}
To compute the determinant of $\opbose$ we take the product over all
the eigenvalues.

The computation of the character-valued index differs from the ordinary
index in that the fermionic Pfaffian now depends on the values
of the eigenvalues $\eig_b$ for the isometry.
Substituting  (\ref{transferm}) into $\opferm = id/dt$ we find
that the eigenvalues of $\opferm$ acting on  the 
periodic fermionic   paths
$\xi^\pm_b$ are
$$ 2 \pi  n \pm  \eig_b.$$
We shall make use of the identity
\begin{equation} \label{fundid}
\prod_{n \ge 1} (1 - \frac{z^2}{4 \pi^2 n^2} ) = 
\frac{\sin z/2 }{z/2} 
\end{equation} 
Combining all factors we find
\begin{equation} \labell{detbos}
\det \opbose = \prod_{\pm} \prod_{n  \in \ZZ } \prod_b 
(2 \pi in + i\eig_b)^2 \pm {F_b }(2 \pi in \pm i\eig_b) 
\end{equation}
while the fermionic Pfaffian is given by 
\begin{equation} \labell{detferm}
({\rm Pf} \opferm)^2 = \prod_{\pm} \prod_b \prod_n 
(2 \pi n \pm \eig_b). 
\end{equation}
Combining (\ref{detbos}) and (\ref{detferm}) we see that
(up to a multiplicative constant from regularization)
\begin{equation} \labell{detcomb}
\frac{({\rm Pf} \opferm)^2}{\det \opbose} = 
\prod_{n \in \ZZ} \prod_{b, \pm} \frac{1}{ 2 \pi n \pm (\eig_b + iF_b)},
\end{equation}
$$ = 
\prod_{b} \frac{1}{\eig_b + i F_b} \prod_{\pm} \prod_{n \ge 1} 
\frac{1}{4 \pi^2 n^2 - (\eig_b + i F_b)^2 } $$
$$ = \prod_b \frac{1}{2 \sin \Bigl (  (\theta_b + i F_b)/2\Bigr ) }. $$
The final answer for the character-valued index of the Dirac operator 
is then 
\begin{eqnarray} \label{finans}
{\rm Index}_g(\dbar)&=&
\frac{1}{(2\pi)^{\dim Z/2}}\int_Z \hat{A} (F^\parallel) 
\prod_{b= 1}^{\dim_{\RR} \norm/2}
\frac{1} {2 \sin \Bigl ( (\eig_b + i F_b)/2\Bigr ) }	\nonumber \\
&=& \int_Z \hat{A} (Z) 
\prod_{b= 1}^{\dim_{\RR} \norm/2}
\frac{1} {2 \sin \Bigl ( (\eig_b + i F_b/(2\pi))/2\Bigr ) }.
\end{eqnarray}






 




\noindent{\bf References:}

1. L. Alvarez-Gaum\'e, Supersymmetry and the Atiyah-Singer
index theorem, \emph{Commun. Math. Phys.} \textbf{90} (1983) 161-173.

2. D. Friedan, P. Windey, Supersymmetric
derivation of the Atiyah-Singer index
theorem and the chiral anomaly, \emph{Nucl. Phys.} {\bf B 235} (1984) 385-416.

3. M. Goodman, Proof of character-valued index theorems,
\emph{Commun. Math. Phys.} \textbf{107} (1986) 391-409.

4. M. Goodman, E. Witten, Global symmetries in four and higher
dimensions, \emph{Nucl. Phys. B} \textbf{271} (1986) 21-52.










\end{document}



