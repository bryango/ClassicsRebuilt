%From: Pierre Deligne <deligne@math.ias.edu>
%Date: Thu, 31 Oct 1996 16:38:52 -0500
%Subject: Witten's Set one, No. 1

%Below is the AMS-TeX file for the
%solution written by D. Freed to Witten's Problems, Set one,
%No. 1, to be put on the web page.

\input amstex
\documentstyle{amsppt}
\magnification=1200
\pagewidth{6.5 true in}
\pageheight{8.9 true in}
\loadeusm

\catcode`\@=11
\def\logo@{}
\catcode`\@=13

\NoRunningHeads

\font\boldtitlefont=cmb10 scaled\magstep1

\def\dspace{\lineskip=2pt\baselineskip=18pt\lineskiplimit=0pt}
\def\wedgeop{\operatornamewithlimits{\wedge}\limits}
\def\w{{\mathchoice{\,{\scriptstyle\wedge}\,}
  {{\scriptstyle\wedge}}
  {{\scriptscriptstyle\wedge}}{{\scriptscriptstyle\wedge}}}}
\def\Le{{\mathchoice{\,{\scriptstyle\le}\,}
  {\,{\scriptstyle\le}\,}
  {\,{\scriptscriptstyle\le}\,}{\,{\scriptscriptstyle\le}\,}}}
\def\Ge{{\mathchoice{\,{\scriptstyle\ge}\,}
  {\,{\scriptstyle\ge}\,}
  {\,{\scriptscriptstyle\ge}\,}{\,{\scriptscriptstyle\ge}\,}}}
\def\vrulesub#1{\hbox{\vrule height7pt depth5pt\,}_{#1}}
\def\mapright#1{\smash{\mathop{\,\longrightarrow\,}%
     \limits^{#1}}}

\def\eps{{\varepsilon}}

\def\Adot{\Dot{A}}
\def\Bdot{\Dot{B}}
\def\Xdot{\Dot{X}}

\def\dbR{{\Bbb R}}

\def\GL{\text{\rm GL}} \def\Hom{\text{\rm Hom}}
\def\aff{\text{\rm aff}} \def\red{\text{\rm red}}


\def\scr#1{{\fam\eusmfam\relax#1}}

\def\scrA{{\scr A}}   \def\scrB{{\scr B}}
\def\scrC{{\scr C}}   \def\scrD{{\scr D}}
\def\scrE{{\scr E}}   \def\scrF{{\scr F}}
\def\scrG{{\scr G}}   \def\scrH{{\scr H}}
\def\scrI{{\scr I}}   \def\scrJ{{\scr J}}
\def\scrK{{\scr K}}   \def\scrL{{\scr L}}
\def\scrM{{\scr M}}   \def\scrN{{\scr N}}
\def\scrO{{\scr O}}   \def\scrP{{\scr P}}
\def\scrQ{{\scr Q}}   \def\scrR{{\scr R}}
\def\scrS{{\scr S}}   \def\scrT{{\scr T}}
\def\scrU{{\scr U}}   \def\scrV{{\scr V}}
\def\scrW{{\scr W}}   \def\scrX{{\scr X}}
\def\scrY{{\scr Y}}   \def\scrZ{{\scr Z}}



\document
\line{{\boldtitlefont Witten's Problems}, Set One --- 
N$^{\text{o}}$. 1 
\hfill (solution written by D. Freed)}
\smallskip
\hbox to \hsize{\hrulefill}

\bigskip
\dspace
\subhead
1
\endsubhead
We denote the Riemannian metric on $M$ by
$(-\,\,,\,\,-)$ and so write the lagrangian density as
$$
\scrL=\tfrac12(\Xdot,\Xdot)dt\,\,.
$$
Here $\Xdot=\frac{dX}{dt}$ is the tangent to the curve.

\medskip\noindent
(a)\enspace
We compute the variation by considering a path of $X$'s,
denoted $X_u$, $u\in(-\eps,\eps)$.
The computation takes place on $(-\eps,\eps)\times\dbR$:
There is a map $X\colon\,(-\eps,\eps)\times\dbR\to
M$ and we use the pullback metric, the pullback tangent
bundle, and the pullback of the Levi-Civita connection,
which we denote $\nabla$.
We use the notation $\delta=\frac{\partial}{\partial
u}$.

Since $[\partial_t, \delta]=0$ and $\nabla$
is torsion free, we have\footnote"$^*\,$"{If $x^i$ are
local coordinates, and $\Gamma_{jk}^i$ the corresponding
Christoffel symbols on $M$, then the pullback by $X$
makes the $x^i$ functions of $t$, $u$ and equation (1.1)
follows, since
$$
\tfrac{\partial^2x^i}{\partial t\partial u}+
\Gamma_{jk}^i\tfrac{\partial X}{\partial t}\tfrac{\partial
X}{\partial u}
$$
is symmetric in $t$, $u$.
(In other words, $\Gamma_{jk}^i$ is symmetric in $j$,
$k$).
A derivation without local coordinates is given at the
end of the exercise in an appendix.}
$$
\nabla_t\delta X=\nabla_\delta \Xdot\,\,.
\tag1.1
$$
So we make the computation
$$
\aligned
\delta\scrL &=\tfrac12\delta(\Xdot,\Xdot)dt\\
&=\left(\nabla_{\delta}\Xdot,\Xdot\right)dt\\
&=\left(\nabla_{t}\delta X,\Xdot\right)dt\\
&=d\left(\delta X,\Xdot\right)-\left(\delta
X,\nabla_{t}\Xdot\right)dt
\endaligned
\tag1.2
$$
Now the first term in the final line is a total
derivative, and the second is linear over the functions 
(as a function of $\delta X$).
So the variational derivative is
$$
\tfrac{\delta\scrL}{\delta X(t)}=-\nabla_t \Xdot\,dt
$$
The space of critical points is then
$$
W=\left\{X\colon\,\, \nabla_{t}\Xdot=0\right\}\,\,.
$$
The image of any such $X$ is a geodesic; it is
parametrized at constant speed.

\medskip\noindent
(b)\enspace
We assume that $M$ is complete.
Then the standard theory of ordinary differential
equations shows that there is a unique solution $X$ to
$\nabla_{t}\Xdot$ given an initial position and
velocity, i.e., a point of $TM$.

\medskip\noindent
(c)\enspace
To compute the symplectic structure we use the exact
term in formula (1.2).
Set
$$
\gamma=\left(\delta X,\Xdot\right)\vrulesub{t_0}\,\,,
\tag1.3
$$
where $t_0$ is some fixed time.
Then the symplectic structure is computed as the
differential of $\gamma$ in the space of paths:
$$
\omega=\delta\gamma=-\left(\delta X,\nabla_{\delta}
\Xdot\right)\vrulesub{t=t_0}=
\left(\nabla_{\delta}\Xdot,\delta
X\right)\vrulesub{t=t_0}
\tag1.4
$$
We elaborate: \ If $\xi_1$, $\xi_2$ are sections of
$X^*(TM)$ --- in other words, are tangent at $X$ to the
space of paths --- then
$$
\omega(\xi_1,\xi_2)=\left(\nabla_{\xi_1}\Xdot(t_0),
\xi_2(t_0)\right)-\left(\nabla_{\xi_2}\Xdot(t_0),
\xi_1(t_0)\right)\,\,.
$$

Now the standard symplectic structure on $T^*M$ is the
differential of the $1$-form
$$
\theta_\alpha(\eta)=\alpha(\pi_*\eta),\qquad
\alpha\in T^*M,\,\,\eta\in T_\alpha(T^*M),\tag1.5
$$
where $\pi\colon\, T^*M\to M$ is projection.
We identify $X\in W$ with the tangent vector\break
$\left<X(t_0),\Xdot(t_0)\right>\in TM$.
Then
$$
\gamma(\eta)=\left(\pi_*\eta,\Xdot(t_0)\right)\,,\qquad
\eta\in T_{X(t_0)}(TM).
\tag1.6
$$
Under the map $TM\to T^*M$ given by the metric, it is
easy to see that (1.5) pulls back to (1.6) at the point
$\left<X(t_0),\Xdot(t_0)\right>\in TM$.

\medskip\noindent
(d)\enspace
The vector field $v$ on $W$ may be described by the
equation
$$
v_X=\Xdot\in C^\infty(\dbR,X^*(TM))\,\,.
$$
To compute $\iota_v\omega$ we substitute $\delta X=\Xdot$
into (1.4):
$$
\align
\iota_v\omega &=\left(\nabla_{t}\Xdot,\delta X\right)
\vrulesub{t_0}-\left(\nabla_{\delta}\Xdot,\Xdot\right)
\vrulesub{t_0}\\
&=-\left(\nabla_{\delta}\Xdot,\Xdot\right)
  \vrulesub{t_0}\,\,,
\endalign
$$
since $X$ satisfies the eqution of motion
$\nabla_{t}\Xdot=0$.
It is easy to see that $\iota_v\omega=dH$ where $H$ is the
function
$$
H(X)=-\tfrac12\,\vert\Xdot(t_0)\vert^2\,\,.
$$
(Of course, this is independent of $t_0$.)

\newpage

\subhead
Remark on {\rm Equation (1.1) (by P. Deligne)}
\endsubhead

An affine space $A$ gives rise to a vector space $V$,
the translations of $A$, and it is convenient
(barycentric calculus) to view $A$ and $V$ as living in
the same vector space $B$ of dimension one more:
$$
0@>>> V@>>> B@>{m}>> \dbR@>>> 0,\qquad
A=m^{-1}(1)\,\,.
$$

If $A$ is an affine space bundle over a manifold $M$
(structural group the affine group
$\GL(n)\ltimes\dbR^n$), a connection $\nabla$ on $A$
allows us to define $\nabla_X a$, for $a$ a section of
$A$, as a section of the associated vector bundle (vector
bundle of translations of $A$), structural group
$\GL(n)$, obtained by applying $\GL(n)\ltimes \dbR^n\to
\GL(n)$.

Let  $T^{\aff}$ be the tangent bundle $T$ to $M$, viewed
as a bundle of affine space.
The vector bundle $T$ is the vector bundle associated to
$T^{\aff}$.
A connection $\nabla$ on $T$ defines as follows a
connection $\nabla^{\aff}$ on $T^{\aff}$: \
$\nabla^{\aff}$ induces $\nabla$ on $T$ and
 for $z$ the
zero section of $T^{\aff}$, one has
$\nabla_X^{\aff}z=-X$ [connexion sans glissement].

The torsion of $\nabla$ is the translation part of the
curvature of $\nabla^{\aff}$.
View $X$ as a map of $\dbR^2$ to the zero section of th
tangent bundle of $M$.
If $\nabla$ is torsion free, vanishing of the
translation part of the curvature gives
$$
\gather
\nabla_t\nabla_u^{\aff}X-\nabla_u\nabla_t^{\aff}X=0\,\,,\qquad
\text{i.e.}\\
-\nabla_t\partial_u X+\nabla_u\partial_t X=0
\endgather
$$

\enddocument



